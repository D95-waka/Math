\documentclass[a4paper]{article}

% packages
\usepackage{inputenc, fontspec, amsmath, amsthm, amsfonts, polyglossia, catchfile}
\usepackage[a4paper, margin=50pt, includeheadfoot]{geometry} % set page margins

% style
\AddToHook{cmd/section/before}{\clearpage}	% Add line break before section
\linespread{1.5}
\setcounter{secnumdepth}{0}		% Remove default number tags from sections
\setmainfont{Libertinus Serif}
\setsansfont{Libertinus Sans}
\setmonofont{Libertinus Mono}
\setdefaultlanguage{hebrew}
\setotherlanguage{english}

% operators
\DeclareMathOperator\cis{cis}
\DeclareMathOperator\Sp{Sp}
\DeclareMathOperator\tr{tr}
\DeclareMathOperator\im{Im}
\DeclareMathOperator\diag{diag}
\DeclareMathOperator*\lowlim{\underline{lim}}
\DeclareMathOperator*\uplim{\overline{lim}}

% commands
\renewcommand\qedsymbol{\textbf{משל}}
\newcommand{\NN}[0]{\mathbb{N}}
\newcommand{\ZZ}[0]{\mathbb{Z}}
\newcommand{\QQ}[0]{\mathbb{Q}}
\newcommand{\RR}[0]{\mathbb{R}}
\newcommand{\CC}[0]{\mathbb{C}}
\newcommand{\getenv}[2][] {
  \CatchFileEdef{\temp}{"|kpsewhich --var-value #2"}{\endlinechar=-1}
  \if\relax\detokenize{#1}\relax\temp\else\let#1\temp\fi
}
\newcommand{\explain}[2] {
	\begin{flalign*}
		 && \text{#2} && \text{#1}
	\end{flalign*}
}

% headers
\getenv[\AUTHOR]{AUTHOR}
\author{\AUTHOR}
\date\today

\title{פתרון מטלה 01 --- תורת הקבוצות האקסיומטית, 80650}

\DeclareMathOperator{\rank}{rank}

\begin{document}
\maketitle
\maketitleprint{}
במטלה זו נניח ZF אלא אם נאמר אחרת.

\Question{}
נוכיח את משפט הרקורסיה עבור יחסים מבוססים היטב. \\*
תהי מחלקה $A$ ויהי $R$ יחס על $A$ כך שהוא מבוסס היטב ודומה־קבוצה (set-like). \\*
תהי $G$ מחלקת פונקציה כך שכל קבוצה במקור שלה, ונוכיח כי קיימת מחלקה יחודית $F$ כך שמתקיים
\[
	\forall x \in A, F(x) = G(F \upharpoonright \{y \mid y R x \})
\]
\begin{proof}
	אם $A$ ריקה אז סיימנו, אחרת נניח ש־$\tilde{a} \in A$ איבר כלשהו, ולכן מתכונת דמיון־קבוצה נסיק כי $\{ x \mid x R \tilde{a} \}$ היא קבוצה, ובתור קבוצה ש־$R$ יחס מבוסס היטב עליה נוכל להסיק כי יש לה מינימום $a$. \\*
	נגדיר מחלקה $F$ כך ש־$F(a) = G(\emptyset)$, ונשתמש בהגדרה זו כבסיס לאינדוקציה על סדרים מבוססים היטב. \\*
	עתה נניח ש־$x \in A$, וש־$\{ y \in A \mid y R x \} \subseteq \dom F$ ונרצה להראות שמתקיים $F(x) - G(F \upharpoonleft \{ y \mid y R x \})$. \\*
	מההנחה שלנו נבחין כי $F \upharpoonright \{ y \mid y R x \}$ היא מחלקת פונקציה ומאקסיומת הפרדה נוכל להניח כי זו בפרט קבוצה ולכן $G$ מוגדרת עליה, אז נקבל ש־$F$ אכן מוגדרת על $x$ ולכן $x \in \dom F$. \\*
	ממשפט האינדוקציה על סדרים מבוססים היטב נסיק כי אכן $F$ מחלקת פונקציה יחידה כך ש־$\dom F = A$ וגם התנאי מתקיים.
\end{proof}

\Question{}
\Subquestion{}
יהי $R$ יחס מבוסס היטב, דומה־קבוצה, על מחלקה $A$. 
נוכיח שקיימת מחלקת פונקציה יחידה $\rank_R : A \to Ord$ המקיימת
\[
	\rank_R(x) = \sup\{ \rank_R(y) + 1 \mid y R x \}
\]
כאשר $\sup \emptyset = 0$. \\*
נוכיח גם ש־$x R y \implies \rank_R(x) < \rank_R(y)$ לכל $x, y \in A$.
\begin{proof}
	נבחר שוב $a \in A$ איבר מינימום, ולכן נגדיר $\rank_R(a) = \sup \emptyset = 0$. \\*
	נגדיר $G$ מחלקת פונקציה כך ש־$G(\rank_R \upharpoonright \{ y \mid y R x \}) = \sup\{ \rank_R(y) + 1 \mid y R x \}$, ולכן ממשפט הרקורסיה לסדרים מבוססים היטב קיימת $\rank_R$ יחידה כזו.

	נניח עתה כי $x, y \in A$ וגם ש־$x R y$.
	נבחין כי $\rank_R(y) + 1 \in \{ \rank_R(y') \mid y' R x \}$ ולכן בהכרח $\rank_R(y) + 1 \le \rank_R(x)$, דהינו $\rank_R(y) < \rank_R(x)$.
\end{proof}

\Subquestion{}
נוכיח ש־$V_\alpha = \{ x \mid \rank_\in(x) < \alpha \}$.
נסיק ש־$\rank_\in(x)$ הוא הסודר $\alpha$ הקטן ביותר כך ש־$x \subseteq V_\alpha$ וש־$\forall x, \rank_\in(x) = \rank(x)$ כפי שהגדרנו בכיתה.
\begin{proof}
	נתחיל בהוכחת טענה קשורה, נוכיח כי$\rank_\in(V_\alpha) = \alpha$ לכל $\alpha \in Ord$. \\*
	נוכיח באינדוקציה על סודרים, תחילה נבחין כי $\rank_\in(V_0) = \rank_\in(\emptyset) = 0$ והטענה נכונה. \\*
	נניח כי הטענה נכונה עבור $\alpha \in Ord$, לכן מהעובדה ש־$V_\alpha \in V_{\alpha + 1}$ נובע
	\[
		\rank_\in(V_{\alpha + 1})
		= \sup\{ \rank_\in(x) + 1 \mid x \in V_{\alpha + 1} \}
		= \rank_\in(V_\alpha) + 1
		= \alpha + 1
	\]
	נבחין כי מעבר זה נכון מטרנזיטיביות $V_{\alpha + 1}$. \\*
	עתה נניח כי $\alpha$ סודר גבולי ונניח שהטענה מתקיימת עבור כל $\beta \in \alpha$, ולכן
	\[
		\rank_\in(V_{\alpha})
		= \sup\{ \rank_\in(x) + 1 \mid x \in V_\alpha \}
		= \sup\{ \rank_\in(V_\beta) + 1 \mid V_\beta \in V_\alpha \}
		= \sup\{ \beta + 1 \mid \beta \in \alpha \}
		= \alpha
	\]
	השלמנו את מהלך האינדוקציה ולכן מתקיים $\forall \alpha \in Ord, \rank_\in(V_\alpha) = \alpha$.

	נעבור עתה להוכחת הטענה, מהמסקנה של הסעיף הקודם יחד עם השוויון שמצאנו זה עתה מתקיים
	\[
		V_\alpha
		= \{ x \mid x \in V_\alpha \}
		= \{ x \mid \rank_\in(x) < \rank_\in(V_\alpha) \}
		= \{ x \mid \rank_\in(x) < \rank_\in(V_\alpha) \}
		= \{ x \mid \rank_\in(x) < \alpha \}
	\]

	מהשוויון שמצאנו נובע שלכל $x$ מתקיים $\rank_\in(x) = \alpha = \rank_\in(V_\alpha)$, בהתאם נקבל ש־$\rank_\in(x) < \rank_\in(V_{\alpha + 1})$ ולכן $x \in V_{\alpha + 1}$ (לא הראינו את נכונות טענה זו אך היא נובעת משלילת טענה זו),
	לכן גם $x \subseteq V_\alpha$, וכמובן מ־$\rank_\in$ נסיק כי זהו גם הסודר הקטן ביותר כך שהכלה זו מתקיימת.
	אבל למעשה זוהי ההגדרה של $\rank$ עצמו, דהינו $\rank_\in(x) = \rank(x)$ לכל $x$.
\end{proof}

\Question{}
\begin{definition*}
	נאמר ש־$\omega$־רקורסיה חלה (holds) אם לכל מחלקת פונקציה $G$ קיימת פונקציה (במובן של קבוצה) $f$ כך ש־$\dom f = \omega$ כך שמתקיים
	\[
		\forall n < \omega, f(n) = G(f \upharpoonright n)
	\]
\end{definition*}
נזכור כי $V_\alpha \models \textbf{Z}$ לכל סודר גבולי $\alpha \ge \omega + \omega$.

\Subquestion{}
נוכיח ש־$\textbf{Z}$ לא מוכיח $\omega$־רקורסיה, על־ידי מציאת מודל של $\textbf{Z}$ אשר יש בו מקרה שנכשל של $\omega$־רקורסיה.
\begin{proof}
	נבחן את המודל $\langle V_{\omega + \omega}, \in \rangle \models \textbf{Z}$. \\*
	נבחין כי $V_\omega \in U$ מטרנזיטיביות, ונגדיר $f(0) = V_\omega$ ו־$\forall n \in \omega, f(n + 1) = \PP(f(n))$, נבחין כי זו אכן קבוצה לכל מקרה סופי. \\*
	אילו היינו מניחים $\omega$־רקורסיה, על־ידי שימוש באקסיומת הפרדה היינו מקבלים כי $\rng\{ f(n) \mid n \in \omega\} = \{ V_{\omega + \alpha} \mid n \in \omega \} = V_{\omega + \omega}$ היא קבוצה, לכן $U \in U$. \\*
	זו כמובן סתירה לאקסיומת היסוד ולכן $\omega$־רקורסיה סותרת את המודל שהגדרנו ל־$\textbf{Z}$.
\end{proof}

\Subquestion{}
תהי $\varphi$ הטענה ''האיחוד של קבוצה בת־מניה כך שהיא מכילה קבוצות בנות־מניה הוא בן־מניה''. \\*
נוכיח ש־$\textbf{ZF} + \varphi$ מוכיח שקיימת קבוצה טרנזיטיבית שמהווה מודל עבור $\textbf{Z} + \text{־רקורסיה}\omega$.
\begin{proof}
	נוכיח באינדוקציה על סודרים כי כל סודר הוא בן־מניה, ברור כי $\omega$ וכל תת־קבוצה שלו בני־מניה. \\*
	נניח ש־$\alpha$ בת־מניה, אז גם $\bigcup \alpha$ בת־מניה, ואיחוד סופי הוא בפרט איחוד בן־מניה ולכן $\alpha + 1$ אף הוא בן־מניה. \\*
	נניח ש־$\alpha$ סודר גבולי ושלכל $\beta \in \alpha$ גם $\beta$ בן־מניה, אז האיחוד הוא בעצמו בן־מניה ואיחוד זה הוא $\alpha$ עצמו, לכן גם הוא בן־מניה. \\*
	אם ככה נוכל להסיק שכל סודר הוא בן־מניה, נבחר סודר גבולי כלשהו $\alpha$ שאיננו בן־מניה ללא $\varphi$ ונבחן את $V_\alpha$. \\*
	כמובן $V_\alpha \models \textbf{Z}$, לכן עלינו רק להוכיח שהוא מקיים גם את $\omega$־רקורסיה. \\*
	נבחין כי כל פונקציית מחלקה $G$ במודל שלנו מתכתבת עם פונקציה $V_\alpha \to V_\alpha$ והיא בת־מניה, לכן נוכל לבנות פונקציה שממפה $\omega$ לקבוצות על־פי תנאי הרקורסיה, אבל פונקציה $f$ כזו מקיימת $f \in V_\alpha$. \\*
	נוכל אם כך להסיק כי גם במודל עצמו $\langle V_\alpha, \in \rangle$ מקיימת את $\omega$־רקורסיה.
\end{proof}

\Question{}
\Subquestion{}
יהי $R$ יחס בינארי על קבוצה $X$.
נוכיח את ההיפוך החלקי של 2 א', דהינו נניח שקיימת פונקציה $f : X \to Ord$ כך שלכל $x, y \in X, y X x \implies f(y) < f(x)$ ונוכיח ש־$R$ מבוסס־היטב.
\begin{proof}
	נתחיל ונראה ש־$x X x \implies f(x) < f(x)$ וזו סתירה ולכן $R$ אנטי־רפלקסיבי. כמו־כן $x X y \land y X x \implies f(x) < f(y) < f(x)$ ולכן גם א־סימטרי, ולבסוף $x R y \land y R z \implies f(x) < f(y) < f(z)$,
	 אז $R$ לא יחס סדר חד אבל הוא כן משרה סדר חד, ובשל השימוש בסודרים סדר זה קווי ומבוסס היטב.
	 נשתמש בביסוס היטב הזה כדי להוכיח שגם $R$ מבוסס היטב. \\*
	 תהי $Y \subseteq X$ ונבחר על־ידי אקסיומת החלפה את $f(Y)$, נקבל קבוצת סודרים, אבל $<$ סדר טוב על הסודרים ולכן קיים מינימלי $\alpha$ כלשהו.
	 עתה נגדיר $A = \{ y \in Y \mid f(y) = \alpha \}$, ונקבל קבוצה של איברים, זוהי קבוצה לא ריקה ולכן נניח $a \in A$ איבר כלשהו. \\*
	 אז $a \in Y$ וכן $\forall y \in Y, a R y$ כמסקנה מהטענה הראשנה.
\end{proof}

\Subquestion{}
תהינה $M, N$ מחלקות טרנזיטיביות המספקות את $\textbf{ZF}$, כך ש־$M \subseteq N$.
יהי $R$ יחס על קבוצה $a \in M$. \\*
נוכיח ש־$R$ הוא מבוסס היטב ב־$M$ אם ורק אם הוא מבוסס היטב ב־$N$.
\begin{proof}
	נניח ש־$R$ מבוסס היטב ב־$M$.
	משאלה 3 נסיק כי קיים $\rank_R : a \to Ord$, כך שגם
	\[
		\forall x, y \in a, x R y \implies \rank_R(x) < \rank_R(y)
	\]
	אנו יודעים ש־$R$ יחס על $a$ ב־$N$, ואנו יודעים כי גם לכל $x, y \in N$ אם $x R y$ (שקורה כאשר $x, y \in a$) אז $\rank_R(x) < \rank_R(y)$, אז מתוצאת סעיף א' מתקבל ש־$R$ מבוסס היטב ב־$N$.

	הכיוון ההפוך דומה.
\end{proof}

\end{document}
