\documentclass[a4paper]{article}

% packages
\usepackage{inputenc, amsmath, amsthm, thmtools, amsfonts, amssymb, luacode, catchfile, tikzducks, hyperref}
\usepackage[a4paper, margin=50pt, includeheadfoot]{geometry} % set page margins
\usepackage[shortlabels]{enumitem}
\usepackage[skip=3pt, indent=0pt]{parskip}

% language
\usepackage[bidi=basic, layout=tabular, provide=*]{babel}
\babelprovide[main, import]{hebrew}
\babelprovide{rl}
\babelfont{rm}{Libertinus Serif}
\babelfont{sf}{Libertinus Sans}
\babelfont{tt}{Libertinus Mono}

% style
\AddToHook{cmd/section/before}{\clearpage}	% Add line break before section
\linespread{1.3}
\setcounter{secnumdepth}{0}		% Remove default number tags from sections, this won't do well with theorems
\AtBeginDocument{\setlength{\belowdisplayskip}{3pt}}
\AtBeginDocument{\setlength{\abovedisplayskip}{3pt}}

% operators
\DeclareMathOperator\cis{cis}
\DeclareMathOperator\Sp{Sp}
\DeclareMathOperator\tr{tr}
\DeclareMathOperator\im{Im}
\DeclareMathOperator\re{Re}
\DeclareMathOperator\diag{diag}
\DeclareMathOperator*\lowlim{\underline{lim}}
\DeclareMathOperator*\uplim{\overline{lim}}
\DeclareMathOperator\rng{rng}
\DeclareMathOperator\Sym{Sym}
\DeclareMathOperator\Arg{Arg}
\DeclareMathOperator\Log{Log}
\DeclareMathOperator\dom{dom}

% commands
%\renewcommand\qedsymbol{\textbf{מש''ל}}
%\renewcommand\qedsymbol{\fbox{\emoji{lizard}}}
\newcommand{\NN}[0]{\mathbb{N}}
\newcommand{\ZZ}[0]{\mathbb{Z}}
\newcommand{\QQ}[0]{\mathbb{Q}}
\newcommand{\RR}[0]{\mathbb{R}}
\newcommand{\CC}[0]{\mathbb{C}}
\newcommand{\FF}[0]{\mathbb{F}}
\newcommand{\PP}[0]{\mathbb{P}}
\newcommand{\TT}[0]{\mathbb{T}}
\newcommand{\acts}[0]{\circlearrowright}
\newcommand{\explain}[2] {
	\begin{flalign*}
		 && \text{#2} && \text{#1}
	\end{flalign*}
}
\newcommand{\maketitleprint}[0]{ \begin{center}
	\begin{tikzpicture}[scale=3]
		\duck[graduate=gray!20!black, tassel=red!70!black]
	\end{tikzpicture}	
\end{center}
}

% theorem commands
\newtheoremstyle{c_remark}
	{}	% Space above
	{}	% Space below
	{}% Body font
	{}	% Indent amount
	{\bfseries}	% Theorem head font
	{}	% Punctuation after theorem head
	{.5em}	% Space after theorem head
	{\thmname{#1}\thmnumber{ #2}\thmnote{ \normalfont{\text{(#3)}}}}	% head content
\newtheoremstyle{c_definition}
	{3pt}	% Space above
	{3pt}	% Space below
	{}% Body font
	{}	% Indent amount
	{\bfseries}	% Theorem head font
	{}	% Punctuation after theorem head
	{.5em}	% Space after theorem head
	{\thmname{#1}\thmnumber{ #2}\thmnote{ \normalfont{\text{(#3)}}}}	% head content
\newtheoremstyle{c_plain}
	{3pt}	% Space above
	{3pt}	% Space below
	{\itshape}% Body font
	{}	% Indent amount
	{\bfseries}	% Theorem head font
	{}	% Punctuation after theorem head
	{.5em}	% Space after theorem head
	{\thmname{#1}\thmnumber{ #2}\thmnote{ \text{(#3)}}}	% head content

\theoremstyle{c_plain}
\newtheorem{theorem}{משפט}[section]
\newtheorem{lemma}[theorem]{למה}
\newtheorem{proposition}[theorem]{טענה}
\newtheorem*{proposition*}{טענה}
%\newtheorem{corollary}[theorem]{אין חלופה עברית}

\theoremstyle{c_definition}
\newtheorem{definition}[theorem]{הגדרה}
\newtheorem*{definition*}{הגדרה}
\newtheorem{example}{דוגמה}[section]
\newtheorem{exercise}{תרגיל}[section]

\theoremstyle{c_remark}
\newtheorem*{remark}{הערה}
\newtheorem*{solution}{פתרון}
\newtheorem{conclusion}[theorem]{מסקנה}
\newtheorem{notation}[theorem]{סימון}

% Questions related commands
\newcounter{question}
\setcounter{question}{1}
\newcounter{sub_question}
\setcounter{sub_question}{1}

\newcommand{\question}[1][0]{
	\ifthenelse{#1 = 0}{}{\setcounter{question}{#1}}
	\subsection{שאלה \arabic{question}}
	\addtocounter{question}{1}
	\setcounter{sub_question}{1}
}

\newcommand{\subquestion}[1][0]{
	\ifthenelse{#1 = 0}{}{\setcounter{sub_question}{#1}}
	\subsubsection{סעיף \localecounter{letters.gershayim}{sub_question}}
	\addtocounter{sub_question}{1}
}

% import lua and start of document
\directlua{common = require ('../common')}

\GetEnv{AUTHOR}

% headers
\author{\AUTHOR}
\date\today

\title{פתרון מטלה 00 --- תורת הקבוצות האקסיומטית, 80650}

\begin{document}
\maketitle
\maketitleprint{}

במטלה זו נניח את סט האקסיומות Z --- Foundation.
\Question{}
נוכיח שאם $A$ מחלקה לא ריקה, אז $\bigcap A$ קבוצה.
\begin{proof}
	נבחין תחילה כי $\bigcap A$ מחלקה (שעלולה להיות ריקה), עוד נתון כי $A$ עצמה לא ריקה ולכן תהי $a \in A$ קבוצה כלשהי.
	עתה נשתמש בסכמת הפרדה ונקבל $\{ x \in a \mid x \in \bigcap A \}$ קבוצה, אבל מהגדרת החיתוך אם $x \in A \iff \forall b \in A, b \in x \implies x \in a$ וסיימנו.
\end{proof}

\Question{}
נוכיח כי לכל $a, b$ קבוצות, גם $a \times b = \{ \langle x, y \rangle = \{ \{ x \}, \{ x, y \}\} \mid x \in a, y \in b \}$ קבוצה.
\begin{proof}
	מאקסיומת קבוצת חזקה נבחין כי $\mathcal{P}(a)$ קיימת, מסכמת הפרדה עבור $\varphi = \forall w, z \in x, w = z$ נקבל קבוצת יחידונים עבור קבוצות ב־$a$, דהינו $c_0 = \{ \{ x \} \mid x \in a \}$.
	מאקסיומת האיחוד ומאקסיומת הזוגות הלא סדורים קיימת הקבוצה $c_1 = \{ \{x, y \} \mid x, y \in a \cup b \}$, וכן קיימת $c_0 \cup c_1$.
	עתה נגדיר $c_2 = \mathcal{P}(c_0 \cup c_1)$ ואת הטענה $\varphi = \forall u \in x (\exists z_0, z_1 \in u \land z_0 \in z_1 \land z_0 \subseteq z_1 \land \forall z_2 \in u (z_2 = z_0 \lor z_2 = z_1))$.
	ונקבל מסכמת הפרדה את קבוצת הקבוצות בגודל 2 כך שקבוצה אחת חלקית לשנייה, נשאר להשתמש שוב בסכמת הפרדה עבור $\psi = \forall u \in x, \exists z_0, z_1 (z_0 \in a \land z_1 \in b \land z_0, z_1 \in u )$
	וקיבלנו את הקבוצה המבוקשת.
\end{proof}

\Question{}
עתה נוכיח את טענת השאלה הקודמת כאשר לא מניחים את אקסיומת קבוצת החזקה אך מניחים את אסיומת סכמת החלפה.
\begin{proof}
	מאקסיומת הזוג הלא סדור יש לנו הצדקה להגדיר את פונקציית המחלקה
	\[
		F_y(x) = \{ x, y \}
	\]
	מאקסיומת החלפה נקבל שעבור $y \in b$ כלשהו, $c_y = \{ \{x, y \} \mid x \in a \}$ היא קבוצה, ולכן יש לנו הצדקה להגדיר פונקציית מחלקה נוספת
	\[
		G(y) = \{ c_y \mid y \}
	\]
	ונקבל $d = \{ c_y \mid y \in b \}$ קבוצה, ונבחין כי גם האיחוד $\bigcup d$ קבוצה, עתה נשתמש בסכמת החלפה עם $H(\{x, y\}) = \{ \langle x, y \rangle, \langle y, x\rangle \}$ ובאיחוד ונקבל את הקבוצה הדרושה.
\end{proof}

\Question{}
נפתור את הסעיפים הבאים תוך שימוש ב־Z.

\Subquestion{}
נמצא נוסחה $\Delta_0$, $\varphi_1(x)$ כך ש־$x$ סודר אם ורק אם $\varphi_1(x)$.
\begin{solution}
	נגדיר נוסחה כך ש־$x$ סדר טוב יחד עם $\in$ אם ורק אם $\psi_0(x)$.
	\[
		\psi_0(x) = \forall a \in \mathcal{P}(x) (\exists b \in a (\forall c \in a (b \in c)))
	\]
	נגדיר נוסחה $\psi_1(x)$ אשר מתקיימת אם ורק אם $x$ קבוצה טרנזיטיבית.
	\[
		\psi_1(x) = \forall a \in x (\forall b \in a ( b \in x))
	\]
	לבסוף נגדיר $\varphi(x) = \psi_1(x) \land \psi_1(x)$.
\end{solution}

\Subquestion{}
נמצא נוסחה $\Delta_0$, $\varphi_2(x, y)$ כך ש־$x = \bigcup y$ אם ורק אם $\varphi_2(x)$.
\begin{solution}
	נגדיר את הנוסחה בהתאם להגדרת האיחוד
	\[
		\varphi_2(x, y) = (\forall a \in x (\exists b \in y (a \in b))) \land (\forall a \in y (\forall b \in a (b \in x)))
	\]
\end{solution}
נמצא נוסחה $\Delta_0$, $\varphi_3(x)$ כך ש־$x$ הוא סודר עוקב אם ורק אם $\varphi_3(x)$.
\begin{solution}
	נוכל להשתמש ב־$\varphi_1(x)$ כדי לוודא ש־$x$ אכן סודר, ונכתוב נוסחה נוספת לבדיקה שהוא אכן עוקב לסודר אחר:
	\[
		\psi(x) = \exists a \in x (x = a \cup \{ a \})
	\]
	נוסחה זו כמובן מקבל משמעות בעקבות אקסיומת הזוג הלא סדור.
	לבסוף נגדיר $\varphi_3(x) = \varphi_1(x) \land \psi(x)$.
\end{solution}

\Subquestion{}
נמצא נוסחה $\Delta_0$, $\varphi_4(x)$ כך ש־$x = \omega$ אם ורק אם $\varphi_4(x)$.
\begin{solution}
	נשתמש בעובדה ש־$\omega$ הסודר הגבולי הראשון, ולכן כל $n \in \omega$ הוא סודר עוקב או הקבוצה הריקה:
	\[
		\varphi_4(x) = \forall a \in x (\varphi_3(a) \lor a = \emptyset)
	\]
\end{solution}

\Question{}
יהיו $A \subseteq B$ קבוצות טרנזיטיביות לא ריקות.

\Subquestion{}
תהי נוסחה $\varphi(y_0, \dots, y_{n - 1})$ מהצורה $\forall x \psi$ כאשר $\psi$ היא $\Delta_0$, ויהיו $p_0, \dots, p_{n - 1} \in A$. \\*
נוכיח כי $\langle B, \in \rangle \models \varphi(p_0, \dots, p_{n - 1})$ גורר ש־$\langle A, \in \rangle \models \varphi(p_0, \dots, p_{n - 1})$.
\begin{proof}
	אפשר להראות ש־$\varphi(p_0, \dots, p_{n - 1}) \vdash \forall x \in B \psi$ ולכן גם $\varphi(p_0, \dots, p_{n - 1}) \models \forall x \in B \psi$ ומהלמה שנלמדה בכיתה נקבל כי הטענה מתקיימת, לכן בפרט גם $\forall x \in A \psi$,
	ושוב מהלמה נקבל $\langle A, \in \rangle \models \forall x \in A \psi$, אבל $\models_A \forall x \in A$, דהינו טענה זו חלה תמיד, ולכן נוכל לקבל גם $\forall x \in A \psi \vdash \varphi(p_0, \dots, p_{n - 1})$.
	בהתאם נקבל $\langle A, \in \rangle \models \varphi(p_0, \dots, p_{n - 1})$.
\end{proof}

\Subquestion{}
הפעם תהי $\varphi(y_0, \dots, y_{n - 1})$ מהצורה $\exists x \psi$ עבור אותה $\psi$ וכאשר $p_0, \dots, p_{n - 1} \in A$.
נוכיח שהפעם $\langle A, \in \rangle \models \varphi(p_0, \dots, p_{n - 1})$ גורר $\langle B, \in \rangle \models \varphi(p_0, \dots, p_{n - 1})$.
\begin{proof}
	נוכל לבחון את הפסוק $\exists x \in A \psi(p_0, \dots, p_{n - 1})$, הוא כמובן עומד בתנאי הלמה ולכן $\langle B, \in \rangle \models \exists x \in A \psi(p_0, \dots, p_{n - 1})$.
	אבל אז מתקיים $\exists x \in A \psi(p_0, \dots, p_{n - 1}) \models \exists x \psi$ דהינו $\langle B, \in \rangle \models \varphi$.
\end{proof}

\end{document}
