\newcounter{english}
\documentclass[a4paper]{article}

% packages
\usepackage{inputenc, fontspec, amsmath, amsthm, amsfonts, polyglossia, catchfile}
\usepackage[a4paper, margin=50pt, includeheadfoot]{geometry} % set page margins

% style
\AddToHook{cmd/section/before}{\clearpage}	% Add line break before section
\linespread{1.5}
\setcounter{secnumdepth}{0}		% Remove default number tags from sections
\setmainfont{Libertinus Serif}
\setsansfont{Libertinus Sans}
\setmonofont{Libertinus Mono}
\setdefaultlanguage{hebrew}
\setotherlanguage{english}

% operators
\DeclareMathOperator\cis{cis}
\DeclareMathOperator\Sp{Sp}
\DeclareMathOperator\tr{tr}
\DeclareMathOperator\im{Im}
\DeclareMathOperator\diag{diag}
\DeclareMathOperator*\lowlim{\underline{lim}}
\DeclareMathOperator*\uplim{\overline{lim}}

% commands
\renewcommand\qedsymbol{\textbf{משל}}
\newcommand{\NN}[0]{\mathbb{N}}
\newcommand{\ZZ}[0]{\mathbb{Z}}
\newcommand{\QQ}[0]{\mathbb{Q}}
\newcommand{\RR}[0]{\mathbb{R}}
\newcommand{\CC}[0]{\mathbb{C}}
\newcommand{\getenv}[2][] {
  \CatchFileEdef{\temp}{"|kpsewhich --var-value #2"}{\endlinechar=-1}
  \if\relax\detokenize{#1}\relax\temp\else\let#1\temp\fi
}
\newcommand{\explain}[2] {
	\begin{flalign*}
		 && \text{#2} && \text{#1}
	\end{flalign*}
}

% headers
\getenv[\AUTHOR]{AUTHOR}
\author{\AUTHOR}
\date\today

\title{Exercise 8 Answer Sheet --- Axiomatic Set Theory, 80650}

\DeclareMathOperator{\crit}{crit}

\begin{document}
\maketitle
\maketitleprint{}

\question{}
\begin{definition}
	Let $X$ be a set. A tree $T$ is set such that,
	\begin{enumerate}
		\item For every $\eta \in T$, $\eta$ is a function from an ordinal $\alpha$ to $X$.
		\item If $\eta \in T$ and $\dom \eta = \alpha > \beta$ then $\eta \restriction \beta \in T$.
	\end{enumerate}
	If $X = 2$ then we say that $T$ is binary tree. \\
	The height of $T$ is the least ordinal $\alpha$ such that $\forall \eta \in T, \dom \eta < \alpha$.
	We define $\operatorname{Lev}(\eta) = \dom \eta$ (the level of $\eta$), and we denote $T_\alpha = \{\eta \in T \mid \operatorname{Lev}(\eta) = \alpha\}$.
	For $\eta, \eta' \in T$ we define $\eta \le_T \eta'$ if $\eta = \eta' \restriction \dom \eta$.
\end{definition}
\begin{definition}
	Let $\kappa$ be a regular cardinal, we say that a tree $T$ is a $\kappa$-tree if the height of $T$ is $\kappa$ and for every $\alpha < \kappa$, $|T_\alpha| < \kappa$.
\end{definition}
\begin{definition}
	Let $T$ be a tree of height $\alpha$.
	A function $b : \alpha \to X$ is a cofinal branch in $T$ if for every $\beta < \alpha$, $b \restriction \beta \in T$.
	We would also use the term cofinal branch for the set $\{ b \restriction \beta \mid \beta < \alpha \}$.
\end{definition}

Let $\kappa$ be an infinite regular cardinal.
Let $T$ be a binary $\kappa$-tree.

We will prove that there is $T' \subseteq T$ of height $\kappa$ such that for every $\alpha < \beta < \kappa$ and $x \in T'$ with $\operatorname{Lev}(x) = \alpha$,
there is $y \in T'$ with $\operatorname{Lev}(y) = \beta$ and $x \le_T y$.
\begin{proof}
	The assumption that this condition isn't met at all will lead to $T$ being of height less than $\kappa$, then there are such elements in $T$.
	Why can't we just take an arbitrary $\kappa$ branch, this would create some linear order of order $\kappa$ and would satisfy the proposition.
\end{proof}

\question{}
We will show that every binary $\omega$-tree has a cofinal branch.
\begin{proof}
	It is direct by the definition of height of a tree.
	From the last question, we can assume $T' \subseteq T$ fulfills the property of arbitrary elements, then we will define recursively the function $b : \omega \to X$ by the following,
	\begin{enumerate}
		\item $b(0) = \eta(0)$, when $\eta$ is any branch $\in T$ (the root of ordered tree is unique).
		\item For every $0 \le n < \omega$, let $b \restriction n + 1 = x$ when $x \in T', \operatorname{Lev}(x) = n + 1$ such that $b \restriction n \le_T x$, given by the last question.
	\end{enumerate}
	This recursive definition ensures us that for every $\alpha < \omega$, indeed $b \restriction \alpha \in T' \subseteq T$, as desired.
\end{proof}

\question{}
We will prove that if there is some cardinal $\mu$ such that $\mu^+ < \kappa$ and $|T_\alpha| \le \mu$ for all $\alpha$, then $T$ has a cofinal branch.
\begin{proof}
	Let us assume there is such a cardinal.
	We assume for contradiction that there is no cofinal branch, meaning that for every branch $b$, there is $\alpha < \kappa$ such that $\dom b < \alpha$.
	From the other hand, if there are no branches of height $\alpha$, we get contradiction to the height of $T$, then there is at least one branch.

	We somehow use the fact that there $|T_\alpha| = 2^\alpha$, this is derived from the fact that $T$ is binary, and if there are not such many branches that reach this level, then there is cofinal branch.
	For each level there has to be lot of branches so they could be cut off afterwards in the right way to not be cofinal.
	We know that there are at least $\mu^+$ levels above $\mu$, each of these ends before $\kappa$, then we look at $T_\mu$,
	we get that $|T_\mu| \ge \mu^+$ so there will be enough branches which are not cofinal, and this is contradiction.
\end{proof}

\question{}
Let us assume that there is a cardinal $\mu < \kappa$ and a function $f : T \to \mu$ such that for all $x, y \in T$, if $x <_T y$ then $f(x) \ne f(y)$. \\
We will prove that there is no cofinal branch in $T$.
\begin{proof}
	Let us assume that $b : \kappa \to X$ is a cofinal branch of $T$.
	By transitivity we get that $f(b \restriction \alpha) \ne f(b \restriction \beta)$ for all $\alpha < \beta < \kappa$.
	We can deduce that $X = f '' \{ b \restriction \alpha \mid \alpha < \kappa \}$ is a set of cardinality of $\kappa$, in contradiction to $X \subseteq \rng f = \mu < \kappa$.
\end{proof}

\question{}
Let $\kappa$ be a measurable cardinal. \\
We will prove that every $\kappa$-tree has a cofinal branch.
\begin{proof}
	If $\kappa \le 2^{\aleph_0}$ then we already know that every $\kappa$-tree has cofinal branch.
	We assume that $2^{\aleph_0} < \kappa$, then the measure is $\kappa$-complete and we can assume that $\kappa$ is inaccessible (strong?).

	By the inaccessibility and $\kappa$ being regular we can deduce there is no mapping from $\alpha < \kappa$ to $|T_\alpha|$ such that $|T_\alpha| = \kappa$.
	Then there is an ordinal $\mu$, by the inaccessibility of $\kappa$, such that $|T_\alpha| \le \mu$ for all $\alpha < \kappa$.
	Then we can use the previous question to deduce that there is a cofinal branch in $T$.
\end{proof}

\end{document}
