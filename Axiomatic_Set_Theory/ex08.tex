\newcounter{english}
\documentclass[a4paper]{article}

% packages
\usepackage{inputenc, fontspec, amsmath, amsthm, amsfonts, polyglossia, catchfile}
\usepackage[a4paper, margin=50pt, includeheadfoot]{geometry} % set page margins

% style
\AddToHook{cmd/section/before}{\clearpage}	% Add line break before section
\linespread{1.5}
\setcounter{secnumdepth}{0}		% Remove default number tags from sections
\setmainfont{Libertinus Serif}
\setsansfont{Libertinus Sans}
\setmonofont{Libertinus Mono}
\setdefaultlanguage{hebrew}
\setotherlanguage{english}

% operators
\DeclareMathOperator\cis{cis}
\DeclareMathOperator\Sp{Sp}
\DeclareMathOperator\tr{tr}
\DeclareMathOperator\im{Im}
\DeclareMathOperator\diag{diag}
\DeclareMathOperator*\lowlim{\underline{lim}}
\DeclareMathOperator*\uplim{\overline{lim}}

% commands
\renewcommand\qedsymbol{\textbf{משל}}
\newcommand{\NN}[0]{\mathbb{N}}
\newcommand{\ZZ}[0]{\mathbb{Z}}
\newcommand{\QQ}[0]{\mathbb{Q}}
\newcommand{\RR}[0]{\mathbb{R}}
\newcommand{\CC}[0]{\mathbb{C}}
\newcommand{\getenv}[2][] {
  \CatchFileEdef{\temp}{"|kpsewhich --var-value #2"}{\endlinechar=-1}
  \if\relax\detokenize{#1}\relax\temp\else\let#1\temp\fi
}
\newcommand{\explain}[2] {
	\begin{flalign*}
		 && \text{#2} && \text{#1}
	\end{flalign*}
}

% headers
\getenv[\AUTHOR]{AUTHOR}
\author{\AUTHOR}
\date\today

\title{Exercise 8 Answer Sheet --- Axiomatic Set Theory, 80650}

\DeclareMathOperator{\crit}{crit}

\begin{document}
\maketitle
\maketitleprint{}

\question{}
\begin{definition}
	Let $X$ be a set. A tree $T$ is set such that,
	\begin{enumerate}
		\item For every $\eta \in T$, $\eta$ is a function from an ordinal $\alpha$ to $X$.
		\item If $\eta \in T$ and $\dom \eta = \alpha > \beta$ then $\eta \restriction \beta \in T$.
	\end{enumerate}
	If $X = 2$ then we say that $T$ is binary tree. \\
	The height of $T$ is the least ordinal $\alpha$ such that $\forall \eta \in T, \dom \eta < \alpha$.
	We define $\operatorname{Lev}(\eta) = \dom \eta$ (the level of $\eta$), and we denote $T_\alpha = \{\eta \in T \mid \operatorname{Lev}(\eta) = \alpha\}$.
	For $\eta, \eta' \in T$ we define $\eta \le_T \eta'$ if $\eta = \eta' \restriction \dom \eta$.
\end{definition}
\begin{definition}
	Let $\kappa$ be a regular cardinal, we say that a tree $T$ is a $\kappa$-tree if the height of $T$ is $\kappa$ and for every $\alpha < \kappa$, $|T_\alpha| < \kappa$.
\end{definition}
\begin{definition}
	Let $T$ be a tree of height $\alpha$.
	A function $b : \alpha \to X$ is a cofinal branch in $T$ if for every $\beta < \alpha$, $b \restriction \beta \in T$.
	We would also use the term cofinal branch for the set $\{ b \restriction \beta \mid \beta < \alpha \}$.
\end{definition}

Let $\kappa$ be an infinite regular cardinal.
Let $T$ be a binary $\kappa$-tree.

We will prove that there is $T' \subseteq T$ of height $\kappa$ such that for every $\alpha < \beta < \kappa$ and $x \in T'$ with $\operatorname{Lev}(x) = \alpha$,
there is $y \in T'$ with $\operatorname{Lev}(y) = \beta$ and $x \le_T y$.
\begin{proof}
	Let us define $T_0 = \{ x \in T \mid \forall \operatorname{Lev}(x) < \alpha < \kappa, \exists y \in T, y \in T_\alpha, x \le_T y \}$.
	If $T_0$ is a $\kappa$-tree then it satisfies the required property, then we will show it is indeed a tree of height $\kappa$.

	For every $x \in T_0$, $x \in T$, hence is a map from an ordinal to $X$.
	Assume $y \in T_0$, and let $x = y \restriction \beta$ for $\beta < \kappa$.
	For every $\beta < \gamma < \kappa$ there is $x \le_T y \le_T z$ such that the given property is satisfied.
	We assume $\alpha < \gamma < \beta$, then $y \restriction \gamma \in T_\gamma$ as $y \in T$.
	We can conclude $x \in T_0$, meaning $T_0$ satisfies definition, namely $T_0$ is a tree.

	We move to proving $T_0$ is of height $\kappa$.
	For certain $x \in T_0$ for every $\dom x < \alpha$ there is $y_\alpha \in T_0$ such that $\dom y_\alpha = \alpha$ for evert such ordinal, then $\sup_{\alpha < \kappa} y_\alpha = \kappa$ as intended.
	The claim is not about $T'$ being $\kappa$-tree (I hope), but we know that each level of $T_0$ must be bounded by the equivalent level in $T$, meaning it is bounded by $\kappa$.

	Lastly, we will check if $T_0$ in not empty, fulfilling our claims stated above.
	By the definition of $T$, if we select $\alpha = 0$, by the height of $T$ the statement is indeed true, indicating $\emptyset \in T_0$.

	We showed that there is such $T' = T_0$.
\end{proof}

\question{}
We will show that every binary $\omega$-tree has a cofinal branch.
\begin{proof}
	From the last question, we can assume $T' \subseteq T$ fulfills the property of arbitrary elements, then we will define recursively the function $b : \omega \to X$ by the following,
	\begin{enumerate}
		\item $b(0) = \eta(0)$, when $\eta$ is any branch $\in T$ (the root of ordered tree is unique).
		\item If $b \restriction n$ is already set, then $b(n) \in T_n$ such that $b(n - 1) \le_T b(n)$, there exists such in $T'$.
	\end{enumerate}
	The result is indeed $b : \omega \to X$ such that $b \restriction n \in T' \subseteq T$ for all $n < \omega$,
	meaning $b$ is cofinal branch of $T$ as desired.
\end{proof}

\question{}
We will prove that if there is some cardinal $\mu$ such that $\mu^+ < \kappa$ and $|T_\alpha| \le \mu$ for all $\alpha$, then $T$ has a cofinal branch.
\begin{proof}
	We assume such $\mu$ exists, as well without loss of generality the arbitrary height of branches is fulfilled.
	For each $x \in T$ such that $\alpha = \operatorname{Lev}(x)$, we let $\beta_x$ be the largest ordinal such that there is no other $y \in T_\alpha$ such that $y \restriction \beta_x = x \restriction \beta_x$.
	In other words, we get the highest level in which $x$ is the only continuation (as of branch) of some branch of that level.
	For each level $\alpha$ we define $f(\alpha) = \sup_{x \in T_\alpha} \beta_x$, $f$ mapping each level to the least level below it such that there is uniquely-extendable branch between the levels.
	Let $\dom f = S = \{ \alpha < \kappa \mid \mu^+ < \alpha \}$, then $S$ is stationary in $\kappa$, and $f : S \to \kappa$.
	By the definition, $f(\alpha) \le \alpha$.
	For every $x \in T_\alpha$ for $\alpha \in S$, we know that $|T_\alpha| \le \mu$, then there cannot be more than $\mu$ levels such that there are more continuations to the restricted branch of $x$,
	but $\operatorname{cf} \alpha \ge \mu^+$, meaning the set of such levels is bounded by $\beta < \mu^+$, in particular $\beta_x < \alpha$, then $f(\alpha) < \alpha$, namely $f$ is regressive.
	By Fodors lemma there is $T \subseteq S$ stationary in $\kappa$ such that $\forall x \in T, f(x) = \gamma$ for $\gamma < \kappa$.
	For some arbitrary $\alpha \in T$, let $x \in T_\alpha$ be a branch for which $\beta_x = \gamma$.
	For each $\alpha \in T$, we can conclude $\beta_x = \gamma$.
	$T$ is stationary therefore unbounded in $\kappa$, then for every $\delta < \kappa$, there is $\delta < \delta' \in T$.
	For this $\delta'$ there is a branch $x \le_T y \in T_{\delta'}$ by the arbitrary height claim, and $\beta_y = \gamma$ as well.
	We can define then $b : \kappa \to X$ by setting $b \restriction \delta = y$ for each $\delta < \kappa$, therefore $b$ is a cofinal branch of $T$.
\end{proof}

\question{}
Let us assume that there is a cardinal $\mu < \kappa$ and a function $f : T \to \mu$ such that for all $x, y \in T$, if $x <_T y$ then $f(x) \ne f(y)$. \\
We will prove that there is no cofinal branch in $T$.
\begin{proof}
	We assume by contradiction that $b : \kappa \to X$ is a cofinal branch of $T$.
	By transitivity of $<_T$ it follows that $f(b \restriction \alpha) \ne f(b \restriction \beta)$ for all $\alpha < \beta < \kappa$.
	We can deduce that for $X = f '' \{ b \restriction \alpha \mid \alpha < \kappa \}$, $|X| = \kappa$, in contradiction to $X \subseteq \rng f = \mu < \kappa$.
\end{proof}

\question{}
Let $\kappa$ be a measurable cardinal. \\
We will prove that every $\kappa$-tree has a cofinal branch.
\begin{proof}
	If $\kappa \le 2^{\aleph_0}$ then we already know that every $\kappa$-tree has cofinal branch.
	We assume that $2^{\aleph_0} < \kappa$, then the measure is $\kappa$-complete and we can assume that $\kappa$ is inaccessible (strong?).

	By the inaccessibility and $\kappa$ being regular we can deduce there is no mapping from $\alpha < \kappa$ to $|T_\alpha|$ such that $|T_\alpha| = \kappa$.
	Then there is an ordinal $\mu$, by the inaccessibility of $\kappa$, such that $|T_\alpha| \le \mu$ for all $\alpha < \kappa$.
	Then we can use the previous question to deduce that there is a cofinal branch in $T$.
\end{proof}

\end{document}
