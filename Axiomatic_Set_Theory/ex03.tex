\documentclass[a4paper]{article}

% packages
\usepackage{inputenc, fontspec, amsmath, amsthm, amsfonts, polyglossia, catchfile}
\usepackage[a4paper, margin=50pt, includeheadfoot]{geometry} % set page margins

% style
\AddToHook{cmd/section/before}{\clearpage}	% Add line break before section
\linespread{1.5}
\setcounter{secnumdepth}{0}		% Remove default number tags from sections
\setmainfont{Libertinus Serif}
\setsansfont{Libertinus Sans}
\setmonofont{Libertinus Mono}
\setdefaultlanguage{hebrew}
\setotherlanguage{english}

% operators
\DeclareMathOperator\cis{cis}
\DeclareMathOperator\Sp{Sp}
\DeclareMathOperator\tr{tr}
\DeclareMathOperator\im{Im}
\DeclareMathOperator\diag{diag}
\DeclareMathOperator*\lowlim{\underline{lim}}
\DeclareMathOperator*\uplim{\overline{lim}}

% commands
\renewcommand\qedsymbol{\textbf{משל}}
\newcommand{\NN}[0]{\mathbb{N}}
\newcommand{\ZZ}[0]{\mathbb{Z}}
\newcommand{\QQ}[0]{\mathbb{Q}}
\newcommand{\RR}[0]{\mathbb{R}}
\newcommand{\CC}[0]{\mathbb{C}}
\newcommand{\getenv}[2][] {
  \CatchFileEdef{\temp}{"|kpsewhich --var-value #2"}{\endlinechar=-1}
  \if\relax\detokenize{#1}\relax\temp\else\let#1\temp\fi
}
\newcommand{\explain}[2] {
	\begin{flalign*}
		 && \text{#2} && \text{#1}
	\end{flalign*}
}

% headers
\getenv[\AUTHOR]{AUTHOR}
\author{\AUTHOR}
\date\today

\title{פתרון מטלה 03 --- תורת הקבוצות האקסיומטית, 80650}

\DeclareMathOperator{\trcl}{trcl}
\DeclareMathOperator{\rank}{rank}

\begin{document}
\maketitle
\maketitleprint{}

\Question{}
נניח $\textbf{ZF} + \textbf{AC}$.
יהי מונה רגולרי אינסופי $\kappa$. נגדיר $H(\kappa) = \{ x \mid |\trcl(x)| < \kappa \}$  כאשר $\trcl(x)$ הסגור הטרנזיטיבי של $x$.

תהי קבוצה טרנזיטיבית $x$.
נוכיח ש־$z = \{\rank y \mid y \in x \}$ קבוצה טרנזיטיבית.
\begin{proof}
	נגדיר $c : z \to x$ פונקציית בחירה כך ש־$c(a) \in x$ וכמובן $\rank c(a) \in z$. \\*
	נוכיח שאם $t = \rank x$, אז לא קיים $l < t$ כך ש־$\forall y \in x, \rank y \ne l$. \\*
	נבחן את $l + 1 = t$ עבור $t$ עוקב. מההגדרה השקולה ל־$\rank$ מהמטלה הקודמת נובע ישירות כי  קיים $\rank y = l, y \in x$. אילו $t$ גבולי, אז מאותה סיבה לכל $l < t$ קיים $y$ כזה אחרת $l + 1$ לא בקבוצה וההגדרה הגבולית לא חלה. \\*
	באינדוקציה מהמבנה הזה ועל־ידי שימוש ב־$c$ נוכל לקבל שאכן לא קיים $y$ כזה. \\*
	לבסוף, אם $b \in a \in z$, אז בהכרח $\exists y \in x, \rank y = b$ ולכן גם $c(b) \in x \implies b \in z$.
\end{proof}

\Question{}
נוכיח ש־$H(\kappa)$ קבוצה.
\begin{proof}
	יהי $x \in H(\kappa)$, ונגדיר $y = \trcl(x)$, אז $|y| < \kappa$. \\*
	נגדיר $\alpha = \rank y$, ונניח $\alpha > \kappa^+$ אז נקבל ש־$|y| > \kappa$ וזו סתירה להנחה, לכן $\rank y < \kappa^+$. \\*
	בהתאם $y \in V_{\kappa^+}$ לכל $y$ ולכן $H(\kappa) \subseteq V_{\kappa^+}$ ובפרט קבוצה.
\end{proof}

\Question{}
נוכיח ש־$\kappa \subseteq H(\kappa)$ וש־$H(\kappa)$ טרנזיטיבי.
\begin{proof}
	יהי $\alpha \in \kappa$, אז $\alpha = \trcl(\alpha)$ ואנו יודעים ש־$\kappa$ מונה ולכן מהגדרה $|\alpha| < \kappa$, שאם לא כן המונה בו $\alpha$ היה מוכל היה גדול מ־$\kappa$, וכמובן $\kappa \notin \kappa$.
	נסיק אם כך ש־$\alpha \in H(\kappa)$ לכל $\alpha \in \kappa$ ולכן $\kappa \subseteq H(\kappa)$. \\*
	נראה ש־$H(\kappa)$ טרנזיטיבי.
	יהי $x \in H(\kappa)$, אז $x$ טרנזיטיבי, ונבחן את $y = \trcl(x)$. כמובן $\trcl(y) = y$ ולכן $y \in H(\kappa)$ ומצאנו ש־$H(\kappa)$ טרנזיטיבית עבור הקבוצות הטרנזיטיביות בה.
	יהי $z \in x$, אז גם $z \in y$ ולכן $z \subseteq y$ ובהתאם $y$ סגור טרנזיטיבי של $z$ ו־$z \in H(\kappa)$ ולכן זו קבוצה טרנזיטיבית.

	מתרגיל 1 והטרנזיטיביות נוכל להסיק ש־$\langle H(\kappa), \in \rangle$ מקיים את אקסיומת ההיקפיות והיסוד והקבוצה הריקה. \\*
	אילו נניח ש־$\omega < \kappa$ אז גם $\omega \in H(\kappa)$ ומתרגיל אחד מתקיימת אקסיומת האינסוף.
\end{proof}

\Question{}
נניח ש־$x \in H(\kappa)$ ו־$f : x \to H(\kappa)$.
נראה ש־$f, \rng f \in H(\kappa)$.
\begin{proof}
	נתון ש־$x \in H(\kappa) \iff |\trcl(x)| < \kappa$ אבל $|x| \le |\trcl(x)|$ ולכן $|x| < \kappa$, אז תהי $g : x \to \alpha$ עבור $\alpha \in \kappa$, חד־חד ערכית. \\*
	נגדיר פונקציית החלפה $c : \rng f \to x$, היא חד־חד ערכית מהגדרה, ולכן גם $c \circ g$ חד־חד ערכית ונסיק $|\rng f| = \alpha$. \\*
	אנו יודעים כי איחוד של קבוצות טרנזיטיביות הוא טרנזיטיבי ולכן $u = \bigcup_{y \in \rng f} \trcl y$ קבוצה טרנזיטיבית. \\*
	עוד נבחין כי $|u| < |\alpha \times \kappa| \le |\kappa \times \kappa| = \kappa$ ולכן $\rng f \in H(\kappa)$. \\*
	לבסוף $|x \times \rng f| < |\kappa \times \kappa| < \kappa$ ולכן גם $f \in H(\kappa)$.

	תחת המודל, נסיק שלכל $x \in H(k)$ ומחלקת פונקציה $F$, גם $\{F(y) \mid y \in x\}$ קבוצה, שכן $F \upharpoonright x : x \to H(\kappa)$. \\*
	יהי $x \in H(\kappa)$, אז מעיקרון הסדר הטוב קיים $\alpha \in Ord$ כך שקיימת $f : x \to \alpha$ הפיכה, ומבחירה נוכל להסיק שגם $\alpha < \kappa$.
\end{proof}

\Question{}
נוכיח ש־$H(\kappa)$ מספקת את אקסיומת האיחוד.
\begin{proof}
	למעשה כבר ראינו את הטענה הזו בסעיף הקודם.
	איחוד של קבוצות טרנזיטיביות הוא טרנזיטיבי ולכן אם $x \in H(\kappa)$ אז $\bigcup x \subseteq \bigcup \trcl(x)$ ולכן $\bigcup x \in H(\kappa)$. \\*
	נקבל אם כך שאקסיומת האיחוד חלה עבור $\langle H(\kappa), \in \rangle$ ולכן זהו מודל טרנזיטיבי של $\textbf{ZFC} - \textbf{Power set}$.
\end{proof}

\Question{}
נראה ש־$\textbf{ZFC} - \textbf{Power Set}$ לא מוכיח את הקיום של קבוצה שאיננה בת־מניה.
\begin{proof}
	נגדיר $\kappa = \omega$ ולכן $\langle H(\omega), \in \rangle$ מודל של $\textbf{ZFC} - \textbf{Power Set}$, אז זהו מודל כך שאין בו קבוצות שאינן־בנות מניה.

	אם נוסיף למודל את האקסיומה שאין קבוצה שלא בת־מניה נקבל שהיא עקבית.
\end{proof}

\end{document}
