\documentclass[a4paper]{article}

% packages
\usepackage{inputenc, amsmath, amsthm, thmtools, amsfonts, amssymb, luacode, catchfile, tikzducks, hyperref}
\usepackage[a4paper, margin=50pt, includeheadfoot]{geometry} % set page margins
\usepackage[shortlabels]{enumitem}
\usepackage[skip=3pt, indent=0pt]{parskip}

% language
\usepackage[bidi=basic, layout=tabular, provide=*]{babel}
\babelprovide[main, import]{hebrew}
\babelprovide{rl}
\babelfont{rm}{Libertinus Serif}
\babelfont{sf}{Libertinus Sans}
\babelfont{tt}{Libertinus Mono}

% style
\AddToHook{cmd/section/before}{\clearpage}	% Add line break before section
\linespread{1.3}
\setcounter{secnumdepth}{0}		% Remove default number tags from sections, this won't do well with theorems
\AtBeginDocument{\setlength{\belowdisplayskip}{3pt}}
\AtBeginDocument{\setlength{\abovedisplayskip}{3pt}}

% operators
\DeclareMathOperator\cis{cis}
\DeclareMathOperator\Sp{Sp}
\DeclareMathOperator\tr{tr}
\DeclareMathOperator\im{Im}
\DeclareMathOperator\re{Re}
\DeclareMathOperator\diag{diag}
\DeclareMathOperator*\lowlim{\underline{lim}}
\DeclareMathOperator*\uplim{\overline{lim}}
\DeclareMathOperator\rng{rng}
\DeclareMathOperator\Sym{Sym}
\DeclareMathOperator\Arg{Arg}
\DeclareMathOperator\Log{Log}
\DeclareMathOperator\dom{dom}

% commands
%\renewcommand\qedsymbol{\textbf{מש''ל}}
%\renewcommand\qedsymbol{\fbox{\emoji{lizard}}}
\newcommand{\NN}[0]{\mathbb{N}}
\newcommand{\ZZ}[0]{\mathbb{Z}}
\newcommand{\QQ}[0]{\mathbb{Q}}
\newcommand{\RR}[0]{\mathbb{R}}
\newcommand{\CC}[0]{\mathbb{C}}
\newcommand{\FF}[0]{\mathbb{F}}
\newcommand{\PP}[0]{\mathbb{P}}
\newcommand{\TT}[0]{\mathbb{T}}
\newcommand{\acts}[0]{\circlearrowright}
\newcommand{\explain}[2] {
	\begin{flalign*}
		 && \text{#2} && \text{#1}
	\end{flalign*}
}
\newcommand{\maketitleprint}[0]{ \begin{center}
	\begin{tikzpicture}[scale=3]
		\duck[graduate=gray!20!black, tassel=red!70!black]
	\end{tikzpicture}	
\end{center}
}

% theorem commands
\newtheoremstyle{c_remark}
	{}	% Space above
	{}	% Space below
	{}% Body font
	{}	% Indent amount
	{\bfseries}	% Theorem head font
	{}	% Punctuation after theorem head
	{.5em}	% Space after theorem head
	{\thmname{#1}\thmnumber{ #2}\thmnote{ \normalfont{\text{(#3)}}}}	% head content
\newtheoremstyle{c_definition}
	{3pt}	% Space above
	{3pt}	% Space below
	{}% Body font
	{}	% Indent amount
	{\bfseries}	% Theorem head font
	{}	% Punctuation after theorem head
	{.5em}	% Space after theorem head
	{\thmname{#1}\thmnumber{ #2}\thmnote{ \normalfont{\text{(#3)}}}}	% head content
\newtheoremstyle{c_plain}
	{3pt}	% Space above
	{3pt}	% Space below
	{\itshape}% Body font
	{}	% Indent amount
	{\bfseries}	% Theorem head font
	{}	% Punctuation after theorem head
	{.5em}	% Space after theorem head
	{\thmname{#1}\thmnumber{ #2}\thmnote{ \text{(#3)}}}	% head content

\theoremstyle{c_plain}
\newtheorem{theorem}{משפט}[section]
\newtheorem{lemma}[theorem]{למה}
\newtheorem{proposition}[theorem]{טענה}
\newtheorem*{proposition*}{טענה}
%\newtheorem{corollary}[theorem]{אין חלופה עברית}

\theoremstyle{c_definition}
\newtheorem{definition}[theorem]{הגדרה}
\newtheorem*{definition*}{הגדרה}
\newtheorem{example}{דוגמה}[section]
\newtheorem{exercise}{תרגיל}[section]

\theoremstyle{c_remark}
\newtheorem*{remark}{הערה}
\newtheorem*{solution}{פתרון}
\newtheorem{conclusion}[theorem]{מסקנה}
\newtheorem{notation}[theorem]{סימון}

% Questions related commands
\newcounter{question}
\setcounter{question}{1}
\newcounter{sub_question}
\setcounter{sub_question}{1}

\newcommand{\question}[1][0]{
	\ifthenelse{#1 = 0}{}{\setcounter{question}{#1}}
	\subsection{שאלה \arabic{question}}
	\addtocounter{question}{1}
	\setcounter{sub_question}{1}
}

\newcommand{\subquestion}[1][0]{
	\ifthenelse{#1 = 0}{}{\setcounter{sub_question}{#1}}
	\subsubsection{סעיף \localecounter{letters.gershayim}{sub_question}}
	\addtocounter{sub_question}{1}
}

% import lua and start of document
\directlua{common = require ('../common')}

\GetEnv{AUTHOR}

% headers
\author{\AUTHOR}
\date\today

\title{פתרון מטלה 01 --- תורת הקבוצות האקסיומטית, 80650}

\begin{document}
\maketitle
\maketitleprint{}

\Question{}
נניח את Z --- Foundation, ותהי קבוצה טרנזיטיבית לא ריקה $x$, נניח גם שלכל $y \in x$ יש קבוצה טרנזיטיבית $z$ כך ש־$y \subseteq z \in x$, ולכל $y \in x$ גם $\mathcal{P}(y) \in x$.

\Subquestion{}
נוכיח ש־$\langle x, \in \rangle$ הוא מודל שמקיים את האקסיומות: ההיקפיות, הקבוצה הריקה, הפרדה והאיחוד.
\begin{proof}
	נוכיח את האקסיומות:
	\begin{itemize}
		\item היקפיות:
			אם $\varphi = \forall y, y' (y = y' \iff \forall z(z \in y \iff z \in y'))$ אז בהתאם $\varphi^x = \forall y, y' \in x (y = y' \iff \forall z \in x (z \in y \iff z \in y'))$. \\*
			זהו כמובן פסוק $\Delta_0$ ולכן נוכל להסיק שהוא מתקיים אם ורק אם $\varphi^x$ מתקיים, והוא אכן מתקיים.
		\item הקבוצה הריקה:
			נתון כי $x$ טרנזיטיבית ולכן $\in$ מגדיר עליה סדר טוב, בהתאם קיים $y \in x$ כך שלכל $z \in x$ גם $y \in z$, לכן $\exists y \in x (\forall z \in x (z \notin y))$ ולמעשה $\emptyset^x = y$.
		\item סכמת החלפה: תהי $y \in x$, אז גם $\PP(y) \in x$ ומטרנזיטיביות גם $\forall z \in z \in \PP(y) \in x \implies z \in x$. \\*
			עתה תהי $A$ מחלקה לא ריקה ב־$x$, אז כמובן $A^x$ מחלקה $\Delta_0$ ובהתאם נוכל לקבל ש־$y \cap A^x \in V$ וגם $y \cap A^x \subseteq y$, לכן $y \cap A^x \in \PP(y)$ אבל מצאנו כי במצב זה $y \cap A^c \in x$.
		\item איחוד:
			תהי $y \in x$, אז קיימת $z \in x$ טרנזיטיבית כך ש־$y \subseteq z$, לכל $a \in y$ מתקיים $a \in z$ ולכן לכל $b \in a$ גם $b \in z$, אם כך נשתמש באקסיומת החלפה על מחלקת האיחוד של $y$ על $z$ ונקבל $\bigcup y$.
	\end{itemize}
\end{proof}

\Question{}
נניח את אקסיומת היסוד ונראה ש־$\langle x, \in \rangle$ מודל המקיים גם את אקסיומת היסוד.
\begin{proof}
	נבחין כי אם $\varphi$ אקסיומת היסוד אז $\varphi$ מתקיים ולכן גם $\varphi^x$ מתקיים, אבל $\varphi^x$ הוא $\Delta_0$ ולכן גם $\langle x, \in \rangle$ מקיים אותו, ולכן $\langle x, \in \rangle \models \varphi$. \\*
	נוכל לראות גם שאם $\langle x, \in \rangle \not\models \varphi$ אז קיים $y \in x$ כך ש־$\langle x, \in \rangle \models \lnot \varphi(y)$ אבל $y \in V$ ולכן $\models \lnot \varphi(y)$ וזו סתירה לאקסיומה שהנחנו.
\end{proof}

\Question{}
נניח עתה שגם
\[
	\forall y, z \in x (y \cup z \in x)
\]
ונוכיח שהמודל מקיים את אקסיומת הזוג הלא סדור.
\begin{proof}
	נגדיר $\varphi(y) = \forall z, z' \in y (z = z')$ הנוסחה שקיים איבר יחיד ב־$y$ ($y$ יחידון), כמובן היא תקפה מאקסיומת ההיקפיות,
	ידוע גם ש$y \in x \implies \mathcal{P}(y) \in x$, ולכן מסכמת הפרדה גם מחלקת היחידונים מ־$y$ קבוצה ומטרנזיטיביות נקבל כי $\{ y \} \in x$, נעשה תהליך זה שוב עבור $z \in x$. \\*
	נקבל מהנוסחה הנתונה כי גם $\{ y \} \cup \{ z \} = \{ y, z \} \in x$ ולכן אקסיומת הזוג הלא סדור חלה.
\end{proof}
נקבל כי $\langle x, \in \rangle$ מקיים את Z --- Infinity.

\Question{}
נניח ש־$\omega \in x$ ונוכיח ש־$\langle x, \in \rangle$ מקיים את אקסיומת האינסוף.
\begin{proof}
	מאקסיומת השלמות קיים $\alpha \in \omega$ כך שחיתוכם זר ב־$x$.
	כמובן $\alpha < \omega$ מאקסיומת הפרדה עבור התכונה של איבר מינימלי נקבל כי $\emptyset \in x$, לכן $\emptyset^x = \emptyset$.
	אז נקבל מכל האקסיומות מלבד אינסוף שהעוקב לקבוצה הריקה, והעוקב ה־$n$־סופי של הקבוצה הריקה כולם מוכלים ב־$x$, ונתון כי גם $\omega$ מוכל בה, ולכן אקסיומת האינסוף מתקיימת.
\end{proof}

\Question{}
נניח עתה את ZF\@. נוכיח שלכל $\alpha$ סודר גבולי ככה ש־$\alpha \ge \omega + \omega$ ולכל $\varphi$ אקסיומה של Z, $\langle V_\alpha, \in \rangle \models \alpha$.
\begin{proof}
	יהי $\alpha$ כזה, נבחין כי $V_\alpha$ טרנזיטיבית (טענה מהכיתה) ונתון כי איננה ריקה. עוד ראינו כי לכל $y \in V_\alpha$ קיים $\beta \in \alpha$ כך ש־$y \subseteq V_\beta$,
	ומטרנזיטיביות $V_\beta \in V_\alpha$. לבסוף, גם $\mathcal{P}(y) \in V_{\alpha + 1}$ ולכן $\mathcal{P}(y) \in V_\alpha$, אז טענת שאלה 1 חלה וכך גם 2 (הנחנו יסוד). \\*
	אם $y, z \in V_\alpha$ אז קיימים $i, j < \alpha$ כך ש־$y \in V_i, z \in V_z$, וכן בהכרח $V_i \le V_j$ או הפוך, נניח $V_i \le V_j$ ולכן מטרנזיטיביות $y, z \in V_j$.
	בהתאם $\{ i, j \} \subseteq V_j$ כמחלקה, ולכן $\{ y, z \} \in V_{j + 1}$, לכן נוכל להסיק כי זו קבוצה, מטרנזיטיביות נקבל $\{ y, z \} \in V_\alpha$. \\*
	מצאנו אם כך שטענת שאלה 3 מתקיימת אף היא.
	ידוע כי $\alpha \ge \omega$ ולכן $\omega \in V_\alpha$ ולכן $\langle V_\alpha, \in \rangle$ מקיים את Z.
\end{proof}

\end{document}
