\newcounter{english}
\documentclass[a4paper]{article}

% packages
\usepackage{inputenc, amsmath, amsthm, thmtools, amsfonts, amssymb, luacode, catchfile, tikzducks, hyperref}
\usepackage[a4paper, margin=50pt, includeheadfoot]{geometry} % set page margins
\usepackage[shortlabels]{enumitem}
\usepackage[skip=3pt, indent=0pt]{parskip}

% language
\usepackage[bidi=basic, layout=tabular, provide=*]{babel}
\babelprovide[main, import]{hebrew}
\babelprovide{rl}
\babelfont{rm}{Libertinus Serif}
\babelfont{sf}{Libertinus Sans}
\babelfont{tt}{Libertinus Mono}

% style
\AddToHook{cmd/section/before}{\clearpage}	% Add line break before section
\linespread{1.3}
\setcounter{secnumdepth}{0}		% Remove default number tags from sections, this won't do well with theorems
\AtBeginDocument{\setlength{\belowdisplayskip}{3pt}}
\AtBeginDocument{\setlength{\abovedisplayskip}{3pt}}

% operators
\DeclareMathOperator\cis{cis}
\DeclareMathOperator\Sp{Sp}
\DeclareMathOperator\tr{tr}
\DeclareMathOperator\im{Im}
\DeclareMathOperator\re{Re}
\DeclareMathOperator\diag{diag}
\DeclareMathOperator*\lowlim{\underline{lim}}
\DeclareMathOperator*\uplim{\overline{lim}}
\DeclareMathOperator\rng{rng}
\DeclareMathOperator\Sym{Sym}
\DeclareMathOperator\Arg{Arg}
\DeclareMathOperator\Log{Log}
\DeclareMathOperator\dom{dom}

% commands
%\renewcommand\qedsymbol{\textbf{מש''ל}}
%\renewcommand\qedsymbol{\fbox{\emoji{lizard}}}
\newcommand{\NN}[0]{\mathbb{N}}
\newcommand{\ZZ}[0]{\mathbb{Z}}
\newcommand{\QQ}[0]{\mathbb{Q}}
\newcommand{\RR}[0]{\mathbb{R}}
\newcommand{\CC}[0]{\mathbb{C}}
\newcommand{\FF}[0]{\mathbb{F}}
\newcommand{\PP}[0]{\mathbb{P}}
\newcommand{\TT}[0]{\mathbb{T}}
\newcommand{\acts}[0]{\circlearrowright}
\newcommand{\explain}[2] {
	\begin{flalign*}
		 && \text{#2} && \text{#1}
	\end{flalign*}
}
\newcommand{\maketitleprint}[0]{ \begin{center}
	\begin{tikzpicture}[scale=3]
		\duck[graduate=gray!20!black, tassel=red!70!black]
	\end{tikzpicture}	
\end{center}
}

% theorem commands
\newtheoremstyle{c_remark}
	{}	% Space above
	{}	% Space below
	{}% Body font
	{}	% Indent amount
	{\bfseries}	% Theorem head font
	{}	% Punctuation after theorem head
	{.5em}	% Space after theorem head
	{\thmname{#1}\thmnumber{ #2}\thmnote{ \normalfont{\text{(#3)}}}}	% head content
\newtheoremstyle{c_definition}
	{3pt}	% Space above
	{3pt}	% Space below
	{}% Body font
	{}	% Indent amount
	{\bfseries}	% Theorem head font
	{}	% Punctuation after theorem head
	{.5em}	% Space after theorem head
	{\thmname{#1}\thmnumber{ #2}\thmnote{ \normalfont{\text{(#3)}}}}	% head content
\newtheoremstyle{c_plain}
	{3pt}	% Space above
	{3pt}	% Space below
	{\itshape}% Body font
	{}	% Indent amount
	{\bfseries}	% Theorem head font
	{}	% Punctuation after theorem head
	{.5em}	% Space after theorem head
	{\thmname{#1}\thmnumber{ #2}\thmnote{ \text{(#3)}}}	% head content

\theoremstyle{c_plain}
\newtheorem{theorem}{משפט}[section]
\newtheorem{lemma}[theorem]{למה}
\newtheorem{proposition}[theorem]{טענה}
\newtheorem*{proposition*}{טענה}
%\newtheorem{corollary}[theorem]{אין חלופה עברית}

\theoremstyle{c_definition}
\newtheorem{definition}[theorem]{הגדרה}
\newtheorem*{definition*}{הגדרה}
\newtheorem{example}{דוגמה}[section]
\newtheorem{exercise}{תרגיל}[section]

\theoremstyle{c_remark}
\newtheorem*{remark}{הערה}
\newtheorem*{solution}{פתרון}
\newtheorem{conclusion}[theorem]{מסקנה}
\newtheorem{notation}[theorem]{סימון}

% Questions related commands
\newcounter{question}
\setcounter{question}{1}
\newcounter{sub_question}
\setcounter{sub_question}{1}

\newcommand{\question}[1][0]{
	\ifthenelse{#1 = 0}{}{\setcounter{question}{#1}}
	\subsection{שאלה \arabic{question}}
	\addtocounter{question}{1}
	\setcounter{sub_question}{1}
}

\newcommand{\subquestion}[1][0]{
	\ifthenelse{#1 = 0}{}{\setcounter{sub_question}{#1}}
	\subsubsection{סעיף \localecounter{letters.gershayim}{sub_question}}
	\addtocounter{sub_question}{1}
}

% import lua and start of document
\directlua{common = require ('../common')}

\GetEnv{AUTHOR}

% headers
\author{\AUTHOR}
\date\today

\title{Exercise 4 Answer Sheet --- Axiomatic Set Theory, 80650}

\begin{document}
\maketitle
\maketitleprint{}

\question{}
Assuming ZF.\@

\subquestion{}
Assuming DC, We'll prove that a relation R of a set X is well founded if and only if there is no infinite sequence $\langle x_n \mid n < \omega \rangle \in X^\omega$ such that for all $n$, $x_{n + 1} R x_n$.
\begin{proof}
	We assume R is well founded.
	Let us assume for contradiction that $\langle x_n \mid n < \omega \rangle$ is a sequence such that $\forall n \in \omega, x_{n + 1} R x_n$.
	We define a subset $Y \subseteq X$ from the list by defining $x_n \in Y$ for all $n$. $R$ is well founded on $X$ therefore exists $y \in Y$ such that $\forall n, y R x_n \in Y$.
	$y \in Y$ then exists $m \in \omega$, $y = x_m$, then in particular $x_{m + 1} R x_m$ in contradiction to $R$ being well founded.

	In the opposite direction, let us assume there is no such sequence, we'll prove that $R$ is well founded as a relation over $X$. \\*
	Let $Y \subseteq X$. If there is an item $y \in Y$ such that there is no $z \in Y$ such that $z R y$ then $R$ is well founded, so we assume otherwise.
	For every element $y \in Y$ there is $z \in Y$ such that $z R y$ from out assumption, then by DC there is an infinite sequence $\langle y_n \mid n < \omega \rangle \in Y^\omega$ such that $y_{n + 1} R y_n$ for all $n$.
	Pay attention that DC was used on the reverse relation of $R$ instead of $R$ directly.
	From out initial assumption we get contradiction, then $R$ is indeed well founded.
\end{proof}

\subquestion{}
We'll show that DC is equivalent to the following version of the L\"owenheim-Skolem theorem: \\*
For every infinite structure $M$ over an countable language, there is a countable elementary substructure $M' \prec M$.
\begin{proof}
	Watch the recording to see what Yair says
\end{proof}

\question{}
Assuming ZF, Let $A$ be a class. We'll use Lev\'y Reflection to conclude the following version of Lev\'y Reflection for $A$: \\*
For every formula $\varphi$ with $n$ free variables, there is a closed unbounded class of ordinals $C_{A, \varphi}$ such that for every $\delta \in C_{A, \varphi}$ and $p_0, \dots, p_{n - 1} \in V_\delta \cap A$,
\[
	\langle A, \in \rangle \models \varphi(p_0, \dots, p_{n - 1})
	\iff \langle V_\delta \cap A, \in \rangle \models \varphi(p_0, \dots, p_{n - 1})
\]
\begin{proof}
	From Lev\'y Reflection there is $\alpha \in Ord$ such that,
	\[
		\forall a_0, \dots, a_{n - 1} \in V_\alpha, \varphi(a_0, \dots, a_{n - 1}) \iff V_\alpha \models \varphi(a_0, \dots, a_{n - 1})
	\]
	and there is a club $C_\varphi$ of such ordinals $\alpha$. It is clear that $C_{A, \varphi} \overset{def}{=} C_\varphi \cap A$ is a club as well, and let $\delta \in C_{A, \varphi}$. \\*
	Assume $p_0, \dots, p_{n - 1} \in V_\delta \cap A$ and that $A \models \varphi(p_0, \dots, p_{n - 1})$, then $p_0, \dots, p_{n - 1} \in V_\delta$ and what


	We can prove by induction on $A$ by defining $A = \{ \langle x_0, \dots, x_{m - 1} \rangle \in V \mid \psi(x_0, \dots, x_{m - 1}) \}$ and prove by induction on $\psi$. \\
	For atomic formulas, we can deduce that $\psi \in \{ (x = x), (x = y), (x \in y), (x \in x) \}$ and therefore $A = V, \emptyset$ and by Lev\'y reflection the statement is true. \\
	For $\psi_0, \psi_1$ we assume the statement is true, then $C_{A, \psi_0}, C_{A, \psi_1}$ exist fulfilling the statement,
	then we can define $C_{A, \psi_0} \cap C_{A, \psi_1}$ for $\psi_0 \land \psi_1$, the statement holds for similar reasons for its holding in the proof of 7.4 (in lecture notes).
	The same statement can be used for $\psi = \lnot \psi_0$. \\
	We assume the statement is true for $\psi_0$ and define $\psi = \exists v, \psi_0$ in order to finish the induction process.
	We can use the same argument from the proof of the theorem in its general form, as it requires only the existence of the clubs and their closeness.
\end{proof}

\question{}
\begin{definition*}[Almost Universal class]
	Let $A$ be a transitive class, we say that $A$ is almost universal if for every ordinal $\gamma$, $A \cap V_\gamma \in A$.
\end{definition*}
\begin{definition*}[$\Delta_0$-Separation]
	A class $A$ will be called a model of $\Delta_0$-Separation if for every $a, p_0, \dots, p_{n - 1} \in A$ and a $\Delta_0$ formula $\varphi(x, y_0, \dots, y_{n - 1})$,
	\[
		b = \{ w \in a \mid \varphi(w, p_0, \dots, p_{n - 1})\} \in A
	\]
\end{definition*}
Let $A$ be a transitive class which is almost universal, and let us assume that $A$ is a model of $\Delta_0$-Separation.

\subquestion{}
We will prove that $Ord \subseteq A$.
\begin{proof}
	We will prove the statement using induction over the ordinals. \\
	As an induction basis, notice that $A \cap V_\emptyset = \emptyset \in A$. \\
	Let us assume the statement is true for $\alpha$ and show that it is also true for $\alpha + 1$.
	We know $\alpha + 1 \in V_{\alpha}$, and using $\Delta_0$-Separation on $V_{\alpha + 1}$ we can deduce from the induction hypothesis that $\{\alpha\} \in A$, and from transitivity $\alpha + 1 \in A$ as well. \\
	Assume the statement is true for all $\beta < \alpha$ and prove it also true for $\alpha$.
	$\beta \in V_\alpha \cap A$ for every $\beta < \alpha$, therefore we can use $\Delta_0$-Separation using localized version of formula representing an ordinal,
	and conclude that $\alpha \in V_\alpha \cap A$, therefore $\alpha \in A$, completing the induction step.
\end{proof}
\begin{remark}
	$A$ is a model of Extensionality, Foundation, Empty Set and Infinity.
\end{remark}

\subquestion{}
We will show that $A$ is a model of Union and Power Set.
\begin{proof}
	Let $x \in A$ be a set, remember we defined $\bigcup x = \{ z \mid \exists y \in x, z \in y \}$, this formula is $\Delta_0$, then we can separate $\operatorname{trcl}(x)$ using it by the assumption of $\Delta_0$-Separation.
	Therefore $A$ is a model of Union.

	In a similar way, we defined $\mathcal{P}(x) = \{ y \mid y \subseteq x \}$, this is of course a $\Delta_0$ formula that can be used on the least cardinal $\operatorname{trcl}(x) < \kappa$.
	Therefore $A$ is also a model of Power Set.
\end{proof}

\subquestion{}
We'll prove that $A$ satisfies the Scheme of Replacement, and in particular Pairing and Separation.
\begin{proof}
	It seems to be the same exact thing with $\sup_{y \in x} \operatorname{trcl} y$ and another $\Delta_0$-Separation.
\end{proof}

\end{document}
