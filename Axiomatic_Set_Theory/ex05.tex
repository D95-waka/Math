\newcounter{english}
\documentclass[a4paper]{article}

% packages
\usepackage{inputenc, fontspec, amsmath, amsthm, amsfonts, polyglossia, catchfile}
\usepackage[a4paper, margin=50pt, includeheadfoot]{geometry} % set page margins

% style
\AddToHook{cmd/section/before}{\clearpage}	% Add line break before section
\linespread{1.5}
\setcounter{secnumdepth}{0}		% Remove default number tags from sections
\setmainfont{Libertinus Serif}
\setsansfont{Libertinus Sans}
\setmonofont{Libertinus Mono}
\setdefaultlanguage{hebrew}
\setotherlanguage{english}

% operators
\DeclareMathOperator\cis{cis}
\DeclareMathOperator\Sp{Sp}
\DeclareMathOperator\tr{tr}
\DeclareMathOperator\im{Im}
\DeclareMathOperator\diag{diag}
\DeclareMathOperator*\lowlim{\underline{lim}}
\DeclareMathOperator*\uplim{\overline{lim}}

% commands
\renewcommand\qedsymbol{\textbf{משל}}
\newcommand{\NN}[0]{\mathbb{N}}
\newcommand{\ZZ}[0]{\mathbb{Z}}
\newcommand{\QQ}[0]{\mathbb{Q}}
\newcommand{\RR}[0]{\mathbb{R}}
\newcommand{\CC}[0]{\mathbb{C}}
\newcommand{\getenv}[2][] {
  \CatchFileEdef{\temp}{"|kpsewhich --var-value #2"}{\endlinechar=-1}
  \if\relax\detokenize{#1}\relax\temp\else\let#1\temp\fi
}
\newcommand{\explain}[2] {
	\begin{flalign*}
		 && \text{#2} && \text{#1}
	\end{flalign*}
}

% headers
\getenv[\AUTHOR]{AUTHOR}
\author{\AUTHOR}
\date\today

\title{Exercise 5 Answer Sheet --- Axiomatic Set Theory, 80650}

\begin{document}
\maketitle
\maketitleprint{}

\question{}
Assuming ZF.\@

\subquestion{}
Assuming DC, We'll prove that a relation R of a set X is well founded if and only if there is no infinite sequence $\langle x_n \mid n < \omega \rangle \in X^\omega$ such that for all $n$, $x_{n + 1} R x_n$.
\begin{proof}
	We assume R is well founded.
	Let us assume for contradiction that $\langle x_n \mid n < \omega \rangle$ is a sequence such that $\forall n \in \omega, x_{n + 1} R x_n$.
	We define a subset $Y \subseteq X$ from the list by defining $x_n \in Y$ for all $n$. $R$ is well founded on $X$ therefore exists $y \in Y$ such that $\forall n, y R x_n \in Y$.
	$y \in Y$ then exists $m \in \omega$, $y = x_m$, then in particular $x_{m + 1} R x_m$ in contradiction to $R$ being well founded.

	In the opposite direction, let us assume there is no such sequence, we'll prove that $R$ is well founded as a relation over $X$. \\*
	Let $Y \subseteq X$. If there is an item $y \in Y$ such that there is no $z \in Y$ such that $z R y$ then $R$ is well founded, so we assume otherwise.
	For every element $y \in Y$ there is $z \in Y$ such that $z R y$ from out assumption, then by DC there is an infinite sequence $\langle y_n \mid n < \omega \rangle \in Y^\omega$ such that $y_{n + 1} R y_n$ for all $n$.
	Pay attention that DC was used on the reverse relation of $R$ instead of $R$ directly.
	From out initial assumption we get contradiction, then $R$ is indeed well founded.
\end{proof}

\subquestion{}
We'll show that DC is equivalent to the following version of the L\"owenheim-Skolem theorem: \\*
For every infinite structure $M$ over an countable language, there is a countable elementary substructure $M' \prec M$.
\begin{proof}
	Assume DC\@, and let $M$ be a model over countable language.
	If for every $x \in M$ there is $y \in M$ such that $M \models x \in y$, then by DC there is a well ordering over $M$ and it is possible to define Skol\'em functions over $M$,
	using Tarski-Vaught criteria there is $M' \prec M$ which is countable.
	If this is false, then there is a maximal element in $M$, and we can directly conclude that the required property is holding.

	Assume that for every infinite structure $M$ over countable language, there is a countable elementary substructure $M' \prec M$.
	Let $X$ be a set and $R \subseteq X^2$ such that for every $x \in X$, there is $y \in X$ such that $\langle x, y \rangle \in R$.
	Define $M = \langle X, R \rangle$ a model, then $M' \prec M$ a countable model.
	Define function $f : \NN \to M'$ such that $f(n) R^{M'} f(n + 1)$, then $\langle x_n \mid n < \omega \rangle$ defined by $x_n = f(0)$ exists, which is DC\@.
\end{proof}

\question{}
Assuming ZF, Let $A$ be a class. We'll use Lev\'y Reflection to conclude the following version of Lev\'y Reflection for $A$: \\*
For every formula $\varphi$ with $n$ free variables, there is a closed unbounded class of ordinals $C_{A, \varphi}$ such that for every $\delta \in C_{A, \varphi}$ and $p_0, \dots, p_{n - 1} \in V_\delta \cap A$,
\[
	\langle A, \in \rangle \models \varphi(p_0, \dots, p_{n - 1})
	\iff \langle V_\delta \cap A, \in \rangle \models \varphi(p_0, \dots, p_{n - 1})
\]
\begin{proof}
	According to localizations definition $A \models \varphi(p_0, \dots, p_{n - 1}) \iff \varphi^A(p_0, \dots, p_{n - 1})$.
	By the same reasoning $V_\delta \cap A \models \varphi(p_0, \dots, p_{n - 1}) \iff V_\delta \models \varphi^A(p_0, \dots, p_{n - 1})$ as well.
	Therefore it is sufficient to prove that $\varphi^A(p_0, \dots, p_{n - 1}) \iff V_\delta \models \varphi^A(p_0, \dots, p_{n - 1})$.
	This proposition is directly concluded using Lev\'y Reflection, as well that there is a club class, denoted $C_{A, \varphi}$, such that for every $\delta \in C_{A, \varphi}$ the result holds.


	Another proof using induction on $A$'s definition: \\
	We can prove by induction on $A$ by defining $A = \{ \langle x_0, \dots, x_{m - 1} \rangle \in V \mid \psi(x_0, \dots, x_{m - 1}) \}$ and prove by induction on $\psi$. \\
	For atomic formulas, we can deduce that $\psi \in \{ (x = x), (x = y), (x \in y), (x \in x) \}$ and therefore $A = V, \emptyset$ and by Lev\'y reflection the statement is true. \\
	For $\psi_0, \psi_1$ we assume the statement is true, then $C_{A, \psi_0}, C_{A, \psi_1}$ exist fulfilling the statement,
	then we can define $C_{A, \psi_0} \cap C_{A, \psi_1}$ for $\psi_0 \land \psi_1$, the statement holds for similar reasons for its holding in the proof of 7.4 (in lecture notes).
	The same statement can be used for $\psi = \lnot \psi_0$. \\
	We assume the statement is true for $\psi_0$ and define $\psi = \exists v, \psi_0$ in order to finish the induction process.
	We can use the same argument from the proof of the theorem in its general form, as it requires only the existence of the clubs and their closeness.
\end{proof}

\question{}
\begin{definition*}[Almost Universal class]
	Let $A$ be a transitive class, we say that $A$ is almost universal if for every ordinal $\gamma$, $A \cap V_\gamma \in A$.
\end{definition*}
\begin{definition*}[$\Delta_0$-Separation]
	A class $A$ will be called a model of $\Delta_0$-Separation if for every $a, p_0, \dots, p_{n - 1} \in A$ and a $\Delta_0$ formula $\varphi(x, y_0, \dots, y_{n - 1})$,
	\[
		b = \{ w \in a \mid \varphi(w, p_0, \dots, p_{n - 1})\} \in A
	\]
\end{definition*}
Let $A$ be a transitive class which is almost universal, and let us assume that $A$ is a model of $\Delta_0$-Separation.

\subquestion{}
We will prove that $Ord \subseteq A$.
\begin{proof}
	We will prove the statement using induction over the ordinals. \\
	As an induction basis, notice that $A \cap V_\emptyset = \emptyset \in A$. \\
	Let us assume the statement is true for $\alpha$ and show that it is also true for $\alpha + 1$.
	We know $\alpha + 1 \in V_{\alpha}$, and using $\Delta_0$-Separation on $V_{\alpha + 1}$ we can deduce from the induction hypothesis that $\{\alpha\} \in A$, and from transitivity $\alpha + 1 \in A$ as well. \\
	Assume the statement is true for all $\beta < \alpha$ and prove it also true for $\alpha$.
	$\beta \in V_\alpha \cap A$ for every $\beta < \alpha$, therefore we can use $\Delta_0$-Separation using localized version of formula representing an ordinal,
	and conclude that $\alpha \in V_\alpha \cap A$, therefore $\alpha \in A$, completing the induction step.
\end{proof}
\begin{remark}
	$A$ is a model of Extensionality, Foundation, Empty Set and Infinity.
\end{remark}

\subquestion{}
We will show that $A$ is a model of Union and Power Set.
\begin{proof}
	We'll show that $A$ is a model of Union.
	Let $x \in A$ and let $\alpha$ be an ordinal such that for all $y \in \bigcup x$, $y \in V_\alpha$.
	For every $z \in x \cap A$, it follows that $y \in z \implies y \in V_\alpha^A = b$.
	But $b \in A$, then $A \models \exists y \in x, z \in y$ for every $y \in x^A$, then by $\Delta_0$-Separation using this formula, $A$ is a model of Union.

	We'll show that $A$ is a model of Power Set as well.
	Let $\alpha$ be an ordinal such that for every subset $y \subseteq x$, $y \in V_\alpha$.
	Again we can assume $y \subseteq x \cap A \implies y \in V_\alpha^A$, then $A \models y \subseteq x \implies y \in V_\alpha$.
	Using our version of separation it directly concluded that $A$ is indeed a model of Power Set.
\end{proof}

\subquestion{}
We'll prove that $A$ satisfies the Scheme of Replacement, and in particular Pairing and Separation.
\begin{proof}
	Let $F : A \to A$ class function and $x \in A$.
	Let us define $B = \{ F(y) \mid y \in x \}$, we want to show that $B \in A$.
	Define $\alpha$ as the least ordinal such that $B \subseteq V_\alpha$, we want to show that $A \models B \subseteq V_\alpha$.
	From Lev\'y Reflection theorem we conclude that there is a club $C_F$ of ordinals fulfilling the theorem, in particular there is $\delta \in C_F$ such that $\alpha < \delta$, therefore $V_\alpha^A \subseteq V_\delta^A \in A$.
	Of course $B \cap A \cap V_\delta = B \cap V_\delta = B$, then $V_\delta^A$ thinks that $B$ definable and exist (using $\Delta_0$-Separation), then $A$ is a model of Replacement.

	We can conclude that Separation holds as well, as we can use $F = Id$ with replacement. \\
	Pairing holds as well, using $2 = \{0, 1\}$ and the class function $1 \mapsto x, 2 \mapsto y$.
\end{proof}

\end{document}
