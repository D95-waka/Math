\newcounter{englishFlag}
\documentclass[a4paper]{article}

% packages
\usepackage{inputenc, amsmath, amsthm, thmtools, amsfonts, amssymb, luacode, catchfile, tikzducks, hyperref}
\usepackage[a4paper, margin=50pt, includeheadfoot]{geometry} % set page margins
\usepackage[shortlabels]{enumitem}
\usepackage[skip=3pt, indent=0pt]{parskip}

% language
\usepackage[bidi=basic, layout=tabular, provide=*]{babel}
\babelprovide[main, import]{hebrew}
\babelprovide{rl}
\babelfont{rm}{Libertinus Serif}
\babelfont{sf}{Libertinus Sans}
\babelfont{tt}{Libertinus Mono}

% style
\AddToHook{cmd/section/before}{\clearpage}	% Add line break before section
\linespread{1.3}
\setcounter{secnumdepth}{0}		% Remove default number tags from sections, this won't do well with theorems
\AtBeginDocument{\setlength{\belowdisplayskip}{3pt}}
\AtBeginDocument{\setlength{\abovedisplayskip}{3pt}}

% operators
\DeclareMathOperator\cis{cis}
\DeclareMathOperator\Sp{Sp}
\DeclareMathOperator\tr{tr}
\DeclareMathOperator\im{Im}
\DeclareMathOperator\re{Re}
\DeclareMathOperator\diag{diag}
\DeclareMathOperator*\lowlim{\underline{lim}}
\DeclareMathOperator*\uplim{\overline{lim}}
\DeclareMathOperator\rng{rng}
\DeclareMathOperator\Sym{Sym}
\DeclareMathOperator\Arg{Arg}
\DeclareMathOperator\Log{Log}
\DeclareMathOperator\dom{dom}

% commands
%\renewcommand\qedsymbol{\textbf{מש''ל}}
%\renewcommand\qedsymbol{\fbox{\emoji{lizard}}}
\newcommand{\NN}[0]{\mathbb{N}}
\newcommand{\ZZ}[0]{\mathbb{Z}}
\newcommand{\QQ}[0]{\mathbb{Q}}
\newcommand{\RR}[0]{\mathbb{R}}
\newcommand{\CC}[0]{\mathbb{C}}
\newcommand{\FF}[0]{\mathbb{F}}
\newcommand{\PP}[0]{\mathbb{P}}
\newcommand{\TT}[0]{\mathbb{T}}
\newcommand{\acts}[0]{\circlearrowright}
\newcommand{\explain}[2] {
	\begin{flalign*}
		 && \text{#2} && \text{#1}
	\end{flalign*}
}
\newcommand{\maketitleprint}[0]{ \begin{center}
	\begin{tikzpicture}[scale=3]
		\duck[graduate=gray!20!black, tassel=red!70!black]
	\end{tikzpicture}	
\end{center}
}

% theorem commands
\newtheoremstyle{c_remark}
	{}	% Space above
	{}	% Space below
	{}% Body font
	{}	% Indent amount
	{\bfseries}	% Theorem head font
	{}	% Punctuation after theorem head
	{.5em}	% Space after theorem head
	{\thmname{#1}\thmnumber{ #2}\thmnote{ \normalfont{\text{(#3)}}}}	% head content
\newtheoremstyle{c_definition}
	{3pt}	% Space above
	{3pt}	% Space below
	{}% Body font
	{}	% Indent amount
	{\bfseries}	% Theorem head font
	{}	% Punctuation after theorem head
	{.5em}	% Space after theorem head
	{\thmname{#1}\thmnumber{ #2}\thmnote{ \normalfont{\text{(#3)}}}}	% head content
\newtheoremstyle{c_plain}
	{3pt}	% Space above
	{3pt}	% Space below
	{\itshape}% Body font
	{}	% Indent amount
	{\bfseries}	% Theorem head font
	{}	% Punctuation after theorem head
	{.5em}	% Space after theorem head
	{\thmname{#1}\thmnumber{ #2}\thmnote{ \text{(#3)}}}	% head content

\theoremstyle{c_plain}
\newtheorem{theorem}{משפט}[section]
\newtheorem{lemma}[theorem]{למה}
\newtheorem{proposition}[theorem]{טענה}
\newtheorem*{proposition*}{טענה}
%\newtheorem{corollary}[theorem]{אין חלופה עברית}

\theoremstyle{c_definition}
\newtheorem{definition}[theorem]{הגדרה}
\newtheorem*{definition*}{הגדרה}
\newtheorem{example}{דוגמה}[section]
\newtheorem{exercise}{תרגיל}[section]

\theoremstyle{c_remark}
\newtheorem*{remark}{הערה}
\newtheorem*{solution}{פתרון}
\newtheorem{conclusion}[theorem]{מסקנה}
\newtheorem{notation}[theorem]{סימון}

% Questions related commands
\newcounter{question}
\setcounter{question}{1}
\newcounter{sub_question}
\setcounter{sub_question}{1}

\newcommand{\question}[1][0]{
	\ifthenelse{#1 = 0}{}{\setcounter{question}{#1}}
	\subsection{שאלה \arabic{question}}
	\addtocounter{question}{1}
	\setcounter{sub_question}{1}
}

\newcommand{\subquestion}[1][0]{
	\ifthenelse{#1 = 0}{}{\setcounter{sub_question}{#1}}
	\subsubsection{סעיף \localecounter{letters.gershayim}{sub_question}}
	\addtocounter{sub_question}{1}
}

% import lua and start of document
\directlua{common = require ('../common')}

\GetEnv{AUTHOR}

% headers
\author{\AUTHOR}
\date\today

\title{פתרון מטלה 04 --- תורת הקבוצות האקסיומטית, 80650}

\DeclareMathOperator{\trcl}{trcl}
\DeclareMathOperator{\rank}{rank}

\begin{document}
\maketitle
\maketitleprint{}

\question{}
נניח ZFC, ונניח גם $\kappa$ מונה רגולרי לא בן־מניה, נבחין כי מהמטלה הקודמת נובע $H(k^+)$ הוא מודל של ZFC --- Power set.

יהי $p \in H(\kappa^+)$, נראה שקיים סל''ח $C \subseteq \kappa$ כך שלכל $\delta \in C$ קיים $M \prec H(\kappa^+)$ כך ש־$p \in M$ וגם $M \cap \kappa = \delta$.
\begin{proof}
	מצאנו בהרצאה כי קבוצת הנוסחות של תורת הקבוצות היא בת־מניה, נגדיר לכל נוסחה $\varphi$ את פונקציית סקולם שלה $f_\varphi$ שמצאנו שקיימת (שימוש בבחירה). \\*
	נבנה סדרה בת־מניה $\langle G_n \mid n < \omega \rangle$ של פונקציות סקולם סגורה להרכבה, נבחין כי היא אכן סגורה להרכבה כהוכחה באינדוקציה על מבנה הפסוק. \\*
	נגדיר $S(\alpha) = \{G_n(\gamma_0, \dots, \gamma_{m - 1}, p) \mid n < \omega, \gamma_0, \dots, \gamma_{m - 1} < \alpha\}$ ונבחין כי לכל $\alpha < \kappa$ מתקיים $S(\gamma) \prec H(\kappa^+)$ ממשפט 6.3. \\*
	נבחין כי $|S(\alpha)| \le \omega \times |\alpha| \le \kappa$ ולכן גם $|S(\alpha)| = |S(\alpha) \cap \kappa|$, לכן נגדיר $R(\alpha) = S(\alpha) \cap \kappa$ עדיין מגדיר תת־מודלים אלמנטריים. \\*
	עוד נראה כי לכל $p \in R(\alpha)$ מהסגירות לנוסחות, ונגדיר. \\*
	לבסוף נגדיר $h(\alpha) = \sup R(\alpha)$ וגם $C = \rng h$, ונקבל ש־$\sup C = \kappa$ וכן ש־$C$ סגורה, זאת ישירות מההגדרה, ולכן מצאנו כי זהו סל''ח המקיים את כל הנדרש.
\end{proof}

\question{}
יהי $M \prec N = H(\kappa^+)$ ויהיו $a_0, \dots, a_{n - 1} \in H(\kappa^+)$. \\*
נוכיח ש־$\{a_0, \dots, a_{n - 1}\} \in M$ אם ורק אם $\forall i < n, a_i \in M$. \\*
נראה גם שטענה זו לא בהכרח תתקיים עבור קבוצה בת־מניה של איברים.
\begin{proof}
	נניח ש־$a = \{a_0, \dots, a_{n - 1}\} \in M$.
	נבחן את $\varphi(x, y) = x \in y$, ידוע ש־$N \models \varphi(a_i, a)$, ולכן בפרט $\exists x, \varphi(x, a)$ ומהשיכון הנתון גם $M \models \exists x, \varphi(x, a)$,
	אבל בהתאם $N \models \varphi(x, a)$ ולכן $x = a_i$ עבור $0 \le i < n$, לכן $M \models a_i \in a$. נוכל להמשיך עתה בתהליך זה ולבנות $n$ נוסחות ובכך נשלים $M \models \forall i, a_i \in a$.

	נבחין כי תהליך זה בדיוק עלול להיכשל במקרים שאינם סופיים.
	למעשה, אנו יודעים כי קיים מודל $|M| = \omega$ עבור $H(\kappa^+)$ ולכן אם נבחר את $\omega_1$ ונגדיר עם סעיף א' שהוא מוכל ב־$M$, נקבל סתירה ל־$|M| = \omega$.
\end{proof}

\question{}
תהי $S \subseteq \kappa$, ונראה ש־$S$ קבוצת שבת אם ורק אם לכל $p \in H(\kappa^+)$ קיים תת־מבנה אלמנטרי $M \prec H(\kappa^+)$ כך ש־$p \in M$ וגם $M \cap \kappa \in S$.
\begin{proof}
	נניח ש־$S$ קבוצת שבת ויהי $p \in H(\kappa^+)$. \\*
	עוד נגדיר $C$ הסל''ח שקיים עבור $p$ לפי שאלה 1, ויהי $\delta \in C \cap S$, אז קיים $M \prec H(\kappa^+)$ עבורו $p \in M$ וגם $M \cap \kappa = \delta \in S$.

	נניח עתה כי לכל $p \in H(\kappa^+)$ קיים תת־מבנה אלמנטרי $M \prec H(\kappa^+)$ כך ש־$p \in M$ וגם $M \cap \kappa \in S$ ונראה ש־$S$ קבוצת שבת. \\*
	יהי סל''ח $C \subseteq \kappa$ ויהי $\alpha \in C$ סודר גבולי שסגור ב־$C$. אז קיים בהתאם מודל $\alpha \in M \prec H(\kappa^+)$, נבנה סדרה לא חסומה ב־$M$ של $\alpha$ על־ידי הגדרת נוסחות מהצורה $\varphi(x, y) = x < y$. \\*
	שימוש בפונקציות סקולם מתאימות ושימוש בתכונות ב־$H(\kappa^+)$ של $\alpha$ יניב לנו סדרה עולה לא חסומה ב־$M$ של סודרים, ובהתאם נוכל להסיק $M \cap p \cap C = M \cap C$ מקיים סגירות ולכן $M \cap \kappa \in C$. \\*
	מצד שני $M \cap \kappa \in S$ ולכן בפרט $S \cap C \ne \emptyset$ ומצאנו כי $S$ אכן קבוצת שבת.

\end{proof}

\question{}
יהי $M \prec H(\kappa^+)$ כך ש־$M \cap \kappa = \delta \in \kappa$. \\*
נוכיח שלכל $C \in M$ סל''ח ב־$\kappa$, מתקיים $\delta \in C$.
\begin{proof}
	נבחין כי $\delta \cap C = M \cap \kappa \cap C = M \cap C$, ונבחן את $\varphi(x, y) = \sup(x) = y$, אנו יודעים כי $\varphi(C, \kappa)$ מההגדרה של $C$. \\*
	נשים לב כי אם $\kappa \in M$ אז $\kappa = \delta \in \delta$ בסתירה להנחה שלנו. \\*
	אם $\sup(\delta \cap C) = \delta$ אז $\delta \in C$ כפי שרצינו ולכן נניח אחרת,
	אבל אז בהכרח $\sup(\delta \cap C) < \delta$ שכן אם $\sup(\delta \cap C) > \delta$ זו תהיה סתירה לאקסיומת היסוד. \\*
	בנוסף מהשוויון שמצאנו נובע $\sup(M \cap C) = \sup(\delta \cap C) < \delta$, דהינו $M \models \varphi(C, \alpha)$ עבור $\alpha < \delta$ כלשהו. \\*
	אבל $M \models \varphi(C, \alpha) \iff H(\kappa^+) \models \varphi(C, \alpha)$, ו־$C$ סל''ח ולכן $\sup C = \kappa$, אז $\alpha = \kappa$ וקיבלנו סתירה להנחה, לכן $\delta \in C$.
\end{proof}

\question{}
תהי $S \subseteq \kappa$ קבוצת שבת ותהי $f : S \to \kappa$ פונקציה יורדת. \\*
יהי $M \prec H(\kappa^+)$ כך ש־$M \cap \kappa = \delta \in S$ ו־$\{f, S\} \in M$. \\*
נגדיר $f(\delta) = \gamma$, נוכיח ש־$X = \{ \alpha \in S \mid f(\alpha) = \gamma \}$ היא קבוצת שבת ותת־קבוצה של $\kappa$. \\*
נסיק שהלמה של פודור חלה.
\begin{proof}
	אנו כבר יודעים מההגדרה ש־$X \subseteq \kappa$, ונניח בשלילה שהיא לא קבוצת שבת, לכן קיים $C \subseteq \kappa$ סל''ח, כך ש־$C \cap X = \emptyset$. \\*
	ידוע ש־$M \cap \kappa = \delta \in S \subseteq \kappa \implies M \cap \kappa \in \kappa$ ולכן תנאי השאלה הקודמת חלים ומתקיים $\delta \in C$. \\*
	$f(\delta) = \gamma$ ולכן $\delta \in X$ ובהתאם $X \cap \delta \ne \emptyset$ בסתירה להנחת השלילה.
\end{proof}

\end{document}
