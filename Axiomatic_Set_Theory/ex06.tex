\newcounter{english}
\documentclass[a4paper]{article}

% packages
\usepackage{inputenc, amsmath, amsthm, thmtools, amsfonts, amssymb, luacode, catchfile, tikzducks, hyperref}
\usepackage[a4paper, margin=50pt, includeheadfoot]{geometry} % set page margins
\usepackage[shortlabels]{enumitem}
\usepackage[skip=3pt, indent=0pt]{parskip}

% language
\usepackage[bidi=basic, layout=tabular, provide=*]{babel}
\babelprovide[main, import]{hebrew}
\babelprovide{rl}
\babelfont{rm}{Libertinus Serif}
\babelfont{sf}{Libertinus Sans}
\babelfont{tt}{Libertinus Mono}

% style
\AddToHook{cmd/section/before}{\clearpage}	% Add line break before section
\linespread{1.3}
\setcounter{secnumdepth}{0}		% Remove default number tags from sections, this won't do well with theorems
\AtBeginDocument{\setlength{\belowdisplayskip}{3pt}}
\AtBeginDocument{\setlength{\abovedisplayskip}{3pt}}

% operators
\DeclareMathOperator\cis{cis}
\DeclareMathOperator\Sp{Sp}
\DeclareMathOperator\tr{tr}
\DeclareMathOperator\im{Im}
\DeclareMathOperator\re{Re}
\DeclareMathOperator\diag{diag}
\DeclareMathOperator*\lowlim{\underline{lim}}
\DeclareMathOperator*\uplim{\overline{lim}}
\DeclareMathOperator\rng{rng}
\DeclareMathOperator\Sym{Sym}
\DeclareMathOperator\Arg{Arg}
\DeclareMathOperator\Log{Log}
\DeclareMathOperator\dom{dom}

% commands
%\renewcommand\qedsymbol{\textbf{מש''ל}}
%\renewcommand\qedsymbol{\fbox{\emoji{lizard}}}
\newcommand{\NN}[0]{\mathbb{N}}
\newcommand{\ZZ}[0]{\mathbb{Z}}
\newcommand{\QQ}[0]{\mathbb{Q}}
\newcommand{\RR}[0]{\mathbb{R}}
\newcommand{\CC}[0]{\mathbb{C}}
\newcommand{\FF}[0]{\mathbb{F}}
\newcommand{\PP}[0]{\mathbb{P}}
\newcommand{\TT}[0]{\mathbb{T}}
\newcommand{\acts}[0]{\circlearrowright}
\newcommand{\explain}[2] {
	\begin{flalign*}
		 && \text{#2} && \text{#1}
	\end{flalign*}
}
\newcommand{\maketitleprint}[0]{ \begin{center}
	\begin{tikzpicture}[scale=3]
		\duck[graduate=gray!20!black, tassel=red!70!black]
	\end{tikzpicture}	
\end{center}
}

% theorem commands
\newtheoremstyle{c_remark}
	{}	% Space above
	{}	% Space below
	{}% Body font
	{}	% Indent amount
	{\bfseries}	% Theorem head font
	{}	% Punctuation after theorem head
	{.5em}	% Space after theorem head
	{\thmname{#1}\thmnumber{ #2}\thmnote{ \normalfont{\text{(#3)}}}}	% head content
\newtheoremstyle{c_definition}
	{3pt}	% Space above
	{3pt}	% Space below
	{}% Body font
	{}	% Indent amount
	{\bfseries}	% Theorem head font
	{}	% Punctuation after theorem head
	{.5em}	% Space after theorem head
	{\thmname{#1}\thmnumber{ #2}\thmnote{ \normalfont{\text{(#3)}}}}	% head content
\newtheoremstyle{c_plain}
	{3pt}	% Space above
	{3pt}	% Space below
	{\itshape}% Body font
	{}	% Indent amount
	{\bfseries}	% Theorem head font
	{}	% Punctuation after theorem head
	{.5em}	% Space after theorem head
	{\thmname{#1}\thmnumber{ #2}\thmnote{ \text{(#3)}}}	% head content

\theoremstyle{c_plain}
\newtheorem{theorem}{משפט}[section]
\newtheorem{lemma}[theorem]{למה}
\newtheorem{proposition}[theorem]{טענה}
\newtheorem*{proposition*}{טענה}
%\newtheorem{corollary}[theorem]{אין חלופה עברית}

\theoremstyle{c_definition}
\newtheorem{definition}[theorem]{הגדרה}
\newtheorem*{definition*}{הגדרה}
\newtheorem{example}{דוגמה}[section]
\newtheorem{exercise}{תרגיל}[section]

\theoremstyle{c_remark}
\newtheorem*{remark}{הערה}
\newtheorem*{solution}{פתרון}
\newtheorem{conclusion}[theorem]{מסקנה}
\newtheorem{notation}[theorem]{סימון}

% Questions related commands
\newcounter{question}
\setcounter{question}{1}
\newcounter{sub_question}
\setcounter{sub_question}{1}

\newcommand{\question}[1][0]{
	\ifthenelse{#1 = 0}{}{\setcounter{question}{#1}}
	\subsection{שאלה \arabic{question}}
	\addtocounter{question}{1}
	\setcounter{sub_question}{1}
}

\newcommand{\subquestion}[1][0]{
	\ifthenelse{#1 = 0}{}{\setcounter{sub_question}{#1}}
	\subsubsection{סעיף \localecounter{letters.gershayim}{sub_question}}
	\addtocounter{sub_question}{1}
}

% import lua and start of document
\directlua{common = require ('../common')}

\GetEnv{AUTHOR}

% headers
\author{\AUTHOR}
\date\today

\title{Exercise 6 Answer Sheet --- Axiomatic Set Theory, 80650}

\begin{document}
\maketitle
\maketitleprint{}

\question{}
Let $\lambda$ be an infinite cardinal.
A $(\lambda^+, \lambda)$-Ulam matric is a collection of sets $\langle A_{\alpha, \rho} \mid \alpha < \lambda^+, \rho < \lambda \rangle$ such that,
\begin{enumerate}
	\item For every $\alpha < \beta < \lambda^+$ and $\rho < \lambda$, $A_{\alpha, \rho} \cap A_{\beta, \rho} = \emptyset$.
	\item For every $\alpha < \lambda^+$, $|\lambda^+ \setminus (\bigcup_\rho A_{\alpha, \rho})| \le \lambda$.
\end{enumerate}

\subquestion{}
We will show that for every infinite $\lambda$, a $(\lambda^+, \lambda)$-Ulam matrix exists.
\begin{proof}
	Let us define for each $0 < \xi < \lambda^+$ a surjection $f_\xi : \lambda \to \xi$ by $f_\xi(x) = x$ if $x < \xi$ and to arbitrary value otherwise. \\
	Define $A_{\alpha, \rho} = \{ \xi < \lambda^+ \mid f_\xi(\rho) = \alpha \}$, and we will show that this definition is fulfilling Ulam matrix definition. \\
	Let $\alpha < \beta < \lambda^+$ and let us fix $\rho < \lambda$, then
	\[
		A_{\alpha, \rho} \cap A_{\beta, \rho}
		= \{ \xi < \lambda^+ \mid f_\xi(\rho) = \alpha \} \cap \{ \xi < \lambda^+ \mid f_\xi(\rho) = \beta \}
		= \{ \xi < \lambda^+ \mid f_\xi(\rho) = \alpha = \beta \}
		= \emptyset
	\]
	Let us fix $\alpha < \lambda^+$, then
	\[
		\left\lvert \lambda^+ \setminus \left(\bigcup_\rho A_{\alpha, \rho}\right) \right\rvert
		= \left\lvert \lambda^+ \setminus \left(\bigcup_\rho \{ \xi < \lambda^+ \mid f_\xi(\rho) = \alpha \} \right) \right\rvert
		= \left\lvert \lambda^+ \setminus \{ \xi < \lambda^+ \mid f_\xi(\lambda) \ni \alpha \} \right\rvert
	\]
	But for each $\alpha$, $f_{\alpha + 1}(\alpha) = \alpha$ we can deduce
	\[
		\left\lvert \lambda^+ \setminus \{ \xi < \lambda^+ \mid f_\xi(\lambda) \ni \alpha \} \right\rvert
		\left\lvert \lambda^+ \setminus \{ \xi < \lambda^+ \mid \alpha + 1 < \lambda^+ \} \right\rvert
		\le \lambda
	\]
\end{proof}

\subquestion{}
Let $\kappa$ be the least cardinal such that there is a $\sigma$-additive, non-trivial, non-atomic measure $\mu$ with $\dom \mu = \mathcal{P}(\kappa)$. \\
We will prove that $\kappa$ is not a successor cardinal.
\begin{proof}
	Let us assume for contradiction that $\kappa$ is indeed a successor cardinal such that $\lambda^+ = \kappa$. \\
	By the last part there is a $(\lambda^+, \lambda)$-Ulam matrix, A, for this specified $\lambda$. \\
	Fixing $\alpha < \lambda^+$, we will find $\rho < \lambda$ such that $\mu(A_{\alpha, \rho}) > 0$.
	We know that $|\kappa \setminus \bigcup_\rho A_{\alpha, \rho}| \le \lambda$,
	then the assumption that all these elements of $A$ fulfilling $A_{\alpha, \rho} \le \lambda$ would lead to contradiction, as their union would be $< \kappa$.
	Then there is an element $A_{\alpha, \rho} > \lambda$ for every $\alpha < \kappa$, for each for these $\mu(A_{\alpha, \rho}) > 0$ as $\mu$ is non-atomic.
	Let $\gamma = \{ \rho < \lambda \mid \alpha < \kappa, \mu(A_{\alpha, \rho}) > 0 \}$, then define $B_\alpha = \bigcup_{\rho \in \gamma} A_{\alpha, \rho}$.
	$B_\alpha \cap B_\beta = \emptyset$ for every $\alpha < \beta < \kappa$ as deducted from Ulam matrix, and then by $\sigma$-additivity of $\mu$ we get contradiction to $\mu(\bigcup B_\alpha) \le 1$.
	By the contradiction it is followed that there is no such $\lambda$, meaning $\kappa$ is not a successor cardinal.
\end{proof}

\question{}
Let $\kappa$ be an uncountable regular cardinal such that there is non-principle filter $\mathcal{F} \subseteq \mathcal{P}(\kappa)$ with the following properties,
\begin{enumerate}
	\item For every $\langle x_\alpha \in \mathcal{F} \mid \alpha < \kappa \rangle$ also $\bigcap_{\alpha < \kappa} x_\alpha \in \mathcal{F}$.
	\item For every collection $\{ X_\alpha \mid \alpha < \omega_1 \} \subseteq \mathcal{P}(\kappa)$ such that $\forall \alpha, \kappa \setminus X_\alpha \notin \mathcal{F}$,
		there are $\alpha < \beta$ such that $X_\alpha \cap X_\beta \ne \emptyset$.
\end{enumerate}
Such an $\mathcal{F}$ is called non-trivial $\sigma$-saturated $\kappa$-complete filter on $\kappa$. \\
We will show that either there is a $\kappa$-complete ultrafilter on $\kappa$ or $\kappa \le 2^{\aleph_0}$ and $\kappa$ is a limit cardinal.
\begin{proof}
	Let $\Ff^+ = \{X \subseteq \kappa \mid \kappa \setminus X \notin \Ff\}$, This set represent the elements of $\Ff$ which are non zero in a sense, a positive subset of the filter.

	Let us assume that for every $B \subseteq A$, $B \in \Ff^+$ or $A \setminus B \in \Ff^+$, this is in a sense the atomic case, in which there is a set that acts as an atom.
	We will show that in this case there is a $\kappa$-complete ultrafilter on $\kappa$.
	Let us define $A$ an atom of $\Ff^+$, from the assumption we made it is clear that there is an atom, such can be constructed by intersecting decreasing series of sets that are all in $\Ff^+$.
	Define $\Uu = \{ x \in \Ff^+ \mid A \subseteq x \}$, this is an ultrafilter which is $\kappa$-complete as required\footnote{Jech T. Set Theory. 2003, 1, 77.}

	Assuming the contrary of our initial assumption, it is directly follows that for every $A \in \Ff^+$ there is $B \subseteq A$ such that $B, A \setminus B \in \Ff^+$.
	This is case is in a sense non-atomic, as for each positive-measure set there is a split of disjoint positive-measure subsets.
	Let us define a left centered standard binary tree $\langle T, f \rangle$ such that $f : 2^{\omega_1} \to \Ff^+$, defined by $f(\langle \rangle) = \kappa$ and for each $t \in \dom f$,
	if $f(t) = A$, from the assumption there is $B \subseteq A$ satisfying the noted property, then we define $f(t \frown \langle 0 \rangle) = B, f(t \frown \langle 1 \rangle) = A \setminus B$.
	Let us define $\operatorname{level}_\alpha(\langle T, f \rangle) = \{ f(t) \mid \operatorname{len}(t) = \alpha$, and $M = \operatorname{level}_{\omega_1}$.
	We know that $|\RR| \le \omega_1$, then by the non-atomic property of $\mu$ we conclude that $\mu(f(t)) > 0$ for all $t \in \dom f$, but then $\sum M = \infty$ and in particular larger than $1$.
	This is of course a contradiction to the implicit assumption that $\kappa > \omega_1$, then we conclude directly $\kappa \le 2^{\aleph_0}$.
	%
	%\textbf{Self Note}:
	%We will construct a binary tree of $2^{\aleph_0}$ of level wise disjoint sets such that their measure is positive. After $\omega_1$ splits it will be a contradiction that the measure is positive,
	%forcing the size of $\kappa$ to be less than $2^{\aleph_0}$.
\end{proof}

\end{document}
