\newcounter{english}
\documentclass[a4paper]{article}

% packages
\usepackage{inputenc, amsmath, amsthm, thmtools, amsfonts, amssymb, luacode, catchfile, tikzducks, hyperref}
\usepackage[a4paper, margin=50pt, includeheadfoot]{geometry} % set page margins
\usepackage[shortlabels]{enumitem}
\usepackage[skip=3pt, indent=0pt]{parskip}

% language
\usepackage[bidi=basic, layout=tabular, provide=*]{babel}
\babelprovide[main, import]{hebrew}
\babelprovide{rl}
\babelfont{rm}{Libertinus Serif}
\babelfont{sf}{Libertinus Sans}
\babelfont{tt}{Libertinus Mono}

% style
\AddToHook{cmd/section/before}{\clearpage}	% Add line break before section
\linespread{1.3}
\setcounter{secnumdepth}{0}		% Remove default number tags from sections, this won't do well with theorems
\AtBeginDocument{\setlength{\belowdisplayskip}{3pt}}
\AtBeginDocument{\setlength{\abovedisplayskip}{3pt}}

% operators
\DeclareMathOperator\cis{cis}
\DeclareMathOperator\Sp{Sp}
\DeclareMathOperator\tr{tr}
\DeclareMathOperator\im{Im}
\DeclareMathOperator\re{Re}
\DeclareMathOperator\diag{diag}
\DeclareMathOperator*\lowlim{\underline{lim}}
\DeclareMathOperator*\uplim{\overline{lim}}
\DeclareMathOperator\rng{rng}
\DeclareMathOperator\Sym{Sym}
\DeclareMathOperator\Arg{Arg}
\DeclareMathOperator\Log{Log}
\DeclareMathOperator\dom{dom}

% commands
%\renewcommand\qedsymbol{\textbf{מש''ל}}
%\renewcommand\qedsymbol{\fbox{\emoji{lizard}}}
\newcommand{\NN}[0]{\mathbb{N}}
\newcommand{\ZZ}[0]{\mathbb{Z}}
\newcommand{\QQ}[0]{\mathbb{Q}}
\newcommand{\RR}[0]{\mathbb{R}}
\newcommand{\CC}[0]{\mathbb{C}}
\newcommand{\FF}[0]{\mathbb{F}}
\newcommand{\PP}[0]{\mathbb{P}}
\newcommand{\TT}[0]{\mathbb{T}}
\newcommand{\acts}[0]{\circlearrowright}
\newcommand{\explain}[2] {
	\begin{flalign*}
		 && \text{#2} && \text{#1}
	\end{flalign*}
}
\newcommand{\maketitleprint}[0]{ \begin{center}
	\begin{tikzpicture}[scale=3]
		\duck[graduate=gray!20!black, tassel=red!70!black]
	\end{tikzpicture}	
\end{center}
}

% theorem commands
\newtheoremstyle{c_remark}
	{}	% Space above
	{}	% Space below
	{}% Body font
	{}	% Indent amount
	{\bfseries}	% Theorem head font
	{}	% Punctuation after theorem head
	{.5em}	% Space after theorem head
	{\thmname{#1}\thmnumber{ #2}\thmnote{ \normalfont{\text{(#3)}}}}	% head content
\newtheoremstyle{c_definition}
	{3pt}	% Space above
	{3pt}	% Space below
	{}% Body font
	{}	% Indent amount
	{\bfseries}	% Theorem head font
	{}	% Punctuation after theorem head
	{.5em}	% Space after theorem head
	{\thmname{#1}\thmnumber{ #2}\thmnote{ \normalfont{\text{(#3)}}}}	% head content
\newtheoremstyle{c_plain}
	{3pt}	% Space above
	{3pt}	% Space below
	{\itshape}% Body font
	{}	% Indent amount
	{\bfseries}	% Theorem head font
	{}	% Punctuation after theorem head
	{.5em}	% Space after theorem head
	{\thmname{#1}\thmnumber{ #2}\thmnote{ \text{(#3)}}}	% head content

\theoremstyle{c_plain}
\newtheorem{theorem}{משפט}[section]
\newtheorem{lemma}[theorem]{למה}
\newtheorem{proposition}[theorem]{טענה}
\newtheorem*{proposition*}{טענה}
%\newtheorem{corollary}[theorem]{אין חלופה עברית}

\theoremstyle{c_definition}
\newtheorem{definition}[theorem]{הגדרה}
\newtheorem*{definition*}{הגדרה}
\newtheorem{example}{דוגמה}[section]
\newtheorem{exercise}{תרגיל}[section]

\theoremstyle{c_remark}
\newtheorem*{remark}{הערה}
\newtheorem*{solution}{פתרון}
\newtheorem{conclusion}[theorem]{מסקנה}
\newtheorem{notation}[theorem]{סימון}

% Questions related commands
\newcounter{question}
\setcounter{question}{1}
\newcounter{sub_question}
\setcounter{sub_question}{1}

\newcommand{\question}[1][0]{
	\ifthenelse{#1 = 0}{}{\setcounter{question}{#1}}
	\subsection{שאלה \arabic{question}}
	\addtocounter{question}{1}
	\setcounter{sub_question}{1}
}

\newcommand{\subquestion}[1][0]{
	\ifthenelse{#1 = 0}{}{\setcounter{sub_question}{#1}}
	\subsubsection{סעיף \localecounter{letters.gershayim}{sub_question}}
	\addtocounter{sub_question}{1}
}

% import lua and start of document
\directlua{common = require ('../common')}

\GetEnv{AUTHOR}

% headers
\author{\AUTHOR}
\date\today

\title{Exercise 6 Answer Sheet --- Axiomatic Set Theory, 80650}

\begin{document}
\maketitle
\maketitleprint{}

\question{}
Let $\lambda$ be an infinite cardinal.
A $(\lambda^+, \lambda)$-Ulam matric is a collection of sets $\langle A_{\alpha, \rho} \mid \alpha < \lambda^+, \rho < \lambda \rangle$ such that,
\begin{enumerate}
	\item For every $\alpha < \beta < \lambda^+$ and $\rho < \lambda$, $A_{\alpha, \rho} \cap A_{\beta, \rho} = \emptyset$.
	\item For every $\alpha < \lambda^+$, $|\lambda^+ \setminus (\bigcup_\rho A_{\alpha, \rho})| \le \lambda$.
\end{enumerate}

\subquestion{}
We will show that for every infinite $\lambda$, a $(\lambda^+, \lambda)$-Ulam matrix exists.
\begin{proof}
	Let us define for each $0 < \xi < \lambda^+$ a surjection $f_\xi : \lambda \to \xi$ by mapping $f_\xi(x) = x$ if $x < \xi$ and into an arbitrary unique (as per $\xi$) value $< \xi$ otherwise.
	We define a matrix $A_{\alpha, \rho} = \{ \xi < \lambda^+ \mid f_\xi(\rho) = \alpha \}$ and want to show that the definition of $A$ fulfills Ulam matrix definition as depicted earlier.
	Let $\alpha < \beta < \lambda^+$ as well let us fix $\rho < \lambda$, therefore
	\[
		A_{\alpha, \rho} \cap A_{\beta, \rho}
		= \{ \xi < \lambda^+ \mid f_\xi(\rho) = \alpha \} \cap \{ \xi < \lambda^+ \mid f_\xi(\rho) = \beta \}
		= \{ \xi < \lambda^+ \mid f_\xi(\rho) = \alpha, f_\xi(\rho) = \beta \}
		= \emptyset
	\]
	Pay attention that the last step follows directly from the fact we fixed $\rho$ and that $f_\xi$ is a function. \\
	Let us fix $\alpha < \lambda^+$, then
	\[
		\left\lvert \lambda^+ \setminus \left(\bigcup_{\rho < \lambda} A_{\alpha, \rho}\right) \right\rvert
		= \left\lvert \lambda^+ \setminus \left(\bigcup_{\rho < \lambda} \{ \xi < \lambda^+ \mid f_\xi(\rho) = \alpha \} \right) \right\rvert
		= \left\lvert \lambda^+ \setminus \{ \xi < \lambda^+ \mid f_\xi(\lambda) \ni \alpha \} \right\rvert
	\]
	when $f_\xi(\lambda) = \im f_\xi = \xi$ as $f_\xi$ is surjective.
	Then
	\[
		\left\lvert \lambda^+ \setminus \{ \xi < \lambda^+ \mid \alpha < \xi \} \right\rvert
		|\lambda^+ \setminus \{ \xi \mid \alpha < \xi < \lambda^+ \}|
		\le \lambda
	\]
	As for every choice of $\alpha < \lambda^+$ the cardinality of $\alpha$ is at most $\lambda$.
\end{proof}

\subquestion{}
Let $\kappa$ be the least cardinal such that there is a $\sigma$-additive, non-trivial, non-atomic measure $\mu$ with $\dom \mu = \Pp(\kappa)$. \\
We will prove that $\kappa$ is not a successor cardinal.
\begin{proof}
	Let us assume in order to prove by contradiction that $\kappa$ is indeed a successor cardinal such that $\lambda^+ = \kappa$.
	By the last proposition, there is a $(\lambda^+, \lambda)$-Ulam matrix, namely $A$, for the specified $\lambda$.
	Fixing $\alpha < \lambda^+$, we will show that there is $\rho < \lambda$ such that $\mu(A_{\alpha, \rho}) > 0$. \\
	It is known that $|\kappa \setminus \bigcup_\rho A_{\alpha, \rho}| \le \lambda$,
	then the assumption that all these elements of $A$ fulfilling $A_{\alpha, \rho} \le \lambda$ would lead to contradiction as their union would be $< \kappa$.
	Then there is an element $A_{\alpha, \rho} > \lambda$ for every $\alpha < \kappa$. For each of these, $\mu(A_{\alpha, \rho}) > 0$, as $\mu$ is non-atomic (and the elements are \textit{big} in relation to $\kappa$).
	Let $\gamma = \{ \rho < \lambda \mid \alpha < \kappa, \mu(A_{\alpha, \rho}) > 0 \}$ in order to define $B_\alpha = \bigcup_{\rho \in \gamma} A_{\alpha, \rho}$.
	From its definition, $B_\alpha \cap B_\beta = \emptyset$ for every $\alpha < \beta < \kappa$, as deducted from Ulam matrix properties,
	and then by $\sigma$-additivity of $\mu$ we get contradiction to $\mu(\bigcup B_\alpha) \le 1$.
	From the contradiction it is followed that there is no such $\lambda$, meaning $\kappa$ is not a successor cardinal.
\end{proof}

\question{}
Let $\kappa$ be an uncountable regular cardinal such that there is non-principle filter $\mathcal{F} \subseteq \mathcal{P}(\kappa)$ with the following properties,
\begin{enumerate}
	\item For every $\langle x_\alpha \in \mathcal{F} \mid \alpha < \kappa \rangle$ also $\bigcap_{\alpha < \kappa} x_\alpha \in \mathcal{F}$.
	\item For every collection $\{ X_\alpha \mid \alpha < \omega_1 \} \subseteq \mathcal{P}(\kappa)$ such that $\forall \alpha, \kappa \setminus X_\alpha \notin \mathcal{F}$,
		there are $\alpha < \beta$ such that $X_\alpha \cap X_\beta \ne \emptyset$.
\end{enumerate}
Such an $\mathcal{F}$ is called non-trivial $\sigma$-saturated $\kappa$-complete filter on $\kappa$. \\
We will show that either there is a $\kappa$-complete ultrafilter on $\kappa$ or $\kappa \le 2^{\aleph_0}$ and $\kappa$ is a limit cardinal.
\begin{proof}
	Let $\Ff^+ = \{X \subseteq \kappa \mid \kappa \setminus X \notin \Ff\}$, This set represents the elements of $\Ff$ which are \textit{non zero} in a sense, a positive subset of the filter.

	Let us assume that for every $B \subseteq A$, $B \in \Ff^+$ or $A \setminus B \in \Ff^+$, this is in a sense the atomic case, in which there is a set that acts as an atom.
	We will show that in this case there is a $\kappa$-complete ultrafilter on $\kappa$.
	We will construct a sequence $\langle x_\alpha \in \Ff^+ \mid \alpha < \kappa \rangle \subseteq \Pp(\kappa)$ by taking such list of decreasing ordinals and using the defined property about inclusion in $\Ff^+$.
	By $\kappa$-completeness of $\Ff$ the intersection, $A$, of the stated sequence is in $\Ff$ and in particular from the assumption we made $A \in \Ff^+$.
	We define $\Uu = \{ x \in \Ff^+ \mid A \subseteq x \}$, this is an principal ultrafilter which is $\kappa$-complete as required\footnote{Jech T. Set Theory. 2003, 1, 77.}
	In details, we found a set which fulfills the definition of an atom, and by defining a measure using this atom, we get $\kappa$-complete ultrafilter, this property carries from $\Ff$ $\kappa$-completeness.

	Assuming the contrary of our initial assumption, it directly follows that for every $A \in \Ff^+$ there is $B \subseteq A$ such that $B, A \setminus B \in \Ff^+$
	(There is also the case that $B, A \setminus B \notin \Ff^+$, which lead to contradiction to the definition of $\Ff$ as a filter).
	This is case is in a sense non-atomic, as for each positive-measure set there is a split of disjoint positive-measure subsets.
	Let us define a left centered standard binary tree $\langle T, f \rangle$ such that $f : 2^{\omega_1} \to \Ff^+$, defined by $f(\langle \rangle) = \kappa$ and for each $t \in \dom f$,
	if $f(t) = A$, from the assumption there is $B \subseteq A$ satisfying our assumption, then we define $f(t \frown \langle 0 \rangle) = B, f(t \frown \langle 1 \rangle) = A \setminus B$.
	From our definition, for every $t \in 2^{\omega_1}$, $f(t \frown \langle 0 \rangle) \cap f(t \frown \langle 1 \rangle) = \emptyset$ and $f(t \frown \langle 0 \rangle) \cup f(t \frown \langle 1 \rangle) = f(t)$.
	Let $M = \{ f(t) \mid t \in 2^{\omega} \}$, then $|M| = 2^{\aleph_0}$ and by $\sigma$-saturation there are $m, m' \in M$ such that $m \cap m' \ne \emptyset$, which is a contradiction.
	Therefore we assume $M \cap \Ff^+ = \emptyset$ (the process to get a contradiction can be done for each branch to obtain this claim).
	But $\bigcup M = \kappa$, then we can conclude that $\kappa \le 2^{\aleph_0}$ as well. \\
	We want to show that $\kappa$ is a limit cardinal, to do that we will use the conclusion of question 1.
	To fulfill the propositions requirements it is needed to define a measure which is $\sigma$-additive, non-trivial and non-atomic.
	We can use the principal measure of $\Ff$ (Which requires extending it to ultrafilter, which in turn requires AC) to obtain such a measure.
\end{proof}

\end{document}
