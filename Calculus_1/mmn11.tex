\documentclass[a4paper]{article}

% packages
\usepackage{inputenc, fontspec, amsmath, amsfonts, polyglossia, catchfile}
\usepackage[a4paper, margin=50pt, includeheadfoot]{geometry} % set page margins

% style
\AddToHook{cmd/section/before}{\clearpage}	% Add line break before section
\setdefaultlanguage{hebrew}
\setotherlanguage{english}
\setmainfont{Libertinus Serif}
\linespread{1.5}
\setcounter{secnumdepth}{0}		% Remove default number tags from sections

% custom operators
\newcommand{\getenv}[2][]{%
  \CatchFileEdef{\temp}{"|kpsewhich --var-value #2"}{\endlinechar=-1}%
  \if\relax\detokenize{#1}\relax\temp\else\let#1\temp\fi}
\getenv[\AUTHOR]{AUTHOR}
\DeclareMathOperator\cis{cis}
\DeclareMathOperator\Sp{Sp}
\DeclareMathOperator\tr{tr}
\DeclareMathOperator\im{Im}
\DeclareMathOperator\diag{diag}
\def\NN{\mathbb{N}}
\def\RR{\mathbb{R}}
\def\CC{\mathbb{C}}

% theorems
\title{פתרון ממ''ן 11 – חשבון אינפיניטסימלי 1 (20474)}
\author{\AUTHOR}
\date\today

\begin{document}
\maketitle
\section{שאלה 1}
\subsection{סעיף א'}
נוכיח כי לכל $n$ טבעי מתקיים
\[
	\binom{2n}{n} \le 4^n
\]
תחילה נראה כי על־פי חוקי חזקות מתקיים
\[
	4^n = 2^{2n} = {(1 + 1)}^{2n}
\]
על־פי נוסחת הבינום של ניטון
\[
	{(1 + 1)}^{2n} = \sum_{i = 0}^{2n} \binom{2n}{i} 1^i 1^{2n - i}
\]
נוכל לרשום מחדש את השוויון כך:
\[
	4^n = \Bigg( \sum_{i = 0, i \ne n}^{2n} \binom{2n}{i} \Bigg)
	+ \binom{2n}{n}
\]
ביטוי הסכום המופיע תמיד שווה ערך לאפס או יותר, לכן בכל מקרה יתקיים
\[
	\binom{2n}{n} \le 4^n
\]

\subsection{סעיף ב'}
נוכיח באינדוקציה כי לכל $n$ טבעי מתקיים:
\[
	\binom{2n}{n} \ge \frac{4^n}{2n + 1}
\]
תחילה נוכיח את הזהות הבאה:
\[
	\binom{2(n + 1)}{n + 1}
	= \binom{2n}{n} \frac{2n + 1}{2}
	\tag{1}
\]
נשתמש בהגדרת הבינום:
\[
	\binom{2n + 2}{n + 1}
	= \frac{(2n + 2)!}{(n + 1)! (n + 1)!}
	= \frac{(2n + 2)(2n + 1) (2n)!}{2 (n + 1) n!}
	= \frac{(2n + 1) (2n)!}{n!}
	= \binom{2n}{n} \frac{2n + 1}{2}
\]
נשתמש בזהות $(1)$ על אי־השוויון המקורי:
\begin{align*}
	& \binom{2n}{n} \ge \frac{4^n}{2n + 1} \\
	& \binom{2n}{n} (2n + 1) \ge 4^n \\
	& 2 \binom{2n + 2}{n + 1} \ge 4^n \tag{2}
\end{align*}
לכן כדי להוכיח את נכונות אי־השוויון המקורי נוכל להוכיח את נכונות $(2)$. \\*
\textbf{בסיס האינדוקציה:}
נציב בביטוי $(2)$ את $n = 1$:
\[
	2 \binom{2 + 2}{1 + 1} = 12 \ge 4 = 4^1
\]
בעזרת חישוב דומה נראה כי אי־השוויון מתקיים גם עבור $n = 2$. \\*
אנו רואים כי אי־השוויון אכן מתקיים ובכך הנחנו בסיס לאינדוקציה. \\*
\textbf{מעבר האינדוקציה:}
נניח כי אי־השוויון מתקיים ונוכיח כי הוא גורר
את נכונות אי השיוויון בהצבה $n + 1$. \\*
נשים לב תחילה כי אי־השוויון הבא מתקיים לכל $n > 2$:
\[
	\frac{2n + 3}{2} \ge 4 \tag{3}
\]
הראינו כי אי־השוויון מתקיים לערכים $n = 1, 2$ ולכן נניח ש־$n > 2$.
נוכל לפי כללי אי־שוויונות להכפיל את אגפי אי־שוייון $(2)$ ב־$(3)$
מבלי לפגוע בנכונות הביטוי:
\[
	2 \binom{2n + 2}{n + 1} \frac{2n + 3}{2}  \ge 4^n \cdot 4
\]
נשתמש בזהות $(1)$:
\[
	2 \binom{2n + 4}{n + 2} \ge 4^{n + 1}
\]
אנו רואים כי מהנחת האינדוקציה עבור $n$ אי־השוויון מתקיים גם עבור $n + 1$,
ולכן אי־השיוויון נכון לכל $n$ טבעי.

\section{שאלה 2}
נוכיח שעבור כל שני מספרים ממשיים $a, b$ כאשר $b \ne 0$,
אם $|a - b| \le b^2$ אז $|\frac{a}{b}| \le |b| + 1$. \\*
נשתמש בטענה 1.48 סעיף 6 ונראה כי מתקיים:
\begin{align*}
	& |a - b| \le |b| \cdot |b| \\
	& \frac{|a - b|}{|b|} \le |b| \\
	& \left|\frac{a - b}{b}\right| \le |b|
	& \text{טענה 1.48} \\
	& \left|\frac{a}{b} - 1\right| \le |b| \\
	& \left|\frac{a}{b} - 1\right| + 1 \le |b| + 1 \\
	& \left|\frac{a}{b} - 1 + 1\right| \le |b| + 1
	& \text{אי־שוויון המשולש} \\
	& \left|\frac{a}{b}\right| \le |b| + 1
\end{align*}
אי־השוויון אכן מתקיים בתנאי זה.

\subsection{סעיף ב'}
נוכיח
\[
	{\left( \frac{a + |a|}{2} \right)}^2
	+ {\left( \frac{a - |a|}{2} \right)}^2
	= a^2
\]
לכל מספר ממשי $n$ מתקיים $a^2 = |a| \cdot |a|$
לפי טענה 1.48 סעיף 6. נשתמש בטענה זו ונפתח סוגריים:
\[
	{\left( \frac{a + |a|}{2} \right)}^2
	+ {\left( \frac{a - |a|}{2} \right)}^2
	=
	\frac{a^2 + 2 a |a| + a^2}{4}
	+ \frac{a^2 - 2 a |a| + a^2}{4}
	=
	\frac{4 a^2 + 2 a |a| - 2 a |a|}{4}
	= a^2
\]

\section{שאלה 3}
\subsection{סעיף א'}
נפתור את אי־השוויון
\[
	\lfloor |x + 1| - |x| \rfloor \ge x^2
\]
כאשר $x \ge 0$ מתקיים $|x + 1| - |x| = x + 1 - x = 1$.
ולכן אי־השוויון שקול לביטוי $1 \ge x^2$.
לכן $0 \le x \le 1$ היא חלק מקבוצת הפתרונות של אי־השוויון. \\*
כאשר $-1 \le x < 0$ מתקיים $|x + 1| - |x| = 1 + x + x = 1 + 2x$
בשל החלק השלם כל תנאי ש־$x \le -\frac{1}{2}$ החלק השלם יהיה $-1$,
ולאי־השוויון לא יהיה פתרון בממשיים, בכל מצב אחר החלק השלם יהיה אפס
ואי־השוויון יתקיים רק כאשר $x = 0$ בסתירה להנחה ש־$x < 0$.
\\*
כאשר $x < -1$ מתקיים $|x + 1| - |x| = -x - 1 + x = -1$
זהו כמובן מספר שלם ולכן אי־השוויון שקול ל־$-1 \ge x^2$,
ולאי־שוויון זה אין פתרון בממשיים. \\*
מצאנו כי אי־השוויון מתקיים רק כאשר $0 \le x \le 1$.

\subsection{סעיף ב'}
(i) נפתור את המשוואה $\lfloor x \rfloor^2 = 16$.
על־ידי העברת אגף ומציאת שורשים נקבל
$(\lfloor x \rfloor - 4)(\lfloor x \rfloor + 4) = 0$.
לכן $\lfloor x \rfloor = \pm 4$.
על־פי טענה 1.64 סעיף 2 תנאים אלה מתקיימים רק כאשר $x - 1 < 4 \le x$
או $x - 1 < -4 \le x$. \\
(ii) נפתור את המשוואה $\lfloor x^2 \rfloor = 3$.
על־פי טענה 1.64 2 המשוואה מתקיימת כאשר $x^2 - 1 < 3 \le x^2$.
אי־השוויון $3 \le x^2$ מתקיים כאשר $\sqrt{3} \le |x|$.
אי־השוויון $x^2 - 1 < 3$ שקול ל־$(x + 2)(x - 2) < 0$.
תנאי זה מתקיים כאשר $-2 < x < 2$. \\*
החיתוך בין שני התנאים הוא שהמשוואה מתקיימת כאשר
$-2 < x \le -\sqrt{3}$ או כאשר $\sqrt{3} \le x < 2$.

\section{שאלה 4}
\subsection{סעיף א'}
תהיה $A$ קבוצת ממשיים הצפופה בקטע $(1, \infty)$. נגדיר
\[
	B = \left\{ \frac{a}{n} \middle| a \in A, n \in \NN \right\}
\]
נוכיח כי הקבוצה $B$ צפופה בקטע $(0, 1)$. \\*
נגדיר $x, y$ מספרים ממשיים כך ש־$x, y \in B$ ו־$0 < x < y < 1$.
קיים מספר טבעי $n$ כך שמתקיים $1 < nx < ny < n$,
נראה כי קיים מספר טבעי הגדול מהמספר ההופכי ל־$x$,
והוא אכן מקיים את תנאי אי־השוויון.
שני המספרים $nx, ny$ איברים ב־$A$
ושייכים ומתקיים $nx, ny \in (1, \infty)$,
לכן בשל הצפיפות קיים מספר $c \in A$ כך ש־$nx < c < ny$.
נחלק את שלושת האגפים ב־$n$ ונקבל $x < c/n < y$,
לכן הקטע $(0, 1)$ צפוף ב־$B$.

\subsection{סעיף ב'}
ננסח הגדרה לחוסר רציפות של קבוצה $A$ בקטע $I$.
תחילה נצרין את הטענה:
\[
	\lnot \forall x, y \in I
	(x < y \rightarrow \exists a \in A (x < a \land a < y)) 
\]
נפשט את הפסוק:
\[
	\exists x, y \in I
	(x < y \land \forall a \in A (x \ge a \lor a \ge y))
\]
ננסח את הפסוק שהתקבל: \\*
קבוצה $A$ תיקרא לא צפופה ב־$I$ אם קיימים שני מספרים $x, y \in I$
כך ש־$x < y$ וגם לכל איבר $a$ ב־$A$ מתקיים ש־$a \le x$ או ש־$a \ge y$.

\subsection{סעיף ג'}
תהי $A$ קבוצה של מספרים ממשיים המוכלת בקטע $(1, \infty)$ וצפופה בו.
נוכיח כי הקבוצה $C$ המוגדרת להלן אינה צפופה בקטע $[0, 1]$.
\[
	C = \left\{ \frac{a}{n^2(a + 1)} \middle| a \in A, n \in \NN \right\}
\]
על־פי ההגדרה שהנחנו בסעיף ב',
כדי שהקבוצה $C$ תהיה לא צפופה ב־$[0, 1]$
אם קיימים שני מספרים $x, y \in C$ כך שלא קיים מספר $c \in [0, 1]$
כך שמתקיים גם $x < c < y$.
ננסה למצוא מספרים $x, y$ כאלה. \\*
עבור כל $a \in A$ מתקיים:
\[
	1 < a < a + 1 \tag{1}
\]
נחלק את הביטוי ב־$a + 1$:
\[
	\frac{1}{a + 1} < \frac{a}{a + 1} < 1
\]
ננצל שוב את טענה $(1)$ ונראה כי:
\[
	\frac{1}{1 + 1} = \frac{1}{2} < \frac{a}{a + 1} < 1 \tag{2}
\]
נגדיר $n$ מספר טבעי כלשהו ונחלק את אי־השוויון בריבועו:
\[
	\frac{1}{2n^2} < \frac{a}{n^2 (a + 1)} < \frac{1}{n^2}
\]
נשים לב כי האגף האמצעי באי־השוויון הוא איבר ב־$C$,
לכן נגדיר אותו להיות $c$:
\[
	\frac{1}{2n^2} < c < \frac{1}{n^2}
\]
נשים לב כי עבור כל $c$ כאשר $1 < n$ מתקיים:
\[
	0 < c < \frac{1}{4} \tag{3}
\]
מאי־השוויונות $(2)$ ו־$(3)$ אנו רואים כי אין $c$ המקיים:
\[
	\frac{1}{4} < c < \frac{1}{2}
\]
לכן נגדיר $x = \frac{1}{4}$ ו־$y = \frac{1}{2}$,
ראינו כי אין איבר ב־$C$ אשר גדול מ־$x$ וקטן מ־$y$
ולכן היא לא צפופה ב־$[0, 1]$.
\end{document}
