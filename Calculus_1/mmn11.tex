\documentclass[a4paper]{article}

% packages
\usepackage{inputenc, fontspec, amsmath, amsfonts, polyglossia, catchfile}
\usepackage[a4paper, margin=50pt, includeheadfoot]{geometry} % set page margins

% style
\AddToHook{cmd/section/before}{\clearpage}	% Add line break before section
\setdefaultlanguage{hebrew}
\setotherlanguage{english}
\setmainfont{Libertinus Serif}
\linespread{1.5}
\setcounter{secnumdepth}{0}		% Remove default number tags from sections

% custom operators
\newcommand{\getenv}[2][]{%
  \CatchFileEdef{\temp}{"|kpsewhich --var-value #2"}{\endlinechar=-1}%
  \if\relax\detokenize{#1}\relax\temp\else\let#1\temp\fi}
\getenv[\AUTHOR]{AUTHOR}
\DeclareMathOperator\cis{cis}
\DeclareMathOperator\Sp{Sp}
\DeclareMathOperator\tr{tr}
\DeclareMathOperator\im{Im}
\DeclareMathOperator\diag{diag}
\def\NN{\mathbb{N}}
\def\ZZ{\mathbb{Z}}
\def\QQ{\mathbb{Q}}
\def\RR{\mathbb{R}}
\def\CC{\mathbb{C}}

\title{פתרון ממ''ן 11 – חשבון אינפיניטסימלי 1 (20474)}
\author{\AUTHOR}
\date\today

\begin{document}
\maketitle
\section{שאלה 1}
\subsection{סעיף א'}
נוכיח כי $a = k + m \sqrt{2}$ כאשר $k, m \in \NN$ הוא מספר אי־רציונלי. \\*
משפט 1.21 גורס כי $\sqrt{2}$ איננו רציונלי, לכן לא קיימים מספרים $p, q \in \ZZ$ כך ש־$\sqrt{2} = \frac{p}{q}$.
אליו היה המספר $m \sqrt{2}$ מספר רציונלי אז היה מתקיים $m \sqrt{2} = \frac{mp}{q}$ בניגוד לאי־רציונליות $\sqrt{2}$,
לכן $m\sqrt{2}$ מספר אי־רציונלי.
נגדיר $p', q' \in \ZZ$, מתקיים:
\[
	\frac{p}{q} + \frac{p'}{q'} = \frac{p q' + p'q}{q q'}
\]
לכן מספר כל חיבור מספרים רציונליים הוא מספר רציונלי, אילו הייתה התוצאה $k + m\sqrt{2}$ רציונלית אז היה נגזר כי גם $m\sqrt{2}$ רציונלית,
בסתירה להיותו אי־רציונלי. לכן המספר $a$ אי־רציונלי.

\subsection{סעיף ב'}
נוכיח כי לכל $n \in \NN$ המספר ${(1 + \sqrt{2})}^n$ לא רציונלי באינדוקציה: \\*
\textbf{בסיס האינדוקציה}:
המספר ${(1 + \sqrt{2})}^1$ הוא כמובן מספר אי־רציונלי בהתאם לסעיף א'. \\*
\textbf{מהלך האינדוקציה}:
נניח כי ${(1 + \sqrt{2})}^n = k + m\sqrt{2}, k, m \in \NN$ אי־רציונלי. מתקיים
\[
	{(1 + \sqrt{2})}^{n + 1}
	= {(1 + \sqrt{2})}^n {(1 + \sqrt{2})}
	= (k + m \sqrt{2}) (1 + \sqrt{2})
	= k + m \sqrt{2} + k \sqrt{2} + 2m
	= (k + 2m) + (m + k) \sqrt{2}
\]
ידוע כי $k + 2m, m + k \in \NN$ ולכן ${(1 + \sqrt{2})}^{n + 1}$ הוא מספר אי־רציונלי על־פי סעיף א' והושלם מהלך האינדוקציה.

\section{שאלה 2}
\subsection{סעיף א'}
יהיו $a, b \in \RR$ נוכיח שמתקיים:
\[
	\left| \sqrt{\left| a \right| + 1} - \sqrt{\left| b \right| + 1} \right| \le \frac{|a - b|}{2}
\]
מתקיים לכל $x, y \in \RR$:
\[
	\sqrt{x} - \sqrt{y}
	= (\sqrt{x} - \sqrt{y}) \frac{(\sqrt{x} + \sqrt{y})}{(\sqrt{x} + \sqrt{y})}
	= \frac{(\sqrt{x} - \sqrt{y}) (\sqrt{x} + \sqrt{y})}{(\sqrt{x} + \sqrt{y})}
	= \frac{x - y}{\sqrt{x} + \sqrt{y}}
\]
נגדיר
\[
	x = |a| + 1, y = |b| + 1
\]
אז
\[
	\left| \sqrt{\left| a \right| + 1} - \sqrt{\left| b \right| + 1} \right|
	= \frac{|a| + 1 - |b| - 1}{\sqrt{\left| a \right| + 1} + \sqrt{\left| b \right| + 1}}
	= \frac{|a| - |b|}{\sqrt{\left| a \right| + 1} + \sqrt{\left| b \right| + 1}}
\]
בשל הגדרת הערך המוחלט מתקיים
\[
	0 \le |a| \rightarrow 1 \le |a| + 1 \rightarrow \sqrt{1} = 1 \le \sqrt{|a| + 1}
\]
ולכן תמיד מתקיים
\[
	2 \le \sqrt{\left| a \right| + 1} + \sqrt{\left| b \right| + 1}
\]
בשל הופעת הביטוי במכנה מתקיים:
\[
	\left| \sqrt{\left| a \right| + 1} - \sqrt{\left| b \right| + 1} \right| = 
	\frac{|a| - |b|}{\sqrt{\left| a \right| + 1} + \sqrt{\left| b \right| + 1}}
	\ge \frac{|a| - |b|}{2}
\]
לפי טענה 1.51:
\[
	\left| \sqrt{\left| a \right| + 1} - \sqrt{\left| b \right| + 1} \right|
	\ge \frac{|a - b|}{2}
\]

\subsection{סעיף ב'}
נוכיח
\[
	{\left( \frac{a + |a|}{2} \right)}^2
	+ {\left( \frac{a - |a|}{2} \right)}^2
	= a^2
\]
לכל מספר ממשי $n$ מתקיים $a^2 = |a| \cdot |a|$
לפי טענה 1.48 סעיף 6. נשתמש בטענה זו ונפתח סוגריים:
\[
	{\left( \frac{a + |a|}{2} \right)}^2
	+ {\left( \frac{a - |a|}{2} \right)}^2
	=
	\frac{a^2 + 2 a |a| + a^2}{4}
	+ \frac{a^2 - 2 a |a| + a^2}{4}
	=
	\frac{4 a^2 + 2 a |a| - 2 a |a|}{4}
	= a^2
\]

\section{שאלה 3}
\subsection{סעיף א'}
נוכיח כי לכל $x, y \in \RR$ אם $x \le y$ אז $\lfloor x \rfloor \le \lfloor y \rfloor$. \\*
נניח כי $x \le y$. נגדיר $x' = \langle x \rangle, y' = \langle y \rangle$ החלקים השבריים של $x, y$ בהאתמה.
מתקיים $x = \lfloor x \rfloor + x', y = \lfloor y \rfloor + y'$ לכן
\[
	\lfloor x \rfloor + x' \le \lfloor y \rfloor + y'
\]
אילו $x' \le y'$ הטענה מתקיימת, לכן נניח $x' > y'$.
מתקיים $\lfloor x \rfloor < \lfloor y \rfloor$, אחרת יתכן כי החלק השלם של המספרים יהיה זהה, אך החלק השברי $x'$ יהיה גדול מ־$y'$ ויגרור $y < x$ בסתירה להנחה.
אז מתקיים בכל מקרה $\lfloor x \rfloor \le \lfloor y \rfloor$.

\subsection{סעיף ב'}
\textbf{(i)}
נפתור את המשוואה
\begin{align*}
	& \lfloor x - \frac{1}{2} \rfloor^2 = 25 \\
	& (\lfloor x - \frac{1}{2} \rfloor^2 + 5)(\lfloor x - \frac{1}{2} \rfloor^2 - 5) = 0 \\
	& \lfloor x - \frac{1}{2} \rfloor = \pm 5
\end{align*}

על־פי טענה 1.64(1)
\begin{align*}
	& \pm 5 \le x - \frac{1}{2} < \pm 5 + 1 \\
	& \pm 5 + \frac{1}{2} \le x < \pm 5 + 1\frac{1}{2}  \\
	& -4 \frac{1}{2} \le x < -3\frac{1}{2}, 5\frac{1}{2} \le x < 6\frac{1}{2}
\end{align*}
\textbf{(ii)}
\begin{align*}
	& \lfloor x^2 \rfloor = 9 \\
	& 9 \le x^2 < 10 \\
	& 9 \le x^2 \land x^2 < 10 \\
	& 0 \le x^2 - 9 \land x^2 - 10 < 0 \\
	& 0 \le (x - 3)(x + 3) \land (x - \sqrt{10})(x + \sqrt{10}) < 0 \\
	& (x \ge 3 \lor x \le -3) \land (x < \sqrt{10} \lor x > -\sqrt{10}) \\
	& 3 \le x < \sqrt{10}, -\sqrt{10} < x \le -3
\end{align*}

\section{שאלה 4}
\subsection{סעיף א'}
תהי הקבוצה
\[
	A = \left\{ q \sqrt{3} \mid q \in (0, \infty) \cap \QQ \right\}
\]
נוכיח כי $A$ צפופה בקטע $[0, 1]$. \\*
ידוע כי $\QQ$ צפופה ב־$[0, 1]$.
נגדיר $f : \QQ \to A$ המוגדרת $f(x) = x \sqrt{3}$. ניתן לראות כי $f$ על וחד־חד ערכית ולכן הפיכה.
לכל שני מספרים $a, b \in A$ אשר מקיימים $a < b$ מתקיים גם $f^{-1}(x) < f^{-1}(y)$, והרי אלו הם מספרים רציונליים אשר ידוע כי צפופים,
נבחר מספר $c$ אשר מקיים $f^{-1}(x) < c < f^{-1}(y)$, ונשים לב כי גם מתקיים $x < f(c) < y$.
מצאנו כי $A$ צפופה לכל שני מספרים שונים המוכלים בה, לכן היא גם צפופה בתחום $[0, 1]$.

\subsection{סעיף ב'}
ננסח הגדרה לחוסר רציפות של קבוצה $A$ בקטע $I$.
תחילה נצרין את הטענה:
\[
	\lnot \forall x, y \in I
	(x < y \rightarrow \exists a \in A (x < a \land a < y)) 
\]
נפשט את הפסוק:
\[
	\exists x, y \in I
	(x < y \land \forall a \in A (x \ge a \lor a \ge y))
\]
ננסח את הפסוק שהתקבל: \\*
קבוצה $A$ תיקרא לא צפופה ב־$I$ אם קיימים שני מספרים $x, y \in I$
כך ש־$x < y$ וגם לכל איבר $a$ ב־$A$ מתקיים ש־$a \le x$ או ש־$a \ge y$.

\subsection{סעיף ג'}
נוכיח כי קבוצת השברים העשרוניים הסופיים שלא מופיעה בהם הספרה 3 אינה צפופה בקטע $[-1, 1]$. \\*
נגדיר $a = 0.3, b = 0.4$, ברור כי $a, b \in [-1, 1], a < b$, אך לכל מספר $x \in A$ מתקיים או $x \le a$ או $x \ge b$,
שכן כל מספר שלא עומד בהגדרה זו חייב להיות מהתצורה $0.3\ldots$ ולכן איננו שייך ל־$A$.
אז לפי הגדרת סעיף ב' הקבוצה $A$ איננה צפופה ב־$[-1, 1]$.
\end{document}
