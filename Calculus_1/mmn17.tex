\documentclass[a4paper]{article}

% packages
\usepackage{inputenc, fontspec, amsmath, amsthm, amsfonts, polyglossia, catchfile}
\usepackage[a4paper, margin=50pt, includeheadfoot]{geometry} % set page margins

% style
\AddToHook{cmd/section/before}{\clearpage}	% Add line break before section
\linespread{1.5}
\setcounter{secnumdepth}{0}		% Remove default number tags from sections
\setmainfont{Libertinus Serif}
\setsansfont{Libertinus Sans}
\setmonofont{Libertinus Mono}
\setdefaultlanguage{hebrew}
\setotherlanguage{english}

% operators
\DeclareMathOperator\cis{cis}
\DeclareMathOperator\Sp{Sp}
\DeclareMathOperator\tr{tr}
\DeclareMathOperator\im{Im}
\DeclareMathOperator\diag{diag}
\DeclareMathOperator*\lowlim{\underline{lim}}
\DeclareMathOperator*\uplim{\overline{lim}}

% commands
\renewcommand\qedsymbol{\textbf{משל}}
\newcommand{\NN}[0]{\mathbb{N}}
\newcommand{\ZZ}[0]{\mathbb{Z}}
\newcommand{\QQ}[0]{\mathbb{Q}}
\newcommand{\RR}[0]{\mathbb{R}}
\newcommand{\CC}[0]{\mathbb{C}}
\newcommand{\getenv}[2][] {
  \CatchFileEdef{\temp}{"|kpsewhich --var-value #2"}{\endlinechar=-1}
  \if\relax\detokenize{#1}\relax\temp\else\let#1\temp\fi
}
\newcommand{\explain}[2] {
	\begin{flalign*}
		 && \text{#2} && \text{#1}
	\end{flalign*}
}

% headers
\getenv[\AUTHOR]{AUTHOR}
\author{\AUTHOR}
\date\today

\title{פתרון ממ''ן 17 – חשבון אינפיניטסימלי 1 (20474)}

\begin{document}
\maketitle
\section{שאלה 1}
תהי $f$ פונקציה רציפה ב־$\RR$ המקבלת מקסימום מקומי בנקודה $x_0$. \\*
נוכיח כי אם אין ל־$f$ נקודות קיצון נוספות אז $f$ מקבלת מקסימום בנקודה $x_0$.
\begin{proof}
	%משהו עם המשפט השני של ויירשטראס.
	ידוע כי $x_0$ נקודת מקסימום מקומי של $x_0$, לכן קיימת סביבה של $x_0$ בה ערך הפונקציה הוא מקסימלי. \\*
	נסמן תחום זה כערכים $|x - x_0| \le \delta$. \\*
	זהו כמובן תחום סגור ולכן על־פי המשפט השני של ויירשטראס לפונקציה יש נקודת מקסימום ומינימום.
	כמובן שנקודת המקסימום בקטע זה מתלכדת עם המקסימום המקומי.
	אנו יודעים כי לא קיימות נקודות קיצון נוספות, דהינו לכל $\delta$ לא קיים בתחום נקודת קיצון נוספת ובהתאם $x_0$ נשארת נקודת הקיצון ולכן המקסימום. \\*
	נניח בשלילה עתה כי קיים $x_1$ כך ש־$f(x_1) > f(x_0)$. \\*
	נבחר $\delta$ כך שמתקיים $|x_0 - x_1| \le \delta$, אז בתחום זה $f(x_0) > f(x)$ לכל $x$, ובפרט גם $f(x_0) > f(x_1)$ בבתירה לטענה. \\*
	אז $x_0$ נקודת מקסימום של $f(x)$.
\end{proof}

\section{שאלה 2}
תהי $f$ פונקציה רציפה בקטע $[a, b]$ וגזירה בקטע $(a, b)$. נניח כי קיימת נקודה $c \in (a, b)$ כך שמתקיים
\[
	(f(c) - f(a))(f(b) - f(c)) < 0
\]
נוכיח כי קיימת נקודה $t \in (a, b)$ עבורה $f'(t) = 0$.
\begin{proof}
	מאי־השוויון נובע ישירות כי $f(c) < f(a)$ או $f(b) < f(c)$ בלבד. \\*
	ללא פגיעה בכלליות ההוכחה נוכל להניח כי אחד השוויונות מתקיימים, ועל־ידי היפוך תפקידי $a, b$ להגיע למצב שהושמט. \\*
	נניח כי $f(c) < f(a)$ ולכן לא יתכן כי $f(b) < f(c)$, ואף לא יתכן כי $f(b) = f(c)$ כנביעה מאי־השוויון הנתון, לכן $f(c) < f(b)$. \\*
	מהמשפט השני של ויירשטראס נובע כי בתחום $[a, b]$ קיים מינימום ל־$f$, אך בשל $f(c)$ אנו יכולים להסיק כי הוא לא ב־$a$ או ב־$b$.
	נגדיר $t$ נקודת המינימום בקטע $[a, b]$ עבור $f$, ולכן ממשפט פרמה נובע כי $f'(t) = 0$.
\end{proof}

\section{שאלה 3}
תהי $f$ פונקציה גזירה בקטע $[0, 1]$ המקיימת $0 \le f'(x) \le 1$ לכל $x \in [0, 1]$. \\*
נוכיח כי קיימת נקודה $x_0 \in [0, 1]$ כך שמתקיים
\[
	f'(x_0) = \frac{3x_0}{\sqrt{3x_0^2 + 6}}
\]
\begin{proof}
	נגדיר פונקציה $g(x) = \sqrt{3x^2 + 6}$.
	\[
		g'(x) = \frac{6x}{2\sqrt{3x^2 + 6}}
	\]
	נשים לב כי $g'(0) = 0$ וכי $g'(1) = 1$. \\*
	ל־$g'(x)$ יש פונקציה קדומה, ולכן ממשפט דארבו נובע כי אני ארנבת מדבר.
\end{proof}

\end{document}

