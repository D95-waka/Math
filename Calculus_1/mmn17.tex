\documentclass[a4paper]{article}

% packages
\usepackage{inputenc, fontspec, amsmath, amsthm, amsfonts, polyglossia, catchfile}
\usepackage[a4paper, margin=50pt, includeheadfoot]{geometry} % set page margins

% style
\AddToHook{cmd/section/before}{\clearpage}	% Add line break before section
\linespread{1.5}
\setcounter{secnumdepth}{0}		% Remove default number tags from sections
\setmainfont{Libertinus Serif}
\setsansfont{Libertinus Sans}
\setmonofont{Libertinus Mono}
\setdefaultlanguage{hebrew}
\setotherlanguage{english}

% operators
\DeclareMathOperator\cis{cis}
\DeclareMathOperator\Sp{Sp}
\DeclareMathOperator\tr{tr}
\DeclareMathOperator\im{Im}
\DeclareMathOperator\diag{diag}
\DeclareMathOperator*\lowlim{\underline{lim}}
\DeclareMathOperator*\uplim{\overline{lim}}

% commands
\renewcommand\qedsymbol{\textbf{משל}}
\newcommand{\NN}[0]{\mathbb{N}}
\newcommand{\ZZ}[0]{\mathbb{Z}}
\newcommand{\QQ}[0]{\mathbb{Q}}
\newcommand{\RR}[0]{\mathbb{R}}
\newcommand{\CC}[0]{\mathbb{C}}
\newcommand{\getenv}[2][] {
  \CatchFileEdef{\temp}{"|kpsewhich --var-value #2"}{\endlinechar=-1}
  \if\relax\detokenize{#1}\relax\temp\else\let#1\temp\fi
}
\newcommand{\explain}[2] {
	\begin{flalign*}
		 && \text{#2} && \text{#1}
	\end{flalign*}
}

% headers
\getenv[\AUTHOR]{AUTHOR}
\author{\AUTHOR}
\date\today

\title{פתרון ממ''ן 17 – חשבון אינפיניטסימלי 1 (20474)}

\begin{document}
\maketitle
\section{שאלה 1}
תהי $f$ פונקציה רציפה ב־$\RR$ המקבלת מקסימום מקומי בנקודה $x_0$. \\*
נוכיח כי אם אין ל־$f$ נקודות קיצון נוספות אז $f$ מקבלת מקסימום בנקודה $x_0$.
\begin{proof}
	%משהו עם המשפט השני של ויירשטראס.
	ידוע כי $x_0$ נקודת מקסימום מקומי של $x_0$, לכן קיימת סביבה של $x_0$ בה ערך הפונקציה הוא מקסימלי. \\*
	נסמן תחום זה כערכים המקיימים $|x - x_0| \le \delta$. \\*
	זהו כמובן תחום סגור ולכן על־פי המשפט השני של ויירשטראס לפונקציה יש נקודת מקסימום ומינימום.
	כמובן שנקודת המקסימום בקטע זה מתלכדת עם המקסימום המקומי $x_0$.
	אנו יודעים כי לא קיימות נקודות קיצון נוספות, דהינו לכל $\delta$ לא קיים בתחום נקודת קיצון נוספת ובהתאם $x_0$ נשארת נקודת הקיצון ולכן המקסימום בקטע זה. \\*
	נניח בשלילה עתה כי קיים $x_1$ כך ש־$f(x_1) > f(x_0)$. \\*
	נבחר $\delta$ כך שמתקיים $|x_0 - x_1| \le \delta$, אז בתחום זה $f(x_0) > f(x)$ לכל $x$, ובפרט גם $f(x_0) > f(x_1)$ בסתירה לטענה. \\*
	בשל כך $x_0$ נקודת מקסימום של $f(x)$.
\end{proof}

\section{שאלה 2}
תהי $f$ פונקציה רציפה בקטע $[a, b]$ וגזירה בקטע $(a, b)$. נניח כי קיימת נקודה $c \in (a, b)$ אשר עבורה
\[
	(f(c) - f(a))(f(b) - f(c)) < 0
\]
נוכיח כי קיימת נקודה $t \in (a, b)$ עבורה $f'(t) = 0$.
\begin{proof}
	מאי־השוויון נובע ישירות כי $f(c) < f(a)$ או $f(b) < f(c)$ בלבד. \\*
	ללא פגיעה בכלליות ההוכחה נוכל להניח כי אחד השוויונות מתקיימים, ועל־ידי היפוך תפקידי $a, b$ להגיע למצב שהושמט. \\*
	נניח כי $f(c) < f(a)$ ולכן לא יתכן כי $f(b) < f(c)$, ואף לא יתכן כי $f(b) = f(c)$ כנביעה מאי־השוויון הנתון, לכן $f(c) < f(b)$. \\*
	מהמשפט השני של ויירשטראס נובע כי בתחום $[a, b]$ קיים מינימום ל־$f$, אך בשל ערך $f(c)$ אנו יכולים להסיק כי הוא לא ב־$a$ או ב־$b$.
	נגדיר $t$ נקודת המינימום בקטע $[a, b]$ עבור $f$, ולכן ממשפט פרמה נובע כי $f'(t) = 0$.
\end{proof}

\section{שאלה 3}
תהי $f$ פונקציה גזירה בקטע $[0, 1]$ המקיימת $0 \le f'(x) \le 1$ לכל $x \in [0, 1]$. \\*
נוכיח כי קיימת נקודה $x_0 \in [0, 1]$ כך שמתקיים
\[
	f'(x_0) = \frac{3x_0}{\sqrt{3x_0^2 + 6}}
\]
\begin{proof}
	נגדיר פונקציה $g(x) = \sqrt{3x^2 + 6}$, ונחשב את נגזרתה על־פי משפטי גזירה:
	\[
		g'(x) = \frac{6x}{2\sqrt{3x^2 + 6}}
	\]
	נשים לב כי $g'(0) = 0$, $g'(1) = 1$ וכי $f'(x) = g'(x)$. \\*
	ממשפט דארבו נובע ישירות כי קיים $x \in [0, 1]$ כך ש־$f'(x) = g'(x)$.
\end{proof}

\section{שאלה 4}
נוכיח כי הפונקציה
\[
	f(x) = \sqrt{x} \sin \sqrt{x}
\]
רציפה במידה שווה בקטע $[0, \infty)$ % chktex 9
\begin{proof}
	נחשב את נגזרתה של $f(x)$:
	\[
		f'(x) = \frac{\sin \sqrt{x}}{2\sqrt{x}} + \frac{1}{2\sqrt{x}} \sqrt{x} \cos \sqrt{x}
		= \frac{\sin \sqrt{x}}{2\sqrt{x}} + \frac{1}{2} \cos \sqrt{x}
	\]
	נשים לב כי
	\begin{align*}
		|f'(x)|
		& = \left\lvert \frac{\sin \sqrt{x}}{2\sqrt{x}} + \frac{1}{2} \cos \sqrt{x} \right\rvert \\
		& \le \left\lvert \frac{\sin \sqrt{x}}{2\sqrt{x}} \right\rvert + \left\lvert \frac{1}{2} \cos \sqrt{x} \right\rvert && \text{אי־שיוויון המשולש} \\
		& = \frac{1}{2\sqrt{x}} \lvert \sin \sqrt{x} \rvert  + \frac{1}{2} \lvert \cos \sqrt{x} \rvert && \text{חיוביות בתחום} \\
		& \le \frac{1}{2\sqrt{x}} + \frac{1}{2} && \text{חסימות $\sin, \cos$} \\
	\end{align*}
	נגדיר גם $x, y \ge 1$ ולכן גם מתקיים בהתאם לאי־השוויון $|f'(x)| \le 1$, לכן הפונקציה $f'(x)$ חסומה. \\*
	משאלה 8.9 נובע כי $f(x)$ רציפה במידה שווה בקטע $[1, \infty)$. \\* % chktex 9
	הפונקציה $f(x)$ רציפה בקטע $[0, 1]$ וחסומה בו, ולכן ממשפט קנטור נובע כי היא רציפה בו במידה שווה. \\*
	בשל כך כמובן נוכל לראות כי לכל $[0, a]$ כאשר $a > 0$ הפונקציה רציפה בו במידה שווה, ובסך־הכול נקבל כי $f(x)$ רציפה במידה שווה ב־$[0, \infty)$. %chktex 9
\end{proof}

\section{שאלה 5}
\subsection{סעיף א'}
תהי $f$ פונקציה גזירה ב־$\RR_{\ge a} = [a, \infty)$. % chktex 9

\textbf{(i)}
נוכיח כי אם קיים קבוע $m > 0$ כך ש־$f'(x) \ge m$ לכל $x \in \RR_{\ge a}$ אז $\lim_{x \to \infty} f(x) = \infty$.
\begin{proof}
	ממשפט 8.18 נובע ישירות כי $f$ עולה ב־$\RR_{\ge a}$, ולכן מהגדרת גבול פונקציות באינסוף מתקיים
	\[
		\lim_{x \to \infty} f(x) = \infty
	\]
\end{proof}

\textbf{(ii)}
נוכיח כי אם קיים קבוע $m > 0$ כך ש־$f'(x) \le -m$ לכל $x \in \RR_{\ge a}$ אז $\lim_{x \to \infty} f(x) = - \infty$.
\begin{proof}
	נובע באופן דומה ממשפט 8.18.
\end{proof}

\subsection{סעיף ב'}
תהי $f$ פונקציה גזירה פעמיים ב־$(0, \infty)$ כך ש־$f''(x) > 0$ לכל $x \in (0, \infty)$. \\*
נתון כי
\[
	\lim_{x \to \infty} f(x) = L
\]

\textbf{(i)}
נוכיח כי $f'(x) < 0$ לכל $x \in (0, \infty)$.
\begin{proof}
	ידוע כי $f''(x) > 0$ לכל $x$ בתחום, ולכן ממשפט 8.18 נובע כי $f'(x)$ עולה לכל $x \in (0, \infty)$. \\*
	נניח בשלילה כי ישנה נקודה $x_0$ אשר בה $f'(x_0) = 0$. \\*
	בשל עלייתה בכל נקודה, בהכרח לכל $x < x_0$ מתקיים $f'(x) < 0$ ובאופן דומה לכל $x > x_0$ גם $f'(x) > 0$. \\*
	בקטע $(x_0, \infty)$ ממשפט 8.18 נובע כי $f(x)$ עולה לכל נקודה, ולכן על־פי הגדרת הגבול לאינסוף
	\[
		\lim_{x \to \infty} f(x) = \infty
	\]
	בסתירה לטענה כי $f(x)$ מתכנסת לערך ממשי $L$. \\*
	בשל כך לא קיימת נקודה $x_0$ עבורה $f'(x) = 0$ ולכל $x \in (0, \infty)$ מתקיים $f'(x) < 0$.
\end{proof}

\textbf{(ii)}
נוכיח כי $\sup f'((0, \infty)) = 0$
\begin{proof}
	ממשפט לופיטל נובע כי
	\[
		0 = \lim_{x \to \infty} \frac{f(x)}{x} = \lim_{x \to \infty} \frac{f'(x)}{1}
	\]
	דהינו
	\[
		\lim_{x \to \infty} f'(x) = 0
	\]
	אנו גם יודעים כי $f'(x) < 0$ לכל $x \in (0, \infty)$. \\*
	משילוב טענות אלה אנו מקבלים כי לכל $\epsilon > 0$ קיים $M$ כך שלכל $x > M$ מתקיים $f'(x) > -\epsilon$.
	מצאנו כי לכל $a < 0$ קיים $b \in (0, \infty)$ כך ש־$f'(b) \ge a$ ולכן נובע כי
	\[
		\sup f'((0, \infty)) = 0
	\]
\end{proof}

\textbf{(iii)}
נוכיח כי
\[
	\lim_{x \to \infty} f'(x) = 0
\]
\begin{proof}
	למעשה כבר הוכחנו זאת בתת־סעיף הקודם.
\end{proof}

\section{שאלה 6}
נחשב את הגבולות הבאים או נוכיח כי אינם קיימים.

\subsection{סעיף א'}
\[
	\lim_{x \to 1} \frac{x^x - x}{\ln x - x + 1}
\]
מחישוב ישיר מתקבל
\[
	1^1 - 1 = 0, \ln 1 - 1 + 1 = 0
\]
נשים לב כי
\[
	(x^x)' = (e^{x \ln x})' = (1 \cdot \ln x + x \frac{1}{x}) e^{x \ln x} = (\ln x + 1) x^x
\]
ולכן על־פי משפט לופיטל
\begin{align*}
	\lim_{x \to 1} \frac{x^x - x}{\ln x - x + 1}
	& = \lim_{x \to 1} \frac{(x^x - x)'}{(\ln x - x + 1)'} \\
	& = \lim_{x \to 1} \frac{(\ln x + 1) x^x - 1}{\frac{1}{x} - 1} \\
	& = \lim_{x \to 1} \frac{\frac{1}{x} x^x + \ln x (\ln x + 1) x^x + (\ln x + 1) x^x}{\frac{-1}{x^2}} && \text{על־פי משפט לופיטל} \\
	& = \frac{\frac{1}{1} 1^1 + \ln 1 (\ln 1 + 1) 1^1 + (\ln 1 + 1) 1^1}{\frac{-1}{1^2}} \\
	& = \frac{1 + 1}{-1} = -2
\end{align*}
ולכן
\[
	\lim_{x \to 1} \frac{x^x - x}{\ln x - x + 1} = -2
\]

\subsection{סעיף ב'}
\[
	\lim_{x \to 0} x (e^\frac{1}{x} - 1)
\]
נחשב את שני הגבולות החד־צדדיים.
\[
	\lim_{x \to 0^+} x (e^\frac{1}{x} - 1)
	= \lim_{t \to \infty} \frac{e^t - 1}{t}
	= \lim_{t \to \infty} \frac{e^t}{1}
	= \infty
\]
על־פי משפט הרכבה עבור $t = \frac{1}{x}$ ומשפט לופיטל.
\[
	\lim_{x \to 0^-} x (e^\frac{1}{x} - 1)
	= \lim_{t \to -\infty} \frac{e^t - 1}{t}
	= \lim_{t \to -\infty} \frac{e^t}{1}
	= 0
\]
על־פי חישוב דומה. \\*
מצאנו כי הגבולות החד־צדדיים שונים ולכן בהתאם הגבול לא מתקיים.

\subsection{סעיף ג'}
\[
	\lim_{x \to \infty} {\left( \frac{2}{\pi} \arctan x \right)}^x
\]
\[
	{\left( \frac{2}{\pi} \arctan x \right)}^x
	= e^{x \ln \left( \frac{2}{\pi} \arctan x \right)}
\]
נבדוק את גבול המעריך:
\begin{align*}
	\lim_{x \to \infty} x \ln \left( \frac{2}{\pi} \arctan x \right)
	& = \lim_{x \to \infty} \frac{\ln \left( \frac{2}{\pi} \arctan x \right)}{\frac{1}{x}} \\
	& = \lim_{x \to \infty} \frac{\frac{\frac{2}{\pi} \frac{1}{1 + x^2}}{\frac{2}{\pi} \arctan x}}{\frac{-1}{x^2}} && \text{לופיטל} \\
	& = \lim_{x \to \infty} \frac{x^2}{-\arctan x (1 + x^2)} \\
	& = \lim_{x \to \infty} \frac{1}{-\arctan x } \lim_{x \to \infty} \frac{x^2}{1 + x^2} \\*
	& = \frac{-2}{\pi}
\end{align*}
ולכן ממשפט פונקציית הרכבה נובע
\[
	\lim_{x \to 0} x (e^\frac{1}{x} - 1) = e^{-\frac{2}{\pi}}
\]

\section{שאלה 7}
\subsection{סעיף א'}
נוכיח כי הפונקציה
\[
	f(x) = \frac{1}{x} + \ln x
\]
מקבלת בקטע $(0, \infty)$ מינימום בנקודה $x = 1$.
\begin{proof}
	ממשפט 8.4 נובע כי בנקודת קיצון של $f(x)$ ערך הנגזרת $f'(x)$ הוא $0$, לכן נחשב את הנגזרת ונשווה ל־$0$:
	\[
		f'(x) = \frac{-1}{x^2} + \frac{1}{x}
	\]
	נשווה ונפתור
	\begin{align*}
		& f'(x) = 0 \\
		& \frac{-1}{x^2} + \frac{1}{x} = 0 \\
		& -1 + x = 0 \\
		& x = 1
	\end{align*}
	אנו יודעים כי ל־$f'(x)$ נקודת איפוס אחת בלבד, ומהגדרתה אנו גם יודעים כי היא רציפה,
	לכן ממשפט ערך הביניים נובע ישירות כי $f'(x) < 0$ לכל $x < 1$, ובאופן דומה גם $f'(x) > 0$ לכל $x > 1$. \\*
	ממשפט 8.18 מתקבל כי $f(x)$ יורדת כאשר $x < 1$, ועולה כאשר $x > 1$, לכן בכל סביבה שנבחר $x = 1$ היא מינימום מקומי של $f(x)$,
	דהינו היא נקודת מינימום של הפונקציה.
\end{proof}

\subsection{סעיף ב'}
נגדיר
\[
	g(x) = e^x \ln x
\]
נוכיח כי הפונקציה מקבלת כל ערך ממשי בדיוק פעם אחת.
\begin{proof}
	תחילה נראה כי מהגדרות רכיבי $g(x)$ ומאריתמטיקה של הגבולות נובע:
	\[
		\lim_{x \to 0^+} g(x) = -\infty
	\]
	מגבול זה אנו יכולים להסיק כי בסביבה החיובית של הפונקציה $0$ $g(x)$ עולה. \\*
	עוד אנו יכולים להסיק מרכיביה את עלייתה גם בתחום $(1, \infty)$. \\*
	ניתן לקבוע כי $g(x)$ עולה לכל $x \in (0, \infty)$ ובהתאם לכל $x_1 < x_2$ מתקיים $g(x_1) < g(x_2)$,
	לכן היא מקבלת כל ערך ממשי פעם אחת בלבד.
\end{proof}

\end{document} % chktex 17
