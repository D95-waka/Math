\documentclass[a4paper]{article}

% packages
\usepackage{inputenc, fontspec, amsmath, amsthm, amsfonts, polyglossia, catchfile}
\usepackage[a4paper, margin=50pt, includeheadfoot]{geometry} % set page margins

% style
\AddToHook{cmd/section/before}{\clearpage}	% Add line break before section
\linespread{1.5}
\setcounter{secnumdepth}{0}		% Remove default number tags from sections
\setmainfont{Libertinus Serif}
\setsansfont{Libertinus Sans}
\setmonofont{Libertinus Mono}
\setdefaultlanguage{hebrew}
\setotherlanguage{english}

% operators
\DeclareMathOperator\cis{cis}
\DeclareMathOperator\Sp{Sp}
\DeclareMathOperator\tr{tr}
\DeclareMathOperator\im{Im}
\DeclareMathOperator\diag{diag}
\DeclareMathOperator*\lowlim{\underline{lim}}
\DeclareMathOperator*\uplim{\overline{lim}}

% commands
\renewcommand\qedsymbol{\textbf{משל}}
\newcommand{\NN}[0]{\mathbb{N}}
\newcommand{\ZZ}[0]{\mathbb{Z}}
\newcommand{\QQ}[0]{\mathbb{Q}}
\newcommand{\RR}[0]{\mathbb{R}}
\newcommand{\CC}[0]{\mathbb{C}}
\newcommand{\getenv}[2][] {
  \CatchFileEdef{\temp}{"|kpsewhich --var-value #2"}{\endlinechar=-1}
  \if\relax\detokenize{#1}\relax\temp\else\let#1\temp\fi
}
\newcommand{\explain}[2] {
	\begin{flalign*}
		 && \text{#2} && \text{#1}
	\end{flalign*}
}

% headers
\getenv[\AUTHOR]{AUTHOR}
\author{\AUTHOR}
\date\today

\title{פתרון ממ''ן 15 – חשבון אינפיניטסימלי 1 (20474)}

\begin{document}
\maketitle
\section{שאלה 1}
\subsection{סעיף א'}
נחשב את הגבול הבא
\[
	\lim_{n \to \infty} {\left( 1 + \sin \frac{1}{n^2} \right)}^{n^2}
\]
מהגדרת $\sin x$ אנו יכולים להסיק כי למשוואה $\sin x = x$ ישנו רק פתרון אחד כאשר $x = 0$,
מהרציפות של שתי הפונקציות והעובדה כי $\sin 1 < 1$ (על־פי חישוב ישיר) נובע כי $\sin x < x$ לכל $0 < x$. \\*
ניתן באופן דומה לראות כי $\alpha x < \sin x$ בסביבה חיובית של $0$
\begin{align*}
	& \alpha x < \sin x < x \\
	& \frac{1}{\alpha n^2} < \sin \frac{1}{n^2} < \frac{1}{n^2} \\
	& 1 + \frac{1}{n^4} < 1 + \sin \frac{1}{n^2} < 1 + \frac{1}{n^2} \\
	& {\left( 1 + \frac{1}{\alpha n^2} \right)}^{\alpha n^2} < {\left( 1 + \sin \frac{1}{n^2} \right)}^{n^2} < {\left(1 + \frac{1}{n^2} \right)}^{n^2} \\
	& e^\alpha \le \lim_{n \to 0} {\left( 1 + \sin \frac{1}{n^2} \right)}^{n^2} \le e
\end{align*}
וכאשר $\alpha \to 1$ נקבל
\[
	\lim_{n \to 0} {\left( 1 + \sin \frac{1}{n^2} \right)}^{n^2} = e
\]

\subsection{סעיף ב'}
נמצא את ערך הגבול
\[
	\lim_{x \to 0} |x|^{\frac{1}{x^2}}
\]
על־פי הרכבת הפונקציה $\frac{1}{x}$:
\begin{align*}
	\lim_{x \to 0} |x|^{\frac{1}{x^2}}
	& = \lim_{t \to \infty} {\left\lvert \frac{1}{t} \right\rvert}^{t^2} \\
	& = \lim_{t \to \infty} \frac{1}{|t|^{t^2} } \\
	& = \infty
\end{align*}
הגבול מתקיים במובן הרחב.

\section{שאלה 2}
תהי פונקציה
\[
	f(x) = e^{-x} + \sin^2 x
\]

\subsection{סעיף א'}
נוכיח כי מתקיים הגבול הבא עבור סדרה
\[
	\lim_{n \to \infty} f(\pi n) = 0
\]
\begin{proof}
	מהגדרת פונקציית $\sin$ אנו יודעים כי $\sin \pi k = 0$ לכל $k \in \ZZ$. \\*
	אז גם $f(\pi k) = e^{-pi k} = {(\frac{1}{e})}^k$, וממשפט 6.9 נובע
	\[
		\lim_{n \to \infty} f(\pi n) = 0
	\]
\end{proof}

\subsection{סעיף ב'}
נוכיח כי
\[
	\inf f([0, \infty)) = 0 % chktex 9
\]
\begin{proof}
	נגדיר $a > 0$, ונראה כי קיים $x_0$ כך ש־$f(x_0) = a$. \\*
	אנו יודעים כי כאשר $x \to \infty$ אז $e^{-x} \to 0$, לכן מהגדרת הגבול ומתחום ההגדרה של $e^{-x}$ אנו יודעים שקיים מספר $x_1 < a$. \\*
	ממשפט ערך הביניים של קושי נובע גם כי $x_0$ כזה המתואר קיים. \\*
	לכן $0 < f(x)$ לכל $x$ בתחום ההגדרה וכל $a > 0$ שנבחר קיים בתמונת $f$, ומהגדרה 3.12 ונבע כי $\inf f([0, \infty)) = 0$. % chktex 9
\end{proof}

\subsection{סעיף ג'}
נוכיח כי הפונקציה $f$ לא מקבלת מינימום בתחום הגדרתה.
\begin{proof}
	נניח בשלילה כי הפונקציה מקבלת מינימום בנקודה $x_0$, ונגדיר כי $f(x_0) = c$. \\*
	כפי שראינו בסעיף הקודם, אילו $c > 0$ אז קיימת נקודה $x_1$ כך ש־$f(x_1) < f(x_0)$.
	לא קיים ערך $x_0$ עבורו $c = 0$, וידוע מהגדרתה כי $f$ לא שלילית לאף ערך $x$. \\*
	על־כן ניתן לקבוע כי לא קיים $x_0$ כזה ובעקבות כך לפונקציה $f$ אין מינימום.
\end{proof}

\section{שאלה 3}
בכל סעיף נמצא את תחום ההגדרה, תחום הרציפות תחום הגזירות, ואת ערך הנגזרת לכל נקודה בתחום הגזירות של הפונקציה המוגדרת.

\subsection{סעיף א'}
\[
	f(x) = \begin{cases}
		\sin^2(x) \sin \frac{1}{x} & x \ne 0 \\
		0 & x = 0
	\end{cases}
\]

\end{document} % chktex 17
