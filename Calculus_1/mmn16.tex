\documentclass[a4paper]{article}

% packages
\usepackage{inputenc, fontspec, amsmath, amsthm, amsfonts, polyglossia, catchfile}
\usepackage[a4paper, margin=50pt, includeheadfoot]{geometry} % set page margins

% style
\AddToHook{cmd/section/before}{\clearpage}	% Add line break before section
\linespread{1.5}
\setcounter{secnumdepth}{0}		% Remove default number tags from sections
\setmainfont{Libertinus Serif}
\setsansfont{Libertinus Sans}
\setmonofont{Libertinus Mono}
\setdefaultlanguage{hebrew}
\setotherlanguage{english}

% operators
\DeclareMathOperator\cis{cis}
\DeclareMathOperator\Sp{Sp}
\DeclareMathOperator\tr{tr}
\DeclareMathOperator\im{Im}
\DeclareMathOperator\diag{diag}
\DeclareMathOperator*\lowlim{\underline{lim}}
\DeclareMathOperator*\uplim{\overline{lim}}

% commands
\renewcommand\qedsymbol{\textbf{משל}}
\newcommand{\NN}[0]{\mathbb{N}}
\newcommand{\ZZ}[0]{\mathbb{Z}}
\newcommand{\QQ}[0]{\mathbb{Q}}
\newcommand{\RR}[0]{\mathbb{R}}
\newcommand{\CC}[0]{\mathbb{C}}
\newcommand{\getenv}[2][] {
  \CatchFileEdef{\temp}{"|kpsewhich --var-value #2"}{\endlinechar=-1}
  \if\relax\detokenize{#1}\relax\temp\else\let#1\temp\fi
}
\newcommand{\explain}[2] {
	\begin{flalign*}
		 && \text{#2} && \text{#1}
	\end{flalign*}
}

% headers
\getenv[\AUTHOR]{AUTHOR}
\author{\AUTHOR}
\date\today

\title{פתרון ממ''ן 16 – חשבון אינפיניטסימלי 1 (20474)}

\begin{document}
\maketitle
\section{שאלה 1}
\subsection{סעיף א'}
נחשב את הגבול הבא
\[
	\lim_{n \to \infty} {\left( 1 + \sin \frac{1}{n^2} \right)}^{n^2}
\]
מהגדרת $\sin x$ אנו יכולים להסיק כי למשוואה $\sin x = x$ ישנו רק פתרון אחד כאשר $x = 0$,
מהרציפות של שתי הפונקציות והעובדה כי $\sin 1 < 1$ (על־פי חישוב ישיר) נובע כי $\sin x < x$ לכל $0 < x$. \\*
ניתן באופן דומה לראות כי $\alpha x < \sin x$ בסביבה חיובית של $0$ כאשר $0 \le \alpha < 1$:
\begin{align*}
	& \alpha x < \sin x < x \\
	& \frac{1}{\alpha n^2} < \sin \frac{1}{n^2} < \frac{1}{n^2} \\
	& 1 + \frac{1}{\alpha n^2} < 1 + \sin \frac{1}{n^2} < 1 + \frac{1}{n^2} \\
	& {\left( 1 + \frac{1}{\alpha n^2} \right)}^{\alpha n^2} < {\left( 1 + \sin \frac{1}{n^2} \right)}^{n^2} < {\left(1 + \frac{1}{n^2} \right)}^{n^2} \\
	& e^\alpha \le \lim_{n \to 0} {\left( 1 + \sin \frac{1}{n^2} \right)}^{n^2} \le e
\end{align*}
וכאשר $\alpha \to 1$ נקבל ממשפט הסדוויץ'
\[
	\lim_{n \to 0} {\left( 1 + \sin \frac{1}{n^2} \right)}^{n^2} = e
\]

\subsection{סעיף ב'}
נמצא את ערך הגבול
\[
	\lim_{x \to 0} |x|^{\frac{1}{x^2}}
\]
על־פי הרכבת הפונקציה $\frac{1}{x}$:
\begin{align*}
	\lim_{x \to 0} |x|^{\frac{1}{x^2}}
	& = \lim_{t \to \infty} {\left\lvert \frac{1}{t} \right\rvert}^{t^2} \\
	& = \lim_{t \to \infty} \frac{1}{|t|^{t^2} } \\
	& = 0
\end{align*}
הגבול מתקיים.

\section{שאלה 2}
תהי פונקציה
\[
	f(x) = e^{-x} + \sin^2 x
\]

\subsection{סעיף א'}
נוכיח כי מתקיים הגבול הבא עבור סדרה
\[
	\lim_{n \to \infty} f(\pi n) = 0
\]
\begin{proof}
	מהגדרת פונקציית $\sin$ אנו יודעים כי $\sin \pi k = 0$ לכל $k \in \ZZ$. \\*
	אז גם $f(\pi k) = e^{-pi k} = {(\frac{1}{e})}^k$, וממשפט 6.9 נובע
	\[
		\lim_{n \to \infty} f(\pi n) = 0
	\]
\end{proof}

\subsection{סעיף ב'}
נוכיח כי
\[
	\inf f([0, \infty)) = 0 % chktex 9
\]
\begin{proof}
	נגדיר $a > 0$, ונראה כי קיים $x_0$ כך ש־$f(x_0) = a$. \\*
	אנו יודעים כי כאשר $x \to \infty$ אז $e^{-x} \to 0$, לכן מהגדרת הגבול ומתחום ההגדרה של $e^{-x}$ אנו יודעים שקיים מספר $x_1 < a$. \\*
	ממשפט ערך הביניים של קושי נובע גם כי $x_0$ כזה המתואר קיים. \\*
	לכן $0 < f(x)$ לכל $x$ בתחום ההגדרה וכל $a > 0$ שנבחר קיים בתמונת $f$, ומהגדרה 3.12 ונבע כי $\inf f([0, \infty)) = 0$. % chktex 9
\end{proof}

\subsection{סעיף ג'}
נוכיח כי הפונקציה $f$ לא מקבלת מינימום בתחום הגדרתה.
\begin{proof}
	נניח בשלילה כי הפונקציה מקבלת מינימום בנקודה $x_0$, ונגדיר כי $f(x_0) = c$. \\*
	כפי שראינו בסעיף הקודם, אילו $c > 0$ אז קיימת נקודה $x_1$ כך ש־$f(x_1) < f(x_0)$.
	לא קיים ערך $x_0$ עבורו $c = 0$, וידוע מהגדרתה כי $f$ לא שלילית לאף ערך $x$. \\*
	על־כן ניתן לקבוע כי לא קיים $x_0$ כזה ובעקבות כך לפונקציה $f$ אין מינימום.
\end{proof}

\section{שאלה 3}
בכל סעיף נמצא את תחום ההגדרה, תחום הרציפות תחום הגזירות, ואת ערך הנגזרת לכל נקודה בתחום הגזירות של הפונקציה המוגדרת.

\subsection{סעיף א'}
\[
	f(x) = \begin{cases}
		\sin^2(x) \sin \frac{1}{x} & x \ne 0 \\
		0 & x = 0
	\end{cases}
\]
מהגדרת פונקציית $\sin$ נובע כי הפונקציה $f(x)$ מוגדרת לכל $x \ne 0$, ומוגדר $f(0) = 0$, לכן $f(x)$ מוגדרת לכל $x \in \RR$. \\*
ממסקנה 5.12, משפט 5.13 והאריתמטיקה של פונקציות רציפות נובע כי $f(x)$ רציפה לכל $x \ne 0$. נוכיח כי $f(x)$ רציפה ב־$x = 0$.
\begin{proof}
	על־פי הגדרת הרציפות בנקודה, עלינו להוכיח כי
	\[
		\lim_{x \to 0} f(x) = f(0) = 0
	\]
	ידוע כי $\sin x$ היא חסומה, ולכן גם $\sin \frac{1}{x}$ חסומה, ומטענה 4.44 נובע כי
	\[
		\lim_{x \to 0} \sin^2 x = 0
	\]
	מהגדרת ההתכנסות בנקודה לפונקציה לפי היינה ומהגדרה 2.13 נובע כי
	\[
		\lim_{x \to 0} \sin^2 x \sin \frac{1}{x} = 0 = f(0)
	\]
\end{proof}
מצאנו כי $f(x)$ מוגדרת ורציפה לכל $x \in \RR$. \\*
נגדיר $f'(x)$ פונקציית הנגזרת של $f(x)$, ונבדוק מתי היא מוגדרת. \\*
על־פי משפט 7.14 ומשפט 7.21 $f(x)$ גזירה לכל $x \ne 0$. נחשבה:
\[
	f'(x) = (\sin^2 x)' \sin \frac{1}{x} + \sin^2 x (\sin \frac{1}{x})'
	= 2 \sin x \cos x \sin \frac{1}{x} - \frac{1}{x^2} \sin^2 x \cos \frac{1}{x}
\]
נבדוק עתה את $x = 0$: \\*
מטענה 7.8 נובע כי נוכל לבדוק את קיום הגבול הבא
\begin{align*}
	\lim_{x \to 0} \frac{f(x) - f(0)}{x - 0} & = \lim_{x \to 0} \frac{\sin^2 x \sin \frac{1}{x}}{x} \\
											 & = \lim_{x \to 0} \frac{\sin x}{x} \cdot \lim_{x \to 0} \sin x \sin \frac{1}{x} \\
											 & = 1 \cdot 0
\end{align*}
מצאנו כי הגבול קיים ולכן מהגדרה 7.8 נובע כי $f'(0) = 0$.

\subsection{סעיף ב'}
\[
	g(x) = |\ln x|
\]
מהגדרה 6.11 והגדרת הערך המוחלט נובע כי $g(x)$ מוגדרת לכל $x \ge 0$. \\*
ממשפט 6.10 והגדרה 6.11 נובע כי $g(x)$ רציפה בכל תחום חיוביותה, דהינו כאשר $x > 1$.
בתחום $0 < x < 1$ הפונקציה $g(x)$ מקיימת $g(x) = - \ln x$ ולכן רציפה גם כן. \\*
נבחן את $x = 1$, בנקודה זו מטענה 6.13 ורציפות הלוגריתם נובע
\[
	\lim_{x \to 1^+} g(x) = \lim_{x \to 1^-} g(x) = 0 = g(1)
\]
ולכן $g(x)$ רציפה בכל תחום הגדרתה. \\*
נמצא את תחום הגזירות של $g(x)$. \\*
בתחום $x > 1$ ממשפט 7.29 נובע ישירות כי $g'(x) = \frac{1}{x}$. בתחום $0 < x < 1$ ראינו כי $g(x) = - \ln x$ ובהתאם $g'(x) = \frac{-1}{x}$. \\*
מטענה 7.10 והרכבת פונקציות נובע כי $g(x)$ לא גזירה כאשר $x = 1$.

\section{שאלה 4}
תהי $f$ פונקציה זוגית ב־$\RR$. נוכיח כי אם $f(x)$ גזירה כאשר $x = 0$ אז $f'(0) = 0$.
\begin{proof}
	נניח כי $f$ גזירה ב־$0$ ומהזוגיות נובע כי $f(x) = f(-x)$ לכל $x$. \\*
	תהי $x_0$ נקודה כלשהי אשר $f$ מוגדרת וגזירה בה. מטענה 7.7 נובע
	\[
		L = \lim_{h \to 0} \frac{f(x_0 + h) - f(x_0)}{h}
	\]
	בשל הגזירות בנקודה הגבול מוגדר. עתה נראה כי מהזוגיות של $f$ נובע
	\[
		\lim_{h \to 0} \frac{f(-x_0 + h) - f(-x_0)}{h}
		= \lim_{h \to 0} \frac{- f(x_0 - h) + f(x_0)}{h}
		= \lim_{h \to 0} \frac{f(x_0 - h) - f(x_0)}{-h}
	\]
	נגדיר $h' = -h$ בהתאם למשפט 4.39
	\[
		L = - \lim_{h' \to 0} \frac{f(x_0 + h') - f(x_0)}{h'}
	\]
	לכן על־פי הגדרה 7.7 הפונקציה $f$ גזירה בנקודה $-x_0$ ומתקיים $f'(-x_0) = -f'(x_0)$. \\*
	ידוע כי $0$ נקודת גזירות, לכן $f'(-0) = -f'(0)$ ונובע בהכרח $f'(0) = 0$.
\end{proof}

\section{שאלה 5}
יהי $a \in \RR$ ופונקציה המוגדרת
\[
	f(x) = \begin{cases}
		x + x e^{\frac{1}{x}} & x < 0 \\
		0 & x = 0 \\
		\frac{a - 2 \cos x}{\sin x} & x > 0
	\end{cases}
\]

\subsection{סעיף א'}
נמצא את כל ערכי $a$ שעבורם הפונקציה $f$ רציפה ב־$x = 0$. \\*
על־פי אריתמטיקה של גבולות, גבול הרכבה והגדרת אקספוננט:
\[
	\lim_{x \to 0^-} f(x) = 
	\lim_{x \to 0^-} x + x e^{\frac{1}{x}}
	= 0 = f(0)
\]
לכן הפונקציה $f$ רציפה משמאל בנקודה $x = 0$. \\*
נבדוק את רציפות $f$ מימין כאשר $a \ne 2$: \\*
מטענה 5.44 נובע כי $a - 2 \cos x$ רציפה בכל הגדרתה, ובהצבה נקבל $a - 2 \cos 0 = a - 2 \ne 0$. \\*
לכן ממשפט 2.43 סעיף ו' נובע (כתלות בהפרש $a - 2$)
\[
	\lim_{x \to 0^+} \frac{a - 2 \cos x}{\sin x} = \frac{\pm1}{\infty} = \pm \infty
\]
לכן הפונקציה $f$ לא רציפה ב־$0$ כאשר $a \ne 2$. \\*
נבדוק עתה את המקרה $a = 2$: \\*
במקרה זה
\[
	\frac{a - 2 \cos x}{\sin x}
	= 2 \frac{1 - \cos x}{\sin x}
	= 2 \frac{(1 - \cos x)(1 + \cos x)}{\sin x(1 + \cos x)}
	= 2 \frac{\sin^2 x}{\sin (1 + \cos x)}
	= \frac{2 \sin x}{1 + \cos x}
\]
לכן בהתאם
\[
	\lim_{x \to 0^+} f(x) = \lim_{x \to 0^+} \frac{2 \sin x}{1 + \cos x} = \frac{2 \cdot 0}{1 + 0} = 0
\]
ממשפט 4.48 נובע כי
\[
	\lim_{x \to 0} f(x) = 0 = f(0)
\]
ולכן $f$ רציפה ב־$0$ כאשר $a = 2$ בלבד.

\subsection{סעיף ב'}
נמצא את כל ערכי $a$ עבורם $f$ גזירה ב־$0$. \\*
נניח בשלילה כי קיים $a \ne 2$ עבורו $f$ גזירה ב־$0$. מטענה 7.9 נובע כי $f$ רציפה ב־$0$, אבל ידוע כי $f$ איננה רציפה בנקודה זו ולכן זוהי סתירה.
אז כאשר $a \ne 2$ הפונקציה אינה גזירה ב־$x = 0$. \\*
נבחן את המקרה $a = 2$. \\*
מטענה 7.8 נובע כי $f$ גזירה ב־$0$ אם ורק אם מתקיים הגבול
\[
	\lim_{x \to 0} \frac{f(x) - f(0)}{x - 0}
	= \lim_{x \to 0} \frac{f(x)}{x}
\]
נחשב את שני הגבולות החלקיים בנקודה לפי משפט 4.48.
\[
	\lim_{x \to 0^-} \frac{f(x)}{x}
	= \lim_{x \to 0^-} \frac{x + xe^{\frac{1}{x}}}{x}
	= \lim_{x \to 0^-} 1 + e^{\frac{1}{x}}
	= 1
\]
\[
	\lim_{x \to 0^+} \frac{f(x)}{x}
	= \lim_{x \to 0^+} \frac{\frac{2 - 2 \cos x}{\sin x}}{x}
	= \lim_{x \to 0^+} \frac{2 \sin x}{x(1 + \cos x)}
	= 1 \cdot \frac{2}{1 + 1} = 1
\]
לכן הגבול מוגדר וערכו $1$ ובהתאם $f$ גזירה בנקודה $x = 0$ כאשר $a = 2$ בלבד.

\end{document} % chktex 17
