\documentclass[a4paper]{article}

% packages
\usepackage{inputenc, fontspec, amsmath, amsfonts, polyglossia, catchfile}
\usepackage[a4paper, margin=50pt, includeheadfoot]{geometry} % set page margins

% style
\AddToHook{cmd/section/before}{\clearpage}	% Add line break before section
\setdefaultlanguage{hebrew}
\setotherlanguage{english}
\setmainfont{Libertinus Serif}
\linespread{1.5}
\setcounter{secnumdepth}{0}		% Remove default number tags from sections

% custom operators
\newcommand{\getenv}[2][]{%
  \CatchFileEdef{\temp}{"|kpsewhich --var-value #2"}{\endlinechar=-1}%
  \if\relax\detokenize{#1}\relax\temp\else\let#1\temp\fi}
\getenv[\AUTHOR]{AUTHOR}
\DeclareMathOperator\cis{cis}
\DeclareMathOperator\Sp{Sp}
\DeclareMathOperator\tr{tr}
\DeclareMathOperator\im{Im}
\DeclareMathOperator\diag{diag}
\DeclareMathOperator*\lowlim{\underline{lim}}
\DeclareMathOperator*\uplim{\overline{lim}}
\def\NN{\mathbb{N}}
\def\RR{\mathbb{R}}
\def\CC{\mathbb{C}}

% theorems
\title{פתרון ממ''ן 13 – חשבון אינפיניטסימלי 1 (20474)}
\author{\AUTHOR}
\date\today

\begin{document}
\maketitle
\section{שאלה 1}
\subsection{סעיף א'}
נגדיר $a_1 = \sqrt{2}$ ולכל $n$:
\[
	a_{n + 1} = 1 - \frac{5}{4a_n + 8}
\]
נוכיח כי הסדרה מוגדרת לכל $n$: \\*
הסדרה איננה מוגדרת במקרה בו קיים $a_n = -2$.
נוכיח באינדוקציה שמקרה זה לא מתקיים על־ידי ההוכחה כי $a_n > 0$ לכל $n$: \\*
\textbf{בסיס האינדוקציה:} נתון $a_1 = \sqrt{2} > 0$. \\*
\textbf{מהלך האינדוקציה:} נניח כי $a_n > 0$ ונראה כי $a_{n + 1} > 0$. \\*
אם $a_n > 0$ אז גם $4a_n > 0$ ובהתאם $4a_n + 8 > 8 > 5$.
נחלק את אי־השוויון בביטוי $4a_n + 8$:
\[
	1 > \frac{5}{4a_n + 8}
	\rightarrow
	1 - \frac{5}{4a_n + 8} > 0
	\rightarrow
	a_{n + 1} > 0
\]
ובכך השלמנו את מהלך האינדוקציה. \\*
הסדרה $(a_n)$ חיובית לכל איבר ולכן בהכרח מוגדרת לכל $n$.

\subsection{סעיף ב'}
נוכיח כי $a_n$ הוא מספר חיובי ואי־רציונלי לכל $n$. \\*
הראינו בסעיף א' כי $a_n$ חיובי לכל $n$,
לכן עלינו רק להוכיח כי הוא גם אי־רציונלי.
נוכיח זאת באינדוקציה: \\*
\textbf{בסיס האינדוקציה:} נתון $a_1 = \sqrt{2}$ וזהו כמובן מספר אי־רציונלי. \\*
\textbf{מהלך האינדוקציה:} נניח $a_n$ אי־רציונלי
ונראה כי גם $a_{n + 1}$ אי־רציונלי. \\*
על־פי הגדרת המספרים האי־רציונליים,
חיבור ומכפלת מספר אי־רציונליים אינם רציונליים.
לכן גם הביטוי $4a_n + 8$ איננו מספר רציונלי.
בהתאם לפעולות החיבור והכפל,
גם הפעולות ההפוכות להם משמרות את תכונת אי־הרציונליות,
לכן גם המספר
\[
	1 - \frac{5}{4a_n + 8}
\]
איננו רציונלי. זהו כמובן $a_{n + 1}$ והשלמנו את מהלך האינדוקציה.
לכן $a_n$ הוא מספר חיובי ואי־רציונלי לכל $n$.

\subsection{סעיף ג'}
נוכיח כי הסדרה $(a_n)$ מתכנסת ונמצא את ערך גבולה. \\*
תחילה נוכיח באינדוקציה כי לכל $n$ מתקיים $a_{n + 1} < a_n$: \\*
\textbf{בסיס האינדוקציה:}
\[
	a_2 = 1 - \frac{5}{4 \sqrt{2} + 8} = 0.633 \ldots < a_1
\]
\textbf{מהלך האינדוקציה:}
נניח כי $a_n < a_{n - 1}$ ונוכיח כי $a_{n + 1} < a_n$.
\begin{align*}
	& a_n < a_{n - 1} \\
	& 4 a_n < 4 a_{n - 1} \\
	& 4 a_n + 8 < 4 a_{n - 1} + 8 \\
	& \frac{1}{4 a_n + 8} > \frac{1}{4 a_{n - 1} + 8} \\
	& -1 + \frac{5}{4 a_n + 8} > -1 + \frac{5}{4 a_{n - 1} + 8} \\
	& 1 - \frac{5}{4 a_n + 8} < 1 - \frac{5}{4 a_{n - 1} + 8} \\
	& a_{n + 1} < a_n
\end{align*}
התנאי מתקיים ולכן הסדרה יורדת. \\*
ידוע כי $a_n > 0$ לכל $n$,
וראינו כי הסדרה יורדת ולכן $a_n < a_1 = \sqrt{2}$ לכל $n$,
לכן הסדרה $(a_n)$ חסומה ויורדת, ולכן לפי משפט 3.16 מתכנסת.
נגדיר $\lim_{n \to \infty} a_n = L$. לכן מתקיים
\begin{align*}
	L
	& = \lim_{n \to \infty} a_{n + 1} \\
	& = \lim_{n \to \infty} 1 - \frac{5}{4 a_{n - 1} + 8} \\
	& = \lim_{n \to \infty} 1 - \lim_{n \to \infty} \frac{5}{4 a_{n - 1} + 8} \\
	& = 1 - \frac{\lim_{n \to \infty} 5}{\lim_{n \to \infty} 4 a_{n - 1} + 8} \\
	L & = 1 - \frac{5}{4 L + 8} \\
	L (4L + 8) & = 1(4L + 8) - 5 \\
	4L^2 + 8L & = 4L + 3 \\
	4L^2 + 4L - 3 & = 0 \\
	L & = \frac{-4 \pm \sqrt{4^2 - 4 \cdot 4 \cdot (-3)}}{2 \cdot 4} \\
	L & = \frac{-1 \pm 2}{2} \\
	L & = \frac{1}{2}, -\frac{3}{2}
\end{align*}
על־פי משפט 3.16 לא יתכן כי $L < 0$, לכן $L = \frac{1}{2}$.

\section{שאלה 2}
\subsection{סעיף א'}
נחשב את הגבול
\[
	\lim_{n \to \infty}
	\frac{{(-3)}^{n + 1} - {(-2)}^n + 5}{3^{n + 2} + 2^n - 5}
\]
נגדיר שתי תת־סדרות $(a_{n_k})$ ו־$(a_{m_k})$
המוגדרות כסדרת האיברים הזוגיים והאי־זוגיים בהתאמה של הסדרה.
סדרות אלה מכסות את הסדרה המקורית, ומתקיים:
\[
	(a_{n_k}) = \frac{-3^{n + 1} - 2^n + 5}{3^{n + 2} + 2^n - 5}
\]
נחשב את גבולה:
\begin{align*}
	& \lim_{n \to \infty} (a_{n_k}) \\
	& = \lim_{n \to \infty} \frac{-3^{n + 1} - 2^n + 5}{3^{n + 2} + 2^n - 5} \\
	& = \lim_{n \to \infty}
	\frac{-\frac{1}{3} - \frac{2^n}{3^{n + 2}} + \frac{5}{3^{n+2}}}
	{1 + \frac{2^n}{3^{n + 2}} - \frac{5}{3^{n+2}}} \\
	& = - \frac{1}{3}
\end{align*}
באופן דומה נראה כי
\[
	(a_{m_k}) = \frac{3^{m + 1} + 2^m + 5}{3^{m + 2} + 2^m - 5}
\]
חישוב דומה יוביל אותנו למסקנה
\[
	\lim_{m \to \infty} (a_{m_k}) = \frac{1}{3}
\]
ראינו כי גבולות תת־הסדרות של הסדרה מתכנסות לערכים שונים,
לכן לפי משפט 3.31 הסדרה עצמה מתבדרת,
וגבולותיה החלקיים הם $-\frac{1}{3}, \frac{1}{3}$.

\subsection{סעיף ב'}
נחשב את הגבול
\[
	\lim_{n \to \infty}
	\frac{{(-3)}^{n + 1} - 4^n + 5}{3^{n + 2} + 2^n - 5}
\]
בדומה לסעיף א' נראה כי גבול סדרת האברים הזוגיים הוא:
\[
	\lim_{n \to \infty}
	\frac{-3^{n + 1} - 4^n + 5}{3^{n + 2} + 2^n - 5}
	= \lim_{n \to \infty}
	\frac{\frac{-3^{n + 1}}{4^n} - 1 + \frac{5}{4^n}}
	{\frac{3^{n + 2}}{4^n} + \frac{2^n}{4^n} - \frac{5}{4^n}}
	-\frac{1}{0^+}
	= - \infty
\]
גבול האברים האי־זוגיים הוא, על־פי חישוב דומה:
\[
	\lim_{n \to \infty}
	\frac{3^{n + 1} - 4^n + 5}{3^{n + 2} + 2^n - 5}
	= - \infty
\]
על־פי משפט 3.31 מתקיים:
\[
	\lim_{n \to \infty}
	\frac{{(-3)}^{n + 1} - 4^n + 5}{3^{n + 2} + 2^n - 5}
	= - \infty
\]

\subsection{סעיף ג'}
נוכיח כי לא מתקיים הגבול
\[
	\lim_{n \to \infty} 2 \lfloor \frac{n}{2} \rfloor - n
\]
נגדיר שתי תת־סדרות $(a_{n_k})$ ו־$(a_{m_k})$
המוגדרות כסדרת האברים הזוגיים והאי־זוגיים בהתאמה של הסדרה המקורית. \\*
כאשר $n$ זוגי $\frac{n}{2}$ הוא מספר שלם, לכן נוכל לכתוב
\[
	a_{n_k} = 2 \frac{n}{2} - n = 0
\]
לכן
\[
	\lim_{n \to \infty}a_{n_k} = 0
\]
כאשר $n$ אי־זוגי, חלוקתו ב־$2$ תוביל לערך חצי,
ערך זה מחוסר בעקבות השמת החלק השלם.
לכן מתקיים
\[
	a_{m_k} = 2 \frac{n - 1}{2} - n = -1
\]
ובהתאם
\[
	\lim_{m \to \infty}a_{m_k} = -1
\]
על־פי משפט 3.31 לא קיים גבול לסדרה המקורית, שכן תת־גבולותיה שונים זה מזה.
ערכי תת־הגבולות של הסדרה הם $0$ ו־$-1$.

\subsection{סעיף ד'}
נוכיח כי הגבול הבא קיים
\[
	\lim_{n \to \infty} \frac{\sqrt[n]{n!}}{n}
\]
תחילה נגדיר את הסדרה $(a_n)$ כך שלכל $n$ מתקיים
\[
	a_n = \frac{n!}{n^n}
\]
מתקיים
\[
	\frac{a_{n + 1}}{a_n}
	= \frac{\frac{(n + 1)!}{{(n + 1)}^{n + 1}}}{\frac{n!}{n^n}}
	= \frac{n^n (n + 1)!}{n! {(n + 1)}^{n + 1}}
	= \frac{n^n (n + 1) n!}{n! (n + 1) {(n + 1)}^n}
	= \frac{n^n}{{(n + 1)}^n}
	= {\left( \frac{n}{n + 1} \right)}^n
\]
לכל $n$ טבעי מתקיים
\[
	0 < n < n + 1
	\rightarrow
	0 < \frac{n}{n + 1} < 1
	\rightarrow 
	0 < {\left( \frac{n}{n + 1} \right)}^n < 1
\]
לכן לפי מבחן המנה לגבולות
\[
	\lim_{n \to \infty} a_n = 0
\]
כמעט לכל $n$ מתקיים על־פי הגבול
\[
	\frac{1}{n^n} < a_n = \frac{n!}{n^n} < 1
\]
לאחר שורש
\[
	\frac{1}{n} < \frac{\sqrt[n]{n!}}{n} < 1
\]
לפי משפט 2.32 מתקיים
\[
	\lim_{n \to \infty} \frac{1}{n}
	= \lim_{n \to \infty} 1
	= \lim_{n \to \infty} \frac{\sqrt[n]{n!}}{n}
	= 1
\]

\section{שאלה 3}
\subsection{סעיף א'}
תהי הסדרה
\[
	a_n = n - \sqrt{n} \lfloor \sqrt{n} \rfloor
\]
נוכיח כי הסדרה חסומה מלרע.
על־פי תכונות החלק השלם
\begin{align*}
	\lfloor \sqrt{n} \rfloor & \le \sqrt{n} \\
	\lfloor \sqrt{n} \rfloor \sqrt{n} & \le \sqrt{n} \sqrt{n} = n \\
	0 & \le n - \lfloor \sqrt{n} \rfloor \sqrt{n} \\
	0 & \le a_n
\end{align*}
הסדרה $(a_n)$ חסומה מלרע על־ידי $0$.

\subsection{סעיף ב'}
נוכיח כי $0$ הוא גבול חלקי של הסדרה $(a_n)$.
נגדיר $(n_k)$ כך שלכל $k$ מתקיים
\[
	n_k = k^2
\]
בסדרה החלקית $(a_{n_k})$ השורש של $n$ תמיד יהיה מספר שלם בשל ההגדרת תת־הסדרה,
לכן מתקיים
\[
	a_{n_k} = n - \sqrt{n} \sqrt{n} = n - n = 0
\]
זוהי כמובן סדרה קבועה ולכן קיים לה גבול והוא זהה לערך כלל איברי הסדרה:
\[
	\lim_{k \to \infty} a_{n_k} = 0
\]

\subsection{סעיף ג'}
נמצא את הגבול התחתון של הסדרה. \\*
תחילה נראה כי $\inf (a_n) = 0$.
ראינו קודם כי $0$ הוא חסם מלרע של הסדרה,
עתה נוכיח שהוא חסם מלרע מקסימלי.
על־פי הגדרת חסם תחתון, זהו האיבר המינימלי בקבוצת ערכי הסדרה.
ראינו כי האיבר $0$ מוכל בקבוצת האברים, וכי אין איבר קטן ממנו,
לכן הוא חסם תחתון של הסדרה. \\*
על־פי הגדרת הגבול התחתון, גבול זה הוא הגבול החלקי הקטן ביותר של סדרה,
אנו רואים כי $0$ הוא גבול חלקי המתלכד עם החסם התחתון של הסדרה,
לכן אין גבול קטן ממנו, לכן
\[
	\lowlim_{n \to \infty} a_n = 0
\]

\subsection{סעיף ד'}
מצאנו בסעיף ג' כי $\inf \{ a_n \mid n \in \NN \} = 0$.

\subsection{סעיף ה'}
יהי $\ell \in \NN$. נוכיח כי לכמעט כל $n$ טבעי מתקיים
\[
	n < \sqrt{n^2 + 2\ell} < n + 1
\]
נגדיר $N = \ell$ ונוכיח כי הטענה נכונה ל־$n > N$. \\*
על־פי השוויון מתקיים כמעט לכל $n$
\[
	\ell < n
\]
$\ell$ הוא מספר טבעי ולכן מתקיים
\[
	n^2 < n^2 + 2\ell
\]
אבל ידוע כי $2\ell < 2n$, לכן
\[
	n^2 < n^2 + 2\ell < n^2 + 2n < n^2 + 2n + 1
\]
נבצע שורש:
\[
	n < \sqrt{n^2 + 2\ell} < n + 1
\]

\subsection{סעיף ו'}
יהי $\ell \in \NN$, נוכיח כי
\[
	\lim_{n \to \infty} n(\sqrt{n^2 + 2\ell} - n) = \ell
\]
נראה כי מתקיים
\begin{align*}
	n(\sqrt{n^2 + 2\ell} - n)
	& = n \frac{(\sqrt{n^2 + 2\ell} - n)(\sqrt{n^2 + 2\ell} + n)}
	{\sqrt{n^2 + 2\ell} + n} \\
	& = n \frac{n^2 + 2\ell - n^2} {\sqrt{n^2 + 2\ell} + n} \\
	& = \frac{2\ell n} {\sqrt{n^2 + 2\ell} + n} \\
\end{align*}
לכן הגבול קיים רק אם קיים הגבול
\[
	\lim_{n \to \infty} \frac{2n} {\sqrt{n^2 + 2\ell} + n}
\]
על־פי סעיף ה'
\begin{align*}
	& n < \sqrt{n^2 + 2\ell} < n + 1 \\
	& \frac{1}{2n} > \frac{1}{\sqrt{n^2 + 2\ell} + n} > \frac{1}{2n + 1} &
	\text{על־פי חוקי אי־שוייונות} \\
	& \frac{2n}{2n} > \frac{2n}{\sqrt{n^2 + 2\ell} + n} > \frac{2n}{2n + 1} \\
\end{align*}
לפי משפט 2.32
\[
	\lim_{n \to \infty} \frac{2n} {\sqrt{n^2 + 2\ell} + n} = 1
\]
לכן לפי משפט 2.18
\[
	\lim_{n \to \infty} n(\sqrt{n^2 + 2\ell} - n) = \ell
\]

\subsection{סעיף ז'}
נוכיח כי כל מספר טבעי הוא גבול חלקי של הסדרה $(a_n)$. \\*
נגדיר $\ell \in \NN$ וסדרת אינדקסים $(n_k)$ המוגדרת
\[
	n_k = n^2 + 2\ell
\]
לכן מתקיים עבור $(a_{n_k})$
\[
	a_{n_k} =
	n^2 + 2\ell - \sqrt{n^2 + 2\ell} \lfloor \sqrt{n^2 + 2\ell} \rfloor
\]
על־פי טענת סעיף ה'
\[
	n < \sqrt{n^2 + 2\ell} < n + 1
	\rightarrow
	n = \lfloor \sqrt{n^2 + 2\ell} \rfloor < n + 1
\]
לכן
\[
	a_{n_k} =
	n^2 + 2\ell - \sqrt{n^2 + 2\ell} n = 2\ell - n(\sqrt{n^2 + 2\ell} - n)
\]
על־פי סעיף ו':
\[
	\lim_{n \to \infty} \left( 2\ell - n(\sqrt{n^2 + 2\ell} - n) \right)
	= \lim_{n \to \infty} 2\ell - \lim_{n \to \infty} n(\sqrt{n^2 + 2\ell} - n)
	= 2\ell - \ell = \ell
\]
ראינו כי לכל מספר טבעי $\ell$ קיים גבול חלקי שווה ל־$\ell$ לסדרה $(a_n)$.

\subsection{סעיף ח'}
הסדרה $(a_n)$ איננה חסומה מלעיל.
נראה כי לכל מספר $c$ שנבחר, קיימת תת־סדרה של $(a_n)$
אשר גבולה הוא $c + 1$, ולכן לכמעט כל $n$ מתקיים 
עבור תת־סדרה זו $c < (a_{n_k})$.
לכן לסדרה אין חסם מלעיל.

\subsection{סעיף ט'}
נמצא את הגבול העליון של $(a_n)$. \\*
בסעיף הקודם ראינו כי לכל מספר $M < 0$ שנבחר,
קיים גבול חלקי לסדרה הגדול ממנו.
לכן על־פי הגדרת הגבול במובן הרחב
\[
	\uplim_{n \to \infty} a_n = \infty
\]

\subsection{סעיף י'}
לא קיים $\sup\{ a_n \mid n \in \NN \}$.
בסעיף הקודם ראינו כי הגבול העליון של הסדרה $(a_n)$
הוא אינסוף, לכן בקבוצת האברים שלה לא יהיה מספר גדול ביותר,
עבור כל מספר שנבחר יהיה איבר גדול ממנו. \\*
בהתאם גם לא קיים איבר מקסימלי לקבוצה $\{ a_n \mid n \in \NN \}$,
שכן רליו היה אחד, אז לפי טענה 3.8 היה גם חסם עליון לקבוצה.
\end{document}
