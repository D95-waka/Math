\documentclass[a4paper]{article}

% packages
\usepackage{inputenc, fontspec, amsmath, amsthm, amsfonts, polyglossia, catchfile}
\usepackage[a4paper, margin=50pt, includeheadfoot]{geometry} % set page margins

% style
\AddToHook{cmd/section/before}{\clearpage}	% Add line break before section
\linespread{1.5}
\setcounter{secnumdepth}{0}		% Remove default number tags from sections
\setmainfont{Libertinus Serif}
\setsansfont{Libertinus Sans}
\setmonofont{Libertinus Mono}
\setdefaultlanguage{hebrew}
\setotherlanguage{english}

% operators
\DeclareMathOperator\cis{cis}
\DeclareMathOperator\Sp{Sp}
\DeclareMathOperator\tr{tr}
\DeclareMathOperator\im{Im}
\DeclareMathOperator\diag{diag}
\DeclareMathOperator*\lowlim{\underline{lim}}
\DeclareMathOperator*\uplim{\overline{lim}}

% commands
\renewcommand\qedsymbol{\textbf{משל}}
\newcommand{\NN}[0]{\mathbb{N}}
\newcommand{\ZZ}[0]{\mathbb{Z}}
\newcommand{\QQ}[0]{\mathbb{Q}}
\newcommand{\RR}[0]{\mathbb{R}}
\newcommand{\CC}[0]{\mathbb{C}}
\newcommand{\getenv}[2][] {
  \CatchFileEdef{\temp}{"|kpsewhich --var-value #2"}{\endlinechar=-1}
  \if\relax\detokenize{#1}\relax\temp\else\let#1\temp\fi
}
\newcommand{\explain}[2] {
	\begin{flalign*}
		 && \text{#2} && \text{#1}
	\end{flalign*}
}

% headers
\getenv[\AUTHOR]{AUTHOR}
\author{\AUTHOR}
\date\today

\title{פתרון ממ''ן 13 – חשבון אינפיניטסימלי 1 (20474)}

\begin{document}
\maketitle
\section{שאלה 1}
\subsection{סעיף א'}
נגדיר $a_1 = 0$ ולכל $n$:
\[
	a_{n + 1} = \frac{1}{4(1 - a_n)}
\]
נוכיח כי הסדרה מוגדרת לכל $n$: \\*
הסדרה איננה מוגדרת במקרה בו קיים $a_n = 1$.
נוכיח באינדוקציה שמקרה זה לא מתקיים על־ידי ההוכחה כי $0 < a_n < \frac{1}{2}$ לכל $n > 1$: \\*
\textbf{בסיס האינדוקציה:} על־פי חישוב $a_2 = \frac{1}{4(1 - 0)} = \frac{1}{4}$ ולכן $0 < a_2 < \frac{1}{2}$. \\*
\textbf{מהלך האינדוקציה:} נניח כי $0 < a_n < 1$ ונראה כי $0 < a_{n + 1} < \frac{1}{2}$. \\*
מתקיים
\begin{align*}
	& 0 < a_n < \frac{1}{2} \\
	& 1 - 0 = 1 > 1 - a_n > \frac{1}{2} = 1 - \frac{1}{2} \\
	& 4 > 4(1 - a_n) > 2 \\
	& 0 < \frac{1}{4} < \frac{1}{4(1 - a_n)} < \frac{1}{2} \\
	& 0 < a_{n + 1} < \frac{1}{2} \\
\end{align*}
מהלך האינדוקציה הושלם ולכן לכל $n > 1$ מתקיים
$$
	0 < a_n < \frac{1}{2}
$$
לפיכך אנו יודעים כי לכל $n > 1$ מתקיים $a_n \ne 1$ ולכן $a_n$ מוגדר לכל $n$.

\subsection{סעיף ב'}
נוכיח כי הסדרה $(a_n)$ מתכנסת ונמצא את ערך גבולה. \\*
תחילה נוכיח באינדוקציה כי לכל $n > 1$ מתקיים $a_{n + 1} > a_n$: \\*
\textbf{בסיס האינדוקציה:}
\[
	a_3 = \frac{1}{4(1 - a_2)} = \frac{1}{4 - 1} = \frac{1}{3} > \frac{1}{4} = a_2
\]
\textbf{מהלך האינדוקציה:}
נניח כי $a_n > a_{n - 1}$ ונוכיח כי $a_{n + 1} > a_n$.
\begin{align*}
	& a_n > a_{n - 1} \\
	& 1 - a_n < 1 - a_{n - 1} \\
	& 4(1 - a_n) < 4(1 - a_{n - 1}) \\
	& \frac{1}{4(1 - a_n)} > \frac{1}{4(1 - a_{n - 1})} \\
	& a_{n + 1} > a_n
\end{align*}
התנאי מתקיים ולכן הסדרה עולה לכל $n > 1$. \\*
ידוע כי $a_n < \frac{1}{2}$ לכל $n$, וראינו כי הסדרה עולה ולכן הסדרה $(a_n)$ חסומה ועולה, ולכן לפי משפט 3.16 מתכנסת.
נגדיר $\lim_{n \to \infty} a_n = L$. לכן מתקיים
\begin{align*}
	L & = \lim_{n \to \infty} a_{n + 1} \\
	  & = \lim_{n \to \infty} \frac{1}{4(1 - a_n)}
	  & = \frac{1}{\displaystyle\lim_{n \to \infty} 4(1 - a_n)} \\
	  & = \frac{1}{4 \displaystyle\lim_{n \to \infty} (1 - a_n)}
	  & = \frac{1}{4 (1 - \displaystyle\lim_{n \to \infty} a_n)} \\
	L & = \frac{1}{4 (1 - L)}
	  & \rightarrow 4L(1 - L) = 1
\end{align*}
\begin{align*}
	& 4L - 4L^2 - 1 = 0 \\
	& 4L^2 - 4L + 1 = 0 \\
	& L = \frac{4 \pm \sqrt{4^2 - 4 \cdot 4}}{2 \cdot 4} = \frac{4 \pm 0}{8} = \frac{1}{2}
\end{align*}
לכן מתקיים
\[
	\lim_{n \to \infty} a_n = \frac{1}{2}
\]

\section{שאלה 2}
\subsection{סעיף א'}
נחשב את הגבול
\[
	\lim_{n \to \infty}
	\frac{{(-5)}^n + 2{(-2)}^n + 3}{5^{n + 1} + 2{(-3)}^n + 3}
\]
נגדיר שתי תת־סדרות $(a_{n_k})$ ו־$(a_{m_k})$
המוגדרות כסדרת האיברים הזוגיים והאי־זוגיים בהתאמה של הסדרה.
סדרות אלה מכסות את הסדרה המקורית, ומתקיים:
\[
	a_{n_k} = \frac{{5}^n + 2 \cdot 2^n + 3}{5^{n + 1} + 2 \cdot 3^n + 3}
\]
נחשב את גבולה בעזרת אריתמטיקה של גבולות:
\begin{align*}
	\lim_{n \to \infty} a_{n_k}
	& = \lim_{n \to \infty} \frac{{5}^n + 2 \cdot 2^n + 3}{5^{n + 1} + 2 \cdot 3^n + 3} \\
	& = \lim_{n \to \infty} \frac{{5}^n/5^n + 2 \cdot 2^n / 5^n + 3/5^n}{5^{n + 1}/5^n + 2 \cdot 3^n/5^n + 3/5^n} \\
	& = \lim_{n \to \infty} \frac{1 + 0 + 0}{5^1 + 0 + 0} \\
	\lim_{n \to \infty} a_{n_k} & = \frac{1}{5}
\end{align*}
באופן דומה נראה כי
\[
	a_{m_k} = \frac{-{5}^m - 2 \cdot 2^m + 3}{5^{m + 1} - 2 \cdot 3^m + 3}
\]
חישוב דומה יוביל אותנו למסקנה
\[
	\lim_{m \to \infty} (a_{m_k}) = -\frac{1}{5}
\]
ראינו כי גבולות תת־הסדרות של הסדרה מתכנסות לערכים שונים,
לכן לפי משפט 3.31 הסדרה עצמה מתבדרת,
וגבולותיה החלקיים הם $-\frac{1}{5}, \frac{1}{5}$.

\subsection{סעיף ב'}
נחשב את הגבול
\[
	\lim_{n \to \infty} \frac{{(-5)}^n + 4^{n + 1} + 3}{({-4})^n + 2{(-2)}^n + 3}
\]
נגדיר סדרות זוגיות ואי־זוגיות כבסעיף א' ונחשב את גבולן:
\[
	\lim_{n \to \infty} \frac{5^n + 4^{n + 1} + 3}{4^n + 2 \cdot 2^n + 3}
	= \lim_{n \to \infty} \frac{5^n/5^n + 4^{n + 1}/5^n + 3/5^n}{4^n/5^n + 2 \cdot 2^n/5^n + 3/5^n}
	= \lim_{n \to \infty} \frac{1}{0^+}
	= \infty
\]
גבול האברים האי־זוגיים הוא, על־פי חישוב דומה:
\[
	\lim_{m \to \infty} \frac{-5^m + 4^{m + 1} + 3}{-4^m - 2 \cdot 2^m + 3}
	= \lim_{m \to \infty} \frac{-1}{0^-}
	= \infty
\]
על־פי משפט 3.31 מתקיים:
\[
	\lim_{n \to \infty} \frac{{(-5)}^n + 4^{n + 1} + 3}{({-4})^n + 2{(-2)}^n + 3} = \infty
\]

\subsection{סעיף ג'}
נוכיח כי לא מתקיים הגבול
\[
	\lim_{n \to \infty} \left( \frac{1}{n} - 1 \right)^n
\]
נגדיר $(a_{n_k})$ סדרת האיברים הזוגיים של $(a_n)$, $(a_{m_k})$ סדרת האיברים האי־זוגיים של $(a_n)$. \\*
נשים לב כי בסדרה $(a_{n_k})$ מתקיים
\[
	\left( \frac{1}{n} - 1 \right)^n
	= (-1)^n \left( 1 - \frac{1}{n} \right)^n
	= \left( 1 - \frac{1}{n} \right)^n
	\tag{1}
\]
כמו־כן מתקיים
\[
	\left( 1 - \frac{1}{n} \right)^n
	= \left( \frac{n - 1}{n} \right)^n
	= \left( \frac{n - 1 + 1}{n - 1} \right)^{-n}
	= \left( 1 + \frac{1}{n - 1} \right)^{-n}
\]
על־פי דוגמה 3.5 ושאלה 20 סעיף א' מתקיים:
\[
	\lim_{n \to \infty} a_{n_k}
	= \lim_{n \to \infty} \left( 1 + \frac{1}{n - 1} \right)^{-n}
	= e^{-1}
\]
באופן דומה עבור $(a_{m_k})$ ו־$(1)$ מתקיים
\[
	\left( \frac{1}{n} - 1 \right)^n
	= -\left( 1 - \frac{1}{n} \right)^n
\]
ולכן על־פי אריתמטיקה של גבולות
\[
	\lim_{m \to \infty} a_{m_k} = - a_{n_k} = -e^{-1}
\]
אנו רואים כי $(a_n)$ לא מתכנסת וגבולותיה החלקיים הם $\pm e^{-1}$.

\subsection{סעיף ד'}
תהי $(a_n)$ סדרה עולה ממש של מספרים שלמים, נוכיח כי מתקיים הגבול
\[
	\lim_{n \to \infty} \left( 1 + \frac{1}{a_n} \right)^{a_n}
\]
על־פי הגדרה 3.24 ומשפט 3.25 מתקיים
\[
	\lim_{n \to \infty} \left( 1 + \frac{1}{a_n} \right)^{a_n}
	= \lim_{n \to \infty} \left( 1 + \frac{1}{n} \right)^{n}
\]
ולכן על־פי דוגמה 3.5 הגבול מתקיים וערכו הוא $e$.

\section{שאלה 3}
תהי $a_n = \langle \sqrt{n} \rangle$.
\subsection{סעיף א'}
נוכיח כי הסדרה $(a_n)$ חסומה. \\*
נגדיר $l = \sqrt{n}$, אז מתקיים $a_n = \langle l \rangle$. על־פי הגדרת החלק השברי מתקיים $0 \le l < 1$, לכן מתקיים גם $0 \le \sqrt{n} < 1$. \\*
נעלה בריבוע ונקבל $0^2 = 0 \le n < 1^2 = 1$. אנו רואים כי לכל $n \in \NN$ מתקיים $0 \le a_n < 1$ ולכן הסדרה $(a_n)$ חסומה ב־1.

\subsection{סעיף ב'}
נחשב את $\displaystyle\lowlim_{n \to \infty} a_n$. \\*
נגדיר $n_k = n^2$, נשים לב כי מתקיים $a_{n_k} = \langle \sqrt{n^2} \rangle = \langle n \rangle = 0$.
לכן גם מתקיים $\lim_{n \to \infty} a_{n_k} = 0$ ובהתאם 0 הוא גבול חלקי של $(a_n)$. \\*
בסעיף הקודם הוכחנו שלכל $n$ מתקיים $0 \le a_n$ ולכן לא יתכן שקיים גבול הקטן מ־0, ובהתאם 0 הוא הגבול החלקי הקטן ביותר של $(a_n)$ ומתקיים
\[
	\lowlim_{n \to \infty} a_n = 0
\]

\subsection{סעיף ג'}
נגדיר $A = \{ a_n \mid n \in \NN \}$. נמצא את $\inf A$. \\*
כפי שראינו בסעיף הקודם, $0 \in A$ מקיים $\forall a \in A, 0 \le a$ ולכן לפי הגדרה 3.12 $\inf A = 0$. \\*
לקבוצה כמובן יש מינימום לפי טענה 3.13, הוא $0$.

\subsection{סעיף ד'}
נוכיח כי לכל $n \in \NN$ מתקיים
\[
	\left\langle \sqrt{n^2 - 1} \right\rangle = \sqrt{n^2 - 1} - n + 1
\]
%נשים לב כי מתקיים $\langle \sqrt{n^2 - 1} \rangle = \sqrt{n^2 - 1} - \lfloor \sqrt{n^2 - 1} \rfloor$.
מתקיים
\begin{align*}
	& -n \le -1 \\
	& -2n \le -2 \\
	& -2n + 1 \le -1 \\
	& n^2 - 2n + 1 \le n^2 - 1 \le n^2 & \text{מתקיים גם} \\
	& (n - 1)^2 \le n^2 - 1 \le n^2 \\
	& n - 1 \le \sqrt{n^2 - 1} \le n \\
	& \left\lfloor \sqrt{n^2 - 1} \right\rfloor = n - 1 & \text{לפי טענה 1.64 3.} \\
	& \sqrt{n^2 - 1} - \left\lfloor \sqrt{n^2 - 1} \right\rfloor = \left\langle \sqrt{n^2 - 1} \right\rangle = \sqrt{n^2 - 1} - n + 1
\end{align*}

\subsection{סעיף ה'}
נגדיר $(b_n)$ סדרה כך שמתקיים
\[
	b_n = \sqrt{n^2 - 1} - n + 1
\]
נוכיח כי יש גבול לסדרה:
\begin{align*}
	\lim_{n \to \infty} \sqrt{n^2 - 1} - n + 1
	& = \lim_{n \to \infty}\sqrt{n + 1} \sqrt{n - 1} - \sqrt{n - 1} \sqrt{n - 1} \\
	& = \lim_{n \to \infty}\sqrt{n - 1} \left( \sqrt{n + 1} - \sqrt{n - 1} \right) \\
	& = \lim_{n \to \infty}\sqrt{n - 1} \frac{\left( \sqrt{n + 1} - \sqrt{n - 1} \right)\left( \sqrt{n + 1} + \sqrt{n - 1} \right)}
		{\left( \sqrt{n + 1} + \sqrt{n - 1} \right)} \\
	& = \lim_{n \to \infty}\sqrt{n - 1} \frac{n + 1 - n + 1}{\left( \sqrt{n + 1} + \sqrt{n - 1} \right)} \\
	& = \lim_{n \to \infty}\frac{2 \sqrt{n - 1}}{ \sqrt{n + 1} + \sqrt{n - 1} }
	= \lim_{n \to \infty}\frac{2 \frac{\sqrt{n - 1}}{\sqrt{n + 1}}}{ \frac{\sqrt{n + 1}}{\sqrt{n + 1}} + \frac{\sqrt{n - 1}}{\sqrt{n + 1}} } \\
	\lim_{n \to \infty} \sqrt{n^2 - 1} - n + 1
	& = \frac{2 \cdot 1}{1 + 1} = 1
\end{align*}

\subsection{סעיף ו'}
נוכיח כי $L = 1$ הוא גבול חלקי של $(a_n)$. \\*
נגדיר סדרת אינדקסים
\[
	n_k = \sqrt{n^2 - 1}
\]
על־פי סעיף ד' מתקיים
\[
	a_{n_k} = \left\langle \sqrt{n^2 - 1} \right\rangle = \sqrt{n^2 - 1} - n + 1
\]
ולכן על־פי סעיף ה'
\[
	\lim_{k \to \infty} a_{n_k} = \lim_{n \to \infty} \sqrt{n^2 - 1} - n + 1 = 1
\]
לכן $L = 1$ גבול חלקי של $(a_n)$.

\subsection{סעיף ז'}
נחשב את $L = \displaystyle\uplim_{n \to \infty} a_n$. \\*
נניח בשלילה כי $L > 1$, לכן קיימת תת־סדרה $a_{n_k}$ אשר מקיימת $\lim_{k \to \infty} a_{n_k} > 1$. \\*
על־פי הגדרת הגבול במצב זה לפי הגדרה 2.9 קיים $n \in \NN$ כך שמתקיים $a_n = 1$ בסתירה למסקנת סעיף א', לכן $L \le 1$.
בסעיף הקודם מצאנו כי $L = 1$ הוא גבול חלקי של הסדרה, וידוע כי $1 \ge L$ לכל גבול חלקי $L$ ולכן בהתאם הוא הגבול החלקי הגדול ביותר של $(a_n)$, דהינו
\[
	\displaystyle\uplim_{n \to \infty} a_n = 1
\]

\subsection{סעיף ח'}
נגדיר $A = \left\{ a_n \mid n \in \NN \right\}$ ונמצא את $\sup A$. \\*
אנו יודעים כי $1 \notin A$, אך על־פי הסעיף הקודם אנו יודעים כי לכל $n < 1$ בסביבה של $1$ מתקיים $n \in A$. \\*
מסיבה זו $l = 1$ הוא חסם מלעיל של $A$, ולכל $k < l$ שנבחר קיים $a \in A$ כך ש־$a > k$ \footnotemark, לכן $l$ חסם עליון, נסמן
\[
	\sup A = 1
\]
כאמור, לקבוצה $A$ אין מקסימום, על־פי טענה $(1)$.

\end{document}
