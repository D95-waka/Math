\documentclass[a4paper]{article}

% packages
\usepackage{inputenc, fontspec, amsmath, amsfonts, polyglossia, catchfile}
\usepackage[a4paper, margin=50pt, includeheadfoot]{geometry} % set page margins

% style
\AddToHook{cmd/section/before}{\clearpage}	% Add line break before section
\setdefaultlanguage{hebrew}
\setotherlanguage{english}
\setmainfont{Libertinus Serif}
\linespread{1.5}
\setcounter{secnumdepth}{0}		% Remove default number tags from sections

% custom operators
\newcommand{\getenv}[2][]{%
  \CatchFileEdef{\temp}{"|kpsewhich --var-value #2"}{\endlinechar=-1}%
  \if\relax\detokenize{#1}\relax\temp\else\let#1\temp\fi}
\getenv[\AUTHOR]{AUTHOR}
\DeclareMathOperator\cis{cis}
\DeclareMathOperator\Sp{Sp}
\DeclareMathOperator\tr{tr}
\DeclareMathOperator\im{Im}
\DeclareMathOperator\diag{diag}
\DeclareMathOperator*\lowlim{\underline{lim}}
\DeclareMathOperator*\uplim{\overline{lim}}
\def\NN{\mathbb{N}}
\def\RR{\mathbb{R}}
\def\CC{\mathbb{C}}

\title{פתרון ממ''ן 14 – חשבון אינפיניטסימלי 1 (20474)}
\author{\AUTHOR}
\date\today

\begin{document}
\maketitle
\section{שאלה 1}
יהיו $f$ ו־$g$ פונקציות מ־$\RR$ ל־$\RR$.
\subsection{סעיף א'}
נוכיח כי אם הפונקציה $f \circ g$ היא על $\RR$ אז $f$ היא על $\RR$. \\*
ההוכחה מבוססת על הוכחת משפט 3.22 בספר תורת הקבוצות מעת שמואל ברגר. \\*
נוכיח כי לכל $c \in (f \circ g)(\RR)$ קיים $b \in \RR$ כך ש־$f(b) = c$.
יהי $c \in \RR$, הפונקציה $f \circ g$ היא על, לכן קיים $a \in \RR$
כך ש־$(f \circ g)(a) = c$, כלומר $f(g(a)) = c$. $g(a) \in \RR$.
נגדיר $b = g(a)$. מצאנו $b$ עבורו $c = f(b)$.

\subsection{סעיף ב'}
נראה כי אם $f \circ g$ היא על, הפונקציה $g$ איננה בהכרח על. \\*
נגידר
\[
	f(x) = \begin{cases}
		x & x < 0 \\
		x - 1 & 0 \le x
	\end{cases}
	g(x) = \begin{cases}
		x & x < 0 \\
		x + 1 & 0 \le x
	\end{cases}
\]
תמונת הפונקציה $g(\RR)$ איננה כוללת את הקטע $(0, 1)$,
לעומת זאת הפונקציה $f \circ g$ היא על, בסתירה לטענה.

\subsection{סעיף ג'}
נוכיח שאם הפונקציה $f \circ g$ היא על ו־$f$
היא חד־חד ערכית אז $g$ היא על $\RR$. \\*
נשים לב כי מתקיים $f(g(\RR)) = \RR$ על־פי הנתון כי $f \circ g$ על.
ידוע כי $f$ היא חד־חד ערכית, ועל־פי סעיף א' גם על, ולכן הפיכה.
נגדיר $f^{-1}$ הפונקציה ההפוכה ל־$f$,
לכן $f^{-1}(f(g(\RR))) = g(\RR) = f^{-1}(\RR)$.
ידוע כי $f^{-1}$ היא על ולכן מתקיים $f^{-1}(\RR) = \RR$,
ולכן גם $g(\RR) = \RR$, אז הפונקציה $g$ היא על.

\subsection{סכיף ד'}
הטענה כי אם $f \circ g$ מונוטונית עולה אז $g$ מונוטונית איננה נכונה. \\*
נראה דוגמה נגדית, נגדיר
\[
	f(x) = x,
	g(x) = x^2
\]
הפונקציה $(f \circ g)(x) = x^3$, פונקציה מונוטונית לכל $\RR$ ועולה,
אבל $g$ היא פרבולה ובהכרח איננה מונוטונית.

\subsection{סעיף ה'}
נוכיח כי אם $f \circ g$ היא מונוטונית עולה ו־$f$ היא מונוטונית יורדת,
אז $g$ היא מונוטונית יורדת. \\*
נראה כי לכל $x_1, x_2 \in \RR, x_1 < x_2$ מתקיים:
\[
	f(g(x_1)) < f(g(x_2))
\]
אבל ידוע כי $f$ מונוטונית יורדת, לכן גם מתקיים
\[
	g(x_1) > g(x_2)
\]
דהינו הפונקציה $g$ מונוטונית יורדת.

\section{שאלה 2}
\subsection{סעיף א'}
נוכיח כי הגבול הבא מתקיים בלשון $\epsilon, \delta$
\[
	\lim_{x \to 4} \sqrt{2x^2 - 7} = 5
\]
נוכיח כי לכל $\epsilon > 0$ קיים $\delta > 0$
כך שלכל $x$ המקיים $0 < |x - 4| < \delta$
מתקיים $|\sqrt{2x^2 - 7} - 5| < \epsilon$. \\*
$x$ בסביבה של $4$ ולכן עבור ערך $\delta$ מתאים מתקיים $|x - 4| < 1$.
לפי אי־שוויון זה גם $-1 < x - 4 < 1$ ולכן $3 < x < 5$.
אנו רואים כי במקרה זה $x < 5$ ולכן $|x + 4| < |4 + 5| = 9$.
נראה כי מתקיים:
\begin{align*}
	& |x - 4| < \delta \\
	& |x - 4| |x + 4| < 9\delta \\
	& |(x - 4)(x + 4)| < 9\delta \\
	& |x^2 - 16| < 9\delta \\
	& |2x^2 - 7 - 25| < 18\delta \\
	& |\sqrt{2x^2 - 7} - 5| |\sqrt{2x^2 - 7} + 5| < 18\delta \\
\end{align*}
נשים לב כי הערך $\sqrt{2x^2 - 7} + 5$ חיובי וגדול מ־$1$ בכל תחום הגדרתו,
לכן
\begin{align*}
	& |\sqrt{2x^2 - 7} - 5| 
	< |\sqrt{2x^2 - 7} - 5| |\sqrt{2x^2 - 7} + 5| < 18\delta \\
\end{align*}
נגדיר 
\[
	\delta = \max\{1, \frac{\epsilon}{18}\}
\]
בשל הגבלת הסביבה, מתקיים
\[
	|\sqrt{2x^2 - 7} - 5| < \epsilon
\]
ולכן מתקיים הגבול $\lim_{x \to 4} \sqrt{2x^2 - 7} = 5$.

\subsection{סעיף ב'}
נוכיח כי הגבול מתקיים בלשון $\epsilon, M$:
\[
	\lim_{x \to \infty} \frac{x + 1}{\lfloor x \rfloor}
\]
הביטוי לא מוגדר כאשר $0 \le x < 1$, לכן נגדיר $M \ge 1$.
נשים לב כי אי־השוויון
\[
	\left| \frac{x-1}{\lfloor x \rfloor} - 1 \right| < \epsilon
\]
מתקיים רק כאשר
\[
	\left| \frac{x - \lfloor x \rfloor - 1}{\lfloor x \rfloor} \right|
	< \epsilon
\]
על־פי ההגדרה $M \ge 1$ תמיד מתקיים $\lfloor x \rfloor > 0$ לכן
\[
	\frac{\left| x - \lfloor x \rfloor - 1\right|}{\lfloor x \rfloor}
	< \epsilon
\]
על־פי חוקי החלק השלם
$x - 1 \le \lfloor x \rfloor < x$
לכן $x - 1 - x < \lfloor x \rfloor - x < 0$
אז $-1 < x - \lfloor x \rfloor - 1 < 0$.
מכנה אי־השוויון תמיד יהיה ערך בין $0$ ל־$1$, אז מתקיים
\[
	\frac{\left| x - \lfloor x \rfloor - 1\right|}{\lfloor x \rfloor}
	< \frac{1}{\lfloor x \rfloor} < \epsilon
\]
אילו נגדיר
\[
	M = 1 + \frac{1}{\epsilon}
\]
יתקיים
\[
	M = 1 + \frac{1}{\epsilon} < x
\]
ולכן גם
\[
	\frac{1}{x - 1} < \epsilon
\]
ועל־פי אי־שוויון החלק השלם גם
\[
	\frac{1}{\lfloor x \rfloor} < \epsilon
\]
ובמקרה זה ראינו כי גם מתקיים
\[
	\left| \frac{x-1}{\lfloor x \rfloor} - 1 \right| < \epsilon
\]
ולכן הגבול מתקיים.

\section{שאלה 3}
\subsection{סעיף א'}
\textbf{(i)}
ננסח את הטענה „לא קיים ל־$f$ גבול סופי כש־$x \to \infty$”
בלשון $\epsilon, M$, על־ידי ניסוח שלילה להגדרה 4.54:
נאמר כי לא קיים $\lim_{x \to \infty} f(x)$
אם לא קיים $L \in \RR$ עבורו $\lim_{x \to \infty} f(x) = L$,
דהינו אם לכל $L \in \RR$ קיים $\epsilon > 0$ כך שלכל $M \in \RR$
קיים $x > M$ עבורו $\left| f(x) - L\right| \ge \epsilon$. \\
\textbf{(ii)}
ננסח את הטענה על־ידי שלילת הגדרת היינה לסדרות כפי שמופיעה בהגדרה 4.54: \\*
לא קיים $\lim_{x \to \infty}f(x)$ אם ורק אם
לכל $L \in \RR$
קיימת סדרה ${(x_n)}_{n = 1}^\infty$
כך ש־$x_n \underset{n \to \infty}{\rightarrow} \infty$
ולא מתקיים $f(x_n) \underset{n \to \infty}{\rightarrow} L$

\subsection{סעיף ב'}
נוכיח כי $L = \frac{1}{2}$ איננו הגבול של $f(x) = \langle x \rangle$
כאשר $x \rightarrow \infty$ בלשון $\epsilon, M$. \\*
קודם כל נראה כי לכל $M \in \RR$ שנבחר, לכל $x > M$ כך ש־$\langle x \rangle = 0$ מתקיים
\[
	\left| \langle x \rangle - \frac{1}{2} \right| = \left| 0 - \frac{1}{2} \right| = \frac{1}{2}
\]
על־פי הגדרת הגבול בלשון $\epsilon, M$ לכל $\epsilon > 0$ קיים $M \in \RR$ כך שלכל $x > M$ מתקיים
\[
	\left| \langle x \rangle - \frac{1}{2} \right| < \epsilon
\]
אך ראינו כי מתקיים
\[
	\left| \langle x \rangle - \frac{1}{2} \right|  = \frac{1}{2} < \epsilon
\]
לכן קיימים ערכי $\epsilon > 0$ כך שלא קיים ערך $M$ כזה, ולכן גבול זה איננו מתקיים.

\subsection{סעיף ג'}
\textbf{(i)}
נוכיח כי לא קיים גבול לפונקציה $f(x) = \langle x \rangle$ כאשר $x \to \infty$ על־פי הגדרת סעיף א' (i) \\*
על־פי ההגדרה, נוכיח כי לכל $L \in \RR$ קיים $\epsilon > 0$ עבורו לכל $M \in \RR$ קיים $x > M$ עבורו
\[
	\left| f(x) - L \right| \ge \epsilon
\]
על־פי הגדרת החלק השברי תמונת $f(x)$ היא $\left[0, 1\right)$, לכן לכל $L$ כך ש־$L > 1$ נגדיר $\epsilon = |L - 1|$. % chktex 9
עבור $\epsilon$ זה לכל $M$ קיים $x > M$ כך ש־$x = \lfloor x \rfloor + 1$ ומתקיים
\[
	\left| f(x) - L \right| = \left| 0 - L \right| > \left| L - 1 \right| = \epsilon
\]
באופן דומה כאשר $L < 0$ נגדיר $\epsilon = \left| 1 - L \right|$ ויתקיים אי־השוויון. \\*
כאשר $0 \le L < 1$ נגדיר $\epsilon = L$ ונראה כי לכל $M \in \RR$ ועבור $x = 1$ מתקיים
\[
	\left| f(x) - L \right| = L \ge \epsilon
\]
ראינו כי אכן ההגדרה מתקיימת לכל $L \in \RR$ ולכן לא קיים גבול כאשר $x \to \infty$ לפונקציה $f$. \\
\textbf{(ii)}
נוכיח כי אין גבול לפונקציה בלשון היינה כפי שהוגדר בסעיף א'. \\*
עבור $L \ne 0$ נגדיר את הסדרה $x_n = \lfloor x \rfloor$.
לכל איבר בסדרה זו מתקיים $f(x) = 0$, לכן
\[
	\lim_{n \to \infty} f(x_n) = 0 \ne L
\]
נגדיר $L = 0$ ו־$x_n = \lfloor x \rfloor + \frac{1}{2}$. אז מתקיים
\[
	\lim_{n \to \infty} f(x_n) = \frac{1}{2} \ne 0
\]
ראינו כי התנאי מתקיים לכל $L \in \RR$ ולכן על־פי הגדרת היינה אין גבול לפונקציה $f$.

\section{שאלה 4}
\subsection{סעיף א'}
על־פי כלל הרכבה לכל $c \in \RR\backslash\{0\}$:
\[
	\lim_{x \to 0} \cos(c x) = \lim_{t \to 0} \cos t = 1 \tag{*}
\]
\begin{align*}
	\lim_{x \to 0} \frac{\tan 5x}{\sin 3x}
	& \overset{\text{כלל לופיטל}}{=} \lim_{x \to 0} \frac{\tan' 5x}{\sin' 3x}\\
	& = \lim_{x \to 0} \frac{\frac{1}{\cos^2 5x} \cdot 5}{\cos(3x) \cdot 3}\\
	& = \lim_{x \to 0} \frac{5}{3} \cdot \frac{1}{\cos(3x) \cos^2(5x)}\\
	& \overset{\text{חוקי גבולות}}{=} \frac{5}{3} \frac{1}{\lim_{x \to 0} \left( \cos(3x) \cos^2(5x)\right)} \\
	& \overset{\text{(*)}}{=} \frac{5}{3} \frac{1}{1}\\
	& = \frac{5}{3}
\end{align*}

\subsection{סעיף ב'}
\[
	\lim_{x \to 0} \frac{\sqrt{\cos x} - \cos x}{x^2}
	= \lim_{x \to 0} \frac{\frac{-\sin x}{2\sqrt{\cos x}} + \sin x}{2x}
	= \lim_{x \to 0} \frac{\sin x}{4x} \frac{2\sqrt{\cos x} - 1}{\sqrt{\cos x}}
\]
על־פי משפט 4.45:
\[
	\lim_{x \to 0} \frac{\sin x}{4 x} = \frac{1}{4}
\]
הביטוי $\sqrt{\cos x}$ מוגדר וחיובי בסביבה החיובית של $0$, לכן גם
\[
	\frac{2\sqrt{\cos x} - 1}{\sqrt{\cos x}} = 2 - \frac{1}{\sqrt{\cos x}}
\]
נשים לב כי מתקיים הגבול
\[
	\lim_{x \to 1} \frac{1}{x} = 1
\]
וניתן להוכיח כי מתקיים הגבול
\[
	\lim_{t \to 0} \sqrt{\cos t} = 1
\]
לכן משפט 4.39
\[
	\lim_{x \to 0} 2 - \frac{1}{\sqrt{\cos x}} = 1
\]
ולכן מתקיים
\[
	\lim_{x \to 0} \frac{\sqrt{\cos x} - \cos x}{x^2}
	= \frac{1}{4} \cdot 1 = \frac{1}{4}
\]

\subsection{סעיף ג'}
\begin{align*}
	\lim_{x \to \infty} \frac{3x^5 - 5x^4 + x \cos x}{3x^2 - 5x^3 + x\sqrt{x}}
	= \lim_{x \to \infty} \frac{\frac{3x^5}{x^5} - 5\frac{x^4}{x^5} + \frac{x \cos x}{x^5}}{3\frac{x^2}{x^5} - 5\frac{x^3}{x^5} + \frac{x\sqrt{x}}{x^5}}
	= \lim_{x \to \infty} \frac{3 - 5\frac{1}{x} + \frac{\cos x}{x^4}}{3\frac{1}{x^3} - 5\frac{1}{x^2} + \frac{\sqrt{x}}{x^4}}
	= \lim_{x \to \infty} \frac{3 + \frac{\cos x}{x^4}}{\frac{\sqrt{x}}{x^4}}
\end{align*}
נשים לב כי מתקיים
\[
	\frac{-1}{x^4} \le \frac{\cos x}{x^4} \le \frac{1}{x^4}
\]
לכל $x \ge 0$, לכן לפי משפט 4.43 מתקיים
\[
	\lim_{x \to \infty} \frac{-1}{x^4} = \lim_{x \to \infty} \frac{\cos x}{x^4} = \lim_{x \to \infty} \frac{1}{x^4} = 0
\]
לכן
\[
	\lim_{x \to \infty} \frac{3 + \frac{\cos x}{x^4}}{\frac{\sqrt{x}}{x^4}}
	= \lim_{x \to \infty} \frac{3 + 0}{\frac{\sqrt{x}}{x^4}}
	= \lim_{x \to \infty} \frac{3x^4}{\sqrt{x}}
\]
קל לראות על־פי הגדרה 4.55
\[
	\lim_{x \to \infty} \frac{3x^4}{\sqrt{x}} = \infty
\]

\subsection{סעיף ד'}
\[
	\lim_{x \to 0} \frac{x + \sin x}{x^2 + \sin^2 x}
	= \lim_{x \to 0} \frac{\frac{x}{x^2} + \frac{\sin x}{x^2}}{\frac{x^2}{x^2} + \frac{\sin^2 x}{x^2}}
	= \lim_{x \to 0} \frac{\frac{1}{x} + \frac{\sin x}{x}\frac{1}{x}}{{(\frac{\sin x}{x})}^2 + 1}
\]
על־פי משפט 4.45:
\[
	\lim_{x \to 0} \frac{\frac{1}{x} + \frac{\sin x}{x}\frac{1}{x}}{{(\frac{\sin x}{x})}^2 + 1}
	= \lim_{x \to 0} \frac{0 + 1 \cdot \frac{1}{x}}
	{1^2 + 1}
	= \lim_{x \to 0} \frac{0 + 1 \cdot 0}{2}
	= 0
\]

\subsection{סעיף ה'}
\[
	\lim_{x \to x_0} \lfloor \tan x \rfloor \cos x, x_0 = 0, \frac{\pi}{2}
\]
על־פי הגדרת $\tan x$ מתקיים כאשר $0 \le x < \frac{\sqrt 2}{2}$:
\[
	\tan x = \frac{\sin x}{\cos x} < 1
\]
ולכן עבור $\delta < 1$ מתקיים
\[
	\lfloor \tan x \rfloor = 0
\]
ובהתאם
\[
	\lim_{x \to 0} \lfloor \tan x \rfloor \cos x
	= \lim_{x \to 0} 0 \cos x
	= \lim_{x \to 0} 0
	= 0
\]
נמצא את הגבול
\[
	\lim_{x \to \frac{\pi}{2}} \lfloor \tan x \rfloor \cos x
\]
נשים לב כי כאשר $\frac{\sqrt{2}}{2} \le x < \frac{\pi}{2}$:
\[
	\tan x - 1 \le \lfloor \tan x \rfloor \le \tan x 
\]
ולאחר הכפלה
\[
	(\tan x - 1) \cos x \le \lfloor \tan x \rfloor \cos x \le \tan x \cos x
\]
בתחום $(\frac{\sqrt{2}}{2}, \frac{\pi}{2})$ נוכל לצמצם את הביטוי $\tan x \cos x$ ל־$\sin x$ בלא שינוי לערך, לכן:
\[
	\sin x - \cos x \le \lfloor \tan x \rfloor \cos x \le \sin x
\]
לכן לפי משפט 4.43
\[
	\lim_{x \to \frac{\pi}{2}} \sin x - \cos x = \lim_{x \to \frac{\pi}{2}} \lfloor \tan x \rfloor \cos x = \lim_{x \to \frac{\pi}{2}} \sin x = 1
\]

\end{document} % chktex 17
