\documentclass[a4paper]{article}

% packages
\usepackage{inputenc, fontspec, amsmath, amsthm, amsfonts, polyglossia, catchfile}
\usepackage[a4paper, margin=50pt, includeheadfoot]{geometry} % set page margins

% style
\AddToHook{cmd/section/before}{\clearpage}	% Add line break before section
\linespread{1.5}
\setcounter{secnumdepth}{0}		% Remove default number tags from sections
\setmainfont{Libertinus Serif}
\setsansfont{Libertinus Sans}
\setmonofont{Libertinus Mono}
\setdefaultlanguage{hebrew}
\setotherlanguage{english}

% operators
\DeclareMathOperator\cis{cis}
\DeclareMathOperator\Sp{Sp}
\DeclareMathOperator\tr{tr}
\DeclareMathOperator\im{Im}
\DeclareMathOperator\diag{diag}
\DeclareMathOperator*\lowlim{\underline{lim}}
\DeclareMathOperator*\uplim{\overline{lim}}

% commands
\renewcommand\qedsymbol{\textbf{משל}}
\newcommand{\NN}[0]{\mathbb{N}}
\newcommand{\ZZ}[0]{\mathbb{Z}}
\newcommand{\QQ}[0]{\mathbb{Q}}
\newcommand{\RR}[0]{\mathbb{R}}
\newcommand{\CC}[0]{\mathbb{C}}
\newcommand{\getenv}[2][] {
  \CatchFileEdef{\temp}{"|kpsewhich --var-value #2"}{\endlinechar=-1}
  \if\relax\detokenize{#1}\relax\temp\else\let#1\temp\fi
}
\newcommand{\explain}[2] {
	\begin{flalign*}
		 && \text{#2} && \text{#1}
	\end{flalign*}
}

% headers
\getenv[\AUTHOR]{AUTHOR}
\author{\AUTHOR}
\date\today

\title{פתרון ממ''ן 14 – חשבון אינפיניטסימלי 1 (20474)}

\begin{document}
\maketitle
\section{שאלה 1}
יהיו $f, g: \RR \to \RR$ פונקציות אשר מקיימות $(f \circ g)(x) = x$\footnotemark{} לכל $x \in \RR$.
\subsection{סעיף א'}
נפריך את הטענה כי $f$ חד־חד ערכית על־ידי דוגמה נגדית. נגדיר
\[
	f(x) = \begin{cases}
		x - 1 & x \ge 1 \\
		1 & 0 \le x < 1 \\
		x & x < 0
	\end{cases},
	g(x) = \begin{cases}
		x + 1 & x \ge 0 \\
		x & x < 0
	\end{cases}
\]
במצב זה אנו רואים כי מתקיים $(1)$, אך $f(2) = f(0)$ בסתירה לטענת החד־חד ערכיות.

\subsection{סעיף ב'}
נוכיח כי $g$ היא חד־חד ערכית. \\*
יהיו $x, y \in \RR$ כך שמתקיים $g(x) = g(y)$.
על שוויון זה נבצע את הפונקציה $f$:
\[
	f(g(x)) = f(g(y))
	\rightarrow
	(f \circ g)(x) = (f \circ g)(y)
\]
אז גם מתקיים $x = y$ אם ורק אם $g(x) = g(y)$, ולכן על־פי הגדרה 4.5 $g$ היא חד־חד ערכית.

\subsection{סעיף ג'}
נוכיח כי $f$ היא על. \\*
יהי $x \in \RR$ מספר, בהתאם ל־$(1)$ הוא מקיים $(f \circ g)(x) = x$.
נגדיר $x' = g(x)$, אז כמובן $f(x') = x$.
נראה כי מצאנו לכל $x$ מספר $x'$ כך ש־$f(x') = x$ ולכן $\im f = \RR$, דהינו, $f$ היא על.

\subsection{סעיף ד'}
נפריך את הטענה כי $g$ היא על על־ידי דוגמה נגדית. \\*
למעשה, הפונקציה $g$ אשר הוגדרה בסעיף א' מקיימת את $(1)$ ואיננה על.

\subsection{סעיף ה'}
נפריך את הטענה כי לכל $x \in \RR$ מתקיים $(g \circ f)(x) = x$ על־ידי דוגמה נגדית. \\*
נגדיר את הפונקציות $f, g$ כשם שהוגדרו בסעיף א', אז $(g \circ f)(0) = 2 \ne 0$ בסתירה לטענה.

\subsection{סעיף ו'}
נוכיח כי אם $g$ היא על, אז $(g \circ f)(x) = x$ לכל $x \in \RR$. \\*
אנו יודעים כי $g$ על ולכן קיים $y \in \RR$ כך ש־$g(x) = y$.
ידוע כי $f(g(x)) = f(y) = x$, נפעיל את $g$ על השוויון האחרון: $g(f(y)) = g(x) = y$,
דהינו $(g \circ f)(x) = x$.

\section{שאלה 2}
\subsection{סעיף א'}
נוכיח כי הגבול הבא מתקיים בלשון $\epsilon, \delta$
\[
	\lim_{x \to \frac{2}{\pi}} \left\lfloor \sin \frac{1}{x} \right\rfloor = 0
\]
נשים לב כי $0 < \sin \frac{1}{x} < 1$ עבור הערכים $\frac{2}{\pi} - \frac{1}{\pi} < x < \frac{2}{\pi}$ על־פי הגדרת $\sin$ וחישוב ישיר.
לכן בתחום זה
\[
	\left\lfloor \sin \frac{1}{x} \right\rfloor = 0
\]
יהי $\epsilon > 0$ ונגדיר $\delta = \min\{ \frac{1}{\pi}, \epsilon \}$.
אז לכל $x$ המקיים $0 < | x - \frac{2}{\pi} | < \delta$ מתקיים
\[
	\left| \left\lfloor \sin \frac{1}{x} \right\rfloor - 0 \right| = 0 < \epsilon
\]
לכן הגבול מתקיים.

\subsection{סעיף ב'}
נוכיח כי הגבול מתקיים בלשון $M_1, M_2$:
\[
	\lim_{x \to \infty} \sqrt{2x - \sin 3x} = \infty
\]
נגדיר $f(x) = \sqrt{2x - \sin 3x} = \infty$. יהי $M_1 \in \RR$. \\*
נגדיר $M_2 = \min\{ 0, M_1^2 + 20 \}$.
נשים לב כי עבור תחום ההגדרה של $f$, היא תמיד חיובית, שכן פעולת השורש לא מחזירה מספרים שליליים בממשיים.
כאשר $M_1 < 0$ לכל $x > M_2$ הפונקציה $f(x) \ge 0 > M_1$.
עוד נראה כי עבור כל $M_1 \ge 0$ כל $x > M_2 = M_1^2 + 20$ מקיים $f(x) > M_1$,
שכן תמונת $\sin$ היא $[-1, 1]$ ולכן $2x - \sin 3x \ge 2x - 1 \ge 2M_1^2 + 19$.
בהתאם $\sqrt{2M_1^2 + 19} > M_1$ וההוכחה נשלמה.

\section{שאלה 3}
\subsection{סעיף א'}
\textbf{(i)}
ננסח את הטענה „לא קיים ל־$f$ גבול סופי כש־$x \to \infty$”
בלשון $\epsilon, M$, על־ידי ניסוח שלילה להגדרה 4.54:
נאמר כי לא קיים $\lim_{x \to \infty} f(x)$
אם לא קיים $L \in \RR$ עבורו $\lim_{x \to \infty} f(x) = L$,
דהינו אם לכל $L \in \RR$ קיים $\epsilon > 0$ כך שלכל $M \in \RR$
קיים $x > M$ עבורו $\left| f(x) - L\right| \ge \epsilon$. \\
\textbf{(ii)}
ננסח את הטענה על־ידי שלילת הגדרת היינה לסדרות כפי שמופיעה בהגדרה 4.54: \\*
לא קיים $\lim_{x \to \infty}f(x)$ אם ורק אם
לכל $L \in \RR$
קיימת סדרה ${(x_n)}_{n = 1}^\infty$
כך ש־$x_n \underset{n \to \infty}{\rightarrow} \infty$
ולא מתקיים $f(x_n) \underset{n \to \infty}{\rightarrow} L$

\subsection{סעיף ב'}
תהי הפונקציה $f: \RR \to \RR$ המוגדרת
\[
	f(x) = \frac{4}{5 + \cos x}
\]
\textbf{(i)}
נוכיח כי ל־$f$ אין גבול סופי כאשר $x \to \infty$ על־פי הגדרת סעיף א' (i).
\begin{proof}
	יהי $L \in \RR$.
	תמונת $\cos$ היא $[-1, 1]$, ולכן בהתאם תמונת $f$ היא $[\frac{2}{3}, 1]$.
	אנו יודעים כי $\cos$ פונקציה מחזורית ולכן גם $f$ בהתאם ולכן לכל $M \in \RR$ שנבחר קיימים ערכי $x > M$ שמגיעים לכל חלקי תמונת $f$. \\*
	נוכל למצוא $x > M$ לכל $M \in \RR$ אשר עבורו $|f(x) - L|$ הוא מספר שונה מאפס, נגדיר $\epsilon = |f(x) - L|/2$. \\*
	אז לפי הגדרת סעיף א' (i) לפונקציה $f$ אין גבול סופי כאשר $x \to \infty$. \\*
\end{proof}
\textbf{(ii)}
נוכיח כי ל־$f$ אין גבול סופי כאשר $x \to \infty$ על־פי הגדרת סעיף א' (ii).
\begin{proof}
	יהי $L \in \RR$. כפי שראינו בתת־סעיף הקודם, קיים $x$ כך ש־$f(x) \ne L$. \\*
	נגדיר $x_n = x + 2 \pi n$, לכן $f(x_n)$ ערך קבוע ושונה מ־$L$.
	אז לפי הגדרת סעיף א' (ii) הפונקציה $f$ לא מתכנסת לערך סופי כאשר $x \to \infty$.
\end{proof}

\section{שאלה 4}
\subsection{סעיף א'}
\[
	\lim_{x \to 0} \frac{1 - \cos x}{x^2}
	= \lim_{x \to 0} \frac{1}{x^2} - \frac{\cos x}{x^2}
	= 0 + 0 = 0
\]

\subsection{סעיף ב'}
\[
	\lim_{x \to 0} \frac{\sin^4 x}{x^7}
	= \lim_{x \to 0} \frac{1}{x^3} {\left( \frac{\sin x}{x} \right)}^4
	= \infty \cdot 1^4
	= \infty
\]

\subsection{סעיף ג'}
\[
	\lim_{x \to \infty} \frac{-3x^5 + 5x^3 + 1}{5x^5 + 3x^3 - 1}
	= \lim_{x \to \infty} \frac{-3x^5/x^5 + 5x^3/x^5 + 1/x^5}{5x^5/x^5 + 3x^3/x^5 - 1/x^5}
	= \lim_{x \to \infty} \frac{-3 + 5/x^2 + 1/x^5}{5 + 3/x^2 - 1/x^5}
	= \frac{-3}{5}
\]

\subsection{סעיף ד'}
\[
	\lim_{x \to -\infty} \sqrt{x^2 - \sin x} + x
\]
נשים לב כי $x^2 = {(-x)}^2, \sin x = - \sin x$, ואילו את $+x$ נוכל לציין כ־$-x$, לכן
\[
	\lim_{x \to -\infty} \sqrt{x^2 - \sin x} + x
	= \lim_{x \to \infty} \sqrt{x^2 - \sin x} - x
\]
מתקיים
\[
	\sqrt{x^2 - \sin x} - x
	= \frac{\left(\sqrt{x^2 - \sin x} + x\right) \left(\sqrt{x^2 - \sin x} - x\right)}{\sqrt{x^2 - \sin x} + x}
	= \frac{ \sqrt{x^2 - \sin x}^2 - x^2}{\sqrt{x^2 - \sin x} + x}
	= \frac{ -\sin x}{\sqrt{x^2 - \sin x} + x}
\]
נשים לב כי מתקיים לכל $x \ge 1$ על־פי תחומי $\sin x$
\[
	0 \le x^2 - \sin x \le x^2 + 1 \le 4x^2
	\rightarrow
	0 \le \sqrt{x^2 - \sin x} \le \sqrt{4x^2}
	\rightarrow
	x \le \sqrt{x^2 - \sin x} + x \le 2x + x
\]
לכן על־פי הביטוי האלגברי ואי־השוויון שמצאנו, ועל־פי חוקי אי־שוויונות.
\[
	\frac{ -\sin x}{x}
	\le
	\frac{ -\sin x}{\sqrt{x^2 - \sin x} + x}
	\le
	\frac{ -\sin x}{3x}
\]
על־פי כלל הסנדויץ'
\[
	\lim_{x \to \infty} \frac{ -\sin x}{x}
	= \lim_{x \to \infty} \frac{ -\sin x}{\sqrt{x^2 - \sin x} + x}
	= \lim_{x \to \infty} \frac{ -\sin x}{3x}
	= 0
\]
לכן מתקיים הגבול
\[
	\lim_{x \to -\infty} \sqrt{x^2 - \sin x} + x = 0
\]

\subsection{סעיף ה'}
\[
	\lim_{x \to x_0} \sin \left( \frac{x}{2} \right) \left\lfloor \sin x \right\rfloor
\]
כאשר $x_0 = 0, \frac{\pi}{2}, \pi$. \\*
נשים לב כי בשל הגדרת $\sin$:
\begin{align*}
	-&\pi \le x < 0 & \rightarrow && & \lfloor \sin x \rfloor = -1 \\
	& 0 < x < \pi & \rightarrow && & \lfloor \sin x \rfloor = 0 \\
	& \pi < x < 2\pi & \rightarrow && & \lfloor \sin x \rfloor = -1
\end{align*}
כאשר $x_0 = \frac{\pi}{2}$ אנו רואים כי
\[
	\lim_{x \to \frac{\pi}{2}} \sin \left( \frac{x}{2} \right) \left\lfloor \sin x \right\rfloor = 0
\]
כאשר $x_0 = 0$ אנו רואים כי עבור סביבת $x_0$ נגדיר פונקציה חדשה:
\[
	h(x) = \begin{cases}
		-\sin{\frac{x}{2}} & x < 0 \\
		0 & x \ge 0
	\end{cases}
\]
ניתן לראות כי בסביבת $x_0$ ערך הפונקציה $h$ שווה לערך הפונקציה המקורית, ובהתאם הגבול בנקודה זהה (אם מתקיים).
על־פי גבול פונקציית $\sin$ ופונקציות קבועות מתקיים
\[
	\lim_{x \to 0} h(x) = 0
\]
ולכן מתקיים גם
\[
	\lim_{x \to 0} \sin \left( \frac{x}{2} \right) \left\lfloor \sin x \right\rfloor = 0
\]
נתבונן עתה במקרה $x_0 = \pi$. \\*
גם במקרה זה נגדיר פונקציה:
\[
	g(x) = \begin{cases}
		0 & x < \pi \\
		-\sin{\frac{x}{2}} & x \ge \pi
	\end{cases}
\]
על־פי ערכי הפונקציה המקורית בסביבת $x_0 = \pi$ הפונקציה $g$ שווה בסביבה לפונקציה המקורית, ולכן הן מתכנסות יחדיו.
נשים לב כי $g$ מורכבת משתי פונקציות אשר מתכנסות ל־$0$ ול־$-1$, ולכן לפי הגדרת התכנסות הפונקציה השלמה $g$ איננה מתכנסת, ובהתאם לא קיים גבול
\[
	\lim_{x \to \pi} \sin \left( \frac{x}{2} \right) \left\lfloor \sin x \right\rfloor
\]

\end{document} % chktex 17
