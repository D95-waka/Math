\documentclass[a4paper]{article}

% packages
\usepackage{inputenc, amsmath, amsthm, thmtools, amsfonts, amssymb, luacode, catchfile, tikzducks, hyperref}
\usepackage[a4paper, margin=50pt, includeheadfoot]{geometry} % set page margins
\usepackage[shortlabels]{enumitem}
\usepackage[skip=3pt, indent=0pt]{parskip}

% language
\usepackage[bidi=basic, layout=tabular, provide=*]{babel}
\babelprovide[main, import]{hebrew}
\babelprovide{rl}
\babelfont{rm}{Libertinus Serif}
\babelfont{sf}{Libertinus Sans}
\babelfont{tt}{Libertinus Mono}

% style
\AddToHook{cmd/section/before}{\clearpage}	% Add line break before section
\linespread{1.3}
\setcounter{secnumdepth}{0}		% Remove default number tags from sections, this won't do well with theorems
\AtBeginDocument{\setlength{\belowdisplayskip}{3pt}}
\AtBeginDocument{\setlength{\abovedisplayskip}{3pt}}

% operators
\DeclareMathOperator\cis{cis}
\DeclareMathOperator\Sp{Sp}
\DeclareMathOperator\tr{tr}
\DeclareMathOperator\im{Im}
\DeclareMathOperator\re{Re}
\DeclareMathOperator\diag{diag}
\DeclareMathOperator*\lowlim{\underline{lim}}
\DeclareMathOperator*\uplim{\overline{lim}}
\DeclareMathOperator\rng{rng}
\DeclareMathOperator\Sym{Sym}
\DeclareMathOperator\Arg{Arg}
\DeclareMathOperator\Log{Log}
\DeclareMathOperator\dom{dom}

% commands
%\renewcommand\qedsymbol{\textbf{מש''ל}}
%\renewcommand\qedsymbol{\fbox{\emoji{lizard}}}
\newcommand{\NN}[0]{\mathbb{N}}
\newcommand{\ZZ}[0]{\mathbb{Z}}
\newcommand{\QQ}[0]{\mathbb{Q}}
\newcommand{\RR}[0]{\mathbb{R}}
\newcommand{\CC}[0]{\mathbb{C}}
\newcommand{\FF}[0]{\mathbb{F}}
\newcommand{\PP}[0]{\mathbb{P}}
\newcommand{\TT}[0]{\mathbb{T}}
\newcommand{\acts}[0]{\circlearrowright}
\newcommand{\explain}[2] {
	\begin{flalign*}
		 && \text{#2} && \text{#1}
	\end{flalign*}
}
\newcommand{\maketitleprint}[0]{ \begin{center}
	\begin{tikzpicture}[scale=3]
		\duck[graduate=gray!20!black, tassel=red!70!black]
	\end{tikzpicture}	
\end{center}
}

% theorem commands
\newtheoremstyle{c_remark}
	{}	% Space above
	{}	% Space below
	{}% Body font
	{}	% Indent amount
	{\bfseries}	% Theorem head font
	{}	% Punctuation after theorem head
	{.5em}	% Space after theorem head
	{\thmname{#1}\thmnumber{ #2}\thmnote{ \normalfont{\text{(#3)}}}}	% head content
\newtheoremstyle{c_definition}
	{3pt}	% Space above
	{3pt}	% Space below
	{}% Body font
	{}	% Indent amount
	{\bfseries}	% Theorem head font
	{}	% Punctuation after theorem head
	{.5em}	% Space after theorem head
	{\thmname{#1}\thmnumber{ #2}\thmnote{ \normalfont{\text{(#3)}}}}	% head content
\newtheoremstyle{c_plain}
	{3pt}	% Space above
	{3pt}	% Space below
	{\itshape}% Body font
	{}	% Indent amount
	{\bfseries}	% Theorem head font
	{}	% Punctuation after theorem head
	{.5em}	% Space after theorem head
	{\thmname{#1}\thmnumber{ #2}\thmnote{ \text{(#3)}}}	% head content

\theoremstyle{c_plain}
\newtheorem{theorem}{משפט}[section]
\newtheorem{lemma}[theorem]{למה}
\newtheorem{proposition}[theorem]{טענה}
\newtheorem*{proposition*}{טענה}
%\newtheorem{corollary}[theorem]{אין חלופה עברית}

\theoremstyle{c_definition}
\newtheorem{definition}[theorem]{הגדרה}
\newtheorem*{definition*}{הגדרה}
\newtheorem{example}{דוגמה}[section]
\newtheorem{exercise}{תרגיל}[section]

\theoremstyle{c_remark}
\newtheorem*{remark}{הערה}
\newtheorem*{solution}{פתרון}
\newtheorem{conclusion}[theorem]{מסקנה}
\newtheorem{notation}[theorem]{סימון}

% Questions related commands
\newcounter{question}
\setcounter{question}{1}
\newcounter{sub_question}
\setcounter{sub_question}{1}

\newcommand{\question}[1][0]{
	\ifthenelse{#1 = 0}{}{\setcounter{question}{#1}}
	\subsection{שאלה \arabic{question}}
	\addtocounter{question}{1}
	\setcounter{sub_question}{1}
}

\newcommand{\subquestion}[1][0]{
	\ifthenelse{#1 = 0}{}{\setcounter{sub_question}{#1}}
	\subsubsection{סעיף \localecounter{letters.gershayim}{sub_question}}
	\addtocounter{sub_question}{1}
}

% import lua and start of document
\directlua{common = require ('../common')}

\GetEnv{AUTHOR}

% headers
\author{\AUTHOR}
\date\today

\title{פתרון ממ''ן 15 – חשבון אינפיניטסימלי 1 (20474)}

\begin{document}
\maketitle
\section{שאלה 1}
נמצא את נקודות הרציפות והאי־רציפות של הפונקציה $f$ המוגדרת:
\[
	f(x) = \lfloor x \rfloor \tan \frac{\pi x}{2}
\]
בתחום $\RR$ ונמיינן. \\*
על־פי משפט 5.13 הפונקציה $\tan \frac{\pi x}{2}$ רציפה בכל תחום הגדרתה, ועל־פי הגדרת הפונקציה אנו יודעים כי היא איננה מוגדרת בערכים
\[
	\left\{ 1 + 2k \mid k \in \ZZ \right\}
\]
מהגדרת החלק השלם והפוקנציה $x$ אנו יודעים כי $\lfloor x \rfloor$ רציפה בכל תחום הגדרתה. \\*
על־פי משפט 5.11 גם $f$ רציפה בכל תחום הגדרתה, והיא כמובן לא מוגדרת ב־$\{ 2x + 1 \mid x \in \ZZ\}$. \\*
יהי $k \in \{ 2x + 1 \mid x \in \ZZ \}$, אנו יודעים כי $f(k)$ איננה מוגדרת, וכי $\lim_{x \to k^\pm} f(x) = \pm \infty$. \\*
לכן בנקודות אלה ל־$f$ נקודות אי־רציפות ממין שני, והפונקציה $f$ רציפה בכל נקודה אחרת.

\section{שאלה 2}
\subsection{סעיף א'}
תהי $f$ פונקציה המוגדרת בסביבת $x_0$. \\*
\textbf{(i)}
ננסח את הטענה כי $f$ איננה רציפה ב־$x_0$ בלשון $\epsilon, \delta$: \\*
הפונקציה $f$ לא רציפה ב־$x_0$ אם ורק אם קיים $\epsilon > 0$ כך שלכל $\delta > 0$ קיים $x$ כך ש־$|x - x_0| < \delta$ וגם $|f(x) - f(x_0)| \ge \epsilon$. \\
\textbf{(ii)}
ננסח את הטענה כי $f$ איננה רציפה ב־$x_0$ בלשון סדרות: \\*
הפונקציה $f$ איננה רציפה ב־$x_0$ אם ורק אם קיימת סדרה ${(x_n)}_{n = 1}^\infty$ המקיימת $x \underset{n \to \infty}{\rightarrow} x_0$
כך שלא מתקיים $f(x_n) \underset{n \to \infty}{\rightarrow} f(x_0)$.

\subsection{סעיף ב'}
נגדיר $g$ פונקציה הרציפה ב־$x_0$ ופונקציה $f$ המוגדרת $f(x) = g(x) D(x)$. \\*
נוכיח כי אם $g(x_0) = 0$ אז $f$ רציפה ב־$x_0$.
\begin{proof}
	מטענה 5.3 נובע כי לכל $\epsilon > 0$ קיים $\delta > 0$ כך שאם $|x - x_0| < \delta$ אז $|g(x)| < \epsilon$. \\*
	על־פי הגדרת פונקציית דיריכלה אנו יודעים כי $D(x) \in \{0, 1\}$ לכל $x \in \RR$,
	לכן תמיד מתקיים $D(x) \le 1$. אז כמובן שמתקיים גם $|g(x)| D(x) \le |g(x)| < \epsilon$ ולכן גם $f(x)$ רציפה ב־$x_0$.
\end{proof}

\subsection{סעיף ג'}
נגדיר $x_0 \in \RR$ אשר עבורו $g(x_0) \ne 0$, כמובן ש־$g$ רציפה ב־$x_0$.

\textbf{(i)}
נוכיח כי הפונקציה $f$ איננה רציפה ב־$x_0$ על־פי ההגדרה מסעיף א' (i)
\begin{proof}
	נמצא $\epsilon > 0$ כך שלכל $\delta > 0$ קיים ערך $x$ כך ש־$|x - x_0| < \delta$ וגם $|f(x) - f(x_0)| \ge \epsilon$. \\*
	נקבע $\epsilon < 1$, ויהי $\delta > 0$ אשר עבורו $|x - x_0| < \delta$. \\*
	מהגדרת פונקציית דיריכלה ואקסיומת הרצף, אנו יכולים להסיק כי קיים מספר $x_1$ אשר מקיים $|x_0 - x_1| < \delta$, ואשר הוא אי־רציונלי.
\end{proof}

\textbf{(ii)}
נוכיח כי הפונקציה $f$ איננה רציפה ב־$x_0$ על־פי ההגדרה מסעיף א' (ii)
\begin{proof}
	נמצא סדרה ${(x_n)}_{n = 1}^\infty$ המקיימת $x \underset{n \to \infty}{\rightarrow} x_0$ כך שלא מתקיים $f(x_n) \underset{n \to \infty}{\rightarrow} f(x_0)$. \\*
	נגדיר $(x_n)$ סדרה אינסופית של מספרים אי־רציונליים, $x_1 < x_0$, ולכל $n \in \NN$ כאשר $1 < n$ גם
	\[
		x_n < x_{n + 1} < x_0
	\]
	הגדרה זו אפשרית כמובן על־פי צפיפות הממשיים. \\*
	מהגדרת הגבול עבור סדרות נובע כי
	\[
		\lim_{n \to \infty} x_n = x_0
	\]
	אבל אנו יודעים שלכל $n$ גם $f(x_n) = 0$ מהגדרת פונקציית דיריכלה, לכן
	\[
		\lim_{n \to \infty} f(x_n) = 0 \ne x_0
	\]
\end{proof}

\textbf{(iii)}
נניח בשלילה כי $f$ רציפה ב־$x_0$ ולכן על־פי משפט 5.11
\[
	\lim_{x \to x_0} f(x)
	= \lim_{x \to x_0} g(x) \lim_{x \to x_0} D(x)
	\rightarrow \lim_{x \to x_0} D(x) \ne 0
\]
אך ממשפט 5.10 נובע כי $D(x)$ אינה רציפה ב־$x_0$ בסתירה לטענה, לכן $f$ אינה רציפה ב־$x_0$.

\section{שאלה 3}
תהי $f$ פונקציה רציפה בקטע $[0, \infty)$. \\* % chktex 9
נוכיח כי אם לכל $x > 0$ מתקיים $|f(x)| > x$, אז $\lim_{x \to \infty} f(x) = \infty$ או $\lim_{x \to \infty} f(x) = -\infty$.
\begin{proof}
% .לובגה תרדגה זאו תילילש קר וא תיבויח קר וא איהש הז םע והשמ	
	תחילה נראה כי הפונקציה $f(x)$ היא חיובית לכל $x$ או שלילית לכל $x$ בתחום. \\*
	נניח בשלילה כי $f(x)$ משנה סימן כאשר $x = x_0$.
	נניח כי קיים $a$ כך ש־$a < x_0$ וגם $f(a) < -a$ וכי קיים $b > x_0$ כך ש־$f(b) > b$. \\*
	ממשפט ערך הביניים של קושי נובע כי קיים מספר $c$ ככה ש־$f(c) = 0$ בניגוד לנתון כי $|f(x)| > x > 0$. \\*
	יכולנו להגדיר את שינוי הסימן ההפוך וההוכחה הייתה נשארת זהה, לכן לא נפגעת הגבלת הכלליות. \\*
	אז אנו יכולים להסיק כי לכל $x$ מתקיים $f(x) > x$, או לחילופין לכל $x$ מתקיים $-f(x) > x$. \\*
	מהגדרת השאיפה לאינסוף ומינוס אינסוף בפונקיות נובע ישירות כי
	\[
		\lim_{x \to \infty} f(x) = \infty
	\]
	או
	\[
		\lim_{x \to \infty} f(x) = -\infty
	\]
\end{proof}

\section{שאלה 4}
תהי $f$ פונקציה רציפה בקטע $\RR_{0\ge} = [0, \infty)$ ויהי $L \in \RR$. ידוע כי מתקיים % chktex 9
\[
	\lim_{x \to \infty} f(x) = L
\]

\subsection{סעיף א'}
נוכיח כי אם $f$ מקבלת מינימום ב־$\RR_{0\ge}$ אז קיים $x_0 \ge 0$ כך ש־$f(x_0) \le L$.
\begin{proof}
	נקבע כי $x_1$ היא נקודת מינימום של $f(x)$ וכי $f(x_1) = c$. \\*
	נבחן שני מקרים, כאשר $c \le L$ אז כמובן שקיים $x_0$ כזה, והוא כאשר $x_0 = x_1$. \\*
	לכן עלינו רק לבחון את המקרה בו $c > L$. \\*
	מהגדרת הגבול בלשון $\epsilon, M$ נובע כי עבור $\epsilon = |c - L| = c - L$ קיים $M$ כך ש־$|f(x) - L| < c - L$ לכל $x > M$. \\*
	אילו היה מתקיים $f(x) < L < c$ והגענו לסתירה לטענה כי $x_1$ נקודת מינימום.
	\[
		f(x) - L < c - L \rightarrow f(x) < c
	\]
	הגענו לסתירה לטענה כי $x_1$ היא נקודת מינימום של הפונקציה $f(x)$ ולכן לא יתכן כי $c > L$ ובהתאם יתכן רק כי $c \le L$.
\end{proof}

\subsection{סעיף ב'}
נוכיח כי אם קיים $x_0 \ge 0$ כך ש־$f(x_0) < L$, אז $f$ מקבלת מינימום ב־$\RR_{0\ge}$.
\begin{proof}
	בשל הגבול של $f(x)$ באינסוף אנו יכולים להסיק כי קיים $M \in \RR_{0\ge}$ עבורו לכל $x > M$ מתקיים $f(x_0) < x$. \\* % Do I need to elaborate?
	מהמשפט השני של ויירשטראס על הקטע $[0, M]$ נובע כי בקטע זה ישנה נקודת מינימום $x_1$. \\*
	כמובן שמתקיים $f(x_0) \ge f(x_1)$, ולכן בהתאם לכל $x > M$ נסיק כי $f(x_1) < x$, ולכן $x_1$ נקודת מינימום בכל הקטע $\RR_{0\ge}$.
\end{proof}

\subsection{סעיף ג'}
נוכיח כי אם קיים $x_0 \ge 0$ כך ש־$f(x_0) = L$, אז $f$ מקבלת מינימום ב־$\RR_{0\ge}$.
\begin{proof}
	אילו $f$ פונקציה קבועה אז כמובן שהיא מקבלת מינימום, ולכן נניח כי $f$ פונקציה רציפה שאיננה קבועה. \\*
	אילו קיימת נקודה $x_1$ עבורה $f(x_1) \le L$ אז על־פי סעיף ב' הפונקציה מקבלת מינימום. \\*
	נניח כי לא קיימת נקודה $x_1 \ne x_0$ אשר מקיימת $f(x_1) \le f(x_0)$, אז זוהי כמובן הגדרת המינימום ובמקרה זה $f$ מקבלת מינימום ב־$x_0$ עצמה.
\end{proof}

\section{שאלה 5}
תהי פונקציה $f$ המוגדרת
\[
	f(x) = \frac{(2x + \sin x) \arctan x}{x^2}
\]
נבדוק האם הפונקציה $f$ מקבלת מינימום ב־$(0, \infty)$. \\*
מהגדרה 5.44 ומהגדרה 5.45 ניתן להסיק כי
\[
	\lim_{x \to \infty} \arctan x = \frac{\pi}{2}
\]
לכן נוכל להסיק בנקל מהאריתמטיקה של הגבולות כי
\[
	\lim_{x \to \infty} f(x) = 0
\]
מהגדרת כלל רכיביה האלגבריים של $f$ אנו למדים כי היא רציפה. \\*
מסעיף א' בשאלה 4 נובע כי אם ב־$f$ ישנה נקודת מינימום, אז גם קיים $x_0 \ge 0$ כך ש־$f(x_0) \le 0$.
אנו יודעים מהגדרתן כי $\arctan x, x^2$ חיוביות לכל $x > 0$. באופן דומה, ניתן להסיק כי $2x + \sin x$ חיובי לכל $x > 0$ מהגדרת $\sin x$. \\*
לכן $f(x) > 0$ לכל $x > 0$, ובהתאם ההנחה כי יש ל־$f$ נקודת מינימום סותרת את הטענה כי $x_0$ אשר מקיים את התנאי קיים בתחום. \\*
לכן אין ל־$f$ מינימום בתחום $(0, \infty)$.

\section{שאלה 6}
\subsection{סעיף א'}
נוכיח כי הפונקציה
\[
	f(x) = \sqrt{x^2 + x}
\]
רציפה במידה שווה בקטע $[0, \infty)$. % chktex 9
\begin{proof}
	נראה כי לכל $a \ge 1$ מתקיים
	\[
		0 \le a
		\rightarrow a^2 \le a^2 + a
		\rightarrow a \le \sqrt{a^2 + a}
		\rightarrow a + \frac{1}{2} \le \sqrt{a^2 + a} + \frac{1}{2} \le 2 \sqrt{a^2 + a}
	\]
	לכן לכל $x_1, x_2 \ge 1$ מתקיים גם
	\begin{align*}
		& x_1 + x_2 + 1 \le 2 \sqrt{x_1^2 + x_1} + 2 \sqrt{x_2^2 + x_2} \\
		& \frac{x_1 + x_2 + 1}{ \sqrt{x_1^2 + x_1} + 2 \sqrt{x_2^2 + x_2}} \le 2
	\end{align*}
	יהי $\delta > 0$ ו־$x_1, x_2$ כך ש־$|x_1 - x_2| < \delta$.
	\begin{align*}
		\left\lvert \sqrt{x_1^2 + x_1} - \sqrt{x_2^2 + x_2} \right\rvert
		& = \left\lvert \frac{\left( \sqrt{x_1^2 + x_1} - \sqrt{x_2^2 + x_2} \right)\left( \sqrt{x_1^2 + x_1} + \sqrt{x_2^2 + x_2} \right)}
			{\sqrt{x_1^2 + x_1} + \sqrt{x_2^2 + x_2}} \right\rvert \\
		& = \left\lvert \frac{ x_1^2 + x_1 - x_2^2 - x_2 }{\sqrt{x_1^2 + x_1} + \sqrt{x_2^2 + x_2}} \right\rvert \\
		& = \left\lvert \frac{ (x_1 + x_2 + 1)(x_1 - x_2) }{\sqrt{x_1^2 + x_1} + \sqrt{x_2^2 + x_2}} \right\rvert \\
		& = \left\lvert \frac{ x_1 + x_2 + 1 }{\sqrt{x_1^2 + x_1} + \sqrt{x_2^2 + x_2}} \right\rvert |x_1 - x_2|
	\end{align*}
	נניח כי גם $x_1, x_2 \ge 1$ ולכן
	\[
		\left\lvert \sqrt{x_1^2 + x_1} - \sqrt{x_2^2 + x_2} \right\rvert \le 2 |x_1 - x_2| < 2 \delta
	\]
	יהי $\epsilon > 0$, אם נגדיר $\delta = \frac{1}{2} \epsilon$ אז לכל $x_1, x_2 \in [1, \infty)$, % chktex 9
	אשר עבורם $|x_1 - x_2| < \delta$ מתקיים $|f(x_1) - f(x_2)| < \epsilon$,
	אז מהגדרה 5.46 נובע כי $f(x)$ רציפה במידה שווה בקטע. \\*
	ממשפט קנטור אנו יכולים להסיק כי $f(x)$ רציפה במידה שווה גם ב־$[0, \infty)$. % chktex 9
\end{proof}

\subsection{סעיף ב'}
נוכיח כי הפונקציה
\[
	f(x) = \sqrt{x} \sin \frac{1}{x}
\]
רציפה במידה שווה בקטע $(0, \infty)$.
\begin{proof}
	\begin{align*}
		\lim_{x \to \infty} f(x) & = \lim_{x \to \infty} \sqrt{x} \sin \frac{1}{x} \\
								 & = \lim_{x \to 0} \sqrt{\frac{1}{x}} \sin x && \text{על־פי 5.14} \\
								 & = \lim_{x \to 0} \sqrt{x} \frac{\sin x}{x} \\
		\lim_{x \to \infty} f(x) & = 0 \cdot 1 = 0 && \text{על־פי משפט 4.45}
	\end{align*}
	אנו יודעים כי $\sin x$ היא פונקציה חסומה, ולכן גם $\sin \frac{1}{x}$ חסומה, לכן ממשפט 4.53 נובע כי
	\[
		\lim_{x \to 0^+} f(x) = 0
	\]
	משני הגבולות האלה ומשפט 5.49 נובע כי $f(x)$ רציפה במידה שווה בקטע $(0, \infty)$.
\end{proof}

\subsection{סעיף ג'}
נפריך את הטענה כי הפונקציה
\[
	f(x) = \sin \frac{1}{x}
\]
רציפה במידה שווה בקטע $(0, 1)$.
\begin{proof}
	נניח בשלילה כי $f(x)$ רציפה במידה שווה, ויהיו $\epsilon, \delta$ אשר מקיימים את הגדרה 5.46. \\*
	נקבע
	\[
		x_1 = \frac{1}{\frac{1}{2} \pi + 2 \pi k},
		x_2 = \frac{1}{-\frac{1}{2} \pi + 2 \pi k}
	\]
	כאשר $k \in \NN$. בשל הקשר בין $k$ לבין גודל המכנה, קיים $k$ כך ש־$|x_1 - x_2| < \delta$. \\*
	אך כמובן ש־$|f(x_1) - f(x_2)| = |1 - (-1)| = 2$, בסתירה לטענה כי קיים ערך $\delta$ לכל $\epsilon$.
\end{proof}

\end{document} % chktex 17
