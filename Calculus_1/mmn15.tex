\documentclass[a4paper]{article}

% packages
\usepackage{inputenc, fontspec, amsmath, amsthm, amsfonts, polyglossia, catchfile}
\usepackage[a4paper, margin=50pt, includeheadfoot]{geometry} % set page margins

% style
\AddToHook{cmd/section/before}{\clearpage}	% Add line break before section
\linespread{1.5}
\setcounter{secnumdepth}{0}		% Remove default number tags from sections
\setmainfont{Libertinus Serif}
\setsansfont{Libertinus Sans}
\setmonofont{Libertinus Mono}
\setdefaultlanguage{hebrew}
\setotherlanguage{english}

% operators
\DeclareMathOperator\cis{cis}
\DeclareMathOperator\Sp{Sp}
\DeclareMathOperator\tr{tr}
\DeclareMathOperator\im{Im}
\DeclareMathOperator\diag{diag}
\DeclareMathOperator*\lowlim{\underline{lim}}
\DeclareMathOperator*\uplim{\overline{lim}}

% commands
\renewcommand\qedsymbol{\textbf{משל}}
\newcommand{\NN}[0]{\mathbb{N}}
\newcommand{\ZZ}[0]{\mathbb{Z}}
\newcommand{\QQ}[0]{\mathbb{Q}}
\newcommand{\RR}[0]{\mathbb{R}}
\newcommand{\CC}[0]{\mathbb{C}}
\newcommand{\getenv}[2][] {
  \CatchFileEdef{\temp}{"|kpsewhich --var-value #2"}{\endlinechar=-1}
  \if\relax\detokenize{#1}\relax\temp\else\let#1\temp\fi
}
\newcommand{\explain}[2] {
	\begin{flalign*}
		 && \text{#2} && \text{#1}
	\end{flalign*}
}

% headers
\getenv[\AUTHOR]{AUTHOR}
\author{\AUTHOR}
\date\today


\title{פתרון ממ''ן 15 – חשבון אינפיניטסימלי 1 (20474)}

\begin{document}
\maketitle
\section{שאלה 1}
נמצא את נקודות הרציפות והאי־רציפות של הפונקציה $f$ המוגדרת:
\[
	f(x) = \lfloor x \rfloor \tan \frac{\pi x}{2}
\]
בתחום $\RR$ ונמיינן. \\*
על־פי משפט 5.13 הפונקציה $\tan \frac{\pi x}{2}$ רציפה בכל תחום הגדרתה, ועל־פי הגדרת הפונקציה אנו יודעים כי היא איננה מוגדרת בערכים
\[
	\left\{ 1 + 2k \mid k \in \ZZ \right\}
\]
מהגדרת החלק השלם והפוקנציה $x$ אנו יודעים כי $\lfloor x \rfloor$ רציפה בכל תחום הגדרתה, ולא מוגדרת בנקודות $x \in \ZZ$.
על־פי משפט 5.11 גם $f$ רציפה בכל תחום הגדרתה, והיא כמובן לא מוגדרת ב־$x \in \ZZ$.
אז כלל הנקודות החשודות באי־רציפות הן $x \in \ZZ$. \\*
נגדיר מעתה $k \in \ZZ$. אנו יודעים כי כאשר $x = 1 + 2k$ אז הפונקציה $f(k)$ איננה מוגדרת, וכי $\lim_{x \to k^\pm} f(x) = \pm \infty$,
לכן בנקודות אלה ל־$f$ נקודות אי־רציפות ממין שני. \\*
כאשר $x = 2k$ אנו וידעים כי $\tan{\pi x}{2}$ רציפה, ואילו $\lfloor x \rfloor$ מקיימת
\[
	\lim_{x \to k^+} f(x) = k - 1
	= \left( \lim_{x \to k^+} \lfloor x \rfloor \right) + \left( \lim_{x \to k^+} \tan \frac{\pi x}{2} \right)
	= k - 1 + 0
	= k - 1
\]
וגם
\[
	\lim_{x \to k^-} f(x) = k - 1
	= k + 0
	= k
\]
לכן על־פי הגדרה 5.22 הנקודות $x = 2k$ הן נקודות אי־רציפות ממין ראשון ב־$f$.

\section{שאלה 2}
\subsection{סעיף א'}
תהי $f$ פונקציה המוגדרת בסביבת $x_0$. \\*
\textbf{(i)}
ננסח את הטענה כי $f$ איננה רציפה ב־$x_0$ בלשון $\epsilon, \delta$: \\*
הפונקציה $f$ לא רציפה ב־$x_0$ אם ורק אם קיים $\epsilon > 0$ כך שלכל $\delta > 0$ קיים $x$ כך ש־$|x - x_0| < \delta$ וגם $|f(x) - f(x_0)| \ge \epsilon$. \\
\textbf{(ii)}
ננסח את הטענה כי $f$ איננה רציפה ב־$x_0$ בלשון סדרות: \\*
הפונקציה $f$ איננה רציפה ב־$x_0$ אם ורק אם קיימת סדרה ${(x_n)}_{n = 1}^\infty$ המקיימת $x \underset{n \to \infty}{\rightarrow} x_0$
כך שלא מתקיים $f(x_n) \underset{n \to \infty}{\rightarrow} f(x_0)$.

\subsection{סעיף ב'}
נגדיר $g$ פונקציה הרציפה ב־$x_0$ ופונקציה $f$ המוגדרת $f(x) = g(x) D(x)$. \\*
נוכיח כי אם $g(x_0) = 0$ אז $f$ רציפה ב־$x_0$. \\*
מטענה 5.3 נובע כי לכל $\epsilon > 0$ קיים $\delta > 0$ כך שאם $|x - x_0| < \delta$ אז $|g(x)| < \epsilon$. \\*
על־פי הגדרת פונקציית דיריכלה אנו יודעים כי $D(x) \in \{0, 1\}$ לכל $x \in \RR$,
לכן תמיד מתקיים $D(x) \le 1$. אז כמובן שמתקיים גם $|g(x)| D(x) \le |g(x)| < \epsilon$ ולכן גם $f(x)$ רציפה ב־$x_0$.

\subsection{סעיף ג'}
נגדיר $x_0 \in \RR$ אשר עבורו $g(x_0) \ne 0$, כמובן גם $g$ רציפה ב־$x_0$.

\textbf{(i)}
נוכיח כי הפונקציה $f$ איננה רציפה ב־$x_0$ על־פי ההגדרה מסעיף א' (i)
\begin{proof}
	נמצא $\epsilon > 0$ כך שלכל $\delta > 0$ קיים ערך $x$ כך ש־$|x - x_0| < \delta$ וגם $|f(x) - f(x_0)| \ge \epsilon$. \\*
	נקבע $\epsilon < 1$, ויהי $\delta > 0$ אשר עבורו $|x - x_0| < \delta$. \\*
	מהגדרת פונקציית דיריכלה ואקסיומת הרצף, אנו יכולים להסיק כי קיים מספר $x_1$ אשר מקיים $|x_0 - x_1| < \delta$, ואשר הוא אי־רציונלי.
\end{proof}

\textbf{(ii)}
נוכיח כי הפונקציה $f$ איננה רציפה ב־$x_0$ על־פי ההגדרה מסעיף א' (ii)
\begin{proof}
	נמצא סדרה ${(x_n)}_{n = 1}^\infty$ המקיימת $x \underset{n \to \infty}{\rightarrow} x_0$ כך שלא מתקיים $f(x_n) \underset{n \to \infty}{\rightarrow} f(x_0)$. \\*
	נגדיר $(x_n)$ סדרה אינסופית של מספרים אי־רציונליים, $x_1 < x_0$, ולכל $n \in \NN$ כאשר $1 < n$ גם
	\[
		x_n < x_{n + 1} < x_0
	\]
	הגדרה זו אפשרית כמובן על־פי צפיפות הממשיים. \\*
	על־פי הגדרת הגבול עבור סדרות, מתקיים
	\[
		\lim_{n \to \infty} x_n = x_0
	\]
	אבל אנו יודעים שלכל $n$ גם $f(x_n) = 0$ מהגדרת פונקציית דיריכלה, לכן
	\[
		\lim_{n \to \infty} f(x_n) = 0 \ne x_0
	\]
\end{proof}

\section{שאלה 3}
תהי $f$ פונקציה רציפה בקטע $[0, \infty)$. \\* % chktex 9
נוכיח כי אם לכל $x > 0$ מתקיים $|f(x)| > x$, אז $\lim_{x \to \infty} f(x) = \infty$ או $\lim_{x \to \infty} f(x) = -\infty$.
\begin{proof}
% .לובגה תרדגה זאו תילילש קר וא תיבויח קר וא איהש הז םע והשמ	
	תחילה נראה כי הפונקציה $f(x)$ היא חיובית לכל $x$ או שלילית לכל $x$ בתחום. \\*
	נניח בשלילה כי $f(x)$ משנה סימן כאשר $x = x_0$.
	נניח כי קיים $a$ כך ש־$a < x_0$ וגם $f(a) < -a$ וכי קיים $b > x_0$ כך ש־$f(b) > b$. \\*
	ממשפט ערך הביניים של קושי נובע כי קיים מספר $c$ ככה ש־$f(c) = 0$ בניגוד לנתון כי $|f(x)| > x > 0$. \\*
	יכולנו להגדיר את שינוי הסימן ההפוך וההוכחה הייתה נשארת זהה, לכן לא נפגעת הגבלת הכלליות. \\*
	אז אנו יכולים להסיק כי לכל $x$ מתקיים $f(x) > x$, או לחילופין לכל $x$ מתקיים $-f(x) > x$. \\*
	מהגדרת השאיפה לאינסוף ומינוס אינסוף בפונקיות נובע ישירות כי
	\[
		\lim_{x \to \infty} f(x) = \infty
	\]
	או
	\[
		\lim_{x \to \infty} f(x) = -\infty
	\]
\end{proof}

\section{שאלה 4}
תהי $f$ פונקציה רציפה בקטע $\RR_{0\ge} = [0, \infty)$ ויהי $L \in \RR$. ידוע כי מתקיים % chktex 9
\[
	\lim_{x \to \infty} f(x) = L
\]

\subsection{סעיף א'}
נוכיח כי אם $f$ מקבלת מינימום ב־$\RR_{0\ge}$ אז קיים $x_0 \ge 0$ כך ש־$f(x_0) \le L$.
\begin{proof}
	נקבע כי $x_1$ היא נקודת מינימום של $f(x)$ וכי $f(x_1) = c$. \\*
	נבחן שני מקרים, כאשר $c \le L$ אז כמובן שקיים $x_0$ כזה, והוא כאשר $x_0 = x_1$. \\*
	לכן עלינו רק לבחון את המקרה בו $c > L$. \\*
	מהגדרת הגבול בלשון $\epsilon, M$ נובע כי עבור $\epsilon = |c - L| = c - L$ קיים $M$ כך ש־$|f(x) - L| < c - L$ לכל $x > M$. \\*
	אילו היה מתקיים $f(x) < L < c$ והגענו לסתירה לטענה כי $x_1$ נקודת מינימום.
	\[
		f(x) - L < c - L \rightarrow f(x) < c
	\]
	הגענו לסתירה לטענה כי $x_1$ היא נקודת מינימום של הפונקציה $f(x)$ ולכן לא יתכן כי $c > L$ ובהתאם יתכן רק כי $c \le L$.
\end{proof}

\subsection{סעיף ב'}
נוכיח כי אם קיים $x_0 \ge 0$ כך ש־$f(x_0) < L$, אז $f$ מקבלת מינימום ב־$\RR_{0\ge}$.
\begin{proof}
	בשל הגבול של $f(x)$ באינסוף אנו יכולים להסיק כי קיים $M \in \RR_{0\ge}$ עבורו לכל $x > M$ מתקיים $f(x_0) < x$. \\* % Do I need to elaborate?
	מהמשפט השני של ויירשטראס על הקטע $[0, M]$ נובע כי בקטע זה ישנה נקודת מינימום $x_1$. \\*
	כמובן שמתקיים $f(x_0) \ge f(x_1)$, ולכן בהתאם לכל $x > M$ נסיק כי $f(x_1) < x$, ולכן $x_1$ נקודת מינימום בכל הקטע $\RR_{0\ge}$.
\end{proof}

\subsection{סעיף ג'}
נוכיח כי אם קיים $x_0 \ge 0$ כך ש־$f(x_0) = L$, אז $f$ מקבלת מינימום ב־$\RR_{0\ge}$.
\begin{proof}
	אילו $f$ פונקציה קבועה אז כמובן שהיא מקבלת מינימום, ולכן נניח כי $f$ פונקציה רציפה שאיננה קבועה. \\*
	אילו קיימת נקודה $x_1$ עבורה $f(x_1) \le L$ אז על־פי סעיף ב' הפונקציה מקבלת מינימום. \\*
	נניח כי לא קיימת נקודה $x_1 \ne x_0$ אשר מקיימת $f(x_1) \le f(x_0)$, אז זוהי כמובן הגדרת המינימום ובמקרה זה $f$ מקבלת מינימום ב־$x_0$ עצמה.
\end{proof}

\section{שאלה 5}
תהי פונקציה $f$ המוגדרת
\[
	f(x) = \frac{(2x + \sin x) \arctan x}{x^2}
\]
נבדוק האם הפונקציה $f$ מקבלת מינימום ב־$(0, \infty)$. \\*
מהגדרה 5.44 ומהגדרה 5.45 ניתן להסיק כי
\[
	\lim_{x \to \infty} \arctan x = \frac{\pi}{2}
\]
לכן נוכל להסיק בנקל מהאריתמטיקה של הגבולות כי
\[
	\lim_{x \to \infty} f(x) = 0
\]
מהגדרת כלל רכיביה האלגבריים של $f$ אנו למדים כי היא רציפה. \\*
מסעיף א' בשאלה 4 אנו למדים כי אם ב־$f$ ישנה נקודת מינימום, אז גם קיים $x_0 \ge 0$ כך ש־$f(x_0) \le 0$.
אנו יודעים מהגדרתן כי $\arctan x, x^2$ חיוביות לכל $x > 0$. באופן דומה, ניתן להסיק כי $2x + \sin x$ חיובי לכל $x > 0$ מהגדרת $\sin x$. \\*
לכן $f(x) > 0$ לכל $x > 0$, ובהתאם ההנחה כי יש ל־$f$ נקודת מינימום סותרת את הטענה כי $x_0$ אשר מקיים את התנאי קיים בתחום. \\*
לכן אין ל$f$ מינימום בתחום $(0, \infty)$.

\section{שאלה 6}
\subsection{סעיף א'}
הוכיחו כי הפונקציה
\[
	f(x) = \sqrt{x^2 + x}
\]
רציפה במידה שווה בקטע $[0, \infty)$. % chktex 9
\begin{proof}
	\begin{align*}
		\left\lvert \sqrt{x_1^2 + x_1} - \sqrt{x_2^2 + x_2} \right\rvert
		= \left\lvert \frac{\left( \sqrt{x_1^2 + x_1} - \sqrt{x_2^2 + x_2} \right)\left( \sqrt{x_1^2 + x_1} + \sqrt{x_2^2 + x_2} \right)}
			{\sqrt{x_1^2 + x_1} + \sqrt{x_2^2 + x_2}} \right\rvert
		= \left\lvert \frac{ x_1^2 + x_1 - x_2^2 - x_2 }
			{\sqrt{x_1^2 + x_1} + \sqrt{x_2^2 + x_2}} \right\rvert
	\end{align*}
\end{proof}
\end{document} % chktex 17
