\documentclass[a4paper]{article}

% packages
\usepackage{inputenc, amsmath, amsthm, thmtools, amsfonts, amssymb, luacode, catchfile, tikzducks, hyperref}
\usepackage[a4paper, margin=50pt, includeheadfoot]{geometry} % set page margins
\usepackage[shortlabels]{enumitem}
\usepackage[skip=3pt, indent=0pt]{parskip}

% language
\usepackage[bidi=basic, layout=tabular, provide=*]{babel}
\babelprovide[main, import]{hebrew}
\babelprovide{rl}
\babelfont{rm}{Libertinus Serif}
\babelfont{sf}{Libertinus Sans}
\babelfont{tt}{Libertinus Mono}

% style
\AddToHook{cmd/section/before}{\clearpage}	% Add line break before section
\linespread{1.3}
\setcounter{secnumdepth}{0}		% Remove default number tags from sections, this won't do well with theorems
\AtBeginDocument{\setlength{\belowdisplayskip}{3pt}}
\AtBeginDocument{\setlength{\abovedisplayskip}{3pt}}

% operators
\DeclareMathOperator\cis{cis}
\DeclareMathOperator\Sp{Sp}
\DeclareMathOperator\tr{tr}
\DeclareMathOperator\im{Im}
\DeclareMathOperator\re{Re}
\DeclareMathOperator\diag{diag}
\DeclareMathOperator*\lowlim{\underline{lim}}
\DeclareMathOperator*\uplim{\overline{lim}}
\DeclareMathOperator\rng{rng}
\DeclareMathOperator\Sym{Sym}
\DeclareMathOperator\Arg{Arg}
\DeclareMathOperator\Log{Log}
\DeclareMathOperator\dom{dom}

% commands
%\renewcommand\qedsymbol{\textbf{מש''ל}}
%\renewcommand\qedsymbol{\fbox{\emoji{lizard}}}
\newcommand{\NN}[0]{\mathbb{N}}
\newcommand{\ZZ}[0]{\mathbb{Z}}
\newcommand{\QQ}[0]{\mathbb{Q}}
\newcommand{\RR}[0]{\mathbb{R}}
\newcommand{\CC}[0]{\mathbb{C}}
\newcommand{\FF}[0]{\mathbb{F}}
\newcommand{\PP}[0]{\mathbb{P}}
\newcommand{\TT}[0]{\mathbb{T}}
\newcommand{\acts}[0]{\circlearrowright}
\newcommand{\explain}[2] {
	\begin{flalign*}
		 && \text{#2} && \text{#1}
	\end{flalign*}
}
\newcommand{\maketitleprint}[0]{ \begin{center}
	\begin{tikzpicture}[scale=3]
		\duck[graduate=gray!20!black, tassel=red!70!black]
	\end{tikzpicture}	
\end{center}
}

% theorem commands
\newtheoremstyle{c_remark}
	{}	% Space above
	{}	% Space below
	{}% Body font
	{}	% Indent amount
	{\bfseries}	% Theorem head font
	{}	% Punctuation after theorem head
	{.5em}	% Space after theorem head
	{\thmname{#1}\thmnumber{ #2}\thmnote{ \normalfont{\text{(#3)}}}}	% head content
\newtheoremstyle{c_definition}
	{3pt}	% Space above
	{3pt}	% Space below
	{}% Body font
	{}	% Indent amount
	{\bfseries}	% Theorem head font
	{}	% Punctuation after theorem head
	{.5em}	% Space after theorem head
	{\thmname{#1}\thmnumber{ #2}\thmnote{ \normalfont{\text{(#3)}}}}	% head content
\newtheoremstyle{c_plain}
	{3pt}	% Space above
	{3pt}	% Space below
	{\itshape}% Body font
	{}	% Indent amount
	{\bfseries}	% Theorem head font
	{}	% Punctuation after theorem head
	{.5em}	% Space after theorem head
	{\thmname{#1}\thmnumber{ #2}\thmnote{ \text{(#3)}}}	% head content

\theoremstyle{c_plain}
\newtheorem{theorem}{משפט}[section]
\newtheorem{lemma}[theorem]{למה}
\newtheorem{proposition}[theorem]{טענה}
\newtheorem*{proposition*}{טענה}
%\newtheorem{corollary}[theorem]{אין חלופה עברית}

\theoremstyle{c_definition}
\newtheorem{definition}[theorem]{הגדרה}
\newtheorem*{definition*}{הגדרה}
\newtheorem{example}{דוגמה}[section]
\newtheorem{exercise}{תרגיל}[section]

\theoremstyle{c_remark}
\newtheorem*{remark}{הערה}
\newtheorem*{solution}{פתרון}
\newtheorem{conclusion}[theorem]{מסקנה}
\newtheorem{notation}[theorem]{סימון}

% Questions related commands
\newcounter{question}
\setcounter{question}{1}
\newcounter{sub_question}
\setcounter{sub_question}{1}

\newcommand{\question}[1][0]{
	\ifthenelse{#1 = 0}{}{\setcounter{question}{#1}}
	\subsection{שאלה \arabic{question}}
	\addtocounter{question}{1}
	\setcounter{sub_question}{1}
}

\newcommand{\subquestion}[1][0]{
	\ifthenelse{#1 = 0}{}{\setcounter{sub_question}{#1}}
	\subsubsection{סעיף \localecounter{letters.gershayim}{sub_question}}
	\addtocounter{sub_question}{1}
}

% import lua and start of document
\directlua{common = require ('../common')}

\GetEnv{AUTHOR}

% headers
\author{\AUTHOR}
\date\today


\title{פתרון ממ''ן 15 – חשבון אינפיניטסימלי 1 (20474)}

\begin{document}
\maketitle
\section{שאלה 1}
נמצא את נקודות הרציפות והאי־רציפות של הפונקציה $f$ המוגדרת:
\[
	f(x) = \lfloor x \rfloor \tan \frac{\pi x}{2}
\]
בתחום $\RR$ ונמיינן. \\*
על־פי משפט 5.13 הפונקציה $\tan \frac{\pi x}{2}$ רציפה בכל תחום הגדרתה, ועל־פי הגדרת הפונקציה אנו יודעים כי היא איננה מוגדרת בערכים
\[
	\left\{ 1 + 2k \mid k \in \ZZ \right\}
\]
מהגדרת החלק השלם והפוקנציה $x$ אנו יודעים כי $\lfloor x \rfloor$ רציפה בכל תחום הגדרתה, ולא מוגדרת בנקודות $x \in \ZZ$.
על־פי משפט 5.11 גם $f$ רציפה בכל תחום הגדרתה, והיא כמובן לא מוגדרת ב־$x \in \ZZ$.
אז כלל הנקודות החשודות באי־רציפות הן $x \in \ZZ$. \\*
נגדיר מעתה $k \in \ZZ$. אנו יודעים כי כאשר $x = 1 + 2k$ אז הפונקציה $f(k)$ איננה מוגדרת, וכי $\lim_{x \to k^\pm} f(x) = \pm \infty$,
לכן בנקודות אלה ל־$f$ נקודות אי־רציפות ממין שני. \\*
כאשר $x = 2k$ אנו וידעים כי $\tan{\pi x}{2}$ רציפה, ואילו $\lfloor x \rfloor$ מקיימת
\[
	\lim_{x \to k^+} f(x) = k - 1
	= \left( \lim_{x \to k^+} \lfloor x \rfloor \right) + \left( \lim_{x \to k^+} \tan \frac{\pi x}{2} \right)
	= k - 1 + 0
	= k - 1
\]
וגם
\[
	\lim_{x \to k^-} f(x) = k - 1
	= k + 0
	= k
\]
לכן על־פי הגדרה 5.22 הנקודות $x = 2k$ הן נקודות אי־רציפות ממין ראשון ב־$f$.

\section{שאלה 2}
\subsection{סעיף א'}
תהי $f$ פונקציה המוגדרת בסביבת $x_0$. \\*
\textbf{(i)}
ננסח את הטענה כי $f$ איננה רציפה ב־$x_0$ בלשון $\epsilon, \delta$: \\*
הפונקציה $f$ לא רציפה ב־$x_0$ אם ורק אם קיים $\epsilon > 0$ כך שלכל $\delta > 0$ קיים $x$ כך ש־$|x - x_0| < \delta$ וגם $|f(x) - f(x_0)| \ge \epsilon$. \\
\textbf{(ii)}
ננסח את הטענה כי $f$ איננה רציפה ב־$x_0$ בלשון סדרות: \\*
הפונקציה $f$ איננה רציפה ב־$x_0$ אם ורק אם קיימת סדרה ${(x_n)}_{n = 1}^\infty$ המקיימת $x \underset{n \to \infty}{\rightarrow} x_0$
כך שלא מתקיים $f(x_n) \underset{n \to \infty}{\rightarrow} f(x_0)$.

\subsection{סעיף ב'}
נגדיר $g$ פונקציה הרציפה ב־$x_0$ ופונקציה $f$ המוגדרת $f(x) = g(x) D(x)$. \\*
נוכיח כי אם $g(x_0) = 0$ אז $f$ רציפה ב־$x_0$. \\*
מטענה 5.3 נובע כי לכל $\epsilon > 0$ קיים $\delta > 0$ כך שאם $|x - x_0| < \delta$ אז $|g(x)| < \epsilon$. \\*
על־פי הגדרת פונקציית דיריכלה אנו יודעים כי $D(x) \in \{0, 1\}$ לכל $x \in \RR$,
לכן תמיד מתקיים $D(x) \le 1$. אז כמובן שמתקיים גם $|g(x)| D(x) \le |g(x)| < \epsilon$ ולכן גם $f(x)$ רציפה ב־$x_0$.

\subsection{סעיף ג'}
נגדיר $x_0 \in \RR$ אשר עבורו $g(x_0) \ne 0$, כמובן גם $g$ רציפה ב־$x_0$.

\textbf{(i)}
נוכיח כי הפונקציה $f$ איננה רציפה ב־$x_0$ על־פי ההגדרה מסעיף א' (i)
\begin{proof}
	נמצא $\epsilon > 0$ כך שלכל $\delta > 0$ קיים ערך $x$ כך ש־$|x - x_0| < \delta$ וגם $|f(x) - f(x_0)| \ge \epsilon$. \\*
	נקבע $\epsilon < 1$, ויהי $\delta > 0$ אשר עבורו $|x - x_0| < \delta$. \\*
	מהגדרת פונקציית דיריכלה ואקסיומת הרצף, אנו יכולים להסיק כי קיים מספר $x_1$ אשר מקיים $|x_0 - x_1| < \delta$, ואשר הוא אי־רציונלי.
\end{proof}

\textbf{(ii)}
נוכיח כי הפונקציה $f$ איננה רציפה ב־$x_0$ על־פי ההגדרה מסעיף א' (ii)
\begin{proof}
	נמצא סדרה ${(x_n)}_{n = 1}^\infty$ המקיימת $x \underset{n \to \infty}{\rightarrow} x_0$ כך שלא מתקיים $f(x_n) \underset{n \to \infty}{\rightarrow} f(x_0)$. \\*
	נגדיר $(x_n)$ סדרה אינסופית של מספרים אי־רציונליים, $x_1 < x_0$, ולכל $n \in \NN$ כאשר $1 < n$ גם
	\[
		x_n < x_{n + 1} < x_0
	\]
	הגדרה זו אפשרית כמובן על־פי צפיפות הממשיים. \\*
	על־פי הגדרת הגבול עבור סדרות, מתקיים
	\[
		\lim_{n \to \infty} x_n = x_0
	\]
	אבל אנו יודעים שלכל $n$ גם $f(x_n) = 0$ מהגדרת פונקציית דיריכלה, לכן
	\[
		\lim_{n \to \infty} f(x_n) = 0 \ne x_0
	\]
\end{proof}

\section{שאלה 3}
תהי $f$ פונקציה רציפה בקטע $[0, \infty)$. \\* % chktex 9
נוכיח כי אם לכל $x > 0$ מתקיים $|f(x)| > x$, אז $\lim_{x \to \infty} f(x) = \infty$ או $\lim_{x \to \infty} f(x) = -\infty$.
\begin{proof}
% .לובגה תרדגה זאו תילילש קר וא תיבויח קר וא איהש הז םע והשמ	
	תחילה נראה כי הפונקציה $f(x)$ היא חיובית לכל $x$ או שלילית לכל $x$ בתחום. \\*
	נניח בשלילה כי $f(x)$ משנה סימן כאשר $x = x_0$.
	נניח כי קיים $a$ כך ש־$a < x_0$ וגם $f(a) < -a$ וכי קיים $b > x_0$ כך ש־$f(b) > b$. \\*
	ממשפט ערך הביניים של קושי נובע כי קיים מספר $c$ ככה ש־$f(c) = 0$ בניגוד לנתון כי $|f(x)| > x > 0$. \\*
	יכולנו להגדיר את שינוי הסימן ההפוך וההוכחה הייתה נשארת זהה, לכן לא נפגעת הגבלת הכלליות. \\*
	אז אנו יכולים להסיק כי לכל $x$ מתקיים $f(x) > x$, או לחילופין לכל $x$ מתקיים $-f(x) > x$. \\*
	מהגדרת השאיפה לאינסוף ומינוס אינסוף בפונקיות נובע ישירות כי
	\[
		\lim_{x \to \infty} f(x) = \infty
	\]
	או
	\[
		\lim_{x \to \infty} f(x) = -\infty
	\]
\end{proof}

\section{שאלה 4}
תהי $f$ פונקציה רציפה בקטע $\RR_{0\ge} = [0, \infty)$ ויהי $L \in \RR$. ידוע כי מתקיים % chktex 9
\[
	\lim_{x \to \infty} f(x) = L
\]

\subsection{סעיף א'}
נוכיח כי אם $f$ מקבלת מינימום ב־$\RR_{0\ge}$ אז קיים $x_0 \ge 0$ כך ש־$f(x_0) \le L$.
\begin{proof}
	נקבע כי $x_1$ היא נקודת מינימום של $f(x)$ וכי $f(x_1) = c$. \\*
	נבחן שני מקרים, כאשר $c \le L$ אז כמובן שקיים $x_0$ כזה, והוא כאשר $x_0 = x_1$. \\*
	לכן עלינו רק לבחון את המקרה בו $c > L$. \\*
	מהגדרת הגבול בלשון $\epsilon, M$ נובע כי עבור $\epsilon = c$ קיים $M$ כך ש־$|f(x) - L| < c$ לכל $x > M$.
\end{proof}

\section{שאלה 5}
\subsection{סעיף א'}
נוכיח כי הפונקציה $f(x) = \sqrt{1 + x^2}$ רציפה במידה שווה בקטע $[0, \infty)$: % chktex 9
\begin{align*}
	\epsilon & > \left| \sqrt{1 + x_0^2} - \sqrt{1 + x_1^2} \right| \\
	& = \frac{\left| \left( \sqrt{1 + x_0^2} - \sqrt{1 + x_1^2} \right) \left( \sqrt{1 + x_0^2} + \sqrt{1 + x_1^2} \right) \right|}
	{\left| \sqrt{1 + x_0^2} + \sqrt{1 + x_1^2} \right|} \\
	& = \frac{\left|  1 + x_0^2 - 1 - x_1^2 \right|} {\left| \sqrt{1 + x_0^2} + \sqrt{1 + x_1^2} \right|} \\
	& = \frac{\left|  (x_0 + x_1)(x_0 - x_1) \right|} {\left| \sqrt{1 + x_0^2} + \sqrt{1 + x_1^2} \right|} \\
	& = \frac{\left|  x_0 + x_1 \right|} {\left| \sqrt{1 + x_0^2} + \sqrt{1 + x_1^2} \right|} |x_0 - x_1| \\
\end{align*}
בקטע הנתון $x_0, x_1 > 0$ ולכן $x_0 + x_1 > 0$, כמו־כן שורשים אלה מוגדרים בכל הקטע וחיוביים בו:
\[
	\frac{x_0 + x_1} {\sqrt{1 + x_0^2} + \sqrt{1 + x_1^2}} |x_0 - x_1| < \epsilon
\]
לכל $x > 0$ מתקיים $\sqrt{x^2 + 1} > x$.
נגדיר את $\delta$:
\[
	\frac{x_0 + x_1} {\sqrt{1 + x_0^2} + \sqrt{1 + x_1^2}} |x_0 - x_1| < 
	\frac{\sqrt{1 + x_0^2} + \sqrt{1 + x_1^2}}{\sqrt{1 + x_0^2} + \sqrt{1 + x_1^2}} |x_0 - x_1| < \delta
\]
ולכן
\[
	|x_0 - x_1| < \delta
\]
כמו קראינו, במצב זה גם מתקיים
\[
	|f(x_0) - f(x_1)| < \epsilon
\]
ולכן הפונקציה $f$ רציפה במידה שווה בקטע $[0, \infty)$. % chktex 9

\subsection{סעיף ב'}
נוכיח כי הפונקציה המוגדרת רציפה במידה שווה בקטע $(0, \infty)$:
\[
	f(x) = (1 - \cos x) \sin \frac{1}{x}
\]
הפונקציה $f$ מוגדרת בכל הקטע הנתון ומורכבת ממכפלת והרכבת פונקציות רציפות ולכן רציפה גם (למצוא תירוץ יותר טוב).
נראה כי מתקיים:
\[
	\lim_{x_0 \to 0^+} 1 - \cos x_0 = 0
\]
הפונקציה $\sin \frac{1}{x}$ אומנם איננה מתכנסת ב־$x_0 = 0$, אבל חסומה ב־$[-1, 1]$ ולכן על־פי הגדרת היינה לגבול חד־צדדי ומשפט 2.22 מתקיים:
\[
	\lim_{x_0 \to 0^+} (1 - \cos x_0) \sin \frac{1}{x_0} = 0
\]
נמצא את הגבול
\[
	\lim_{x_0 \to \infty^-} (1 - \cos x_0) \sin \frac{1}{x_0}
\]
במקרה זה $\lim_{x \to \infty} \sin \frac{1}{x} = 0$ על־פי גבול $\sin x \underset{x \to 0}{\to} 0$ והרכבת פונקציה עם $\frac{1}{x}$.
הפונקציה $1 - \cos x$ חסומה בקטע $[-1, 1]$ ובאופן דומה:
\[
	\lim_{x_0 \to \infty^-} (1 - \cos x_0) \sin \frac{1}{x_0} = 0
\]
יש בספר משפט שמרחיב את משפט 5.49 לקטעים אינסופיים, תשתמש בזה.

\subsection{סעיף ג'}
נוכיח כי לכל $y \ge x \ge 1$ מתקיים
\begin{align*}
	y^2 \arctan y - x^2 \arctan x & \ge (y^2 - x^2) \arctan x \\
	y^2 \arctan y & \ge y^2 \arctan x \\
	\arctan y & \ge \arctan x \\
\end{align*}
על־פי טענה 5.44 ומשפט 5.43 מתקיים $\arctan y \ge \arctan x$ אם $y \ge x$ ולכן אי־השוויון מתקיים. \\
נוכיח כי הפונקציה $f$, המוגדרת:
\[
	f(x) = x^2 \arctan x
\]
איננה רציפה במידה שווה בקטע $[1, \infty)$: \\* %chktex 9
נניח בשלילה כי הפונקציה $f$ רציפה במידה שווה לכל $\epsilon > 0$, לכן לכל $x, y \in \RR$ שעבורם מתקיים $|x - y| < \delta$, מתקיים
\[
	\left| y^2 \arctan y - x^2 \arctan x \right| < \epsilon
\]
לכן גם מתקיים:
\[
	\left| (x^2 - y^2) \arctan x \right| < \epsilon
\]
הפונקציה $\arctan x$ חיובית לכל $x \ge 0$ ולכן
\[
	\left| (x - y)(x + y) \right| \arctan x = \left| x - y \right| (x + y) \arctan x < \epsilon
\]

\end{document} % chktex 17
