\documentclass[a4paper]{article}

% packages
\usepackage{inputenc, fontspec, amsmath, amsfonts, polyglossia, catchfile}
\usepackage[a4paper, margin=50pt, includeheadfoot]{geometry} % set page margins

% style
\AddToHook{cmd/section/before}{\clearpage}	% Add line break before section
\setdefaultlanguage{hebrew}
\setotherlanguage{english}
\setmainfont{Libertinus Serif}
\linespread{1.5}
\setcounter{secnumdepth}{0}		% Remove default number tags from sections

% custom operators
\newcommand{\getenv}[2][]{%
  \CatchFileEdef{\temp}{"|kpsewhich --var-value #2"}{\endlinechar=-1}%
  \if\relax\detokenize{#1}\relax\temp\else\let#1\temp\fi}
\getenv[\AUTHOR]{AUTHOR}
\DeclareMathOperator\cis{cis}
\DeclareMathOperator\Sp{Sp}
\DeclareMathOperator\tr{tr}
\DeclareMathOperator\im{Im}
\DeclareMathOperator\diag{diag}
\def\NN{\mathbb{N}}
\def\RR{\mathbb{R}}
\def\CC{\mathbb{C}}

% theorems
\title{פתרון ממ''ן 12 – חשבון אינפיניטסימלי 1 (20474)}
\author{\AUTHOR}
\date\today

\begin{document}
\maketitle
\section{שאלה 1}
\subsection{סעיף א'}
נוכיח כי מתקיים הגבול:
\[
	\lim_{n \to \infty} \frac{3n^2 - 4}{n^2 - 4} = 3
\]
נוכיח כי הגבול מתקיים על־פי הגדרת הגבול בלשון $\epsilon, N$.
על־פי ההגדרה, לכל $\epsilon > 0$ עלינו למצוא ערך $N$ עבורו לכל $N < n$
מתקיים אי־השוויון
\[
	\left| \frac{3n^2 - 4}{n^2 - 4} - 3 \right| =
	\left| \frac{3n^2 - 4 - 3n^3 + 12}{n^2 - 4} \right| =
	\left| \frac{8}{n^2 - 4} \right|
	< \epsilon \tag{1}
\]
נשים לב כי לכל $2 < n$ ערך המונה והמכנה באגף השמאלי הוא חיובי,
לכן בוודאי גם הערך הפנימי לערך המוחלט חיובי, ונוכל להשמיט את הערך המוחלט:
\begin{align*}
	& \frac{8}{n^2 - 4} < \epsilon \\
	& \frac{8}{\epsilon} < n^2 - 4 \\
	& \frac{8}{\epsilon} + 4 < n^2 \\
	& 2 \sqrt{\frac{2}{\epsilon} + 1} < n \\
\end{align*}
נגדיר
\[
	N = \max\left(3, 2 \sqrt{\frac{2}{\epsilon} + 1}\right)
\]
הראינו כי לכל $\epsilon > 0$ קיים $N$
כך שעבור כל $N < n$ מתקיים אי־שוויון $(1)$ ולכן מתקיים
\[
	\lim_{n \to \infty} \frac{3n^2 - 4}{n^2 - 4} = 3
\]

\subsection{סעיף ב'}
\textbf{(i)}
יהיו $(a_n)$ סדרה ו־$L$ מספר ממשי. ננסח בלשון $\epsilon, N$ את הטענה
\[
	\lim_{n \to \infty} a_n \ne L
\]
תחילה נצרין את הטענה:
\[
	\lnot \forall \epsilon > 0 (
	\exists N \in \NN (
	\forall n > N (\left| a_n - L \right| < \epsilon)))
\]
נפשט את הפסוק:
\[
	\exists \epsilon > 0 (
	\forall N \in \NN (
	\exists n > N (\left| a_n - L \right| \ge \epsilon)))
\]
ננסח פסוק זה במילים: \\*
הטענה $\lim_{n \to \infty} a_n \ne L$ מתקיימת אם ורק אם
קיים $\epsilon > 0$ עבורו לכל $N$ טבעי קיים $n > N$
כך שמתקיים $\left| a_n - L \right| \ge \epsilon$. \\
\textbf{(ii)}
על־פי הגדרת התבדרות (2.13) סדרה נקראת מתבדרת כאשר אין מספר $L$
שעבורו מתקיים הגבול $\lim_{n \to \infty} a_n = L$.
ננסח את הטענה בעזרת הגדרת התכנסות בלשון $\epsilon, N$: \\*
סדרה $(a_n)$ היא מתבדרת אם ורק אם
לכל $L$ ממשי קיים $\epsilon > 0$ עבורו לכל $N$ טבעי קיים $n > N$
כך שמתקיים $\left| a_n - L \right| \ge \epsilon$.

\subsection{סעיף ג'}
נוכיח שהסדרה $a_n = \langle \sqrt{n} \rangle$ מתבדרת בלשון $\epsilon, N$. \\*
נוכיח כי לכל $L$ ממשי שנבחר קיים $\epsilon > 0$ שעבורו לכל $N$
טבעי קיים $n > N$ שעבורו $\left| a_n - L \right| \ge \epsilon$. \\*
תחילה נבחין כי לכל $n$ מתקיים $0 \le a_n \le 1$, על־פי הגדרת החלק השברי.
לכן לכל $L$ שאיננו בתחום $[-1, 1]$ ו־$\epsilon = 1$
קיים $n$ שהוא שורש שלם של מספר טבעי וגדול מ־$N$ כלשהו. \\*
הראינו כי התנאי נכון מלבד בתחום $[-1, 1]$, נותר עתה לראות את נכונותו אף בתחום.
נגדיר $\epsilon = \left| L \right|$ ו־$n$ להיות ריבוע מספר טבעי הגדול מ־$N$.
במקרה זה $a_n = 0$ ובהתאם אי־השוויון $\left| a_n - L \right| \ge \epsilon$
מתאפס ל־$\left| L \right| \ge \epsilon = \left| L \right|$ והתנאי מתקיים,
לכן הסדרה $(a_n)$ מתבדרת.

\section{שאלה 2}
נחשב את הגבולות הבאים, או נוכיח שאינם קיימים:

\subsection{למה א'}
לכל $k \in \NN$ ו־$h \in \RR$ מתקיים
\[
	\lim_{n \to \infty} \frac{h}{n^k} = 0
\]
על־פי אריתמטיקה של גבולות וטענה 2.10:
\[
	\lim_{n \to \infty} \frac{h}{n^k} = 0
	= \lim_{n \to \infty} h \cdot
	{\left(\lim_{n \to \infty} \frac{1}{n}\right)}^k
	= h {0}^k
	= 0
\]

\subsection{סעיף א'}
\begin{align*}
	\lim_{n \to \infty} \frac{2n^3 - 5n^5 + 9}{2n^4 - 4n^7 - \pi}
	& = \lim_{n \to \infty} \frac{2n^3/n^7 - 5n^5/n^7 + 9/n^7}
	{2n^4/n^7 - 4n^7/n^7 - \pi/n^7} \\
	& = \lim_{n \to \infty} \frac{2/n^4 - 5/n^2 + 9/n^7}
	{2/n^3 - 4 - \pi/n^7} \\
	& = \lim_{n \to \infty} \frac{0}
	{-4} & \text{על־פי למה א'} \\
	& = 0
\end{align*}

\subsection{סעיף ב'}
\begin{align*}
	\lim_{n \to \infty} \frac{2n^3 - 5n^5 + 9}{2n^4 - 4n^5 - \pi}
	& = \lim_{n \to \infty} \frac{2n^3/n^5 - 5n^5/n^5 + 9/n^5}
	{2n^4/n^5 - 4n^5/n^5 - \pi/n^5} \\
	& = \lim_{n \to \infty} \frac{2/n^2 - 5 + 9/n^5}
	{2/n - 4 - \pi/n^5} \\
	& = \lim_{n \to \infty} \frac{-5}
	{-4} & \text{על־פי למה א'} \\
	& = \frac{5}{4}
\end{align*}

\subsection{סעיף ג'}
\begin{align*}
	\lim_{n \to \infty} \sqrt{n^2 + 2n} - \sqrt{n^2 - 2n}
	& = \lim_{n \to \infty} \frac
		{(\sqrt{n^2 + 2n} - \sqrt{n^2 - 2n})
		(\sqrt{n^2 + 2n} + \sqrt{n^2 - 2n})}
		{\sqrt{n^2 + 2n} + \sqrt{n^2 - 2n}} \\
	& = \lim_{n \to \infty} \frac
		{n^2 + 2n - n^2 - 2n}
		{\sqrt{n^2 + 2n} + \sqrt{n^2 - 2n}} \\
	& = \lim_{n \to \infty} \frac
		{4n}
		{n \sqrt{1 + \frac{2}{n}} + n \sqrt{1 - \frac{2}{n}}} \\
	& = \lim_{n \to \infty} \frac
		{4}
		{\sqrt{1 + \frac{2}{n}} + \sqrt{1 - \frac{2}{n}}}
	& \tag{1} \\
\end{align*}
נוכיח כי מתקיים:
\[
	\lim_{n \to \infty} \sqrt{1 \pm \frac{2}{n}} = 1
\]
לכל $n$ מתקיים:
\[
	1 < 1 \pm \frac{2}{n} < {(1 \pm \frac{2}{n})}^2
\]
נפעיל שורש:
\[
	1 < \sqrt{1 \pm \frac{2}{n}} < 1 \pm \frac{2}{n}
\]
על־פי משפט 2.32:
\[
	\lim_{n \to \infty} 1
	= \lim_{n \to \infty} \sqrt{1 \pm \frac{2}{n}} 
	= \lim_{n \to \infty} 1 \pm \frac{2}{n} = 1
\]
נציב ב־$(1)$:
\[
	\frac {4} {1 + 1} = 2
\]

\subsection{סעיף ד'}
\[
	\lim_{n \to \infty} \sqrt[n]{ 2^n - n^2 }
	= \lim_{n \to \infty} \sqrt[n]{ 2^n (1 - \frac{n^2}{2^n}) }
	= \lim_{n \to \infty} 2 \sqrt[n]{ 1 - \frac{n^2}{2^n} }
\]
אילו מוגדר הגבול
\[
	\lim_{n \to \infty} \sqrt[n]{ 1 - \frac{n^2}{2^n} } = L \tag{1}
\]
אז על־פי האריתמטיקה של הגבולות ערך הגבול המבוקש $2L$.
נוכיח כי גבול זה מתקיים ונמצאהו.
הביטוי $\frac{n^2}{2^n}$ מוגדר וחיובי לכל $n$ ולכן
\[
	1 - \frac{n^2}{2^n} < 1
\]
נבצע שורש:
\[
	\sqrt[n]{1 - \frac{n^2}{2^n}} < \sqrt[n]{1} = 1 \tag{2}
\]
אנו רואים כי לסדרה בגבול $(1)$ מתקיים שכמעט כל איבר שלה קטן מ־$1$. \\*
על־פי מסקנה 2.49 מתקיים
\[
	\lim_{n \to \infty} \frac{n^2}{2^n} = 0
\]
לכן לכמעט לכ $n$ מתקיים
\[
	\frac{n^2}{2^n} < \frac{1}{2}
	\rightarrow
	1 - \frac{1}{2} = \frac{1}{2} < 1 - \frac{n^2}{2^n}
\]
לאחר הפעלת שורש:
\[
	\frac{1}{\sqrt[n]{2}} < \sqrt[n]{1 - \frac{n^2}{2^n}}
\]
על־פי טענה 2.35 ואריתמטיקה של הגבולות:
\[
	\lim_{n \to \infty} \frac{1}{\sqrt[n]{2}} = 1 \tag{3}
\]

משילוב טענות $(2)$ ו־$(3)$ ועל־פי משפט 2.32 מתקיים:
\[
	\lim_{n \to \infty} \sqrt[n]{1 - \frac{n^2}{2^n}} = 1
\]
משילוב בטענה $(1)$ אנו מקבלים:
\[
	\lim_{n \to \infty} \sqrt[n]{ 2^n - n^2 } = 2
\]

\subsection{סעיף ה'}
\[
	\lim_{n \to \infty} \frac{n}{n + 2} \sum_{k = 1}^n \frac{k}{k + 3}
\]
נכתוב מחדש את ערך הסדרה:
\[
	\frac{n}{n + 2} \sum_{k = 1}^n \frac{k}{k + 3}
	= \frac{n}{n + 2} n \frac{1}{n} \sum_{k = 1}^n \frac{k}{k + 3}
\]
מתקיים:
\[
	\lim_{n \to \infty} \frac{n}{n + 2} \sum_{k = 1}^n \frac{k}{k + 3}
	=
	\left( \lim_{n \to \infty} \frac{n^2}{n + 2} \right)
	\left( \lim_{n \to \infty} \frac{1}{n}
	\sum_{k = 1}^n \frac{k}{k + 3} \right)
\]
אם ורק אם שני הגבולות מתקיימים על־פי האריתמטיקה של הגבולות,
לכן נוכיח ששני הגבולות קיימים ונמצא את ערכם. \\*
תחילה נוכיח את התכנסות הגבול
\[
	\lim_{n \to \infty} \frac{n^2}{n + 2}
\]
מתקיים:
\[
	\frac{n^2}{n + 2} = \frac{1}{\frac{1}{n} + \frac{2}{n^2}}
\]
לכן
\[
	\lim_{n \to \infty} \frac{n^2}{n + 2}
	= \lim_{n \to \infty} \frac{1}{\frac{1}{n} + \frac{2}{n^2}}
	= \frac{\lim_{n \to \infty} 1}{\lim_{n \to \infty}
	\frac{1}{n} + \frac{2}{n^2}}
	= \frac{1}{0^+}
	= \infty
\]
על־פי האריתמטיקה של גבולות אינסופיים. \\*
נוכיח כי הגבול
\[
	\lim_{n \to \infty} \frac{1}{n} \sum_{k = 1}^n \frac{k}{k + 3}
\]
מתקיים אף הוא. \\*
על־פי משפט 2.51 אם הגבול קיים אז מתקיים גם
\[
	\lim_{n \to \infty} \frac{1}{n} \sum_{k = 1}^n \frac{k}{k + 3}
	= \lim_{n \to \infty} \frac{n}{n + 3}
	= \lim_{n \to \infty} \frac{1}{1 + \frac{3}{n}}
	= \frac{1}{1 + \lim_{n \to \infty} \frac{3}{n}}
	= \frac{1}{1 + 0}
	= 1
\]
ראינו כי הגבול מתקיים ושווה ל־$0$,
לכן מתקיים:
\[
	\lim_{n \to \infty} \frac{n}{n + 2} \sum_{k = 1}^n \frac{k}{k + 3}
	= \infty \cdot 1
	= \infty
\]
דהינו, על־פי האריתמטיקה של גבולות אינסופיים הגבול שואף לאינסוף,
וכמובן מתכנס במובן הרחב.

\section{שאלה 3}
יהיו $(a_n)$ ו־$(b_n)$ סדרות כך שמתקיים $(1) \lim_{n \to \infty} a_n b_n = 1$.

\subsection{סעיף א'}
נוכיח כי אם כמעט כל אברי $(a_n)$ חיוביים, אז כמעט כל אברי $(b_n)$ חיוביים.
נניח בשלילה כי כמעט כל אברי $(b_n)$ הם שליליים. קיים $N \in \NN$
כך שלכל $n > N$ מתקיים $a_n > 0$ ו־$b_n < 0$, לכן $a_n b_n < 0$,
אבל ידוע כי לכל $n > N$ גם מתקיים $a_n b_n > 0$ על־פי $(1)$.
זוהי סתירה ולכןכמעט כל אברי $(b_n)$ מקיימים $b_n > 0$.
נראה באופן דומה כי לא יתכן ש־$b_n = 0$ כמעט לכל $n$.
במצב זה הסדרה תתכנס ל־$0$, ונתון כי זהו לא המצב,
לכן אם כמעט כל איבר ב־$(a_n)$ חיובי, אז גם כמעט כל איבר ב־$(b_n)$ חיובי.

\subsection{סעיף ב'}
נראה כי אם הסדרות $(a_n)$ ו־$(b_n)$ חיוביות,
אז לא בהכרח לפחות אחת מהן מתכנסת. \\*
נגדיר
\[
	a_n = \begin{cases}
		1 & \text{ אי־זוגי} n \\
		3 & \text{ זוגי} n
	\end{cases}, b_n = \frac{1}{a_n}
\]
ניתן לראות כי לכל $n$ מתקיים $a_n b_n = 1$,
ולכן מכפלת הסדרות מתכנסת ל־$1$.
לעומת זאת, הסדרות $(a_n)$ ו־$(b_n)$ שתיהן מתבדרות: \\*
לכל $n$ מתקיים $a_n = {(-1)}^n + 2$,
ועל־פי טענה 2.14 סדרה זו היא הזזה של סדרה מתבדרת, ומתבדרת בעצמה.
הסדרה $(b_n)$ מוגדרת לפי $a_n$ לכל $n$, ומתבדרת גם היא.

\subsection{סעיף ג'}
נוכיח כי אם $\lim_{n \to \infty} b_n = \infty$ אז $\lim_{n \to \infty} = 0$.
על־פי הגדרת שאיפה לאינסוף, כמעט לכל $n$ מתקיים $b_n > 0$,
לכן בגבול
\[
	\lim_{n \to \infty} \frac{a_n b_n}{b_n}
	= \lim_{n \to \infty} a_n
\]
מוגדר וערכו לפי האריתמטיקה של הגבולות האינסופיים היא $0$.

\subsection{סעיף ד'}
נראה כי אם $\lim_{n \to \infty} a_n = 0$ אז לא תמיד
$\lim_{n \to \infty} b_n = \infty$.
נגדיר
\[
	a_n = \frac{1}{-n}, b_n = -n
\]
לכן $a_n b_n = 1$ לכל $n$
ו־$\lim_{n \to \infty} a_n = 0$
אבל $\lim_{n \to \infty} b_n = -\infty$.

\subsection{סעיף ה'}
נוכיח כי אם $(a_n)$ סדרה חיובית, אז קיים $N \in \NN$ כך שלכל $n > N$
מתקיים $b_n > \frac{1}{2a_n}$. \\*
על־פי הגבול $(1)$ כמעט לכל $n$ מתקיים
\[
	\left| a_n b_n - 1 \right| < \frac{1}{2}
\]
לכן גם
\[
	-\frac{1}{2} < a_n b_n - 1 < \frac{1}{2}
	\rightarrow
	1 - \frac{1}{2} < a_n b_n < 1 + \frac{1}{2}
\]
לכן בפרט
\[
	a_n b_n > \frac{1}{2}
\]
ידוע כי $(a_n)$ חיובית ולכן נחלק בערכה:
\[
	b_n > \frac{1}{2 a_n}
\]

\subsection{סעיף ו'}
נוכיח כי אם $(a_n)$ חיובית ואפסה אז $\lim_{n \to \infty} b_n = \infty$. \\*
בדומה לסעיף ג' מתקיים
\[
	\lim_{n \to \infty} b_n
	= \lim_{n \to \infty} \frac{a_n b_n}{a_n}
	= \frac{\lim_{n \to \infty} a_n b_n}{\lim_{n \to \infty} a_n}
	= \frac{1}{0^+}
	= \infty
\]
על־פי האריתמטיקה של הגבולות האינסופיים.

\subsection{סעיף ז'}
נוכיח כי אם $\lim_{n \to \infty} |a_n| = 1$
אז $\lim_{n \to \infty} |b_n| = 1$. \\*
בהכפלת הגבול $(1)$ בעצמו נקבל
\[
	\lim_{n \to \infty} a_n^2 b_n^2 = 1^2 = 1
\]
ידוע כי $\lim_{n \to \infty} a_n^2 = 1$, לכן
\[
	\lim_{n \to \infty} \frac{a_n^2 b_n^2}{a_n^2} = \frac{1}{1}
	\rightarrow
	\lim_{n \to \infty} b_n^2 = 1
\]
לכן לכל $\epsilon > 0$ כמעט לכל $n$ מתקיים
\[
	|b_n^2 - 1| < \epsilon
\]
לפי נוסחת הכפל המקוצר מתקיים
\[
	\left| \left| b_n \right| - 1 \right|
	\left| \left| b_n \right| + 1 \right| < \epsilon
\]
הביטוי $\left| \left| b_n \right| + 1 \right| > 1$ לכל $n$, לכן
\[
	\left| \left| b_n \right| - 1 \right| <
	\left| \left| b_n \right| - 1 \right|
	\left| \left| b_n \right| + 1 \right| < \epsilon
\]
לכן בפרט
\[
	\left| \left| b_n \right| - 1 \right| < \epsilon
\]
לכן
\[
	\lim_{n \to \infty} |b_n| = 1
\]
\end{document}
