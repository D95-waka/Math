\documentclass[a4paper]{article}

% packages
\usepackage{inputenc, fontspec, amsmath, amsfonts, polyglossia, catchfile}
\usepackage[a4paper, margin=50pt, includeheadfoot]{geometry} % set page margins

% style
\AddToHook{cmd/section/before}{\clearpage}	% Add line break before section
\setdefaultlanguage{hebrew}
\setotherlanguage{english}
\setmainfont{Libertinus Serif}
\linespread{1.5}
\setcounter{secnumdepth}{0}		% Remove default number tags from sections

% custom operators
\newcommand{\getenv}[2][]{%
  \CatchFileEdef{\temp}{"|kpsewhich --var-value #2"}{\endlinechar=-1}%
  \if\relax\detokenize{#1}\relax\temp\else\let#1\temp\fi}
\getenv[\AUTHOR]{AUTHOR}
\DeclareMathOperator\cis{cis}
\DeclareMathOperator\Sp{Sp}
\DeclareMathOperator\tr{tr}
\DeclareMathOperator\im{Im}
\DeclareMathOperator\diag{diag}
\def\NN{\mathbb{N}}
\def\RR{\mathbb{R}}
\def\CC{\mathbb{C}}

\title{פתרון ממ''ן 12 – חשבון אינפיניטסימלי 1 (20474)}
\author{\AUTHOR}
\date\today

\begin{document}
\maketitle
\section{שאלה 1}
\subsection{סעיף א'}
נוכיח בלשון $\epsilon, N$ כי מתקיים:
\begin{align}
	\lim_{n \to \infty} \sqrt{\frac{4n + 1}{n}} = 2
\end{align}
נשים לב תחילה כי מתקיים
\[
	\frac{4n + 1}{n}
	= \frac{\frac{4n}{n} + \frac{1}{n}}{\frac{n}{n}}
	= \frac{4 + \frac{1}{n}}{1}
	= 4 + \frac{1}{n}
\]
לכל $n \ne 0$. לכן נוכל להוכיח את קיום הגבול
\[
	\lim_{n \to \infty} \sqrt{4 + \frac{1}{n}} = 2
\]
נגדיר $\epsilon > 0$ מספר ממשי. על־פי הגדרת הגבול צריך להתקיים:
\[
	\left| \sqrt{4 + \frac{1}{n}} - 2 \right| < \epsilon
\]
תוכן השורש הוא תמיד לפחות 4, ולכן תוצאתו תמיד גדולה מ־2, בהתאם תוכן הערך המוחלט חיובי תמיד ומתקיים:
\[
	\left| \sqrt{4 + \frac{1}{n}} - 2 \right|
	= \sqrt{4 + \frac{1}{n}} - 2 < \epsilon
\]
לכל $n > N$ כאשר $N \in \NN$. נשים לב כי
\[
	{\left(2 + \sqrt{\frac{1}{n}}\right)}^2 = 4 + 2\sqrt{\frac{1}{n}} + \frac{1}{n} > 4 + \frac{1}{n} = {\left(\sqrt{4 + \frac{1}{n}}\right)}^2
	\rightarrow \sqrt{4 + \frac{1}{n}} < 2 + \sqrt{\frac{1}{n}}
\]
נקבע
\[
	\sqrt{4 + \frac{1}{n}} - 2 < 2 + \sqrt{\frac{1}{n}} - 2 = \frac{1}{\sqrt{n}} < \epsilon
\]
נגדיר \[
	N = \left\lceil \frac{1}{\epsilon^2} \right\rceil
\]
במצב זה הגבול $(1)$ מתקיים לכל $\epsilon > 0$.

\subsection{סעיף ב'}
\textbf{(i)}
יהיו $(a_n)$ סדרה ו־$L$ מספר ממשי. ננסח בלשון $\epsilon, N$ את הטענה
\[
	\lim_{n \to \infty} a_n \ne L
\]
תחילה נצרין את הטענה:
\[
	\lnot \forall \epsilon > 0 (
	\exists N \in \NN (
	\forall n > N (\left| a_n - L \right| < \epsilon)))
\]
נפשט את הפסוק:
\[
	\exists \epsilon > 0 (
	\forall N \in \NN (
	\exists n > N (\left| a_n - L \right| \ge \epsilon)))
\]
ננסח פסוק זה במילים: \\*
הטענה $\lim_{n \to \infty} a_n \ne L$ מתקיימת אם ורק אם
קיים $\epsilon > 0$ עבורו לכל $N$ טבעי קיים $n > N$
כך שמתקיים $\left| a_n - L \right| \ge \epsilon$. \\
\textbf{(ii)}
על־פי הגדרת התבדרות (2.13) סדרה נקראת מתבדרת כאשר אין מספר $L$
שעבורו מתקיים הגבול $\lim_{n \to \infty} a_n = L$.
ננסח את הטענה בעזרת הגדרת התכנסות בלשון $\epsilon, N$: \\*
סדרה $(a_n)$ היא מתבדרת אם ורק אם
לכל $L$ ממשי קיים $\epsilon > 0$ עבורו לכל $N$ טבעי קיים $n > N$
כך שמתקיים $\left| a_n - L \right| \ge \epsilon$.

\subsection{סעיף ג'}
נוכיח שהסדרה $(a_n)$ מתבדרת בלשון $\epsilon, N$, כאשר
\[
	a_n = \frac{{(-1)}^n n + 1}{n + 2}
\]
נוכיח כי לכל $L$ ממשי שנבחר קיים $\epsilon > 0$ שעבורו לכל $N$ טבעי קיים $n > N$ שעבורו $\left| a_n - L \right| \ge \epsilon$. \\*
מתקיים על־פי חישוב ישיר
\begin{align*}
	& n = 2 \rightarrow a_n = \frac{3}{4} \\
	& n = 3 \rightarrow a_n = -\frac{2}{5} \\
\end{align*}
נגדיר $\epsilon = \frac{1}{4}, N = 1$.
אם $L \ge 1$ אז $|a_2 - L| > \epsilon$ והתנאי מתקיים.
אילו $L \le 0$ אז $|a_3 - L| < \epsilon$ והתנאי מתקיים גם כן. \\*
כאשר $0 < L < \frac{3}{4}$ נגדיר $\epsilon = a_2 - L$ ונראה כי התנאי מתקיים שכן מספר זה מקיים את התנאי $\epsilon > 0$.
כאשר $\frac{3}{4} \le L < 1$ אז נגדיר את $\epsilon = L - a_3$ ובהתאם התנאי עדיין יתקיים.
בסך־הכול ראינו כי הסדרה מתבדרת לכל $L \in \RR$ ולכן מתבדרת לפי הגדרה ב'(ii).

\section{שאלה 2}
נחשב את הגבולות הבאים, או נוכיח שאינם קיימים:

\subsection{סעיף א'}
\begin{align*}
	\lim_{n \to \infty} \sqrt{n^2 + {(-1)}^n} - n
	& = \lim_{n \to \infty} \frac{\left(\sqrt{n^2 + {(-1)}^n} + n\right)\left(\sqrt{n^2 + {(-1)}^n} - n\right)}{\sqrt{n^2 + {(-1)}^n} + n} \\
	& = \lim_{n \to \infty} \frac{n^2 + {(-1)}^n - n^2}{\sqrt{n^2 + {(-1)}^n} + n} \\
	& = \lim_{n \to \infty} \frac{{(-1)}^n}{\sqrt{n^2 + {(-1)}^n} + n} \\
\end{align*}
נוכל לראות כי מכנה הגבול חיובי וגדול מ־$2n - 1$ כמעט לכל $n$ ולכן המכנה שואף ל־$\infty$, כאשר המונה חסום ב־$1$,
לכן לפי משפט 2.22 והאריתמטיקה של הגבולות לקבוע כי
\[
	\lim_{n \to \infty} \sqrt{n^2 + {(-1)}^n} - n = 0
\]

\subsection{סעיף ב'}
\begin{align*}
	\lim_{n \to \infty} \frac{3n^3 - 2n^6 - 1}{n^4 - \pi n^5 + 5n}
	& = \lim_{n \to \infty} \frac{3n^3/n^6 - 2n^6/n^6 - 1/n^6}{n^4/n^6 - \pi n^5/n^6 + 5n/n^6} \\
	& = \lim_{n \to \infty} \frac{3/n^3 - 2 - 1/n^6}{1/n^2 - \pi /n + 5/n^5} \\
	& = \frac{-2}{0^+} & \text{על־פי אריתמטיקה של הגבולות} \\
	& = -\infty
\end{align*}
לכן הסדרה מתכנסת ל־$-\infty$ במובן הרחב.

\subsection{סעיף ג'}
\begin{align*}
	& \frac{\sqrt{3} n^2 - 1}{n^4} \le \frac{\lfloor \sqrt{3} n^2 \rfloor}{n^4} \le \frac{\sqrt{3} n^2}{n^4} \\
	& \frac{\sqrt{3}/n^2 - 1/n^4}{1} \le \frac{\lfloor \sqrt{3} n^2 \rfloor}{n^4} \le \frac{\sqrt{3}}{n^2} \\
\end{align*}
נראה כי גם
\[
	\lim_{n \to \infty} \left(\frac{\sqrt{3}}{n^2} - \frac{1}{n^4} \right)
	=
	\lim_{n \to \infty} \frac{\sqrt{3}}{n^2}
	= 0
\]
לכן לפי כלל הסנדוויץ' מתקיים גם
\[
	\lim_{n \to \infty} \frac{\lfloor \sqrt{3} n^2 \rfloor}{n^4} = 0
\]

\subsection{סעיף ד'}
לפי אי־שוויון הממוצעים
\[
	(1) \sqrt[n]{\frac{\displaystyle \prod_{i = 1}^{n} 2n - 1}{\displaystyle \prod_{i = 1}^{n} 2n }}
	= \sqrt[n]{\displaystyle \prod_{i = 1}^{n} \frac{2n - 1}{2n}}
	\le \frac{\displaystyle \sum_{i = 1}^{n} \frac{2n - 1}{2n}}{n}
	= \displaystyle \sum_{i = 1}^{n} \frac{2n - 1}{2n \cdot n}
	\le \displaystyle n \cdot \frac{2n - 1}{2n \cdot n}
	= \frac{2n - 1}{2n} 
	= 1 - \frac{1}{2n} 
לפי אי־שוויון הממוצעים
\]
נשים לב כי $(1)$ הוא שורש של מספר גדול מ־1 ולכן נוכל להניח כי הוא גדול שווה ל־1.
עוד נראה כי
\[
	\lim_{n \to \infty} 1 = \lim_{n \to \infty} \left( 1 - \frac{1}{2n} \right) = 1
\]
לכן לפי כלל הסנדוויץ' סדרה $(1)$ מתכנסת ל־1.

\section{שאלה 3}
יהיו $(a_n)$ ו־$(b_n)$ סדרות כך שמתקיים $(1) \lim_{n \to \infty} a_n b_n = \infty$.

\subsection{סעיף א'}
נוכיח כי אם כמעט כל אברי $(a_n)$ חיוביים, אז כמעט כל אברי $(b_n)$ חיוביים.
נניח בשלילה כי כמעט כל אברי $(b_n)$ הם שליליים. קיים $N \in \NN$
כך שלכל $n > N$ מתקיים $a_n > 0$ ו־$b_n < 0$, לכן $a_n b_n < 0$,
אבל ידוע כי לכל $n > N$ גם מתקיים $a_n b_n > 0$ על־פי $(1)$.
זוהי סתירה ולכןכמעט כל אברי $(b_n)$ מקיימים $b_n > 0$.
נראה באופן דומה כי לא יתכן ש־$b_n = 0$ כמעט לכל $n$.
במצב זה הסדרה תתכנס ל־$0$, ונתון כי זהו לא המצב,
לכן אם כמעט כל איבר ב־$(a_n)$ חיובי, אז גם כמעט כל איבר ב־$(b_n)$ חיובי.

\subsection{סעיף ב'}
נראה כי אם הסדרות $(a_n)$ ו־$(b_n)$ חיוביות,
אז לא בהכרח לפחות אחת מהן מתכנסת. \\*
נגדיר
\[
	a_n = \begin{cases}
		1 & \text{ אי־זוגי} n \\
		3 & \text{ זוגי} n
	\end{cases}, b_n = \frac{1}{a_n}
\]
ניתן לראות כי לכל $n$ מתקיים $a_n b_n = 1$,
ולכן מכפלת הסדרות מתכנסת ל־$1$.
לעומת זאת, הסדרות $(a_n)$ ו־$(b_n)$ שתיהן מתבדרות: \\*
לכל $n$ מתקיים $a_n = {(-1)}^n + 2$,
ועל־פי טענה 2.14 סדרה זו היא הזזה של סדרה מתבדרת, ומתבדרת בעצמה.
הסדרה $(b_n)$ מוגדרת לפי $a_n$ לכל $n$, ומתבדרת גם היא.

\subsection{סעיף ג'}
נוכיח כי אם $\lim_{n \to \infty} b_n = \infty$ אז $\lim_{n \to \infty} = 0$.
על־פי הגדרת שאיפה לאינסוף, כמעט לכל $n$ מתקיים $b_n > 0$,
לכן בגבול
\[
	\lim_{n \to \infty} \frac{a_n b_n}{b_n}
	= \lim_{n \to \infty} a_n
\]
מוגדר וערכו לפי האריתמטיקה של הגבולות האינסופיים היא $0$.

\subsection{סעיף ד'}
נראה כי אם $\lim_{n \to \infty} a_n = 0$ אז לא תמיד
$\lim_{n \to \infty} b_n = \infty$.
נגדיר
\[
	a_n = \frac{1}{-n}, b_n = -n
\]
לכן $a_n b_n = 1$ לכל $n$
ו־$\lim_{n \to \infty} a_n = 0$
אבל $\lim_{n \to \infty} b_n = -\infty$.

\subsection{סעיף ה'}
נוכיח כי אם $(a_n)$ סדרה חיובית, אז קיים $N \in \NN$ כך שלכל $n > N$
מתקיים $b_n > \frac{1}{2a_n}$. \\*
על־פי הגבול $(1)$ כמעט לכל $n$ מתקיים
\[
	\left| a_n b_n - 1 \right| < \frac{1}{2}
\]
לכן גם
\[
	-\frac{1}{2} < a_n b_n - 1 < \frac{1}{2}
	\rightarrow
	1 - \frac{1}{2} < a_n b_n < 1 + \frac{1}{2}
\]
לכן בפרט
\[
	a_n b_n > \frac{1}{2}
\]
ידוע כי $(a_n)$ חיובית ולכן נחלק בערכה:
\[
	b_n > \frac{1}{2 a_n}
\]

\subsection{סעיף ו'}
נוכיח כי אם $(a_n)$ חיובית ואפסה אז $\lim_{n \to \infty} b_n = \infty$. \\*
בדומה לסעיף ג' מתקיים
\[
	\lim_{n \to \infty} b_n
	= \lim_{n \to \infty} \frac{a_n b_n}{a_n}
	= \frac{\lim_{n \to \infty} a_n b_n}{\lim_{n \to \infty} a_n}
	= \frac{1}{0^+}
	= \infty
\]
על־פי האריתמטיקה של הגבולות האינסופיים.

\subsection{סעיף ז'}
נוכיח כי אם $\lim_{n \to \infty} |a_n| = 1$
אז $\lim_{n \to \infty} |b_n| = 1$. \\*
בהכפלת הגבול $(1)$ בעצמו נקבל
\[
	\lim_{n \to \infty} a_n^2 b_n^2 = 1^2 = 1
\]
ידוע כי $\lim_{n \to \infty} a_n^2 = 1$, לכן
\[
	\lim_{n \to \infty} \frac{a_n^2 b_n^2}{a_n^2} = \frac{1}{1}
	\rightarrow
	\lim_{n \to \infty} b_n^2 = 1
\]
לכן לכל $\epsilon > 0$ כמעט לכל $n$ מתקיים
\[
	|b_n^2 - 1| < \epsilon
\]
לפי נוסחת הכפל המקוצר מתקיים
\[
	\left| \left| b_n \right| - 1 \right|
	\left| \left| b_n \right| + 1 \right| < \epsilon
\]
הביטוי $\left| \left| b_n \right| + 1 \right| > 1$ לכל $n$, לכן
\[
	\left| \left| b_n \right| - 1 \right| <
	\left| \left| b_n \right| - 1 \right|
	\left| \left| b_n \right| + 1 \right| < \epsilon
\]
לכן בפרט
\[
	\left| \left| b_n \right| - 1 \right| < \epsilon
\]
לכן
\[
	\lim_{n \to \infty} |b_n| = 1
\]
\end{document}
