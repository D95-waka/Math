\documentclass[a4paper]{article}

% packages
\usepackage{inputenc, fontspec, amsmath, amsthm, amsfonts, polyglossia, catchfile}
\usepackage[a4paper, margin=50pt, includeheadfoot]{geometry} % set page margins

% style
\AddToHook{cmd/section/before}{\clearpage}	% Add line break before section
\linespread{1.5}
\setcounter{secnumdepth}{0}		% Remove default number tags from sections
\setmainfont{Libertinus Serif}
\setsansfont{Libertinus Sans}
\setmonofont{Libertinus Mono}
\setdefaultlanguage{hebrew}
\setotherlanguage{english}

% operators
\DeclareMathOperator\cis{cis}
\DeclareMathOperator\Sp{Sp}
\DeclareMathOperator\tr{tr}
\DeclareMathOperator\im{Im}
\DeclareMathOperator\diag{diag}
\DeclareMathOperator*\lowlim{\underline{lim}}
\DeclareMathOperator*\uplim{\overline{lim}}

% commands
\renewcommand\qedsymbol{\textbf{משל}}
\newcommand{\NN}[0]{\mathbb{N}}
\newcommand{\ZZ}[0]{\mathbb{Z}}
\newcommand{\QQ}[0]{\mathbb{Q}}
\newcommand{\RR}[0]{\mathbb{R}}
\newcommand{\CC}[0]{\mathbb{C}}
\newcommand{\getenv}[2][] {
  \CatchFileEdef{\temp}{"|kpsewhich --var-value #2"}{\endlinechar=-1}
  \if\relax\detokenize{#1}\relax\temp\else\let#1\temp\fi
}
\newcommand{\explain}[2] {
	\begin{flalign*}
		 && \text{#2} && \text{#1}
	\end{flalign*}
}

% headers
\getenv[\AUTHOR]{AUTHOR}
\author{\AUTHOR}
\date\today

\title{פתרון ממ''ן 12 – חשבון אינפיניטסימלי 1 (20474)}

\begin{document}
\maketitle
\section{שאלה 1}
\subsection{סעיף א'}
\explain{נוכיח בלשון $\epsilon, N$ כי מתקיים}{$\lim_{n \to \infty} \sqrt{\frac{4n + 1}{n}} = 2$}
\explain{נשים לב תחילה כי מתקיים}{
$
	\frac{4n + 1}{n}
	= \frac{\frac{4n}{n} + \frac{1}{n}}{\frac{n}{n}}
	= \frac{4 + \frac{1}{n}}{1}
	= 4 + \frac{1}{n}
$}
לכל $n \ne 0$. לכן נוכל להוכיח את קיום הגבול
$$
	\lim_{n \to \infty} \sqrt{4 + \frac{1}{n}} = 2
$$
נגדיר $\epsilon > 0$ מספר ממשי. על־פי הגדרת הגבול צריך להתקיים:
$$
	\left| \sqrt{4 + \frac{1}{n}} - 2 \right| < \epsilon
$$
תוכן השורש הוא תמיד לפחות 4, ולכן תוצאתו תמיד גדולה מ־2, בהתאם תוכן הערך המוחלט חיובי תמיד ומתקיים:
$$
	\left| \sqrt{4 + \frac{1}{n}} - 2 \right|
	= \sqrt{4 + \frac{1}{n}} - 2 < \epsilon
$$
לכל $n > N$ כאשר $N \in \NN$. נשים לב כי
$$
	{\left(2 + \sqrt{\frac{1}{n}}\right)}^2 = 4 + 2\sqrt{\frac{1}{n}} + \frac{1}{n} > 4 + \frac{1}{n} = {\left(\sqrt{4 + \frac{1}{n}}\right)}^2
	\rightarrow \sqrt{4 + \frac{1}{n}} < 2 + \sqrt{\frac{1}{n}}
$$
\explain{נקבע} {$\sqrt{4 + \frac{1}{n}} - 2 < 2 + \sqrt{\frac{1}{n}} - 2 = \frac{1}{\sqrt{n}} < \epsilon$ }
\explain{נגדיר}{ $N = \left\lceil \frac{1}{\epsilon^2} \right\rceil$ }
במצב זה הגבול $(1)$ מתקיים לכל $\epsilon > 0$.

\subsection{סעיף ב'}
\textbf{(i)}
יהיו $(a_n)$ סדרה ו־$L$ מספר ממשי. ננסח בלשון $\epsilon, N$ את הטענה
$$
	\lim_{n \to \infty} a_n \ne L
$$
תחילה נצרין את הטענה:
$$
	\lnot \forall \epsilon > 0 (
	\exists N \in \NN (
	\forall n > N (\left| a_n - L \right| < \epsilon)))
$$
נפשט את הפסוק:
$$
	\exists \epsilon > 0 (
	\forall N \in \NN (
	\exists n > N (\left| a_n - L \right| \ge \epsilon)))
$$
ננסח פסוק זה במילים: \\*
הטענה $\lim_{n \to \infty} a_n \ne L$ מתקיימת אם ורק אם
קיים $\epsilon > 0$ עבורו לכל $N$ טבעי קיים $n > N$
כך שמתקיים $\left| a_n - L \right| \ge \epsilon$. \\
\textbf{(ii)}
על־פי הגדרת התבדרות (2.13) סדרה נקראת מתבדרת כאשר אין מספר $L$
שעבורו מתקיים הגבול $\lim_{n \to \infty} a_n = L$.
ננסח את הטענה בעזרת הגדרת התכנסות בלשון $\epsilon, N$: \\*
סדרה $(a_n)$ היא מתבדרת אם ורק אם
לכל $L$ ממשי קיים $\epsilon > 0$ עבורו לכל $N$ טבעי קיים $n > N$
כך שמתקיים $\left| a_n - L \right| \ge \epsilon$.

\subsection{סעיף ג'}
נוכיח שהסדרה $(a_n)$ מתבדרת בלשון $\epsilon, N$, כאשר
$$
	a_n = \frac{{(-1)}^n n + 1}{n + 2}
$$
נוכיח כי לכל $L$ ממשי שנבחר קיים $\epsilon > 0$ שעבורו לכל $N$ טבעי קיים $n > N$ שעבורו $\left| a_n - L \right| \ge \epsilon$. \\*
מתקיים על־פי חישוב ישיר
\begin{align*}
	& n = 2 \rightarrow a_n = \frac{3}{4} \\
	& n = 3 \rightarrow a_n = -\frac{2}{5} \\
\end{align*}
נגדיר $\epsilon = \frac{1}{4}, N = 1$.
אם $L \ge 1$ אז $|a_2 - L| > \epsilon$ והתנאי מתקיים.
אילו $L \le 0$ אז $|a_3 - L| < \epsilon$ והתנאי מתקיים גם כן. \\*
כאשר $0 < L < \frac{3}{4}$ נגדיר $\epsilon = a_2 - L$ ונראה כי התנאי מתקיים שכן מספר זה מקיים את התנאי $\epsilon > 0$.
כאשר $\frac{3}{4} \le L < 1$ אז נגדיר את $\epsilon = L - a_3$ ובהתאם התנאי עדיין יתקיים.
בסך־הכול ראינו כי הסדרה מתבדרת לכל $L \in \RR$ ולכן מתבדרת לפי הגדרה ב'(ii).

\section{שאלה 2}
נחשב את הגבולות הבאים, או נוכיח שאינם קיימים:

\subsection{סעיף א'}
\begin{align*}
	\lim_{n \to \infty} \sqrt{n^2 + {(-1)}^n} - n
	& = \lim_{n \to \infty} \frac{\left(\sqrt{n^2 + {(-1)}^n} + n\right)\left(\sqrt{n^2 + {(-1)}^n} - n\right)}{\sqrt{n^2 + {(-1)}^n} + n} \\
	& = \lim_{n \to \infty} \frac{n^2 + {(-1)}^n - n^2}{\sqrt{n^2 + {(-1)}^n} + n} \\
	& = \lim_{n \to \infty} \frac{{(-1)}^n}{\sqrt{n^2 + {(-1)}^n} + n} \\
\end{align*}
נוכל לראות כי מכנה הגבול חיובי וגדול מ־$2n - 1$ כמעט לכל $n$ ולכן המכנה שואף ל־$\infty$, כאשר המונה חסום ב־$1$,
לכן לפי משפט 2.22 והאריתמטיקה של הגבולות לקבוע כי
$$
	\lim_{n \to \infty} \sqrt{n^2 + {(-1)}^n} - n = 0
$$

\subsection{סעיף ב'}
\begin{align*}
	\lim_{n \to \infty} \frac{3n^3 - 2n^6 - 1}{n^4 - \pi n^5 + 5n}
	& = \lim_{n \to \infty} \frac{3n^3/n^6 - 2n^6/n^6 - 1/n^6}{n^4/n^6 - \pi n^5/n^6 + 5n/n^6} \\
	& = \lim_{n \to \infty} \frac{3/n^3 - 2 - 1/n^6}{1/n^2 - \pi /n + 5/n^5} \\
	& = \frac{-2}{0^+} & \text{על־פי אריתמטיקה של הגבולות} \\
	& = -\infty
\end{align*}
לכן הסדרה מתכנסת ל־$-\infty$ במובן הרחב.

\subsection{סעיף ג'}
\begin{align*}
	& \frac{\sqrt{3} n^2 - 1}{n^4} \le \frac{\lfloor \sqrt{3} n^2 \rfloor}{n^4} \le \frac{\sqrt{3} n^2}{n^4} \\
	& \frac{\sqrt{3}/n^2 - 1/n^4}{1} \le \frac{\lfloor \sqrt{3} n^2 \rfloor}{n^4} \le \frac{\sqrt{3}}{n^2} \\
\end{align*}
נראה כי גם
$$
	\lim_{n \to \infty} \left(\frac{\sqrt{3}}{n^2} - \frac{1}{n^4} \right)
	=
	\lim_{n \to \infty} \frac{\sqrt{3}}{n^2}
	= 0
$$
לכן לפי כלל הסנדוויץ' מתקיים גם
$$
	\lim_{n \to \infty} \frac{\lfloor \sqrt{3} n^2 \rfloor}{n^4} = 0
$$

\subsection{סעיף ד'}
לפי אי־שוויון הממוצעים
$$
	(1) \sqrt[n]{\frac{\displaystyle \prod_{i = 1}^{n} 2n - 1}{\displaystyle \prod_{i = 1}^{n} 2n }}
	= \sqrt[n]{\displaystyle \prod_{i = 1}^{n} \frac{2n - 1}{2n}}
	\le \frac{\displaystyle \sum_{i = 1}^{n} \frac{2n - 1}{2n}}{n}
	= \displaystyle \sum_{i = 1}^{n} \frac{2n - 1}{2n \cdot n}
	\le \displaystyle n \cdot \frac{2n - 1}{2n \cdot n}
	= \frac{2n - 1}{2n} 
	= 1 - \frac{1}{2n} 
לפי אי־שוויון הממוצעים
$$
נשים לב כי $(1)$ הוא שורש של מספר גדול מ־1 ולכן נוכל להניח כי הוא גדול שווה ל־1.
עוד נראה כי
$$
	\lim_{n \to \infty} 1 = \lim_{n \to \infty} \left( 1 - \frac{1}{2n} \right) = 1
$$
לכן לפי כלל הסנדוויץ' סדרה $(1)$ מתכנסת ל־1.

\section{שאלה 3}
יהיו $(a_n)$ ו־$(b_n)$ סדרות כך שמתקיים $(1) \lim_{n \to \infty} a_n b_n = \infty$.

\subsection{סעיף א'}
נפריך את הטענה כי אם כמעט כל אברי $(a_n)$ ו־$(b_n)$ חיוביים אז אחת משתי הסדרות מתכנסת לאינסוף לפחות על־ידי דוגמה נגדית. \\*
נגדיר
$$
	a_n = \begin{cases}
		1 & \text{ זוגי} n \\
		n & \text{ אי־זוגי} n \\
	\end{cases},
	b_n = \begin{cases}
		n & \text{ זוגי} n \\
		1 & \text{ אי־זוגי} n \\
	\end{cases}
$$
ניתן לראות כי מכפלת הסדרות היא הסדרה $c_n = n$ אשר מקיימת את תנאי $(1)$, אך שתי הסדרות מתבדרות.

\subsection{סעיף ב'}
נוכיח כי אם כמעט כל אברי $(b_n)$ חיוביים אז גם כמעט כל אברי $(a_n)$ חיוביים. \\*
נגדיר כי $(b_n)$ חיובית לכל $n$ לפי משפט 2.44 בלא פגיעה בכלליות ההוכחה.
בשל גבול $(1)$ ידוע כי לכל $M \in \RR$ קיים $N \in \NN$ כך שלכל $n > N$ מתקיים $a_n b_n > M$.
ידוע כי $b_n$ חיובי וכי $M$ חיובי, אז גם $a_n$ חיובי אף הוא, שאם לא כן לא יתקיים השוויון $a_n b_n > M$.
לכן $a_n$ חיובי לכל $n$.

\subsection{סעיף ג'}
נפריך את הטענה כי $\lim_{n \to \infty} b_n \ne 0$ בעזרת דוגמה נגדית. \\*
נגדיר
$$
	a_n = n^3,
	b_n = \frac{1}{n}
$$
לכן
$$
	a_n b_n = n^2
$$
ובהתאם תנאי $(1)$ מתקיים. למרות זאת, לפי טענה 2.10 מתקיים $\lim_{n \to \infty} b_n = 0$ בסתירה לטענה.

\subsection{סעיף ד'}
נוכיח כי קיים $N \in \NN$ כך שלכל $n > N$ מתקיים $b_n \ne 0$. \\*
הוכחת טענה זו דומה ביותר להוכחת סעיף ב', אנו רואים כי $a_n b_n > M$, ולכן נוכל להניח גם $a_n b_n > 0$, לכן $a_n \ne 0 \land b_n \ne 0$.

\subsection{סעיף ה'}
נוכיח כי אם $\lim_{n \to \infty} b_n = 5$, אז $\lim_{n \to \infty} a_n = \infty$. \\*
נניח כי הגבול הראשון מתקיים והגבול השני לא בשלילה.
אילו $a_n$ מתבדרת אז מכפלת הסדרות תתבדר גם היא, בניגוד ל־$(1)$, לכן נניח ש־$(a_n)$ מתכנסת.
אילו הייתה הסדרה מתכנסת למספר סופי, אז לפי האריתמטיקה של הגבולות $a_n b_n$ הייתה מתכנסת אף היא למספר סופי בניגוד ל־$(1)$,
לכן $(a_n)$ מתכנסת ל־$\pm \infty$. אנו יודעים כי $(b_n)$ חיובית לכמעט כל $n$, אילו הייתה $(a_n)$ מתכנסת ל־$-\infty$ אז היא הייתה שלילית לכמעט כל $n$.
במקרה זה, מכפלת הסדרות הייתה שלילית לכמעט כל $n$, אך אנו יודעים כי היא חיובית על־פי גבול $(1)$ ולכן בסך־הכול $\lim_{n \to \infty} a_n = \infty$.

\subsection{סעיף ו'}
נפריך את הטענה כי אם $b_n < a_n$ כמעט לכל $n$ אז $\lim_{n \to \infty} a_n = \infty$ בעזרת דוגמה נגדית: \\*
נגדיר
$$
	a_n = -n, b_n = -n - 1, a_n b_n = n (n + 1)
$$
במקרה זה גבול $(1)$ מתקיים ואף $b_n < a_n$ לכל $n$ אך $\lim_{n \to \infty} a_n = -\infty$ בסתירה לטענה.

\subsection{סעיף ז'}
נוכיח כי אם $0 < b_n < a_n$ כמעט לכל $n$ אז $\lim_{n \to \infty} a_n = \infty$. \\*
על־פי הגדרת שאיפה לאינסוף, לכל $M \in \RR$ קיים $N \in \NN$ כך שלכל $n > N$ מתקיים $a_n b_n > M$.
לכן מתקיים גם $a_n a_n > a_n b_n > M$, אז $a_n > \sqrt{M}$. נגדיר $M' = \sqrt{M}$ ואנו רואים כי הגדרת השאיפה חלה על $(a_n)$.
לכן
$$
	\lim_{n \to \infty} a_n = \infty
$$
\end{document}
