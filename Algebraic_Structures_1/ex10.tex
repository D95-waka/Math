\documentclass[a4paper]{article}

% packages
\usepackage{inputenc, fontspec, amsmath, amsthm, amsfonts, polyglossia, catchfile}
\usepackage[a4paper, margin=50pt, includeheadfoot]{geometry} % set page margins

% style
\AddToHook{cmd/section/before}{\clearpage}	% Add line break before section
\linespread{1.5}
\setcounter{secnumdepth}{0}		% Remove default number tags from sections
\setmainfont{Libertinus Serif}
\setsansfont{Libertinus Sans}
\setmonofont{Libertinus Mono}
\setdefaultlanguage{hebrew}
\setotherlanguage{english}

% operators
\DeclareMathOperator\cis{cis}
\DeclareMathOperator\Sp{Sp}
\DeclareMathOperator\tr{tr}
\DeclareMathOperator\im{Im}
\DeclareMathOperator\diag{diag}
\DeclareMathOperator*\lowlim{\underline{lim}}
\DeclareMathOperator*\uplim{\overline{lim}}

% commands
\renewcommand\qedsymbol{\textbf{משל}}
\newcommand{\NN}[0]{\mathbb{N}}
\newcommand{\ZZ}[0]{\mathbb{Z}}
\newcommand{\QQ}[0]{\mathbb{Q}}
\newcommand{\RR}[0]{\mathbb{R}}
\newcommand{\CC}[0]{\mathbb{C}}
\newcommand{\getenv}[2][] {
  \CatchFileEdef{\temp}{"|kpsewhich --var-value #2"}{\endlinechar=-1}
  \if\relax\detokenize{#1}\relax\temp\else\let#1\temp\fi
}
\newcommand{\explain}[2] {
	\begin{flalign*}
		 && \text{#2} && \text{#1}
	\end{flalign*}
}

% headers
\getenv[\AUTHOR]{AUTHOR}
\author{\AUTHOR}
\date\today

\title{פתרון מטלה 10 --- מבנים אלגבריים 1 (80445)}

\begin{document}
\maketitle
\maketitleprint{}

\Question{}
יהי $\FF$ שדה, נוכיח שהחוג $R = M_n(\FF)$ לא מכיל אידאלים לא טריוויאליים.
\begin{proof}
	נניח כי קיים $\{ 0 \} \ne I \triangleleft R$.
	לכן קיימת מטריצה $M \in I$.
	אילו מטריצה זו הפיכה אז $M^{-1} \in R$ ובהתאם $M M^{-1} = I_n \in I$ ולכן נוכל להסיק כי $I = R$, ולכן נניח כי $M$ לא הפיכה.
	מלינארית נסיק כי קיימת מטריצה ממטריצות הבסיס הסטנדרטי שלא מאפסת את $M$ (אחרת $M = 0$ בסתירה לטענה).
	נקבל אם כן שקיימת מטריצה מהבסיס הסטנדרטי ב־$I$, ועל־ידי שימוש במטריצות שינוי שורה ועמודה (שפגשנו בתרגול) נוכל לבנות את כלל מטריצת היחידה, דהינו $I_n \in I$ וקיבלנו שוב כי $I = R$. \\*
	מצאנו אם כן כי יש רק אידאלים טריוויאליים ל־$R$.
\end{proof}

\Question{}
יהי חוג $R$ ו־$I \triangleleft R$.

\Subquestion{}
נוכיח כי לכל חבורה חיבורית $J \le R$ כך ש־$I \subseteq J$, תת־החבורה $\overline{J} \le R / I$ היא אידאל אם ורק אם $J \triangleleft R$.
\begin{proof}
	נניח ש־$J \triangleleft R$ ונוכיח ש־$\overline{J} \triangleleft R / I$.
	למעשה, זו כבר חבורה חיבורית ולכן מספיק להראות את תכונת הסגירות לכפל בחוג.
	יהיו $a + I \in \overline{J}, b + I \in R / I$, נראה כי $(a + I)(b + I) = ab + I$ הוא האיבר המתקבל מההטלה הקאנונית $\pi : R \to R / I$ מ־$a \in J, b \in R$, ולכן $ab \in J$ ובהתאם $ab + I \in \overline{J}$.
	נסיק אם כן ש־$\overline{J} \triangleleft R / I$.

	הכיוון ההפוך זהה לחלוטין, נשתמש בהטלה הקאנונית ובמשפט ההתאמה.
\end{proof}

\Subquestion{}
יהי תת־חוג $S \subseteq R$, נוכיח ש־$I + S$ הוא תת־חוג של $R$, ש־$S \cap I \triangleleft S$ ושההומומורפיזם של החבורות החיבוריות המתקבל ממשפט האיזומורפיזם השני לחבורות
\[
	S / (S \cap I) \simeq (S + I) / I
\]
הוא איזומורפיזם של חוגים.
\begin{proof}
	נראה ש־$S + I$ תת־חוג. \\*
	זוהי כמובן תת־חבורה חיבורית כפי שראינו בעבר, ולכן עלינו רק לבדוק את הסגירות לכפל וקיום יחידה. \\*
	נבחר $0 \in S, 1 \in I$ ולכן נקבל $0 + 1 = 1 \in S + I$. \\*
	נראה שהקבוצה סגורה לכפל. למעשה, $S$ סגורה וכפל וכך גם $I$ ולכן נותר לבדוק את המקרה $ab$ כאשר $a \in I, b \in S$. \\*
	אנו יודעים כי $I$ אידאל וגם כי $b \in S \in R$ ולכן $ab \in I$ ונסיק כי $S + I$ סגורה לכפל ולכן תת־חוג.

	נראה ש־$S \cap I \triangleleft S$. \\*
	חיתוך של תת־חבורות חיבוריות קומוטטיביות הוא חבורה חיבורית כפי שראינו בעבר, $0, 1 \in S \cap I$ כפי שגם ראינו כבר, ולכן נשאר לבדוק סגירות לכפל באיבר מ־$S$. \\*
	יהיו $a \in S \cap I, b \in S$. $a, b \in S$ ולכן $ab \in S$, ו־$I$ אידאל ו־$a \in R$ ולכן $ab \in I$ ונסיק כי $ab \in S \cap I$ ומצאנו כי $S \cap I \triangleleft S$.

	נגדיר הומומורפיזם $\varphi : S / (S \cap I) \to (S + I) / I$ על־פי משפט ההומוורפיזם השני לחבורות, ונבדוק אם הוא משמר כפל. \\*
	יהיו $a + S \cap I, b + S \cap I \in S / (S \cap I)$,
	אז $\varphi(a + S \cap I) = a + I$ ובהתאם גם $\varphi(b + S \cap I) = b + I$ ונקבל $\varphi(ab + S \cap I) = ab + I$ וגם $\varphi(a + S \cap I)\varphi(b + S \cap I) = (a + I)(b + I) = ab + I$. \\*
	מצאנו אם כן ש־$\varphi$ הומומורפיזם של חוגים, ואיזומורפיזם מבדיקה ישירה של הופכיות.
\end{proof}

\Question{}
יהי $R$ חוג קומוטטיבי.

\Subquestion{}
נוכיח ש־$R$ הוא שדה אם ורק אם יש לו רק את שני האידאלים הטריוויאליים בלבד, וכי הם שונים.
\begin{proof}
	נניח ש־$R$ שדה ולכן לכל איבר יש הופכי (מלבד אפס), יהי $I \triangleleft R$, ויהי $x \in I$, אז $x^{-1} \in R$ קיים, ובהתאם $x x^{-1} = 1 \in I$ ובהתאם $I = R$ ומצאנו כי קיימים שני אידאלים בלבד.

	נניח מצד שני כי $R$ חוג קומוטטיבי עם שני האידאלים בלבד.
	אילו נניח כי קיים $x \in R$ שאין לו הופכי נקבל כי $(x) \triangleleft R$ ו־$(x) \ne R$, בסתירה להנחה כי אין אידאלים לא טריוויאליים, ולכן נוכל להסיק כי לכל איבר יש הופכי, וכי $0, 1 \in R$ וגם $0 \ne 1$.
	נקבל אם כן ש־$R$ שדה.
\end{proof}

\Subquestion{}
יהיו $I, J \triangleleft R$ כך ש־$I + J = R$, נגדיר
\[
	\pi : R \to R / I \times R / J
\]
המוגדר על־ידי $\pi(x) = (x + I, x + J)$ הוא על.
\begin{proof}
	יהי $x \in R$, מהנתון $I + J = R$ נסיק כי קיימים $y \in I, z \in J$ כך ש־$x = y + z$, ובהתאם $\pi(x) = (y + I, z + J)$. \\*
	נוכל אם כן לבחור $(y + I, z + J)$ כלשהם, ולקבל $y + z \in R$ וכי קיים $x \in R$ עבורו $x = y + z$ ומכאן נסיק שההעתקה היא על.
\end{proof}

\Subquestion{}
נוכיח שלכל $a_1, \dots, a_n \in R$ קיים איזומורפיזם
\[
	R[x_1, \dots, x_n] / (x_1 - a_1, \dots, x_n - a_n) \simeq R
\]
\begin{proof}
	נשתמש בתכונה של פולינומים ש־$x - a = (x - a_1) + (a_1 - a)$, ו־$x^2 - a = x(x - a) + a(x - a) + (a^2 - a)$ ופירוק דומה לכל מעלה כדי להסיק שכל איבר בחוג הנתון מתאפס פרט לאיבר החופשי. \\*
	נוכל כמובן להעביר תהליך זהה עבור איברים מעורבים מהצורה $k x_i x_j$ על־ידי קיבוע $x_i$ וביצוע פירוק דומה לפירוק שהוצג זה עתה, ונקבל כי האיבר החופשי היחיד שלא מתאפס ובהתאם נקבל את האיזומורפיה.
\end{proof}

\Question{}
יהי $R$ תחום שלמות, ונגדיר את שדה הרציונליים על $R$ על־ידי מחלקות השקילות
\[
	Q(R) := \{ (a, b) \mid a \in R, 0 \ne b \in R \} / \sim
\]
כאשר $(a, b) \sim (c, d) \iff ad = bc$, נגדיר גם
\[
	(a, b) + (c, d) = (ad + bc, bd),
	\qquad
	(a, b) \cdot (c, d) = (ac, bd)
\]
ולבסוף נגדיר את $\iota : R \to Q(R)$ על־ידי $\iota(r) = (r, 1)$.

\Subquestion{}
נוכיח כי $\sim$ יחס שקילות על $R \times (R \setminus \{ 0 \})$.
\begin{proof}
	הוכח במסגרת הקורס מבוא לתורת הקבוצות, שאלה חמש בקובץ הבא: \href{https://github.com/D95-waka/Math/blob/master/Set_Theory/bin/ex03.pdf}{מטלה 3}
\end{proof}

\Subquestion{}
נוכיח כי הפעולות $+, \cdot$ מוגדרות היטב על $Q(R)$.
\begin{proof}
	הוכח במסגרת הקורס מבוא לתורת הקבוצות, שאלה חמש בקובץ הבא: \href{https://github.com/D95-waka/Math/blob/master/Set_Theory/bin/ex03.pdf}{מטלה 3}
\end{proof}

\Subquestion{}
נוכיח כי $(Q(R), +, \cdot, (0, 1), (1, 1))$ הוא שדה.
\begin{proof}
	אנו יודעים כי $R$ תחום שלמות ולכן נוכל להסיק שגם $Q(R)$, ולכן עלינו רק לבדוק סגירות להופכי. \\*
	למעשה, אנו כבר יודעים כי לכל $(a, b) \in Q(R)$ כאשר $b \ne 0$ מתקיים $(a, b) \cdot (b, a) = (1, 1)$ ולכן מצאנו הופכי ונסיק ש־$Q(R)$ שדה.
\end{proof}

\Subquestion{}
נוכיח כי $\iota : R \to Q(R)$ היא הומומורפיזם חוגים חד־חד ערכי.
\begin{proof}
	נראה
	\[
		\forall a, b \in R : \iota(a + b) = (a + b, 1) = (a, 1) + (b, 1) = \iota(a) + \iota(b)
	\]
	באופן דומה נקבל
	\[
		\forall a, b \in R : \iota(a \cdot b) = (a \cdot b, 1) = (a, 1) \cdot (b, 1) = \iota(a) \cdot \iota(b)
	\]
	ונשאר לבדוק חד־חד ערכיות.
	נניח ש־$a \ne b$ הפעם, ונקבל $\iota(a) = (a, 1) \ne (b, 1) = \iota(b)$.
\end{proof}

\end{document}
