\documentclass[a4paper]{article}

% packages
\usepackage{inputenc, fontspec, amsmath, amsthm, amsfonts, polyglossia, catchfile}
\usepackage[a4paper, margin=50pt, includeheadfoot]{geometry} % set page margins

% style
\AddToHook{cmd/section/before}{\clearpage}	% Add line break before section
\linespread{1.5}
\setcounter{secnumdepth}{0}		% Remove default number tags from sections
\setmainfont{Libertinus Serif}
\setsansfont{Libertinus Sans}
\setmonofont{Libertinus Mono}
\setdefaultlanguage{hebrew}
\setotherlanguage{english}

% operators
\DeclareMathOperator\cis{cis}
\DeclareMathOperator\Sp{Sp}
\DeclareMathOperator\tr{tr}
\DeclareMathOperator\im{Im}
\DeclareMathOperator\diag{diag}
\DeclareMathOperator*\lowlim{\underline{lim}}
\DeclareMathOperator*\uplim{\overline{lim}}

% commands
\renewcommand\qedsymbol{\textbf{משל}}
\newcommand{\NN}[0]{\mathbb{N}}
\newcommand{\ZZ}[0]{\mathbb{Z}}
\newcommand{\QQ}[0]{\mathbb{Q}}
\newcommand{\RR}[0]{\mathbb{R}}
\newcommand{\CC}[0]{\mathbb{C}}
\newcommand{\getenv}[2][] {
  \CatchFileEdef{\temp}{"|kpsewhich --var-value #2"}{\endlinechar=-1}
  \if\relax\detokenize{#1}\relax\temp\else\let#1\temp\fi
}
\newcommand{\explain}[2] {
	\begin{flalign*}
		 && \text{#2} && \text{#1}
	\end{flalign*}
}

% headers
\getenv[\AUTHOR]{AUTHOR}
\author{\AUTHOR}
\date\today

\title{מבנים אלגבריים 1}

\begin{document}
\maketitle
\maketitleprint{}

\section{שיעור 1 --- 6.5.2024}
הקורס עוסק בעיקרו בתורת החבורות, ממנה גם מתחילים. \\*
חבורה (באנגלית Group) היא מבנה מתמטי. \\*
ברעיון חבורה מייצגת סימטריה, אוסף השינויים שאפשר לעשות על אובייקט ללא שינוי שלו, קרי שהוא ישאר שקול לאובייקט במקור. \\*
מה הן הסימטריות שיש לריבוע? אני יכול לסובב ולשקף אותו בלי לשנות את הצורה המתקבלת והיא תהיה שקולה. חשוב להגיד שהפעולות האלה שקולות שכן התוצאה הסופית זהה למקורית. \\*
אפשר לסובב ספציפית אפס, תשעים מאה שמונים ומאתיים שבעים מעלות, נקרא לפעולות האלה A, B, C בהתאמה. \\*
בנוסף אפשר לשקף סביב ציר האמצע, ציר האמצע מלמעלה, ועל האלכסונים, ניתן גם לאלה שמות, נקרא לפעולות אלה בהתאמה $D, E, F, G, H$. \\*
אלה הפעולות הבסיסיות ואי אפשר לעשות פעולה שלא בקבוצה הזאת, אבל אפשר להרכיב את הפעולות האלה והתוצאה הסופית תהיה שקולה לפעולה מהקבוצה. \\*
נגדיר את הפעולות:
\[
	D_4 = \{ A, B, C, D, E, F, G, H \},
	\circ : D_4 \times D_4 \to D_4
\]
נראה כי הרכבת פעולות שקולה לפעולה קיימת:
\[
	E \circ G = C,
	E \circ B = H,
	B \circ F = F
\]
חשוב לשים לב שהפעולה הזאת לא חילופית: $X \circ Y \ne Y \circ X$. \\*
היא כן קיבוצית: $X \circ (Y \circ Z) = (X \circ Y) \circ Z$. \\*
תכונה נוספת היא קיום האיבר הנייטרלי, במקרה הזה $A$. איבר זה לא משפיע על הפעולה הסופית, והרכבה איתו מתבטלת ומשאירה רק את האיבר השני:
\[
	\forall X \in D_4 : A \circ X = X \circ A = X
\]
התכונה האחרונה היא קיום איבר נגדי:
\[
	\forall X \in D_4 \exists Y \in D_4 : X \circ Y = Y \circ X = A
\]

\subsection{הגדרה: חבורה}
חבורה היא קבוצה $G$ עם $\circ : G \times G \to G$ ואיבר $e \in G$ כך שמתקיימות התכונות הבאות:
\begin{enumerate}
	\item אסוציאטיביות (חוק הקיבוץ): $\forall x, y, z \in G : (x \circ y) \circ z = x \circ (y \circ z)$.
	\item קיום איבר נייטרלי: לכל $x \in G$ מתקיים $x \circ e = e \circ x = x$.
	\item קיום איבר נגדי: לכל $x \in G$ קיים $y \in G$ כך שמתקיים $x \circ y = y \circ x = e$.
\end{enumerate}
חשוב לציין כי זו היא לא הגדרה מיניממלית, ניתן לצמצם אותה, לדוגמה להגדיר שלכל איבר יש הופכי משמאל בלבד (יש להוכיח שקילות).

\subsection{למה: קיום איבר נייטרלי יחיד}
אם $e_1, e_2 \in G$ נייטרליים אז $e_1 = e_2$.
\begin{proof}
	$e_1 = e_1 \circ e_2 = e_2$
\end{proof}
דהינו, קיים איבר נייטרלי יחיד.

\subsection{דוגמות}
הקורס מבוסס על הספר ''מבנים אלגבריים'' מאת דורון פודר, אלכס לובוצקי ואהוד דה שליט, אך יש הבדלים, חשוב לשים לב אליהם. ניתן לקרוא שם דוגמות. \\*
דוגמות כלליות לחבורות,
עבור $(\FF, +, \cdot, 0, 1)$ שדה:
\begin{enumerate}
	\item חבורה החיבורית היא $(\FF, +, 0)$
	\item החבורה הכפלית היא $(\FF, \cdot, 1)$
\end{enumerate}
 הסימון הכי נפוץ לפעולה של החבורה היא כפל או נקודה או לא בכלל: $xy = x \cdot y$.

 \subsection{הגדרה: חבורה קומוטטיבית}
חבורה $G$ תיקרא קומוטטיבית או חילופית או אבלית (על שם המתטיקאי אבל) אם $xy = yx$ לכל $x, y \in G$. \\*
חשוב להבין, למה שסימטריות תהינה חילופיות.

\subsubsection{דוגמות לחבורות קומוטטיביות}
$(\ZZ, +, 0)$ חבורת החיבור מעל השלמים, היא חבורה קומוטטיבית. \\*
באופן דומה גם $(\ZZ_n, +, 0)$.

\subsubsection{דוגמות לחבורות שאינן קומוטטיביות}
\begin{itemize}
	\item $(D_4, \circ, A)$ אשר מייצג את הריבוע עליו דובר בתחילת ההרצאה
	\item $S_n$ תמורות על $1, \dots, n$ עם הרכבה. \\*
		תמורה היא פעולה שמחליפה שני איברים כפונקציה, לדוגמה $s(1) = 2, s(2) = 1, s(n) = n$. \\*
		$S_n$ הוא מקרה פרטי של תמורות על קבוצה $\{1, \dots, n \}$
	\item $\text{Sym}(X) = \{ f : X \to X \mid f \text{ הופכית, חח''ע ועל} \}$ \\*
		תמורות הן סימטריה של קבוצה, כל תמורה היא העתקה חד־חד ערכית ועל שמשמרת את מבנה הקבוצה.
	\item $GL_n(\FF)$ מטריצות $n \times n$ הפיכות מעל שדה $\FF עם כפל$.
	\item אם $V$ מרחב וקטורי מעל שדה $\FF$ אז \\*
$GL(V) = \{ f : V \to V \mid f \text{ לינארית וחד חד ערכית} \}$
\end{itemize}
נשים לב כי $GL_n(\FF) \cong GL(\FF^n)$, דהינו הם איזומורפיים. זה לא אומר שהם שווים, רק שיש להם בדיוק אותן תכונות. \\*
גם בקבוצות שתי קבוצות עם אתו גודל הן איזומורפיות אך לא שקולות.

\section{תרגול 1 --- 7.5.2024}
\subsection{דוגמות לחבורות}
\begin{align*}
	& (\ZZ, \cdot, 1) & \text{לא חבורה בגלל $0$} \\
	& (M_{n \times n}(\RR), \circ, I_n) & \text{לא חבורה בגלל מטריצות רגולריות ומטריצת האפס לדוגמה} \\
	& (\ZZ_4, +_4, 0) & \text{אכן חבורה} \\
	& (\ZZ_3, +_3, 0) & \text{אכן חבורה} \\
	& (\ZZ_4^*, \cdot, 1) & \text{לא חבורה, $2 \cdot 2 = 0$} \\
	& (\ZZ_3^*, \cdot, 1) & \text{אכן חבורה, מבוסס על מספר ראשוני} \\
\end{align*}
הערה לא קשורה: הסימון של כוכבית מסמן הסרת כלל האיברים הלא הפיכים מהקבוצה. \\*
כל שלישייה $(\ZZ_p\setminus\{0\}, \cdot_p, 1)$ היא חבורה בתנאי ש־$p$ הוא ראשוני.

\subsection{תכונות בסיסיות של חבורות}
\begin{align*}
	& e_1 = e_1 e_2 = e_2 & \text{יחידות האיבר הנייטרלי} \\
	& x \in G, y, y_1 = x^{-1} : y = y \cdot e = y x y_1 = e \cdot y_1 = y_1 & \text{יחידות ההופכי}
\end{align*}
תהי $G$ חבורה, $g = x_1 \cdot \hdots \cdot x_n$ ביטוי לא תלוי בהצבת סוגריים, טענה זו אפשר להוכיח באינדוקציה. \\*
לכל $n, m \in \NN$ מתקיים גם ${(x^n)}^m = x^{n\cdot m}$ ואף $x^n \cdot x^m = x^{n + m}$.

\subsection{תתי־חבורות}
תהי חבורה $(G, \cdot_G, e_G)$, ותהי $H \subseteq G$ תת־קבוצה, אז $(H, \cdot_G, e_G)$ תיקרא תת־חבורה אם היא מהווה חבורה תקינה. נסמן $H \le G$. \\*
לדוגמה נראה $(2\ZZ, +, 0) \le (\ZZ, +, 0)$ חבורת הזוגיים בחיבור היא תת־חבורה של השלמים. \\*
$(\diag_n(\RR), \circ, I_n) \le (GL_n(\RR), \circ, I_n)$ חבורת המטריצות האלכסוניות היא תת־חבורה של המטריצות. \\*
$(GL_n(\QQ), \circ, I_n) \le (GL_n(\RR), \circ, I_n)$ מטריצות הפיכות מעל הרציונליים חלקיות למטריצות הפיכות מעל הממשיים.

\subsubsection{קריטריון מקוצר לתת־חבורה}
תהי $G$ חבורה ותהי קבוצה $H \subseteq G$ אז $H \le G$ (תת־חבורה של $G$) אם ורק אם:
\begin{enumerate}
	\item $e_G \in H$, איבר היחידה נמצא ב־$H$
	\item $\forall x \in H : x^{-1} \in H$, לכל איבר גם האיבר ההופכי לו נמצא בקבוצה
	\item $\forall x, y \in H : x \cdot y \in H$, הקבוצה סגורה לכפל האיברים בה
\end{enumerate}

\subsubsection{דוגמות}
\begin{align*}
	& (\NN_0, +, 0) \not\subseteq (\ZZ, +, 0) & 1 \in \NN_0 \land -1 \not\in \NN_0 \\
	& \{0, 2, 4, 6, 8\} \subseteq (\ZZ_{10}, +_{10}, 0) & \text{כלל התנאים מתקיימים} \\
\end{align*}

\subsubsection{טענה: תת־חבורה לחבורה סופית}
אם חבורה היא סופית, אז תנאי 2 איננו הכרחי לתתי־חבורות.
\begin{proof}
	תהי $G$ חבורה סופית ותהי $H \subseteq G$ אשר מקיימת את סעיפים 1 ו־3 בקריטריון. \\*
	יהי $x \in H$, נבחין כי $\{ x^n \mid n \in \NN \} \subseteq H$ בעקבות סעיף 3 של הקריטריון. \\*
	לכן קיימים שני מספרים $n, m \in \NN$ כך ש־$m < n$ אשר מקיימים $x^n = x^m$. \\*
	כמובן מתקיים $x^n \cdot x^{-m} = e$ ומהסגירות לכפל נובע כי $x^{n - m} \in H$ ומצאנו כי התנאי השני מתקיים.
\end{proof}

\subsection{חבורת התמורות}
תהי $X$ קבוצה, אז $\text{Sym}(X)$ היא קבוצת הפונקציות החד־חד ערכיות ועל מ־$X$ לעצמה. \\*
$(\text{Sym}(X), \circ, Id)$ היא חבורה, מורכבת מכלל התמורות, הרכבת פונקציות ופונקציית הזהות. \\*
אם $X$ היא קבוצה סופית אז $S_n = \text{Sym}(X)$, ובדרך כלל נגדיר $X = [n] = \{1, \hdots, n\}$, וחבורת התמורות תהיה $(S_n, \circ, Id)$.

\subsubsection{הגדרה: סדר של חבורה}
סדר של חבורה הוא מספר האיברים בחבורה. \\*
אילו $G$ אז נגיד שסדר החבורה הוא אינסוף. \\*
נסמן את הסדר $|G|$. \\*
אילו $G$ חבורה ו־$x \in G$, הסדר של $x$ הוא $n \in \NN$ המינימלי כך שמתקיים $x^n = e$, נסמנו $|x|$ או $\sigma(x)$.

\subsubsection{חזרה לתמורות}
נשים לב שמתקיים $|S_n| = n !$. \\* % chktex 40
$\sigma \in \S_n$, נכתוב את התמורה כך:
\[
	\begin{pmatrix}
		1 & 2 & \cdots & n \\
		\sigma(1) & \sigma(2) & \cdots & \sigma(n)
	\end{pmatrix}
\]
לדוגמה $\begin{pmatrix}
	1 & 2 & 3 \\
	2 & 1 & 3
\end{pmatrix}$. \\*
אילו $\sigma \in S_n$ ו־$i \in [n]$ נקיים $\sigma(i) = i$ אז $i$ נקרא \textbf{נקודת שבט} של $\sigma$. \\*
בדוגמה שנתנו, $\sigma(3) = 3$ ולכן זוהי נקודת שבט של $\sigma$.

\subsubsection{תתי־חבורות של חבורת התמורות}
גודמה ראשונה:
\[
	\left\{
		\begin{pmatrix}
			1 & 2 & 3 \\
			1 & 2 & 3
		\end{pmatrix},
		\begin{pmatrix}
			1 & 2 & 3 \\
			2 & 1 & 3
		\end{pmatrix}
	\right\}
	\subseteq S_3
\]
היא תת־חבורה של $S_3$ שכן כללי הקריטריון מתקיימים מבדיקה. \\*
גם $\{ \sigma \in S_n \mid \sigma(1) = 1 \}$ היא תת־חבורה, שכן $\sigma(\tau(1)) = \tau(\sigma(1)) = 1$. \\*
לעומת זאת $\{ \sigma \in S_n \mid \sigma(1) \in \{1, 2, 3\}\}$ איננה חבורה. נראה כי אם $\sigma, \tau$ המקיימות $\sigma(4) = 2, \sigma(2) = 4, \tau(2) = 1, \tau(1) = 2$
וכל השאר נקודות שבט, $\sigma(\tau(1)) = 4$ שלא נמצא בקבוצה על־פי הגדרתה.

\subsubsection{מחזורים}
מחזור הוא רצף של איברים שהתמורה מחזירה כרצף, זאת אומרת שהתמורה עבור האיבר הראשון במחזור תחזיר את השני, השני את השלישי וכן הלאה. \\*
\textbf{הגדרה:} מחזור פשוט $\sigma \in S_n$ יקרא \textbf{$l$־מחזור} אם קיימים $x_1, \hdots, x_l \in [n]$ כך שלכל $0 \le i < l$ מתקיים $\sigma(x_i) = x_{i + 1}$ ו־$\sigma(x_l) = x_0$. \\*
\textbf{טענה:} כל תמורה היא הרכבה של מספר כלשהו של מחזורים, ההוכחה מסתמכת על היכולת לשרשר את ערכי המחזור משרשראות שאינן נוגעות אחת לשנייה. \\*
לדוגמה, נבחין כי אם
\[
	\sigma = \begin{pmatrix}
		1 & 2 & 3 & 4 & 5 & 6 & 7 \\
		6 & 2 & 7 & 5 & 1 & 4 & 3
	\end{pmatrix}
\]
אז נוכל להרכיב $\sigma = (1 \, 6 \, 4 \, 5)(2)(3 \, 7)$. \\*
נשים לב למקרה מיוחד, יהי $\sigma \in S_n$ כך ש־$\sigma$ הוא $l$־מחזור, ונגדיר $\sigma = (x_1 \, x_2 \, \hdots \, x_l)$. \\*
בהינתן $\tau \in S_n$, מתקיים
\[
	\tau \circ \sigma \circ \tau^{-1} = (\tau(x_1) \, \tau(x_2) \, \hdots \, \tau(x_n))
\]
זאת שכן לדוגמה $\sigma(\tau^{-1}(\tau(x_1))) = \sigma(x_1)$ ובהתאם $(\tau \circ \sigma \circ \tau^{-1}) (x_1) = \tau(x_1)$.

\section{שיעור 2 --- 8.5.2024}
\subsection{מבוא לאיזומורפיות}
%אנחנו מקבלים את האחריות על תהליך הלמידה בקורס מבחינת שיעורי בית, דהינו מטלות. בבקשה תפתור לבד כמה שאפשר ואשכרה תחשוב על כל שאלה, כדי ללמוד ממנה. \\*
המטרה שלנו היא להבין מתי שתי חבורות שונות הן שקולות, ולחקור את מושג האיזומורפיות. \\*
נבחן את $\ZZ/2$ ואת $(\{\pm 1\}, \cdot)$ ובשתיהן יש רק שני איברים, אחד נייטרלי ואחד לא, ובשתיהן הפעולות מתנהגות אותו דבר בדיוק.
\[
	1 \leftrightarrow -1,
	1 \leftrightarrow 0
\]
עוד דוגמה היא $(\RR, +)$ ו־$(\RR^{>0}, \cdot)$.
\[
	(\RR, +) \xrightarrow{\exp} (\RR^{>0}, \cdot), \exp(x + y) = \exp(a) \exp(b)
\]
\subsubsection{הגדרה: הומומורפיזם}
עבור $G$ ו־$H$ חבורות: \\*
\textbf{הומומורפיזם} מ־$G$ ל־$H$ היא פונקציה $\varphi : G \to H$ שמקיימת:
\begin{enumerate}
	\item $\varphi(e_G) = e_H$
	\item $\varphi(x y) = \varphi(x) \varphi(y)$
	\item $\varphi(x^{-1}) = {\varphi(x)}^{-1}$
\end{enumerate}

\subsubsection{למה: תנאי הכרחי להומומורפיזם}
$\varphi : G \to H$ היא הומומורפיזם אם ורק אם לכל $x, y \in G$ מתקיים $\varphi(xy) = \varphi(x) \varphi(y)$.
\begin{proof}
	נראה ששלושת התכונות מתקיימות:
	\begin{enumerate}
		\item נבחר $x \in G$ ונראה כי $\varphi(x) = \varphi(e_G x) = \varphi(e_G) \varphi(x) \iff e_H = \varphi(e_G)$.
		\item נתון
		\item $\varphi(e_G) = \varphi(x x^{-1}) = \varphi(x) \varphi(x^{-1}) = e_H \implies \varphi(x^{-1}) = \varphi^{-1}(x) e_H$
	\end{enumerate}
	ומצאנו כי שלושת התנאים מתקיימים.
\end{proof}

\subsubsection{הגדרה: איזומורפיזם}
איזומורפיזם $G$ ל־$H$ הוא הומומורפיזם חד־חד ערכי ועל ומסומן $\varphi : G \xrightarrow{\sim} H$.

\subsubsection{למה: הופכי לאיזומורפיזם}
עבור $\varphi : G \xrightarrow{\sim} H$ גם ההופכי הומומורפיזם (ולכן גם איזומורפיזם).
\begin{proof}
	נראה כי לכל $x, y \in H$:
	\[
		\varphi^{-1}(xy)
		= \varphi^{-1}(\varphi(\varphi^{-1}(x)) \varphi(\varphi^{-1}(y)))
		= \varphi^{-1}(x) \varphi^{-1}(y)
	\]
	ומצאנו כי התנאי ההכרחי להומומורפיזם מתקיים.
\end{proof}

\subsubsection{מסקנה: תנאי הכרחי לאיזומורפיזם}
המומורפיזם $\varphi : G \to H$ הוא איזומורפיזם אם ורק אם קיים הומומורפיזם $\psi : H \to G$ כך שמתקיים $\varphi \circ \psi = \psi \circ \varphi = Id_G$.

\subsubsection{הגדרה: איזומורפיות}
נגדיר שתי חבורות כ\textbf{איזומורפיות} אם ורק אם קיים איזומורפיזם ביניהן. \\*
נשים לב שמספר האיזומורפיזמים בין החבורות, גם אם הוא אינסופי, הוא חסר משמעות, ובמקום אנו מסתכל על עצם האיזומורפיות.

דוגמה לחבורות איזומורפיות הן $(\{\pm 1\}, \cdot) \cong \ZZ/2$ כפי שראינו בהתחלה. \\*
חשוב לשים לב שגם אם יש כמות איברים זהה בין החבורות, הן לא בהכרח תהינה איזומורפיות, לדוגמה
$GL_2(\FF_2)$, חבורת המטריצות ההפיכות מעל שדה עם שני איברים. יש בשורה העליונה 3 אפשרויות, ובשורה השנייה 2 ולכן יש 6 איברים בחבורה הזו.
גם ב־$S_3$ יש בדיוק שישה איברים, אבל $GL_2(\FF_2) \not\cong S_3$. גם החבורה החיבורית $\ZZ/6$ היא חבורה עם שישה איברים. החבורה הראשונה לא קומוטטיבית והשנייה כן, כי כפל מטריצות לא ניתן לשינוי סדר.

\subsubsection{למה: הרכבת הומומורפיזמים}
$\varphi : G \to H$ ו־$\psi : H \to K$ שני הומומורפיזמים, אז גם $\psi \circ \varphi : G \to K$ הוא הומומורפיזם.
\begin{proof}
	$\forall x, y \in G : (\psi \circ \varphi) (xy) = \psi(\varphi(xy)) = \psi(\varphi(x) \varphi(y)) = \psi(\varphi(x)) \psi(\varphi(y)) = (\psi \circ \varphi)(x) (\psi \circ \varphi)(y)$
\end{proof}

\subsubsection{מסקנה: הרכבת איזומורפיזמים}
הרכבה של איזומורפיזמים היא איזומורפיזם.

\subsubsection{הגדרה: אוטומורפיזם}
אוטומורפיזם של $G$ הוא איזומורפיזם $G \xrightarrow{\sim} G$. נסמן ב־$Aut(G)$ את קבוצת האוטומורפיזמים של $G$.

\subsubsection{למה: חבורת האוטומורפיזמים}
$Aut(G)$ היא חבורה ביחס להרכבה.
\begin{proof}
	הרכבה היא אסוציאטיבית, העתקת הזהות מוכלת בקבוצה ונייטרלי להרכבה, והוכחנו שלכל אוטומורפיזם $\varphi$ יש הופכי $\varphi^{-1} \in Aut(G)$.
\end{proof}
מהי $Aut(\ZZ)$? לדוגמה $\varphi(n) = n + 1$. פונקציה זו איננה אוטומורפיזם שכן $\varphi(1 + 3) = \varphi(4) = 5, \varphi(1) + \varphi(3) = 6$. \\*
פונקציית הזהות היא אוטומורפיזם, והפונקציה $\varphi(n) = -n$ על־פי בדיקה ישירה של הגדרות. \\*
נבחן את פונקציית הכפל בקבוע, $\varphi(n) = 2n$, נראה כי $\varphi(n + m) = 2(n + m) = 2n + 2m, \varphi(n) + \varphi(m) = 2n + 2m$. הומומורפיזם, אבל לא כל איבר שייך לקבוצה השנייה ולכן לא אוטומורפיזם.
\[
	Aut(\ZZ) = \{Id, -Id\} \cong \ZZ/2
\]

\subsubsection{טענה, ערך $Aut(\ZZ)$}
$Aut(\ZZ) = \{Id, -Id\}$.
\begin{proof}
	יהי $\varphi : \ZZ \xrightarrow{\sim} \ZZ$, ראשית נראה כי $\varphi(n) = n\varphi(1)$. \\*
	עבור $n = 0$ ברור, עבור $ n > 1$ נראה כי $\varphi(n) = \varphi(1 + \cdots + 1) = \varphi(1) + \cdots + \varphi(1) = n \varphi(1)$. \\*
	עבור $n \le 1$ נשתמש ב־$\varphi(-1) = -1$ ובהתאם $\varphi(-n) = (-n)\varphi(1)$. תתקן אחר כך את הסימנים. \\*
	לכן $\varphi(1) = \pm 1 \implies \varphi = \pm Id$.
\end{proof}

\subsubsection{הגדרה: מכפלת חבורות}
אם $G$ ו־$H$ הן חבורות, המכפלה הישרה ל $G$ ו־$H$ או $G \times H$ היא החבורה שמקיימת $G \times H = \{ (x, y) \mid x \in G, y \in H \}$.
עם הפעולה $(x_1, y_1) \cdot (x_2, y_2) = (x_1 x_2, y_1 y_2)$ והנייטרלי $e = (e_G, e_H) \in G \times H$ \\*
נראה בהמשך שמתקיים $\ZZ/6 \cong \ZZ_2 \times \ZZ_3$. אבל $\ZZ/4 \not\cong \ZZ/2 \times \ZZ/2$.

\subsubsection{הגדרה: תת־חבורה}
$G$ חבורה, ותהי תת־קבוצה $H \subseteq G$ נקראת תת־חבורה אם
\begin{enumerate}
	\item $e \in H$
	\item $x, y \in H \implies xy \in H$
	\item $x \in H \implies x^{-1} \in H$
\end{enumerate}
נשים לב  כי תת־קבוצה $H \subseteq G$ היא תת־חבורה אם ורק אם $H$ חבורה ביחס לאותה פעולה של $G$. \\*
מסמנים $H \le G$ תת־חבורה. \\*
דוגמות:
\begin{itemize}
	\item $\{ 0^\circ, 90^\circ, 180^\circ, 270^\circ \} \le D_4$
	\item $\{ \sigma \in S_n \mid \sigma(1) = 1 \} \le S_n$
		\subitem{-} תהי $G$ חבורה סופית אז $Aut(G) \le Sym(G) \cong S_n$
	\item $SL_n(\FF) \le GL_n(\FF)$ מטריצות עם דטרמיננטה 1 הן חלקיות למטריצות הפיכות.
	\item $B_n(\FF) \le GL_n(\FF)$ מטריצות משולשיות עליונות עם אלכסון 1 הן חלקיות אף הן להפיכות.
	\item $O_n(\FF) \le GL_n(\FF)$ חבורת המטריצות האורתוגונליות חלקיות לחבורת המטריצות ההפיכות. $O_n(\FF) = \{ A \in GL_n(\FF) \mid I_n = AA^t = A^t A$.
\end{itemize}

\subsubsection{למה: חיתוך תת־חבורות}
לכל קבוצה $S$ ומשפחה $\{ H_\alpha \le G \mid \alpha \in S\}$. של תת־חבורה של $G$ אז $\bigcap_{\alpha \in S} H_\alpha \le G$ תת־חבורה. \\*
הערה קטנה: משפחה היא קבוצה של קבוצות ככה שאפשר לזהות כל אחת לפי מספר, אפשר להשתמש בלמה גם בקבוצות כרגיל.
\begin{proof}
	\begin{itemize}
		\item $e \in H_\alpha$ לכל $\alpha \in S$ ולכן $e \in \bigcap_{\alpha \in S}$.
		\item $x, y \in \bigcap_{\alpha \in S}$ אם ורק אם לכל $\alpha$ מתקיים $x, y \in H_\alpha$ ולכן $xy \in H_\alpha$ ובהתאם $xy \in \bigcap_{\alpha \in S}$.
	\end{itemize}
	ומצאנו כי זוהי חבורה.
\end{proof}
למשל $SO_n = SL_n(\RR) \cap O_n \le GL_n(\RR)$.

\subsubsection{הגדרה: תת־חבורה נוצרת}
$G$ חבורה ו־$S \subseteq G$, תת־קבוצה, התת־חבורה הנוצרת על־ידי $S$ מוגדרת להיות:
\[
	\langle S \rangle = \bigcap_{S \subseteq H \le G} H
\]
ונשים לב כי על־פי הלמה האחרונה מתקבל כי זוהי אכן תת־חבורה.

\end{document}
