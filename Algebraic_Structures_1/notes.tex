\documentclass[a4paper]{article}

% packages
\usepackage{inputenc, fontspec, amsmath, amsthm, amsfonts, polyglossia, catchfile}
\usepackage[a4paper, margin=50pt, includeheadfoot]{geometry} % set page margins

% style
\AddToHook{cmd/section/before}{\clearpage}	% Add line break before section
\linespread{1.5}
\setcounter{secnumdepth}{0}		% Remove default number tags from sections
\setmainfont{Libertinus Serif}
\setsansfont{Libertinus Sans}
\setmonofont{Libertinus Mono}
\setdefaultlanguage{hebrew}
\setotherlanguage{english}

% operators
\DeclareMathOperator\cis{cis}
\DeclareMathOperator\Sp{Sp}
\DeclareMathOperator\tr{tr}
\DeclareMathOperator\im{Im}
\DeclareMathOperator\diag{diag}
\DeclareMathOperator*\lowlim{\underline{lim}}
\DeclareMathOperator*\uplim{\overline{lim}}

% commands
\renewcommand\qedsymbol{\textbf{משל}}
\newcommand{\NN}[0]{\mathbb{N}}
\newcommand{\ZZ}[0]{\mathbb{Z}}
\newcommand{\QQ}[0]{\mathbb{Q}}
\newcommand{\RR}[0]{\mathbb{R}}
\newcommand{\CC}[0]{\mathbb{C}}
\newcommand{\getenv}[2][] {
  \CatchFileEdef{\temp}{"|kpsewhich --var-value #2"}{\endlinechar=-1}
  \if\relax\detokenize{#1}\relax\temp\else\let#1\temp\fi
}
\newcommand{\explain}[2] {
	\begin{flalign*}
		 && \text{#2} && \text{#1}
	\end{flalign*}
}

% headers
\getenv[\AUTHOR]{AUTHOR}
\author{\AUTHOR}
\date\today

\title{מבנים אלגבריים 1}

\begin{document}
\maketitle
\maketitleprint{}

\section{שיעור 1 --- 6.5.2024}
הקורס עוסק בעיקרו בתורת החבורות, ממנה גם מתחילים. \\*
חבורה (באנגלית Group) היא מבנה מתמטי. \\*
ברעיון חבורה מייצגת סימטריה, אוסף השינויים שאפשר לעשות על אובייקט ללא שינוי שלו, קרי שהוא ישאר שקול לאובייקט במקור. \\*
מה הן הסימטריות שיש לריבוע? אני יכול לסובב ולשקף אותו בלי לשנות את הצורה המתקבלת והיא תהיה שקולה. חשוב להגיד שהפעולות האלה שקולות שכן התוצאה הסופית זהה למקורית. \\*
אפשר לסובב ספציפית אפס, תשעים מאה שמונים ומאתיים שבעים מעלות, נקרא לפעולות האלה A, B, C בהתאמה. \\*
בנוסף אפשר לשקף סביב ציר האמצע, ציר האמצע מלמעלה, ועל האלכסונים, ניתן גם לאלה שמות, נקרא לפעולות אלה בהתאמה $D, E, F, G, H$. \\*
אלה הפעולות הבסיסיות ואי אפשר לעשות פעולה שלא בקבוצה הזאת, אבל אפשר להרכיב את הפעולות האלה והתוצאה הסופית תהיה שקולה לפעולה מהקבוצה. \\*
נגדיר את הפעולות:
\[
	D_4 = \{ A, B, C, D, E, F, G, H \},
	\circ : D_4 \times D_4 \to D_4
\]
נראה כי הרכבת פעולות שקולה לפעולה קיימת:
\[
	E \circ G = C,
	E \circ B = H,
	B \circ F = F
\]
חשוב לשים לב שהפעולה הזאת לא חילופית: $X \circ Y \ne Y \circ X$. \\*
היא כן קיבוצית: $X \circ (Y \circ Z) = (X \circ Y) \circ Z$. \\*
תכונה נוספת היא קיום האיבר הנייטרלי, במקרה הזה $A$. איבר זה לא משפיע על הפעולה הסופית, והרכבה איתו מתבטלת ומשאירה רק את האיבר השני:
\[
	\forall X \in D_4 : A \circ X = X \circ A = X
\]
התכונה האחרונה היא קיום איבר נגדי:
\[
	\forall X \in D_4 \exists Y \in D_4 : X \circ Y = Y \circ X = A
\]

\subsection{הגדרה: חבורה}
חבורה היא קבוצה $G$ עם $\circ : G \times G \to G$ ואיבר $e \in G$ כך שמתקיימות התכונות הבאות:
\begin{enumerate}
	\item אסוציאטיביות (חוק הקיבוץ): $\forall x, y, z \in G : (x \circ y) \circ z = x \circ (y \circ z)$.
	\item קיום איבר נייטרלי: לכל $x \in G$ מתקיים $x \circ e = e \circ x = x$.
	\item קיום איבר נגדי: לכל $x \in G$ קיים $y \in G$ כך שמתקיים $x \circ y = y \circ x = e$.
\end{enumerate}
חשוב לציין כי זו היא לא הגדרה מיניממלית, ניתן לצמצם אותה, לדוגמה להגדיר שלכל איבר יש הופכי משמאל בלבד (יש להוכיח שקילות).

\subsection{למה: קיום איבר נייטרלי יחיד}
אם $e_1, e_2 \in G$ נייטרליים אז $e_1 = e_2$.
\begin{proof}
	$e_1 = e_1 \circ e_2 = e_2$
\end{proof}
דהינו, קיים איבר נייטרלי יחיד.

\subsection{דוגמות}
הקורס מבוסס על הספר ''מבנים אלגבריים'' מאת דורון פודר, אלכס לובוצקי ואהוד דה שליט, אך יש הבדלים, חשוב לשים לב אליהם. ניתן לקרוא שם דוגמות. \\*
דוגמות כלליות לחבורות,
עבור $(\FF, +, \cdot, 0, 1)$ שדה:
\begin{enumerate}
	\item חבורה החיבורית היא $(\FF, +, 0)$
	\item החבורה הכפלית היא $(\FF, \cdot, 1)$
\end{enumerate}
 הסימון הכי נפוץ לפעולה של החבורה היא כפל או נקודה או לא בכלל: $xy = x \cdot y$.

 \subsection{הגדרה: חבורה קומוטטיבית}
חבורה $G$ תיקרא קומוטטיבית או חילופית או אבלית (על שם המתטיקאי אבל) אם $xy = yx$ לכל $x, y \in G$. \\*
חשוב להבין, למה שסימטריות תהינה חילופיות.

\subsubsection{דוגמות לחבורות קומוטטיביות}
$(\ZZ, +, 0)$ חבורת החיבור מעל השלמים, היא חבורה קומוטטיבית. \\*
באופן דומה גם $(\ZZ_n, +, 0)$.

\subsubsection{דוגמות לחבורות שאינן קומוטטיביות}
\begin{itemize}
	\item $(D_4, \circ, A)$ אשר מייצג את הריבוע עליו דובר בתחילת ההרצאה
	\item $S_n$ תמורות על $1, \dots, n$ עם הרכבה. \\*
		תמורה היא פעולה שמחליפה שני איברים כפונקציה, לדוגמה $s(1) = 2, s(2) = 1, s(n) = n$. \\*
		$S_n$ הוא מקרה פרטי של תמורות על קבוצה $\{1, \dots, n \}$
	\item $\text{Sym}(X) = \{ f : X \to X \mid f \text{ הופכית, חח''ע ועל} \}$ \\*
		תמורות הן סימטריה של קבוצה, כל תמורה היא העתקה חד־חד ערכית ועל שמשמרת את מבנה הקבוצה.
	\item $GL_n(\FF)$ מטריצות $n \times n$ הפיכות מעל שדה $\FF עם כפל$.
	\item אם $V$ מרחב וקטורי מעל שדה $\FF$ אז \\*
$GL(V) = \{ f : V \to V \mid f \text{ לינארית וחד חד ערכית} \}$
\end{itemize}
נשים לב כי $GL_n(\FF) \cong GL(\FF^n)$, דהינו הם איזומורפיים. זה לא אומר שהם שווים, רק שיש להם בדיוק אותן תכונות. \\*
גם בקבוצות שתי קבוצות עם אתו גודל הן איזומורפיות אך לא שקולות.

\end{document}
