\documentclass[a4paper]{article}

% packages
\usepackage{inputenc, fontspec, amsmath, amsthm, amsfonts, polyglossia, catchfile}
\usepackage[a4paper, margin=50pt, includeheadfoot]{geometry} % set page margins

% style
\AddToHook{cmd/section/before}{\clearpage}	% Add line break before section
\linespread{1.5}
\setcounter{secnumdepth}{0}		% Remove default number tags from sections
\setmainfont{Libertinus Serif}
\setsansfont{Libertinus Sans}
\setmonofont{Libertinus Mono}
\setdefaultlanguage{hebrew}
\setotherlanguage{english}

% operators
\DeclareMathOperator\cis{cis}
\DeclareMathOperator\Sp{Sp}
\DeclareMathOperator\tr{tr}
\DeclareMathOperator\im{Im}
\DeclareMathOperator\diag{diag}
\DeclareMathOperator*\lowlim{\underline{lim}}
\DeclareMathOperator*\uplim{\overline{lim}}

% commands
\renewcommand\qedsymbol{\textbf{משל}}
\newcommand{\NN}[0]{\mathbb{N}}
\newcommand{\ZZ}[0]{\mathbb{Z}}
\newcommand{\QQ}[0]{\mathbb{Q}}
\newcommand{\RR}[0]{\mathbb{R}}
\newcommand{\CC}[0]{\mathbb{C}}
\newcommand{\getenv}[2][] {
  \CatchFileEdef{\temp}{"|kpsewhich --var-value #2"}{\endlinechar=-1}
  \if\relax\detokenize{#1}\relax\temp\else\let#1\temp\fi
}
\newcommand{\explain}[2] {
	\begin{flalign*}
		 && \text{#2} && \text{#1}
	\end{flalign*}
}

% headers
\getenv[\AUTHOR]{AUTHOR}
\author{\AUTHOR}
\date\today

\title{מבנים אלגבריים 1}
\usepackage{hyperref}
\setcounter{secnumdepth}{2}

\hypersetup{}
\begin{document}
\maketitle
\maketitleprint{}

\tableofcontents

\section{שיעור 1 --- 6.5.2024}
\subsection{מבוא לחבורות}
הקורס עוסק בעיקרו בתורת החבורות, ממנה גם מתחילים. \\*
חבורה (באנגלית Group) היא מבנה מתמטי. \\*
ברעיון חבורה מייצגת סימטריה, אוסף השינויים שאפשר לעשות על אובייקט ללא שינוי שלו, קרי שהוא ישאר שקול לאובייקט במקור. \\*
מה הן הסימטריות שיש לריבוע? אני יכול לסובב ולשקף אותו בלי לשנות את הצורה המתקבלת והיא תהיה שקולה. חשוב להגיד שהפעולות האלה שקולות שכן התוצאה הסופית זהה למקורית. \\*
אפשר לסובב ספציפית אפס, תשעים מאה שמונים ומאתיים שבעים מעלות, נקרא לפעולות האלה A, B, C בהתאמה. \\*
בנוסף אפשר לשקף סביב ציר האמצע, ציר האמצע מלמעלה, ועל האלכסונים, ניתן גם לאלה שמות, נקרא לפעולות אלה בהתאמה $D, E, F, G, H$. \\*
אלה הפעולות הבסיסיות ואי אפשר לעשות פעולה שלא בקבוצה הזאת, אבל אפשר להרכיב את הפעולות האלה והתוצאה הסופית תהיה שקולה לפעולה מהקבוצה. \\*
נגדיר את הפעולות:
\[
	D_4 = \{ A, B, C, D, E, F, G, H \},
	\circ : D_4 \times D_4 \to D_4
\]
נראה כי הרכבת פעולות שקולה לפעולה קיימת:
\[
	E \circ G = C,
	E \circ B = H,
	B \circ F = F
\]
חשוב לשים לב שהפעולה הזאת לא חילופית: $X \circ Y \ne Y \circ X$. \\*
היא כן קיבוצית: $X \circ (Y \circ Z) = (X \circ Y) \circ Z$. \\*
תכונה נוספת היא קיום האיבר הנייטרלי, במקרה הזה $A$. איבר זה לא משפיע על הפעולה הסופית, והרכבה איתו מתבטלת ומשאירה רק את האיבר השני:
\[
	\forall X \in D_4 : A \circ X = X \circ A = X
\]
התכונה האחרונה היא קיום איבר נגדי:
\[
	\forall X \in D_4 \exists Y \in D_4 : X \circ Y = Y \circ X = A
\]

\begin{definition}[חבורה]
	חבורה היא קבוצה $G$ עם $\circ : G \times G \to G$ ואיבר $e \in G$ כך שמתקיימות התכונות הבאות:
	\begin{enumerate}
		\item אסוציאטיביות (חוק הקיבוץ): $\forall x, y, z \in G : (x \circ y) \circ z = x \circ (y \circ z)$.
		\item קיום איבר נייטרלי: לכל $x \in G$ מתקיים $x \circ e = e \circ x = x$.
		\item קיום איבר נגדי: לכל $x \in G$ קיים $y \in G$ כך שמתקיים $x \circ y = y \circ x = e$.
	\end{enumerate}
	חשוב לציין כי זו היא לא הגדרה מיניממלית, ניתן לצמצם אותה, לדוגמה להגדיר שלכל איבר יש הופכי משמאל בלבד (יש להוכיח שקילות).
\end{definition}

\begin{lemma}[קיום איבר נייטרלי יחיד]
	אם $e_1, e_2 \in G$ נייטרליים אז $e_1 = e_2$.
\end{lemma}
\begin{proof}
	$e_1 = e_1 \circ e_2 = e_2$
\end{proof}
דהינו, קיים איבר נייטרלי יחיד.

\subsection{דוגמות}
הקורס מבוסס על הספר ''מבנים אלגבריים'' מאת דורון פודר, אלכס לובוצקי ואהוד דה שליט, אך יש הבדלים, חשוב לשים לב אליהם. ניתן לקרוא שם דוגמות. \\*
דוגמות כלליות לחבורות,
עבור $(\FF, +, \cdot, 0, 1)$ שדה:
\begin{enumerate}
	\item חבורה החיבורית היא $(\FF, +, 0)$
	\item החבורה הכפלית היא $(\FF, \cdot, 1)$
\end{enumerate}
הסימון הכי נפוץ לפעולה של החבורה היא כפל או נקודה או לא בכלל: $xy = x \cdot y$.
\begin{definition}[חבורה קומוטטיבית]
חבורה $G$ תיקרא קומוטטיבית או חילופית או אבלית (על שם המתטיקאי אבל) אם $xy = yx$ לכל $x, y \in G$. \\*
חשוב להבין, למה שסימטריות תהינה חילופיות.
\end{definition}
\begin{example}[לחבורות קומוטטיביות]
	$(\ZZ, +, 0)$ חבורת החיבור מעל השלמים, היא חבורה קומוטטיבית. \\*
	באופן דומה גם $(\ZZ_n, +, 0)$.
\end{example}
\begin{example}[חבורות לא קומוטטיביות]
	נבחין במספר דוגמות לחבורות שאין בהן חילופיות.
	\begin{itemize}
		\item $(D_4, \circ, A)$ אשר מייצג את הריבוע עליו דובר בתחילת ההרצאה
		\item $S_n$ תמורות על $1, \dots, n$ עם הרכבה. \\*
			תמורה היא פעולה שמחליפה שני איברים כפונקציה, לדוגמה $s(1) = 2, s(2) = 1, s(n) = n$. \\*
			$S_n$ הוא מקרה פרטי של תמורות על קבוצה $\{1, \dots, n \}$
		\item $\text{Sym}(X) = \{ f : X \to X \mid f \text{ הופכית, חח''ע ועל} \}$ \\*
			תמורות הן סימטריה של קבוצה, כל תמורה היא העתקה חד־חד ערכית ועל שמשמרת את מבנה הקבוצה.
		\item $GL_n(\FF)$ מטריצות $n \times n$ הפיכות מעל שדה $\FF עם כפל$.
		\item אם $V$ מרחב וקטורי מעל שדה $\FF$ אז \\*
	$GL(V) = \{ f : V \to V \mid f \text{ לינארית וחד חד ערכית} \}$
	\end{itemize}
	נשים לב כי $GL_n(\FF) \cong GL(\FF^n)$, דהינו הם איזומורפיים. זה לא אומר שהם שווים, רק שיש להם בדיוק אותן תכונות. \\*
	גם בקבוצות שתי קבוצות עם אתו גודל הן איזומורפיות אך לא שקולות.
\end{example}

\section{תרגול 1 --- 7.5.2024}
\subsection{חבורות ותתי־חבורות}
\begin{example}
	\begin{align*}
		& (\ZZ, \cdot, 1) & \text{לא חבורה בגלל $0$} \\
		& (M_{n \times n}(\RR), \circ, I_n) & \text{לא חבורה בגלל מטריצות רגולריות ומטריצת האפס לדוגמה} \\
		& (\ZZ_4, +_4, 0) & \text{אכן חבורה} \\
		& (\ZZ_3, +_3, 0) & \text{אכן חבורה} \\
		& (\ZZ_4^*, \cdot, 1) & \text{לא חבורה, $2 \cdot 2 = 0$} \\
		& (\ZZ_3^*, \cdot, 1) & \text{אכן חבורה, מבוסס על מספר ראשוני} \\
	\end{align*}
\end{example}
הערה לא קשורה: הסימון של כוכבית מסמן הסרת כלל האיברים הלא הפיכים מהקבוצה. \\*
כל שלישייה $(\ZZ_p\setminus\{0\}, \cdot_p, 1)$ היא חבורה בתנאי ש־$p$ הוא ראשוני.
\begin{lemma}[בסיסיות של חבורות]
	\begin{align*}
		& e_1 = e_1 e_2 = e_2 & \text{יחידות האיבר הנייטרלי} \\
		& x \in G, y, y_1 = x^{-1} : y = y \cdot e = y x y_1 = e \cdot y_1 = y_1 & \text{יחידות ההופכי}
	\end{align*}
	תהי $G$ חבורה, $g = x_1 \cdot \hdots \cdot x_n$ ביטוי לא תלוי בהצבת סוגריים, טענה זו אפשר להוכיח באינדוקציה. \\*
	לכל $n, m \in \NN$ מתקיים גם ${(x^n)}^m = x^{n\cdot m}$ ואף $x^n \cdot x^m = x^{n + m}$.
\end{lemma}
\begin{definition}[תת־חבורה]
	תהי חבורה $(G, \cdot_G, e_G)$, ותהי $H \subseteq G$ תת־קבוצה, אז $(H, \cdot_G, e_G)$ תיקרא תת־חבורה אם היא מהווה חבורה תקינה. נסמן $H \le G$.
\end{definition}
\begin{example}
	$(2\ZZ, +, 0) \le (\ZZ, +, 0)$ חבורת הזוגיים בחיבור היא תת־חבורה של השלמים. \\*
	$(\diag_n(\RR), \circ, I_n) \le (GL_n(\RR), \circ, I_n)$ חבורת המטריצות האלכסוניות היא תת־חבורה של המטריצות. \\*
	$(GL_n(\QQ), \circ, I_n) \le (GL_n(\RR), \circ, I_n)$ מטריצות הפיכות מעל הרציונליים חלקיות למטריצות הפיכות מעל הממשיים.
\end{example}
\begin{proposition}[מקוצר לתת־חבורה]
	תהי $G$ חבורה ותהי קבוצה $H \subseteq G$ אז $H \le G$ (תת־חבורה של $G$) אם ורק אם:
	\begin{enumerate}
		\item $e_G \in H$, איבר היחידה נמצא ב־$H$
		\item $\forall x \in H : x^{-1} \in H$, לכל איבר גם האיבר ההופכי לו נמצא בקבוצה
		\item $\forall x, y \in H : x \cdot y \in H$, הקבוצה סגורה לכפל האיברים בה
	\end{enumerate}
\end{proposition}
\begin{example}
	\begin{align*}
		& (\NN_0, +, 0) \not\subseteq (\ZZ, +, 0) & 1 \in \NN_0 \land -1 \not\in \NN_0 \\
		& \{0, 2, 4, 6, 8\} \subseteq (\ZZ_{10}, +_{10}, 0) & \text{כלל התנאים מתקיימים} \\
	\end{align*}
\end{example}
\begin{proposition}[תת־חבורה לחבורה סופית]
	אם חבורה היא סופית, אז תנאי 2 איננו הכרחי לתתי־חבורות.
	\begin{proof}
		תהי $G$ חבורה סופית ותהי $H \subseteq G$ אשר מקיימת את סעיפים 1 ו־3 בקריטריון. \\*
		יהי $x \in H$, נבחין כי $\{ x^n \mid n \in \NN \} \subseteq H$ בעקבות סעיף 3 של הקריטריון. \\*
		לכן קיימים שני מספרים $n, m \in \NN$ כך ש־$m < n$ אשר מקיימים $x^n = x^m$. \\*
		כמובן מתקיים $x^n \cdot x^{-m} = e$ ומהסגירות לכפל נובע כי $x^{n - m} \in H$ ומצאנו כי התנאי השני מתקיים.
	\end{proof}
\end{proposition}

\subsection{חבורת התמורות}
תהי $X$ קבוצה, אז $\text{Sym}(X)$ היא קבוצת הפונקציות החד־חד ערכיות ועל מ־$X$ לעצמה. \\*
$(\text{Sym}(X), \circ, Id)$ היא חבורה, מורכבת מכלל התמורות, הרכבת פונקציות ופונקציית הזהות. \\*
אם $X$ היא קבוצה סופית אז $S_n = \text{Sym}(X)$, ובדרך כלל נגדיר $X = [n] = \{1, \hdots, n\}$, וחבורת התמורות תהיה $(S_n, \circ, Id)$.

\begin{definition}[סדר של חבורה]
	סדר של חבורה הוא מספר האיברים בחבורה. \\*
	אילו $G$ אז נגיד שסדר החבורה הוא אינסוף. \\*
	נסמן את הסדר $|G|$. \\*
	אילו $G$ חבורה ו־$x \in G$, הסדר של $x$ הוא $n \in \NN$ המינימלי כך שמתקיים $x^n = e$, נסמנו $|x|$ או $\sigma(x)$.
\end{definition}
\subsubsection{חזרה לתמורות}
נשים לב שמתקיים $|S_n| = n !$. \\* % chktex 40
$\sigma \in S_n$, נכתוב את התמורה כך:
\[
	\begin{pmatrix}
		1 & 2 & \cdots & n \\
		\sigma(1) & \sigma(2) & \cdots & \sigma(n)
	\end{pmatrix}
\]
לדוגמה $\begin{pmatrix}
	1 & 2 & 3 \\
	2 & 1 & 3
\end{pmatrix}$. \\*
אילו $\sigma \in S_n$ ו־$i \in [n]$ נקיים $\sigma(i) = i$ אז $i$ נקרא \textbf{נקודת שבט} של $\sigma$. \\*
בדוגמה שנתנו, $\sigma(3) = 3$ ולכן זוהי נקודת שבט של $\sigma$.

\subsubsection{תתי־חבורות של חבורת התמורות}
דוגמה ראשונה:
\[
	\left\{
		\begin{pmatrix}
			1 & 2 & 3 \\
			1 & 2 & 3
		\end{pmatrix},
		\begin{pmatrix}
			1 & 2 & 3 \\
			2 & 1 & 3
		\end{pmatrix}
	\right\}
	\subseteq S_3
\]
היא תת־חבורה של $S_3$ שכן כללי הקריטריון מתקיימים מבדיקה. \\*
גם $\{ \sigma \in S_n \mid \sigma(1) = 1 \}$ היא תת־חבורה, שכן $\sigma(\tau(1)) = \tau(\sigma(1)) = 1$. \\*
לעומת זאת $\{ \sigma \in S_n \mid \sigma(1) \in \{1, 2, 3\}\}$ איננה חבורה. נראה כי אם $\sigma, \tau$ המקיימות $\sigma(4) = 2, \sigma(2) = 4, \tau(2) = 1, \tau(1) = 2$
וכל השאר נקודות שבט, $\sigma(\tau(1)) = 4$ שלא נמצא בקבוצה על־פי הגדרתה.

\subsubsection{מחזורים}
מחזור הוא רצף של איברים שהתמורה מחזירה כרצף, זאת אומרת שהתמורה עבור האיבר הראשון במחזור תחזיר את השני, השני את השלישי וכן הלאה.
\begin{definition}
	מחזור פשוט $\sigma \in S_n$ יקרא \textbf{$l$־מחזור} אם קיימים $x_1, \hdots, x_l \in [n]$ כך שלכל $0 \le i < l$ מתקיים $\sigma(x_i) = x_{i + 1}$ ו־$\sigma(x_l) = x_0$.
\end{definition}
\begin{proposition}
	כל תמורה היא הרכבה של מספר כלשהו של מחזורים, ההוכחה מסתמכת על היכולת לשרשר את ערכי המחזור משרשראות שאינן נוגעות אחת לשנייה.
\end{proposition}
\begin{example}
	נבחין כי אם
	\[
		\sigma = \begin{pmatrix}
			1 & 2 & 3 & 4 & 5 & 6 & 7 \\
			6 & 2 & 7 & 5 & 1 & 4 & 3
		\end{pmatrix}
	\]
	אז נוכל להרכיב $\sigma = (1 \, 6 \, 4 \, 5)(2)(3 \, 7)$. \\*
	נשים לב למקרה מיוחד, יהי $\sigma \in S_n$ כך ש־$\sigma$ הוא $l$־מחזור, ונגדיר $\sigma = (x_1 \, x_2 \, \hdots \, x_l)$. \\*
	בהינתן $\tau \in S_n$, מתקיים
	\[
		\tau \circ \sigma \circ \tau^{-1} = (\tau(x_1) \, \tau(x_2) \, \hdots \, \tau(x_n))
	\]
	זאת שכן לדוגמה $\sigma(\tau^{-1}(\tau(x_1))) = \sigma(x_1)$ ובהתאם $(\tau \circ \sigma \circ \tau^{-1}) (x_1) = \tau(x_1)$.
\end{example}

\section{שיעור 2 --- 8.5.2024}
\subsection{מבוא לאיזומורפיות}
%אנחנו מקבלים את האחריות על תהליך הלמידה בקורס מבחינת שיעורי בית, דהינו מטלות. בבקשה תפתור לבד כמה שאפשר ואשכרה תחשוב על כל שאלה, כדי ללמוד ממנה. \\*
המטרה שלנו היא להבין מתי שתי חבורות שונות הן שקולות, ולחקור את מושג האיזומורפיות. \\*
נבחן את $\ZZ/2$ ואת $(\{\pm 1\}, \cdot)$ ובשתיהן יש רק שני איברים, אחד נייטרלי ואחד לא, ובשתיהן הפעולות מתנהגות אותו דבר בדיוק.
\[
	1 \leftrightarrow -1,
	1 \leftrightarrow 0
\]
עוד דוגמה היא $(\RR, +)$ ו־$(\RR^{>0}, \cdot)$.
\[
	(\RR, +) \xrightarrow{\exp} (\RR^{>0}, \cdot), \exp(x + y) = \exp(a) \exp(b)
\]
\begin{definition}[הומומורפיזם]
	עבור $G$ ו־$H$ חבורות,
	\textbf{הומומורפיזם} מ־$G$ ל־$H$ היא פונקציה $\varphi : G \to H$ שמקיימת:
	\begin{enumerate}
		\item $\varphi(e_G) = e_H$
		\item $\varphi(x y) = \varphi(x) \varphi(y)$
		\item $\varphi(x^{-1}) = {\varphi(x)}^{-1}$
	\end{enumerate}
\end{definition}
\begin{lemma}[תנאי הכרחי להומומורפיזם]
	$\varphi : G \to H$ היא הומומורפיזם אם ורק אם לכל $x, y \in G$ מתקיים $\varphi(xy) = \varphi(x) \varphi(y)$.
\end{lemma}
\begin{proof}
	נראה ששלושת התכונות מתקיימות:
	\begin{enumerate}
		\item נבחר $x \in G$ ונראה כי $\varphi(x) = \varphi(e_G x) = \varphi(e_G) \varphi(x) \iff e_H = \varphi(e_G)$.
		\item נתון
		\item $\varphi(e_G) = \varphi(x x^{-1}) = \varphi(x) \varphi(x^{-1}) = e_H \implies \varphi(x^{-1}) = \varphi^{-1}(x) e_H$
	\end{enumerate}
	ומצאנו כי שלושת התנאים מתקיימים.
\end{proof}

\begin{definition}[איזומורפיזם]
	איזומורפיזם $G$ ל־$H$ הוא הומומורפיזם חד־חד ערכי ועל ומסומן $\varphi : G \xrightarrow{\sim} H$.
\end{definition}

\begin{lemma}[הופכי לאיזומורפיזם]
	עבור $\varphi : G \xrightarrow{\sim} H$ גם ההופכי הומומורפיזם (ולכן גם איזומורפיזם).
\end{lemma}
\begin{proof}
	נראה כי לכל $x, y \in H$:
	\[
		\varphi^{-1}(xy)
		= \varphi^{-1}(\varphi(\varphi^{-1}(x)) \varphi(\varphi^{-1}(y)))
		= \varphi^{-1}(x) \varphi^{-1}(y)
	\]
	ומצאנו כי התנאי ההכרחי להומומורפיזם מתקיים.
\end{proof}

\begin{conclusion}[תנאי הכרחי לאיזומורפיזם]
	המומורפיזם $\varphi : G \to H$ הוא איזומורפיזם אם ורק אם קיים הומומורפיזם $\psi : H \to G$ כך שמתקיים $\varphi \circ \psi = \psi \circ \varphi = Id_G$.
\end{conclusion}

\begin{definition}[איזומורפיות]
	נגדיר שתי חבורות כ\textbf{איזומורפיות} אם ורק אם קיים איזומורפיזם ביניהן. \\*
	נשים לב שמספר האיזומורפיזמים בין החבורות, גם אם הוא אינסופי, הוא חסר משמעות, ובמקום אנו מסתכל על עצם האיזומורפיות.
\end{definition}
דוגמה לחבורות איזומורפיות הן $(\{\pm 1\}, \cdot) \cong \ZZ/2$ כפי שראינו בהתחלה. \\*
חשוב לשים לב שגם אם יש כמות איברים זהה בין החבורות, הן לא בהכרח תהינה איזומורפיות, לדוגמה
$GL_2(\FF_2)$, חבורת המטריצות ההפיכות מעל שדה עם שני איברים. יש בשורה העליונה 3 אפשרויות, ובשורה השנייה 2 ולכן יש 6 איברים בחבורה הזו.
גם ב־$S_3$ יש בדיוק שישה איברים, אבל $GL_2(\FF_2) \not\cong S_3$. גם החבורה החיבורית $\ZZ/6$ היא חבורה עם שישה איברים. החבורה הראשונה לא קומוטטיבית והשנייה כן, כי כפל מטריצות לא ניתן לשינוי סדר.
\begin{lemma}[הרכבת הומומורפיזמים]
	$\varphi : G \to H$ ו־$\psi : H \to K$ שני הומומורפיזמים, אז גם $\psi \circ \varphi : G \to K$ הוא הומומורפיזם.
\end{lemma}
\begin{proof}
	$\forall x, y \in G : (\psi \circ \varphi) (xy) = \psi(\varphi(xy)) = \psi(\varphi(x) \varphi(y)) = \psi(\varphi(x)) \psi(\varphi(y)) = (\psi \circ \varphi)(x) (\psi \circ \varphi)(y)$
\end{proof}
\begin{conclusion}[הרכבת איזומורפיזמים]
	הרכבה של איזומורפיזמים היא איזומורפיזם.
\end{conclusion}
\begin{definition}[אוטומורפיזם]
	אוטומורפיזם של $G$ הוא איזומורפיזם $G \xrightarrow{\sim} G$. נסמן ב־$Aut(G)$ את קבוצת האוטומורפיזמים של $G$.
\end{definition}
\begin{lemma}[חבורת האוטומורפיזמים]
	$Aut(G)$ היא חבורה ביחס להרכבה.
\end{lemma}
\begin{proof}
	הרכבה היא אסוציאטיבית, העתקת הזהות מוכלת בקבוצה ונייטרלי להרכבה, והוכחנו שלכל אוטומורפיזם $\varphi$ יש הופכי $\varphi^{-1} \in Aut(G)$.
\end{proof}
מהי $Aut(\ZZ)$? לדוגמה $\varphi(n) = n + 1$. פונקציה זו איננה אוטומורפיזם שכן $\varphi(1 + 3) = \varphi(4) = 5, \varphi(1) + \varphi(3) = 6$. \\*
פונקציית הזהות היא אוטומורפיזם, והפונקציה $\varphi(n) = -n$ על־פי בדיקה ישירה של הגדרות. \\*
נבחן את פונקציית הכפל בקבוע, $\varphi(n) = 2n$, נראה כי $\varphi(n + m) = 2(n + m) = 2n + 2m, \varphi(n) + \varphi(m) = 2n + 2m$. הומומורפיזם, אבל לא כל איבר שייך לקבוצה השנייה ולכן לא אוטומורפיזם.
\[
	Aut(\ZZ) = \{Id, -Id\} \cong \ZZ/2
\]
\begin{proposition}[ערך $Aut(Z)$]
	$Aut(\ZZ) = \{Id, -Id\}$.
\end{proposition}
\begin{proof}
	יהי $\varphi : \ZZ \xrightarrow{\sim} \ZZ$, ראשית נראה כי $\varphi(n) = n\varphi(1)$. \\*
	עבור $n = 0$ ברור, עבור $ n > 1$ נראה כי $\varphi(n) = \varphi(1 + \cdots + 1) = \varphi(1) + \cdots + \varphi(1) = n \varphi(1)$. \\*
	עבור $n \le 1$ נשתמש ב־$\varphi(-1) = -1$ ובהתאם $\varphi(-n) = (-n)\varphi(1)$. תתקן אחר כך את הסימנים. \\*
	לכן $\varphi(1) = \pm 1 \implies \varphi = \pm Id$.
\end{proof}
\begin{definition}[מכפלת חבורות]
	אם $G$ ו־$H$ הן חבורות, המכפלה הישרה ל $G$ ו־$H$ או $G \times H$ היא החבורה שמקיימת $G \times H = \{ (x, y) \mid x \in G, y \in H \}$.
	עם הפעולה $(x_1, y_1) \cdot (x_2, y_2) = (x_1 x_2, y_1 y_2)$ והנייטרלי $e = (e_G, e_H) \in G \times H$ \\*
	נראה בהמשך שמתקיים $\ZZ/6 \cong \ZZ_2 \times \ZZ_3$. אבל $\ZZ/4 \not\cong \ZZ/2 \times \ZZ/2$.
\end{definition}
\begin{definition}[תת־חבורה]
	$G$ חבורה, ותהי תת־קבוצה $H \subseteq G$ נקראת תת־חבורה אם
	\begin{enumerate}
		\item $e \in H$
		\item $x, y \in H \implies xy \in H$
		\item $x \in H \implies x^{-1} \in H$
	\end{enumerate}
\end{definition}
נשים לב  כי תת־קבוצה $H \subseteq G$ היא תת־חבורה אם ורק אם $H$ חבורה ביחס לאותה פעולה של $G$. \\*
מסמנים $H \le G$ תת־חבורה. \\*
דוגמות:
\begin{itemize}
	\item $\{ 0^\circ, 90^\circ, 180^\circ, 270^\circ \} \le D_4$
	\item $\{ \sigma \in S_n \mid \sigma(1) = 1 \} \le S_n$
		\subitem{-} תהי $G$ חבורה סופית אז $Aut(G) \le Sym(G) \cong S_n$
	\item $SL_n(\FF) \le GL_n(\FF)$ מטריצות עם דטרמיננטה 1 הן חלקיות למטריצות הפיכות.
	\item $B_n(\FF) \le GL_n(\FF)$ מטריצות משולשיות עליונות עם אלכסון 1 הן חלקיות אף הן להפיכות.
	\item $O_n(\FF) \le GL_n(\FF)$ חבורת המטריצות האורתוגונליות חלקיות לחבורת המטריצות ההפיכות. $O_n(\FF) = \{ A \in GL_n(\FF) \mid I_n = AA^t = A^t A$.
\end{itemize}
\begin{lemma}[חיתוך תת־חבורות]
	לכל קבוצה $S$ ומשפחה $\{ H_\alpha \le G \mid \alpha \in S\}$. של תת־חבורה של $G$ אז $\bigcap_{\alpha \in S} H_\alpha \le G$ תת־חבורה. \\*
	הערה קטנה: משפחה היא קבוצה של קבוצות ככה שאפשר לזהות כל אחת לפי מספר, אפשר להשתמש בלמה גם בקבוצות כרגיל.
\end{lemma}
\begin{proof}
	\begin{itemize}
		\item $e \in H_\alpha$ לכל $\alpha \in S$ ולכן $e \in \bigcap_{\alpha \in S}$.
		\item $x, y \in \bigcap_{\alpha \in S}$ אם ורק אם לכל $\alpha$ מתקיים $x, y \in H_\alpha$ ולכן $xy \in H_\alpha$ ובהתאם $xy \in \bigcap_{\alpha \in S}$.
	\end{itemize}
	ומצאנו כי זוהי חבורה.
\end{proof}
למשל $SO_n = SL_n(\RR) \cap O_n \le GL_n(\RR)$.
\begin{definition}[תת־חבורה נוצרת]
	$G$ חבורה ו־$S \subseteq G$, תת־קבוצה, התת־חבורה הנוצרת על־ידי $S$ מוגדרת להיות:
	\[
		\langle S \rangle = \bigcap_{S \subseteq H \le G} H
	\]
\end{definition}
נשים לב כי על־פי הלמה האחרונה מתקבל כי זוהי אכן תת־חבורה.

\section{שיעור 3 --- 15.5.2024}
\subsection{תת־חבורות}
\begin{definition}[תת־חבורה נוצרת]
	תהי $S \subseteq G$ תת־קבוצה לחבורה, נגדיר
	\[
		\langle S \rangle = \bigcup_{S \subseteq H \le G} H \le G
	\]
\end{definition}
\begin{lemma}[תת־חבורה מינימלית]
	$S \subseteq G$ התת־חבורה המינימלית $\langle S \rangle$ היא התת־חבורה המינימלית של $G$ המכילה את $S$.
\end{lemma}
קצת קשה לעבור על זה, איזה אפיון נוסף יש לדבר הזה?

\begin{proposition}[תת־חבורה נוצרת מפורשת]
	$S \subseteq G$ אז
	\[
		\langle S \rangle = \overline{S} \equiv \{ x_1^{\epsilon_1}x_2^{\epsilon_2} \cdots x_n^{\epsilon_n} \mid x_i \in S, \epsilon_i = \pm 1 \}
	\]
\end{proposition}
\begin{proof}
	\textbf{כיוון ראשון:}
	נניח שעבור תת־חבורה $H$ המכילה של $S$ סגיורת $H$ לכפל והופכי גוררת שהקבוצה $\overline{S}$ הנתונה מוכלת ב־$H$. \\*
	מצד שני נראה שזוהי כבר תת־חבורה.
	\begin{itemize}
		\item $1 \in \overline{S}$ מכפלה ריקה.
		\item $x, y \in \overline{S}$ אז נסמן
			\[
				x = x_1^{\epsilon_1}x_2^{\epsilon_2} \cdots x_n^{\epsilon_n},
				y = y_1^{\epsilon_1}y_2^{\epsilon_2} \cdots y_n^{\epsilon_n},
				xy = x_1^{\epsilon_1}x_2^{\epsilon_2} \cdots x_n^{\epsilon_n} y_1^{\epsilon_1}y_2^{\epsilon_2} \cdots y_n^{\epsilon_n}
			\]
		\item $x \in \overline{S}$ אז
			\[
				x^{-1} = x_1^{-\epsilon_1}x_2^{-\epsilon_2} \cdots x_n^{-\epsilon_n},
			\]
			וידוע כי $(x y)(x^{-1} y^{-1}) = x y x^{-1} y^{-1} = x x^{-1} = 1$
	\end{itemize}
\end{proof}

\begin{definition}[שלמות תת־חבורה יוצרת]
	אם $\langle S \rangle = G$ אומרים ש־$S - e$ יוצרת את $G$.
\end{definition}
\begin{example}
	מתקיים $\langle -1 \rangle = \langle 1 \rangle = \ZZ$. כקונספט כללי $\langle d \rangle = d\ZZ$. \\*
	מה לגבי $\ZZ/n$? מתקיים $\langle 1 \rangle = \ZZ/n$.
\end{example}
\begin{definition}[חבורה ציקלית]
	חבורה $G$ נקראת \textbf{ציקלית} אם היא נוצרת על־ידי איבר אחד, דהינו קיים $x \in G$ כך ש־$\langle x \rangle = G$.
\end{definition}
\begin{proposition}
	כל חבורה ציקלית $G$ מקיימת $G = \overset{\sim}{=} \ZZ$ או $G \overset{\sim}{=} \ZZ/n$ הוכחה בתרגיל.
\end{proposition}
\begin{example}
	$G = D_4$. \\*
	נגדיר את $\sigma$ להיות סיבוב בתשעים מעלות, ואת $\tau$ להיות היפוך על ציר האיקס. \\*
	אז יש לנו את $\langle \sigma \rangle = \{e, \sigma, \sigma^2, \sigma^3\}$ \\*
	וגם $\langle \tau \rangle = \{e, \tau\}$. \\*
	אנחנו יכולים להכפיל כל שני איברים משתי הקבוצות שסימנו עכשיו.
	\[
		D_4
		= \langle \sigma, \tau \rangle
		= \{e, \sigma, \sigma^2, \sigma^3,
		\tau, \sigma \tau, \sigma^2 \tau, \sigma^3 \tau \}
	\]
	נראה כי לדוגמה $\tau \sigma = \sigma^3 \tau, \sigma^4 = e, \tau^2 = e$. \\*
	ונראה כי $\tau \sigma \tau^{-1} = \sigma^3 = \sigma^{-1}$.
\end{example}
\begin{proposition}[תת־חבורות של Z]
	לכל $H \le \ZZ$ קיים $d \ge 0$ יחיד כך ש־$H = d \ZZ$.
\end{proposition}
\begin{proof}
	אם $H \ne \{0\}$ אז קיים $0 < d \in H$ וניקח את $d$ להיות המינימלי שמקיים את אי־השוויון. \\*
	מצד אחד $\langle d \rangle = d \ZZ \subseteq H$. \\*
	מצד שני, עבור $a \in H$ וידוע $a > 0$ אז נכתוב $a = nd + r$ כאשר $0 \le r < d$ שארית. \\*
	נקבל כי $r = a - nd \in H$. מהמינימליות של $d$ נובע כי $r = 0$ ולכן $a = nd \in d\ZZ$.
\end{proof}
יחידות של זה: תרגיל
נגלה בהמשך שתת־חבורה של חבורה ציקלית היא בעצמה ציקלית.

\begin{definition}[gcb]
	עבור שני מספרים $a, b \in \ZZ$ שלא שניהם $0$ נגדיר $\gcd(a, b) = d$ (Greatest common divisor) מחלק משותף מקסימלי כך שמתקיים:
	$d \mid a, b$ וגם לשלכל $m \mid a, b$ מתקיים גם $m \mid d$.
\end{definition}
\begin{proof}
	$\langle a, b \rangle = d\ZZ$, לאיזשהו $d \ge 0$ יחיד. \\*
	נראה ש־$d = \gcd(a, b)$. \\*
	מצד אחד $a, b \in d \ZZ$ ולכן $d \mid a, b$. \\*
	מצד שני אם $n \mid a, b$ אז $d \in d \ZZ = \{a, b \} \subseteq m \ZZ$. ולכן $m \mid d$ והוא מחלק מקסימלי.
\end{proof}
 \begin{example}
	 עבור $2 \ZZ = \langle 2 \rangle = \langle 6, 10 \rangle$
 \end{example}
 \begin{conclusion}[הלמה של Bézout]
	 לכל $a, b \in \ZZ$ קיימים $n, m \in \ZZ$ עבורם $\gcd(a, b) = na + mb$.
\end{conclusion}

\subsection{מחלקות (Cosets)}
\begin{definition}[מחלקה ימנית ושמאלית]
	תהי $G$ חבורה ו־$H \le G$ ו־$x \in G$. 
	נגדיר את המחלקה המשלאתי של $x$ על־ידי
	\[
		x H = \{x h \mid h \in H\}
	\]
	ואת המחלקה הימנית של $x$ בהתאם
	\[
		H x = \{h x \mid h \in H\}
	\]
\end{definition}
תרגיל: להוכיח שהמחלקה הימנית והשמאלית הן איזומורפיות. וזה לא נכון במונואיד.

\begin{lemma}[שיוך למחלקה]
	$y \in xH \iff yH = xH$
\end{lemma}
\begin{proof}
	\[
		y \in xH
		\iff  y = x h
		\iff x^{-1} y \in H
		\iff y^{-1} x \in H
		\iff x \in yH, y \in xH
		\iff xH = yH
	\]
\end{proof}
\begin{conclusion}
	לכל $x, y \in G$ מתקיים \\*
	$x H = y H$ (אם ורק אם $x^{-1} y \in H$). \\*
	או $xH \cup yH = \emptyset$
\end{conclusion}
\begin{proof}
	אם $z \not\in xH \cup yH$ אז מהלמה הקודמת $yH = ZH = xH$.
\end{proof}

\begin{proposition}[כיסוי זר]
	$G \le H$ התת־קבוצות מהצורה $xH$ עבור $x \in G$ מהוות כיסוי זר של $G$.
\end{proposition}
\begin{proof}
	נשאר לשים לב $x \in xH$ ולכן כיסוי ומהמסקנה זר.
\end{proof}

\begin{proposition}
	לכל $x, y \in G$ יש התאמה חד־חד ועל ערכית של קבוצות $xH \xrightarrow{\sim} yH$. \\*
	בפרט אם $H$ סופית אז לכל המחלקות אותו גודל, $|xH| = |yH|$.
\end{proposition}
\begin{proof}
	נגדיר
	$\varphi : xH \to yH$ על־ידי $\varphi(z) = y x^{-1} z$. \\*
	ונגדיר פונקציה חדשה $\psi : yH \to xH$ על־ידי $\psi(z) = x y^{-1} z$. \\*
	אז מתקיים $\psi = \varphi^{-1}$ ובהתאם נובע כי $\varphi$ איזומורפיזם.
\end{proof}

\begin{definition}[אוסף מחלקות]
	$H \le G$ אז נסמן
	\[
		G / H = \{ xH \mid x \in G \},
		H \backslash G = \{ Hx \mid x \in G \}
	\]
	אוסף המחלקות השמאליות והימניות בהתאמה.
\end{definition}
\begin{theorem}[משפט לאגרנז']
	אם $G$ חבורה סופית, אז לכל $H \le G$ מתקיים $\left. |H| \Big| |G| \right. $.
\end{theorem}
\begin{proof}
	ל־$G$ יש כיסוי זר על־ידי מחלקות שמאליות של $H$ ולכן הגודל של $|G| = |H| \cdot |G / H|$. \\*
	הגודל של $|G / H| = |G| / |H|$.
\end{proof}
\begin{notation}
	$|G / H| = |G : H|$ \textbf{האינדקס} של $H$ ב־$G$.
\end{notation}

\begin{example}
	המחלקות של $3\ZZ \le \ZZ$:
	\[
		3\ZZ + 0 = 3\ZZ + 3, 3\ZZ + 1, 3\ZZ + 2
	\]
	הקבוצה $\ZZ/3\ZZ$ היא השאריות האפשריות בחלוקה לשלוש.
\end{example}

\section{שיעור 4 --- 20.5.2024}
\subsection{סדר}
\begin{definition}[סדר של חבורה]
	$G$ חבורה ו־$x \in G$ מסומן $o(x)$ הוא המספר הקטן ביותר כך ש־$1 \le n \in \NN, x^n = e$. או $\infty$ אם לא קיים $n$ כזה.
\end{definition}
\begin{lemma}[סדר]
	\[
		o(x) = | \langle x\rangle|
	\]
\end{lemma}
\begin{proof}
	נוכיח שאם $o(x)$ סופי אז
	\[
		\langle x \rangle = \{ 1, x, x^2, \hdots, x^{o(x) - 1} \} \tag{1}
	\]
	ואם $o(x) = \infty$ אז
	\[
		\langle x \rangle = \{ 1, x, x^2, \hdots, \} \cup \{ x^{-1}, x^{-2}, \hdots \} \tag{2}
	\]
	הוכחה ל־$(1)$. \\*
	$(1)$ תת־חבורה:
	\begin{itemize}
		\item $x^k \cdot x^m = x^{(m + k) \mod o(x)}$.
		\item ${(x^n)}^{-1} = x^{o(x) - n}$.
	\end{itemize}
	כל ההאיברים שונים כי אם $x^k = x^m$ ל־$0 \le k < k \le o(x)$ אז
	\[
		1 = x^0 = m^{m - k}
	\]
	ונקבל $1 \le m - k < o(x)$ בסתירה למינימליות של $o(x)$. \\*
	הוכחה ל־$(2)$: \\*
	אם $H = \langle x \rangle$. \\*
	סופיות נתונה בקבוצה.
	\[
		\{1, x, x^2, \hdots \} \subseteq H
	\]
	מסופיות קיימים $0 \le k < m$ עבורם
	\[
		x^k = x^m \implies x^{m - k} = 1
	\]
	ולכן ל־$x$ יש סדר סופי, משובך היונים. \\*
	$2$ תרגיל.
\end{proof}
\begin{conclusion}[משפט לגרנז' לחבורה סופית]
	$G$ חבורה סופית, אז לכל $x \in G$ מתקיים
	\[
		o(x) \Big| |G|
	\]
\end{conclusion}
\begin{conclusion}
	אם קיים $x \in G$ עבורו $o(x) = |G|$ אז $G$ ציקלית.
\end{conclusion}

\begin{proposition}[בסיס למשפט השאריות הסיני]
	לכל $a, b \ge 1$ זרים אז $\gcd(a, b) = 1$, מתקיים
	\[
		\ZZ/a \times \ZZ/b \cong \ZZ/ab
	\]
\end{proposition}
\begin{proof}
	נראה שהסדר של $x = (1, 1) \in \ZZ/a \times \ZZ/b$ הוא $ab$ ונסיק מההבחנה. \\*
	ראשית, $x^{ab} = (ab, ab) = (0, 0) = 1$. \\*
	מצד שני, אם $x^n = 1$ אז $(n, n) = (0, 0) \in \ZZ/a \times \ZZ/b$ כלומר
	\[
		0 = n \in \ZZ/a, \qquad 0 = n \in \ZZ/b
	\]
	ולכן $a | n, b | n$, $a, b$ זרים ולכן $ab | n$. \\*
	מכיוון ש־$|\ZZ/a \times \ZZ/b| = |\ZZ/a| \cdot |\ZZ/b| = ab$ \\*
	נובע ש־$\ZZ/a \times \ZZ/b$ ציקלית מגודל $ab$ ולכן איזומורפית ל־$\ZZ/ab$.
\end{proof}

\subsection{פעולות של חבורה על קבוצה}
נתעסק בחבורות לא אבליות ואיך הן מופיעות כסימטריות פעמים רבות.
הסיבה שאנחנו מתעסקים בחבורות היא לראות את הפעולות שלהן על דברים.
\begin{definition}[פעולה]
	פעולה של חבורה $G$ על קבוצה $X$ זו פונקציה $\cdot : G \times X \to X$, $(g, x) \mapsto g \cdot x$ כך שמתקיים:
	\begin{enumerate}
		\item $1 \cdot x = x$ לכל $x \in X$.
		\item $h \cdot (g \cdot x) = (hg) \cdot x$ לכל $x \in X, g, h \in G$.
	\end{enumerate}
	סימון: $G \acts X$. באנגלית Group action.
\end{definition}
\begin{example}[לפעולות]
	מספר פעולות:
	\begin{enumerate}
		\item $S_n$ פועלת על הקבוצה $X = \{1, 2, \hdots, n\}$ על־ידי
			\[
				S_n \times \{1, \hdots n\} \to \{1, \hdots, n\}
			\]
			כאשר $(\sigma, k) \mapsto \sigma(k)$.
		\item $D_n \le S_n$ כפי שהגדרנו בתרגיל. \\*
			$D_n$ פועלת על $\{1, 2, \hdots, n \}$ באותו אופן כמו $S_n$, והיא אינטואיטיבית שקולה לביצוע פעולה סימטרית נתונה על מצב מסוים של הריבוע.
		\item $\RR^n \acts GL_n(\RR)$ על־ידי
			\[
				GL_n(\RR) \times \RR^n \to \RR^n, \qquad (A, v) \mapsto Av
			\]
			קבלת וקטור ומטריצה וכפל הווקטור במטריצה. \\*
			$\RR^n \acts O_n(\RR) \le GL_n(\RR)$ פעולה אורתוגונלית על וקטורים, שקול למעשה ל־$S^{n - 1}$. \\*
			$SO_2(\RR) = O_2(\RR) \cap SL_n(\RR)$. אף היא פעולה על $\RR$. \\*
			\textbf{הערה:} הסימון $O(n) = O_n(\RR)$ הוא קבוצת האורתוגונליים על $\RR$, באופן דומה $SO_n(\RR)$ קבוצת האורתוגונליים עם דטרמיננטה $1$.
		\item דוגמה 0: המקרה הטריוויאלי, כל חבורה $G$ ולכל קבוצה $X$ יש את הפעולה הטריוויאלית של $G$ על $X$ והיא
			\[
				g \cdot x = x, \forall g \in G, x \in X
			\]
	\end{enumerate}
\end{example}

הרציונל מאחורי ההגדרה הזאת הוא שאנחנו יכולים לפרק את החבורות מתוך פעולות שאנחנו כבר מכירים ולחקור את התכונות של הפעולות האלה באופן ריגורזי ושיטתי.
נשים לב לדוגמה ש־$D_4 \acts \{ D_1, D_2 \}$, אנחנו יכולים לחקור את המקרה היחסית טריוויאלי הזה של סימטריה גאומטרית על־ידי הגדרת הפעולה המתאימה.

\begin{definition}[אינבולוציה]
	נבחן את הפעולה של $\ZZ/2$ על $X$. האיבר הנייטרלי לא עושה כלום ולכן קל להגדיר אותו, יש להגדיר פעולה רק עבור איבר לא נייטרלי. \\*
	זה אותו דבר בגדול כמו פונצקיה $\tau : X \to X$ שמקיימת $\tau \circ \tau = Id_X$, זאת שכן
	\[
		\ZZ/2 \times X \to X, \qquad g \cdot x \mapsto \begin{cases}
			x, & g = 0 \\
			\tau(x), & g = 1
		\end{cases}
	\]
	לפונקציה כזאת קוראים אינבולוציה, פעולה שריבועה הוא $Id$, באנגלית Involution, וכבר ראינו פונקציות רבות כאלה.
\end{definition}

כדוגמה יש לנו לפחות שלוש פעולות $\ZZ/2$ על $\RR^2$ כאלה
\[
	\tau( \begin{bmatrix} x \\ y \end{bmatrix}) = \begin{bmatrix} -x \\ y \end{bmatrix}, \qquad
	\tau( \begin{bmatrix} x \\ y \end{bmatrix}) = \begin{bmatrix} x \\ -y \end{bmatrix}, \qquad
	\tau( \begin{bmatrix} x \\ y \end{bmatrix}) = \begin{bmatrix} x \\ y \end{bmatrix}
\]

\begin{definition}[הפעולה הרגולרית]
$G$ חבורה, \textbf{הפעולה הרגולרית (השמאלית)} של $G$ על $G$ שנתונה על־ידי
\[
	g \cdot x = gx
\]
פעולה המוגדרת על־ידי הכפל של החבורה. זוהי כמובן פעולה והסימון הוא $G \acts G$.
\end{definition}
האם פעולה ימנית גם עומדת בהגדרת הפעולה? \\*
נבדוק את $G \times G \to G$ המוגדרת על־ידי $(g, x) \mapsto xg$: \\*
נבדוק אסוציאטיביות
\[
	h \cdot (g \cdot x) = h \cdot (xg) = (xg)h, \quad (hg) \cdot x = x (hg), \quad (xg) h \ne x (hg)
\]
ומצאנו כי הביטויים לא שווים ואין שמירה על אסוציאטיביות כחלק מהגדרת הפעולה, ולכן כמובן זוהי לא פעולה. \\*
נשתמש במקום זאת בהופכית ונגדיר $(g, x) \mapsto x g^{-1}$ \\*
פעולה זאת היא אכן פעולה מוגדרת והיא נקראת \textbf{הפעולה הרגולרית הימנית}.

יש עוד פעולה מעניינת של חבורה על עצמה, על־ידי הצמדה
\begin{definition}[הצמדה]
	\[
		G \times G \to G, \quad (g, x) \mapsto x g x^{-1}
	\]
	היא פעולת ההצמדה, נחקור אותה בתרגיל. באנגלית Conjugacy.
	באופן דומה הפעולה היא $conjugate$.
\end{definition}

בהינתן פעולה של $G \acts X$ נגדיר פונצקיה $f : G \to Sym(X) \subseteq End(X)$ על־ידי
\[
	f(g)(x) = g \cdot x
\]
זאת שכן $G \times X \to X$ שקול ל־$G \to \{ X \to X \}$.

\begin{proposition}[הצמדה היא הומומורפיזם]
	$f$ היא הומומורפיזם של חבורות.
\end{proposition}
\begin{proof}
	\[
		f(hg)(x) = (hg) \cdot x = h \cdot (g \cdot x) = f(h)(g \cdot x) = f(h)(f(g)(x)) = (f(h) \cdot f(g))(x)
	\]
\end{proof}

למה $f(g) \in Sym(X)$? \\*
כי $f(g^{-1}) \cdot f(g) = f(g^{-1} g) = f(1) = Id$.
גם $f(g) \cdot f(g^{-1}) = f(g g^{-1}) = f(1) = Id$.

בשיעור הבא נגדיר המון דברים על פעולות על קבוצות, אז צריך להבין את זה ואת הדוגמות באופן מאוד כבד ושלם.

\section{תרגול 3 --- 21.5.2024}
\subsection{שאלות מתרגיל 1}
\subsubsection{שאלה 1}
\[
	End(X) = \{ f : X \to X \}
\]
והיה צריך להוכיח שזה מונואיד. וזה חבורה רק כשהקבוצה היא הקבוצה הריקה או יחידון או משהו כזה. \\*
הסעיף השני הוא שיהא $M$ מונואיד כך שלכל $x \in M$ קיים הופכי משמאל ומראים ש־$M$ חבורה.
\begin{proof}[פתרון]
	יש לי $x \in M$ וצריך להראות שקיים $y \im M$ כך ש־$xy = yx = e$. \\*
	נתון קיום של $y \in M$ כך ש־$yx = e$ ואנחנו רוצים להראות שגם $xy \in M$.
	\[
		xy = e \implies {(xy)}^2 = e = x (yx) y = xy = e
	\]
	ולכן $\exists t \in M: tz = e$ ונקבל $z = tz^2 = tz = e$.
\end{proof}
עכשיו נגיד שיש לנו מונואיד $M$ כך ש־$x \in M$ ול־$x$ יש הופכי מימין והופכי משמאל וצריך להראות שהם שווים.
\begin{proof}[פתרון]
	קיימים $y, z, xz = yx = e$. \\*
	לכן
	\[
		z = ez = (yx) z = y (xz) = y
	\]
\end{proof}
הסעיף האחרון הוא לתת דוגמה לאיבר במונואיד עם הופכי משמאל ולא מימין. \\*
נבחן את $End(\NN)$ ונבחר את $f(x) = x + 1$ ו־$g(x) = \begin{cases}
	1, & x = 1 \\
	n - 1, & n > 1
\end{cases}$

\subsubsection{שאלה 4}
סעיף ב', צריך להראות שזה איזומורפי
\[
	\varphi : (\RR^\times, \cdot) \to \ZZ/2 \times \RR^+
\]
ונאחנו משתמשים בבינאריות של $\ZZ/2$, ואנחנו יודעים שלוגריתם משמר פעולות.
\[
	\varphi(x) = \begin{cases}
		(1, \ln|x|), & x < 0 \\
		(0, \ln|x|), & x > 0
	\end{cases}
\]
ועכשיו לסעיף ג': \\*
צריך למצור פונקציה
\[
	\varphi : GL_2(\ZZ/2) \xrightarrow{\sim} S(\{v_1, v_2, v_3\}), \qquad v_1 = (1, 0), v_2 = (0, 1), v_3 = (1, 1)
\]
\[
	\varphi(T) = \begin{pmatrix}
		v_1 & v_2 & v_3 \\
		T(v_1) & T(v_2) & T(v_3)
	\end{pmatrix}
\]
\[
	\varphi(T) \varphi(S) =
	\begin{pmatrix}
		v_1 & v_2 & v_3 \\
		T(v_1) & T(v_2) & T(v_3)
	\end{pmatrix}
	\begin{pmatrix}
		v_1 & v_2 & v_3 \\
		S(v_1) & S(v_2) & S(v_3)
	\end{pmatrix}
	=
	\begin{pmatrix}
		v_1 & v_2 & v_3 \\
		T(S(v_1)) & T(S(v_2)) & T(S(v_3))
	\end{pmatrix}
\]
וזה מן הסתם עובד די טוב. אז בקיצור זה איזומורפיזם. ועכשיו נתחיל באשכרה תרגול.

\subsection{מחלקות שקילות}
\begin{definition}
	תהא $G$ חבורה, ו־$H \le G$.
	מחלקות השקילות השמאליות של $H$ הן קבוצות מהצורה $g H, g \in G$.
\end{definition}
\begin{lemma}[תכונות מחלקות שקילות]
	תהי $H \le G$ חבורה ותת־חבורה, אז הטענות הבאות מתקיימות:
	\begin{enumerate}
		\item $gH = H \iff g \in H$
		\item אם $H$ סופית אז לכל $g \in G$ מתקיים $|g H| = |H|$.
		\item $\forall g \in G : g H = H g \iff g H g^{-1} \subseteq H$.
		\item ישנה התאמה בין הקבוצות $g H$ ל־$H g$.
	\end{enumerate}
\end{lemma}
\begin{definition}[אינדקס]
	תהי $H \le G$ חבורה ותת־חבורתה. \\*
	נגדיר $[G : H]$ להיות מספר המחלקות השמאליות של $H$.
	אם מספר זה אינסופי אז נגדיר את האינדקס $[G : H] = \infty$.
	מספר זה נקרא אינדקס של $H$ ב־$G$.
\end{definition}
\begin{example}
	נתבונן ב־$D_3$.
	חבורת הסימטריות על משולש שווה צלעות.
	יש לנו שלושה צירי סימטריה, ויש לנו שלושה סיבובים לעשות.
	\[
		D_3 = \{ r, r^2, f, fr, fr^2 \}
	\]
	וזה מן הסתם מקיים $D_3 = \langle r, f \rangle$. \\*
	נגדיר $H_1 = \{e, f_2\}, H_2 = \{e, r, r^2\}$. \\*
	נראה כי מחלקות שקילות הן:
	\[
		r H_1 = \{ r, rf \},
		r^2 H_1 = \{ r^2, r^2f \},
		H_1 = H_1
	\]
	ומהצד השני:
	\[
		H_1 r = \{ r, fr \},
		H_1 r^2 = \{ r^2, f r^2 \}
	\]
	ועבור $H_2$:
	\[
		f H_2 = \{ f, fr, fr^2 \}, etc
	\]
\end{example}
עתה נדבר על סדר.
\subsection{משפט לגרנז'}
\begin{definition}[סדר של איבר]
	תהא $G$ חבורה סופית ו־, לכן $g \in G$ נגדיר את הסדר של $g$, או $|g| = ord(g)$ הוא המינימום של המספרים הטבעיים כך ש־$g^n = e$.
\end{definition}
\begin{theorem}[משפט לגרנז']
	תהא $G$ חבורה סופית ו־$H$ תת־חבורה של $G$. אז
	\[
		[G : H] = \frac{|G|}{|H|}
	\]
	ובפרט $|H| \Big| |G|$.
\end{theorem}
\begin{conclusion}
	תהא $G$ סופית ו$g \in G$ אז $ord(g) \Big| |G|$.
\end{conclusion}
\begin{proof}
	על־ידי התבוננות ב־$H = \langle g \rangle$.
\end{proof}
\begin{lemma}
	$|H| = ord(g)$.
\end{lemma}
\begin{proof}
	נגדיר $\varphi : \ZZ/ord(g) \to H$ על־ידי $\varphi(b) = g^n$. \\*
	נראה כי $\varphi$ חד־חד ערכית ועל. \\*
	יהיו $n, m \in \ZZ/ord(g)$ ונניח כי $\varphi(n) = \varphi(m)$, אזי $g^n = g^m$ ולכן $g^{n - m} = e$ ולכן $n - m = 0$, שאם לא כן יש סתירה למינימליות של $ord(g)$. \\*
	מה החבורה הנוצרת על־ידי $\langle g \rangle = \{ g^n \mid n \in \NN\}$. \\*
	יהא $n \in \ZZ$ נחלק את $n$ עם שארית בסדר של $g$, $n = m \cdot ord(g) + r$ ו־$r \in \ZZ/ord(g)$. לכן $g^n = g^{m \cdot ord(g) + r} = g^r$. \\*
	הראינו כי $|H| = ord(g)$ ולכן הסדר של $ord(g) \Big| |G|$.
\end{proof}
\begin{conclusion}
	תהיה $G$ חבורה סופית.
	\[
		\forall g \in G, g^{|G|} = e
	\]
\end{conclusion}
\begin{proof}
	לפי המסקנה הקודמת
	\[
		g^{|G|} = g^{k \cdot ord(g)} = g^{ord(g)} = e
	\]
\end{proof}
\begin{conclusion}
	יהיה $p$ ראשוני, ו־$G$ חבורה מסדר $p$. אז
	\begin{enumerate}
		\item $G$ ציקלית.
		\item $G$ איזומורפית ל־$\ZZ/p$.
		\item כל החבורות מגודל $p$ איזומורפיות.
	\end{enumerate}
\end{conclusion}
\begin{proof}
	$G$ היא לא חבורה טריוויאלית בגלל $p$ ולכן נוכל להגדיר $g \in G \setminus \{e\}$. \\*
	נשים לב כי $1 < ord(g)$ אך מצד שני $|\langle g \rangle| = ord(g) \big|p$ \\*
	לכן $\langle g \rangle = G, |\langle g \rangle| = p$. \\*
	סעיף ב' בתרגיל 2.
\end{proof}
\begin{theorem}[משפט פרמה הקטן]
	יהיה $p$ ראשוני, ו־$a \in \ZZ$, אם $\gcd(a, p) = 1$ אז $a^{p - 1} \equiv 1 (\mod p)$
\end{theorem}
\begin{proof}
	נתבונן בחבורה הכפלית של $\ZZ/p$, מסומנת $\ZZ^\times_{/p}$ שהוא השדה בלי 0 \\*
	הגודל של $\ZZ^\times_{/p}$ הוא $p - 1$ ולכן לכל $x$ בחבורה הזאת $x^{p - 1} =_{\ZZ/p} 1$. \\*
	כעת נחלק את $a$ ב־$p$ עם שארית, ונקבל $a = np + r$ כאשר $0 < r \le p - 1$, וזה נכון כי הם זרים, דהינו $r \in \ZZ^\times_{/p}$. \\*
	נשים לב כי
	\[
		a^{p - 1} = {(mp + r)}^{p - 1} \implies a^{p - 1} = {(mp + r)}^{p - 1} \mod p = \sum_{i = 0}^{n - 1} \binom{p - 1}{i} {(m p)}^{p - 1} \cdot r
		= r^{p - 1} \mod p
	\]
	לכן $a^{p - 1} = r^{p - 1} = 1$.
\end{proof}

\subsection{שאלה 4 סעיף א'}
היה צריך למצוא תת־חבורה של $GL_n(\FF)$ שאיזומורפית ל־$S_n$.
\begin{proof}[פתרון]
	אוסף מטריצות הפרמוטציה, $H = \{ A \in M_n(\FF) \mid \text{בכל שורה או עמודה יש איבר בודד שאיננו אפס והוא אחת}\}$. \\*
	המטריצות האלה הן כידוע מטריצות שפשוט מחליפות אגפים בווקטורים ולמעשה זה פשוט תמורה על הווקטורים מסדר $n$. \\*
	$S_n = S([n])$ ולכן נגדיר $\varphi : H \to S_n$ על־ידי $\varphi(A) = \text{התמורה שפועלת על $A$}$.
\end{proof}

\section{שיעור 5 --- 22.5.2024}
נניח שיש לי $G$ חבורה סופית. מלגרז' נובע ש־$H \le G \implies |H| \Big| |G|$.
משפט קושי אומר שאם $p \Big| |G|$ ו־$p$ ראשוני אז קיימת חבורה $H \le G$ כך ש־$|H| = p$. למעשה קיים $x \in G$ עם $o(x) = p$.

\subsection{פעולות על קבוצות}
\begin{notation}
	בהינתן $G \acts X$ נסמן עבור $x, y \in X$ את $x \sim y$ כיחס שמתקיים אם $\exists g \in G : g \cdot x = y$.
\end{notation}
במילים פשוטות, שני איברים בקבוצה הם דומים אם קיים איבר בחבורה שמוביל מאחד מהם לשני. רעיונית מדובר בסימטריה, ולכן הגיוני לשאול אם שני מצבים הם סימטריים ללא קשר למה הפעולה שמשרה את הסימטריה.

\begin{proposition}[יחס שקילות בפעולה על קבוצות]
	$\sim$ הוא יחס שקילות.
\end{proposition}
\begin{proof}
	נבחין כי הגדרת יחס השקילות מתקיימת:
	\begin{itemize}
		\item רפלקסיבי $e \cdot x = x$.
		\item סימטרי: $x \sim y \implies \exists g \in G g \cdot x = y \implies g^{-1} y = x \implies y \sim x$.
		\item טרנזיטיבי: $x \sim y, y \sim z \implies \exists g, h \in G, gx = y, hy = z \implies (hg)x = h (gx) = hy = z \implies x \sim z$
	\end{itemize}
\end{proof}
משמעות הדבר היא שסימטריות הן שקולות. שוב, מדובר ברעיון מאוד הגיוני שכן אם בוחנים את הכול בעיניים של סימטריה. כלל המצבים שסימטריים בזוגות גם סימטריים בכללי.

\begin{definition}[מסלולים]
	בהינתן $G \acts X$, המסלולים של $G$ הם מחלקות השקילות של $\sim$ והמסלול של $x \in X$ הוא
	\[
		O(x) = \{ y \in X \mid y \sim x\} = \{ y \in x \mid \exists g \in G : g \cdot x = y \}
	\]
	\textbf{סימון:} קבוצת המסלולים מסומנת $G \backslash X$.
\end{definition}
\begin{conclusion}
	$X = \bigcup_{O \in G\backslash X} O$, דרך מזעזעת להגיד שהקבוצה המקורית מורכבת מהחלוקה למסלולים שלה.
\end{conclusion}
מהותית אנו מדברים פה על החלוקה של $X$ לפי השקילות, בכל קבוצה יהיו רק איברים ששקולים אחד לשני.
\begin{definition}[נקודת שבת]
	$x \in X$ נקודת שבת של $G$ אם $|O(x)| = 1$.\\*
	כלומר $\forall g \in G : g \cdot x = x$.
\end{definition}
הרעיון הוא שהפעולה על איבר מסוים תמיד מחזירה אותו עצמו, ללא קשר לאיזו סימטריה מהחבורה אנחנו בוחרים.
\begin{definition}[טרנזיטיבית]
	פעולה $G \acts X$ נקראת טרנזיטיבית אם $|G \backslash X| = 1$. \\*
	הפעולה היא טרנזיטיבית אם יש רק קבוצת מסלולים (שהיא חלוקת שקילות) אחת, דהינו שכל איבר בקבוצה סימטרי לכל איבר אחר.
\end{definition}
\begin{conclusion}
	$H \backslash G$ קבוצת המסלולים של $H \acts G$ רגולרית משמאל שקולה ל־$H \backslash G$ קבוצת המחלקות הימניות של $H$ ב־$G$.\\*
	באופן דומה $G / H$ המסלולים של הפעולה $H \acts G$ הרגולרית מימין. \\*
	יש פה התכנסות מאוד אלגנטית גם של הרעיון של מחלקות ימניות ושל השקילויות מבחינת רגולרית משמאל, זו הרי מהותית מגדירה הכפלה של האיברים משמאל, ולכן גם המסלולים מעל התת־חבורה הם המחלקות האלה.
\end{conclusion}
\begin{example}
	נבחין בכמה פעולות שונות וחשובות:
	\begin{enumerate}
		\item $G \acts G$ פעולה רגולרית שמאלית. $\forall x, y \in G, x \sim y \iff g \in G : gx = y$ ותמיד קיים $g$ כזה והוא אף יחיד, $g = yx^{-1}$. לכן יש מסלול אחד והפעולה טרנזיטיבית.
		\item יהי $H \le G$, ונבחן את $H \acts G$, רגולרית משמאל, הפעם $x \sim y \iff \exists h \in H : hx = y \iff yx^{-1} \in H \iff Hx = Hy$ מחלקות ימניות. \\*
			מצאנו הפעם כי יש מסלול בין איברים רק אם הם באותה מחלקה ימנית (על אף שמדובר על רגולרית שמאלית). נראה את המסקנה האחרונה.
		\item $GL_2(\RR) \acts \RR^2$ מטריצות הפיכות פועלות על המרחב $\RR^2$. \\*
			מסלולים: $\{ \{ 0 \}, \RR^2 \setminus \{ 0 \} \}$.\\*
			ביתר פירוט, מטריצות הפיכות משמרות את האי־איפוס, אבל כן נוכל להגיע מכל וקטור לכל וקטור אחר עם המטריצה הנכונה.
			לעומת זאת וקטור אפס ישאר אפס מכל מטריצה שתוכפל בו, ולכן הוא לא סימטרי לאף וקטור אחר בפעולה.
		\item $O_2(\RR) \acts \RR^2$, ידוע כי $O_2(\RR) \le GL_2(\RR)$. הפעם כל וקטור צריך להגיע רק לווקטור מאותו גודל.\\*
			מסלולים: $\{ \{0\}, \{ \{ v \in \RR^2 \mid |v| = a\} \mid a > 0 \} \}$. \\*
			לכל וקטור שנבחר, כל מטריצה בחבורה משמרת את הנורמה שלו, אבל לא את הכיוון, ובהתאם נוכל להסיק שכל שני וקטורים עם אותה נורמה שקולים ונמצאים באותה קבוצה.
		\item $S_n \acts \{1, \dots, n\}$ הפעולה הזו היא טרנזיטיבית. \\*
			זה די טריוויאלי בגדול, נוכל לסדר מחדש את רשימת המספרים בכל דרך על־ידי איזושהי תמורה, ובהתאם כל הסדרים דומים אחד לשני ויש ביניהם מסלול.
		\item כל הדגלים שמחולקים לשלושה פסים בשלושה צבעים, וכל האופציות לבחור את של שלושת הצבעים. יש מן הסתם שמונה דגלים כאלה.\\*
			אפשר להגדיר פעולה $\ZZ/2$ של סיבוב ב־$180^\circ$ ואז אפשר לראות אילו דגלים מתקשרים לאילו דגלים אחרים. יש שישה מסלולים.
	\end{enumerate}
\end{example}

\begin{definition}[מקבע]
	תהינה $G \acts X$, עבור $g \in G$, ונגדיר את ה\textbf{מקבע} להיות $ Fix(g) = \{ x \in X \mid gx = x \}$.\\*
	עוד סימון הוא $X^g$, אבל לא מומלץ להשתמש בו, הוא יחסית מבלבל. \\*
	עבור איבר בחבורה, המקבע הוא כל האיברים בקבוצה שהפעולה לא משנה, הם לא בהכרח נקודות שבת כי אנחנו מדברים פה בהקשר של סימטריה ספציפית.
\end{definition}
\begin{definition}[מייצב]
	יהיו $G \acts X$, אז נגדיר  את המייצב של $x \in X$ להיות $Stab(x) = \{ g \in G \mid g x = x \}$, באנגלית Stabilizer. \\*
	סימון נוסף הוא $G_x$. \\*
	במילים זוהי קבוצת איברי החבורה שלא משנים את $x$, או לחילופין שולחים אותו לעצמו.
\end{definition}
האינטואציה היא שיש איברים שסימטריות מסוימות פשוט לא משפיעות עליהם, ובהתאם המייצב הוא קבוצת הסימטריות הכאלה שנייטרליות לאיבר שבחרנו.

\begin{lemma}[מייצב הוא תת־חבורה]
	$G_x$ תת־חבורה של $G$.
\end{lemma}
\begin{proof}
	נבדוק את הגדרת תת־החבורה:
	\begin{enumerate}
		\item איבר נייטרלי: $e \cdot x = x \implies e \in G_x$.
		\item סגירות לכפל: $\forall g, h \in G, g \cdot x, h \cdot x = x \implies (gh) \cdot x = g \cdot (h \cdot x) = g \cdot x = x \implies gh \in G_x$.
		\item קיום הופכי: $g \in G \implies g \cdot x = x \implies x = g^{-1} \cdot x \implies g^{-1} \in G_x$.
	\end{enumerate}
	מצאנו כי כלל התכונות מתקיימות ולכן $G_x$, המייצב של $x$, הוא תת־חבורה של $G$.
\end{proof}
\begin{definition}[פעולה חופשית]
	$G \acts X$ נקראת חופשית אם $G_x = \{e\}$ לכל $x \in X$. במילים אחרות, הפעולה לעולם לא שולחת איבר לעצמו. \\*
	היא נקראת נאמנה אם $\bigcap_{x \in X} G_x = \{e\}$, החיתוך הזה בכללי גם נקרא גרעין.
\end{definition}
נאמנה זה שם קצת מוזר אבל הוא בגדול מבטיח שאין איבר בחבורה שכל איברי הקבוצה נייטרליים אליו, חוץ מהאיבר הנייטרלי עצמו. \\*
עניין הגרעין הוא די דומה למה שקורה בלינארית גם, איבר שהפעולה איתו לא משפיעה על אף איבר בקבוצה.
\begin{definition}
	נבחן את $G \acts G$ על־ידי הצמדה.
	\[
		O(x) = \{ g x g^{-1} \mid g \in G\}
	\]
	המסלול של $x$ הוא קבוצת האיברים שמקיימים $g x g^{-1} = y$, באופן מאוד דומה למטריצות דומות.
	נקרא למסלול הזה מחלקת צמידות.
\end{definition}
\begin{definition}[מרכז]
	ישנו המרכז של $x$ ב־$G$ והוא $C_G(x) = G_x = \{ g \in G \mid g x g^{-1} = x\} \iff gx = xg$. באנגלית Centrilizer.\\*
	מרכז הוא סוג של מייצב במקרה שבו $X = G$.
\end{definition}
\begin{theorem}[מסלול־מייצב]
	$G \acts X$ ו־$x \in X$. $|O(x)| = [G : G_x]$. וזה נכון גם כשהחבורה לא סופית. $O(x) \xrightarrow{\sim} G/G_x$.\\*
	בפרט אם $G$ סופית אז $|O(x)| = \frac{|G|}{|G_x|}$ וונובע שהגודל של כל מסלול מחלק את גודל החבורה. \\*
	במילים הטענה היא שהמסלול של $x$, שהוא מספר האיברים שאפשר להגיע אליהם ממנו, שווה לאינדקס של המייצב, דהינו מספר מחלקות השקילות השונות שאפשר ליצור בעזרת מחלקות שמאליות עם התת־חבורה שלא מושפעת מ־$x$.
\end{theorem}
\begin{proof}
	נגדיר $f : G/G_x \to O(x)$ ונראה שהיא חד־חד ערכית ועל.\\*
	נבחר $f(gG_x) = g \cdot x$. זה לא בהכרח מוגדר היטב ולכן נבדוק למה זה כן.\\*
	אם יש איבר $g' \in gG_x$ אז $g' = g \cdot h$ כך ש־$h \in G_x$. מתקיים ש־$g' \cdot x = g h x \overset{h \in G_x}{=} g \cdot x$.\\*
	על: לפי הגדרה. \\*
	חד־חד ערכי: נניח ש־$g \cdot x = f(g G_x) = f(g'G_x) = g' \cdot x = {(g')}^{-1} g x = x \implies {(g')}^{-1} g \in G_x \overset{\text{סגירות להופכי}}{\implies} g'G_x = g G_x$.
\end{proof}
\begin{example}
	תהינה חבורה $H \le G$ ותת־חבורתה, יש פעולה ''רגולרית'' של $G$ על $G / H$:
	\[
		g \cdot (x H) = (g \cdot x) H
	\]
\end{example}

\begin{theorem}[משפט קושי]
	יהיו $G$ חבורה סופית ו־$p$ ראשוני כך ש־$p \Big| |G|$. אז קיים $x \in G$ כך ש־$ord(x) = p$.
\end{theorem}
\begin{proof}
	נגדיר פעולה של החבורה $\ZZ_{/p}$ על הקבוצה $X = \{ (g_1, \dots, g_p) \in G^p \mid g_1g_2 \cdots g_p = e\}$. \\*
	הפעולה פועלת על־ידי שיפט ציקלי: $u \in \{0, 1, \dots, p - 1\}$ אז $k \cdot (g_1, \dots, g_p) = (g_{k + 1}, g_{k + 2}, \dots, g_{p \mod p}, g_{1}, \dots, g_k)$.\\*
	אז $k (g_{k + 1}, \dots, g_p) = e$ וגם $(g_{k + 1}, \dots, g_p)(g_1, \dots, g_k) = e$. \\*
	נבחין כי כלל המסלולים בפעולה הם אחד משני סוגים:
	\begin{itemize}
		\item מסלולים בגודל $p$. אם לא כל האיברים זהים, מעגל שלם יקח ככמות האיברים והיא מוגדרת להיות $p$.
		\item מסלולים בגודל $1$. אם כל האיברים זהים אז שיפט יחזיר את האיבר עצמו.
	\end{itemize}
	ממשפט מסלול־מייצב $|O(x)| \Big| p \iff |O(x)| = 1, p$.\\*
	עתה נבחין כי אם ישנו מסלול בגודל $p$ אז הוא כמובן ממלא את טענת ההוכחה ולכן נניח שאין כזה. \\*
	נראה כי מסלול בגודל $1$ הוא מסלול שמקיים $(g_1, \dots, g_p) = (g_2, \dots, g_p, g_1)$ כלומר $x = (g, \dots, g)$, $g^p = e$.\\*
	נשים לב כי נוכל לחלק את הקבוצה המקורית ומתקיים $X = \displaystyle \bigcup_{O \in \ZZ_{/p} \backslash X} O$, ובהתאם מהאיחוד הזר נקבל גם $|X| = \displaystyle \sum_{O \in \ZZ_{/p} \backslash X} |O|$. \\*
	אם $(e, \dots, e)$ היה נקודת השבת היחידה אז $\displaystyle \sum_{O \in \ZZ_{/p} \backslash X} |O| = 1 (\mod p)$, שכן כל מסלול כולל $p$ חילופים ונקודת השבת היחידה הייתה תורמת $1$ בלבד. \\*
	לכן מצד אחד $p \Big| |G|^{p - 1}$ ומצד שני $|G|^{p - 1} \cong 1 (\mod p)$ ולכן קיים $x \ne e$ עם $x^n = e$.
\end{proof}
\begin{remark}
	ההוכחה מוויקיפדיה הרבה יותר ברורה.
\end{remark}

\section{שיעור 6 --- 27.5.2024}
\subsection{מקבעים של פעולות}
ניזכר בהגדרת המקבע, $X^g = \{ x \in X \mid gx = x \}$, דהינו האיברים ב־$X$ שלא משתנים על־ידי הסימטריה $g$.

לדגומה עבור החבורה $D_4$ ו־$X = \{1, 2, 3, 4\}$ אוסף קודקודי ריבוע נבחן את $g$ סיבוב על האלכסון: $g = (1\ 3)$ ואת $h = (1\ 2)(3\ 4)$ שיקוף על האמצע.
אז כמובן המקבע של $g$ ב־$X$ הוא $X^g = \{1, 3\}$ אוסף הקודקודים שלא מושפעים מהסימטריה $g$.
באופן דומה $X^h = \emptyset$, דהינו הסמיטריה $h$ תמיד משנה את כל הקודקודים ובהתאם המקבע הוא ריק.

\begin{lemma}[הלמה של ברנסייד]
	תהיה חבורה סופית $G$ ופעולה $G \acts X$ כאשר $X$ סופית גם היא.
	יהי $g \in G$ ונסמן $Fix(g) = X^g = \{ x \in X \mid gx = x\} \subseteq X$. \\*
	אז מספר המסלולים (מסומן גם $X/G$) הוא
	\[
		|X/G| = \frac{1}{|G|} \sum_{g \in G} |Fix(g)|
	\]
	דהינו ממוצע כמות האיברים שנשארים במקום היא ככמות המסלולים השונים.
\end{lemma}
\begin{proof}
	תהי חבורה סופית $G$. עבור $X$ סופית ופעולה $G \acts X$ נגדיר
	\[
		E(X) = \frac{1}{|G|} \sum_{g \in G} |X^g|
	\]
	נוכיח כי $E(x) = |X/G|$.

	נשים לב שאם $X, Y$ קבוצות זרות עם פעולה של $G$, אז נובע מהזרות ומהגדרת המסלולים של הקבוצות כי
	\[
		(X \sqcup Y)/G = X/G \sqcup Y/G
		\implies
		|(X \sqcup Y)/G| = |X/G| + |Y/G|
	\]
	עוד נראה כי	$\forall g \in G : {(X \sqcup Y)}^g = X^g \sqcup Y^g$
	ולכן גם $|{(X \sqcup Y)}^g| = |X^g| + |Y^g|$,
	ונוכל להסיק ש־$|E(X \sqcup Y)| = E(X) + E(Y)$. \\*
	נסיק כי אילו הלמה נכונה עבור $X, Y$ זרים, אז היא מתקיימת גם עבור איחודם $X \sqcup Y$.

	תהי $X$ קבוצה כלשהי, נוכל לכתוב גם
	\[
		X = \bigsqcup_{O \in G\backslash X} O
	\]
	במילים ש־$X$ היא איחוד זר של קבוצות המסלולים השונות שמוגדרות על־ידי הפעולה $G \acts X$. \\*
	על־כן מהטענה שהוכחנו זה עתה מספיק להוכיח את הטענה כאשר ל־$X$ יש מסלול יחיד $x = O$, ובמקרה הכללי נוכל לאחד איחוד זר של מסלולים. \\*
	נניח מעתה כל $X \ne \emptyset$ עם מסלול יחיד (פעולה טרנזיטיבית).
	במקרה הזה צריך להוכיח
	\[
		\frac{1}{|G|} \sum_{g \in G} |X^g| = E(X) = 1
	\]
	נגדיר עבור $x \in X, g \in G$ את $s(g, x)$ על־ידי
	\[
		s(g, x) = \begin{cases}
			1, & gx = x \\
			0, & gx \ne x
		\end{cases}
	\]
	דהינו $s$ מחזירה $1$ אם $g$ מקבעת את $x$. ואנו יודעים כי $X^g = \{ g \in G \mid gx = x \} = \{ g \in G \mid s(g, x) = 1 \}$.
	לכן נוכל להסיק שמתקיים
	\[
		|X^g| = \sum_{x \in G} s(g, x)
	\]
	ועתה נציב ונקבל כי
	\[
		\sum_{g \in G} |X^g|
		= \sum_{g \in G} \sum_{x \in X} s(g, x)
		= \sum_{x \in X} \sum_{g \in G}  s(g, x)
		\overset{(1)}{=} \sum_{x \in X} |G_x|
		\overset{(2)}{=} |X| \cdot |G_x| = |G|
	\]
	$(1)$ נובע ישירות מההגדרה של מייצב $G_x = \{ g \in G \mid g x = x\}$. \\*
	$(2)$ ממשפט מסלול־מייצב נקבל כי $|G| = |G_x| \cdot |O(x)|$ אבל ידוע שהפעולה טרנזיטיבית ולכן $\forall x \in X: O(x) = X$, לכן $|G| = |X| \cdot |G_x|$. \\*
	קיבלנו כי $|E(X)| = 1$ ולכן נוכל להסיק כי הטענה מתקיימת תמיד.
\end{proof}
\begin{example}
	בתזכורת הראינו כי עבור $D_4$ ו־$X = \{1, 2, 3, 4\}$ מתקיים $|X^g| = 2, |X^h| = 0$.
	נחשב את כלל המקבעים ונקבל על־פי הלמה:
	\[
		\frac{1}{8} (4 + 2 + 2 + 0 + 0 + 0 + 0 + 0) = 1 = |D_4 \backslash X|
	\]
	דהינו $D_4$ טרנזיטיבית לפי הלמה, שכן יש לה רק מסלול אחד.
\end{example}
דוגמה נוספת היא בחירת $G$ סופית ופעולה שלה על עצמה עם הצמדה. \\*
לכן $g(h) = ghg^{-1}$ ונשים לב כי המקבע הוא $C(g) = G^g = \{ h \in G \mid ghg^{-1} = h\}$. \\*
כמות מחלקות הצמידות --- היא מספר המסלולים על־פי הצמדה --- ניתנת לחישוב על־ידי
\[
	\frac{1}{|G|} \sum_{g \in G} |C(g)|
\]
\begin{definition}[מרכז חבורה]
	נגדיר את ה\textbf{מרכז} של חבורה $G$, המסומן $Z(G)$, להיות קבוצת האיברים שנייטרליים לסדר ההכפלה בהם:
	\[
		Z(G) = \{ h \in G \mid \forall g \in G : g h = h g\}
	\]
	לחילופין הגדרה שקולה היא קבוצת האיברים שצמודים לעצמם בלבד. \\*
	נגדיר גם $C_x$ מחלקת הצמידות של $x$, דהינו
	\[
		C_x = \{ g \in G \mid g x g^{-1} = x \}
	\]
\end{definition}
\begin{proposition}[מרכז הוא תת־חבורה]
	תהי $G$ חבורה, אז $Z(G) \subseteq G$ היא תת־חבורה.
\end{proposition}
\begin{proof}
	נראה כי תכונות החבורה חלות על $Z(G)$:
	\begin{enumerate}
		\item איבר נייטרלי: $\forall g \in G: eg = ge \implies e \in Z(G)$
		\item סגירות לכפל: $\forall a, b \in G : \forall g \in G, a b g = a g b = g a b \implies ab \in Z(G)$.
		\item סגירות להופכי: $n \in Z(G) : ng = gn \implies \forall g \in G n^{-1} g = g h^{-1}$
	\end{enumerate}
	לכן $Z(G)$ חבורה וחלקית ל־$G$ ולכן נובע $Z(G) \le G$.
\end{proof}

\begin{lemma}[חיתוך מרכזים]
	תהי $G$ חבורה, ניזכר כי המרכז של $x \in G$ מוגדר על־ידי
	\[
		C_G(x) = C(x) = \{ g \in G \mid g x g^{-1} = g \}
	\]
	ומתקיים
	\[
		Z(G) = \bigcap_{x \in G} C(x)
	\]
\end{lemma}
\begin{proof}
	נובע ישירות מההגדרות
\end{proof}
לכן נשים לב שחיתוך המרכזים הוא המרכז של החבורה, והיא תת־חבורה אבלית.

\begin{notation}[מחלקות צמידות]
	תהי חבורה $G$, אז נסמן את אוסף מחלקות הצמידות שלה:
	\[
		cong(G) := \{ X \subseteq G \mid \forall x, y \in X \exists g \in G : x = g y g^{-1} \}
	\]
\end{notation}
נשים לב שמרכז עבור צמידות מסומן באופן מיוחד, וגם כאן זהו סימון מיוחד עבור $G \backslash G$ עם פעולת ההצמדה. \\*
כל איבר ב־$cong(G)$ הוא קבוצה שכלל האיברים בה צמודים זה לזה.
נשתמש בהגדרת המרכז ונכתוב גם
\[
	cong(G) = \{ X \subseteq G \mid \forall x, y \in X : y \in C(x) \}
\]
ונסמן $[g] \in cong(G)$ איבר כלשהו מייצג מכל מחלקת צמידות. \\*
נסמן גם $C_h$ מחלקת הצמידות של $h$ ומתקיים
\[
	C_h = \{ g \in G \mid \exists k \in G : k h k^{-1} = g \}
\]

\begin{proposition}[נוסחת המחלקות]
	תהי חבורה סופית $G$, אז מתקיים
	\[
		|G| = |Z(G)| + \sum_{[h] \in cong(G), h \notin Z(G)} \frac{|G|}{|G_h|}
	\]
\end{proposition}
\begin{proof}
	תחילה נבחין כי נוכל לפרק את $G$:
	\[
		G = \bigsqcup_{[h] \in cong(G)} C_h
	\]
	ונבחין כי לכל $h \in G$ מתקיים
	\[
		h \in Z(G)
		\iff
		|C_h| = 1
		\iff
		\forall g \in G : g h g^{-1} = h
	\]
	אז נוכל לראות כי
	\[
		G = Z(G) \sqcup \bigsqcup_{[h] \in cong(G), h \notin Z(G)} C_h
	\]
	ומכאן נסיק
	\[
		|G|
		= |Z(G)| + \sum_{[h] \in cong(G), h \notin Z(G)} |C_h|
		\overset{\text{מסלול־מייצב}}{=} |Z(G)| + \sum_{[h] \in cong(G), h \notin Z(G)} \frac{|G|}{|G_h| (= |C_G(h))|}
	\]
\end{proof}

\section{תרגול 4 --- 28.5.2024}
\subsection{צביעות}
\begin{definition}[צביעה]
	תהי קבוצה $X$ ותהי צביעה עם $m$ צבעים, אז אז \textbf{צביעה} של $X$ עם $m$ היא פונקציה $f : x \to [m]$.
\end{definition}
הרעיון פה הוא שאנחנו יכולים לקחת את הקבוצה ולסווג לכל איבר בה צבע (מספר) ומן הסתם יש לנו ${[m]}^{|X|}$ צביעות רעיוניות כאלה.
\begin{proposition}[צביעה מעל פעולה]
	תהי קבוצה $X$ ו־$G \acts X$ חבורה ופעולה המסומנת על־ידי $\ldotp$, ויהי ${[m]}^X$ אוסף הצביעות ב־$m$ של $X$. \\*
	אז הפונקציה $\ldotp : G \times {[m]}^X \to {[m]}^X$ המוגדרת על־ידי
	\[
		\forall g \in G, f \in {[m]}^X, \forall x \in X : g \ldotp f(x) = f(g^{-1} \ldotp x)
	\]
	היא פעולה של $G$ על ${[m]}^X$.
\end{proposition}
\begin{proof}
	אנו צריכים לבדוק ששתי התכונות של פעולה של החבורה על הקבוצה מתקיימות.
	\begin{itemize}
		\item נייטרליות האיבר הנייטרלי: $\forall f \in {[m]}^X, x \in X : e \ldotp f(x) = f(e^{-1} x) = f(x)$.
		\item סגירות לכפל: $\forall f \in {[m]}^X, x \in X : g \ldotp (h \ldotp f)(x) = (h \ldotp f)(g^{-1} \ldotp x) = f(h^{-1} g^{-1} \ldotp x) = (gh) \ldotp f(x)$
	\end{itemize}
	ומצאנו כי התנאים לפעולה מתקיימים ומתקיים $G \acts {[m]}^X$.
\end{proof}
מה שבעצם עשינו פה הוא להרחיב פעולה של $G$ על $X$ להשרות פעולה מעל אוסף הצביעות השונות שלו, ועשינו את זה על־ידי שימוש בכפל בהופכי.
מאוד חשוב לשים לב שאנחנו מקבלים את הצביעה כפונקציה של אוסף האיברים ב־$X$ לאוסף הצבעים, אבל זה עדיין איבר בקבוצת הצביעות.

\begin{definition}[שימור צביעה]
	נגדיר שצביעה $f \in {[m]}^X$ נשמרת על־ידי $g \in G$ אם $f \in Fix(g)$, דהינו ש־$g \cdot f = f$.
\end{definition}

\subsection{טטרההדרון}
נבחן עתה את הטטרההדרון (ארבעון) שמרכזו הוא $0 \in \RR^3$ ושקודקודיו מסומנים על־ידי $v_0, \dots, v_3$ ונגדיר אותו מעתה להיות $\Delta^3$.
ונגדיר את חבורת הסימטריה $\Sym(\Delta^3)$ להיות אוסף האיזומטריות הלינאריות שמשמרות את הטטרההדרון:
\[
	\Sym(\Delta^3) = \left\{ T \in GL_3(\RR) \middle| |\det T| = 1, T\Delta^3 = \Delta^3\right\}
\]
ונגדיר גם את חבורת הסימטריות האיזומטריות שנוצרות על־ידי פעולות נוקשות:
\[
	\Sym_+(\Delta^3) = \left\{ T \in \Sym(\Delta^3) \middle| \det T = 1 \right\}
\]

נשים לב כי כל $T \in \Sym(\Delta^3)$ היא למעשה תמורה בין קודקודי הטטרההדרון. יותר מזה גם נשים לב כי אם שתי העתקות סימטריות משנות את הקודקודים באופן זהה אז הן מתנהגות באופן זהה. \\*
נגדיר אם כן את התורה $\sigma_T$ כתמורה שמזיזה את הקודקודים על־פי $T \in \Sym(\Delta^3)$.

\begin{proposition}[פעולת סימטריות על הקודקודים]
	הפעולה $\cdot : \Sym(\Delta^3) \times \{v_0, \dots, v_3\} \to \{v_0, \dots, v_3\}$ הנתונה על־ידי $T \cdot v_i = T(v_i)$ היא פעולה על הקבוצה $\{v_0, \dots, v_3\}$.
\end{proposition}
\begin{proof}
	בתרגיל
\end{proof}
\begin{conclusion}[איזומורפיות הסימטריות]
	הפונקציה $\varphi : \Sym(\Delta^3) \to S(\{v_0, \dots, v_3\})$ המוגדרת על־ידי $\varphi(T) = \sigma^T$ היא איזומורפיזם.
\end{conclusion}
\begin{proof}
	מספיק להוכיח ש־$\varphi$ היא הומומורפיזם ושכל מחזור מהצורה $(v_i, v_j)$ הוא בתמונת $\varphi$.
	העובדה שהיא הומומורפיזם נובעת מיידית מהיותה פעולה על הקבוצה.
	יהיו $i \ne j$ המתארים קודקודים, אז ישנו מישור העובר בין שני הקודקודים האחרים ודרך $\frac{v_i + v_j}{2}$.
	השיקוף סביב מרחב זה שולח את $v_i$ ל־$v_j$ והפוך, בלי להשפיע על שאר הקודקודים.
	לכן $(v_i, v_j) \in \varphi(\Sym(\Delta^3))$. נראה כי $\varphi(\Sym(\Delta^3))$ היא תת־חבורה של $S(\{v_0, \dots, v_3\})$ ולכן היא מכילה קבוצה יוצרת, ומכאן נקבל
	$\varphi(\Sym(\Delta^3)) = S(\{v_0, \dots, v_3\})$, מהטענות הקודמות נקבל גם חד־חד ערכיות.
\end{proof}
מעתה נתייחס באופן שקול ל־$T \in \Sym(\Delta^3)$ ו־$\sigma_T$.
\begin{conclusion}[טרנזיטיביות הפעולה]
	הפעולה של $\Sym(\Delta^3)$ על הקודקודים היא טרנזיטיבית.
\end{conclusion}
\begin{proof}
	נסיק מכך שכל $(v_i, v_j) \in \Sym(\Delta^3)$ שהמסלול של הגעה מכל קודקוד לכל קודקוד הוא יחיד, ולכן ככלל יש מסלול יחיד בפעולה.
\end{proof}

נבחן עתה את הפעולה של $\Sym(\Delta^3)$ על ${[m]}^X$ כאשר $X = \{v_0, \dots, v_3\}$ כפי שהגדרנו בחלק הקודם.

\begin{proposition}[מקבעי הסימטריות]
	יהי $T \in \Sym(\Delta^3)$, אז נוכיח כי $|Fix(T)|$ תלוי בסוג המחזור של $T$ בלבד.
\end{proposition}
\begin{proof}
	נכתוב את כלל סוגי המחזורים ב־$\Sym(\Delta^3)$ על־פי אורכם:
	\begin{align*}
		& 1\ 1\ 1\ 1 \\
		& 2\ 1\ 1 \\
		& 2\ 2 \\
		& 3\ 1 \\
		& 4
	\end{align*}
	מספר התמורות מכל סוג ב־$S_4$ הן $1, 6, 3, 8, 6$ בהתאמה. עתה נחשב את הצביעות המשתמרות על כל מקרה. \\*
	עבור $1\ 1\ 1\ 1$ ישנה רק תמורת הזהות, ובהתאם היא משמרת את הצבע של כל קודקוד, ולכן $|Fix(e)| = m^4$. \\*
	עתה נבחן מחזור בגודל $2$, דהינו $\sigma = (i, j)$. התמורה הזו תשמר את הצביעה של קודקודים אם ורק אם $v_i, v_j$ הם מאותו הצבע.
	לכן לשני הקודקודים $v_i, v_j$ יכולות להיות $m$ צביעות שונות כך שהתמורה תשמר את הצביעה, כאשר שאר הקודקודים בלתי תלויים, ולכן במקרה זה ישנן $m^3$ צביעות משתמרות. \\*
	באופן דומה יש $m^2$ צביעות משתמרות עבור שרשור שני מחזורים מגודל $2$. \\*
	כאשר בוחנים מחזורים בגודל $3$ אז יכולה להיות רק צביעה אחת לשלושת הקודקודים כך שהצביעה תשתמר, ולקודקוד הנותר הצבע חופשי, ונקבל $m^2$. \\*
	עבור תמורות שהן מחזור בודד מגודל $4$ אז על כלל הקודקודים להיות באותו צבע, ונקבל כמובן את מספר הצבעים עצמו $m$.
\end{proof}

נשתמש בלמה של ברנסייד כדי לחשב את מספר המסלולים של סימטריות על קודקודים על צביעות שונות של הקודקודים.
\[
	|\Sym(\Delta^3) \backslash {[m]}^X| = \frac{1}{|\Sym(\Delta^3)|} \sum_{T \in \Sym(\Delta^3)} |Fix(T)| = \frac{1m^4 + 6m^3 + 11m^2 + 6m}{24}
\]
\begin{conclusion}[מסלולים מעל צביעה]
	בעוד הפעולה של הסימטריות על $X$ היא טרנזיטיבית, הפעולה מעל הצביעות היא עצמה לא כזו בהכרח, דהינו הטרנזיטיביות של פעולה לא מעידה על טרנזיטיביות הצביעה מעליה.
\end{conclusion}
\begin{proposition}[כמות הצביעות בסימטריות חיוביות]
	נבחן את הפעולה של $\Sym_+(\Delta^3)$ על הצביעות של הקודקודים ונחשב את כמות המסלולים השונים בה.
\end{proposition}
\begin{proof}[פתרון]
	נובע ממשפט לגרנז'. סיבובים ללא היפוך יכולים להיות מורכבים רק מסיבוב סביב אחת הפאות, ולכן רק ממחזורים מהצורה $(i\ j\ k)$. \\*
	יש כמובן $8$ סיבובים אפשריים כאלה (סביב כל פאה יש שניים).
	לכן יש בחבורה $\Sym_+(\delta^3)$ לפחות 9 איברים יחד עם הנייטרלי, וממשפט לגרנז' נובע כי $\Sym_+(\delta^3) \Big| 24$ ולכן $\Sym_+(\delta^3) \in \{12, 24\}$. \\*
	אבל אנו יודעים כי $|\Sym_+(\delta^3)| < |\Sym(\delta^3)|$ שכן ישנן העתקות שהופכות את הצורה, ולכן נקבל $|\Sym_+(\delta^3)| = 12$. \\*
	נחפש אם כן את שלוש התמורות החסרות. נשים לב כי תמורות מהצורה $(i\ j)(l\ l)$ מוכלות גם הן ב־$\Sym_+(\delta^3)$ שכן הן הופכות את סימן הדטרמיננטה פעמיים.
	לכן נוכל לבחור את התמורה בין שלושה זוגות כפולים של קודקודים ונקבל את שלוש התמורות החסרות.
\end{proof}
\begin{definition}[מספר המסלולים בסימטריות סיבוביות]
	נשתמש בלמה של ברנסייד ונקבל כי מספר המסלולים של $\Sym_+(\delta^3)$ על ${[m]}^X$ היא
	\[
		|\Sym_+(\Delta^3) \backslash {[m]}^X| = \frac{1}{|\Sym_+(\Delta^3)|} \sum_{T \in \Sym_+(\Delta^3)} |Fix(T)| = \frac{1m^4 + 11m^2}{12}
	\]
\end{definition}
\begin{remark}[צביעה של פאות]
	נשים לב כי ישנן ארבע פאות ולכן נוכל לקשר כל פאה לקודקוד ונקבל כי מספר הצביעות של פאות שקול למספר הצביעות של הקודקודים.
\end{remark}

\section{שיעור 7 --- 29.5.2024}
\subsection{חבורות p}
\subsubsection{תזכורת: מרכז של חבורה}
המרכז של חבורה $Z(G)$ הוא תת־חבורה נורמלית של איברים שמתחלפים עם כלל האיברים בחבורה המקורית.
\[
	Z(G) = \{ g \in G \mid \forall h \in, g h = hg \}
\]

\begin{definition}[חבורת p]
	תהי חבורה סופית $G$, אז נקרא ל־$G$ חבורת $p$ אם קיים $p$ ראשוני ו־$n \in \NN$ כך שמתקיים $|G| = p^n$.
\end{definition}
\begin{proposition}[מרכז של חבורת p]
	אם $G$ חבורת $p$ ו־$|G| \ne 1$ ($G$ לא טריוויאלית) אז $|Z(G)| > 1$.
\end{proposition}
\begin{proof}
	למעשה נוכיח ש־$p \Big| |Z(G)|$ ולכן $|Z(G)| \ge p$. \\*
	נשתמש בנוסחת המחלקות
	\[
		|G| = |Z(G)| + \sum_{[h] \in cong(G), n \notin Z(G)} \frac{|G|}{|C_G(h)|}
	\]
	ידוע כבר כי $|G|$ מתחלק ב־$p$ ומספיק לבדוק את הסכום ולקבל את החלוקה. \\*
	כמובן ש־$|G|$ מחולק על־ידי $p$, ולכן גם חלוקתו בגודל מרכז מחולק ב־$p$ או ב־$1$. \\*
	אם $|C(h)| = |G|$ אז $C(h) = G$ ולכן $h \in Z(G)$. ולכן נניח ש־$|C(h)| < |G|$ בלי להגביל את כלליות ההוכחה ונקבל כי $p \Big| \frac{|G|}{|C(h)}$ וקיבלנו כי הסכום מחולק על־ידי $p$.
\end{proof}
\begin{example}
	עבור $S_3$, נקבל $|S_3| = 6$, והמרכז כולל רק את האיבר הטריוויאלי ולכן $|Z(S_n) = 1|$,
	מחלקות הצמידות בתמורות הן תמורות שקולות מחזור ולכן ישנן שלוש מחלקות צמידות, מתוכן שתיים לא במרכז.
	אז נקבל
	\[
		6 = 1 + \frac{6}{3} + \frac{6}{2}
	\]
\end{example}

\subsection{הומומורפיזמים}
ניזכר בהגדרת ההומומורפיזם.
תהינה $G, H$ חבורות אז הומומורפיזם $\varphi : G \to H$ היא העתקה שמקיימת
\[
	\varphi(g_1 g_2) = \varphi(g_1) \varphi(g_2)
\]
ומכאן נובע גם $\varphi(e_G) = e_H$ וגם $\varphi(g^{-1}) = \varphi^{-1}(g)$.
\begin{definition}
	אם $\varphi$ חד־חד ערכית אז נאמר שהיא \textbf{מונומורפיזם}. \\*
	אם היא על היא תיקרא \textbf{אפימורפיזם}. \\*
	אם היא חד־חד ערכית ועל אז היא תיקרא \textbf{איזומורפיזם}.
\end{definition}
\begin{definition}[גרעין]
	יהי $\varphi$ הומומורפיזם $\varphi : G \to H$.
	ה\textbf{גרעין} של $\varphi$ ושמסומן $\ker(\varphi)$ מוגדר להיות
	\[
		\ker(\varphi) = \{ g \in G \mid \varphi(g) = e_H \}
	\]
	כלל האיברים שההעתקה שולחת לאיבר הנייטרלי.
\end{definition}

\begin{definition}[תמונה]
	יהי $\varphi : G \to H$ המומורפיזם, ה\textbf{תמונה} של $\varphi$ המסומנת $\im(\varphi)$ מוגדרת על־ידי
	\[
		\im(\varphi) = \{ h \in H \mid \exists y \in G : \varphi(y) = h \}
	\]
	בדומה לתמונה של פונקציות.
\end{definition}

\begin{proposition}[גרעין ותמונה הם תת־חבורות]
	אם $\varphi : G \to H$ הומומורפיזם אז:
	\begin{enumerate}
		\item $\im(\varphi)$ תת־חבורה של $H$.
		\item $\ker(\varphi)$ תת־חבורה של $G$.
	\end{enumerate}
\end{proposition}
\begin{proof}
	נתחיל בטענה הראשונה, על־פי הגדרת תת־חבורה:
	\begin{enumerate}
		\item איבר נייטרלי: $e_h = \varphi(e_G) \implies e_H \in \im(\varphi)$
		\item סגירות לכפל: $h_1, h_2 \in \im(\varphi) \implies \exists g_1, g_2 : \varphi(g_1) = h_1, \varphi(g_2) = h_2$
		\item סגירות להופכי: $h \in \im(G) \implies \exists g \in \varphi(G) = h \implies \varphi(g) = h^{-1} \implies h^{-1} \in \im(\varphi)$
	\end{enumerate}
	ונוכיח את הטענה השנייה באופן דומה:
	\begin{enumerate}
		\item איבר נייטרלי: $e_G \in \ker(\varphi)$ נובע מ־$\varphi(e_G) = e_H$
		\item סגירות לכפל: $ g_1, g_2 \in \ker(\varphi) \implies \varphi(g_1) = e_H, \varphi(g_2) = e_H \implies \varphi(g_1g_2) = e_H e_H \implies g_1g_2 \in \ker(\varphi)$.
		\item סגירות להופכי: $g \in \ker(\varphi) \implies\ \varphi(g) = e_H \implies \varphi(g^{-1}) = \varphi^{-1}(g) = e_H$
	\end{enumerate}
\end{proof}

\begin{proposition}[תנאי מספיק לאפימורפיזם ומונומורפיזם]
	אם $\varphi$ הומומורפיזם אז:
	\begin{enumerate}
		\item $\im(\varphi) = H$ אם $\varphi$ על (אפימורפיזם).
		\item $\ker(\varphi) = \{e\}$ אם ורק אם $\varphi$ חד־חד ערכית (מונומורפיזם).
	\end{enumerate}
\end{proposition}
\begin{proof}
	טענה 1 היא טריוויאלית ונובעת מההגדרה, נוכיח את הטענה השנייה. \\*
	אם $\varphi$ חד־חד ערכית אז הטענה ברורה. \\*
	נניח כעת כי $\ker(\varphi)$ הוא טריוויאלי ונוכיח כי $\varphi$ חד־חד ערכית. \\*
	נניח בשלילה כי $\exists g_1, g_2 \in G : g_1 \ne g_2, \varphi(g_1) = \varphi(g_2)$. \\*
	נסתכל על $g_2 g_1^{-1} \ne e_G$
	אבל $\varphi(g_2g_1^{-1}) = \varphi(g_2) \varphi(g_1^{-1}) = \varphi(g_2)\varphi^{-1}(g_1) = e_H$.
\end{proof}

נראה עתה מספר דוגמות להומומורפיזמים:
\begin{example}[דטרמיננטה]
	נשים לב כי הדטרמיננטה המוגדרת על־ידי $\det : GL_n(\RR) \to \RR^\times$ היא הומומורפיזם, שכן $|AB| = |A| \cdot |B|$. \\*
	נראה גם כי $\im(|\cdot|) = \RR^\times$ וגם $\ker(| \cdot|) = SL_n(\RR)$.
\end{example}
\begin{example}[מטריצה שקולה למרוכביבם]
	יהי הומומורפיזם $\varphi : C^\times \to GL_2(\RR)$ המוגדר על־ידי 
	\[
		a + bi \mapsto
		\begin{pmatrix}
			a & b \\
			-b & a
		\end{pmatrix}
	\]
	נוכיח כי זהו הומומורפיזם:
	\[
		\varphi(a + ib)\varphi(c + id) = 
		\begin{pmatrix}
			a & b \\
			-b & a
		\end{pmatrix}
		\begin{pmatrix}
			c & d \\
			-d & c
		\end{pmatrix}
		=
		\begin{pmatrix}
			ac - bd & ad + bc \\
			-ad + bc & ac - bd
		\end{pmatrix}
		= \varphi(ac - bd + i(ad + bc))
		= \varphi((a + ib)(c + id))
	\]
	זוהי למעשה העתקה איזומורפית למרוכבים המשמרת כפל מרוכבים.
\end{example}
\begin{example}[העתקות לינאריות]
	כל העתקה לינארית $T : \RR^d \to \RR^m$ היא לינארית ולכן הומומורפיזם.
\end{example}
\begin{example}[בלוקי ז'ורדן]
	ההעתקה $\varphi : \RR \to GL_2(\RR)$ המוגדרת על־ידי
	\[
		a \mapsto \begin{pmatrix}
			1 & a \\
			0 & 1
		\end{pmatrix}
	\]
	היא הומומורפיזם, נוכיח:
	\[
		\varphi(a)\varphi(b)
		=
		\begin{pmatrix}
			1 & a \\
			0 & 1
		\end{pmatrix}
		\begin{pmatrix}
			1 & b \\
			0 & 1
		\end{pmatrix}
		=
		\begin{pmatrix}
			1 & a + b \\
			0 & 1
		\end{pmatrix}
		= \varphi(a + b)
	\]
	נשים לב כי העתקה זו מגדירה עבור כל מספר את בלוק הז'ורדן המתאים אליו, דהינו בלוק ז'ורדן משמר את תכונתו בכפל.
\end{example}
\begin{example}[מטריצה בתמורה]
	נגדיר את ההעתקה $\varphi : S_n \to GL_n(\RR)$ על־ידי
	\[
		\tau \mapsto P_\tau,
		\qquad
		{(P_\tau)}_{ij} = \delta_{i\ \tau(j)}
	\]
	כאשר $\delta_{ij}$ מוגדרת על־ידי
	\[
		(\delta_{ij}) = \begin{cases}
			1, & i = j \\
			0, & i \ne j \\
		\end{cases}
	\]
	זוהי למעשה פונקציה המקשרת תמורה למטריצה הפיכה, על־ידי שינוי סדר השורות להיות על־פי התמורה. נוכיח כי זהו הומומורפיזם:
	\[
		\varphi(\tau)\varphi(\sigma)
		= P_\tau P_\sigma
		= \sum_{k = 1}^{n} {(P_\tau)}_{ik} {(P_\sigma)}_{kj}
		= \delta_{i\ \tau(\sigma(j))}
	\]
	ולכן $P_\tau P_\sigma = P_{\tau \circ \sigma}$ וקיבלנו כי ההעתקה היא הומומורפיזם. \\*
	נוכל לראות כי זהו גם איזומורפיזם, דהינו יש יצוג יחיד לכל תמורה כמטריצה בצורה הנתונה, והפוך.
\end{example}
\begin{example}[צמצום להומומורפיזם]
	אם $\varphi : G \to H$ הומומורפיזם, אז עבור $\im(\varphi) \subseteq H' \subseteq H$ היא תת־חבורה ומתקיים
	\[
		\varphi' : G \to H',
		\qquad
		\varphi'(g) = \varphi(g)
	\]
\end{example}
\begin{example}[שרשור הומומורפיזמים]
	אם $\varphi : G \to H$ וגם $\phi : H \to K$ שני הומומורפיזמים, אז גם $\phi \circ \varphi : G \to K$ הומומורפיזם. \\*
	נוכיח:
	\[
		\phi \circ \varphi(g_1 g_2) = \phi(\varphi(g_1 g_2)) = \phi(\varphi(g_1) \varphi(g_2)) = (\phi \circ \varphi)(g_1) (\phi \circ \varphi)(g_2)
	\]
\end{example}
\begin{example}[סימן של תמורה]
	נבחן את שרשור ההומומורפיזמים:
	\[
		S_n \xrightarrow{P} GL_n(\RR) \xrightarrow{\det} \RR^\times
	\]
	תמונת השרשור היא $\{ -1, 1 \}$ בלבד, נשתמש בהומומורפיזם זה כדי להגדיר סימן לתמורות. \\*
	לתמורות עם סימן חיובי נקרא תמורות זוגיות ולשליליות נקרא אי־זוגיות. \\*
	נגדיר את ההעתקה:
	\[
		sign : S_n \to \{1, -1\} \cong \RR_{/2}
	\]
	ואף נגדיר את תת־חבורת התמורות החיוביות
	\[
		A_n := \ker(sign)
	\]
	אוסף התמורות הזוגיות. \\*
	כך לדוגמה $|A_3| = 3 = |\{e, (1\ 2\ 3), (3\ 2\ 1)\}|$.
\end{example}
\begin{example}[פעולה על חבורה]
	תהי קבוצה $X$ ותהי פעולה $G \acts X$. הפעולה ניתנת להגדרה על־ידי ההעתקה $\varphi : G \to \Sym(X)$, \\*
	שכן $\varphi(g_1 g_2) = \varphi(g_1) \varphi(g_2)$.
	לכן פעולות על קבוצות שקולות להומומורפיזמים מחבורות לסימטריות של $X$.
	נוכיח:
	\begin{proof}
		נגדיר
		\[
			\varphi(g) \in \Sym(X),
			\qquad
			\varphi(g) = fx
		\]
		נבחן את $\varphi(g_1g_2)$ על־ידי הצבה:
		\[
			\varphi(g_1g_2)(x)
			= (g_1 g_2)(x)
			= g_2(g_1(x))
			= \varphi(g)(g_2(x))
			= \varphi(g_1)(\varphi(g_2)(x))
			= (\varphi(g_1) \circ \varphi(g_2))(x)
		\]
	\end{proof}
	זאת למעשה טענה חזקה במיוחד, שכן היא קושרת כל פעולה על חבורה להומומורפיזם בין חבורה לסימטריות של קבוצה ומאפשרת לנו להסיק עוד מסקנות על הפעולה.
\end{example}
\begin{example}[שיכון]
	יהי חבורה ותת־חבורה שלה $H \le G$. \\*
	אז אפשר לבנות את העתקת השיכון ונקבל $\varphi(h \in H) = h \in G$ ונקבל $\im(\varphi) = H$, דהינו כל תת־חבורה יכולה להוות תמונה להומומורפיזם כלשהו.
\end{example}
\begin{proposition}[צמוד לגרעין]
	יהי $\varphi : G \to H$ הומומורפיזם. \\*
	לכל $g \in G$ מתקיים
	\[
		g \ker(\varphi) g^{-1} = \ker(\varphi)
	\]
\end{proposition}
\begin{proof}
	יהי $h \in \ker(\varphi)$ ו־$g \in G$ אז
	\[
		\varphi(g h g^{-1}) = \varphi(g) \varphi(h) \varphi(g^{-1}) = \varphi(g) e_H \varphi^{-1}(g) = e_H
	\]
	וקיבלנו כי השוויון מתקיים.
\end{proof}

\begin{definition}[תת־חבורה נורמלית]
$N \le G$ תת־חבורה של חבורה $G$ נקראת \textbf{נורמלית} אם לכל $g \in G$ מתקיים $g N g^{-1} = N$. \\*
נסמן $N \trianglelefteq G$. \\*
נבחין כי מההגדרה נובע כי כל איבר ב־$N$ הוא חילופי לשאר איברי $G$. \\*
נשים לב כי מצאנו שלכל $\varphi : G \to H$ הומומורפיזם נובע מיידית ש־$\ker(\varphi) \trianglelefteq G$.
\end{definition}

\begin{theorem}[משפט האיזומורפיזם הראשון]
	יהי $\varphi : G \to H$ הומומורפיזם, אז $\im(\varphi) \xrightarrow{\sim} G/\ker{\varphi}$. \\*
	דהינו התמונה של הומומורפיזם והמחלקות השמאליות של הגרעין הן איזומורפיות.
\end{theorem}
\begin{proof}
	נסמן $N = \ker(\varphi)$ אז
	\[
		g N \mapsto \varphi(g) \varphi(N) = \varphi(g) \in \im(g)
	\]
	נוכל לבחור נציג לכל מחלקה שכן:
	\[
		\forall g_1, g_2 \in G : g_1 N = g_2 N \iff g_1 g_2^{-1} \in N \iff \varphi(g g_2^{-1}) = e_h \iff \varphi(g_1) = \varphi(g_2)
	\]
	ומצאנו כי זהו הומומורפיזם. קל לראות כי הוא אף הפיך, ולכן גם איזומורפיזם.
\end{proof}

\section{שיעור 8 --- 3.6.2024}
\subsection{הומומורפיזמים}
\begin{proposition}[תנאי התמונה לאיזומורפיזם]
	העתקה $f : G \to H$ היא חד־חד ערכית אם ורק אם $G \xrightarrow{\sim} \im(f)$.
\end{proposition}
\begin{example}[דוגמות להומומורפיזמים]
	$D_n \hookrightarrow S_n$ על־פי הגדרה. \\*
	גם $P \cdot S_n \hookrightarrow GL_n(\FF)$ מטריצות הפרמוטציה היא שיכון ואף אחד מאוד חשוב. \\*
	ראינו כי $P : S_n \hookrightarrow GL_n(\FF) \xrightarrow{\det} \RR^\times$ שמייצג סימן עבור תמורות. \\*
	ראינו גם את $\CC^\times \hookrightarrow GL_2(\RR)$ על־ידי $a + bi \mapsto \begin{pmatrix} a & -b \\ b & a \end{pmatrix}$.
\end{example}
ניזכר כי מצאנו קשר בין פעולה לבין הומומורפיזם וננסחו כלמה.
\begin{lemma}[הומומורפיזם ופעולה]
	הומומורפיזם $G \xrightarrow{f} \Sym(X)$ היא זהה לפעולה $G \acts X$  כך שמתקיים
	\[
		\forall g \in G, \pi_g \in \Sym(X) : \pi_g(x) = g \cdot x, \pi_g \circ \pi_h = \pi_{gh}
	\]
	ונסיק $f(g) = \pi_g$ הומומורפיזם. \\*
	עוד נבחין כי $\ker(f) = \{ g \in G \mid \pi_g = Id_X \}$
	ונסיק כי $g \in \ker(G) \iff gx = x \forall x \in X$
	ולכן $\ker(f) = \bigcap_{x \in X} G_x$.
\end{lemma}
\begin{theorem}{משפט קיילי}
	לכל חבורה $G$ קיימת קבוצה $X$ ושיכון $G \hookrightarrow \Sym(X)$. \\*
	אם $|G| = n$ אז יש שיכון $G \hookrightarrow S_n$.
\end{theorem}
\begin{proof}
	$G$ פועלת רגולרית (משמאל) על $G$.
	כלומר $\forall x \in G : G_x = \{e\}$ שכן $gx = x \iff g = e$. \\*
	בפרט אם $f : G \to \Sym(G)$ ההומומורפיזם המתאים אז $\ker(f) = \cap_{x \in G} G_x = \{ e \}$
	וקיבלנו כי $f$ חד־חד ערכית.
\end{proof}
\begin{example}
	נקבל כי $D_n \hookrightarrow S_{2n}$, עוד נקבל מהמשפט שאפשר ליצור את השיכון $S_n \hookrightarrow S_{n!}$.
	זה לא הכי עוזר לנו אבל זה כן אפשרי, אנו רואים כי המשפט מבטיח שיכון אבל הוא עלול להיות די חסר תועלת ומהיכרות עם החבורה נוכל לבנות שיכון מוצלח יותר.
\end{example}
\begin{notation}
	העתקה חד־חד ערכית מסומנת $\hookrightarrow$,
	העתקה על מסומנת $\twoheadrightarrow$.
\end{notation}
\begin{proposition}[תנאי לתת־חבורה נורמלית]
	התנאים הבאים הם שקולים ואם אחד מהם מתקיים אז $N$ תת־חבורה נורמלית.
	\begin{enumerate}
		\item $\forall g \in G : g N g^{-1} \subseteq N$.
		\item $\forall g \in G : g g N g^{-1} = N$.
		\item $\forall g \in G : g N = N g$.
	\end{enumerate}
	ההוכחה בתרגיל.
\end{proposition}
\begin{conclusion}
	לא קיים הומומורפיזם $f : S_3 \to H$ כך שמתקיים $\ker(f) = \{ Id, (1\ 2) \}$.
\end{conclusion}
\begin{proof}
	נבחיו כי $\{ Id, (1\ 2)\}$ היא לא תת־חבורה נורמלית של $S_3$ כי $(1\ 3)(1\ 2)(3\ 1) = (1)(2\ 3)$.
\end{proof}
דהינו לא כל תת־חבורה יכולה לשמש כגרעין, נשאל את עצמנו האם כל תת־חבורה נורמלית היא גרעין של הומומורפיזם כלשהו, על שאלה זו נענה עתה.

\begin{proposition}[תמונת תת־חבורה נורמלית]
	כאשר $f : G \to H$ הומומורפיזם ו־$N = \ker(f)$ אז $f^{-1}(f(x)) = x N$, התמונה ההפוכה של תמונת $x$ היא המחלקה $x N$. \\*
	יתרה מכך הפונקציה $\im(f) \to G/N$ המוגדרת על־ידי $h \mapsto f^{-1}(h)$ היא חד־חד ערכית ועל.
\end{proposition}
\begin{proof}
	תחילה נבחין כי מתקיים
	\[
		{f(x)}^{-1} f(y) = x^{-1} y \in N \iff xN = yN
	\]
	נראה כי ההעתקה היא על:
	\[
		f^{-1}(f(x)) = xN
	\]
	נראה כי ההעתקה היא גם חד־חד ערכית,
	עבור $f(x), f(y) \in \im(f)$ מתקיים
	\[
		f^{-1}(f(x)) = f^{-1}(f(y)) = yN
		\iff
		x^{-1} y \in N
		\iff
		f(x^{-1} y) = e
	\]
\end{proof}

\subsection{חבורת המנה}
תהינה $N \triangleleft G$ ונגדיר $G/N$ מבנה של חבורה.

\begin{proposition}[מכפלת מחלקות]
	$N$ נורמלית אם ורק אם $\forall x, y \in G : (xN) \cdot (yN) = (xy)N$.
\end{proposition}
\begin{proof}
	$(x N)(y N) = x (Ny) N \overset{\text{נורמליות}}{=} x (yN) N = (xy)(NN) = (xy)N$
\end{proof}

\begin{proposition}[חבורת כפל מחלקות]
	$G/N$ עם הכפל של מחלקות היא חבורה עם האיבר הנייטרלי $eN$.
\end{proposition}
\begin{proof}
	נבדוק את התנאים לחבורה:
	\begin{enumerate}
		\item איבר נייטרלי: $\forall x \in N: (eN)(xN) = xN = (xN)(eN)$.
		\item אסוציאטיביות: $((xN)(yN))(zN) = ((xy)z)N = (xyz)N = (xN)(yN)(zN)$.
		\item סגירות להופכי: $(xN)(x^{-1}N) = (xx^{-1})N = eN$.
	\end{enumerate}
\end{proof}
\begin{proposition}
	תהי הפונקציה $\pi : G \to G/N$ המוגדרת על־ידי $x \mapsto x N$. \\*
	הפונקציה $\pi$ היא הומומורפיזם כך שגם $\ker(\pi) = N$.
\end{proposition}
\begin{proof}
	$\pi(x) \cdot \pi(y) = (xN)(yN) = (xy)N = \pi(xy)$. \\*
	עוד נבחין כי $x N = \pi(x) = N \iff x \in N$.
\end{proof}
\begin{example}
	נבחין בחבורות המנה הבאות:
	\begin{enumerate}
		\item עבור החבורה $\ZZ$. זוהי חבורה אבלית ולכן כל תת־חבורה שלה היא נורמלית ומתקיים $n \ZZ \triangleleft \ZZ$. \\*
			בהתאם $\ZZ/n \cong \ZZ / n \ZZ = \{ n \ZZ, 1 + n \ZZ, \dots, (n - 1) + n \ZZ \}$. \\*
			ונראה גם $(a + n \ZZ) + (b + n \ZZ) = ((a + b) + n \ZZ) = (a + b \mod n) + n \ZZ$.
		\item ראינו בתרגול כי $GL_n(\FF) / SL_n(\FF) \cong \FF^\times, A \cdot SL_n(\FF) \mapsto \det(A)$. \\*
			ואנחנו רואים כי $\det : GL_n(\FF) \twoheadrightarrow \FF^\times$ וגם כי $SL_n(\FF) = \ker(\det)$.
	\end{enumerate}
\end{example}

\section{תרגול 5 --- 4.6.2024}
\subsection{תת־חבורות נורמליות}
\begin{example}
	תהי $H \subseteq GL_n(\FF)$ חבורת הייזנברג, המוגדרת על־ידי
	\[
		H = \left\{
			\begin{pmatrix}
				1 & a & c \\
				0 & 1 & b \\
				0 & 0 & 1
			\end{pmatrix}
			\middle|
			a, b, c \in \FF
		\right\}
	\]
	נבחין כי זו אכן חבורה שכן מטריצות מולשיות סגורות לפעולת הכפל ומכילות הופכי. \\*
	נגדיר גם
	\[
		H = \left\{
			\begin{pmatrix}
				1 & 0 & c \\
				0 & 1 & 0 \\
				0 & 0 & 1
			\end{pmatrix}
			\middle|
			c \in \FF
		\right\}
	\]
	נבחין כי $Z \trianglelefteq H$ ואף מתקיים $H/Z \cong \FF^2$.
\end{example}
\begin{lemma}
	תזכורת: אם $|G| = p$ אז $G$ היא ציקלית.
\end{lemma}
\begin{lemma}
	אם $|G| = p^2$ כאשר $p$ ראשוני, אז $G$ אבלית.
\end{lemma}
\begin{proof}
	ידוע כי $Z(G)$ לא טריוואלית, ולפי משפט לגרנז' מתקיים $|Z(G)| \Big| |G|$ אז נקבל כי $|Z(G)| \in \{ p, p^2 \}$. \\*
	נקבל כי $G/Z(G)$ היא מגודל $1$ או מגודל $p$. לכן נקבל כי החלוקה הזו היא ציקלית ואז נובע כי היא אבלית.
\end{proof}
נבחין כי לא בהכרח כל $G$ ציקלית היא מגודל $p^2$, לדוגמה ${(\ZZ_{/p})}^2$ היא לא ציקלית כלל שכן לא כל האיברים הם מסדר $p$ ועל־כן אי־אפשר ליצור את החבורה מאיבר בודד. נשים לב לכן גם ש־${(\ZZ/p)}^2 \not\cong \ZZ_{/p^2}$.
\begin{proposition}
	יהי $p$ ראשוני ו־$G$ חבורה. אם $|G| = p^2$ אז $G$ איזומורפית לאחת החבורות
	\[
		\ZZ_{/p^2}, \ZZ_{/p} \times \ZZ_{/p}
	\]
	אם $|G| = p^3$ בהתאם היא איזומורפית לאחת החבורות
	\[
		\ZZ_{/p^3}, \ZZ_{/p^2} \times \ZZ_{/p}, \ZZ_{/p} \times \ZZ_{/p} \times \ZZ_{/p}
	\]
\end{proposition}

\section{שיעור 9 --- 5.6.2024}
בשבוע הבא השיעור בשני יועבר על ידי יונתן והשיעור ברביעי לא יתקיים בעקבות שבועות.

\subsection{משפטי האיזומורפיזם}
בשיעור הקודם דיברנו על זה שאם יש לנו הומומורפיזם $f : G \to H$ אז $\ker(f) \trianglelefteq G$.
מצד שני אם $N \trianglelefteq G$ אז קיים $\pi : G \to G/N$ העתקה מהחבורה למחלקות השמאליות של $N$ על־ידי כפל תת־חבורות וזוהי חבורה.
מה שאמרנו זה ששתי הטענות הן כמעט הופכיות אחת לשנייה. מצאנו כי $\ker(\pi) = N$.
נבחין כי $N \subseteq G, N \in G/N$.
בזמן שאי־אפשר לשחזר את הפונקציה המקורית, אנחנו כן יכולים להסיק על התמונה שלה על־פי הגרעין.

\begin{theorem}[משפט האיזומורפיזם הראשון]
	תהינה $G, H$ חבורות, ו־$f : G \to H$ הומומורפיזם אז נובע $G/\ker(f) \xrightarrow{\sim} \im(f)$. \\*
	אף קיים איזומורפיזם יחיד $\alpha$ כך ש־$\alpha \circ \pi = f$.
\end{theorem}
\begin{proof}
	בנינו פונקציה חד־חד ערכית ועל $\alpha : G/\ker(f) \to \im(f)$ על־ידי $\alpha(x \ker(f)) = f(x)$. \\*
	נראה ש־$\alpha$ הומומורפיזם.
	\[
		\alpha(x \ker(f)) \alpha(y \ker(f)) = f(x) f(y) = f(xy) = \alpha((xy) \ker(f))
	\]
	נותר להוכיח את היחידות של $\alpha$. \\*
	לכל $y \in G/\ker(f)$ קיים $x \in G$ כך ש־$y = x \ker(f)$ ולכן
	\[
		\alpha(y) = \alpha(x \ker(f)) = \alpha(\pi(x)) = f(x)
	\]
	וקיבלנו כי $f = \alpha \circ \pi$ וזהו אכן איזומורפיזם יחיד.
\end{proof}
מתברר שכל הומומורפיזם בעולם הם הרכבה של חלוקה למחלקות גרעין, הליכה לתמונה ואז הפעלת אוטומורפיזם כלשהו.
\begin{example}
	יהי $\ZZ \xrightarrow{\mod n} \ZZ_{/n}$ וראינו כי $n\ZZ \trianglelefteq \ZZ$ ונקבל $\ZZ / n\ZZ \xrightarrow{\sim} \ZZ_{/n}$ ממשפט האיזומורפיזם הראשון.
\end{example}
\begin{example}
	יהי $GL_n(\RR) \xrightarrow{\det} \RR^\times$ הומומורפיזם שהוא על. הגרעין הוא הדטרמיננטות עם גודל 1, דהינו $SL_n(\RR) \trianglelefteq GL_n(\RR)$.
	לכן גם $GL_n(\RR) / SL_n(\RR) \cong \RR^\times$.
\end{example}
\begin{example}
	$GL_n(\RR)$, ונראה את המרכז $Z(GL_n(\RR)) = \{ a I_n \mid a \in \RR^\times \}$, המטריצות הסקלריות. \\*
	 נחלק עתה את חבורה במרכזה ונקבל
	 \[
		 GL_n(\RR) / Z(GL_n(\RR)) := PGL_n(\RR)
	 \]
\end{example}
\begin{example}
	אם יש שתי חבורות $G, H$, נבחן את $H \xleftarrow{\pi_H} G \times H \xrightarrow{\pi_G} G$, כאשר $\pi_G(gh) = g$ ו־$\pi_H(gh) = h$. \\*
	$\ker(\pi_G) = \{(e, h) \mid h \in H \} = \{e\} \times H$ ובאופן דומה $\ker(\pi_H) = G \times \{e \}$. \\*
	ממשפט האיזומורפיזם הראשון אנו מקבלים כי $(G \times H) / (G \times \{e\}) \cong H$, גם אינטואיטיבית זה מאוד הגיוני שכן אנו מקבצים לפי איברי $G$.
\end{example}
\begin{remark}
	אם $G$ סופית ו־$N \trianglelefteq G$ אז $|G| = |N| \cdot |G/N|$ כנביעה ממשפט לגרנז'.
\end{remark}
בהינתן שתי חבורות $G, H$, אז נוכלי לבנות חבורה $E$ כך ש־$G \trianglelefteq E$ כך ש־$H \xrightarrow{\sim} E/G$.
כך לדוגמה אם נבחן את $\ZZ_{/2} = G = H$ אז נוכל להגדיר $E = \ZZ_{/2} \times \ZZ_{/2}$ או גם $E = \ZZ_{/4}$, ולכן נקבל $\ZZ_{/2} \times \ZZ_{/2} \xrightarrow{\sim} \ZZ_{/2}$ אשר מקיימת את הטענה.

בהינתן חבורות $G, H, K$ והומומורפיזמים $K \xrightarrow{\alpha} G, K \xrightarrow{\beta} H$
אז נוכל לבנות גם $K \xrightarrow{(\alpha, \beta)} G \times H$ על־ידי $(\alpha, \beta)(x) = (\alpha(x), \beta(x))$. \\*
הגרעין מקיים במקרה זה $\ker(\alpha, \beta) = \ker(\alpha) \cap \ker(\beta)$.

בהינתן $\ZZ_{/b} \xleftarrow{\pi_b} \ZZ \xrightarrow{\pi_a} \ZZ_{/a}$ נוכל להגדיר $\ZZ \xrightarrow{\pi} \ZZ_{/a} \times \ZZ_{/b}$ ונקבל $\ker(\pi) = a\ZZ \cap b\ZZ = lcm(a, b) \ZZ$.
\begin{conclusion}
	$\ZZ_{/lcm(a, b)} \cong \im(\pi) \le \ZZ_{/a} \times \ZZ_{/b}$ ואם $\gcd(a, b) = 1$ אז $lcm(a, b) = ab$ ונקבל את משפט השאריות הסיני:
	\[
		\ZZ_{/ab} \xrightarrow{\sim} \ZZ_{/a} \times \ZZ_{/b}
	\]
\end{conclusion}

נבחן עתה על $G, G/N$ בלבד, ומה המבנה שלה. נגדיר כי $G \xrightarrow{\pi} G/N$. אם $K \le G$ אז בהתאם $\pi(K) \le G/N$ כנביעה ישירה מהעובדה ש־$\pi$ הומומורפיזם. \\*
אם $L \le G/N$ אז $\pi^{-1}(L) \ge N$, שכן כל איבר בחבורת המנה מיתרגם למספר איברים (למעשה מחלקות שקילות שלמות) בחבורה המקורית, וכל תת־חבורה מכילה איבר נייטרלי שמיתרגם לחבורה $N$ עצמה במקור.
נסמן את התמונה על־ידי $\overline{\pi}(K) = \{ \pi(x) \mid x \in K\}$ מטעמי נוחות.
\begin{theorem}
	תהי $G$ חבורה ו־$N \trianglelefteq G$ תת־חבורה נורמלית שלה, אז
	\[
		\{ K \le G \mid N \le K \} \xrightarrow{\overline{\pi}} \{ L \le G/N \}
		\qquad
		\{ L \le G/N \} \xrightarrow{\pi^{-1}} \{ K \le G \mid N \le K \} 
	\]
\end{theorem}
\begin{proof}
	\textbf{כיוון ראשון:}
	יהי $L \le G/N$ ונקבל מהגדרת $\pi$ כי
	\[
		\overline{\pi}(\pi^{-1}(L)) \subseteq L
	\]
	מצד שני נטען כי $L \subseteq \overline{\pi}(\pi^{-1}(L))$ שכן $y \in L \implies y = \pi(x)$ לאיזשהו $x \in \pi^{-1}(L)$.
	לכן $y = \pi(x) \in \overline{\pi}(\pi^{-1}(L))$.

	\textbf{כיוון שני:}
	תהי $N \le K \le G$ ונחשב
	\[
		K \overset{\text{לפי הגדרה}}{\subseteq} \pi^{-1}(\overline{\pi}(K)) \overset{(1)}{\subseteq} K
	\]
	ונסביר את $(1)$:
	\[
		\overline{\pi}(K) = \{ \pi(x) \mid x \in K \} = \{ xN \mid x \in K \}
	\]
	ולכן
	\[
		\pi^{-1}(\overline{\pi}(K)) = \bigcup_{x \in K} \pi^{-1}(x N) = \bigcup_{x \in K} x N \subseteq K
	\]
\end{proof}
\begin{remark}
	שתי הפונקציות $\pi^{-1}, \overline{\pi}$ משמרות הכלה.
\end{remark}
\begin{notation}
	אם $N \subseteq K \le G$ אז נסמן $\overline{\pi}(K) = K / N$.
\end{notation}
\begin{theorem}[משפט האיזומורפיזם השלישי]
	תהי $N \trianglelefteq G$ אז לכל $N \trianglelefteq K \le G$ מתקיים
	\[
		K \trianglelefteq G \iff K/N \trianglelefteq G/N
	\]
	ובמקרה זה
	\[
		G/K \cong (G/N)/(K/N)
	\]
\end{theorem}
\begin{proof}
	נניח $K/N \trianglelefteq G/N$ ונסתכל על ההומומורפיזם
	\[
		G \overset{\pi}{\twoheadrightarrow} G/N \overset{\varphi}{\twoheadrightarrow} (G/N)/(K/N)
	\]
	אז
	\[
		\ker(\varphi \circ \pi) = \pi^{-1}(\ker(\varphi)) = \pi^{-1}(K/N) = K
	\]
	ממשפט האיזומורפיזם הראשון נקבל $G/K \xrightarrow{\sim} (G/N)/(K/N)$. \\*
	שכן קיבלנו כי $G/\ker(\varphi \circ \pi) \xrightarrow{\sim} \im(\varphi \circ \pi)$.

	כיוון שני:
	נניח כי $N \trianglelefteq K \trianglelefteq G$ ונסתכל על הפונקציה
	\[
		\alpha : G/N \to G/K,
		\qquad
		xN \mapsto (xN)K = x(NK) = xK
	\]
	ונראה ש־$\alpha$ הומומורפיזם. \\*
	פונקציה זו היא בבירור הומומורפיזם שכן מדובר על כפל חבורות.
	נבחין כי
	\[
		\ker(\alpha) = \{ xN \mid xK = K \}
		= \{ xN \mid x \in K \}
		= K/N
	\]
	ונוכל להסיק ממשפט האיזומורפיזם הראשון כי
	\[
		(G/N)/(K/N) \xrightarrow{\sim} G/K
	\]
\end{proof}
\begin{example}
	נגדיר את ההומומורפיזם $\ZZ \to \ZZ / n\ZZ$. אנו יודעים כי $n \ZZ \le d \ZZ \le \ZZ$ לכל $d \big| n$, ולכן נוכל להשתמש במשפט האיזומורפיזם השלישי ונקבל
	\[
		(\ZZ/n\ZZ)/(d\ZZ/n\ZZ) \xrightarrow{\sim} \ZZ_{/d}
	\]
	ולמעשה הצלחנו לפשט משמעותית את חבורת המנה הזו.
\end{example}

\end{document}
