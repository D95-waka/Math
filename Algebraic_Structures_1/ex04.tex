\documentclass[a4paper]{article}

% packages
\usepackage{inputenc, fontspec, amsmath, amsthm, amsfonts, polyglossia, catchfile}
\usepackage[a4paper, margin=50pt, includeheadfoot]{geometry} % set page margins

% style
\AddToHook{cmd/section/before}{\clearpage}	% Add line break before section
\linespread{1.5}
\setcounter{secnumdepth}{0}		% Remove default number tags from sections
\setmainfont{Libertinus Serif}
\setsansfont{Libertinus Sans}
\setmonofont{Libertinus Mono}
\setdefaultlanguage{hebrew}
\setotherlanguage{english}

% operators
\DeclareMathOperator\cis{cis}
\DeclareMathOperator\Sp{Sp}
\DeclareMathOperator\tr{tr}
\DeclareMathOperator\im{Im}
\DeclareMathOperator\diag{diag}
\DeclareMathOperator*\lowlim{\underline{lim}}
\DeclareMathOperator*\uplim{\overline{lim}}

% commands
\renewcommand\qedsymbol{\textbf{משל}}
\newcommand{\NN}[0]{\mathbb{N}}
\newcommand{\ZZ}[0]{\mathbb{Z}}
\newcommand{\QQ}[0]{\mathbb{Q}}
\newcommand{\RR}[0]{\mathbb{R}}
\newcommand{\CC}[0]{\mathbb{C}}
\newcommand{\getenv}[2][] {
  \CatchFileEdef{\temp}{"|kpsewhich --var-value #2"}{\endlinechar=-1}
  \if\relax\detokenize{#1}\relax\temp\else\let#1\temp\fi
}
\newcommand{\explain}[2] {
	\begin{flalign*}
		 && \text{#2} && \text{#1}
	\end{flalign*}
}

% headers
\getenv[\AUTHOR]{AUTHOR}
\author{\AUTHOR}
\date\today

\title{פתרון מטלה 04 --- מבנים אלגבריים 1 (80445)}

\begin{document}
\maketitle
\maketitleprint{}

\Question{}
\Subquestion{}
הכוונה ברורה לי אבל עוד לפני שקראתי את שאר השאלה אני רוצה אינטואיטיבית להשתמש ב־$D_n$ וצביעה מעל קבוצה.

\Subquestion{}
נגדיר $X_{n, q} = {[q]}^{[n]}$.
נוכיח שהחבורה $D_n$ משרה פעולה על הקבוצה $X_{n, q}$ על־ידי
\[
	\forall f \in X_{n, q} \forall \sigma \cdot f(k) = f(\sigma^{-1}(k))
\]
\begin{proof}
	אנו כבר יודעים כי החבורה $D_n$ היא פעולה מעל $[n]$ על־ידי $\forall g \in D_n, x \in [n] : g \cdot x = g(x)$. \\*
	בתרגול הוכחנו כי בהינתן קבוצה ופעולה עליה, ניתן להרחיב את הפעולה לצביעה של הקבוצה על־ידי
	\[
		\forall g \in D_4, f \in {[q]}^{[n]} : \forall x \in [n], g \cdot f(x) = f(g^{-1} \cdot x)
	\]
	ומצאנו כי הטענה נכונה.
\end{proof}

\Subquestion{}
נחשב את מספר המסלולים של $D_n$ על $X_{n, q}$.

נגדיר $\Xi \subseteq \NN$ קבוצת הראשוניים, דהינו מתקיים 
\[
	\Xi = \left\{ n \in \NN \mid \forall k \in \NN\setminus\{1, n\} : k \nmid n\right\}
\]
נגדיר $\xi : \NN \to \Xi$ על־ידי $\xi(n) = \{ k \in \Xi \mid k | n \}$, פונקציה אשר מחזירה את הראשוניים אשר המספר מתחלק בהם. \\*
עתה נגדיר גם $r = (1 \dots n)$ תמורת סיבוב ו־$s(k) = n - k + 1$ פונקציית ההיפוך, לכן $D_n = \langle r, s \rangle$. \\*
תהי צביעה $f \in {[q]}^{[n]}$ כלשהי, אז נבחין כי אילו היא מורכבת מיותר צבע אחד, דהינו $\lnot \exists c : \forall k \in [n] : f(k) = c$, אז מתחייב כי $\exists n : f(n) \ne f(n + 1)$ ונוכל להסיק
\[
	Fix(f) = \{ f\} \implies |Fix(f)| = 1
\]
יהי $m \in \NN$ כך ש־$1 < m \le n$, ונבחן את $r^m$, סיבוב ב־$m$ איברים.
אילו $m \big| n$ אז קל לראות כי ישנן $q^m$ צביעות אפשריות, אילו לעומת זאת $m \nmid n$ אז נוכל להוכיח בדומה למקרה $m = 1$ שמספר הצביעות האפשרי הוא צביעה אחידה, דהינו $q$ אפשרויות.
נגדיר פונצקיה $\mu : \NN \to \NN$ המשקפת את מספר הצביעות המושרה על־ידי $r^m$ כזה, נגדירה להיות
\[
	\mu(m) = \begin{cases}
		q^m & m \mid n \\
		q & m \nmid n
	\end{cases}
\]
ואף מתקיים $Fix(r^m) = \mu(m)$ ישירות על־פי הגדרה. \\*
נבחן את המקרה השני והוא $s r^m \in D_n$ כשאר $0 \le m < n$. \\*
נשים לב כי $sr^m = {(sr^m)}^{-1}$ ולכן נובע כי $\forall k \in [n] : f(k) = f(sr^m k)$ ולמעשה נוכל להשתמש בהצמדה זו לקבוע כי ישנן $\lfloor \frac{n}{2} \rfloor q$ צביעות. \\*
נסכם:
ב־$\langle r, s \rangle$ ישנם $2n$ איברים בדיוק, כולם מהצורה $r^m$ או $s r^m$ עבור $0 \le m < n$. נבנה פונקציה $\nu : D_n \to \NN$ המייצגת את כמות הצביעות על־פי פעולה מהחבורה:
\[
	Fix(s^l r^m) = \nu(s^l r^m) = \begin{cases}
		q \lfloor \frac{n}{2} \rfloor & l = 1 \\
		\mu(m) & l = 0
	\end{cases}
\]
ונשתמש בלמה של ברנסייד לקבל כי מספר המסלולים מוגדר על־ידי
\[
	|X_{n, q} / D_n| = \frac{1}{|D_n|} \sum_{s^l r^m \in D_n, l \in \{0, 1\}, 0 \le m < n} \nu(s^l r^m)
	= \frac{1}{|D_n|} \sum_{s^l r^m \in D_n, l \in \{0, 1\}, 0 \le m < n} \begin{cases}
		q \lfloor \frac{n}{2} \rfloor & l = 1 \\
		q^m & l = 0, m \mid n \\
		q & l = 0, m \nmid n
	\end{cases}
\]

\end{document}
