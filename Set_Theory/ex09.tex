\documentclass[a4paper]{article}

% packages
\usepackage{inputenc, amsmath, amsthm, thmtools, amsfonts, amssymb, luacode, catchfile, tikzducks, hyperref}
\usepackage[a4paper, margin=50pt, includeheadfoot]{geometry} % set page margins
\usepackage[shortlabels]{enumitem}
\usepackage[skip=3pt, indent=0pt]{parskip}

% language
\usepackage[bidi=basic, layout=tabular, provide=*]{babel}
\babelprovide[main, import]{hebrew}
\babelprovide{rl}
\babelfont{rm}{Libertinus Serif}
\babelfont{sf}{Libertinus Sans}
\babelfont{tt}{Libertinus Mono}

% style
\AddToHook{cmd/section/before}{\clearpage}	% Add line break before section
\linespread{1.3}
\setcounter{secnumdepth}{0}		% Remove default number tags from sections, this won't do well with theorems
\AtBeginDocument{\setlength{\belowdisplayskip}{3pt}}
\AtBeginDocument{\setlength{\abovedisplayskip}{3pt}}

% operators
\DeclareMathOperator\cis{cis}
\DeclareMathOperator\Sp{Sp}
\DeclareMathOperator\tr{tr}
\DeclareMathOperator\im{Im}
\DeclareMathOperator\re{Re}
\DeclareMathOperator\diag{diag}
\DeclareMathOperator*\lowlim{\underline{lim}}
\DeclareMathOperator*\uplim{\overline{lim}}
\DeclareMathOperator\rng{rng}
\DeclareMathOperator\Sym{Sym}
\DeclareMathOperator\Arg{Arg}
\DeclareMathOperator\Log{Log}
\DeclareMathOperator\dom{dom}

% commands
%\renewcommand\qedsymbol{\textbf{מש''ל}}
%\renewcommand\qedsymbol{\fbox{\emoji{lizard}}}
\newcommand{\NN}[0]{\mathbb{N}}
\newcommand{\ZZ}[0]{\mathbb{Z}}
\newcommand{\QQ}[0]{\mathbb{Q}}
\newcommand{\RR}[0]{\mathbb{R}}
\newcommand{\CC}[0]{\mathbb{C}}
\newcommand{\FF}[0]{\mathbb{F}}
\newcommand{\PP}[0]{\mathbb{P}}
\newcommand{\TT}[0]{\mathbb{T}}
\newcommand{\acts}[0]{\circlearrowright}
\newcommand{\explain}[2] {
	\begin{flalign*}
		 && \text{#2} && \text{#1}
	\end{flalign*}
}
\newcommand{\maketitleprint}[0]{ \begin{center}
	\begin{tikzpicture}[scale=3]
		\duck[graduate=gray!20!black, tassel=red!70!black]
	\end{tikzpicture}	
\end{center}
}

% theorem commands
\newtheoremstyle{c_remark}
	{}	% Space above
	{}	% Space below
	{}% Body font
	{}	% Indent amount
	{\bfseries}	% Theorem head font
	{}	% Punctuation after theorem head
	{.5em}	% Space after theorem head
	{\thmname{#1}\thmnumber{ #2}\thmnote{ \normalfont{\text{(#3)}}}}	% head content
\newtheoremstyle{c_definition}
	{3pt}	% Space above
	{3pt}	% Space below
	{}% Body font
	{}	% Indent amount
	{\bfseries}	% Theorem head font
	{}	% Punctuation after theorem head
	{.5em}	% Space after theorem head
	{\thmname{#1}\thmnumber{ #2}\thmnote{ \normalfont{\text{(#3)}}}}	% head content
\newtheoremstyle{c_plain}
	{3pt}	% Space above
	{3pt}	% Space below
	{\itshape}% Body font
	{}	% Indent amount
	{\bfseries}	% Theorem head font
	{}	% Punctuation after theorem head
	{.5em}	% Space after theorem head
	{\thmname{#1}\thmnumber{ #2}\thmnote{ \text{(#3)}}}	% head content

\theoremstyle{c_plain}
\newtheorem{theorem}{משפט}[section]
\newtheorem{lemma}[theorem]{למה}
\newtheorem{proposition}[theorem]{טענה}
\newtheorem*{proposition*}{טענה}
%\newtheorem{corollary}[theorem]{אין חלופה עברית}

\theoremstyle{c_definition}
\newtheorem{definition}[theorem]{הגדרה}
\newtheorem*{definition*}{הגדרה}
\newtheorem{example}{דוגמה}[section]
\newtheorem{exercise}{תרגיל}[section]

\theoremstyle{c_remark}
\newtheorem*{remark}{הערה}
\newtheorem*{solution}{פתרון}
\newtheorem{conclusion}[theorem]{מסקנה}
\newtheorem{notation}[theorem]{סימון}

% Questions related commands
\newcounter{question}
\setcounter{question}{1}
\newcounter{sub_question}
\setcounter{sub_question}{1}

\newcommand{\question}[1][0]{
	\ifthenelse{#1 = 0}{}{\setcounter{question}{#1}}
	\subsection{שאלה \arabic{question}}
	\addtocounter{question}{1}
	\setcounter{sub_question}{1}
}

\newcommand{\subquestion}[1][0]{
	\ifthenelse{#1 = 0}{}{\setcounter{sub_question}{#1}}
	\subsubsection{סעיף \localecounter{letters.gershayim}{sub_question}}
	\addtocounter{sub_question}{1}
}

% import lua and start of document
\directlua{common = require ('../common')}

\GetEnv{AUTHOR}

% headers
\author{\AUTHOR}
\date\today

\title{פתרון מטלה 09 --- תורת הקבוצות (80200)}

\DeclareMathOperator\dom{dom}
\DeclareMathOperator\add{Add}
\DeclareMathOperator\mult{Mult}
\begin{document}
\maketitle
\maketitleprint{}

\Question{}
תהי קבוצה $A$ ונוכיח כי התנאים הבאים שקולים:
\begin{enumerate}
	\item $A$ טרנזיטיבית
	\item $A \subseteq \mathcal{P}(A)$
	\item $\bigcup A \subseteq A$
\end{enumerate}
\begin{proof}
	$1 \to 2$:
	נניח כי $A$ טרנזיטיבית. \\*
	נראה ש־$\forall x \in A \implies x \subseteq A \iff x \in \mathcal{P}(A)$. \\*
	ההגדרה של הכלה היא כמובן $\forall x \in A \implies x \in B \iff A \subseteq B$ וקיבלנו כי $\forall x \in A \implies x \in \mathcal{P}(A)$ ולכן נסיק כי $A \subseteq \mathcal{P}(A)$.

	$2 \to 3$:
	נניח כי $A \subseteq \mathcal{P}(A)$. \\*
	בחלק הקודם מצאנו $x \subseteq A \iff x \in \mathcal{P}(A)$ ולכן נסיק,
	ונתון כי $\forall x \in A \implies x \in \mathcal{P}(A)$, ומשרשור הטענות נקבל
	\[
		\forall x \in A \implies x \subseteq A
	\]
	וכמובן גם $\bigcup x \in A = A$ על־פי הגדרה, ולכן נקבל כי גם $\bigcup A \subseteq A$.

	$3 \to 1$:
	נניח כי $\bigcup A \subseteq A$. \\*
	יהי $x \in A$, אז נקבל כי $x \subseteq \bigcup A \subseteq A$ מההגדרה של איחוד, וקיבלנו כי $\forall x, x \in A \implies x \subseteq A$.
\end{proof}

\Question{}
\Subquestion{}
נוכיח כי אם $A$ טרנזיטיבית אז גם $\mathcal{P}(A)$ טרנזיטיבית.
\begin{proof}
	נניח כי $A$ טרנזיטיבית ולכן מהשאלה הקודמת נקבל $A \subseteq \mathcal{P}(A)$ ונסיק
	\[
		x \in \mathcal{P}(A) \implies x \subseteq A \subseteq \mathcal{P}(A)
	\]
	ומצאנו כי $\mathcal{P}(A)$ טרנזיטיבית.
\end{proof}

\Subquestion{}
נניח כי $A \ne \emptyset$ קבוצה של קבוצות טרנזיטיביות, נוכיח כי $\bigcup A, \bigcap A$ טרנזיטיביות.
\begin{proof}
	תהי $x \in A$, נתון כי היא טרנזיטיבית ולכן משאלה 1 נקבל $\bigcup x \subseteq x$, נפעיל את פעולת האיחוד על שני האגפים ונקבל $\bigcup \bigcup A \subseteq \bigcup A$ ולכן נסיק משאלה 1 כי גם $\bigcup A$ טרנזיטיבית.

	יהי $y \in \bigcap A$, דהינו $\forall x \in A\enspace y \in x$, ונבחין כי גם $\forall x \in A \enspace y \in x \implies y \subseteq x \implies y \subseteq \bigcap A$ וקיבלנו כי גם החיתוך הוא טרנזיטיבי.
\end{proof}

\Question{}
יהי $\alpha$ סודר ונסמן $\alpha + 1 = s(\alpha) = \alpha \cup \{ \alpha \}$. \\*
נוכיח כי $\alpha + 1$ סודר וכי הוא העוקב של $\alpha$.
\begin{proof}
	נבדוק אם $\alpha + 1$ הוא טרנזיטיבי. \\*
	אם $x \in \alpha + 1$ אז נקבל כי $x \in \alpha \lor \alpha \in \{ \alpha \}$, דהינו $x \in \alpha$ או $x = \alpha$. \\*
	אם $x = \alpha$ אז כמובן $x = \alpha \subseteq \alpha + 1$, ואם $x \in \alpha$ אז $x \subseteq \alpha \subseteq \alpha \cup \{ \alpha \} = \alpha + 1$. \\*
	נראה כי $(\alpha + 1, \in)$ סדר קווי. אנו כבר יודעים כי $(\alpha, \in)$ סדר קווי וזהו למעשה אותו הסדר עם תוספת של איבר יחיד $\{ \alpha \}$, ולכן מספיק לבדוק אותו בסדר זה. \\*
	נראה כי אם $x \in \alpha$ אז כמובן $x \in \alpha + 1$ כפי שכבר ראינו, ולמעשה הראינו כי תכונת הסדר הקווית נשמרת. \\*
	נסיק אם כן ש־$\alpha + 1$ הוא סודר.

	נוכיח כי $\alpha + 1$ הוא העוקב של $\alpha$.
	ראינו כבר כי $\alpha \in \alpha + 1$, ואנו יודעים כי $\alpha$ איבר מקסימלי ב־$(\alpha, \in)$, ונסיק כי $\forall x \in \alpha \in \alpha + 1$. \\*
	אם כן החשוד היחידי הוא $x = \{ \alpha \}$ עצמו, האיבר היחיד שנוסף ב־$(\alpha + 1, \in)$, וכמובן $x \in \alpha + 1$ ולכן נסיק כי אין איברים שסותרים את תכונת העוקב, ולכן $\alpha + 1$ אכן עוקב של $\alpha$.
\end{proof}

\Question{}
\Subquestion{}
נוכיח שאם $A \ne \emptyset$ קבוצת סודרים אז $\bigcup A, \bigcap A$ סודרים ושמתקיים $\min A = \bigcap A$ וגם $\sup A = \bigcup \alpha$.
\begin{proof}
	בשאלה 2 מצאנו כי שתי קבוצות אלה טרנזיטיביות ולכן מספיק להוכיח כי הן מגדירות יחס סדר קווי יחד עם $\in$ כדי להראות שהן סודרים. \\*
	נשתמש בטענה מהתרגול הטוענת כי כל שני סודרים $\alpha, \beta$ מתקיים $\alpha = \beta$ או $\alpha \in \beta$ וזהו ללא הגבלת הכלליות. \\*
	אם $\alpha = \beta$ כמובן $(\alpha \cup \beta, \in)$ הוא סדר קווי, ולכן נניח כי $\alpha \in \beta$. \\*
	יהיו $x \in \alpha, y \in \beta$, ואנו יודעים כי $x \in \alpha \in \beta \implies x \in \beta$, ולכן נקבל כי $x, y \in \beta$ ומהסדר הטוב $(\beta, \in)$ נסיק כי $x = y \lor x \in y \lor y \in x$. \\*
	נוכל אם כן לבצע אותו הליך באופן רקורסיבי לכל הקבוצות ב־$A$ ונקבל כי $(\bigcup A, \in)$ הוא סדר קווי ולכן $\bigcup A$ סודר. \\*
	נעבור לבדוק את הסדר הממוגדר על־ידי $\bigcap A$, יהיו $\alpha, \beta \in A$, אם $\alpha = \beta$ אז $\alpha \cap \beta = \alpha$ ומצאנו כי זהו סודר, ולכן נניח ללא הגבלת הכלליות כי $\alpha \in \beta$. \\*
	נוכל אם כן להסיק כי גם $\alpha \subseteq \beta$ ונסיק ישירות ש־$\alpha \cap \beta = \alpha$ ובהתאם גם $\bigcap A$ הוא סודר.

	ראינו כי $\alpha \in \beta \implies \alpha \cap \beta = \alpha$ ולכן אם נבחר את הסודר המינימלי ב־$A$ (כפי שמותר לנו לעשות שכן מצאנו ש־Ord סדר קווי עם יחס ההכלה) ונגדירו $\alpha$.
	נוכל אם כן להסיק $\forall \beta \in A \enspace \alpha \cap \beta = \alpha$ ולכן נוכל להסיק כי $\bigcap A = \alpha = \min A$. \\*
	באופו דומה מצאנו כי $\forall \alpha, \beta \in A \enspace \alpha \in \beta \implies \alpha \cup \beta = \beta$,
	לכן נקבל כי $\forall \alpha \in A \enspace \alpha \in \bigcup A$ ואם נניח $\gamma$ סודר כך ש־$\alpha \in \gamma$ לכל $\alpha \in A$ אז נקבל כי גם $\bigcup A \in \gamma$ מהטענה הקודמת, ונוכל להסיק כי $\sup A = \bigcup A$.
\end{proof}

\Subquestion{}
נוכיח שאם ל־$A$ אין מקסימום אז $\sup A$ הוא סודר גבולי, דהינו לא קיים $\alpha$ סודר כך ש־$\sup A = \alpha + 1$.
\begin{proof}
	נניח כי ל־$A$ אין מקסימום ונניח בשלילה ש־$\sup A = \alpha + 1$ עבור סודר $\alpha$ כלשהו. \\*
	מצאנו כי $\sup A = \bigcup A$ ולכן $\alpha + 1 \in x$ עבור $x \in A$ כלשהו, ולכן מטענות שראינו בהוכחה הקודמת נסיק $\forall y \in A \enspace y \in x$ ומצאנו כי $x$ מקסימום ב־$A$ בסתירה להנחה.
\end{proof}

\Question{}
\Subquestion{}
יהי $\alpha$ סודר.

\subsubsection{i.}
נוכיח שאם $\alpha = \beta + 1$ הוא עוקב אז $\bigcup \alpha = \beta$.
\begin{proof}
	נבחין כי $\alpha = \beta \cup \{ \beta \}$ ולכן $\bigcup \alpha = $
\end{proof}

\end{document}
