\documentclass[a4paper]{article}

% packages
\usepackage{inputenc, amsmath, amsthm, thmtools, amsfonts, amssymb, luacode, catchfile, tikzducks, hyperref}
\usepackage[a4paper, margin=50pt, includeheadfoot]{geometry} % set page margins
\usepackage[shortlabels]{enumitem}
\usepackage[skip=3pt, indent=0pt]{parskip}

% language
\usepackage[bidi=basic, layout=tabular, provide=*]{babel}
\babelprovide[main, import]{hebrew}
\babelprovide{rl}
\babelfont{rm}{Libertinus Serif}
\babelfont{sf}{Libertinus Sans}
\babelfont{tt}{Libertinus Mono}

% style
\AddToHook{cmd/section/before}{\clearpage}	% Add line break before section
\linespread{1.3}
\setcounter{secnumdepth}{0}		% Remove default number tags from sections, this won't do well with theorems
\AtBeginDocument{\setlength{\belowdisplayskip}{3pt}}
\AtBeginDocument{\setlength{\abovedisplayskip}{3pt}}

% operators
\DeclareMathOperator\cis{cis}
\DeclareMathOperator\Sp{Sp}
\DeclareMathOperator\tr{tr}
\DeclareMathOperator\im{Im}
\DeclareMathOperator\re{Re}
\DeclareMathOperator\diag{diag}
\DeclareMathOperator*\lowlim{\underline{lim}}
\DeclareMathOperator*\uplim{\overline{lim}}
\DeclareMathOperator\rng{rng}
\DeclareMathOperator\Sym{Sym}
\DeclareMathOperator\Arg{Arg}
\DeclareMathOperator\Log{Log}
\DeclareMathOperator\dom{dom}

% commands
%\renewcommand\qedsymbol{\textbf{מש''ל}}
%\renewcommand\qedsymbol{\fbox{\emoji{lizard}}}
\newcommand{\NN}[0]{\mathbb{N}}
\newcommand{\ZZ}[0]{\mathbb{Z}}
\newcommand{\QQ}[0]{\mathbb{Q}}
\newcommand{\RR}[0]{\mathbb{R}}
\newcommand{\CC}[0]{\mathbb{C}}
\newcommand{\FF}[0]{\mathbb{F}}
\newcommand{\PP}[0]{\mathbb{P}}
\newcommand{\TT}[0]{\mathbb{T}}
\newcommand{\acts}[0]{\circlearrowright}
\newcommand{\explain}[2] {
	\begin{flalign*}
		 && \text{#2} && \text{#1}
	\end{flalign*}
}
\newcommand{\maketitleprint}[0]{ \begin{center}
	\begin{tikzpicture}[scale=3]
		\duck[graduate=gray!20!black, tassel=red!70!black]
	\end{tikzpicture}	
\end{center}
}

% theorem commands
\newtheoremstyle{c_remark}
	{}	% Space above
	{}	% Space below
	{}% Body font
	{}	% Indent amount
	{\bfseries}	% Theorem head font
	{}	% Punctuation after theorem head
	{.5em}	% Space after theorem head
	{\thmname{#1}\thmnumber{ #2}\thmnote{ \normalfont{\text{(#3)}}}}	% head content
\newtheoremstyle{c_definition}
	{3pt}	% Space above
	{3pt}	% Space below
	{}% Body font
	{}	% Indent amount
	{\bfseries}	% Theorem head font
	{}	% Punctuation after theorem head
	{.5em}	% Space after theorem head
	{\thmname{#1}\thmnumber{ #2}\thmnote{ \normalfont{\text{(#3)}}}}	% head content
\newtheoremstyle{c_plain}
	{3pt}	% Space above
	{3pt}	% Space below
	{\itshape}% Body font
	{}	% Indent amount
	{\bfseries}	% Theorem head font
	{}	% Punctuation after theorem head
	{.5em}	% Space after theorem head
	{\thmname{#1}\thmnumber{ #2}\thmnote{ \text{(#3)}}}	% head content

\theoremstyle{c_plain}
\newtheorem{theorem}{משפט}[section]
\newtheorem{lemma}[theorem]{למה}
\newtheorem{proposition}[theorem]{טענה}
\newtheorem*{proposition*}{טענה}
%\newtheorem{corollary}[theorem]{אין חלופה עברית}

\theoremstyle{c_definition}
\newtheorem{definition}[theorem]{הגדרה}
\newtheorem*{definition*}{הגדרה}
\newtheorem{example}{דוגמה}[section]
\newtheorem{exercise}{תרגיל}[section]

\theoremstyle{c_remark}
\newtheorem*{remark}{הערה}
\newtheorem*{solution}{פתרון}
\newtheorem{conclusion}[theorem]{מסקנה}
\newtheorem{notation}[theorem]{סימון}

% Questions related commands
\newcounter{question}
\setcounter{question}{1}
\newcounter{sub_question}
\setcounter{sub_question}{1}

\newcommand{\question}[1][0]{
	\ifthenelse{#1 = 0}{}{\setcounter{question}{#1}}
	\subsection{שאלה \arabic{question}}
	\addtocounter{question}{1}
	\setcounter{sub_question}{1}
}

\newcommand{\subquestion}[1][0]{
	\ifthenelse{#1 = 0}{}{\setcounter{sub_question}{#1}}
	\subsubsection{סעיף \localecounter{letters.gershayim}{sub_question}}
	\addtocounter{sub_question}{1}
}

% import lua and start of document
\directlua{common = require ('../common')}

\GetEnv{AUTHOR}

% headers
\author{\AUTHOR}
\date\today

\title{פתרון מטלה 03 --- תורת הקבוצות (80200)}

\begin{document}
\maketitle
\maketitleprint{}

\Question{}
\Subquestion{}
תהי $A = \{ X \in \mathcal{P}(\NN) \mid X \text{ אינסופית}\}$. נוכיח כי היא איננה בת־מניה.
\begin{proof}
	נניח בשלילה ש־$A$ בת־מניה ולכן קיימת פונקציה $f : \NN \to A$ חד־חד ערכית ועל. \\*
	נוכיח באופן דומה להוכחה של קנטור, נגדיר קבוצה $B$. לכל $n \in \NN$, אם $n \notin f(n)$ אז נגדיר $n \in B$. \\*
	ידוע כי $f$ על ולכן קיים $a \in \NN$ כך ש־$f(a) = B$, ולכן על־פי הגדרתנו $a \notin B$. אבל לפי הגדרתנו $a \notin f(n) = B \implies n \in B$ וזו סתירה.\\*
	לכן $A$ לא בת־מניה.
\end{proof}

\Question{}
\Subquestion{}
נוכיח שקבוצת כל הפונקציות $f : \NN \to \NN$ שהן מונוטוניות יורדות היא בת־מניה.
\begin{proof}
	תהי פונקציה $f$ כזו, על־פי הנתון יהי $k = f(0)$, אז $\forall n \in \NN : 0 \le f(n) \le k$. \\*
	הפונקציה $f$ אם כן ''יכולה'' לרדת $k$ פעמים, ולאחר מכן היא תהיה הפונקציה הקבועה $f(n) = 0$. נוכל לבנות סדרה סופית באורך לכל היותר $k$ שמתארת את המספרים בתחום עבורם הפונקציה האכן יורדת. \\*
	כדי להגדיר את הסדרה בצורה פורמלית תהי הסדרה $(a_n)$ כך ש־$a_1 = f(0)$, ומתקיים $f(a_n) = f(a_{n + 1}) + 1$, נשים לב כי הגדרה זו מכסה גם את המקרים בהם הירידה היא ביותר מיחידה. \\*
	לכל פונקציה כזו מתאימה סדרה כמתואר יחידה בלבד, ואם שתי פונקציות מתוארות על־ידי אותה הסדרה אז על־פי הגדרתה הפונקציות עצמן זהות. \\*
	תהי $A$ קבוצת הפונקציות המדוברות ו־$B$ סדרת הסדרות הסופיות מעל הטבעיים, אז $g : A \to B$ היא חד־חד ערכית ולכן גם $|A| \le |B| \le |\NN|$ כפי שהוכח בהרצאה. \\*
	לכיוון ההפוך $h : B \to A$ נגדיר על־ידי $h(n)(0) = n, h(n)(x > 0) = 0$, זוהי כמובן פונקציה חד־חד ערכית והפונקציות בטווח הן אכן מונוטוניות יורדות, ולכן $|\NN| \le |A|$.\\*
	מצאנו כי $A$ בת־מניה.
\end{proof}

\Subquestion{}
נוכיח כי עוצמת קבוצת הפונקציות מהשלמים לשלמים היורדות ממש היא $0$.
\begin{proof}
	נניח כי קיימת פונקציה $f : \NN \to \NN$ מונוטונית יורדת ממש, ונניח $f(0) = k \in \NN$. \\*
	נראה כי $f(1) \le f(0) - 1 = k - 1$ וכי אם נניח ש־$f(n) \le k - n$ אז $\forall n \in \NN : f(n + 1) \le f(n) - 1 = k - n - 1$ וזוהי הוכחה באינדוקציה, לכן
	\[
		\forall n \in \NN\setminus\{0\} f(n) \le k - n
	\]
	ובפרט עבור $n = k$ מתקיים $0 \le f(n) \le n - n = 0$ ולכן $f(n) = 0$. \\*
	ידוע כי $0 \le f(n + 1) < f(n) = 0$ ולכן $0 < 0$ וזוהי כמובן סתירה, לכן לא קיימת פונקציה $f$ שעומדת בהגדרה. \\*
	בהתאם קבוצת הפונקציות האלה היא ריקה ועוצמתה $0$.
\end{proof}

\Subquestion{}
נוכיח שקבוצת הפונקציות המונוטויות העולות ממש $C$ מהטבעיים לעצמם היא איננה בת־מניה.
\begin{proof}
	נגדיר פונקציה מהקבוצה משאלה 1 $A$ ל־$C$, $f : A \to C$. \\*
	תהי קבוצת טבעיים אינסופית כלשהי $D$, אז נוכל נבחר את סדרת המספרים אשר מתקבלת מסידור הקבוצה על־פי גודל המספרים המרכיבים אותה, נקרא לסדרה זו ${(a_n)}_{n = 1}^\infty$. 
	סדרה זו כמובן מוגדרת ישירות מההגדרה של הקבוצה, ולכל $n$ מתקיים $a_{n + 1} > a_n$ מיחידות ההכלה של הקבוצה המקורית והסידור לפי סדר עולה. \\*
	עתה נגדיר $f(D) = \{ \langle n, a_n \rangle \mid n \in \NN \}$, מיחידות האינדקס בסדרה נובע כי הפונקציה אכן מוגדרת ונובע $\forall n > m \in \NN : f(n) > f(m)$ כפי שרצינו.
	נבחין כי הפונקציה אכן חד־חד ערכית, אם $\forall D, E \in A : f(D) = f(E) \overset{(1)}{\implies} f(D)(\NN) = f(E)(\NN) \implies D = E$. \\*
	$(1)$ נובע ישירות מהגדרת הסדרה לשלוח לאיברי הקבוצה המקורית. \\*
	קיבלנו כי $f$ חד־חד ערכית ולכן בהתאם $|\NN| < |A| \le |C|$ ולכן בהתאם $C$ לא בת־מניה.
\end{proof}

\Question{}
\Subquestion{}
נוכיח ישירות מטיעון האלכסון כי $\QQ^\NN = \{ f : \NN \to \QQ\}$ איננה בת־מניה.
\begin{proof}
	נניח בשלילה כי $\QQ^\NN$ בת־מניה ובהתאם קיימת פונקציה חד־חד ערכית ועל $f : \NN \to \QQ^\NN$. \\*
	נגדיר פונקציה $g : \NN \to \QQ$ על־ידי $\forall n \in \NN : f(n)(0) \ne 2 \iff g(n) = 2$. \\*
	ידוע כי $g \in \QQ^\NN$ ולכן קיים $k \in \NN$ כך ש־$f(k) = g$ ולכן $f(k)(0) = g(0) = 2$ אם ורק אם $g(0) \ne 2$ וזוהי סתירה, ולכן $\QQ^\NN$ לא בת־מניה.
\end{proof}

\Subquestion{}
נוכיח ישירות על־ידי טיעון האלכסון כי $A = \{ f \in \NN^\NN \mid f : \NN \to \NN \text{ חד־חד ערכית} \}$ איננה בת־מניה.
\begin{proof}
	נניח בשלילה כי $A$ בת־מניה ולכן קיימת פונקציה $f : \NN \to A$. חד־חד ערכית ועל. \\*
	נגדיר פונקציה חד־חד ערכית $g : \NN \to \NN \in A$ כך ש־$\forall n \in \NN f(n)(n) = n \iff g(2n) = 2n + 1 \land g(2n + 1) = 2n$. \\*
	 כדי לשמור על החד־חד ערכיות נגדיר שאם $f(n)(2n) \ne n$ אז $g(2n + 1) = 2n + 1, g(2n) = 2n$. \\*
	 הגדרה זו אכן שומרת על חד־חד ערכיות, שכן לכל $n \in \NN$ מתקיים $g(2n) = 2n, g(2n + 1) = 2n + 1$ או $g(2n) = 2n + 1, g(2n + 1) = 2n$ ובכל מקרה אין עוד ערך $m \ne n$ כך ש־$f(m) = 2n + 1$ או $f(m) = 2n$. \\*
	 עתה נראה כי $g \in A \implies \exists k \in \NN : f(k) = g$, ו־$f(k)(2k) = k$ אם ורק אם $g(2k) = 2k + 1$ וזוהי כמובן סתירה, ולכן נסיק ש־$A$ לא בת־מניה.
\end{proof}

\Question{}
\Subquestion{}
נוכיח כי
\[
	\rng \varphi^+ \cup \rng \varphi^- = (\NN \times \NN) / \sim_\ZZ,
	\qquad
	\rng \varphi^+ \cap \rng \varphi^- = \{ {[\langle 0, 0 \rangle]}_{\sim_\ZZ} \}
\]
\begin{proof}
	יהי $\langle n, m \rangle \in \NN \times \NN$. אם $n \ge m$ אז נוכל להגדיר $n = n' + m$ ומתקיים $\langle n, m \rangle \sim_\ZZ \langle n', 0 \rangle$,
	ובהתאם ${[\langle n, m \rangle]}_{\sim_\ZZ} \in \rng \varphi^+$. \\*
	אם $m \ge n$ אז בהתאם קיים $m' \in \NN$ כך ש־$m = n + m'$ ולכן ${[\langle n, m \rangle]}_{\sim_\ZZ} \in \rng \varphi^-$. \\*
	מצאנו כי לכל $\langle n, m \rangle \in \NN^2$ קיימת מחלקת שקילות כך ש־${[\langle n, m \rangle]}_{\sim_\ZZ} \in \rng \varphi^+ \cup \rng \varphi^-$.
	בהתאם נובע
	\[
		\rng \varphi^+ \cup \rng \varphi^- = (\NN \times \NN) / \sim_\ZZ,
	\]
	ראינו כי עבור $\langle n, m \rangle$ אז $n \ge m \iff {[\langle n, m \rangle]}_{\sim_\ZZ} \in \rng \varphi^+$ וגם כי
	$n \le m \iff {[\langle n, m \rangle]}_{\sim_\ZZ} \in \rng \varphi^-$.
	לכן
	\[
		n = m \iff n \le m \land n \ge m \iff {[\langle n, m \rangle]}_{\sim_\ZZ} \in \rng \varphi^+ \land {[\langle n, m \rangle]}_{\sim_\ZZ} \in \rng \varphi^-
	\]
	ואנו יודעים כי $\langle n, n \rangle \sim_\ZZ \langle 0, 0 \rangle$ ולכן $\rng \varphi^+ \cap \rng \varphi^- = \{ {[\langle 0, 0 \rangle]}_{\sim_\ZZ} \}$
\end{proof}

\Subquestion{}
נוכיח כי הפעולה
\[
	\langle a, b \rangle * \langle c, d \rangle := \langle a \cdot c, b \cdot d \rangle
\]
לא משרה פעולה מוגדרת היטב על $(\NN \times \NN) / \sim_\ZZ$.
\begin{proof}
	נראה כי $\langle 1, 2 \rangle \sim_\ZZ \langle 0, 1 \rangle$. נראה כי
	\[
		\langle 1, 2\rangle * \langle 1, 3 \rangle = \langle 1, 6\rangle, 
		\qquad
		\langle 0, 1 \rangle * \langle 1, 3 \rangle = \langle 0, 3 \rangle
	\]
	לעומת זאת $\langle 1, 6 \rangle \not\sim_\ZZ \langle 1, 5 \rangle$ בסתירה להגדרת הסגירות של פעולה על מחלקות שקילות, ולכן הפונקציה $*$ לא משרה פעולה על $(\NN \times \NN) / \sim_\ZZ$.
\end{proof}

\Subquestion{}
נגדיר את הפונקציה $\cdot$ על־ידי
\[
	\langle a, b \rangle \cdot \langle c, d \rangle := \langle ac + bd, ad + bc \rangle
\]
נוכיח כי פונקציה זו משרה פעולה על $(\NN \times \NN) / \sim_\ZZ$.
\begin{proof}
	נראה שבחירת נציגים ממחלקת השקילות אינם משפיעים על תוצאת הפונקציה. \\*
	יהיו $\langle a, b \rangle \sim_\ZZ \langle a', b' \rangle, \langle c, d \rangle \sim_\ZZ \langle c', d' \rangle$ 
	ולכן $a + b' = b + a', c + d' = d + c'$, מכפל שוויונות אלה נקבל
	\[
		(a + b')(c + d') = (b + a')(d + c')
		\iff ac + ad' + b'c + b'd' = bd + bc' + a'd + a'c' \tag{1}
	\]
	אז:
	\begin{align*}
		& \langle a, b \rangle \cdot \langle c, d \rangle = \langle ac + bd, ad + bc \rangle,
		& & \langle a', b' \rangle \cdot \langle c', d' \rangle = \langle a'c' + b'd', a'd' + b'c' \rangle, \\
		& \langle a, b \rangle \cdot \langle c, d \rangle \sim_\ZZ \langle a', b' \rangle \cdot \langle c', d' \rangle
		& \iff  & \langle ac + bd, ad + bc \rangle \sim_\ZZ \langle a'c' + b'd', a'd' + b'c' \rangle \\
		& \iff ac + bd + a'd' + b'c' = ad + bc + a'c' + b'd'
	\end{align*}
	\begin{align*}
		& \iff ac + bd + ad + bc + 2a'd' + 2b'c' = 2ad + 2bc + a'c' + b'd' + a'd' + b'c' \\
		& \overset{(1)}{\iff} 2a'd' + 2b'c' = 2ad + 2bc \\
		& \iff a'd' + b'c' = ad + bc \\
		& \overset{(1)}{\iff} 0 = 0
	\end{align*}
	ומצאנו כי הפעולה אכן סגורה לבחירת איברים ממחלקות השקילות.
\end{proof}

\Question{}
\Subquestion{}
על $\ZZ \times (\NN \setminus \{0\})$ יחס $\sim_\QQ$ על־ידי
\[
	\langle p, q \rangle \sim_\QQ \langle r, s \rangle \iff p \cdot s = r \cdot q
\]
נוכיח כי זהו יחס שקילות.
\begin{proof}
	נבדוק את התנאים המגדירים יחס שקילות:
	\begin{enumerate}
		\item רפלקסיביות: $\langle p, q \rangle \sim_\QQ \langle p, q \rangle \iff p \cdot q = p \cdot q$ ואכן התכונה מתקיימת.
		\item סימטריה: $\langle p, q \rangle \sim_\QQ \langle r, s \rangle \iff ps = rq \iff rq = ps \iff \langle r, s \rangle \sim_\QQ \langle p, q \rangle$
		\item טרנזיטיביות: $\langle p, q \rangle \sim_\QQ \langle r, s \rangle, \langle r, s \rangle \sim_\QQ \langle t, u \rangle
			\iff ps = rq, ru = ts \iff psu = ruq = tsq \iff pu = tq \iff \langle p, q \rangle \sim_\QQ \langle t, u \rangle$
	\end{enumerate}
	ומצאנו כי זהו אכן יחס שקילות.
\end{proof}

\Subquestion{}
נגדיר $\QQ := (\ZZ \times (\NN \setminus\{0\}))/\sim_{\QQ}$. \\*
עוד נגדיר
\[
	\langle a, b \rangle + \langle c, d\rangle := \langle ad + bc, bd \rangle,
	\qquad
	\langle a, b \rangle \cdot \langle c, d\rangle := \langle ac, bd \rangle
\]
ונוכיח כי פעולות אלה משרות פעולות מוגדרות היטב ב־$\QQ$.
\begin{proof}
	נבדוק את סגירות פעולת החיבור שהגדרנו זה עתה:
	\begin{align*}
		& \langle a, b \rangle + \langle c, d\rangle = \langle ad + bc, bd \rangle, \langle a', b' \rangle + \langle c', d'\rangle = \langle a' d' + b' c', b' d' \rangle, \\
		& a d = b c, a' d' = b' c' \\
		\iff &
		\langle a, b \rangle \sim_\QQ \langle a', b' \rangle, \langle c, d\rangle \sim_\QQ \langle c', d'\rangle \\
		\iff &
		\langle a, b \rangle + \langle c, d\rangle \sim_\QQ \langle a', b' \rangle + \langle c', d'\rangle \\
		\iff &
		\langle ad + bc, bd \rangle \sim_\QQ \langle a' d' + b' c', b' d' \rangle \\
		\iff &
		(ad + bc)(b' d') = (a' d' + b' c')(bd) \\
		\iff &
		ad b' d'+ bc b' d' = a' d' b d + b' c' bd \\
	\end{align*}
	וקיבלנו כי הטענה אכן מתקיימת והפעולה אכן ממוגדרת היטב ב־$\QQ$. נבחין כי הפעולה $\cdot$ כמעט זהה והתהליך האלגברי לבדיקתה דומה.
\end{proof}

\Subquestion{}
נגדיר את הפונקציה $\varphi : \ZZ \to \QQ$ על־ידי
\[
	\varphi({[\langle n, m \rangle]}_{\sim_\ZZ}) = {[{[\langle n, m \rangle]}_{\sim_\ZZ}, 1]}_{\sim_\QQ}
\]
נוכיח כי היא חד־חד ערכית
\begin{proof}
	נשתמש בסימון למספרים אלה על־פי הגדרות המחלקות ונקבל
	\[
		\varphi(n - m) = \frac{n - m}{1}
	\]
	נשים לב כי זהו סימון בלבד ואלו הן מחלקות שקילות יחד עם הפעולות שהגדרנו.
	נראה כי
	\[
		\forall n, m, n', m' : \varphi(n - m) = \varphi(n' - m') \iff \frac{n - m}{1} = \frac{n' - m'}{1} \iff 1(n - m) = 1(n' - m') \iff n - m = n' - m'
	\]
	וקיבלנו כי היא אכן חד־חד ערכית.
\end{proof}

\end{document}
