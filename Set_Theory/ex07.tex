\documentclass[a4paper]{article}

% packages
\usepackage{inputenc, fontspec, amsmath, amsthm, amsfonts, polyglossia, catchfile}
\usepackage[a4paper, margin=50pt, includeheadfoot]{geometry} % set page margins

% style
\AddToHook{cmd/section/before}{\clearpage}	% Add line break before section
\linespread{1.5}
\setcounter{secnumdepth}{0}		% Remove default number tags from sections
\setmainfont{Libertinus Serif}
\setsansfont{Libertinus Sans}
\setmonofont{Libertinus Mono}
\setdefaultlanguage{hebrew}
\setotherlanguage{english}

% operators
\DeclareMathOperator\cis{cis}
\DeclareMathOperator\Sp{Sp}
\DeclareMathOperator\tr{tr}
\DeclareMathOperator\im{Im}
\DeclareMathOperator\diag{diag}
\DeclareMathOperator*\lowlim{\underline{lim}}
\DeclareMathOperator*\uplim{\overline{lim}}

% commands
\renewcommand\qedsymbol{\textbf{משל}}
\newcommand{\NN}[0]{\mathbb{N}}
\newcommand{\ZZ}[0]{\mathbb{Z}}
\newcommand{\QQ}[0]{\mathbb{Q}}
\newcommand{\RR}[0]{\mathbb{R}}
\newcommand{\CC}[0]{\mathbb{C}}
\newcommand{\getenv}[2][] {
  \CatchFileEdef{\temp}{"|kpsewhich --var-value #2"}{\endlinechar=-1}
  \if\relax\detokenize{#1}\relax\temp\else\let#1\temp\fi
}
\newcommand{\explain}[2] {
	\begin{flalign*}
		 && \text{#2} && \text{#1}
	\end{flalign*}
}

% headers
\getenv[\AUTHOR]{AUTHOR}
\author{\AUTHOR}
\date\today

\title{פתרון מטלה 07 --- תורת הקבוצות (80200)}

\DeclareMathOperator\Pfin{\mathcal{P}_\text{fin}}
\DeclareMathOperator\lex{\le_\text{lex}}
\begin{document}
\maketitle
\maketitleprint{}

\Question{}
נוכיח שאם $\langle A, \le_A \rangle$ קבוצה סדורה חלקית, אז יש שיכון של $\langle A, \le_A \rangle$ לתוך $\langle \mathcal{P}(A), \subseteq \rangle$.
\begin{proof}
	יהי איבר $b \in A$, ונגדיר $B = \{ \langle n, m \rangle \in \le_A \mid m = b \}$, ולכן $\forall c \in B : c \le b$, נגדיר $f : \langle A, \le_A \rangle \to \langle \mathcal{P}(A), \subseteq \rangle$, על־ידי
	\[
		f(x) = \{ \langle a, x \rangle \in \le_A \}
	\]
	למעשה פונקציה זו מחזירה קבוצת האיברים הקטנים מהאיבר המקורי (כולל אותו עצמו), ולכן מהתכונות של סדר נוכל להסיק $a \le_A b \implies f(a) \subseteq f(b)$ ומצאנו שיכון.
\end{proof}

\Question{}
לכל קבוצה $A$ נסמן $\Pfin(A) = \{ X \subseteq A \mid X \text{ is finite} \}$.

\Subquestion{}
נמצא שיכון של $\langle \ZZ, \le \rangle$ לתוך $\langle \mathcal{P}(\NN), \subseteq \rangle$.

נגדיר $f : \ZZ \to \NN$ על־ידי
\[
	f(z) = \begin{cases}
		2z + 1 & z \le 0 \\
		-2z & z < 0
	\end{cases}
\]
יחד עם יחס הסדר $\preceq = \{ \langle n, m \rangle \in \NN^2 \mid m = 1 (\mod 2) \implies n \le m \lor n = 0 (\mod 2) \implies n \le m \lor (n = 0 (\mod 2) \land m = 1 (\mod 2)) \}$. \\*
מהותית סדר שבודק את הזוגיות ומשמר את הסדר המקורי של $\ZZ$, ועתה נרכיב את השיכון משאלה 1 ונקבל שיכון כפי שנתבקשנו.

\Subquestion{}
נוכיח כי לא קיים שיכון של $\langle \ZZ, \le \rangle$ לתוך $\langle \Pfin(\NN), \subseteq \rangle$.
\begin{proof}
	נניח בשלילה כי ישנו שיכון $f$ כזה. \\*
	לכן $f(-1) \subseteq f(0)$, וכמו־כן גם $\forall n \in -\NN : f(n) \subseteq f(0)$, ולכן נוכל להסיק כי $|f(0)| = \aleph_0$ בסתירה להנחה כי כל איבר ב־$\Pfin(\NN)$ הוא סופי.
\end{proof}

\Subquestion{}
נמצא שיכון של $\langle Seq(\NN), \trianglelefteq \rangle$ לתוך $\langle \NN \setminus \{ 0 \}, \mid \rangle$.

נגדיר $\varphi : \NN \to \NN$ פונקציה המחזירה ראשוניים הולכים וגדלים. \\*
נגדיר $f : \langle Seq(\NN), \trianglelefteq \rangle \to \langle \NN \setminus \{ 0 \}, \mid \rangle$ על־ידי
\[
	f(\langle n_1, \dots, n_d \rangle) = \prod_{i = 1}^d \varphi({\varphi(i)}^{n_i})
\]
נניח כי $v, u \in Seq(\NN)$ כך ש־$v \trianglelefteq u$ אז נובע כי $v = \langle v_1, \dots, v_d \rangle, u = \langle v_1, \dots, v_d, u_{d + 1}, \dots, u_k \rangle$, \\*
אז נובע כי $f(u) = f(v) \cdot C$ ולכן $f(v) \mid f(u)$. \\*
בכיוון ההפוך נניח כי $f(v) \mid f(u)$ ונקבל מפירוק $f(u)$ למספרים ראשוניים וחישוב כי $v_1 = u_1$ וכן הלאה, ונוכל להסיק $v \trianglelefteq u$.

\Subquestion{}
נוכיח כי $\langle Seq(\NN), \trianglelefteq \rangle$ ו־$\langle \NN \setminus \{ 0 \}, \mid \rangle$ לא איזומורפיים.
\begin{proof}
	יהיו $a, b \in \NN$, נבחין כי קיים $c \in \NN$ כך ש־$a, b \mid c$ תמיד ($c = ab$ לדוגמה). \\*
	לעומת זאת יהיו $u, v \in Seq(\NN)$, אם $u \ne v$ והם מאותו האורך, אז אין $w \in Seq(\NN)$ כך ש־$a \trianglelefteq c$ וגם $b \trianglelefteq c$, בסתירה לשימור מקסימלי לתת־קבוצות בין קבוצות איזומורפיות.
\end{proof}

\Subquestion{}
נמצא שיכון של $\langle Seq(\NN), \trianglelefteq \rangle$ לתוך $\langle \Pfin(\NN), \subseteq\rangle$.

נשתמש בפונקציה שהגדרנו בשאלה 1, עלינו לבדוק רק שכל קבוצה שתתקבל היא אכן סופית. \\*
יהי $u \in Seq(\NN)$, ונניח כי גודל $u$ הוא $n$, אז נסיק כי $u$ גדול מ־$n$ איברים, הם הצמצומים שלו, וסדרה ריקה, ולכן $|f(u)| = n$, ולכן $f(u) \in \Pfin(\NN)$ לכל $u \in Seq(\NN)$.

\Subquestion{}
נוכיח כי $\langle Seq(\NN), \trianglelefteq \rangle$ ו־$\langle \Pfin(\NN), \subseteq\rangle$ לא איזומורפיים.
\begin{proof}
	למעשה, ההוכחה זהה לסעיף ד', שכן לכל שתי קבוצות $X, Y \in \Pfin(\NN)$ קיים $X \cup Y$ אשר מהווה איבר גדול משתיהן. \\*
	לעומת זאת, ראינו כי לא לכל שני איברים יש איבר משותף גדול משניהם ב־$Seq(\NN)$.
\end{proof}

\Question{}
הגדרנו עבור סדרים חלקיים $\langle A, \le_A \rangle, \langle B, \le_B \rangle$ את הסדר המילוני על $A \times B$ על־ידי
\[
	\langle a, b \rangle \lex \langle a', b' \rangle \iff a <_A a' \lor (a = a' \land b \le_B b')
\]
נניח כי $A, B$ קבוצות לא ריקות.

\Subquestion{}
נוכיח כי $\langle A \times B, \lex \rangle$ קבוצה סדורה חלקית.
\begin{proof}
	נבדוק את תכונות הסדר החלקי:
	\begin{enumerate}
		\item רפלקסיביות: $\forall (a, b) \in A \times B \implies a = a' \land b \le_B b \iff \langle a, b \rangle \lex \langle a, b \rangle$.
		\item טרנזיטיבי: $\forall \langle a_1, b_1 \rangle, \langle a_2, b_2 \rangle, \langle a_3, b_3 \rangle \in A \times B : \langle a_1, b_1 \rangle \lex \langle a_2, b_2 \rangle \land \langle a_2, b_2 \lex \langle a_3, b_3 \rangle \iff (a_1 <_A a_2 \lor (a_1 = a_2 \land b_1 \le_B b_2)) \land (a_2 <_A a_3 \lor (a_2 = a_3 \land b_2 \le_B b_3)) \iff (a_1 <_A a2 \land a_2 <_A a_3) \lor (a_1 = a_2 \land b_2 \le_B b_3 \land a_2 = a_3 \land b_2 \le_B b_3) \iff (a_1 <_A a_3) \lor (a_1 = a_3 \land b_1 \le_B b_3) \iff \langle a_1, b_1 \rangle \lex \langle a_3, b_3 \rangle$.
		\item אנטי־סימטרי: $\forall \langle a, b \rangle, \langle a', b' \rangle \in A \times B : \langle a, b \rangle \lex \langle a', b' \rangle \land \langle a', b' \rangle \lex \langle a, b \rangle \iff (a < a' \land a' < a) \lor (a = a' \land (b \le b' \land b' \le b)) \iff a = a' \land b = b' \iff \langle a, b \rangle = \langle a', b' \rangle$.
	\end{enumerate}
	ומצאנו כי שלושת התנאים מתקיימים וזהו אכן יחס סדר חלקי.
\end{proof}

\Subquestion{}
נוכיח כי $\langle A \times B, \lex \rangle$ סדורה קווית אם ורק אם $\langle A, \le_A \rangle, \langle B, \le_B \rangle$ סדורות קווית.
\begin{proof}
	\textbf{כיוון ראשון:}
	נניח כי $\langle A \times B, \lex \rangle$ סדורה קווית. \\*
	נבחן את $\langle a, b \rangle, \langle a', b \rangle$, אם הם שווים נובע $a = a'$,
	אם $\langle a, b \rangle \lex \langle a', b \rangle$ אז נובע $a \le_A a$ ואם $\langle a', b \rangle \lex \langle a, b \rangle$ אז באופן דומה נקבל $a' \le_A a$.
	קיבלנו כי $\langle A, \le_A \rangle$ קווית, ובאותו אופן בדיוק עבור $\langle a, b \rangle, \langle a, b' \rangle$ נקבל כי גם $\langle B, \le_B \rangle$ קווית.

	\textbf{כיוון שני:}
	נניח כי $\langle A, \le_A \rangle, \langle B, \le_B \rangle$ שניהם סדרים קוויים. \\*
	נבחן את $\langle a, b \rangle, \langle a', b' \rangle$, אם $a = a', b = b'$ אז גם $\langle a, b \rangle = \langle a', b' \rangle$. \\*
	אם $a \le_A a' \land b \le_B b'$ אז נובע כי $a <_A a'$ או $a = a'$ ולכן $\langle a, b \rangle \lex \langle a', b' \rangle$. \\*
	אם $a' \le_A a \land b' \le_B b$ אז באופן זהה השוויון מתקיים לכיוון השני. \\*
	אם $a <_A a'$ אז כמובן היחס מתקיים, וגם $a >_A a'$, ולכן נניח ש־$a = a'$. \\*
	אם $b \le_B b'$ אז $\langle a, b \rangle \lex \langle a', b' \rangle$, הכיוון השני זהה, וסיימנו לעבור על כל המקרים וראינו כי תמיד היחס מוגדר, ובהתאם היחס קווי.
\end{proof}

\Subquestion{}
נוכיח כי $\langle A \times B, \lex \rangle$ מבוסס היטב אם ורק אם $\langle A, \le_A \rangle, \langle B, \le_B \rangle$ מבוססים היטב שניהם.
\begin{proof}
	\textbf{כיוון ראשון:}
	נניח כי $\langle A \times B, \lex \rangle$ מבוסס היטב. \\*
	לכל תת־קבוצה $A' \subseteq A$ נקבל כי לקבוצה $A' \times \{ b \}$ (כאשר $b$ קבוע) יש איבר מינימלי. תחת ההגבלה נקבל כי $\lex \iff a <_A a' \lor a = a' \iff a \le_A a' \iff \le_A$,
	דהינו נקבל כי לכל $A' \subseteq A$ לא ריקה קיים איבר מינימלי עבור $\le_A$ ולכן $\langle A, \le_A \rangle$ מבוסס היטב. \\*
	ההוכחה ל־$B$ זהה לחלוטין ונובעת מהעובדה ש־$\langle a, b \rangle \lex \langle a, b' \rangle \iff b \le_B b'$.

	\textbf{כיוון שני:}
	נניח כי $\langle A, \le_A \rangle, \langle \langle B, \le_B \rangle$ סדרים מבוססים היטב, ותהינה $A' \subseteq A, B' \subseteq B$ כאשר $A', b' \ne \emptyset$. \\*
	לכן קיימים איברים מינימליים $a \in A', b \in B'$.
	נבדוק עתה את $\langle a', b' \rangle \in A' \times B'$ איבר כלשהו. \\*
	נניח ש־$\langle a', b' \rangle \lex \langle a, b \rangle$ ולכן נקבל $a \le_A a' \land b \le_B b' \land (a' <_A a \lor (a' = a \land b' \le_B b)) \iff a = a' \land b = b'$ ולכן $\langle a, b \rangle$ מינימלי ב־$A' \times B'$.
\end{proof}

\Question{}
נוכיח כי לכל $n, m \in \NN$ מתקיים $\langle [n] \times [m], \lex \rangle \cong \langle [n \cdot m ], \le \rangle$.
\begin{proof}
	נתחיל במציאת שיכון בין הסדרים. \\*
	נגדיר $f : \langle [n] \times [m], \lex \rangle \to \langle [n \cdot m ], \le \rangle$ על־ידי
	\[
		f(x, y) = mx + y
	\]
	ונקבל $\forall \langle x, y \rangle, \langle x', y' \rangle \in [n] \times [m] : \langle x, y \rangle \lex \langle x', y' \rangle \iff x < x' \lor (x = x' \land y \le y') \iff mx < mx' \lor (nx + y \le mx' + y') \iff mx + y \le mx' + y' \iff f(x, y) \lex f(x', y')$ כנביעה מ־$y < m$. \\*
	נבחין עתה כי כל $z \in [n \cdot m ]$ ניתן לייצוג על־ידי $z = mx + y$ ולכן נוכל לקבוע כי $f$ היא על ובהתאם $\langle [n] \times [m], \lex \rangle \cong \langle [n \cdot m ], \le \rangle$.
\end{proof}

\directlua{Q_number = 6}
\Question{}
נניח כי $\langle A, \le \rangle$ סדר קווי צפוף ללא מקסימום ומינימום. \\*
יהיו $\langle B, \le_B \rangle, \langle B', \le_{B'} \rangle$ סדרים קוויים בעלי סדר שלם ללא מינימום ומקסימום, \\*
$g : A \to B, g' : A \to B'$ שיכונים של הסדרים כך ש־$\rng(g), \rng(g')$ צפופים ב־$B, B'$ בהתאמה. \\*
נוכיח שקיים איזומורפיזם יחיד של סדרים $h : B \to B'$ המקיים $h \circ g = g'$.
\begin{proof}
	
\end{proof}

\end{document}
