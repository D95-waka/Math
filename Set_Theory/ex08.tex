\documentclass[a4paper]{article}

% packages
\usepackage{inputenc, fontspec, amsmath, amsthm, amsfonts, polyglossia, catchfile}
\usepackage[a4paper, margin=50pt, includeheadfoot]{geometry} % set page margins

% style
\AddToHook{cmd/section/before}{\clearpage}	% Add line break before section
\linespread{1.5}
\setcounter{secnumdepth}{0}		% Remove default number tags from sections
\setmainfont{Libertinus Serif}
\setsansfont{Libertinus Sans}
\setmonofont{Libertinus Mono}
\setdefaultlanguage{hebrew}
\setotherlanguage{english}

% operators
\DeclareMathOperator\cis{cis}
\DeclareMathOperator\Sp{Sp}
\DeclareMathOperator\tr{tr}
\DeclareMathOperator\im{Im}
\DeclareMathOperator\diag{diag}
\DeclareMathOperator*\lowlim{\underline{lim}}
\DeclareMathOperator*\uplim{\overline{lim}}

% commands
\renewcommand\qedsymbol{\textbf{משל}}
\newcommand{\NN}[0]{\mathbb{N}}
\newcommand{\ZZ}[0]{\mathbb{Z}}
\newcommand{\QQ}[0]{\mathbb{Q}}
\newcommand{\RR}[0]{\mathbb{R}}
\newcommand{\CC}[0]{\mathbb{C}}
\newcommand{\getenv}[2][] {
  \CatchFileEdef{\temp}{"|kpsewhich --var-value #2"}{\endlinechar=-1}
  \if\relax\detokenize{#1}\relax\temp\else\let#1\temp\fi
}
\newcommand{\explain}[2] {
	\begin{flalign*}
		 && \text{#2} && \text{#1}
	\end{flalign*}
}

% headers
\getenv[\AUTHOR]{AUTHOR}
\author{\AUTHOR}
\date\today

\title{פתרון מטלה 08 --- תורת הקבוצות (80200)}

\DeclareMathOperator\dom{dom}
\DeclareMathOperator\add{Add}
\DeclareMathOperator\mult{Mult}
\begin{document}
\maketitle
\maketitleprint{}

\directlua{ Q_number = 3; }
\Question{}
תהי $r$ קבוצה. נוכיח כי $\{ x : \exists y (\langle x, y \rangle \in r )\}$ היא קבוצה וגם $\{ y : \exists x(\langle x, y \rangle \in r )\}$ היא קבוצה. \\*
נסיק שאם $f$ פונקציה אז $\dom(f), \rng(f)$ קבוצות.
\begin{proof}
	נגדיר תכונה $p(x) = \exists y (\langle x, y \rangle \in r)$, ונוכל לבחור $\cup \cup r$ ואז מאקסיומת ההפרדה נקבל כי קיימת הקבוצה הראשונה, על־ידי הגדרת תכונה דומה והפעלת איחוד שלוש פעמים נקבל כי גם השנייה קיימת. \\*
	כמובן כל פונקציה $f$ היא קבוצת זוגות סדורים, וההגדרה של $\dom(f), \rng(f)$ היא שקולה לקבוצה הראשונה והשנייה שמצאנו כי קיימות זה עתה, ולכן נסיק כי התחום והתמונה של פונקציה הן קבוצות תמיד.
\end{proof}

\Question{}
תהי $X$ קבוצה. נוכיח במדויק על־ידי ZF את הטענות הבאות.

\Subquestion{}
נוכיח כי קיימת קבוצת כל יחסי השקילות על $X$.
\begin{proof}
	אנו יודעים כי יחס שקילות הוא יחס עם קיום תכונות מוגדרות נוספות, ולכן נגדיר $p(x)$ תכונה של קיום תכונות יחס שקילות. \\*
	מצאנו בהרצאה כי קבוצת הזוגות הסדורים $X \times X$ קיימת, ולכן מאקסיומת ההפרדה נוכל לטעון כי גם $\{ E \in X \times X \mid p(E) \}$ קיימת, וזו למעשה קבוצת כל יחסי השקילות על $X$.
\end{proof}

\Subquestion{}
נוכיח כי אם $E$ יחס שקילות על $X$ ו־$D$ מחלקת שקילות ביחס $E$, אז $D$ קבוצה.
\begin{proof}
	מצאנו עכשיו כי $E$ קבוצה, נגדיר $e \in X$ איבר כלשהו, ונגדיר $p(x) = \exists y \in X (\langle e, x \rangle \in E)$. \\*
	נשתמש שוב באקסיומת ההפרדה ונקבל כמובן כי $\{ x \in E \mid p(x) \}$ --- מחלקת שקילות המושרית על־ידי $e$ --- היא קבוצה. \\*
	נשאר לנו לקחת איבר כלשהו $e \in D$ ונקבל כי היא אכן קבוצה.
\end{proof}

\Subquestion{}
נוכיח שאם $E$ יחס שקילות על $X$, אז $X / E$ קבוצה.
\begin{proof}
	בסעיף הקודם מצאנו כי אם $x \in X$ אז מחלקת השקילות המכילה את $x$ היא קבוצה. \\*
	אנו יודעים כי $\mathcal{P}(X)$ היא קבוצה, ונבחין כי $D$ מחלקת השקילות כך ש־$x \in D$ מקיימת $D \in \mathcal{P}(X)$. \\*
	לכן אם כן נוכל להסיק כי התכונה $p(x) = \forall y, y' \in x \implies \langle \langle y, y' \rangle \rangle \in E$ ניתנת להפרדה ונקבל כי קבוצת מחלקות השקילות היא אכן קבוצה.
\end{proof}

\Question{}
תהי $X$ קבוצה, נוכיח שקיימת קבוצה $Y$ כך ש־$X \subseteq Y$ ולכל $y \in Y$ מתקיים $y \subseteq Y$.
\begin{proof}
	נגדיר פונקציה $g$ על־ידי $g(0) = X$ ו־$g(n + 1) = \bigcup_{a \in g(n)} a$. \\*
	על־ידי אקסיומת החלפה ומשפט הרקורסיה נסיק כי הפונקציה $g$ אכן קיימת ויחידה, ומאקסיומת האיחוד נקבל כי $Y = \bigcup_n g(n)$ אף היא קבוצה. \\*
	נותר רק לבדוק שהתנאי אכן מתקיים, יהי $y \in Y$, לכן קיים $n$ עבורו $y \in g(n)$, ואנו יודעים כי $y \subseteq \{ y \} \subseteq g(n + 1)$, ולכן קיבלנו כי $y \subseteq Y$ במתבקש.
\end{proof}

\Question{}
\Subquestion{}
ניעזר בפונקציה $\add : \NN \times \NN \to \NN$ כדי לבנות פונקציה $\mult : \NN \times \NN \to \NN$ המתארת על פעולת הכפל על הטבעיים.

נבחין כי על הפונקציה לקיים את התכונות:
\[
	\mult(0, 0) = 0
	\qquad
	\mult(n + 1, m) = \mult(n, m) + m
	\qquad
	\mult(n, m + 1) = \mult(n, m) + n
\]
כאשר אנו רושמים $X + n$ כסימון לביטוי $\add(X, n)$ ו־$X + 1$ כסימון ל־$s(X)$.

הוכחת קיום ויחידות הפונקציה $\mult$ כפי שהגדרנו אותה זה עתה נוכל לעשות בתהליך זהה להוכחה שראינו בהרצאה, נקבע סדרת פונקציות $f_n(m + 1) = m_n(m) + n$ עם בסיס $f_n(0) = 0$ ונקבל כי הן קיימות ומגדירות את $\mult$.
בשלב השני נוכיח באינדוקציה כי לכל $n$, לכל בחירת $m$, מתקיים גם $f_{n + 1}(m) = f_n(m) + m$.

\end{document}
