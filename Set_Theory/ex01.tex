\documentclass[a4paper]{article}

% packages
\usepackage{inputenc, fontspec, amsmath, amsthm, amsfonts, polyglossia, catchfile}
\usepackage[a4paper, margin=50pt, includeheadfoot]{geometry} % set page margins

% style
\AddToHook{cmd/section/before}{\clearpage}	% Add line break before section
\linespread{1.5}
\setcounter{secnumdepth}{0}		% Remove default number tags from sections
\setmainfont{Libertinus Serif}
\setsansfont{Libertinus Sans}
\setmonofont{Libertinus Mono}
\setdefaultlanguage{hebrew}
\setotherlanguage{english}

% operators
\DeclareMathOperator\cis{cis}
\DeclareMathOperator\Sp{Sp}
\DeclareMathOperator\tr{tr}
\DeclareMathOperator\im{Im}
\DeclareMathOperator\diag{diag}
\DeclareMathOperator*\lowlim{\underline{lim}}
\DeclareMathOperator*\uplim{\overline{lim}}

% commands
\renewcommand\qedsymbol{\textbf{משל}}
\newcommand{\NN}[0]{\mathbb{N}}
\newcommand{\ZZ}[0]{\mathbb{Z}}
\newcommand{\QQ}[0]{\mathbb{Q}}
\newcommand{\RR}[0]{\mathbb{R}}
\newcommand{\CC}[0]{\mathbb{C}}
\newcommand{\getenv}[2][] {
  \CatchFileEdef{\temp}{"|kpsewhich --var-value #2"}{\endlinechar=-1}
  \if\relax\detokenize{#1}\relax\temp\else\let#1\temp\fi
}
\newcommand{\explain}[2] {
	\begin{flalign*}
		 && \text{#2} && \text{#1}
	\end{flalign*}
}

% headers
\getenv[\AUTHOR]{AUTHOR}
\author{\AUTHOR}
\date\today

\title{פתרון מטלה 01 --- תורת הקבוצות (80200)}

\begin{document}
\maketitle
\maketitleprint{}

\Question{}
\Subquestion{}
נוכיח כי אם $F$ קבוצת זוגות סדורים אז $F$ היא פונקציה אם ורק אם $\forall x, y, y': \langle x, y\rangle, \langle x, y' \rangle \in F \implies y = y'$.
\begin{proof}
	\textbf{כיוון ראשון:}
	נניח כי $F$ פונקציה. \\*
	נגדיר $F : A \to B$, לכן לכל $x \in A$ קיים זוג סדור יחיד ב־$F$ אשר רכיבו השמאלי הוא $x$. לכן נובע ישירות כי אם $\langle x, y\rangle, \langle x, y' \rangle \in F$ אז $y = y'$. \\*
	\textbf{כיוון שני:}
	נניח $\forall x, y, y': \langle x, y\rangle, \langle x, y' \rangle \in F \implies y = y'$. \\*
	נבחר $A$ קבוצת כל ה־$x$־ים המקיימים את הטענה.
	לכן $\forall x \in A \exists y : \langle x, y \rangle \in F$. \\*
	עתה נשים לב שנתון כי אם $y, y'$ מקיימים את הטענה עבור $x \in A$ כלשהו, אז $y = y'$, ולכן נובע גם
	$\forall x \in A \exists ! y : \langle x, y \rangle \in F$. \\*
	לכן $F$ היא פונקציה על־פי הגדרה.
\end{proof}

\Subquestion{}
יהיו $f, g$ פונקציות, נוכיח כי $f \cap g$ היא פונקציה.
\begin{proof}
	נגדיר $A = dom(f), B = dom(g)$. אז לכל איבר $c \in A \cap B$ מתקיים מהנתון $f(c) = g(c)$. \\*
	לכן קבוצת הזוגות הסדורים $A \cap B$ מכילה זוג סדור אחד ויחיד לכל איבר ב־$A \cap B$ ומההגדרה נקבל כי זוהי פונקציה.
\end{proof}

\Subquestion{}
נראה דוגמה לפונקציות $f, g$ כך ש־$f \cup g$ לא פונקציה: \\*
נגדיר $A = \{0, 1\}$, ואת הפונקציות $f, g : A \to A$ על־ידי
\[
	f(x) = 1, g(x) = 0
\]
ולכן נובע
\[
	f = \{(0, 1), (1, 1)\}, g = \{(0, 0), (1, 0)\}
\]
אז
\[
	f \cup g = \{(0, 1), (1, 1), (0, 0), (1, 0)\}
\]
וזוהי כמובן לא פונקציה.

\Subquestion{}
נראה דוגמה לפונקציות $f, g$ כך שמתקיים $dom(f \cap g) \ne dom(f) \cap dom(g)$. \\*
נגדיר $A$ כמו בסעיף הקודם ו־$f, g : A \to A$. נגדיר
\[
	f = \{(0, 1), (1, 0)\}, g = \{(0, 0), (1, 0)\}, f \cap g = \{(1, 0)\}
\]
לכן
\[
	dom(f \cap g) = \{1\} \ne dom(f) \cap dom(g) = \{0, 1\} \cap \{0, 1\} = \{0, 1\}
\]

\Question{}
תהי $f : A \to B$.
נוכיח כי שלושת התנאים הבאים שקולים:
\begin{enumerate}
	\item $f$ הפיכה.
	\item $f$ חד־חד ערכית ועל.
	\item קיימת פונקציה $g : B \to A$ כך ש־$f \circ g = id_B, g \circ f = id_A$.
\end{enumerate}
\begin{proof}
	\textbf{1 \leftarrow{} 2}:
	ידוע כי $f$ הפיכה ולכן $f^{-1}$ היא פונקציה. \\*
	נניח בשלילה ש־$f$ לא חד־חד ערכית ולכן קיימים $a, b \in A$ כך ש־$a \ne b, f(a) = f(b) = y$.
	לכן על־פי הגדרה נובע כי $\langle y, a \rangle, \langle y, b \rangle \in f^{-1}$.
	אבל ידוע כי $f^{-1}$ פונקציה והגענו לסתירה, לכן $f$ חד־חד ערכית. \\*
	נניח בשלילה ש־$f$ לא על, לכן קיים $y \in B$ כך ש־$\forall a \in A, \langle a, y \rangle \not\in f$. אבל ידוע ש־$f^{-1}$ פונקציה ולכן $\langle y, x \rangle \in f^{-1}$ עבור $x \in A$ כלשהו.
	זוהי סתירה ולכן $f$ חד־חד ערכית ועל. \\*
	\textbf{2 \leftarrow{} 3}:
	נניח כי $f$ חד־חד ערכית ועל, ונגדיר $g = \{ \langle y, x \rangle \mid \langle x, y \rangle \in f \}$.
	ידוע כי $f$ חד־חד ערכית, ולכן כל איבר באגף ימני של $f$ לא חוזר על עצמו ובהתאם אין חזרה באגף ימני של $g$, והיא עומדת בהגדרת פונקציה.
	נניח שקיים $b \in B$ כך ש־$g(b)$ לא מוגדר, לכן אין $a \in A$ כך ש־$f(a) = b$, אך זו סתירה להיותה של $g$ על, ולכן לא קיים $b$ כזה.
	במילים אחרות, לכל $b \in B$ מתקיים $g(b) = a$ עבור $a$ כלשהו.
	עתה נראה כי $\forall a \in A, g(f(a)) = g(b) = a$ שכן אם $\langle a, b \rangle \in f$ אז $\langle b, a \rangle \in g$.
	באופן דומה נראה כי $\forall b \in B, f(g(b)) = f(a) = b$ ולכן מתקיים $f \circ g = id_B, g \circ f = id_A$. \\*
	\textbf{3 \leftarrow{} 1}:
	נניח כי קיימת פונקציה $g : B \to A$ כך ש־$f \circ g = id_B, g \circ f = id_A$. \\*
	מהנתון נובע כי אם $\langle a, b \rangle \in f$ אז $\langle b, a\rangle \in g$, ונתון כי היא פונקציה.
	לכן גם $f^{-1} = g$.
\end{proof}

\Question{}
תהינה $f : A \to B, g : B \to C$.

\Subquestion{}
נוכיח כי אם $g, f$ הן על, אז גם $g \circ f$ היא על.
\begin{proof}
	יהי $c \in C$, $g$ היא על ולכן קיים $b \in B$ כך ש־$g(b) = c$. \\*
	ידוע כי $f$ היא על ולכן קיים $a \in A$ כך ש־$f(a) = b$, ולכן $g(f(a)) = c$ ומצאנו כי $g \circ f$ היא על.
\end{proof}

\Subquestion{}
נפריך את הטענה כי אם $g$ על אז גם $g \circ f$ על על־ידי דוגמה נגדית: \\*
נגדיר $A = B = C = \{0, 1\}$, ונגדיר $g = id_A$ ולכן על. עוד נגדיר $f(x) = 1$. \\*
נראה כי $g(f(x)) = 1$ לכל $x$ ולכן לא קיים $x$ כך ש־$g(f(x)) = 0$ ובהתאם היא לא על.

\Subquestion{}
נסתור את הטענה כי אם $g$ היא חד־חד ערכית אם $g \circ f$ היא חד־חד ערכית על־ידי דוגמה נגדית: \\*
נגדיר $A = \{0\}, B = C = \{0, 1\}$ וגם $f(0) = 0$ ו־$g(x) = 0$. \\*
ניתן להבחין כי $f$ חד־חד ערכית, וגם כי $g \circ f$ היא חד־חד ערכית, אבל $g(0) = g(1)$.

\Question{}
תהינה קבוצות $A, B, C, D$ כך שמתקיים $|A| = |C|, |B| = |D|$.

\Subquestion{}
נוכיח כי $|A \times B| = |C \times D|$.
\begin{proof}
	משוויון העוצמות נניח שיש שתי פונקציות הפיכות $f : A \to C$ ו־$g : B \to D$. \\*
	נגדיר פונקציה חדשה $h : A \times B \to B \times D$ על־ידי $h(a, b) = \langle f(a), g(b) \rangle$. \\*
	זוהי כמובן פונקציה הפיכה שכן $f, g$ הפיכות, ולכן מתקיים שוויון העוצמות $|A \times B| = |C \times D|$.
\end{proof}

\Subquestion{}
נפריך את הטענה כי $|A \cup B| = |C \cup D|$ על־ידי דוגמה נגדית: \\*
נגדיר $A = \{0, 1\}, B = \{1, 2\}, C = D = \{0, 1\}$. \\*
ברור כי כלל הקבוצות בעלות עוצמה זהה, ובפרט $|A| = |C|, |B| = |D|$, אבל $A \cup B = \{0, 1, 2\}$ ואילו $C \cup D = \{0, 1\}$ ועוצמותיהן לא שוות.

\Question{}
נוכיח כי אם $A, B$, סופיות, ו־$|A| = n, |B| = m$, אז $|A \times B| = n \cdot m$.
\begin{proof}
	נגדיר $A = [n], B = [m]$, שאם לא כן נוכל להגדיר פונקציה הפיכה בין הקבוצות ועוצמותיהן שוות, לכן לא פגענו בכלליות ההוכחה. \\*
	נשים לב כי $A \times B = \{ \langle a, b \rangle \mid 0 \le a < n, 0 \le b < m \}$ \\*
	נגדיר $C = \{ 2^a \cdot 3^b \mid 0 \le a < n, 0 \le b < m \}$, בקבוצה זו יש בדיוק $nm$ איברים. \\*
	נראה כי הפונקציה $f : A \times B \to C$ המוגדרת על־ידי $f(a, b) = 2^a 3^b$ היא חד־חד ערכית ועל ולכן נקבל כי $|A \times B| = nm$.
\end{proof}

\Question{}
\Subquestion{}
נוכיח באינדוקציה על $n \in \NN$ שאין פונקציה חד־חד ערכית מ־$[n + 1]$ ל־$[n]$.
\begin{proof}
	נבחין כי עבור $n = 1$ הפונקציה היחידה האפשרית היא $f(0) = f(1) = 0$ וזו כמובן לא חד־חד ערכית וזהו בסיס האינדוקציה. \\*
	נניח כי טענת האינדוקציה נכונה עבור $n - 1$, דהינו אין פונקצה חד־חד ערכית מ־$[n]$ ל־$[n - 1]$. \\*
	תהי פונקציה כזו, ונרחיב אותה להיות מוגדרת מ־$[n + 1]$ ל־$[n]$ על־ידי $f(n + 1) = f(n)$. ידוע כי היא לא חד־חד ערכית, לכן ישנם לכל איבר בטווח 
\end{proof}

\end{document}
