\documentclass[a4paper]{article}

% packages
\usepackage{inputenc, fontspec, amsmath, amsthm, amsfonts, polyglossia, catchfile}
\usepackage[a4paper, margin=50pt, includeheadfoot]{geometry} % set page margins

% style
\AddToHook{cmd/section/before}{\clearpage}	% Add line break before section
\linespread{1.5}
\setcounter{secnumdepth}{0}		% Remove default number tags from sections
\setmainfont{Libertinus Serif}
\setsansfont{Libertinus Sans}
\setmonofont{Libertinus Mono}
\setdefaultlanguage{hebrew}
\setotherlanguage{english}

% operators
\DeclareMathOperator\cis{cis}
\DeclareMathOperator\Sp{Sp}
\DeclareMathOperator\tr{tr}
\DeclareMathOperator\im{Im}
\DeclareMathOperator\diag{diag}
\DeclareMathOperator*\lowlim{\underline{lim}}
\DeclareMathOperator*\uplim{\overline{lim}}

% commands
\renewcommand\qedsymbol{\textbf{משל}}
\newcommand{\NN}[0]{\mathbb{N}}
\newcommand{\ZZ}[0]{\mathbb{Z}}
\newcommand{\QQ}[0]{\mathbb{Q}}
\newcommand{\RR}[0]{\mathbb{R}}
\newcommand{\CC}[0]{\mathbb{C}}
\newcommand{\getenv}[2][] {
  \CatchFileEdef{\temp}{"|kpsewhich --var-value #2"}{\endlinechar=-1}
  \if\relax\detokenize{#1}\relax\temp\else\let#1\temp\fi
}
\newcommand{\explain}[2] {
	\begin{flalign*}
		 && \text{#2} && \text{#1}
	\end{flalign*}
}

% headers
\getenv[\AUTHOR]{AUTHOR}
\author{\AUTHOR}
\date\today

\title{פתרון מטלה 02 --- תורת הקבוצות (80200)}

\begin{document}
\maketitle
\maketitleprint{}

\Question{}
\Subquestion{}
נוכיח שאם $A$ קבוצה סופית וגם $|A| = n$ אז $|\mathcal{P}(A)| = 2^n$.
\begin{proof}
	נקבע ללא פגיעה בכלליות ההוכחה כי הקבוצה $A = [n]$. \\*
	נגדיר פונקציה $f : \mathcal{P}(A) \to [2^n]$ על־ידי
	\[
		f(X) = \sum_{i = 0}^{n} \begin{cases}
			2^i, & i \in X \\
			0, & i \not\in X
		\end{cases}
	\]
	פונקציה זו מתאימה מספר אי־שלילי לכל קבוצה $X \subset A$ על־ידי שימוש בבסיס בינארי, בדומה לייצוג עשרוני, לכן. \\*
	לכן נניח שהפונקציה חד־חד ערכית. לכל מספר $r \in [2^n]$ נוכל למצוא יצוג בינארי מהצורה $r_0 2^0 + r_1 2^1 + \hdots + r_n 2^n$ כאשר $r_i \in \{0, 1\}$ לכל $i \in [n]$. \\*
	אילו נבנה קבוצה $X = \{i \mid r_i = 1\}$ אז נקבל $f(X) = r$ ולכן הפונקציה גם על. \\*
	לכן נובע $|\mathcal{P}(A)| = |[2^n]| = 2^n$
\end{proof}

\Subquestion{}
נוכיח כי אם $|A| = |B|$ אז גם $|\mathcal{P}(A)| = |\mathcal{P}(B)|$.
\begin{proof}
	אילו הקבוצות $A, B$ סופיות אז הטענה נובעת ישירות מהסעיף הקודם. \\*
	נניח כי $A, B$ אינן סופיות. משוויון העוצמות נובע שקיימת $f : A \to B$ הפיכה. \\*
	נגדיר $g : \mathcal{P}(A) \to \mathcal{P}(B)$ על־ידי
	\[
		g(X) = \{ f(x) \mid x \in A \}, \qquad
		g^{-1}(X) = \{ f^{-1}(x) \mid x \in B \}
	\]
	הראינו פונקציה שמוכיחה כי הפונקציה היא על, ומהחד־חד ערכיות של $f$ נובעת חד־חד ערכיות של $g$, לכן $|\mathcal{P}(A)| = |\mathcal{P}(B)|$.
\end{proof}

\Subquestion{}
נוכיח כי מתקיים השוויון
\[
	\mathcal{P}(A) \cap \mathcal{P}(B) = \mathcal{P}(A \cap B)
\]
\begin{proof}
	\[
		\forall X \subseteq A \cap B
		\iff X \subseteq A \land X \subseteq B
		\iff X \in \mathcal{P}(A) \land X \in \mathcal{P}(B)
		\iff X \in \mathcal{P}(A \cap B)
	\]
	ולכן $\mathcal{P}(A) \cap \mathcal{P}(B) = \mathcal{P}(A \cap B)$.
\end{proof}

\Subquestion{}
נוכיח כי
\[
	\mathcal{P}(A) \cup \mathcal{P}(B) \subseteq \mathcal{P}(A \cup B)
\]
\begin{proof}
	\[
		\forall X \in \mathcal{P}(A) \cup \mathcal{P}(B)
		\iff X \subseteq A \lor X \subseteq B
		\implies X \subseteq A \cup B
		\iff X \in \mathcal{P}(A \cup B)
	\]
\end{proof}
נראה דוגמה שבה זו הכלה ממש:
\[
	A = \{0\}, B = \{1\}, 
	\mathcal{P}(A) \cup \mathcal{P}(B) = \{\emptyset, \{0\}, \{1\}\} \subset \mathcal{P}(A \cup B)
	= \{\emptyset, \{0\}, \{1\}, \{0, 1\} \}
\]

\Subquestion{}
נמצא קבוצה אינסופית $A$ כך ש־$A \subseteq \mathcal{P}(A)$. \\*
נגדיר איבר $\emptyset_0 = \emptyset$ ולכל $n \in \NN$ נגדיר $\emptyset_{n + 1} = \{\emptyset_n\}$. \\*
עתה נגדיר $A = \{\emptyset_i \mid i \in \NN \}$. \\*
נובע כי לכל $n$ גם $\emptyset_n \in A \land \emptyset_n \subseteq A$ ולכן גם $\emptyset_n \in \mathcal{P}(A)$ ובמסתכם $A \subseteq \mathcal{P}(A)$.

\Question{}
נבדוק את הרפלקסיביות, הסימטריה והטרנזיטיביות של היחסים הבאים:

\Subquestion{}
\[
	R_1 = \{ \langle x, y \rangle \in \QQ^2 \mid x - y \in \ZZ \}
\]
היחס רפלקסיבי שכן לכל $\langle x, x \rangle$ מתקיים $x - x = 0 \in \ZZ$. \\*
היחס סימטרי שכן $\forall \langle x, y \rangle \in R_1 : x - y \in \ZZ \iff y - x \in \ZZ$. \\*
היחס גם טרנזיטיבי שכן $\forall x, y, z \in R_1 : x R_1 y \land y R_1 z \implies x - y + y - z \in \ZZ, x - z \in \ZZ \implies \langle x, z \rangle \in R_1$.

\Subquestion{}
\[
	R_2 = \{ \langle A, B \rangle \in {(\mathcal{P}(X))}^2 \mid A \cap B \text{ אינסופי} \}
\]
היחס רפלקסיבי שכן $A \cap A = A$ לכל קבוצה, וסימטרי כנביעה ישירה מסימטריית חיתוך הקבוצות. \\*
היחס לא טרנזיטיבי. נגדיר $X = \NN, A = 2\NN, B = \NN, C = 2\NN + 1$ אז $A \cap B = A$ ולכן אינסופית, וגם $B \cap C = C$ אינסופית, אבל $A \cap C = \emptyset$.

\Subquestion{}
\[
	R_2 = \{ \langle A, B \rangle \in {(\mathcal{P}(X))}^2 \mid A \triangle B \text{ אינסופי} \}
\]
היחס לא רפלקסיבי, לדוגמה $\NN \triangle \NN = \emptyset$ והיא איננה אינסופית. \\*
היחס סימטרי כנביעה מסימטריית ההפרש הסימטרי. \\*
היחס גם לא טרנזיטיבי, נראה כי אם $A = C = \NN, B = \emptyset$ אז $A \triangle B = C \triangle B = \NN$ אינסופי אבל $A \triangle C = \emptyset$.

\Question{}
\Subquestion{}
תהי $A$ קבוצה ו־$E \subseteq A \times A$ יחס שקילות. \\*
נוכיח ש־$A / E$ היא חלוקה של $A$.
\begin{proof}
	יהיו $a, b \in A$ ונגדיר $A_a, A_b$ מחלקות השקילות של $a, b$ בהתאמה. \\*
	נשים לב ש־$A_a = A_b \iff aEb$, ולכן נניח מעתה כי $\langle a, b \rangle \not\in E$. \\*
	נקבל בהתאם כי $A_a \ne A_b$, ונראה שגם $A_a \cap A_b = \emptyset$, שאם לא כן קיים איבר $c \in A$ שמקיים $\langle a, c \rangle, \langle b, c \rangle \in E$ ויחד עם טרנזיטיביות $A$ נקבל סתירה. \\*
	כל $a \in A$ מגדיר איזושהי מחלקת שקילות, דהינו אין איבר כך שהוא לא נמצא באף מחלקת שקילות, ולכן איחודן הוא $A$ עצמה. \\*
	מצאנו כי $A/E$ היא חלוקה של $A$.
\end{proof}

\Subquestion{}
תהי $A$ קבוצה ו־$Q \subseteq \mathcal{P}(A)$ חלוקה של $A$, נגדיר
\[
	E_Q = \{ \langle a, b \rangle \in A^2 \mid \exists X \in Q : a \in X \land b \in X \}
\]
נוכיח כי $E_Q$ הוא יחס שקילות.
\begin{proof}
	נבדוק את כל התנאים ליחס שקילות:
	\begin{itemize}
		\item רפלקסיביות: $\forall a \in A \exists X \in Q : a \in X \land a \in X \implies \langle a, a \rangle \in E_Q$.
		\item סימטריה: $\forall \langle a, b \rangle \in E_Q \exists X : b \in X \land a \in X \implies \langle b, a \rangle \in E_Q$.
		\item טרנזיטיביות: $\forall \langle a, b \rangle, \langle b, c \rangle \exists X_1, X_2 : a \in X_1, b \in X_2, b \in X_2, c \in X_2 \implies X_1 = X_2 \implies c \in X_1, \langle a, c \rangle \in E_Q$.
	\end{itemize}
\end{proof}

\Subquestion{}
\subsubsection{i.}
נוכיח כי לכל יחס שקילות $A \subseteq A \times A$ מתקיים $E_{A/E} = E$.
\begin{proof}
	מצאנו בסעיף א' כי $A/E$ היא חלוקה של $A$, ובסעיף ב' מצאנו כי היחס המושרה על־ידי חלוקה מהווה יחס שקילות בין רכיבי החלוקה, ולכן נובע
	\[
		\forall \langle a, b \rangle \in E \iff \langle a, b \rangle \in A_{A/E}
	\]
	ומצאנו כי הטענה נכונה.
\end{proof}

\subsubsection{ii.}
נוכיח כי לכל חלוקה $Q \subseteq \mathcal{P}(A)$ מתקיים $A/E_Q = Q$.
\begin{proof}
	לא יודע
\end{proof}

\end{document}
