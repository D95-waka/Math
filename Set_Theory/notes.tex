\documentclass[a4paper]{article}

% packages
\usepackage{inputenc, amsmath, amsthm, thmtools, amsfonts, amssymb, luacode, catchfile, tikzducks, hyperref}
\usepackage[a4paper, margin=50pt, includeheadfoot]{geometry} % set page margins
\usepackage[shortlabels]{enumitem}
\usepackage[skip=3pt, indent=0pt]{parskip}

% language
\usepackage[bidi=basic, layout=tabular, provide=*]{babel}
\babelprovide[main, import]{hebrew}
\babelprovide{rl}
\babelfont{rm}{Libertinus Serif}
\babelfont{sf}{Libertinus Sans}
\babelfont{tt}{Libertinus Mono}

% style
\AddToHook{cmd/section/before}{\clearpage}	% Add line break before section
\linespread{1.3}
\setcounter{secnumdepth}{0}		% Remove default number tags from sections, this won't do well with theorems
\AtBeginDocument{\setlength{\belowdisplayskip}{3pt}}
\AtBeginDocument{\setlength{\abovedisplayskip}{3pt}}

% operators
\DeclareMathOperator\cis{cis}
\DeclareMathOperator\Sp{Sp}
\DeclareMathOperator\tr{tr}
\DeclareMathOperator\im{Im}
\DeclareMathOperator\re{Re}
\DeclareMathOperator\diag{diag}
\DeclareMathOperator*\lowlim{\underline{lim}}
\DeclareMathOperator*\uplim{\overline{lim}}
\DeclareMathOperator\rng{rng}
\DeclareMathOperator\Sym{Sym}
\DeclareMathOperator\Arg{Arg}
\DeclareMathOperator\Log{Log}
\DeclareMathOperator\dom{dom}

% commands
%\renewcommand\qedsymbol{\textbf{מש''ל}}
%\renewcommand\qedsymbol{\fbox{\emoji{lizard}}}
\newcommand{\NN}[0]{\mathbb{N}}
\newcommand{\ZZ}[0]{\mathbb{Z}}
\newcommand{\QQ}[0]{\mathbb{Q}}
\newcommand{\RR}[0]{\mathbb{R}}
\newcommand{\CC}[0]{\mathbb{C}}
\newcommand{\FF}[0]{\mathbb{F}}
\newcommand{\PP}[0]{\mathbb{P}}
\newcommand{\TT}[0]{\mathbb{T}}
\newcommand{\acts}[0]{\circlearrowright}
\newcommand{\explain}[2] {
	\begin{flalign*}
		 && \text{#2} && \text{#1}
	\end{flalign*}
}
\newcommand{\maketitleprint}[0]{ \begin{center}
	\begin{tikzpicture}[scale=3]
		\duck[graduate=gray!20!black, tassel=red!70!black]
	\end{tikzpicture}	
\end{center}
}

% theorem commands
\newtheoremstyle{c_remark}
	{}	% Space above
	{}	% Space below
	{}% Body font
	{}	% Indent amount
	{\bfseries}	% Theorem head font
	{}	% Punctuation after theorem head
	{.5em}	% Space after theorem head
	{\thmname{#1}\thmnumber{ #2}\thmnote{ \normalfont{\text{(#3)}}}}	% head content
\newtheoremstyle{c_definition}
	{3pt}	% Space above
	{3pt}	% Space below
	{}% Body font
	{}	% Indent amount
	{\bfseries}	% Theorem head font
	{}	% Punctuation after theorem head
	{.5em}	% Space after theorem head
	{\thmname{#1}\thmnumber{ #2}\thmnote{ \normalfont{\text{(#3)}}}}	% head content
\newtheoremstyle{c_plain}
	{3pt}	% Space above
	{3pt}	% Space below
	{\itshape}% Body font
	{}	% Indent amount
	{\bfseries}	% Theorem head font
	{}	% Punctuation after theorem head
	{.5em}	% Space after theorem head
	{\thmname{#1}\thmnumber{ #2}\thmnote{ \text{(#3)}}}	% head content

\theoremstyle{c_plain}
\newtheorem{theorem}{משפט}[section]
\newtheorem{lemma}[theorem]{למה}
\newtheorem{proposition}[theorem]{טענה}
\newtheorem*{proposition*}{טענה}
%\newtheorem{corollary}[theorem]{אין חלופה עברית}

\theoremstyle{c_definition}
\newtheorem{definition}[theorem]{הגדרה}
\newtheorem*{definition*}{הגדרה}
\newtheorem{example}{דוגמה}[section]
\newtheorem{exercise}{תרגיל}[section]

\theoremstyle{c_remark}
\newtheorem*{remark}{הערה}
\newtheorem*{solution}{פתרון}
\newtheorem{conclusion}[theorem]{מסקנה}
\newtheorem{notation}[theorem]{סימון}

% Questions related commands
\newcounter{question}
\setcounter{question}{1}
\newcounter{sub_question}
\setcounter{sub_question}{1}

\newcommand{\question}[1][0]{
	\ifthenelse{#1 = 0}{}{\setcounter{question}{#1}}
	\subsection{שאלה \arabic{question}}
	\addtocounter{question}{1}
	\setcounter{sub_question}{1}
}

\newcommand{\subquestion}[1][0]{
	\ifthenelse{#1 = 0}{}{\setcounter{sub_question}{#1}}
	\subsubsection{סעיף \localecounter{letters.gershayim}{sub_question}}
	\addtocounter{sub_question}{1}
}

% import lua and start of document
\directlua{common = require ('../common')}

\GetEnv{AUTHOR}

% headers
\author{\AUTHOR}
\date\today

\usepackage{hyperref}
\hypersetup{}
\setcounter{secnumdepth}{2}
\title{תורת הקבוצות}

\begin{document}
\maketitle
\maketitleprint{}

\tableofcontents

\section{שיעור 1 --- 8.5.2024}

מרצה: עומר בן־נריה, מייל: omer.bn@mail.huji.ac.il

\subsection{מבוא}
הקורס בנוי מחצי של תורת הקבוצות הנאיבית, בה מתעסקים בקבוצה באופן כללי ולא ריגורזי, ומחצי של תורת הקבוצות האקסיומטית, בה יש הגדרה חזקה להכול. \\*
הסיבה למעבר לתורה אקסיומטית נעוצה בפרדוקסים הנוצרים ממתמטיקה לא מוסדרת, לדוגמה הפרדוקס של בנך־טרסקי. \\*
עוד דוגמה היא פרדוקס ראסל, אם במתמטקיה שואלים אילו קבוצות קיימות, אינטואיטיבית אפשר להניח שכל קבוצה קיימת, הפרדוקס מתאר שזה לא ממש אופציונלי. נניח שכל קבוצה קיימת, אז ניקח את הקבוצה $y = \{x \mid x \not\in x\}$.
מה אפשר להגיד על $y \in y$ ועל $y \not\in y$, אז נראה כי $y \in y \implies y \not\in y, y\not\in y \implies y \in y$ ואלו הן סתירות מן הסתם. \\*
התוכנית של הילברט, היא ניסיון להגדיר אקסיומטית בסיס רוחבי למתמטיקה, אבל ניתן להוכיח שגם זה לא עובד בלא מעט מקרים.
מומלצת קריאה נוספת על Zermelo Frankel ZF בהקשר לסט האקסיומות הבסיסי המקובל היום.

\subsection{עוצמות}
העוצמה של קבוצה $A$ היא הגודל של $A$. \\*
שאלות: איך משווים בין גדלים של קבוצות $A$ ו־$B$? \\*
הגדרה: נאמר כי זוג קבוצות $A$ ו־$B$ הן שוות עוצמה ונסמן $|A| = |B|$, אם ורק אם יש פונקציה הפיכה $F: A \to B$.

\subsection{תזכורת על פונקציות}
סימון: הזוג הסדור של אובייקטים $x, y$ יסומן $\langle x, y \rangle$. \\*
הערה: אם $x \ne y$ אז $\langle x, y \rangle \ne \langle y, x\rangle$ \\*
המכפלה הקרטזית של קבוצות $A, B$ היא הקבוצה
\[
	A \times B = \{ \langle a, b \rangle \mid a \in A, b \in B \}
\]
הגדרה: יחס בין $A$ ל־$B$ קבוצות, הוא תת־קבוצה $R$ של המכפלה הקרטזית, $R \subseteq A \times B$. \\*
הגדרה: פונקציה $F : A \to B$ היא יחס $F \subseteq A \times B$ המקיים כי $\forall a \in A \exists ! b \in B : \langle a, b \rangle \in F$. \\*
הערה חשובה: $\exists! $ קיים מקרה אחד בלבד כך שמתקיימת טענה. \\*
דוגמה 1: $A = \{0, 1\}, B = \{3, \pi \}, R_1 = \{\langle 0, 3\rangle\}$ לא פונקציה. \\*
דוגמה 2: אותן קבוצות, אבל $R_2 = \{ \langle 0, \pi\rangle, \langle 1, \pi\rangle\}$, היא אכן פונקציה. \\*
דוגמה 3: לכל קבוצה $X$ נסמן $Id_X = \{ \langle a, a \rangle \mid a \in X \}$ מתקיים $Id_X : X \to X$ והיא פונקציית הזהות. \\*
הגדרה: יהי יחס $R \subseteq A \times B$ נגדיר $dom(R) = \{ a \in A \mid \exists b \in B \langle a, b\rangle \in R\}$. \\*
נגדיר $rng(R) = \{ b \in A \mid \exists a \in A \langle a, b\rangle \in R\}$, נקרא לזה גם תמונה של $R$. \\*
הבחנה: אם $R \subseteq A \times B$ הוא פונקציה מ־$A$ ל־$B$ אז $dom(R) = A$ ועוד נראה כי $rng(R) \subseteq B$. \\*
הגדרות בסיסיות נוספות:
\begin{enumerate}
	\item בהינתן $F : A \to B$ אז נסמן לכל $a \in A$ את $F(a)$ להיות $b \in B$ היחיד עבורו מתקיים $\langle a, b \rangle \in F$.
	\item פונקציה $F : A \to B$ היא חד־חד ערכית אם לכל $a_1 \ne a_2$ איברים $a_1, a_2 \in A$ אז מתקיים $F(a_1) \ne F(a_2)$.
	\item פונקציה $F: A \to B$ תיקרא על אם לכל $b \in B$ קיים $a \in A$ כך ש־$\langle a, b \rangle \in R$, או גם $rng(F) = B$.
	\item בהינתן יחס $R$ נגדיר את היחס ההופכי $R^{-1} \subseteq B \times A$ להיות $R^{-1} = \{ \langle b, a\rangle \mid \langle a, b \rangle \in R\}$.
	\item פונקציה $F : A \to B$ נקראת הפיכה אם היחס ההופכי $F^{-1}$ הוא פונקציה מ־$B$ ל־$A$ ונרשום $F^{-1} : B \to A$.
\end{enumerate}
תרגיל: $F : A \to B$ היא הפיכה, אם ורק אם היא חד־חד ערכית ועל $B$. \\*
מסקנה: אם $F : A \to B$ היא פונקציה חד־חד ערכית ועל אז גם הפונקציה ההופכית שלה $F^{-1} : B \to A$ היא חד־חד ערכית ועל. \\*
הוכחה: נתון $F : A \to B$ ונתון כי היא חד־חד ערכית ועל, נסיק כי $F$ הפיכה גם כן ולכן הגדרת ההפיכה מעידה כי $F^{-1} : B \to A$ היא פונקציה. \\*
לכן ${(F^{-1})}^{-1}$ היא פונקציה ולכן $F^{-1}$ היא הפיכה על־פי הגדרה ובהתאם גם חח''ע ועל. \\*
הגדרה: הרכבת יחסים. נניח כי קיימים שני יחסים $R \subseteq A \times B, S \subseteq B \times C$ אז נגדיר $S \circ R \subseteq A \times C$ על־ידי
\[
	S \circ R = \{ \langle a, c \rangle \mid a \in A, c \in C, \exists b \in B : \langle a, b \rangle \in R \land \langle b, c \rangle \in S \}
\]
תרגיל: אם $F : A \to B$ ו־$G : B \to C$ אז $G \circ F \subseteq A \times C$ הוא יחס שהוא גם פונקציה. \\*
הבחנות שהן גם תרגיל: בהינתן פונקציות כמו שהגדרנו השנייה אז מתקיימים המצבים הבאים:
\begin{enumerate}
	\item אם $F, G$ הן חד־חד ערכיות, אז גם $G \circ F$ היא חד־חד ערכית.
	\item אם $F, G$ על אז גם $G \circ F$ היא על.
	\item $F, G$ הפיכות אז $G \circ F$ הפיכה גם היא.
	\item $F$ הפיכה אז $Id_A = F^{-1} \circ F$ וגם $Id_B = F \circ F^{-1}$
\end{enumerate}
נחזור לעוצמות: \\*
נראה כי שוויון עוצמות הוא יחס שקילות:
\begin{enumerate}
	\item אם יש $F: A \to B$ הפיכה אז גם יש $F^{-1} : B \to A$ ולכן $|A| = |B| \iff |B| = |A|$. כלומר יחס שוויון עוצמה הוא סימטרי.
	\item לכל $A$ מתקיים $|A| = |A|$ שכן $Id_A : A \to A$ היא הפיכה לעצמה.
	\item אם $|A| = |B|$ וגם $|B| = |C|$ אז גם $|A| = |C|$ בגלל היכולת להרכיב פונקציות הפיכות מתאימות.
\end{enumerate}

\subsection{קבוצות סופיות}
סימון לכל $n \ge 0$ נסמן $[n] = \{0, 1, \hdots, n - 1 \}$. \\*
הגדרה זמנית: הקבוצה $A$ נקראת סופית אם קיים $n \in \NN$ כך שמתקיים $|A| = |[n]|$. \\*
הבחנה: לכל קבוצה סופית $A \ne \emptyset$ אם $A^*$ מתקבלת מ־$A$ על־ידי השמטת איבר אז $|A| \ne |A^*|$. \\*
טענה: קבוצת כל המספרים הטבעיים $\NN = \{0, 1, \hdots\}$ אינה סופית. \\*
הוכחה: נסמן $\NN^* = \NN\setminus\{0\}$ ונגדיר $F : \NN \to \NN^*$ על־ידי $F(n) = n + 1$, בבירור $F$ חד־חד ערכית ועל $\NN^*$ ולכן $|\NN| = |\NN^*|$.

צריך להשלים את הסוף של ההאצאה.

\section{שיעור 2 --- 15.5.2024}
\subsection{תוצאות ראשונות בשוויון עוצמות}
\subsubsection{הקדמה למשפט קנטור}
לכל מספר $x \in \RR$ יש חלק שלם וחלק שברי כך שמתקיים $x = \lfloor x \rfloor + \langle x \rangle$. \\*
במקרה זה $\lfloor x \rfloor = n \in \ZZ$, כאשר $n \le x$. \\*
נובע כי $0 \le x - \lfloor x \rfloor < 1$. נגדיר $\langle x \rangle = x - \lfloor x \rfloor$. \\*
כל מספר $\langle x \rangle$ ניתן להצגה כהצגה בצור
\[
	\langle x \rangle = 0.x_1x_2 \hdots x_k \hdots
\]
נשים לב כי צורת הצגה זו היא יחידה פרט למקרה בודד בו ''הזנב'' של הספרות נגמר ב־$x_k = 0$ או כאשר הזנב נגמר ב־$x_k = 9$. \\*
לדוגמה $0.359999\hdots = 0.360000\hdots$.

\subsubsection{מונח: פיתוח סטנדרטי}
לכל מספר $x$ עבורו ל־$\langle x \rangle$ יש פיתוח יחיד נקרא לו פיתוח \textbf{סטנדרטי}. \\*
אחרת אם ל־$\langle x \rangle$ יש שני פיתוחים, אז נבחר את זה המסתיים ב־$x_k = 0$ להיות הסטנדרטי.

\subsubsection{משפט קנטור}
\begin{proof}
	נראה כי לכל פונקציה $f : \NN \to \RR$ אז $f$ איננה על $\RR$. \\*
	לכל $n \in \NN$ נרשום את הפיתוח הסטנדרטי של $\langle f(n) \rangle$:
	\[
		\langle f(n) \rangle = 0.x_0^n x_1^n x_2^n \hdots
	\]
	\[
		\begin{matrix}
			\langle f(0) \rangle & 0.x_0^0 & x_1^0 & x_2^0 & \hdots \\
			\langle f(1) \rangle & 0.x_0^1 & x_1^1 & x_2^1 & \hdots \\
			\langle f(2) \rangle & 0.x_0^2 & x_1^2 & x_2^2 & \hdots \\
		\end{matrix}
	\]
	ונבחן את האלכסונים, ונבנה מספר כך שלכל ערך אלכסוני נבחר ספרה שונה מהערך האלכסוני. לכן נוכל לבנות מספר שלא מופיע בכלל ברשימה הזו. \\*
	נתבונן כעת במספר $y \in \RR$ המוגדר על־ידי הפיתוח $y = 0.y_1 y_2 \hdots$ כאשר לכל $n \in \NN$ אנו מגדירים
	\[
		y_n = \begin{cases}
			2, & x_n^n \ne 2 \\
			7, & x_n^n = 2
		\end{cases}
	\]
	מכיוון שכל הספרות בפיתוח הנתון הן 2 או 7 אז פיתוח זה הוא הפיתוח הסטנדרטי של $y$. \\*
	לכל $n \in \NN$ לא יתכן ש־$y = f(n)$ שכן אחרת $\langle y \rangle = \langle f(n) \rangle$ ומכאן של־$\langle y \rangle$ ול־$\langle f(n) \rangle$ אותו פיתוח סטנדרטי בסתירה לכך ש־$y_n \le x_n^n$. \\*
	מסיקים $\forall n \in \NN : y \ne f(n)$ ולכן $y \not\in rng(f)$ ובהתאם $f$ איננה על $\RR$.
\end{proof}

הגדרות נוספות:
\subsubsection{אי־שוויון עוצמות}
עבור קבוצות $A, B$ נאמר שעוצמת $A$ קטנה מעוצמת $B$ או $|A| \le |B|$ כאשר יש פונקציה חד־חד ערכית $f : A \to B$. \\*
נאמר שעוצמת $A$ קטנה ממש מעוצמת $B$ אם $|A| \le |B| \land |A| \ne |B|$.

\textbf{מסקנה:} $|\NN| < |\RR|$ \\*
זאת משום ש־$\NN \subseteq \RR$ ולכן $|\NN| \le |\RR|$ והוכחנו במשפט קנטור ש־$|\NN| \ne |\RR|$.

\subsection{שאלות המשך}
\subsubsection{שאלה 1}
\[
	\NN \subseteq \ZZ \subseteq \QQ \subseteq \text{Alg}_\RR \subseteq \RR
\]
מהן עוצמות קבוצות הביניים בין $\NN$ ל־$\RR$?

\subsubsection{שאלה 2}
האם יש גודל אינסופי מירבי?

\subsubsection{קבוצה בת־מנייה}
קבוצה $A$ ששוות עוצמה ל־$\NN$ תיקרא \textbf{בת־מנייה}.

\subsubsection{קבוצה מעוצמת הרצף}
קבוצה $A$ ששוות עוצמה ל־$\RR$ תיקרא בעוצמת הרצף.

\subsubsection{משפט קנטור־שרדר־ברנשטיין}
תהינה שתי קבוצות $A, B$, אם $|A| \le |B|$ וגם $|B| \le |A|$ אז $|A| = |B|$.
\begin{proof}
	נדחה לסוף הפרק, יושלם בהמשך
\end{proof}

\subsubsection{טענה: עוצמת הטבעיים ומכפלת הטבעיים בעצמם}
\[
	|\NN| = |\NN \times \NN|
\]
נתאהר שתי הוכחות שונות למשפט.
\begin{proof}[בניית הנחש]
	\[
		\begin{matrix}
			(0, 0) & (0, 1) & (0, 2) & \hdots & (0, n) & \hdots \\
			(1, 0) & (1, 1) & (1, 2) & \hdots & (1, n) & \hdots \\
			(2, 0) & (2, 1) & (2, 2) & \hdots & (2, n) & \hdots \\
			\vdots \\
			(m, 0) & (m, 1) & (m, 2) & \hdots & (m, n) & \hdots
		\end{matrix}
	\]
	ונעבור על המטריצה הזאת באופן אלכסוני. \\*
	נגדיר $f : \NN \times \NN \to \NN$ על־ידי
	\[
		f(i, j) = \frac{(i + j)(i + j + 1)}{2} + i
	\]
\end{proof}
\begin{proof}[שימוש במשפט קנטור־שרדר־ברנשטיין]
	נמצא שתי פונקציות
	\[
		f : \NN \times \NN \to \NN,
		g : \NN \to \NN \times
	\]
	את $f$ נגדיר על־ידי $f(n) = (0, n)$. \\*
	ונגדיר $g(i, j) = 2^i 3^j$. שתי הפונקציות כמובן חד־חד ערכיות. \\*
	נובע מיחידות הצגת מספרים טבעיים כמכפלת ראשוניים.
\end{proof}

\subsubsection{טענה: מכפלת קבוצות בנות מנייה}
אם $A, B$ קבוצות בנות מנייה, אז גם $A \times B$ בת מנייה.
\begin{proof}
	נתון $A, B$ בנות מנייה אז ניקח פונקציה $h_B : \NN \to B$ חד־חד ערכית על $B$, \\*
	וניקח $h_A : \NN \to A$
	נקבע פונקציה חד־חד ערכית ועל $f : \NN \to \NN \times \NN$ (מטענה קודמת), $f(n) = (i_n, j_n)$, ונגדיר
	\[
		H : \NN \to A \times B, H(n) = (h_A(i_n), h_B(j_n)) \in A \times B
	\]
	נראה כי $H$ חד־חד ערכית, נניח $n \ne m$ אז $f(n) \ne f(m)$ (כי $f$ חד־חד ערכית). \\*
	אז או $i_n \ne i_m$ או $j_n \ne j_m$ ונקבל $H(n) \ne H(m)$. \\*
	$H$ גם על: $a \in A, b \in B$ וקיימים $i, j \in \NN$ כך ש־$a = h_A(i), b = h_B(j)$, נובע מזה שהן על. \\*
	ידוע כי יש $n \in \NN$ כך ש־$f(n) = (i, j)$ ולכן מחיבור הטענות נקבל כי $H(n) = (a, b)$.
\end{proof}

\subsubsection{הגדרה: חזקה קרטזית}
לכל קבוצה $A$ ו־$k \in \NN$ נגדיר $A^k$ באופן הבא: \\*
אילו $k = 1$ אז $A^k = A$ ובמקרה ש־$k > 1$ אז $A^{k + 1} = A^k \times A$. \\*
סימון: נסמן את אברי $A^k$ על־ידי $(a_1, a_2, \hdots, a_k)$, זאת למרות שבמציאות הקבוצה מוגדרת כ־$(((a_1, a_2), \hdots), a_k)$

\subsubsection{טענה: חזקה קרטזית בת מנייה}
לכל קבוצה $A$ בת־מנייה ו־$k \ge 1$ טבעי נובע $A^k$ בת־מנייה.
\begin{proof}
	באינדוקציה על $k$ ושימוש בטענה האחרונה.
\end{proof}

\subsubsection{קבוצת הרציונליים היא בת־מנייה}
$\QQ$ היא בת־מנייה.
\begin{proof}
	נשתמש במשפט קנטור־שרדר־ברנשטיין
	\[
		\NN \subseteq \QQ \implies |\NN| \le |\QQ|
	\]
	כדי להראות ש־$|\QQ| \le |\NN|$ מספיק לבנות פונקציה חד־חד ערכית לקבוצה בת מנייה כלשהי. \\*
	נגדיר $f : \QQ \to A$. לכל מספר רציונלי $z \ne 0$ יש הצגה יחידה בצורה $z = \pm \frac{p}{q}$ כאשר $p, q > 0$ טבעיים וזרים. \\*
	נגדיר $f : \QQ \to \NN \times \NN \times \NN$ על־ידי
	\[
		f(z) = \begin{cases}
			(0, 0, 0), & z = 0 \\
			(1, p, q), & z > 0 \\
			(2, p, q), & z < 0
		\end{cases}
	\]
	נובע מהגדרתה כי $f$ היא חד־חד ערכית ולכן $|\QQ| \le |\NN \times \NN \times \NN| = |\NN^3| = |\NN|$.
\end{proof}

\section{שיעור 3 --- 22.5.2024}
\subsection{קבוצת הסדרות הסופיות}
\subsubsection{הגדרה}
בהינתן קבוצה $A$ נגדיר 
\[
	seq(A) = \bigcup_{k \ge 1} A^k
\]
קבוצת כל הסדרות הסופיות של $A$.

\subsubsection{טענה: קבוצת הסדרות הסופיות היא בת־מניה}
לכל קבוצה בת־מניה $A$ גם $seq(A)$ היא בת־מניה.

\textbf{טענת עזר:} נניח ש־$B_n$ סדרת קבוצות ו־$h_n$ סדרת פונקציות. $h_n : \NN \to B_n$ הפיכה.\\*
בפרט מתקבל כי $B_n$ בת־מניה, אז הקבוצה
\[
	\bigcup_{n \in \NN} B_n = \{ b \mid \exists n \in \NN, b \in B_n \}
\]
נוכיח ראשית את הטענה בהינתן טענת העזר. \\*
תהי $h : \NN \to A$ הפיכה.
נתון כי $A$ בת־מניה, ונגדיר סדרת פונקציות ${(h_k)}_{k = 1}^\infty$, $h_k : \NN \to A^k$. \\*
נבחר $h_1 = h$. בהינתן $h_k$ נגדיר את $h_{k + 1}$ באופן הבא:
\[
	\tilde{h}_{k + 1} = h_k \times h_1 : \NN \times \NN \to A^k \times A
\]
אנו יודעים כי $\tilde{h}_{k + 1}$ הפיכה, ונשתמש בפונקציה ההפיכה $f : \NN \to \NN \times \NN$ מהשיעור הקודם ונגדיר
\[
	h_{k + 1} = \tilde{h}_{k + 1} \circ f : \NN \to A^{k + 1}
\]
אז תיארנו סדרה של פונקציות $h_k : \NN \to A^k$ הפיכות ומטענת העזר נסיק $seq(A) = \bigcup_{k \ge 1} A^k$ היא בת־מניה. \\*
נוכיח את טענת העזר: \\*
נשתמש במשפט קנטור־ברנשטיין ונראה כי$|\NN| \le | \bigcup_{n \in \NN}B_n|$ הוא א' ו־$|\NN| \ge | \bigcup_{n \in \NN}B_n|$ הוא ב'. \\*
א': נתון כי $B_0$ בת־מנייה, תהי $f_0 : \NN \to B_0$ הפיכה. נשים לב כי ניתן להתייחס ל־$f_0$ כפונקציה לאיחוד והיא עדיין חד־חד ערכית, לכן $|\NN| \le | \bigcup_{n \in \NN}B_n|$. \\*
עתה לב'. מכיוון ש־$|\NN \times \NN| = |\NN|$ די להראות כי קיימת פונקציה חד־חד ערכית
\[
	g : \bigcup_{n \in \NN} \to \NN \times \NN
\]
לכל $n$ נסמן $h_n$ ההופכית של $g_n$ פונקציה על. \\*
נגדיר את $g$ באופן הבא. יהי $b \in \bigcup_{n \in \NN} B_n$ נסמן $n(b) \in \NN$ המספר הטבעי הקטן ביותר עבורו מתקיים $b \in B_{n(b)}$. \\*
נשים לב כי $b \in B_{n(b)} \implies g_{n(b)}(b) \in \NN$ ובפרט מוגדר. \\*
ניקח
\[
	g(b) = \langle n(b), g_{n(b)}(b) \rangle
\]
נבדוק כי $g$ היא חד־חד ערכית. \\*
יהיו $b \ne b^*$ איברים באיחוד. \\*
נפריד לשני מקרים:
\begin{enumerate}
	\item אם $n(b) \ne n(b^*)$ בוודאי $g(b) \ne g(b^*)$.
	\item אם $n(b) = n(b^*)$ נסמן $n(b) = n(b^*) = m$ אז נסיק ש־$b, b^* \in B_m$ ו־$b \ne b^*$. מכיוון ש־$g_m$ חד־חד ערכית אז נקבל $g_{n(b)}(b) = g_m(b) \ne g_m(b^*) = g_{n(b)}(b)$ ובפרט $g(b) \ne g(b^*)$.
\end{enumerate}

\subsection{משפט קנטור על קבוצת החזקה}
\subsubsection{הגדרה}
בהינתן קבוצה $A$ מגדירים
\[
	\mathcal{P}(A) = \{ B \mid B \subseteq A \}
\]

\subsubsection{דוגמה}
$\mathcal{P}(\emptyset) = \{ \emptyset \}$. \\*
$| \mathcal{P}(\{1, 2, \dots, n \})| = |[2^n]|$

\subsubsection{משפט קנטור}
לכל קבוצה $A$ מתקיים
\[
	|\mathcal{P}(A)| > |A|
\]
\begin{proof}
	הוכחת $A \le \mathcal{P}(A)$: \\*
	נגדיר פונקציה $f : A \to \mathcal{P}(A)$ המוגדרת על־ידי $f(a) = \{a\} \in \mathcal{P}(A)$. \\*
	$f$ חד־חד ערכית ועונה על המבוקש.\\*
	כיוון $|A| \ne |\mathcal{P}(A)|$: \\*
	נוכיח כי לא קיימת פונקציה $g : A \to \mathcal{P}(A)$ שהיא על $\mathcal{P}(A)$. \\*
	תהי $g$ כלשהי, ונגדיר $B \subseteq A$ באופן הבא
	\[
		B = \{ a \in A \mid a \not\in g(a) \}
	\]
	כמובן ש־$B \in \mathcal{P}(A)$ ונטען כי $B \not\in rng(g)$ ומכאן ש־$g$ אינה על $\mathcal{P}(A)$. \\*
	נניח אחרת, אז יש $a^* \in A$ כך ש־$B = g(a^*)$. \\*
	נבדוק האם $a^* \in B$. אם $a^* \in B \overset{\text{הנחת השלילה}}{\iff} a^* \in g(a^*) \overset{\text{הגדרת $B$}}{\iff} a^* \not\in B$. \\*
	קיבלנו סתירה להנחת השלילה ולכן $|A| \ne |\mathcal{P}(A)|$.
\end{proof}

\subsubsection{עוצמות אינסופיות}
נקבל עכשיו ש־$|\NN| < |\mathcal{P}(\NN)| < |\mathcal{P}(\mathcal{P}(\NN)) |$ ונוכל לקבל שלכל $n \in \NN$ מתקיים
\[
	|\mathcal{P}^n(\NN)| < |\mathcal{P}^{n + 1}(\NN)|
\]
נגדיר
\[
	\bigcup_{k \ge 1} \mathcal{P}^k(\NN) = \mathcal{P}^\omega(\NN)
\]
תרגיל:
\[
	\forall k \in \NN \mathcal{P}^k(\NN) < \mathcal{P}^\omega(\NN)
\]
וכמובן גם
\[
	\mathcal{P}^\omega(\NN) = \mathcal{P}(\mathcal{P}^\omega(\NN))
\]
האם קיימת עוצמה גדולה ביותר?

\subsection{פעולות על מחלקות שקילות}
\subsubsection{תזכורת: יחס שקילות}
יחס $E \subseteq X \times X$ הוא יחס שקילות אם הוא רפלקסיבי, סימטרי וטרנזיטיבי.

\subsubsection{דוגמות}
\begin{enumerate}
	\item $X_1 = \ZZ$ והיחס $E_1 = \{ (a, b) \in (\ZZ, \ZZ) \mid a^2 = b^2 \}$
	\item $X_2 = \NN \times \NN$ ו־$E_2 = \{ ((n, m), (n', m')) \mid n + m' = n' + m \}$.
\end{enumerate}

בהינתן יחס שקילות $E$ על קבוצה $X$ מגדירים לכל $x \in X$ את
\[
	{[x]}_E = \{ y \in X \mid (x, y) \in E\}
\]
תכונה חשובה, לכל $x, x^*$ אם ${[x]}_E \cap {[x^*]}_E \ne \emptyset$ אז ${[x]}_E ={[x^*]}_E$. \\*
בדוגמה 1 ${[1]}_E = \{1, -1 \}$ ו־${[0]}_E = \{ 0 \}$. \\*
בנוגע לדוגמה 2 תרגיל בדקו כי זהו יחס שקילות ונראה כי מחלקות השקילות הן $\forall (n, m) \in \NN \times \NN$
מתקיים רק אחד מהשניים:
\begin{enumerate}
	\item $n \ge m$ ולכן $(n - m, 0) \in {[(n, m)]}_{E_2}$
	\item $n < m$ ולכן $(0, m - n) \in {[(n, m)]}_{E_2}$.
\end{enumerate}
אנחנו רואים כי לכל $l \in \NN \setminus \{0\}$ מתאים למחלקות שקילות של ${[(l, 0)]}_{E_2}, {[(0, l)]}_{E_2}$.

\subsubsection{שאלה מנחה}
בהינתן פעולה או יחס על קבוצה $X$, ויחס שקילות $E$ מתי ניתן להגדיר פעולה או יחס מושרית על קבוצת מחלקות השקילות?
\[
	X / E = \{ {[x]}_E \mid x \in X\}
\]
תהי $*$ פעולה על זוגות איברי $X$, דהינו $\forall x_1, x_2 \in X \implies x_1 + x_2 \in X$.\\*
הרעיון, בהינתן מחלקות שקילות $C_1, C_2 \in X/E$ נבקש להגדיר $C_1 * C_2 \in X/E$ נגדיר באופן הבא: \\*
נבחר נציג $x_1 \in C_1$ ו־$x_2 \in C_2$ וננסה להגדיר $C_1 * C_2 = {[x_1 + x_2]}_E$. \\*
הקושי הוא שכדי לקבל פעולה מוגדרת היטב על מחלקות יש לבדוק כי ההגדרה איננה תלויה בבחירת נציגים.
כלומר לכל $x_1, x_1' \in C_1, x_2, x_2' \in C_2$ יתקיים ${[x_1 + x_2]}_E = {[x_1' + x_2']}_E$ ובמקרה כזה נאמר כי הפעולה $*$ על $X$ מוגדרת היטב על קבוצת המנה $X/E$.

\section{שיעור 4 --- 29.5.2024}
\subsection{מושג העוצמה}
\subsubsection{תזכורת}
בהינתן יחס שקילות $E$ על קבוצה $X$ נסמן $X/E = \{ {[x]}_E \mid x \in X\}$ קבומצ מחלקות השקילות. \\*
בהינתן פעולה (יחס) $*$ על $X$. \\*
$*$ משרה פעולה מוגדרת היטב על $X/E$ אם מתקיימת התכונה הבאה:
\[
	\forall (x_1, x_2) \in E, \forall (y_1, y_2) \in E : (x_1 * y_1, x_2 * y_2) \in E
\]
אי־תלות בנציגים $E$ ביחס לפעולה $*$.

נגדיר את $*$ על $X/E$ על־ידי
\[
	{[x_1]}_E * {[y_1]}_E = {[x_1 * y_1]}_E
\]

נתבונן ביחס
\[
	E = \{ (A, B) \mid \exists f : A \to B \text{ הפיכה} \}
\]
ראינו כי $E$:

\begin{itemize}
	\item רפלקסיבי
	\item סימטרי
	\item טרנזיטיבי
\end{itemize}
ולכן $E$ מקיים את התכונות של יחס שקילות.

\subsubsection{הגדרה (זמנית): עוצמה}
עוצמה היא מחלקת שקילות לפי $E$. \\*
נסמן ב־$|A|$ את מחלקת השקילות של $A$.
סימונים מקובלים נוספים:
\begin{itemize}
	\item $|\NN| = \aleph_0$
	\item $|\RR| = \aleph$ או גם מסמנים ב־$\mathfrak{C}$, מה שנקרא C גותית.
	\item באופן כללי משתמשים באותיות גותיות כדי לסמן את העוצמות של קבוצות, לדוגמה $|A| = \mathfrak{a}, |B| = \mathfrak{b}$.
	\item לקבוצה סופית $[n]$ נסמן גם $|[n]| = n$.
\end{itemize}

\subsubsection{דוגמות}
\begin{enumerate}
	\item $\{1, 2, 3\} \in |[3]|$, $\{\pi, e, \frac{1}{7}\}$.
	\item $|\ZZ| = |\QQ| = \aleph_0$.
	\item באופן דומה נקבל גם $\mathfrak{C} = |\RR| = |\mathbb{C}| = |\RR\backslash a| = |[0, 1]|$.
	\item $0 = |\emptyset|$.
\end{enumerate}

\subsection{פעולות חשבון על עוצמות}
נבקש להגדיר לכל זוג עוצמות $\mathfrak{a}, \mathfrak{b}$ עוצמות נוספות. \\*
חיבור $\mathfrak{a} + \mathfrak{b}$, כפל $\mathfrak{a} \cdot \mathfrak{b}$ וחזקה $\mathfrak{a}^\mathfrak{b}$.

\subsubsection{כפל}
נתבונן בפעולת המכפלה הקרטזית $\times$ על קבוצה $A \times B$. \\*
נרצה להראות שהיא מגדירה פעולה מוגדרת היטב למחלקות עוצמה.
\begin{proof}
	צריך להוכיח כי בהינתן $A_1, A_2$ שוות עוצמה ו־$B_1, B_2$ שוות עוצמה, אז שמתקיים $|A_1 \times B_1| = |A_2 \times B_2|$.
	נבחר $f : A_1 \to A_2$ הפיכה וגם $g : A_2 \to B_2$ הפיכה שאנו יודעים שקיימות ונבחן את
	\[
		(f \times g) : A_1 \times B_1 \to A_2 \times B_2
	\]
	המוגדרת על־ידי
	\[
		(f \times g)(a, b) = (f(a), g(b))
	\]
	ולכן $f \times g$ היא חד־חד ערכית ועל $A_2 \times B_2$ שכן $f, g$ הן חד־חד ערכיות ועל בנפרד. \\*
	מסקנה: $|A_1 \times B_1| = |A_2 \times B_2|$.
\end{proof}

\subsubsection{הגדרה: כפל עוצמות}
בהינתן עוצמות $\mathfrak{a}, \mathfrak{b}$ נגדיר $\mathfrak{a} \cdot \mathfrak{b}$ באופן הבא: \\*
תהינה $A, B$ קבוצות, $|A| = \mathfrak{a}, |B| = \mathfrak{b}$ אז נגדיר $\mathfrak{a} \cdot \mathfrak{b} = |A \times B|$.

\subsubsection{דוגמה}
\begin{enumerate}
	\item לכל $n, m$ סופיים נראה כי
		\[
			|[n]| \cdot |[m]| = |[n] \times [m]| = |[n \cdot m]|
		\]
	\item $\aleph_0 \cdot \aleph_0 = | \NN \times \NN| = |\NN| = \aleph_0$.
	\item באינדוקציה על $1 \le k$ נגדיר $\overbrace{\aleph_0 \cdot \cdots \aleph_0}^k = \aleph_0$
\end{enumerate}

\subsubsection{פעולת החזקה}
בהינתן קבוצות $A$ ו־$B$ נתבונן בקבוצה $A^B = \{ f : B \to A \}$. \\*
נבקש ללבדוק כי הפעולה הזו לא תלויה בבחירת נציגים ולכן מוגדרת היטב.
\begin{proof}
	צריך להוכיח:
	אם $|A_1| = |A_2|, |B_1| = |B_2|$ אז נראה כי $|A_1^{B_1}| = |A_2^{B_2}|$. \\*
	דהינו
	\[
		|\{ f : B_1 \to A_1 \}| = |\{ f : B_2 \to A_2 \}|
	\]
	נקבע פונקציות הפיכות $f : A_1 \to A_2$ ו־$g : B_1 \to B_2$ הפיכות. \\*
	נגדיר $\varphi : A_1^{B_1} \to A_2^{B_2}$ על־ידי
	\[
		h_1 : B_1 \to A_1,
		\varphi(h_1) = f \circ h_1 \circ g^{-1} : B_2 \to A_2
	\]
	נרצה להראות כי $\varphi$ הפיכה על־ידי מציגת פונקציה הופכית:
	\[
		\psi(h_2) : A_2^{B_2} \to A_1^{B_1},
		\qquad
		\psi(h_2) = f^{-1} \circ h_2 \circ g
	\]
	נבדוק את הרכבת הפונקציות:
	\begin{align*}
		& \psi \circ \varphi = id_{A_1^{B_1}} \\
		& \varphi \circ \psi = id_{A_2^{B_2}}
	\end{align*}
	ונסיק מהקריטריון השקול להפיכות כי $\varphi$ הפיכה ומתקיים $\varphi^{-1} = \psi$. \\*
	נבדוק את ההרכבה הראשונה:
	\[
		\forall h_1 \in A_1^{B_1}, (\psi \circ \varphi)(h_1)
		= \psi(f \circ h_1 \circ g^{-1})
		= f^{-1} \circ f \circ h_1 \circ g^{-1} \circ g
		= (f^{-1} \circ f) \circ h_1 \circ (g^{-1} \circ g)
		= (id_{A_1}) \circ h_1 \circ (id_{B_1})
		= h_1
	\]
	והצד השני דומה.
\end{proof}

\subsubsection{הגדרה: פעולת חזקה על עוצמות}
בהינתן עוצמות $\mathfrak{a}, \mathfrak{b}$ נגדיר עוצמה $\mathfrak{a}^\mathfrak{b}$ באופן הבא: \\*
נבחר קבוצות המקיימות $|A| = \mathfrak{a}, |B| = \mathfrak{b}$ ונגדיר $\mathfrak{a}^\mathfrak{b} = |A^B|$.

\subsubsection{דוגמות}
\begin{enumerate}
	\item יהיו $n, m$ סופיים, מתקיים $|{[n]}^{[m]}| = |[n^m]|$
	\item תהי $\mathfrak{a} = |A|$ ו־$\mathfrak{b} = 0 = |\emptyset|$.
		נשים לב כי $f : \emptyset \to A$ היא פונקציה שכן $\emptyset \subseteq \emptyset \times A$ הוא יחס באופן ריק. הפונקציה היא אכן פונקציה באופן ריק. \\*
		נסיק מכך כי מתקיים $A^\emptyset = \{ \emptyset \}$ ולכן $\mathfrak{a}^\mathfrak{b} = |\{ \emptyset \} = 1$.
\end{enumerate}

\subsubsection{טענה: }
לכל קבוצה $A$ מתקיים $2^{|A|} = |\mathcal{P}(A)|$.
\begin{proof}
	על־פי ההגדרה העוצמה $2^{|A|}$ שווה למחלקת העוצמה של ${\{0, 1\}}^A = \{ h : A \to \{0, 1\}\}$. \\*
	לכן כדי להוכיח את הטענה די להראות קיום פונקציה הפיכה בין ${\{0, 1\}}^A$ לבין $\mathcal{P}(A)$. \\*
	נגדיר $\varphi : {\{0, 1\}}^A \to \mathcal{P}(A)$ על־ידי
	\[
		\varphi(h) = h^{-1}(\{1\}) = \{ a \in A \mid h(a) = 1\}
	\]
	נוכיח כי $\varphi$ חד־חד ערכית ועל $\mathcal{P}(A)$:
	\[
		\forall h_1, h_2 : A \to \{0, 1\}, h_1 \ne h_2,
		\exists a \in A : h_1(a) \ne h_2(a)
		\implies
		a \in h_1^{-1}(\{1\}) \triangle h_2^{-1}(\{1\})
	\]
	בפרט הקבוצות $\varphi(h_1), \varphi(h_2)$ הן שונות. \\*
	נוכיח על: יהי $B \subseteq A$, דהינו $B \in \mathcal{P}(A)$. ניקח
	\[
		l_B : A \to \{0, 1\}, l_B(a) = \begin{cases}
			0, & a \notin B \\
			1, & a \in B
		\end{cases}
	\]
	נובע מהגדרת $\varphi$ ש־$\varphi(l_b) = B$ והיא על. \\*
	הראינו כי קיימת פונקציה הפיכה $\varphi : {\{0, 1\}}^A \to \mathcal{P}(A)$ ולכן מתקיים $2^{|A|} = |\mathcal{P}(A)|$.
\end{proof}

\subsubsection{מסקנה}
נובע ממשפט קנטור כי לכל עוצמה $\mathfrak{a}$ מתקיים $\mathfrak{a} < 2^\mathfrak{a}$.

\subsubsection{טענה: שקילות חיבור עוצמות}
יהיו $|A_1| = |A_2|, |B_1| = |B_2|$, אם $\emptyset = A_1 \cap B_1$ וגם $\emptyset = A_2 \cap B_2$. \\*
אז $|A_1 \cup B_1| = |A_2 \cup B_2|$. \\*
הוכחה בתרגיל.

\subsubsection{הגדרה: חיבור עוצמות}
תהינה עוצמות $\mathfrak{a}, \mathfrak{b}$ אז נגדיר את $\mathfrak{a} + \mathfrak{b}$ באופן הבא: \\*
ניקח קבוצות זרות $A, B$ כך ש־$|A| = \mathfrak{a}, |B| = \mathfrak{b}$. \\*
נגדיר את $\mathfrak{a} + \mathfrak{b}$ להיות העוצמה $A \cup B$.

\subsubsection{הגדרה שקולה}
הערה: לכל קבוצה $A$ מתקיים $|\{0\} \times A| = |A| = |\{1\} \times A|$. \\*
לכל זוג $A, B$ קבוצות נראה כי
\[
	\emptyset = (\{ 0 \} \times A) \cap (\{ 1 \} \times B)
\]
ולכן נגדיר
\[
	|A| + |B| = |(\{ 0 \} \times A) \cup (\{ 1 \} \times B)|
\]

\subsubsection{הגדרה: אי־שוויון בין עוצמות}
בהינתן עוצמות $\mathfrak{a}, \mathfrak{b}$ נגדיר כי $\mathfrak{a} \le \mathfrak{b}$ ($\mathfrak{a} < \mathfrak{b}$) \\*
אם יש נציגים $|A| = \mathfrak{a}, |B| = \mathfrak{b}$ כך שקיימת פונקציה חד־חד ערכית $f : A \to B$ (וגם לא קיימת $f$ כזו שהיא גם על).

נבחין כי ההגדרה איננה תלויה בבחירת נציגים $A, B$.

\subsubsection{הערה: ניסוח שקול למשפט קנטור־שרדר ברנשטיין}
ניסוח שקול למשפט הוא שלכל $\mathfrak{a}, \mathfrak{b}$ עוצמות אם גם $\mathfrak{a} \le \mathfrak{b}$ וגם $\mathfrak{b} \le \mathfrak{a}$ אז $\mathfrak{a} = \mathfrak{b}$.

\subsubsection{משפט: כללי חשבון בסיסיים}
\begin{enumerate}
	\item לכל $\mathfrak{a}, \mathfrak{b}$ עוצמות מתקיים $\mathfrak{a} + \mathfrak{b} = \mathfrak{b} + \mathfrak{a}$ וגם $\mathfrak{a} \cdot \mathfrak{b} = \mathfrak{b} \cdot \mathfrak{a}$.
	\item לכל שלוש עוצמות $\mathfrak{a}, \mathfrak{b}_1, \mathfrak{b}_2$ מתקיים $\mathfrak{a} \cdot \mathfrak{b}_1 + \mathfrak{a} \cdot \mathfrak{b}_2 = \mathfrak{a}(\mathfrak{b}_1 + \mathfrak{b}_2)$
	\item לכל שלוש עוצמות $\mathfrak{a}, \mathfrak{b}_1, \mathfrak{b}_2$ מתקיים $\mathfrak{a}^{\mathfrak{b}_1 + \mathfrak{b}_2} = \mathfrak{a}^{\mathfrak{b}_1} \cdot \mathfrak{a}^{\mathfrak{b}_2}$
		וגם ${{(\mathfrak{a}^{\mathfrak{b}_1})}^{\mathfrak{b}_2}} = \mathfrak{a}^{\mathfrak{b}_1 \mathfrak{b}_2}$.
\end{enumerate}

\section{שיעור 5 --- 5.6.2024}
\subsection{עוצמת המנייה ועוצמת הרצף}
\begin{theorem}
	$2^{\aleph_0} = \mathfrak{C} = |(0, 1)|$
\end{theorem}
\begin{proof}
	נשתמש במשפט CSB, תחילה נראה את $2^{\aleph_0} \le \mathfrak{C}$. \\*
	ניקח נציגים, עבור $2^{\aleph_0}$ נבחר $\{ f : \NN \{0, 1\} \} = {(\{0, 1\})}^\NN$, ו־$(0, 1)$ עבור $\mathfrak{C}$. \\*
		נגדיר פונקציה $G : {\{0, 1\}}^\NN \to (0, 1)$ על־ידי
		\[
			\forall f : \NN \to \{0, 1\},
			\qquad
			G(f) = 0.1f(0) f(1) f(2) \dots f(n)\dots
		\]
		בפיתוח עשרוני. יחידות הפיתוח של המספרים המדוברים מבטיחה כי $G$ היא חד־חד ערכית. \\*
		קיבלנו כי $2^{\aleph_0} = |{\{0, 1\}}^\NN| \le |(0, 1)| = \mathfrak{C}$.

		עתה נוכיח את אי־השוויון לכיוון השני $\mathfrak{C} \le 2^{\aleph_0}$. \\*
		נשים לב כי $2^{\aleph_0} \le 10^{\aleph_0} \le 16^{\aleph_0} = 2^{4 \cdot \aleph_0} = 2^{\aleph_0}$
		ומכאן נסיק כי $2^{\aleph_0} \le 10^{\aleph_0} \le 2^{\aleph_0}$ ומ־CSB נקבל $2^{\aleph_0} = 10^{\aleph_0}$
		ולכן מספיק להוכיח כי $\mathfrak{C} \le 10^{\aleph_0}$.
		נבחר נציגים $(0, 1)$ עבור $\mathfrak{C}$ ואת ${\{0, \dots, 9\}}^\NN$ עבור $10^{\aleph_0}$. \\*
		נגדיר פונקציה $H : (0, 1) \to {\{0, \dots, 9\}}^\NN$ באופן הבא: \\*
		לכל $x \in (0, 1)$ ותהי $0.x_0 x_1 \dots x_n\dots$ ההצגה הסטנדרטית היחידה של $x$ ונגדיר $H(x) : \NN \to \{0, \dots, 9\}$ על־ידי $H(x)(n) = x_n$. \\*
		יחידות ההצגה הסטנדרטית של המספר מבטיחה כי $H$ היא חד־חד ערכית. \\*
		$H$ מעידה על כך ש־$\mathfrak{C} \le 10^{\aleph_0} = 2^{\aleph_0}$ כמבוקש.
\end{proof}
נעבור עתה להוכיח את משפט קנטור־שרדר־ברנשטיין
\begin{theorem}[משפט קנטור־שרדר־ברנשטיין]
	לכל זוג קבוצות $A, B$ אם קיימות פונקציות חד־חד ערכיות $f : A \to B, g : B \to A$ אז קיימת פונקציה הפיכה $h : A \to B$.
\end{theorem}
\begin{proof}
	נוכיח בשני שלבים. \\*
	שלב ראשון (רדוקציה):
	המשפט שקול לטענה הבאה: לכל זוג קבוצות $B \subseteq A$ אם קיימת פונקציה $f : A \to B$ חד־חד ערכית אז קיימת פונקציה $h : A \to B$ הפיכה.

	הוכחת הטענה: \\*
	תהינה $A, B$ אשר מקיימות את משפט CSB ולכן קיימות $f : A \to B, g : B \to A$ חד־חד ערכיות.
	נגדיר $B^* = g[B] \subseteq A$. נשים לב כי על־פי הגדרת $B^*$ נוכל לבחון את $g : B \to B^*$ וזו פונקציה חד־חד ערכית וכמובן גם על ולכן הפיכה. \\*
	נגדיר $f^* = g \circ f : A \to B^*$ ולכן $f^*$ חד־חד ערכית כהרכבה של $f, g$ חד־חד ערכיות. \\*
	נקבל כי $A, B^*$ מקיימות את ההנחות בנוסח שמופיע בטענה.
	לכן כמסקנה מהנוסח החלופי יש פונקציה $h^* : A \to B^*$ חד־חד ערכית ועל ונגדיר $h = g^{-1} \circ h^*$

	שלב ב', נוכיח את הניסוח השקול שמופיע בשלב א'. \\*
	נניח $B \subseteq A$ ותהי פונקציה חד־חד ערכית $f : A \to B$.

	יהיו $B \subseteq A$ ונתונה פונקציה חד־חד ערכית $f : A \to B$ ונגדיר
	\[
		A^* = \{ x \in A \mid \exists n \in \NN \exists a \in A \setminus B : x = f^n(a) \}
	\]
	מתקיים
	\begin{enumerate}
		\item $A \setminus B \subseteq A^*$.
		\item לכל $x \in A^*$ גם $f(x) \in A^*$
	\end{enumerate}
	נגדיר $h : A \to B$ על־ידי
	\[
		h(x) = \begin{cases}
			x & x \in A \setminus A^* \\
			f(x) & x \in A^*
		\end{cases}
	\]
	$h$ מוגדרת היטב שכן לכל $x \in A^*$ גם $h(x) = f(x) \in B$ כי $f : A \to B$ ולכל $x \in A \setminus A^*$ מתקיים $h(x) = x \in B$ שכן $A \setminus \subseteq A^*$. \\*
	נוכיח כי $h$ חד־חד ערכית ונפריד למקרים. \\*
	אם $x_1, x_2 \in A^*$ אז $h(x_1) = f(x_1) \ne f(x_2) = h(x_2)$. \\*
	אם $x_1, x_2 \in A \setminus A^*$ אז $h(x_1) = x_1 \ne x_2 = h(x_2)$. \\*
	אם $a_1 \in A^*$ ו־$a_2 \in A \setminus A^*$ ללא הגבלת הכלליות אז $h(x_1) = f(x_1) \in A^*, h(x_2) = x_2 \notin A^*$ ולכן כמובן $h(x_1) \ne h(x_2)$. \\*
	נוכיח כי $h$ על $B$.
	יהי $b \in B$, ונפריד למקרים. \\*
	אם $b \in A\setminus A^*$ אז $h(b) = b$ ואם $b \in A^*$ אז ניקח $a \in A \setminus B$ ו־$n \in \NN\setminus\{0\}$ כך ש־$b = f^n(a)$. \\*
	יהי $b' = f^{n - 1}(a)$ ונסיק כי $h(b') = f(b') = f(f^{n - 1}(a)) = b$.
\end{proof}

\subsection{מבוא לתורת הקבוצות האקסיומתית}
נבחן מספר שאלות פתוחות שיש לנו:
\begin{enumerate}
	\item יהיו $A, B$ קבוצות כך שקיימת $g : B \twoheadrightarrow A$, האם בהכרח $|A| \le |B|$? \\*
		ננסה לענות: נגדיר $a \in A$ ונגדיר $B_a = \{ b \in B \mid g(b) = a \} \ne \emptyset$. \\*
		הבחנה: אם קיימת פונקציה $f : A \to B$ כך שלכל $a \in A$ מתקבל $f(a) \in B_a$ אז $f$ חד־חד ערכית כמבוקש ו־$|A| \le |B|$.
\end{enumerate}
\begin{definition}[אקסיומת הבחירה]
	לכל סדרת קבוצות לא ריקות $\langle B_i \mid i \in I \rangle$ כאשר $I$ קבוצת אינדקסים קיימת פונקציה (פונקציית בחירה) $f : I \to \bigcup_{i \in I} B_i$, כלומר
	\[
		\forall i \in I : f(i) \in B_i
	\]
\end{definition}
תרגיל:
הראו כי בהינתן אקסיומת הבחירה כי לכל $A, B$ אם יש $g : B \to A$ על $A$ אז $|A| \le |B|$.

\section{שיעור 6 --- 19.5.2024}
\subsection{הגישה האקסיומטית לתורת הקבוצות}
\begin{definition}[כללי יסוד]
	\begin{enumerate}
		\item כל האובייקטים המתמטיים הם קבוצות.
		\item כל הקבוצות מתוארות בשפה היסודית הכוללת סימנים של $=, \in$ וגם סימנים לוגיים סטנדרטים כמו סוגריים, כמתים, קשרים ומשתנים וכן הלאה.
	\end{enumerate}
\end{definition}
\begin{example}
	המספרים הטבעיים הם קבוצות, שכן הם אובייקטים מתמטיים, נגדיר
	\[
		0 = \emptyset, 1 = \{ \emptyset \}, n + 1 = n \cup \{ n \}
	\]
\end{example}
\begin{example}
	זוג סדור הוא קבוצה:
	\[
		\langle x, y \rangle = \{ \{x\}, \{x, y \}\}
	\]
\end{example}
\begin{example}
		נתאר את הקבוצה הריקה
		\[
			\varphi_0(x) = \forall y : y \notin x
		\]
		תכונה זו משמעותה היא שעבור $x$ לכל קבוצה $y$ אז $y \notin x$.
\end{example}
\begin{example}
	הכלה בין קבוצות $x \subseteq y$ תיכתב על־ידי
	\[
		\forall z : z \in x \implies z \in y
	\]
\end{example}
\begin{example}
	כיצד נבטא את הביטוי $x = 1$ בשפה פורמלית? \\*
	אנו יודעים כי $1 = \{ \emptyset \}$ ובשפה הפורמלית
	\[
		\varphi_1(x) = \exists y : \varphi_0(y) \land y \in x \land \forall z (z \in x \implies z = y)
	\]
	את הביטוי $x = 2$ נוכל לכתוב על־ידי
	\[
		\varphi_2(x) = \forall z ( z \in x \implies (\varphi_0(z) \lor \varphi_1(z)) \land \exists x_0 (x_0 \in x \land \varphi_0(x_0)) \land \exists x_1 (x_1 \in x \land \varphi_1(x_1)))
	\]
\end{example}
\begin{definition}[תכונה]
	תכונה $p(x)$ של קבוצות היא כזו הניתנת לתיאור בשפה הפורמלית.
\end{definition}
\begin{example}
	$x \subseteq y$ היא אכן תכונה.
\end{example}
\begin{remark}
	נרשה לעשות שימוש בקבוצות $a$ כחלק מתיאור של תכונה.
\end{remark}
\begin{example}
	$a = 2$. התכונה $p(x) = x \subseteq 2$ ניתנת לתיאור בשפה הפורמלית.
\end{example}
עתה נשאל את עצמנו, מהן קבוצות? \\*
הניסיון הנאיבי הוא לכל $p(x)$ יש קבוצה $\{ x \mid p(x) \}$ (קבוצת האובייקטים שמקיימים $p$).

ניתקל כך בבעיה, ראינו כי $p(x) := x \notin x$ הוא תכונה אבל לא תיתכן קבוצה $\{ x \mid x \notin x \}$.

הגישה שלנו היא לרום רשימת אקסיומות ZFC = ZF + AC, שתכולתן לתת לנו תיאור של הקבוצות הקיימות.
\begin{remark}
	באופן כללי, לאוסף מהצורה
	\[
		\{ x \mid p(x) \}
	\]
	כאשר $p(x)$ תכונה כלשהי, יקרא מחלקה.
\end{remark}
\begin{remark}
	כל קבוצה $A$ היא מחלקה, שכן ניתן לרשום
	\[
		A = \{ x \mid x \in A \}
	\]
\end{remark}
מחלקה שאינה קבוצה תיקרא מחלקה נאותה.
\begin{example}
	\[
		\{ x \mid x \notin x\}
	\]
	היא מחלקה נאותה על־פי משפט ראסל ו־ZF. % chktex 13
\end{example}
\begin{definition}[המערכת ZF]
	\begin{enumerate}
		\item אקסיומת ההיקפיות:
			שתי קבוצות שוות אם ורק אם יש בהן אותם איברים,
			\[
				\forall x \forall y ( x = y \iff (\forall z, z \in x \iff x \in y))
			\]
		\item אקסיומת הקבוצה הריקה:
			קיימת קבוצה ריקה
			\[
				\exists x\ \varphi_0(x)
			\]
		\item אקסיומת הזוג הלא סדור:
			לכל קבוצות $x$ ו־$y$ קיימת $\{x, y\}$,
			\[
				\forall x \forall y \ \exists z [ x \in z \land y \in z \land \forall w (w \in z \implies w = x \lor w = y)]
			\]
			הערה: נסיק קיום יחידונים לכל $x$ על־ידי $\{ x, x \} = \{ x \}$.
		\item אקסיומת האיחוד:
			לכל קבוצה $x$ קיימת הקבוצה $\cup x$,
			\[
				\forall x \exists y \ \forall w [w \in y \iff \exists z (z \in x \land w \in z)]
			\]
			הערה: נסיק כי לכל זוג קבוצות $a, b$ קיימת הקבוצה $a \cup b$, שכן מאקסיומת הזוגות קיימת הקבוצה $x = \{ a, b \}$ ומאקסיומת האיחוד קיימת קבוצה $a \cup b = \cup x$.
		\item אקסיומת קבוצת החזקה:
			לכל קבוצה $x$ קיימת קבוצת חזקה $\mathcal{P}(x)$ (קבוצה שאיבריה הם תתי־הקבוצות של $x$),
			\[
				\forall x\exists y \forall z (z \in y \iff z \subseteq x)
			\]
		\item אקסיומת הסדירות (ביסוס היטב):
			אם $x$ קבוצה לא ריקה אז קיים איבר $y \in x$ שהוא מזערי ביחס $\in$ (כלומר אין $z \in x$ כך ש־$z \in y$),
			\[
				\forall x [ \lnot \varphi_0(x) \implies \exists y ( y \in x \land \forall z (z \in x \implies \lnot z \in y))]
			\]
			מסקנה: לא קיים $x$ כך ש־$x \in x$, שכן אחרת $x = \{ a \}$ סותרת את אקסיומת הסדירות. \\*
			מסקנה 2: לא יתכן $a, b$ קבוצות כך ש־$b \in a \land a \in b$, על־ידי $x = \{a, b\}$. \\*
			מסקנה 3: לא תיתכן סדרה אינסופית יורדת ביחס $\in$ של קבוצות, דהינו
			\[
				x_0 \ni x_1 \dots \ni x_n \ni \dots
			\]
			שכן $x = {\{ x_n \}}_{n \in \NN}$ לא מקיימת את אקסיומת הסדירות.
		\item אקסיומת האינסוף:
			הכנה: \\*
			סימון: לכל קבוצה $x$ נגדיר $s(x) = x \cup \{ x \}$ (מובטח קיום מתוך האקסיומות הקודמות). \\*
			הגדרה: קבוצה $I$ נקראת אינדוקטיבית אם $\emptyset \in I$, וגם $\forall x (x \in I \implies s(x) \in I)$. \\*
			האקסיומה משמעותה היא שיש קבוצה אינדוקטיבית,
			\[
				\exists y [ \emptyset \in y \land \forall x (x \in y \implies s(x) \in y)]
			\]
		\item אקסיומת (סכמת) הפרדה:
			לכל תכונה $p(x)$ וקבוצה $A$ קיימת קבוצה $\{ x \in A \mid p(x) \}$, \\*
			נניח כי $p(x)$ היא תכונה מוגדרת על־ידי נוסחה $\varphi(x, a_1, \dots, a_k)$, אז
			\[
				\forall x \forall a_1, \dots, a_k \exists y [\forall z \ z \in y \iff z \in x \land \varphi(z, a_1, \dots, a_k)]
			\]
			הערה: זו סכמה של אקסיומות במובן שלכל נוסחה (תכונה) כותבים אקסיומה עבורה.
		\item אקסיומת (סכמת) החלפה: \\*
			הכנה: \\*
			נאמר כי תכונה $p(x, y)$ מקיימת את תנאי הפונקציה אם לכל $x_1, y_1, x_2, y_2$ אם $p(x_1, y_1)$ וגם $p(x_2, y_2)$ וגם $x_1 = x_2$ אז $y_1 = y_2$. \\*
			הערה: אם $p(x, y)$ מקיימת את תנאי הפונקציה אז נאמר כי המחלקה $F = \{ \langle x, y \rangle \mid p(x, y) \}$ אז היא מחלקה המקיימת את תנאי הפונקציה. \\*
			האקסיומה קובעת כי לכל תכונה $p(x, y)$ המקיימת את תנאי הפונקציה ולכל קבוצה $A$ קיימת קבוצה שהיא
			\[
				F[A] = \{ y \mid \exists x \in A\ \overbrace{\langle x, y \rangle \in F}^{p(x, y)} \}
			\]
			תרגיל: לרשום את האקסיומה (סכמה) בשפה הפורמלית.
	\end{enumerate}
\end{definition}
תשובה לשאלה המקורית היא שקבוצה היא כל אוסף שהאקסיומות מוכיחות שהוא קבוצה.

\subsection{בניות ראשונות ב־ZF}
\begin{proposition}
	לכל קבוצות $a, b$ קיימת קבוצה $\{ \{a\}, \{a, b\}\}$ והיא הזוג הסדור של $a, b$, מסומן ב־$\langle a, b \rangle$ או $(a, b)$.
\end{proposition}
\begin{exercise}
	הראו כי לכל קבוצות $a_1, b_1, a_2, b_2$ אם $\langle a_1, b_1 \rangle = \langle a_2, b_2 \rangle$ אז $a_1 = a_2$ וגם $b_1 = b_2$.
\end{exercise}
\begin{proposition}
	לכל זוג קבוצות $A, B$ קיימת המכפלה הקרטזית
	\[
		A \times B = \{ \langle a, b \rangle \mid a \in A, b \in B \}
	\]
\end{proposition}
\begin{proof}
	יהיו $A, B$ קבוצות. הראינו כי קיימת הקבוצה $A \cup B$. \\*
	מאקסיומת חזקה קיימת קבוצת חזקה $\mathcal{P}(A \cup B)$. 
	נשים לב כי לכל $a \in A, b \in B$ מתקיים $\{a\}, \{b\} \in \mathcal{P}(A \cup B)$. \\*
	גם מאקסיומת החזקה קיימת גם $\mathcal{P}(\mathcal{P}(A \cup B))$, ולכן נשים לב כי לכל $a \in A, b \in B$ מתקיים
	\[
		\langle a, b \rangle = \{\{a\}, \{a, b\}\} \in \mathcal{P}(\mathcal{P}(A \cup B))
	\]
	נשתמש באקסיומת הפרדה עבור התכונה
	\[
		p(z) = \exists a \exists b (a \in A \land b \in B \land z = \langle a, b \rangle)
	\]
	מאקסיומת ההחלפה נסיק כי קיימת קבוצה
	\[
		\{ \langle a, b \rangle \mid a \in A, b \in B \} = \{ z \in \mathcal{P}(\mathcal{P}(A \cup B)) \mid p(z) \}
	\]
\end{proof}

\section{שיעור 7 --- 26.6.2024}
\subsection{התורה האקסיומתית}
תזכורת: לפי הגישה האקסיומטית כל אובייקט הוא קבוצה. ישנה רשימת אקסיומות ZF אשר קובעת כללים למהן הקבוצות. \\*
מחלקה היא אוסף של קבוצות הנתונה על־ידי $p(x)$ תכונה כלשהי. \\*
מחלקות נאותות הן מחלקות שאינן קבוצות. באופן פורמלי (בשפה הפורמלית) לא ניתן להשתמש במחלקות נאותות כקבוצות. \\*
אף־על־פי כן בהינתן מחלקות $A = \{ x \mid p_A(x) \}$ ו־$B = \{ x \mid P_B(x) \}$ הנקבעות על־ידי תכונות $P_A, P_B$ אנו רושמים לעיתים באופן לא פורמלי $x \in A$ או $A \subseteq B$. \\*
המשמעות של ביטויים אלה הם
\[
	P_A(x),
	\quad
	\forall x (P_A(x) \implies P_B(x))
\]
בהתאמה.

\begin{notation}
	$V = \{ x \mid x = x \}$ מחלקת כל הקבוצות.
\end{notation}
\begin{proposition}
	$V$ אינו קבוצה (היא מחלקה נאותה).
\end{proposition}
\begin{proof}
	נניח כי $V$ קבוצה ונתבונן בתכונה $p(x) = x \notin x$, מאקסיומת ההפרדה נסיק כי יש קבוצה
	\[
		\{ x \in V \mid x \notin x \} = \{ x \mid x \notin x \}
	\]
	בסתירה לטיעון הפרדוקס של ראסל.
\end{proof}
בסוף השיעור הקודם הוכחנו את הטענות
\begin{proposition}
	לכל זוג קבוצות $a, b$ קיימת הקבוצה
	\[
		\langle a, b \rangle = \{ \{a\}, \{a, b\} \}
	\]
\end{proposition}
\begin{proposition}
	לכל זוג קבוצות $X, Y$ קיימת קבוצת המכפלה הקרטזית
	\[
		X \times Y = \{ \langle a, b \rangle \mid a, X, b \in Y \}
	\]
\end{proposition}
\begin{proposition}
	לכל זוג קבוצות $X, Y$ קיימת קבוצת כל היחסים $R \subseteq X \times Y$.
\end{proposition}
\begin{proof}
	מהטענה הקודמת $X \times Y$ קיימת ונשים לב כי קבוצת כל היחסים בין $X$ ל־$Y$ היא $\mathcal{P}(X \times Y)$, ולכן הקיום נובע מאקסיומת החזקה.
\end{proof}
\begin{proposition}
	לכל זוג קבוצות $X, Y$ קיימת קבוצת כל הפונקציות $f : X \to Y$.
\end{proposition}
\begin{proof}
	כל פונקציה $f : X \to Y$ היא יחס $f \subseteq X \times Y$ שמקיים את תכונת הפונקציה $p_f(x)$ ומאקסיומת ההפרדה קיימת.
	\[
		\{ f \in \mathcal{P}(X \times Y) \mid p_f(f) \}
	\]
\end{proof}
\begin{proposition}
	בהינתן קבוצות $X, Y$ ותכונה $p(x, y)$ שמקיימת את תנאי הפונקציה וכך שלכל $x \in X$ קיים $y \in Y$ כך ש־$p(x, y)$ אז קיימת פונקציה $f : X \to Y$ המתארת את  $p$.
\end{proposition}
\begin{proposition}
	באופן דומה לטענה הקודמת לכל תכונה $p(x, y)$ המקיימת את תכונת יחס השקילות על קבוצה נתונה $X$ אז קיים יחס שקילות $E \subseteq X \times X$ שמתאים לתיאור של $p(x, y)$.
\end{proposition}
הוכחה כתרגיל.
\begin{proposition}
	לקבוצת קבוצות $X \ne \emptyset$ קיימת קבוצה
	\[
		\cap X = \{ z \mid \forall y \in X : z \in y \}
	\]
\end{proposition}
\begin{proof}
	מאקסיומת האיחוד קיימת $\cup X$ ונשים לב כי
	\[
		\{ z \mid \forall y \in X : z \notin y \} = \{ z \in \cup X \mid \forall y \in X : z \in y \}
	\]
	מאקסיומת ההפרדה קיימת הקבוצה
	\[
		\{ z \in \cup X \mid \forall y \in x : z \in y \}
	\]
	ולכן $\cap X$ קיים.
\end{proof}

\subsection{קבוצת הטבעיים}
הגדרנו לכל קבוצה $x$ את הקבוצה $s(x) = x \cup \{ x \}$, והוכחנו בשיעור הקודם מ־ZF שהיא אכן קיימת לכל $x$. \\*
נרצה להשתמש במושג $s(x)$ כדי להגדיר את המספרים הטבעיים, הרעיון הוא
\[
	0 = \emptyset,
	\quad
	1 = s(0) = \{ \emptyset \},
	\quad
	2 = s(1) = 1 \cup \{ 1 \} = \{ \emptyset, \{ \emptyset \} \},
	\quad
	\dots
\]
תזכורת: אקסיומת האינסוף מבטיחה קיום קבוצה אינדוקטיבית $I$, כך שמתקיים
\[
	\emptyset \in I,
	\qquad
	\forall x \in I, s(x) \in I
\]
\begin{proposition}
	קיימת קבוצה אינדוקטיבית מינימלית $I^*$ ביחס הכלה, כלומר $I^* \subseteq I$ לכל $I$ אינדוקטיבית.
\end{proposition}
\begin{proof}
	\textbf{שלב א':}
	נוכיח כי בהינתן $X \ne \emptyset$ קבוצה שאבריה הן קבוצות אינדוקטיביות אז גם $I_x = \cap X$ אינדוקטיבית. \\*
	$\emptyset \in \cap X$ כי לכל $I \in X$ $I$ היא אינדוקטיבית ולכן $\emptyset \in I$. \\*
	באופן דומה לכל $x \in \cap X$ נקבל $x \in I$ לכל $I \in X$ אינדוקטיבית. \\*
	לכן $s(x) \in I$ לכל $I \in X$ ולכן $s(x) \in \cap X$.

	\textbf{שלב ב':}
	ניקח קבוצה אינדוקטיבית כלשהי $I_0$, מאקסיומת החזקה והפרדה נסיק שקיימת $X = \{ I \subseteq I_0 \mid I \text{ אינדוקטיבית} \}$. \\*
	$X \ne \emptyset$ שכן $I_0 \in X$. משלב א' נקבל $I^* = \cap X$ היא אינדוקטיבית.

	\textbf{שלב ג':}
	נטען כי $I^*$ אינדוקטיבית מינימלית ב־$\subseteq$. \\*
	תהי $J$ קבוצה אינדוקטיבית אז $J \cap I_0$ אינדוקטיבית ותת־קבוצה של $I_0$ ולכן שייכת ל־$X$.
	נסיק כי $I^* = \cap X \subseteq J \cap I_0 \subseteq J$,
	כלומר $I^* \subseteq J$ כמבוקש.
\end{proof}
\begin{definition}[קבוצת הטבעיים]
	קבוצת הטבעיים $\NN$ היא הקבוצה האינדוקטיבית המינימלית ($I^*$ מהטענה האחרונה).
\end{definition}
\begin{conclusion}
	מספר טבעי היא קבוצה $n \in \NN$.
\end{conclusion}
\begin{example}
	$0 = \emptyset \in \NN$.
	נקבל גם $1 = s(0) \in \NN$.
\end{example}
\begin{theorem}[אינדוקציה על הטבעיים]
	תהי $p(x)$ תכונה. אם $p(0)$ מתקיים וגם לכל $n \in \NN$ מתקיים $p(n) \implies p(n + 1)$ אז $p(n)$ מתקיים לכל $n \in \NN$.
\end{theorem}
\begin{proof}
	בהינתן תכונה $p$ נגדיר $I_p$ קבוצת הטבעיים המקיימים את $p$. \\*
	נשים לב כי מההנחות על $p$ מתקיים $0 = \emptyset \in I_p$ ולכל $n \in I_p$ גם $I_p \ni s(n)$. \\*
	נסיק כי $\NN \supseteq I_p$ היא קבוצה אינדוקטיבית. \\*
	מכיוון ש־$\NN$ היא אינדוקטיבית מינימלית ב־$\subseteq$ נסיק $I_p = \NN$ ולכן $\forall n \in \NN : p(n)$.
\end{proof}
\begin{definition}[יחס הסדר על הטבעיים]
	נגדיר עבור קבוצות $n, m \in \NN$ את $n < m$ אם ורק אם $n \in m$. \\*
	נבקש להראות כי $<$ הוא יחס סדר על הטבעיים וכי הוא סדר טוב.
\end{definition}
נוכיח מספר טענות למטרה זו.
\begin{proposition}
	לכל $n \in \NN$ מתקיים $n \subseteq \NN$.
\end{proposition}
\begin{proof}
	באינדוקציה על $n$ כמובן $0 = \emptyset \subseteq \NN$
	נניח כי $n \in \NN$ מקיים $n \subseteq \NN$ ונקבל $s(n) = n \cup \{ n \} \subseteq \NN$
\end{proof}
\begin{remark}
	בעבר סימנו לכל $n \in \NN$ את הקבוצה $[n] = \{ m \in \NN \mid m < n \}$. \\*
	נשים לב כי בפרשנות שלנו ל־$\NN$ נקבל $[n] = \{ m \in \NN \mid m \in n \} = n$.
\end{remark}
\begin{proposition}
	לכל $n, m \in \NN$ אם $n \in m$ אז $s(n) \in m$ או $s(n) = m$.
\end{proposition}
\begin{proof}
	באינדוקציה על התכונה $p(m)$ לפיה לכל $n \in m$ מתקיים $s(n) \in m$ או $s(n) = m$. \\*
	$p(0)$ נכון באופן ריק. \\*
	לצעד האינדוקציה נניח כי $p(m)$ ונבקש להראות את $p(s(m))$, כלומר כי לכל $n \in s(m)$ חייב להתקיים $s(n) \in s(m)$ או $s(n) = s(m)$. \\*
	ניזכר $s(m) = m \cup \{ m \}$.
	יהי $n \in s(m)$ ונחלק למקרים:
	\begin{itemize}
		\item אם $n \in m$ אז מהנחת האינדוקציה $s(n) = m$ או $s(n) \in m$ ובכל מקרה מכיוון ש־$m \subseteq s(m)$ נסיק כי $s(n) \in s(m)$.
		\item אם $n = m$ אז $s(n) = s(m)$.
	\end{itemize}
\end{proof}
\begin{proposition}
	לכל $m \in \NN$ מתקיים ש־$0 \in m$ או ש־$0 = m$.
\end{proposition}
מושאר כתרגיל.
\begin{proposition}
	לכל $n, m \in \NN$ מתקיים $n \in m$ או ש־$m = n$ או ש־$m \in n$.
\end{proposition}
\begin{proof}
	לכל $n \in \NN$ נתבונן בתכונה $p_n(m)$ שאומרת ש־$n \in m$ או ש־$n = m$ או ש־$m \in n$. \\*
	נקבע $n \in \NN$ ונוכיח באינדוקציה $p_n(m)$ מתקיים לכל $m$. \\*
	בסיס: $m = 0$ מתקיים מהטענה הקודמת. \\*
	נניח $p_m(n)$ ונוכיח $p_n(s(m))$, נפריד בין שלושה מקרים פשוטים:
	\begin{itemize}
		\item אם $n \in m$ אז $n \in s(m)$.
		\item אם $n = m$ אז $n \in s(m)$.
		\item אם $m \in n$ אז מטענה קודמת נקבל $s(m) \in n$ או ש־$s(m) = n$.
	\end{itemize}
\end{proof}

\section{שיעור 8 --- 3.7.2024}
\subsection{קבוצת הטבעיים --- המשך}
בשיעור הקודם הגדרנו את $\NN$ להיות הקבוצה האינדוקטיבית הקטנה ביותר. \\*
מספר טבעי $n \in \NN$ הוא קבוצה.
כבר הוכחנו כי לכל $n, m \in \NN$ מתקיים $n \in m$ או $n = m$ או $m \in n$.
נרצה להוכיח את המשפט הבא
\begin{theorem}
	היחס $\in$ על $\NN$ הוא יחס סדר טוב, כלומר
	\begin{enumerate}
		\item $\in$ יחס סדר חזק על $\NN$ (טרנזיטיבי ואנטי־סימטרי חזק).
		\item $\in$ יחס סדר קווי (לינארי) על $\NN$.
		\item $\in$ מבוסס היטב על $\NN$.
	\end{enumerate}
\end{theorem}
\begin{proposition}
	לכל $n \in \NN$ אם $m \in n$ אז $m \subseteq n$.
\end{proposition}
\begin{proof}
	באינדוקציה על $n$. \\*
	\textbf{בסיס:}
	הטענה נכונה באופן ריק.

	\textbf{צעד:}
	נניח שהטענה נכונה עבור $n$ ונוכיח עבור $s(n)$ (והחל מעכשיו נסמנו על־ידי $n + 1$). \\*
	ניזכר כי $s(n) = n \cup \{n\}$ ויהי $m \in s(n)$ אם $m \in n$ נקבל מהנחת האינדוקציה $m \subseteq s(n)$. \\*
	במקרה השני אם $m = n$ אז בבירור $m \subseteq s(n)$.
\end{proof}
\begin{proof}[הוכחת המשפט]
	\begin{enumerate}
		\item נתחיל בבדיקת יחס סדר. נוכיח טרנזיטיביות.
			נניח $n, m, k \in \NN$ כך ש־$m \in n, k \in m$ ומטענת העזר נקבל $m \in n \implies m \subseteq n$ ולכן $k \in n$. \\*
			נבדוק כי היחס הוא אנטי־סימטרי חזק. נרצה להוכיח שאין $n, m \in \NN$ כך ש־$n \in m \land m \in n$. \\*
			אחרת הקבוצה $x = \{n, m\}$ סותרת את אקסיומת הסדירות.
		\item $\in$ קווי, נובע מיידית מהטענה שהוכחנו בשבוע שעבר ומופיעה בתזכורת.
		\item נוכיח כי הוא מבוסס היטב. תהי $A \subseteq \NN$ לא ריקה. מאקסיומת הסדירות קיים $n \in A$ כך שלכל $m \in A$ מתקיים $\lnot(m \in n)$. \\*
			מכיוון ש־$\in$ קווי על $\NN$ אז המספר $n$ הזה הוא גם המספר המינימלי ב־$A$.
	\end{enumerate}
\end{proof}
\begin{definition}
	נגדיר את $<$ על $\NN$ להיות $\in$. \\*
	מהמשפט האחרון נסיק כי $<$ הוא סדר טוב.
\end{definition}

\subsection{רקורסיה על N}
\begin{theorem}[רקורסיה]
	תהי $A \ne \emptyset$ ו־$f : A \to A$ ו־$a \in A$. \\*
	קיימת פונקציה $g : \NN \to A$ עבורה מתקיים $g(0) = a, g(n + 1) = f(g(n))$.
\end{theorem}
\begin{proof}
	בתרגיל
\end{proof}
\begin{theorem}[רקורסיה בגרסת המחלקה]
	תהי $A$ מחלקה לא ריקה (המתוארת על־ידי תכונה $p_A(x)$) ו־$F$ מחלקה (המתוארת על־ידי תכונה $P_F(x, y)$) שאיבריה הם זוגות $\langle a, b \rangle$ כאשר $a, b \in A$,
	וגם $F$ מקיימת את תנאי הפונקציה על $A$, כלומר לכל קבוצה $a$ אם $P_A(a)$ אז קיימת ויחידה קבוצה $b$ עם $f_A(b)$ כך ש־$P_F(a, b)$. \\*
	לכל $a \in A$ קיימת פונקציה $g : \NN \to A$ כך ש־$g(0) = a, \forall n\ g(n + 1) = F(g(n))$, דהינו $P_F(g(n), F(g(n)))$.
\end{theorem}
באופן לא פורמלי: לכל $a \in A$ קיים ויחיד $b \in B$ כך ש־$\langle a, b \rangle \in F$. נרשום $F : A \to A$.

נעבור עתה לבחון את השימושים.
\begin{theorem}
	לכל קבוצה $X$ קיימת סדרה $\langle x^n \mid n \in \NN \setminus \{0\}\rangle$ החזקות הקרטזיות של $X$, \\*
	הממומשת על־ידי $g$ כאשר $dom(g) = \NN$ ו־$\forall n\ g(n) = X^{n + 1}$.
\end{theorem}
\begin{proof}
	נתבונן במחלקה $V$ המוגדרת על־ידי
	\[
		V = \{ x \mid x = x \}
	\]
	מחלקת כל הקבוצות ונתבונן בפונקציית המחלקה $F$ המוגדרת על־ידי
	\[
		\forall y \in V\ F(y) = X \times y
	\]
	פורמלית $F$ מתוארת על־ידי התכונה
	\[
		P_F(r, u) = u = X \times r
	\]
	ניקח את $V \ni a = X$ אז ממשפט הרקורסיה יש $g : \NN \to V$ כך ש־$g(0) = X$ ומתקיים
	\[
		g(n + 1) = F(g(n)) = X \times g(n)
	\]
	ולכן למעשה $g(0) = X, g(1) = X \times X, g(2) = X \times (X \times X)$ וכן הלאה.
\end{proof}
נתאר הגדרה פורמלית של פעולת החיבור על־ידי רקורסיה על הטבעיים.
\begin{theorem}[פונקציית חיבור הטבעיים]
	קיימת פונקציה $\text{add} : \NN \times \NN$ המקיימת $\text{add}(0, 0) = 0$
	ו־$\forall n, m\ \text{add}(n, m + 1) = \text{add}(n, m) + 1 = s(\text{add}(n, m))$
	וגם $\forall n, m\ \text{add}(n + 1, m) = \text{add}(n, m) + 1$.
\end{theorem}
\begin{proof}
	נוכיח קיום פונקציה כזו בשני שלבים
	בשלב הראשון נוכיח קיום סדרת פונקציות $\langle a_m \mid m \in \NN \rangle$ לכל $m$ $a_m : \NN \to \NN$
	כך שלכל $m$ יתקיים
	\[
		a_m(0) = m,
		\qquad
		a_m(n + 1) = a_m(n) + 1
	\]
	בשלב השני נגדיר $\text{add}(n, m)$ על־ידי $\text{add}(n, m) = a_n(m)$.

	נתחיל אם כן בשלב א', נעשה שימוש במשפט הרקורסיה. \\*
	תהי $A$ קבוצת כל הפונקציות $h : \NN \to \NN$ ו־$a = id_\NN = \{ \langle n, n \rangle \mid n \in \NN \}$.
	ניקח $F : A \to A$ פונקציה המוגדרת באופן הבא: בהינתן $A \ni h : \NN \to \NN$ נגדיר
	\[
		F(h)(m) = h(m) + 1,
		\qquad
		F(h) : \NN \to \NN
	\]
	ממשפט הרקורסיה קיימת $g : \NN \to A$ המקיימת
	\[
		g(0) = a = id_\NN,
		\qquad
		g(n + 1) = F(g(n))
	\]
	כלומר לכל $m$ נקבל $g(n + 1)(m) = g(n)(m) + 1$. \\*
	הסדרה $\langle a_n, n \in \NN \rangle$ ממומשת על־ידי $g$, $a_n = g(n)$. \\*
	צריך להוכיח כי הסדרה הזו מקיימת את תנאי המשפט, לכל $n$, $g(n)(0) = n$. \\*
	נראה גם כי $g(n)(k + 1) = g(n)(k) + 1$ מתכונת הרקורסיה שמקיימת את $g$.

	בשלב ב' נבדוק את התוצאה. \\*
	נגדיר $\text{add}(n, m) = a_n(m) = g(n)(m)$, וצריך להוכיח add מקיימת את שלוש התכונות. \\*
	נראה כי $\text{add}(0, 0) = g(0, 0) = id_\NN(0) = 0$. \\*
	נראה גם כי $\text{add}(n + 1, m) = g(n)(m) + 1$ כתכונה של $g$ וערכה. \\*
	ונשאר להוכיח את הטענה כי $g(n)(k + 1) = g(n)(k) + 1$.
\end{proof}
\begin{proof}[הוכחת הטענה]
	נוכיח באינדוקציה על $n$ שכן לכל $k$ נקבל $g(n)(k + 1) = g(n)(k) + 1$. \\*
	עבור $n = 0$ קיבלנו כי $g(0) = id_\NN$ ונקבל $g(0)(k + 1) = id(k + 1) = k + 1$.

	\textbf{צעד:}
	נניח כי עבור $n$ הטענה נכונה ונבדוק עבור $s(n) = n + 1$, נקבל
	\[
		g(n + 1)(k + 1) = g(n)(k + 1) + 1 = g(n)(k) + 1 + 1 = (g(n + 1)(k)) + 1 = g(n + 1)(k) + 1
	\]

	נותר להוכיח כי לכל $n$ מתקיים $g(n)(0) = n$ באינדוקציה על $n$ (תרגיל).
\end{proof}
\begin{exercise}
	הוכיחו קיום פונקציה $\text{mul} : \NN \times \NN \to \NN$ המממשרת את פעולת הכפל בטבעיים.
\end{exercise}

\subsection{השוואת סדרים טובים וסודרים}
\begin{definition}[פונקציה שומרת סדר]
	יהיו $(X, \le_X), (Y, \le_Y)$ סדרים חלקיים. \\*
	פונקציה $f : X \to Y$ היא שומרת סדר אם
	\[
		\forall x_1, x_2 \in X : x_1 \le_X x_2 \implies f(x_1) \le_Y f(x_2)
	\]
\end{definition}
\begin{definition}[שיכון]
	נאמר כי $f : X \to Y$ היא שיכון בין הסדרים אם היא חד־חד ערכית וגם
	\[
		\forall x_1, x_2 \in X : x_1 \le_X x_2 \iff f(x_1) \le_Y f(x_2)
	\]
\end{definition}
\begin{definition}[איזומורפיזם]
	נאמר כי $f : X \to Y$ היא איזומורפיזם אם היא שיכון של $X$ ב־$Y$ וגם על $Y$.
\end{definition}
\begin{definition}
	יהי $\langle X, \le_X \rangle$ סדר חלקי. \\*
	קבוצה $X^* \subseteq X$ היא תחילית (רישא) אם
	\[
		\forall x_1 \in X^* \forall x_2 \le_X x_1 : x_2 \in X^*
	\]
\end{definition}
\begin{proposition}
	אם $(X, \le_X)$ סדר טוב אז לכל תחילית $X^* \subseteq X$ חייב להתקיים או $X = X^*$ או שקיים $y \in X$ כך ש־$X^* = \{ x \in X \mid x <_X y \}$.
\end{proposition}
\begin{theorem}[משפט ההשוואה]
	יהיו $(X, \le_X), (Y, \le_Y)$ סדרים טובים, אז מתקיים אחד מהבאים
	\begin{enumerate}
		\item $(X, \le_X)$ היא איזומורפית לתחילית של $(Y, \le_Y)$.
		\item $(Y, \le_Y)$ היא איזומורפית לתחילית של $(X, \le_X)$.
	\end{enumerate}
\end{theorem}
נוכיח הכול בשיעור הבא.

\section{שיעור 9 --- 10.7.2024}
\subsection{תורה בסיסית של סדרים טובים}
\begin{definition}[סדר טוב]
	יחס סדר $(X, \le_X)$ הוא סדר טוב אם הוא
	\begin{enumerate}
		\item קווי
		\item מבוסס היטב
	\end{enumerate}
\end{definition}
\begin{theorem}[אינדוקציה על סדרים טובים]
	יהי $(X, \le_X)$ סדר טוב ו־$p(r)$ תכונה. \\*
	אם מתקיים: לכל $x \in X$ אם לכל $y <_X x$, $p(y)$ אז $p(x)$, אז $p(x)$ לכל $x \in X$.
\end{theorem}
\begin{proof}
	נסמן $A = \{ x \in X \mid \lnot p(x) \}$.
	צריך להוכיח ש־$A = \emptyset$, אחרת ניקח את $x_0 = \min_{<_X}(A)$ האיבר המינימלי ב־$A$ לפי $<_X$.
	נשים לב כי לכל $y \in X, y <_X x_0$ מתקיים $y \notin A$.
	נקבל מהגדרת $A$ כי $p(y)$.
	מהנחת האינדוקציה של הטענה נסיק כי $p(x_0)$ וזו סתירה לכך ש־$x_0 \in A$ והגדרת $A$.
\end{proof}
\begin{definition}[שומרת סדר חזק]
	יהי $(X, \le_X)$ סדר, פונקציה $f : X \to X$ היא שומרת סדר חזק אם
	\[
		\forall x, y \in X \, x <_X \implies f(x) <_X f(y)
	\]
\end{definition}
\begin{proposition}
	יהי $(X, \le_X)$ סדר טוב, לכל פונקציה $f : X \to X$ ששומרת סדר חזק מתקיים
	\[
		\forall x \in X \, f(x) \ge_X x
	\]
\end{proposition}
\begin{proof}
	באינדוקציה עבור התכונה $p(x) = x \le_X p(x)$. \\*
	יהי $x \in X$ ונניח ש־$\forall y \in X, y <_X x \implies p(y)$.
	צריך להוכיח $p(x)$.

	נניח אחרת, דהינו $f(x) <_X x$, נסמן $y = f(x)$.
	מההנחה $p(y)$ אנו יודעים כי $y \le_X f(y)$.
	נסיק כי $y <_X x$ וגם $f(y) \ge_X y = f(x)$
	בסתירה להנחה כי $f$ שומרת סדר חזק.
\end{proof}
\begin{definition}[רישא]
	יהי $(X, \le_X)$ סדר, רישא (תחילית) היא תת־קבוצה $X' \subseteq X$ המקיימת
	\[
		\forall x \in X', y \in X, x <_X x \implies y \in X'
	\]
	דהינו, היא סגורה כלפי מטה
\end{definition}
\begin{example}
	ב־$(\NN, <)$ נקבל $X' = 0 = \emptyset$ תחילית באופן ריק. \\*
	גם $X' = 10 = \{0, \dots, 9\}$ תחילית. \\*
	$X' = \NN$ תחילית.
\end{example}
\begin{remark}
	לכל סדר $(X, \le_X)$ ו־$y \in X$ הקבוצה $X_y = \{ x \in X \mid x <_X y\}$ היא תחילית של $X$.
\end{remark}
תחת אילו תנאים נוכל להגיד שכל התחיליות הן מהצורה הזאת בלבד?
\begin{example}
	ניקח את $(\QQ, \le)$ סדר רגיל על הרציונליים, ונבחר $X' = (-\infty, \pi) \cap \QQ$.
	$X'$ תחילית אבל אינה מהצורה $X_y$ עבור $y \in \QQ$.
\end{example}
\begin{proposition}
	יהי $(X, \le_X)$ סדר טוב, לכל תחילית $X' \subseteq X$ חייב להתקיים או $X' = X$ או קיים $y \in X$ כך ש־$X' = X_y$.
\end{proposition}
\begin{proof}
	תהי $X' \subseteq X$ תחילית של $X$, אם $X' = X$ אז סיימנו, אחרת נסמן $\emptyset \ne A = X \setminus X'$. \\*
	כיוון ש־$(X, \le_X)$ סדר טוב אז קיים $y = \min_{\le_X}(A)$.
	נבדוק שהתחילית הנתונה היא $X' = X_y$. \\*
	נבדוק $X' \subseteq X_y$: יהי $x' \in X'$, אם $x' \not<_X y$ אז $x' \ge y$, לכן או ש־$x' = y$ או ש־$x' >_X y$.
	אם $x' \ni x' = y$ וזו סתירה ל־$y \in X \setminus X'$ ואם $x' >_X y$ אז מהעובדה ש־$X$ תחילית נקבל $x' \ni y$ וסתירה דומה. \\*
	מסכימים כי $x' \in X' \implies x' <_X y \implies x' \in X_y$.

	נבדוק $X_y \subseteq X'$.
	לכל $x' \in X_y$ נקבל מהגדרת $X_y$ כי $x' <_X y$, מהמינימליות של $y \in X \setminus X'$ נקבל $x' \in X'$ וסיימנו.
\end{proof}
\begin{theorem}[השוואת סדרים טובים]
	לכל זוג סדרים טובים $(X, \le_X), (Y, \le_Y)$ אחד הסדרים הוא איזומורפי לרישא (תחילית) של השני.
\end{theorem}
\begin{proposition}[טענת עזר]
	לכל קבוצה סדורה היטב $(X, \le_X)$ מתקיים
	\begin{enumerate}
		\item אין שיכון של $X$ בתחילית $X' \subsetneq X$
		\item האיזומורפיזם היחיד בין יחס הסדר לעצמו הוא הזהות
		\item לכל יחס סדר $(Y, \le_Y)$ אם הוא איזומורפי ל־$(X, \le_X)$ אז קיים איזומורפיזם יחיד.
		\item לכל $(Y, \le_Y)$ אם הוא איזומורפי לתחילית $X'$ לש $X$ אז התחילית הזאת היא יחידה.
	\end{enumerate}
\end{proposition}
\begin{proof}
	\begin{enumerate}
		\item אחרת, יש שיכון $f : (X, \le_X) \to (X', \le_{X'})$. אנו יודעים כי $X' \subsetneq X$ ולכן היא מהצורה $X_y$ עבור $y \in X$ כלשהו.
			בפרט נקבל $f : X \to X$ שומרת סדר חזק אבל $f(y) <_X y$ בסתירה לטענה הקודמת.
		\item יהי $f : X \to X$ איזומורפיזם של $(X, \le_X)$ עם עצמו. \\*
			נשים לב כי גם הפונקציה ההופכית $f^{-1} : X \to X$ היא איזומורפיזם.
			נבחין ש־$f, f^{-1}$ שתיהן שומרות סדר חזק ולכן בהינתן $x \in X$ ו־$y = f(x)$ נקבל $x = f^{-1}(y)$. \\*
			נסיק מהטענה הקודמת כי $x \le_X f(x) = y, y \le_X f^{-1}(y) = x$ ולכן $x = y$ ו־$f$ היא פונקציית הזהות.
		\item נניח ש־$(Y, \le_Y)$ סדר ו־$f, g : (X, \le_X) \to (Y, \le_Y)$ זוג איזומורפיזמים, צריך להוכיח $f = g$. \\*
			נשים לב כי $g^{-1} \circ f : X \to X$ היא איזומורפיזם של $(X, \le_X)$ ועצמו, ומהסעיף הקודם נסיק כי $g^{-1} \circ f = id_X$.\\*
			נסיק מהרכבה על שני הצדדים כי $g \circ (g^{-1} \circ f) = g \circ id_X$ ולכן $f = g$.
		\item נניח ש־$(Y, \le_Y)$ איזומורפי לתחיליות $X', X'' \subseteq X$ וצריך להוכיח כי $X' = X''$. \\*
			מכיוון שכל תחילית של $X$ היא מהצורה $X' = X$ או $X' = X_x$ עבור $x \in X$ כלשהו, חייב כי כל זוג תחיליות של $X$ ניתנות להשוואה ב־$\subseteq$. \\*
			אחרת $X' \ne X''$ ומכאן בלי הגבלת הכלליות נקבל כי $X' \subsetneq X''$.
			נסיק כי $(X'', \le_X \upharpoonright x'') \simeq (Y, \le_Y) \simeq (X', \le_X \upharpoonright x')$ אבל $X' \subsetneq X''$ תחילית בסתירה לסעיף 1.
	\end{enumerate}
\end{proof}
\begin{proof}[הוכחת המשפט]
	יהיו $(X, \le_X), (Y, \le_Y)$ זוג סדרים טובים. \\*
	נאמר כי זוג $(x, y) \in X \times Y$ הוא מתאים אם $X_x \simeq Y_y$.
	נסמן $R = \{ (x, y) \in X \times Y \mid (x, y) \text{ מתאימים} \}$.
	הבחנות:
	\begin{enumerate}
		\item אם $(x, y)$ מתאים ו־$(x, y')$ מתאים אז $y = y'$ כנביעה מסעיף 4 בטענת העזר
		\item באופן דומה אם $(x, y)$ ו־$(x', y)$ מתאימים אז $x = x'$
		\item אם $(x, y)$ מתאים ו־$x' <_X x$ אז קיים ויחיד $y' <_X y$ כך ש־$(x', y')$ זוג מתאים
		\item אם $(x, y)$ מתאים ו־$y' <_X y$ אז קיים ויחיד $x' <_X x$ כך ש־$(x', y')$ זוג מתאים
	\end{enumerate}
	נתבונן בקבוצה $X' = \text{dom} R \subseteq X$ ומהבחנה 3 נקבל $X'$ היא רישא. \\*
	באופן דומה הקבוצה $Y' = \rng(R) = \text{dom}(R^{-1}) \subseteq Y$ גם רישא מהבחנה 4. \\*
	נסיק מהבחנות 1 ו־2 כי
	\[
		R : X' \leftrightarrow Y'
	\]
	פונקציה חד־חד ערכית ועל, ושומרת סדר.
	לכן $R : (X', \le_X \upharpoonright x') \to (Y', \le_Y \upharpoonright y')$ איזומורפיזם בין הרישאות. \\*
	נותר להראות כי לפחות אחת מבין $X', Y'$ היא רישא מקסימלית, דהינו $X' = X$ או $Y' = Y$. \\*
	נניח אחרת. לכן יש איברים $x^* \in X, y^* \in Y$ כך ש־$X' = X_{x^*}, Y' = Y_{y^*}$.
	נסיק כי $R$ הוא איזומורפיזם בין $X_{x^*}$ לבין $Y_{y^*}$ ומכאן $(x^*, y^*) \in R$ מתאים.
	לכן $X_{x^*} = \text{dom}(R) \ni x^*$ ולכן $x^* <_X x^*$ וזו סתירה.
\end{proof}

\subsection{סודרים}
תזכורת: ההגדרה שנתנו למספרים הטבעיים נתנה לנו כי לכל $n \in \NN$ מתקיים
\begin{enumerate}
	\item $n$ קבוצה טרנזיטיבית ($\forall m \in n \implies m \subseteq n$)
	\item $(n, \in)$ הוא סדר טוב.
\end{enumerate}
\begin{definition}[סודר]
	קבוצה $\alpha$ תיקרא \textbf{סודר} אם היא מקיימת
	\begin{enumerate}
		\item $\alpha$ קבוצה טרנזיטיבית, דהינו $\forall x \in \alpha, x \subseteq \alpha$
		\item $(\alpha, \in)$ סדר קווי (שקול לטוב)
	\end{enumerate}
\end{definition}
\begin{exercise}
	איך נראים סודרים? כמה סדרים יש?
\end{exercise}
\begin{example}
	$0 = \emptyset$ הוא סודר. \\*
	באופן כללי, כל $n \in \NN$ הוא סודר.
\end{example}
\begin{example}
	הקבוצה $\omega = \NN$ היא טרנזיטיבית כי לכל $n \in \NN$ הראינו כי $n \subseteq \NN$, והיחס $(\NN, \in)$ הוא קווי כפי שכבר הוכחנו.
\end{example}
\begin{conclusion}
	$\omega$ הוא סודר אינסופי.
\end{conclusion}
\begin{proposition}
	לכל סודר $\alpha$, גם $s(\alpha) = \alpha \cup \{ \alpha \}$ היא סודר.
\end{proposition}
\begin{proof}
	נראה כי $s(\alpha)$ טרנזיטיבית, יהי $x \in s(\alpha)$, אם חייב ש־$x \in \alpha$ או $x = \alpha$.
	אם $x \in \alpha$ אז $x \subseteq \alpha \subseteq s(\alpha)$, ואם $x = \alpha$ אז $x \subseteq s(\alpha) = \alpha \cup \{ \alpha \}$. \\*
	יהיו $x, y \in s(\alpha)$ כך ש־$x \ne y$, חייב כי $x, y \in \alpha$ או שבדיוק אחד מבין $x, y$ הוא $\alpha$.
	אם $x, y \in \alpha$ אז חיבי $x \in y$ או $y \in x$, אחרת בלי הגבלת הכלליות $x = \alpha$ אז $y \in \alpha = x$.
\end{proof}
\begin{proposition}
	אם $\emptyset \ne S$ היא קבוצה של סודרים, אז $\alpha_S = \cup S$ הוא סודר.
\end{proposition}

\section{שיעור 10 --- 17.7.2024}
\subsection{סודרים}
ניזכר כי $\alpha$ היא סודר אם היא קבוצה טרנזיטיבית וסדר קווי יחד עם יחס ההכלה.
להיות קווי זה להיות שקול לסדר טוב במקרה זה.
ראינו גם כי אם $\alpha$ סודר אז גם $\alpha + 1$ הוא סודר, וראינו כי גם כל הטבעיים הם סודר, וגם הטבעיים עצמם הם סודר, ובמקרה זה נסמן אותם על־ידי $\omega$.

עוד הוכחנו בתרגול כי אם $\alpha$ סודר אז לכל $\beta \in \alpha$ גם $\beta$ הוא סודר, ואפשר להוסיף בהקשר זה שהוא גם סודר ולכן גם רישא והצמצום של הסדר של $\alpha$ ל־$\beta$ הוא $(\beta, \in)$ עצמו.
מצאנו גם כי לכל זוג סודרים או $\alpha \in \beta$ או ש־$\beta \in \alpha$.
ומצאנו כי גם אם $\alpha \ne \beta$ סודרים אז $(\alpha, \in) \not\cong (\beta, \in)$, דהינו הם לא איזומורפיים. \\*
דבר אחרון שמצאנו בתרגול הוא שמחלקת הסודרים $\text{Ord} = \{ \alpha \mid \alpha \text{ סודר} \}$ היא מחלקה נאותה, דהינו אין קבוצה של כלל הסודרים.

\begin{proposition}
	לכל קבוצת סודרים $S \ne \emptyset$ גם $\bigcup S$ הוא סודר.
\end{proposition}
\begin{proof}
	נסמן $\sigma = \bigcup S$ ונוכיח כי $\sigma$ טרנזיטיבית וסדר קווי יחד עם $\in$.

	יהי $\beta \in \sigma$, מהגדרת $\sigma$ נקבל כי קיים $\alpha \in S$ כך ש־$\beta \in \alpha$, ו־$\alpha$ סודר ולכן $\beta \in \alpha \implies \beta \subseteq \alpha$ וקיבלנו כי $\alpha \in S$.

	נניח כי קיימים $\beta, \beta' \in \sigma$. ניקח $\alpha, \alpha' \in S$ כך ש־$\beta \in \alpha, \beta' \in \alpha'$, אבל $\alpha \in \alpha'$ או הפוך מהטענות שמצאנו ומופיעות בתזכורת.
	נניח ללא הגבלת הכלליות $\alpha \in \alpha$ ולכן $\beta, \beta' \in \alpha'$ ולכן נקבל או $\beta \in \beta'$ או $\beta' \in \beta$ ולכן $(\sigma, \in)$ סדר קווי ומצאנו כי $\bigcup S$ סודר.
\end{proof}
\begin{theorem}
	לכל סדר טוב $(X, <_X)$ קיים ויחיד $\alpha$ סודר כך ש־$(X, <_X) \simeq (\alpha, \in)$.
\end{theorem}
\begin{proof}
	יהי $(X, <_X)$ סדר טוב. \\*
	לכל $x \in X$ אנו יודעים כי הסדר המצומצם $(X_x, <_X \upharpoonright X_x)$ כאשר $X_x = \{ y \in X \mid y <_X x \}$. \\*
	נשתמש במסקנה מהתזכורת ונקבל שאם $X_x$ איזומורפי לסודר אז הסודר הזה הוא יחיד ונסמנו $\alpha_x$. \\*
	נגדיר $D = \{ x \in X \mid (X_x, <_X \upharpoonright X_x) \simeq (\alpha, \in) \}$ קבוצת האיברים ב־$X$ שאיזומורפיים לסודר כלשהו.
	נקבל כי $D \subseteq X$ מההגדרה, נשתמש באקסיומת ההחלפה על המחלקה $F = \{ (x, \alpha_x) \mid x \in D \}$ ומההבחנה האחרונה $F$ מקיימת את תנאי הפונקציה.
	מכיוון ש־$D$ קבוצה נקבל מאקסיומת ההחלפה כי גם התכונה $S = F(D) = \{ \alpha \mid \exists x, \alpha = \alpha_x \}$ ומהטענה האחרונה $\sigma = \bigcup S$ סודר.
	לכן גם $\sigma + 1$ סודר.
	נשים לב כי $\sigma + 1 \notin S$ שכן אחרת נקבל $\sigma + 1 \in \sigma + 1$.
	נפעיל את משפט ההשוואה על סדרים טובים עבור הסדר הטוב $(X, <_X)$ שלנו ו־$(\sigma + 1, \in)$ ואנו יודעים שיש סדר קווי בין כל הסודרים, לכן עלינו רק לפסול ש־$X$ מכיל את $\sigma + 1$. \\*
	לפי המשפט חייב כי $(\sigma + 1, \in)$ איזומורפי לרישא $X_x$ או ש־$(X, <_X)$ איזומורפי לרישא של $\sigma + 1$. \\*
	המקרה השני מסיים את ההוכחה ולכן מספיק לפסול את אופציה א', נניח כי היא נכונה ונקבל כי $\alpha_x = \sigma + 1$ עבור איזשהו $x \in X$ ולכן $\sigma + 1 \in S$ וזו סתירה.
\end{proof}
\begin{theorem}
	לכל סודר $\alpha$ קיים סודר $\beta$ כך ש־$|\alpha| < |\beta|$.
\end{theorem}
\begin{proof}
	נגדיר $A_\le = \{ \beta \in \text{Ord} \mid |\beta| \le |\alpha| \}$ מחלקה ונבחן על $A_= = \{ \beta \in \text{Ord} \mid |\beta| = |\alpha| \}$. \\*
	מספיק להוכיח ש־$A_\le$ קבוצה, שכן אז נסיק כי יש סודר $\beta \in \text{Ord} \setminus A_\le$ ועל־פי הגדרת $A_\le$ חייב להתקיים $|\alpha| < |\beta|$ כמבוקש. \\*
	נשים לב כי $A_\le = A_= \cup \alpha$. \\*
	עבור $\supseteq$ נבחין כי $\forall \beta \in \alpha \implies \beta \subseteq \alpha \implies |\beta| \le |\alpha|$. \\*
	עבור $\subseteq$, לכל $\beta \in A_\le$ אם $\beta \notin \alpha$ אז $\alpha \in \beta$ אבל אז נקבל גם $|\alpha| \le |\beta|$ אבל ידוע כי $|\beta| \le |\alpha|$ ולכן נובע שהעומות שוות ולכן $\beta \in A_=$.

	השוויון $A_\le = A_= \cup \alpha$ מראה כי מספיק להוכיח ש־$A_=$ היא קבוצה כדי לקבל ש־$A_\le$ קבוצה. \\*
	נשים לב כי לכל $\beta \in A_=$ קיים סדר טוב $<_\alpha^\beta$ על $\alpha$ (דהינו $<_\alpha^\beta \subseteq \alpha \times \alpha$) כך ש־$(\alpha, <_\alpha^\beta) \simeq (\beta, \in)$,
	נבנה $<_\alpha^\beta$ באופן הבא:
	ניקח $g : \alpha \to \beta$ הפיכה (שקיימת בהתאם לשוויון העוצמות) ללא קשר לסדרים ונגדיר $<_\alpha^\beta$ באופן הבא
	\[
		\forall \alpha_1, \alpha_2, \alpha_1 <_\alpha^\beta \alpha_2 \iff g(\alpha_1) \in g(\alpha_2)
	\]
	מיידי מהגדרת $g$ כי היא איזומורפיזם בין הסדרים $(\beta, \in) \simeq (\alpha, <_\alpha^\beta)$. \\*
	נתבונן בקבוצה $D = \{ R \subseteq \alpha \times \alpha \mid R \text{ סדר טוב} \alpha, \exists \beta \text{ סודר}, (\alpha, R) \simeq (\beta, \in) \}$.
	נראה ש־$D \subseteq \mathcal{P}(\alpha \times \alpha)$ ולכל $R \in D$ קיים ויחיד $\beta_R$ כך ש־$(\alpha, R) \simeq (\beta, \in) \}$. \\*
	נתבונן ב־$F = \{ (R, \beta_R) \mid R \in D, \text{$\beta_R$ סודר יחיד מתאים ל־$R$}\}$,
	$F$ מקיימת את תנאי הפונקציה ולכן התחום של $F$ היא קבוצה והיא $D$ ולכן מאקסיומת ההחלפה התמונה של $F$ היא קבוצה, אבל $\im(F) = A_=$ הוא קבוצה כמבוקש.
\end{proof}
\begin{notation}
	לכל סודר $\alpha$ נסמן $\alpha^+$ את הסודר המינימלי $\beta$ כך ש־$|\beta| > |\alpha|$.
\end{notation}
\begin{example}
	נבחין כי $0^+ = 1$, וכך גם $21^+ = 2$ בכלל $n^+ = n + 1$. \\*
	לעומת זאת, $\omega^+ \ne \omega + 1$ וגם $\omega^+ \ne \omega + 2$.
	אז איפה נמצא $\omega^+ := \omega_1$?
	קיומו נובע מהמשפט, אבל אין לנו דרך ישירה לגשת אליו ולחשבו.
\end{example}
\begin{definition}[מונה]
	סודר $\kappa$ יקרא \textbf{מונה} אם לכל $\alpha \in \kappa$ מתקיים $|\alpha| < |\kappa|$.
\end{definition}
\begin{example}
	כל $n \in \NN$ הוא מונה, גם $\omega$ הוא מונה, וגם $\omega^+ = \omega_1$ הוא מונה.
\end{example}
\begin{definition}[היררכיית מונים]
	נגדיר לכל סודר $\alpha$ מונה $\omega_\alpha$ באופן הבא
	\[
		\omega_0 = \omega
	\]
	ולכל $\alpha$ בהינתן $\omega_\alpha$ נגדיר $\omega_{\alpha + 1} = \omega_\alpha^+$. \\*
	אם $\delta$ הוא סודר שאינו עוקב (סודר גבולי) אז $\delta = \bigcup \delta$ ונגדיר
	\[
		\omega_\delta = \bigcup_{\alpha \in \delta} \omega_\alpha
	\]
\end{definition}
\begin{definition}[עוצמת המונים האינסופיים]
	נגדיר לכל מונה $\omega_\alpha$ את
	\[
		\aleph_\alpha = |\omega_\alpha|
	\]
\end{definition}
אז $\omega_0 = \omega$ ואיפשהו אחרי זה מופיע $\omega_1 = \omega_0^+$ וכן הלאה, ולבסוף נקבל את האיחוד $\omega_\omega = \bigcup_{n \in \omega} \omega_n$.
עוצמות אלה מסומנות על־ידי אלף, דהינו $\aleph_0, \alpha_1, \dots, \aleph_\omega$.
\begin{definition}[עיקרון הסדר הטוב]
	עיקרון הסדר הטוב אומר כי לכל קבוצה $X$ קיים סדר טוב $(X, <_X)$.
\end{definition}
\begin{conclusion}
	מ־ZF ועיקרון הסדר הטוב נקבל
	\begin{enumerate}
		\item לכל קבוצה $X$ קיים סודר $\alpha$ כך ש־$|X| = |\alpha|$
		\item לכל קבוצה $X$ קיים מונה $\kappa$ כך ש־$|X| = |\kappa|$
		\item לכל זוג קבוצות $X, Y$ מתקיים $|X| \le |Y|$ או $|Y| \le |X|$
	\end{enumerate}
\end{conclusion}
\begin{theorem}[שקילות לאקסיומת הבחירה]
	התנאים הבאים שקולים (תחת ZF):
	\begin{enumerate}
		\item אקסיומת הבחירה (AC)
		\item עיקרון הסדר הטוב
		\item הלמה של צורן
	\end{enumerate}
\end{theorem}

\subsection{הלמה של צורן}
\begin{definition}
	יהי $(Z, <_Z)$ יחס סדר חלקי
	\begin{enumerate}
		\item תת־קבוצה $C \subseteq Z$ תיקרא \textbf{שרשרת} אם (ב־$<_Z$) אם $<_Z$ הוא קווי על $C$
		\item איבר $x \in Z$ הוא \textbf{חסם מלעיל} של שרשרת $C$ אם $\forall y \in C, y \le_Z x$
		\item איבר $m \in Z$ יקרא \textbf{איבר מקסימלי} אם לא קיים $y \in Z$ כך ש־$m <_Z y$
	\end{enumerate}
\end{definition}
\begin{definition}[הלמה של צורן]
	לכל יחס סדר חלקי $(Z, <_Z)$ אם לכל שרשרת $C \subseteq Z$ יש חסם מלעיל אז יש איבר מקסימלי בסדר $(Z, <_Z)$.
\end{definition}

\section{שיעור 11 --- 24.7.2024}
\subsection{הלמה של צורן ומשפט השקילות}
ניזכר בלמה של צורן
\begin{definition}[הלמה של צורן]
	עבור $X \ne \emptyset$, לכל יחס סדר $(X, <_X)$ אם לכל שרשרת $C \subseteq X$ יש חסם מלעיל אז יש ב־$(X, <_X)$ איבר מירבי.
\end{definition}
נבחן דוגמה שמשתמשת בלמה של צורן
\begin{remark}[תזכורת על בסיס למרחב לינארי]
	יהי $V$ מרחב וקטורי מעל שדה $\FF$, קבוצה $S \subseteq V$ היא בלתי תלויה לינארית אם ורק אם לכל $v_1, \dots, v_n \in S$ שונים ולכל $a_1, \dots, a_n \in \FF$ שלא כולם 0 מחייבים
	\[
		\sum_{i = 1}^{n} a_i v_i = a_1 v_1 + \cdots + a_n v_n \ne 0
	\]
	קבוצה $S \subseteq V$ אם לכל $u \in V$ קיימים $v_1, \dots, v_n \in S$ ו־$a_1, \dots, a_n \in \FF$ כך ש־$\sum_{i = 1}^{n} v_i a_i = u$. \\*
	$S \subseteq V$ היא בסיס אם היא פורשת ובלתי תלויה לינארית.
\end{remark}
ניזכר גם במשפט מאלגברה לינארית
\begin{theorem}
	$S \subseteq V$ היא בסיס אם ורק אם היא בלתי תלויה לינארית מקסימלית.
\end{theorem}
\begin{theorem}[בסיס מרחב וקטורי על־ידי ZF + Zorn's Lemma]
	לכל מרחב וקטורי מעל שדה $\FF$ יש בסיס.
\end{theorem}
\begin{proof}
	נתון $V$ מרחב וקטורי מעל שדה $\FF$ ולפי המשפט האחרון נצטרך להוכיח קיום $S \subseteq V$ בלתי תלויה לינארית ומקסימלית,
	דהינו לכל $u \in V \setminus S$, $S \cup \{ u \}$ היא תלויה לינארית. \\*
	נגדיר $(X, <_X)$ כאשר $X = \{ S \subseteq V \mid S \text{ בלתי תלויה לינארית}\}$ ו־$<_X = \subsetneq$. \\*
	צריך להוכיח כי $X \ne \emptyset$ וכי לכל שרשרת $C \subseteq X$ יש חסם מלעיל, ואז נוכל להפעיל את הלמה של צורן.

	אכן $\emptyset \subseteq V$ בלתי תלויה לינארית ולכן $\emptyset \in X$ ובהתאם $X \ne \emptyset$. \\*
	תהי $C \subseteq X$ שרשרת, כלומר לכל $S, S^* \in C$ חייב להתקיים $S \subseteq S^* \lor S^* \subseteq S$.
	נגדיר $S_C = \bigcup C = \bigcup \{ s \mid s \in C \}$.
	נבדוק ש־$S_C \in X$, דהינו שהיא בלתי תלויה לינארית.
	יהי $n \in \omega$ ויהיו $v_1, \dots, v_n \in S_C$ ו־$a_1, \dots, a_n \in \FF$ שלא כולם אפס, מהגדרת $S_C$ לכל $i = 1, \dots, n$ קיים $S_i \in C$ כך ש־$v_i \in S_i$.
	מכיוון שמדובר בשרשרת אז הקבוצות $S_1, \dots, S_n$ משוות על־ידי $\subseteq$ ולכן חייב שיש $i^*$ כך ש־$S_1, \dots, S_n \subseteq S_{i^*}$.
	בפרט נסיק כי $v_1, \dots, v_n \in S_{i^*}$, מכיוון ש־$S_{i^*} \in X$ בלתי תלויה לינארית חייב $\sum_{i = 1}^{n} a_i v_i \ne 0$.

	$\forall S \in C, S \subseteq S_C$ ולכן $S_C \in X$ היא חסומה מלעיל ב־$X$ של $C$.

	בדקנו ש־$(X, <_X)$ מקיים את התנאים של הלמה של צורן ולכן נקבל כי קיים $S^* \in X$ מירבי ב־$\subseteq$.
	על־פי הגדרת $X$ נסיק כי $S^* \subseteq V$ והיא בלתי תלויה מקסימלית.
\end{proof}
\begin{remark}
	יש מרחבים וקטוריים בעלי מימד (גודל בסיס) גדול מ־$\aleph_0$.
\end{remark}
\begin{example}
	נבחן את $V = \QQ^\QQ$ מרחב הפונקציות $f : \QQ \to \QQ$ מעל $\FF = \QQ$. \\*
	נקבל כי $|V| = \aleph_0^{\aleph_0} = 2^{\aleph_0}$.

	יהי $S \subseteq \QQ^\QQ$ בסיס, נטען כי $\aleph_0 < |S|$, מכיוון ש־$S$ בסיס נסיק כי
	\[
		\QQ^\QQ = V = \bigcup_b \{ \sum_{i = 1}^{n} a_1 v_1 \mid a_1, \dots, a_n \in \QQ, v_1, \dots, v_n \in S \}
	\]
	אילו $|S| = \aleph_0$ אז לכל $n$
	\[
		\aleph_0 = |S^n| = |\QQ^n|
	\]
	ולכן גם
	\[
		\aleph_0 = \left\lvert \bigcup_b \left\{ \sum_{i = 1}^{n} a_1 v_1 \middle\vert a_1, \dots, a_n \in \QQ, v_1, \dots, v_n \in S \right\} \right\rvert
	\]
	אבל אז נסיק כי $|V| = \aleph_0$ וזו סתירה.
\end{example}
\begin{theorem}[משפט השקילות]
	נניח ZF, התנאים הבאים שקולים
	\begin{enumerate}
		\item עיקרון הסדר הטוב
		\item אקסיומת הבחירה
		\item הלמה של צורן
	\end{enumerate}
	ולכן בפרט כל אלה נכונים תחת ZFC = ZF + AC.\@
\end{theorem}
\begin{proof}[סקיצה של הוכחה]
	$1 \implies 2$:
	יהיו $\{ X_i \mid i \in I \}$ קבוצה של קבוצות לא ריקות $X_i \ne \emptyset$. \\*
	אנו רוצים להראות פונקציית בחירה $f : I \to \bigcup_{i \in I} X_i$ כך שלכל $i \in I$ נקבל $f(i) \in X_i$. \\*
	נסמן $X = \bigcup_{i \in I} X_i$ ומעיקרון הסדר הטוב קיים סדר טוב $(X, <_X)$. \\*
	נגדיר באמצעות הסדר $<_X$ פונקציה $f$ על־ידי
	\[
		f(i) = \min_{<_X} X_i
	\]
	הגיוני כי $\emptyset \ne X_i \subseteq X$ ולכן יש ב־$X_i$ איבר מינימלי יחיד לפי $<_X$.

	$2 \implies 3$:
	יהי $(X, <_X)$ יחס סדר המקיים את תנאי הלמה של צורן.
	נרצה להשתמש באקסיומת הבחירה כדי למצוא איבר מירבי ב־$X$. \\*
	נגדיר $I = \{ C \subseteq X \mid C \text{ שרשרת} \}$, לכל $C \in I$ נסמן $\emptyset \ne X_C = \{ x \in X \mid x \text{ חסם מלעיל מעל $C$} \}$.
	ניקח פונקציית בחירה $f : I \to \bigcup_{C \in I} X_C$.
	כלומר לכל שרשרת $C$, נקבל $f(C) \in X$ חסם מלעיל.
	ניקח עוד פונקציית בחירה $g : X^* \to X$ עבור $X^* = \{ x \in X \mid x \text{ לא מירבי} \}$.
	נגדיר באמצעות $f$ ו־$g$ איברים $x_\alpha \in X$ ו־$\alpha$ סודר, כל עוד $x_\alpha$ אינו מקסימלי
	\[
		x_0 = f(\emptyset) \in X
	\]
	בהינתן $x_\alpha$ נגדיר $x_{\alpha + 1} = g(x_\alpha) >_X x_\alpha$ בשלב (סודר) גבולי $\delta$. נתבוננן ב־$C = \{ x_\alpha \mid \alpha \in \delta \}$ שרשרת, וניקח $x_\delta = f(C)$.
	מכיוון ש־Ord מחלקה נאותה ו־$X$ קבוצה ו־$x_\beta <_X x_\alpha$ לכל $\beta \in \alpha$, הבנייה חייבת להיעצר בסודר $\alpha$ כלשהו, כלומר $x_\alpha$ חייב להיות מירבי.

	$3 \implies 1$:
	נתונה קבוצה $A$ ונרצה להשתמש בלמה של צורן כדי למצוא סדר טוב על $A$. \\*
	נגדיר $X = \{ \langle B, <_B \rangle \mid B \subseteq A, <_B \text{ סדר טוב על $B$} \}$, \\*
	$<_B$ מוגדר כך
	\[
		(B_1, <_{B_1}) <_X (B_2, <_{B_2}) \iff B_1 \subsetneq B_2 \land <_{B_1} = <_{B_2} \upharpoonright B_1
	\]
	וגם כל איברי $B_2 \setminus B_1$ מעל איברי $B_1$ לפי $<_{B_2}$. \\*
	בודקים כי $(X, <_X)$ מקיים את תנאי הלמה של צורן (תרגיל). \\*
	מהלמה נקבל איבר מקסימלי (מירבי) $(B, <_B) \in X$ סדר טוב. עתה $A = B$, אחרת ניקח $a \in A \setminus B$ ונגדיר $<_{B \cup \{ a \}}$ על $B \cup \{ a \}$ על־ידי הוספת $a$ אחרון.
\end{proof}

\subsection{סיכום}
נבחן את הקשר שבין התורה הנאיבית לבין התורה האקסיומתית
\begin{enumerate}
	\item בתורה הנאיבית הוכחנו שבהינתן סדרת קבוצות $\langle A_n \mid n \in \NN \rangle$ וסדרת פונקציות $\langle h_n \mid n \in \NN \rangle$ כאשר $h_n : \NN \to A_n$ הפיכה אז גם $\bigcup_{n \in \NN} A_n$ בת־מנייה. \\*
		לעומת זאת בתורה האקסיומתית מצאנו שאיחוד $\bigcup_{n \in \NN} A_n$ של קבוצות בנות מנייה $|A_n| = \aleph_0$ לכל $n$ אז גם $\bigcup_{n \in \NN} A_n$ בן־מנייה. \\*
		המעבר בין שתי הטענות מבוסס על AC, פונקציית בחירה שבוחרת לכל $n \in \NN$ את $h_n : \NN \to A_n$ הפיכה (מעידה על כך ש־$A_n$ בת־מנייה).
	\item בחלק הנאיבי הוכחנו כי לכל $A$ בת־מנייה (למשל $A = \emptyset$) גם $\text{seq}(A) = \bigcup_{n \in \NN} A^n$ בת־מנייה. \\*
		ההוכחה שנתנו לא עשתה שימוש ב־AC, במקום זאת הגדרנו סדרה $\langle h_n \mid n \in \NN \rangle$ באופן מפורש לפי מתכון
		\[
			h_1 : \NN \to A,
			h_n : \NN \to A^n,
			h_1 \times h_n : \NN \times \NN \to A^{n + 1}
		\]
		הכלי הפורמלי ב־ZF שמאפשר זאת (שימוש ברקורסיה כדי להוכיח את הטענה) הוא רקורסיה על הטבעיים.
	\item נשארה השאלה מהתורה הנאיבית האם לכל זוג קבוצות $A, B$ מתקיים בהכרח $|A| \le |B|$ או $|B| \le |A|$. \\*
		בעבודה אקסיומתית ובעיקר על־ידי שימוש ב־ZFC נקבל מעיקרון הסדר הטוב אנו יודעים כי קיימים סודרים $\alpha, \beta$
		כך ש־$|A| = |\alpha|$ ו־$|B| = |\beta|$ ומכיוון ש־$\alpha, \beta$ סודרים מתקיים $\alpha \subseteq \beta$ או $\beta \subseteq \alpha$ ולכן בהכרח $|\alpha| \le |\beta|$ או $|\beta| \le |\alpha|$.
	\item בתורה הנאיבית עלתה השאלה האם ניתן לעבוד עם מחלקות עוצמה $|A| = \mathfrak{a}$ כמו קבוצות רגילות (החשש הוא ש־$\mathfrak{a}$ מחלקות נאותות).
		למשל הטענה כי $\forall \mathfrak{a}, \mathfrak{b}, \mathfrak{a} \cdot \mathfrak{b} = \mathfrak{b} \cdot \mathfrak{a}$. \\*
		אקסיומתית מאפשר לנו לפתור את הבעיה הזאת על־ידי עבודה עם מונים.
		הרעיון הוא שתחת ZFC בכל מחלקת עוצמה $\mathfrak{a}$ יש מונה יחיד $\kappa$, ולכן $\mathfrak{a} = |\kappa|$.
		נוכל אם כך לטעון ולעבוד עם מונים בלבד.
		למשל $\forall \kappa, \lambda$ מונים מתקיים $|\kappa \times \lambda| = |\lambda \times \kappa|$.
	\item בחלק הנאיבי הוכחנו כי $\aleph_0 = \aleph_0 \cdot \aleph_0$. \\*
		בחלק האקסיומתי נוכל להוכיח שלכל עוצמה $\mathfrak{a}$ מתקיים $\mathfrak{a} = \mathfrak{a} \cdot \mathfrak{a}$ (מוכיחים כי לכל מונה $\kappa$ מתקיים $|\kappa| = |\kappa \times \kappa|$).
	\item הראינו כי לכל $A, B$ קבוצות אם קיימת $f : A \to B$ חד־חד ערכית אז קיימת $g : B \to A$ על בתורה הנאיבית. \\*
		עתה עם אקסיומת הבחירה נקבל כי לכל $A, B$ קבוצות התנאים הבאים שקולים: קיימת $f : A \to B$ חד־חד ערכית, קיימת $g : B \to A$ על.
		בהינתן $g : B \to A$ על, נגדיר $I = A$ ולכל $a \in I$ נגדיר $\emptyset \ne B_a = g^{-1}(\{a\}) = \{ b \in B \mid g(b) = a \}$ ומאקסיומת הבחירה יש פונקציה $f : I \to \bigcup_a B_a = B$.
		$a \in I, f(a) \in B_a$ ונקבל כי $f$ חד־חד ערכית שכן לכל $a_1 \ne a_2 \in A$ נקבל $B_{a_1} \cap B_{a_2} = \emptyset$ ולכן $f(a_1) \ne f(a_2)$ וכמסקנה $f : A \to B$ חד־חד ערכית.
	\item עוד שאלה שניסחנו בהקשר הנאיבי היא האם $2^{\aleph_0} = |\RR|$ היא המינימלית הגדולה מ־$\aleph_0$.
		ננסח את השאלה בצורה מדויקת יותר, האם קיימת $A \subseteq \RR$ כך ש־$|A| \ne n, \aleph_0, 2^{\aleph_0}$. \\*
		תחת ZFC נקבל ש־$2^{\aleph_0}$ מיוצגת על־ידי מונה $\omega < \kappa$ ואנו יודעים כי אחרי $\omega$ מגיע $\omega_1$ וכן הלאה והם מסומנים על־ידי $\aleph_0, \aleph_1$ וכן הלאה.
		כלומר יש $\alpha$ סודר כך ש־$2^{\aleph_0} = \aleph_\alpha = |\omega_\alpha|$.
		ניסוח מדויק יותר הוא האם $2^{\aleph_0} = \aleph_1$.
		לשאלה זאת קוראים השערת הרצף (CH), קנטור ניסח את השאלה ושיער ש־$2^{\aleph_0} = \aleph_1$.
\end{enumerate}
\begin{theorem}[כהן]
	לא ניתן להכריע את CH ב־ZFC.\@
\end{theorem}
הקורס הבא של תורת הקבוצות, כפייה ואי־תלות, עונה על השאלה הזאת ומלמד את הטכניקה הזאת.

\end{document}
