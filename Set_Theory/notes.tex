\documentclass[a4paper]{article}

% packages
\usepackage{inputenc, fontspec, amsmath, amsthm, amsfonts, polyglossia, catchfile}
\usepackage[a4paper, margin=50pt, includeheadfoot]{geometry} % set page margins

% style
\AddToHook{cmd/section/before}{\clearpage}	% Add line break before section
\linespread{1.5}
\setcounter{secnumdepth}{0}		% Remove default number tags from sections
\setmainfont{Libertinus Serif}
\setsansfont{Libertinus Sans}
\setmonofont{Libertinus Mono}
\setdefaultlanguage{hebrew}
\setotherlanguage{english}

% operators
\DeclareMathOperator\cis{cis}
\DeclareMathOperator\Sp{Sp}
\DeclareMathOperator\tr{tr}
\DeclareMathOperator\im{Im}
\DeclareMathOperator\diag{diag}
\DeclareMathOperator*\lowlim{\underline{lim}}
\DeclareMathOperator*\uplim{\overline{lim}}

% commands
\renewcommand\qedsymbol{\textbf{משל}}
\newcommand{\NN}[0]{\mathbb{N}}
\newcommand{\ZZ}[0]{\mathbb{Z}}
\newcommand{\QQ}[0]{\mathbb{Q}}
\newcommand{\RR}[0]{\mathbb{R}}
\newcommand{\CC}[0]{\mathbb{C}}
\newcommand{\getenv}[2][] {
  \CatchFileEdef{\temp}{"|kpsewhich --var-value #2"}{\endlinechar=-1}
  \if\relax\detokenize{#1}\relax\temp\else\let#1\temp\fi
}
\newcommand{\explain}[2] {
	\begin{flalign*}
		 && \text{#2} && \text{#1}
	\end{flalign*}
}

% headers
\getenv[\AUTHOR]{AUTHOR}
\author{\AUTHOR}
\date\today

\usepackage{hyperref}
\hypersetup{}
\title{תורת הקבוצות}

\begin{document}
\maketitle
\maketitleprint{}

\tableofcontents

\section{שיעור 1 --- 8.5.2024}

מרצה: עומר בן־נריה, מייל: omer.bn@mail.huji.ac.il

\subsection{מבוא}
הקורס בנוי מחצי של תורת הקבוצות הנאיבית, בה מתעסקים בקבוצה באופן כללי ולא ריגורזי, ומחצי של תורת הקבוצות האקסיומטית, בה יש הגדרה חזקה להכול. \\*
הסיבה למעבר לתורה אקסיומטית נעוצה בפרדוקסים הנוצרים ממתמטיקה לא מוסדרת, לדוגמה הפרדוקס של בנך־טרסקי. \\*
עוד דוגמה היא פרדוקס ראסל, אם במתמטקיה שואלים אילו קבוצות קיימות, אינטואיטיבית אפשר להניח שכל קבוצה קיימת, הפרדוקס מתאר שזה לא ממש אופציונלי. נניח שכל קבוצה קיימת, אז ניקח את הקבוצה $y = \{x \mid x \not\in x\}$.
מה אפשר להגיד על $y \in y$ ועל $y \not\in y$, אז נראה כי $y \in y \implies y \not\in y, y\not\in y \implies y \in y$ ואלו הן סתירות מן הסתם. \\*
התוכנית של הילברט, היא ניסיון להגדיר אקסיומטית בסיס רוחבי למתמטיקה, אבל ניתן להוכיח שגם זה לא עובד בלא מעט מקרים.
מומלצת קריאה נוספת על Zermelo Frankel ZF בהקשר לסט האקסיומות הבסיסי המקובל היום.

\subsection{עוצמות}
העוצמה של קבוצה $A$ היא הגודל של $A$. \\*
שאלות: איך משווים בין גדלים של קבוצות $A$ ו־$B$? \\*
הגדרה: נאמר כי זוג קבוצות $A$ ו־$B$ הן שוות עוצמה ונסמן $|A| = |B|$, אם ורק אם יש פונקציה הפיכה $F: A \to B$.

\subsection{תזכורת על פונקציות}
סימון: הזוג הסדור של אובייקטים $x, y$ יסומן $\langle x, y \rangle$. \\*
הערה: אם $x \ne y$ אז $\langle x, y \rangle \ne \langle y, x\rangle$ \\*
המכפלה הקרטזית של קבוצות $A, B$ היא הקבוצה
\[
	A \times B = \{ \langle a, b \rangle \mid a \in A, b \in B \}
\]
הגדרה: יחס בין $A$ ל־$B$ קבוצות, הוא תת־קבוצה $R$ של המכפלה הקרטזית, $R \subseteq A \times B$. \\*
הגדרה: פונקציה $F : A \to B$ היא יחס $F \subseteq A \times B$ המקיים כי $\forall a \in A \exists ! b \in B : \langle a, b \rangle \in F$. \\*
הערה חשובה: $\exists! $ קיים מקרה אחד בלבד כך שמתקיימת טענה. \\*
דוגמה 1: $A = \{0, 1\}, B = \{3, \pi \}, R_1 = \{\langle 0, 3\rangle\}$ לא פונקציה. \\*
דוגמה 2: אותן קבוצות, אבל $R_2 = \{ \langle 0, \pi\rangle, \langle 1, \pi\rangle\}$, היא אכן פונקציה. \\*
דוגמה 3: לכל קבוצה $X$ נסמן $Id_X = \{ \langle a, a \rangle \mid a \in X \}$ מתקיים $Id_X : X \to X$ והיא פונקציית הזהות. \\*
הגדרה: יהי יחס $R \subseteq A \times B$ נגדיר $dom(R) = \{ a \in A \mid \exists b \in B \langle a, b\rangle \in R\}$. \\*
נגדיר $rng(R) = \{ b \in A \mid \exists a \in A \langle a, b\rangle \in R\}$, נקרא לזה גם תמונה של $R$. \\*
הבחנה: אם $R \subseteq A \times B$ הוא פונקציה מ־$A$ ל־$B$ אז $dom(R) = A$ ועוד נראה כי $rng(R) \subseteq B$. \\*
הגדרות בסיסיות נוספות:
\begin{enumerate}
	\item בהינתן $F : A \to B$ אז נסמן לכל $a \in A$ את $F(a)$ להיות $b \in B$ היחיד עבורו מתקיים $\langle a, b \rangle \in F$.
	\item פונקציה $F : A \to B$ היא חד־חד ערכית אם לכל $a_1 \ne a_2$ איברים $a_1, a_2 \in A$ אז מתקיים $F(a_1) \ne F(a_2)$.
	\item פונקציה $F: A \to B$ תיקרא על אם לכל $b \in B$ קיים $a \in A$ כך ש־$\langle a, b \rangle \in R$, או גם $rng(F) = B$.
	\item בהינתן יחס $R$ נגדיר את היחס ההופכי $R^{-1} \subseteq B \times A$ להיות $R^{-1} = \{ \langle b, a\rangle \mid \langle a, b \rangle \in R\}$.
	\item פונקציה $F : A \to B$ נקראת הפיכה אם היחס ההופכי $F^{-1}$ הוא פונקציה מ־$B$ ל־$A$ ונרשום $F^{-1} : B \to A$.
\end{enumerate}
תרגיל: $F : A \to B$ היא הפיכה, אם ורק אם היא חד־חד ערכית ועל $B$. \\*
מסקנה: אם $F : A \to B$ היא פונקציה חד־חד ערכית ועל אז גם הפונקציה ההופכית שלה $F^{-1} : B \to A$ היא חד־חד ערכית ועל. \\*
הוכחה: נתון $F : A \to B$ ונתון כי היא חד־חד ערכית ועל, נסיק כי $F$ הפיכה גם כן ולכן הגדרת ההפיכה מעידה כי $F^{-1} : B \to A$ היא פונקציה. \\*
לכן ${(F^{-1})}^{-1}$ היא פונקציה ולכן $F^{-1}$ היא הפיכה על־פי הגדרה ובהתאם גם חח''ע ועל. \\*
הגדרה: הרכבת יחסים. נניח כי קיימים שני יחסים $R \subseteq A \times B, S \subseteq B \times C$ אז נגדיר $S \circ R \subseteq A \times C$ על־ידי
\[
	S \circ R = \{ \langle a, c \rangle \mid a \in A, c \in C, \exists b \in B : \langle a, b \rangle \in R \land \langle b, c \rangle \in S \}
\]
תרגיל: אם $F : A \to B$ ו־$G : B \to C$ אז $G \circ F \subseteq A \times C$ הוא יחס שהוא גם פונקציה. \\*
הבחנות שהן גם תרגיל: בהינתן פונקציות כמו שהגדרנו השנייה אז מתקיימים המצבים הבאים:
\begin{enumerate}
	\item אם $F, G$ הן חד־חד ערכיות, אז גם $G \circ F$ היא חד־חד ערכית.
	\item אם $F, G$ על אז גם $G \circ F$ היא על.
	\item $F, G$ הפיכות אז $G \circ F$ הפיכה גם היא.
	\item $F$ הפיכה אז $Id_A = F^{-1} \circ F$ וגם $Id_B = F \circ F^{-1}$
\end{enumerate}
נחזור לעוצמות: \\*
נראה כי שוויון עוצמות הוא יחס שקילות:
\begin{enumerate}
	\item אם יש $F: A \to B$ הפיכה אז גם יש $F^{-1} : B \to A$ ולכן $|A| = |B| \iff |B| = |A|$. כלומר יחס שוויון עוצמה הוא סימטרי.
	\item לכל $A$ מתקיים $|A| = |A|$ שכן $Id_A : A \to A$ היא הפיכה לעצמה.
	\item אם $|A| = |B|$ וגם $|B| = |C|$ אז גם $|A| = |C|$ בגלל היכולת להרכיב פונקציות הפיכות מתאימות.
\end{enumerate}

\subsection{קבוצות סופיות}
סימון לכל $n \ge 0$ נסמן $[n] = \{0, 1, \hdots, n - 1 \}$. \\*
הגדרה זמנית: הקבוצה $A$ נקראת סופית אם קיים $n \in \NN$ כך שמתקיים $|A| = |[n]|$. \\*
הבחנה: לכל קבוצה סופית $A \ne \emptyset$ אם $A^*$ מתקבלת מ־$A$ על־ידי השמטת איבר אז $|A| \ne |A^*|$. \\*
טענה: קבוצת כל המספרים הטבעיים $\NN = \{0, 1, \hdots\}$ אינה סופית. \\*
הוכחה: נסמן $\NN^* = \NN\setminus\{0\}$ ונגדיר $F : \NN \to \NN^*$ על־ידי $F(n) = n + 1$, בבירור $F$ חד־חד ערכית ועל $\NN^*$ ולכן $|\NN| = |\NN^*|$.

צריך להשלים את הסוף של ההאצאה.

\section{שיעור 2 --- 15.5.2024}
\subsection{תוצאות ראשונות בשוויון עוצמות}
\subsubsection{הקדמה למשפט קנטור}
לכל מספר $x \in \RR$ יש חלק שלם וחלק שברי כך שמתקיים $x = \lfloor x \rfloor + \langle x \rangle$. \\*
במקרה זה $\lfloor x \rfloor = n \in \ZZ$, כאשר $n \le x$. \\*
נובע כי $0 \le x - \lfloor x \rfloor < 1$. נגדיר $\langle x \rangle = x - \lfloor x \rfloor$. \\*
כל מספר $\langle x \rangle$ ניתן להצגה כהצגה בצור
\[
	\langle x \rangle = 0.x_1x_2 \hdots x_k \hdots
\]
נשים לב כי צורת הצגה זו היא יחידה פרט למקרה בודד בו ''הזנב'' של הספרות נגמר ב־$x_k = 0$ או כאשר הזנב נגמר ב־$x_k = 9$. \\*
לדוגמה $0.359999\hdots = 0.360000\hdots$.

\subsubsection{מונח: פיתוח סטנדרטי}
לכל מספר $x$ עבורו ל־$\langle x \rangle$ יש פיתוח יחיד נקרא לו פיתוח \textbf{סטנדרטי}. \\*
אחרת אם ל־$\langle x \rangle$ יש שני פיתוחים, אז נבחר את זה המסתיים ב־$x_k = 0$ להיות הסטנדרטי.

\subsubsection{משפט קנטור}
\begin{proof}
	נראה כי לכל פונקציה $f : \NN \to \RR$ אז $f$ איננה על $\RR$. \\*
	לכל $n \in \NN$ נרשום את הפיתוח הסטנדרטי של $\langle f(n) \rangle$:
	\[
		\langle f(n) \rangle = 0.x_0^n x_1^n x_2^n \hdots
	\]
	\[
		\begin{matrix}
			\langle f(0) \rangle & 0.x_0^0 & x_1^0 & x_2^0 & \hdots \\
			\langle f(1) \rangle & 0.x_0^1 & x_1^1 & x_2^1 & \hdots \\
			\langle f(2) \rangle & 0.x_0^2 & x_1^2 & x_2^2 & \hdots \\
		\end{matrix}
	\]
	ונבחן את האלכסונים, ונבנה מספר כך שלכל ערך אלכסוני נבחר ספרה שונה מהערך האלכסוני. לכן נוכל לבנות מספר שלא מופיע בכלל ברשימה הזו. \\*
	נתבונן כעת במספר $y \in \RR$ המוגדר על־ידי הפיתוח $y = 0.y_1 y_2 \hdots$ כאשר לכל $n \in \NN$ אנו מגדירים
	\[
		y_n = \begin{cases}
			2, & x_n^n \ne 2 \\
			7, & x_n^n = 2
		\end{cases}
	\]
	מכיוון שכל הספרות בפיתוח הנתון הן 2 או 7 אז פיתוח זה הוא הפיתוח הסטנדרטי של $y$. \\*
	לכל $n \in \NN$ לא יתכן ש־$y = f(n)$ שכן אחרת $\langle y \rangle = \langle f(n) \rangle$ ומכאן של־$\langle y \rangle$ ול־$\langle f(n) \rangle$ אותו פיתוח סטנדרטי בסתירה לכך ש־$y_n \le x_n^n$. \\*
	מסיקים $\forall n \in \NN : y \ne f(n)$ ולכן $y \not\in rng(f)$ ובהתאם $f$ איננה על $\RR$.
\end{proof}

הגדרות נוספות:
\subsubsection{אי־שוויון עוצמות}
עבור קבוצות $A, B$ נאמר שעוצמת $A$ קטנה מעוצמת $B$ או $|A| \le |B|$ כאשר יש פונקציה חד־חד ערכית $f : A \to B$. \\*
נאמר שעוצמת $A$ קטנה ממש מעוצמת $B$ אם $|A| \le |B| \land |A| \ne |B|$.

\textbf{מסקנה:} $|\NN| < |\RR|$ \\*
זאת משום ש־$\NN \subseteq \RR$ ולכן $|\NN| \le |\RR|$ והוכחנו במשפט קנטור ש־$|\NN| \ne |\RR|$.

\subsection{שאלות המשך}
\subsubsection{שאלה 1}
\[
	\NN \subseteq \ZZ \subseteq \QQ \subseteq \text{Alg}_\RR \subseteq \RR
\]
מהן עוצמות קבוצות הביניים בין $\NN$ ל־$\RR$?

\subsubsection{שאלה 2}
האם יש גודל אינסופי מירבי?

\subsubsection{קבוצה בת־מנייה}
קבוצה $A$ ששוות עוצמה ל־$\NN$ תיקרא \textbf{בת־מנייה}.

\subsubsection{קבוצה מעוצמת הרצף}
קבוצה $A$ ששוות עוצמה ל־$\RR$ תיקרא בעוצמת הרצף.

\subsubsection{משפט קנטור־שרדר־ברנשטיין}
תהינה שתי קבוצות $A, B$, אם $|A| \le |B|$ וגם $|B| \le |A|$ אז $|A| = |B|$.
\begin{proof}
	נדחה לסוף הפרק, יושלם בהמשך
\end{proof}

\subsubsection{טענה: עוצמת הטבעיים ומכפלת הטבעיים בעצמם}
\[
	|\NN| = |\NN \times \NN|
\]
נתאהר שתי הוכחות שונות למשפט.
\begin{proof}[בניית הנחש]
	\[
		\begin{matrix}
			(0, 0) & (0, 1) & (0, 2) & \hdots & (0, n) & \hdots \\
			(1, 0) & (1, 1) & (1, 2) & \hdots & (1, n) & \hdots \\
			(2, 0) & (2, 1) & (2, 2) & \hdots & (2, n) & \hdots \\
			\vdots \\
			(m, 0) & (m, 1) & (m, 2) & \hdots & (m, n) & \hdots
		\end{matrix}
	\]
	ונעבור על המטריצה הזאת באופן אלכסוני. \\*
	נגדיר $f : \NN \times \NN \to \NN$ על־ידי
	\[
		f(i, j) = \frac{(i + j)(i + j + 1)}{2} + i
	\]
\end{proof}
\begin{proof}[שימוש במשפט קנטור־שרדר־ברנשטיין]
	נמצא שתי פונקציות
	\[
		f : \NN \times \NN \to \NN,
		g : \NN \to \NN \times
	\]
	את $f$ נגדיר על־ידי $f(n) = (0, n)$. \\*
	ונגדיר $g(i, j) = 2^i 3^j$. שתי הפונקציות כמובן חד־חד ערכיות. \\*
	נובע מיחידות הצגת מספרים טבעיים כמכפלת ראשוניים.
\end{proof}

\subsubsection{טענה: מכפלת קבוצות בנות מנייה}
אם $A, B$ קבוצות בנות מנייה, אז גם $A \times B$ בת מנייה.
\begin{proof}
	נתון $A, B$ בנות מנייה אז ניקח פונקציה $h_B : \NN \to B$ חד־חד ערכית על $B$, \\*
	וניקח $h_A : \NN \to A$
	נקבע פונקציה חד־חד ערכית ועל $f : \NN \to \NN \times \NN$ (מטענה קודמת), $f(n) = (i_n, j_n)$, ונגדיר
	\[
		H : \NN \to A \times B, H(n) = (h_A(i_n), h_B(j_n)) \in A \times B
	\]
	נראה כי $H$ חד־חד ערכית, נניח $n \ne m$ אז $f(n) \ne f(m)$ (כי $f$ חד־חד ערכית). \\*
	אז או $i_n \ne i_m$ או $j_n \ne j_m$ ונקבל $H(n) \ne H(m)$. \\*
	$H$ גם על: $a \in A, b \in B$ וקיימים $i, j \in \NN$ כך ש־$a = h_A(i), b = h_B(j)$, נובע מזה שהן על. \\*
	ידוע כי יש $n \in \NN$ כך ש־$f(n) = (i, j)$ ולכן מחיבור הטענות נקבל כי $H(n) = (a, b)$.
\end{proof}

\subsubsection{הגדרה: חזקה קרטזית}
לכל קבוצה $A$ ו־$k \in \NN$ נגדיר $A^k$ באופן הבא: \\*
אילו $k = 1$ אז $A^k = A$ ובמקרה ש־$k > 1$ אז $A^{k + 1} = A^k \times A$. \\*
סימון: נסמן את אברי $A^k$ על־ידי $(a_1, a_2, \hdots, a_k)$, זאת למרות שבמציאות הקבוצה מוגדרת כ־$(((a_1, a_2), \hdots), a_k)$

\subsubsection{טענה: חזקה קרטזית בת מנייה}
לכל קבוצה $A$ בת־מנייה ו־$k \ge 1$ טבעי נובע $A^k$ בת־מנייה.
\begin{proof}
	באינדוקציה על $k$ ושימוש בטענה האחרונה.
\end{proof}

\subsubsection{קבוצת הרציונליים היא בת־מנייה}
$\QQ$ היא בת־מנייה.
\begin{proof}
	נשתמש במשפט קנטור־שרדר־ברנשטיין
	\[
		\NN \subseteq \QQ \implies |\NN| \le |\QQ|
	\]
	כדי להראות ש־$|\QQ| \le |\NN|$ מספיק לבנות פונקציה חד־חד ערכית לקבוצה בת מנייה כלשהי. \\*
	נגדיר $f : \QQ \to A$. לכל מספר רציונלי $z \ne 0$ יש הצגה יחידה בצורה $z = \pm \frac{p}{q}$ כאשר $p, q > 0$ טבעיים וזרים. \\*
	נגדיר $f : \QQ \to \NN \times \NN \times \NN$ על־ידי
	\[
		f(z) = \begin{cases}
			(0, 0, 0), & z = 0 \\
			(1, p, q), & z > 0 \\
			(2, p, q), & z < 0
		\end{cases}
	\]
	נובע מהגדרתה כי $f$ היא חד־חד ערכית ולכן $|\QQ| \le |\NN \times \NN \times \NN| = |\NN^3| = |\NN|$.
\end{proof}

\section{שיעור 3 --- 22.5.2024}
\subsection{קבוצת הסדרות הסופיות}
\subsubsection{הגדרה}
בהינתן קבוצה $A$ נגדרי 
\[
	seq(A) = \bigcup_{k \ge 1} A^k
\]
קבוצת כל הסדרות הסופיות של $A$.

\subsubsection{טענה: קבוצת הסדרות הסופיות היא בת־מניה}
לכל קבוצה בת־מניה $A$ גם $seq(A)$ היא בת־מניה.

\textbf{טענת עזר:} נניח ש־$B_n$ סדרת קבוצות ו־$h_n$ סדרת פונקציות. $h_n : \NN \to B_n$ הפיכה.\\*
בפרט מתקבל כי $B_n$ בת־מניה, אז הקבוצה
\[
	\bigcup_{n \in \NN} B_n = \{ b \mid \exists n \in \NN, b \in B_n \}
\]
נוכיח ראשית את הטענה בהינתן טענת העזר. \\*
תהי $h : \NN \to A$ הפיכה.
נתון כי $A$ בת־מניה, ונגדיר סדרת פונקציות ${(h_k)}_{k = 1}^\infty$, $h_k : \NN \to A^k$. \\*
נבחר $h_1 = h$. בהינתן $h_k$ נגדיר את $h_{k + 1}$ באופן הבא:
\[
	\tilde{h}_{k + 1} = h_k \times h_1 : \NN \times \NN \to A^k \times A
\]
אנו יודעים כי $\tilde{h}_{k + 1}$ הפיכה, ונשתמש בפונקציה ההפיכה $f : \NN \to \NN \times \NN$ מהשיעור הקודם ונגדיר
\[
	h_{k + 1} = \tilde{h}_{k + 1} \circ f : \NN \to A^{k + 1}
\]
אז תיארנו סדרה של פונקציות $h_k : \NN \to A^k$ הפיכות ומטענת העזר נסיק $seq(A) = \bigcup_{k \ge 1} A^k$ היא בת־מניה. \\*
נוכיח את טענת העזר: \\*
נשתמש במשפט קנטור־ברנשטיין ונראה כי$|\NN| \le | \bigcup_{n \in \NN}B_n|$ הוא א' ו־$|\NN| \ge | \bigcup_{n \in \NN}B_n|$ הוא ב'. \\*
א': נתון כי $B_0$ בת־מנייה, תהי $f_0 : \NN \to B_0$ הפיכה. נשים לב כי ניתן להתייחס ל־$f_0$ כפונקציה לאיחוד והיא עדיין חד־חד ערכית, לכן $|\NN| \le | \bigcup_{n \in \NN}B_n|$. \\*
עתה לב'. מכיוון ש־$|\NN \times \NN| = |\NN|$ די להראות כי קיימת פונקציה חד־חד ערכית
\[
	g : \bigcup_{n \in \NN} \to \NN \times \NN
\]
לכל $n$ נסמן $h_n$ ההופכית של $g_n$ פונקציה על. \\*
נגדיר את $g$ באופן הבא. יהי $b \in \bigcup_{n \in \NN} B_n$ נסמן $n(b) \in \NN$ המספר הטבעי הקטן ביותר עבורו מתקיים $b \in B_{n(bb)}$. \\*
נשים לב כי $b \in B_{n(b)} \implies g_{n(b)}(b) \in \NN$ ובפרט מוגדר. \\*
ניקח
\[
	g(b) = \langle n(b), g_{n(b)}(b) \rangle
\]
נבדוק כי $g$ היא חד־חד ערכית. \\*
יהיו $b \ne b^*$ איברים באיחוד. \\*
נפריד לשני מקרים:
\begin{enumerate}
	\item אם $n(b) \ne n(b^*)$ בוודאי $g(b) \ne g(b^*)$.
	\item אם $n(b) = n(b^*)$ נסמן $n(b) = n(b^*) = m$ אז נסיק ש־$b, b^* \in B_m$ ו־$b \ne b^*$. מכיוון ש־$g_m$ חד־חד ערכית אז נקבל $g_{n(b)}(b) = g_m(b) \ne g_m(b^*) = g_{n(b)}(b)$ ובפרט $g(b) \ne g(b^*)$.
\end{enumerate}

\subsection{משפט קנטור על קבוצת החזקה}
\subsubsection{הגדרה}
בהינתן קבוצה $A$ מגדירים
\[
	\mathcal{P}(A) = \{ B \mid B \subseteq A \}
\]

\subsubsection{דוגמה}
$\mathcal{P}(\emptyset) = \{ \emptyset \}$. \\*
$| \mathcal{P}(\{1, 2, \dots, n \})| = |[2^n]|$

\subsubsection{משפט קנטור}
לכל קבוצה $A$ מתקיים
\[
	|\mathcal{P}(A)| > |A|
\]
\begin{proof}
	הוכחת $A \le \mathcal{P}(A)$: \\*
	נגדיר פונקציה $f : A \to \mathcal{P}(A)$ המוגדרת על־ידי $f(a) = \{a\} \in \mathcal{P}(A)$. \\*
	$f$ חד־חד ערכית ועונה על המבוקש.\\*
	כיוון $|A| \ne |\mathcal{P}(A)|$: \\*
	נוכיח כי לא קיימת פונקציה $g : A \to \mathcal{P}(A)$ שהיא על $\mathcal{P}(A)$. \\*
	תהי $g$ כלשהי, ונגדיר $B \subseteq A$ באופן הבא
	\[
		B = \{ a \in A \mid a \not\in g(a) \}
	\]
	כמובן ש־$B \in \mathcal{P}(A)$ ונטען כי $B \not\in rng(g)$ ומכאן ש־$g$ אינה על $\mathcal{P}(A)$. \\*
	נניח אחרת, אז יש $a^* \in A$ כך ש־$B = g(a^*)$. \\*
	נבדוק האם $a^* \in B$. אם $a^* \in B \overset{\text{הנחת השלילה}}{\iff} a^* \in g(a^*) \overset{\text{הגדרת $B$}}{\iff} a^* \not\in B$. \\*
	קיבלנו סתירה להנחת השלילה ולכן $|A| \ne |\mathcal{P}(A)|$.
\end{proof}

\subsubsection{עוצמות אינסופיות}
נקבל עכשיו ש־$|\NN| < |\mathcal{P}(\NN)| < |\mathcal{P}(\mathcal{P}(\NN)) |$ ונוכל לקבל שלכל $n \in \NN$ מתקיים
\[
	|\mathcal{P}^n(\NN)| < |\mathcal{P}^{n + 1}(\NN)|
\]
נגדיר
\[
	\bigcup_{k \ge 1} \mathcal{P}^k(\NN) = \mathcal{P}^\omega(\NN)
\]
תרגיל:
\[
	\forall k \in \NN \mathcal{P}^k(\NN) < \mathcal{P}^\omega(\NN)
\]
וכמובן גם
\[
	\mathcal{P}^\omega(\NN) = \mathcal{P}(\mathcal{P}^\omega(\NN))
\]
האם קיימת עוצמה גדולה ביותר?

\subsection{פעולות על מחלקות שקילות}
\subsubsection{תזכורת: יחס שקילות}
יחס $E \subseteq X \times X$ הוא יחס שקילות אם הוא רפלקסיבי, סימטרי וטרנזיטיבי.

\subsubsection{דוגמות}
\begin{enumerate}
	\item $X_1 = \ZZ$ והיחס $E_1 = \{ (a, b) \in (\ZZ, \ZZ) \mid a^2 = b^2 \}$
	\item $X_2 = \NN \times \NN$ ו־$E_2 = \{ ((n, m), (n', m')) \mid n + m' = n' + m \}$.
\end{enumerate}

בהינתן יחס שקילות $E$ על קבוצה $X$ מגדירים לכל $x \in X$ את
\[
	{[x]}_E = \{ y \in X \mid (x, y) \in E\}
\]
תכונה חשובה, לכל $x, x^*$ אם ${[x]}_E \cap {[x^*]}_E \ne \emptyset$ אז ${[x]}_E ={[x^*]}_E$. \\*
בדוגמה 1 ${[1]}_E = \{1, -1 \}$ ו־${[0]}_E = \{ 0 \}$. \\*
בנוגע לדוגמה 2 תרגיל בדקו כי זהו יחס שקילות ונראה כי מחלקות השקילות הן $\forall (n, m) \in \NN \times \NN$
מתקיים רק אחד מהשניים:
\begin{enumerate}
	\item $n \ge m$ ולכן $(n - m, 0) \in {[(n, m)]}_{E_2}$
	\item $n < m$ ולכן $(0, m - n) \in {[(n, m)]}_{E_2}$.
\end{enumerate}
אנחנו רואים כי לכל $l \in \NN \setminus \{0\}$ מתאים למחלקות שקילות של ${[(l, 0)]}_{E_2}, {[(0, l)]}_{E_2}$.

\subsubsection{שאלה מנחה}
בהינתן פעולה או יחס על קבוצה $X$, ויחס שקילות $E$ מתי ניתן להגדיר פעולה או יחס מושרית על קבוצת מחלקות השקילות?
\[
	X / E = \{ {[x]}_E \mid x \in X\}
\]
תהי $*$ פעולה על זוגות איברי $X$, דהינו $\forall x_1, x_2 \in X \implies x_1 + x_2 \in X$.\\*
הרעיון, בהינתן מחלקות שקילות $C_1, C_2 \in X/E$ נבקש להגדיר $C_1 * C_2 \in X/E$ נגדיר באופן הבא: \\*
נבחר נציג $x_1 \in C_1$ ו־$x_2 \in C_2$ וננסה להגדיר $C_1 * C_2 = {[x_1 + x_2]}_E$. \\*
הקושי הוא שכדי לקבל פעולה מוגדרת היטב על מחלקות יש לבדוק כי ההגדרה איננה תלויה בבחירת נציגים.
כלומר לכל $x_1, x_1' \in C_1, x_2, x_2' \in C_2$ יתקיים ${[x_1 + x_2]}_E = {[x_1' + x_2']}_E$ ובמקרה כזה נאמר כי הפעולה $*$ על $X$ מוגדרת היטב על קבוצת המנה $X/E$.

\section{שיעור 4}
\subsection{מושג העוצמה}
\subsubsection{תזכורת}
בהינתן יחס שקילות $E$ על קבוצה $X$ נסמן $X/E = \{ {[x]}_E \mid x \in X\}$ קבומצ מחלקות השקילות. \\*
בהינתן פעולה (יחס) $*$ על $X$. \\*
$*$ משרהר פעולה מוגדרת הייטב על $X/E$ אם מתקיימת התכונה הבאה:
\[
	\forall (x_1, x_2) \in E, \forall (y_1, y_2) \in E : (x_1 * y_1, x_2 * y_2) \in E
\]
אי־תלות בנציגים $E$ ביחס לפעולה $*$.

נגדיר את $*$ על $X/E$ על־ידי
\[
	{[x_1]}_E * {[y_1]}_E = {[x_1 * y_1]}_E
\]

נתבונן ביחס
\[
	E = \{ (A, B) \mid \exists f : A \to B \text{ הפיכה} \}
\]
ראינו כי $E$:

\begin{itemize}
	\item רפלקסיבי
	\item סימטרי
	\item טרנזיטיבי
\end{itemize}
ולכן $E$ מקיים את התכונות של יחס שקילות.

\subsubsection{הגדרה (זמנית): עוצמה}
עוצמה היא מחלקת שקילות לפי $E$. \\*
נסמן ב־$|A|$ את מחלקת השקילות של $A$.
סימונים מקובלים נוספים:
\begin{itemize}
	\item $|\NN| = \aleph_0$
	\item $|\RR| = \aleph$ או גם מסמנים ב־$\mathfrak{C}$ גותית שאני לא יודע לעשות בלאטך.
	\item באופן כללי משתמשים באותיות גותעות כדי לסמן את העוצמות של קבוצות, לדוגמה $|A| = \mathfrak{a}, |B| = \mathfrak{b}$.
	\item לקבוצה סופית $[n]$ נסמן גם $|[n]| = n$.
\end{itemize}

\subsubsection{דוגמות}
\begin{enumerate}
	\item $\{1, 2, 3\} \in |[3]|$, $\{\pi, e, \frac{1}{7}\}$.
	\item $|\ZZ| = |\QQ| = \aleph_0$.
	\item באופן דומה נקבל גם $\mathfrak{C} = |\RR| = |\mathbb{C}| = |\RR\backslash a| = |[0, 1]|$.
	\item $0 = |\emptyset|$.
\end{enumerate}

\subsection{פעולות חשבון על עוצמות}
נבקש להגדיר לכל זוג עוצמותת $\mathfrak{a}, \mathfrak{b}$ עוצמות נוספות. \\*
חיבור $\mathfrak{a} + \mathfrak{b}$, כפל $\mathfrak{a} \cdot \mathfrak{b}$ וחזקה $\mathfrak{a}^\mathfrak{b}$.

\subsubsection{כפל}
נתבונן בפעולת המכפלה הקרטזית $\times$ על קבוצה $A \times B$. \\*
נרצה להראות שהיא מגדירה פעולה מוגדרת היטב למחלקות עוצמה.
\begin{proof}
	צריך להוכיח כי בהינתן $A_1, A_2$ שוות עוצמה ו־$B_1, B_2$ שוות עוצמה, אז שמתקיים $|A_1 \times B_1| = |A_2 \times B_2|$.
	נבחר $f : A_1 \to A_2$ הפיכה וגם $g : A_2 \to B_2$ הפיכה שאנו יודעים שקיימות ונבחן את
	\[
		(f \times g) : A_1 \times B_1 \to A_2 \times B_2
	\]
	המוגדרת על־ידי
	\[
		(f \times g)(a, b) = (f(a), g(b))
	\]
	ולכן $f \times g$ היא חד־חד ערכית ועל $A_2 \times B_2$ שכן $f, g$ הן חד־חד ערכיות ועל בנפרד. \\*
	מסקנה: $|A_1 \times B_1| = |A_2 \times B_2|$.
\end{proof}

\subsubsection{הגדרה: כפל עוצמות}
בהינתן עוצמות $\mathfrak{a}, \mathfrak{b}$ נגדיר $\mathfrak{a} \cdot \mathfrak{b}$ באופן הבא: \\*
תהינה $A, B$ קבוצות, $|A| = \mathfrak{a}, |B| = \mathfrak{b}$ אז נגדיר $\mathfrak{a} \cdot \mathfrak{b} = |A \times B|$.

\subsubsection{דוגמה}
\begin{enumerate}
	\item לכל $n, m$ סופיים נראה כי
		\[
			|[n]| \cdot |[m]| = |[n] \times [m]| = |[n \cdot m]|
		\]
	\item $\aleph_0 \cdot \aleph_0 = | \NN \times \NN| = |\NN| = \aleph_0$.
	\item באינדוקציה על $1 \le k$ נגדיר $\overbrace{\aleph_0 \cdot \cdots \aleph_0}^k = \aleph_0$
\end{enumerate}

\subsubsection{פעולת החזקה}
בהינתן קבוצות $A$ ו־$B$ נתבונן בקבוצה $A^B = \{ f : B \to A \}$. \\*
נבקש ללבדוק כי הפעולה הזו לא תלויה בבחירת נציגים ולכן מוגדרת היטב.
\begin{proof}
	צריך להוכיח:
	אם $|A_1| = |A_2|, |B_1| = |B_2|$ אז נראה כי $|A_1^{B_1}| = |A_2^{B_2}|$. \\*
	דהינו
	\[
		|\{ f : B_1 \to A_1 \}| = |\{ f : B_2 \to A_2 \}|
	\]
	נקבע פונקציות הפיכות $f : A_1 \to A_2$ ו־$g : B_1 \to B_2$ הפיכות. \\*
	נגדיר $\varphi : A_1^{B_1} \to A_2^{B_2}$ על־ידי
	\[
		h_1 : B_1 \to A_1,
		\varphi(h_1) = f \circ h_1 \circ g^{-1} : B_2 \to A_2
	\]
	נרצה להראות כי $\varphi$ הפיכה על־ידי מציגת פונקציה הופכית:
	\[
		\psi(h_2) : A_2^{B_2} \to A_1^{B_1},
		\qquad
		\psi(h_2) = f^{-1} \circ h_2 \circ g
	\]
	נבדוק את הרכבת הפונקציות:
	\begin{align*}
		& \psi \circ \varphi = id_{A_1^{B_1}} \\
		& \varphi \circ \psi = id_{A_2^{B_2}}
	\end{align*}
	ונסיק מהקריטריון השקול להפיכות כי $\varphi$ הפיכה ומתקיים $\varphi^{-1} = \psi$. \\*
	נבדוק את ההרכבה הראשונה:
	\[
		\forall h_1 \in A_1^{B_1}, (\psi \circ \varphi)(h_1)
		= \psi(f \circ h_1 \circ g^{-1})
		= f^{-1} \circ f \circ h_1 \circ g^{-1} \circ g
		= (f^{-1} \circ f) \circ h_1 \circ (g^{-1} \circ g)
		= (id_{A_1}) \circ h_1 \circ (id_{B_1})
		= h_1
	\]
	והצד השני דומה.
\end{proof}

\subsubsection{הגדרה: פעולת חזקה על עוצמות}
בהינתן עוצמות $\mathfrak{a}, \mathfrak{b}$ נגדיר עוצמה $\mathfrak{a}^\mathfrak{b}$ באופן הבא: \\*
נבחר קבוצות המקיימות $|A| = \mathfrak{a}, |B| = \mathfrak{b}$ ונגדיר $\mathfrak{a}^\mathfrak{b} = |A^B|$.

\subsubsection{דוגמות}
\begin{enumerate}
	\item יהיו $n, m$ סופיים, מתקיים $|{[n]}^{[m]}| = |[n^m]|$
	\item תהי $\mathfrak{a} = |A|$ ו־$\mathfrak{b} = 0 = |\emptyset|$.
		נשים לב כי $f : \emptyset \to A$ היא פונקציה שכן $\emptyset \subseteq \emptyset \times A$ הוא יחס באופן ריק. הפונקציה היא אכן פונקציה באופן ריק. \\*
		נסיק מכך כי מתקיים $A^\emptyset = \{ \emptyset \}$ ולכן $\mathfrak{a}^\mathfrak{b} = |\{ \emptyset \} = 1$.
\end{enumerate}

\subsubsection{טענה: }
לכל קבוצה $A$ מתקיים $2^{|A|} = |\mathcal{P}(A)|$.
\begin{proof}
	על־פי ההגדרה העוצמה $2^|A|$ שווה למחלקת העוצמה של ${\{0, 1\}}^A = \{ h : A \to \{0, \}\}$. \\*
	לכן כדי להוכיח את הטענה די להראות קיום פונקציה הפיכה בין ${\{0, 1\}}^A$ לבין $\mathcal{P}(A)$. \\*
	נגדיר $\varphi : {\{0, 1\}}^A \to \mathcal{P}(A)$ על־ידי
	\[
		\varphi(h) = h^{-1}(\{1\}) = \{ a \in A \mid h(a) = 1\}
	\]
	נוכיח כי $\varphi$ חד־חד ערכית ועל $\mathcal{P}(A)$:
	\[
		\forall h_1, h_2 : A \to \{0, 1\}, h_1 \ne h_2,
		\exists a \in A : h_1(a) \ne h_2(a)
		\implies
		a \in h_1^{-1}(\{1\}) \triangle h_2^{-1}(\{1\})
	\]
	בפרט הקבוצות $\varphi(h_1), \varphi(h_2)$ הן שונות. \\*
	נוכיח על: יהי $B \subseteq A$, דהינו $B \in \mathcal{P}(A)$. ניקח
	\[
		l_B : A \to \{0, 1\}, l_B(a) = \begin{cases}
			0, & a \notin B \\
			1, & a \in B
		\end{cases}
	\]
	נובע מהגדרת $\varphi$ ש־$\varphi(l_b) = B$ והיא על. \\*
	הראינו כי קיימת פונקציה הפיכה $\varphi : {\{0, 1\}}^A \to \mathcal{P}(A)$ ולכן מתקיים $2^{|A|} = |\mathcal{P}(A)|$.
\end{proof}

\subsubsection{מסקנה}
נובע ממשפט קנטור כי לכל עוצמה $\mathfrak{a}$ מתקיים $\mathfrak{a} < 2^\mathfrak{a}$.

\subsubsection{טענה: שקילות חיבור עוצמות}
יהיו $|A_1| = |A_2|, |B_1| = |B_2|$, אם $\emptyset = A_1 \cap B_1$ וגם $\emptyset = A_2 \cap B_2$. \\*
אז $|A_1 \cup B_1| = |A_2 \cup B_2|$. \\*
הוכחה בתרגיל.

\subsubsection{הגדרה: חיבור עוצמות}
תהינה עוצמות $\mathfrak{a}, \mathfrak{b}$ אז נגדיר את $\mathfrak{a} + \mathfrak{b}$ באופן הבא: \\*
ניקח קבוצות זרות $A, B$ כך ש־$|A| = \mathfrak{a}, |B| = \mathfrak{b}$. \\*
נגדיר את $\mathfrak{a} + \mathfrak{b}$ להיות העוצמה $A \cup B$.

\subsubsection{הגדרה שקולה}
הערה: לכל קבוצה $A$ מתקיים $|\{0\} \times A| = |A| = |\{1\} \times A|$. \\*
לכל זוג $A, B$ קבוצות נראה כי
\[
	\emptyset = (\{ 0 \} \times A) \cap (\{ 1 \} \times B)
\]
ולכן נגדיר
\[
	|A| + |B| = |(\{ 0 \} \times A) \cup (\{ 1 \} \times B)|
\]

\subsubsection{הגדרה: אי־שוויון בין עוצמות}
בהינתן עוצמות $\mathfrak{a}, \mathfrak{b}$ נגדיר כי $\mathfrak{a} \le \mathfrak{b}$ ($\mathfrak{a} < \mathfrak{b}$) \\*
אם יש נציגים $|A| = \mathfrak{a}, |B| = \mathfrak{b}$ כך שקיימת פונקציה חד־חד ערכית $f : A \to B$ (וגם לא קיימת $f$ כזו שהיא גם על).

נבחין כי ההגדרה איננה תלויה בבחירת נציגים $A, B$.

\subsubsection{הערה: ניסוח שקול למשפט קנטור־שרדר ברנשטיין}
ניסוח שקול למשפט הוא שלכל $\mathfrak{a}, \mathfrak{b}$ עוצמות אם גם $\mathfrak{a} \le \mathfrak{b}$ וגם $\mathfrak{b} \le \mathfrak{a}$ אז $\mathfrak{a} = \mathfrak{b}$.

\subsubsection{משפט: כללי חשבון בסיסיים}
\begin{enumerate}
	\item לכל $\mathfrak{a}, \mathfrak{b}$ עוצמות מתקיים $\mathfrak{a} + \mathfrak{b} = \mathfrak{b} + \mathfrak{a}$ וגם $\mathfrak{a} \cdot \mathfrak{b} = \mathfrak{b} \cdot \mathfrak{a}$.
	\item לכל שלוש עוצמות $\mathfrak{a}, \mathfrak{b}_1, \mathfrak{b}_2$ מתקיים $\mathfrak{a} \cdot \mathfrak{b}_1 + \mathfrak{a} \cdot \mathfrak{b}_2 = \mathfrak{a}(\mathfrak{b}_1 + \mathfrak{b}_2)$
	\item לכל שלוש עוצמות $\mathfrak{a}, \mathfrak{b}_1, \mathfrak{b}_2$ מתקיים $\mathfrak{a}^{\mathfrak{b}_1 + \mathfrak{b}_2} = \mathfrak{a}^{\mathfrak{b}_1} \cdot \mathfrak{a}^{\mathfrak{b}_2}$
		וגם ${{(\mathfrak{a}^{\mathfrak{b}_1})}^{\mathfrak{b}_2}} = \mathfrak{a}^{\mathfrak{b}_1 \mathfrak{b}_2}$.
\end{enumerate}


\end{document}
