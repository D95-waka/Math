\documentclass[a4paper]{article}

% packages
\usepackage{inputenc, fontspec, amsmath, amsthm, amsfonts, polyglossia, catchfile}
\usepackage[a4paper, margin=50pt, includeheadfoot]{geometry} % set page margins

% style
\AddToHook{cmd/section/before}{\clearpage}	% Add line break before section
\linespread{1.5}
\setcounter{secnumdepth}{0}		% Remove default number tags from sections
\setmainfont{Libertinus Serif}
\setsansfont{Libertinus Sans}
\setmonofont{Libertinus Mono}
\setdefaultlanguage{hebrew}
\setotherlanguage{english}

% operators
\DeclareMathOperator\cis{cis}
\DeclareMathOperator\Sp{Sp}
\DeclareMathOperator\tr{tr}
\DeclareMathOperator\im{Im}
\DeclareMathOperator\diag{diag}
\DeclareMathOperator*\lowlim{\underline{lim}}
\DeclareMathOperator*\uplim{\overline{lim}}

% commands
\renewcommand\qedsymbol{\textbf{משל}}
\newcommand{\NN}[0]{\mathbb{N}}
\newcommand{\ZZ}[0]{\mathbb{Z}}
\newcommand{\QQ}[0]{\mathbb{Q}}
\newcommand{\RR}[0]{\mathbb{R}}
\newcommand{\CC}[0]{\mathbb{C}}
\newcommand{\getenv}[2][] {
  \CatchFileEdef{\temp}{"|kpsewhich --var-value #2"}{\endlinechar=-1}
  \if\relax\detokenize{#1}\relax\temp\else\let#1\temp\fi
}
\newcommand{\explain}[2] {
	\begin{flalign*}
		 && \text{#2} && \text{#1}
	\end{flalign*}
}

% headers
\getenv[\AUTHOR]{AUTHOR}
\author{\AUTHOR}
\date\today

\usepackage{hyperref}
\hypersetup{}
\title{תורת הקבוצות}

\begin{document}
\maketitle
\maketitleprint{}

\tableofcontents

\section{שיעור 1 --- 8.5.2024}

מרצה: עומר בן־נריה, מייל: omer.bn@mail.huji.ac.il

\subsection{מבוא}
הקורס בנוי מחצי של תורת הקבוצות הנאיבית, בה מתעסקים בקבוצה באופן כללי ולא ריגורזי, ומחצי של תורת הקבוצות האקסיומטית, בה יש הגדרה חזקה להכול. \\*
הסיבה למעבר לתורה אקסיומטית נעוצה בפרדוקסים הנוצרים ממתמטיקה לא מוסדרת, לדוגמה הפרדוקס של בנך־טרסקי. \\*
עוד דוגמה היא פרדוקס ראסל, אם במתמטקיה שואלים אילו קבוצות קיימות, אינטואיטיבית אפשר להניח שכל קבוצה קיימת, הפרדוקס מתאר שזה לא ממש אופציונלי. נניח שכל קבוצה קיימת, אז ניקח את הקבוצה $y = \{x \mid x \not\in x\}$.
מה אפשר להגיד על $y \in y$ ועל $y \not\in y$, אז נראה כי $y \in y \implies y \not\in y, y\not\in y \implies y \in y$ ואלו הן סתירות מן הסתם. \\*
התוכנית של הילברט, היא ניסיון להגדיר אקסיומטית בסיס רוחבי למתמטיקה, אבל ניתן להוכיח שגם זה לא עובד בלא מעט מקרים.
מומלצת קריאה נוספת על Zermelo Frankel ZF בהקשר לסט האקסיומות הבסיסי המקובל היום.

\subsection{עוצמות}
העוצמה של קבוצה $A$ היא הגודל של $A$. \\*
שאלות: איך משווים בין גדלים של קבוצות $A$ ו־$B$? \\*
הגדרה: נאמר כי זוג קבוצות $A$ ו־$B$ הן שוות עוצמה ונסמן $|A| = |B|$, אם ורק אם יש פונקציה הפיכה $F: A \to B$.

\subsection{תזכורת על פונקציות}
סימון: הזוג הסדור של אובייקטים $x, y$ יסומן $\langle x, y \rangle$. \\*
הערה: אם $x \ne y$ אז $\langle x, y \rangle \ne \langle y, x\rangle$ \\*
המכפלה הקרטזית של קבוצות $A, B$ היא הקבוצה
\[
	A \times B = \{ \langle a, b \rangle \mid a \in A, b \in B \}
\]
הגדרה: יחס בין $A$ ל־$B$ קבוצות, הוא תת־קבוצה $R$ של המכפלה הקרטזית, $R \subseteq A \times B$. \\*
הגדרה: פונקציה $F : A \to B$ היא יחס $F \subseteq A \times B$ המקיים כי $\forall a \in A \exists ! b \in B : \langle a, b \rangle \in F$. \\*
הערה חשובה: $\exists! $ קיים מקרה אחד בלבד כך שמתקיימת טענה. \\*
דוגמה 1: $A = \{0, 1\}, B = \{3, \pi \}, R_1 = \{\langle 0, 3\rangle\}$ לא פונקציה. \\*
דוגמה 2: אותן קבוצות, אבל $R_2 = \{ \langle 0, \pi\rangle, \langle 1, \pi\rangle\}$, היא אכן פונקציה. \\*
דוגמה 3: לכל קבוצה $X$ נסמן $Id_X = \{ \langle a, a \rangle \mid a \in X \}$ מתקיים $Id_X : X \to X$ והיא פונקציית הזהות. \\*
הגדרה: יהי יחס $R \subseteq A \times B$ נגדיר $dom(R) = \{ a \in A \mid \exists b \in B \langle a, b\rangle \in R\}$. \\*
נגדיר $rng(R) = \{ b \in A \mid \exists a \in A \langle a, b\rangle \in R\}$, נקרא לזה גם תמונה של $R$. \\*
הבחנה: אם $R \subseteq A \times B$ הוא פונקציה מ־$A$ ל־$B$ אז $dom(R) = A$ ועוד נראה כי $rng(R) \subseteq B$. \\*
הגדרות בסיסיות נוספות:
\begin{enumerate}
	\item בהינתן $F : A \to B$ אז נסמן לכל $a \in A$ את $F(a)$ להיות $b \in B$ היחיד עבורו מתקיים $\langle a, b \rangle \in F$.
	\item פונקציה $F : A \to B$ היא חד־חד ערכית אם לכל $a_1 \ne a_2$ איברים $a_1, a_2 \in A$ אז מתקיים $F(a_1) \ne F(a_2)$.
	\item פונקציה $F: A \to B$ תיקרא על אם לכל $b \in B$ קיים $a \in A$ כך ש־$\langle a, b \rangle \in R$, או גם $rng(F) = B$.
	\item בהינתן יחס $R$ נגדיר את היחס ההופכי $R^{-1} \subseteq B \times A$ להיות $R^{-1} = \{ \langle b, a\rangle \mid \langle a, b \rangle \in R\}$.
	\item פונקציה $F : A \to B$ נקראת הפיכה אם היחס ההופכי $F^{-1}$ הוא פונקציה מ־$B$ ל־$A$ ונרשום $F^{-1} : B \to A$.
\end{enumerate}
תרגיל: $F : A \to B$ היא הפיכה, אם ורק אם היא חד־חד ערכית ועל $B$. \\*
מסקנה: אם $F : A \to B$ היא פונקציה חד־חד ערכית ועל אז גם הפונקציה ההופכית שלה $F^{-1} : B \to A$ היא חד־חד ערכית ועל. \\*
הוכחה: נתון $F : A \to B$ ונתון כי היא חד־חד ערכית ועל, נסיק כי $F$ הפיכה גם כן ולכן הגדרת ההפיכה מעידה כי $F^{-1} : B \to A$ היא פונקציה. \\*
לכן ${(F^{-1})}^{-1}$ היא פונקציה ולכן $F^{-1}$ היא הפיכה על־פי הגדרה ובהתאם גם חח''ע ועל. \\*
הגדרה: הרכבת יחסים. נניח כי קיימים שני יחסים $R \subseteq A \times B, S \subseteq B \times C$ אז נגדיר $S \circ R \subseteq A \times C$ על־ידי
\[
	S \circ R = \{ \langle a, c \rangle \mid a \in A, c \in C, \exists b \in B : \langle a, b \rangle \in R \land \langle b, c \rangle \in S \}
\]
תרגיל: אם $F : A \to B$ ו־$G : B \to C$ אז $G \circ F \subseteq A \times C$ הוא יחס שהוא גם פונקציה. \\*
הבחנות שהן גם תרגיל: בהינתן פונקציות כמו שהגדרנו השנייה אז מתקיימים המצבים הבאים:
\begin{enumerate}
	\item אם $F, G$ הן חד־חד ערכיות, אז גם $G \circ F$ היא חד־חד ערכית.
	\item אם $F, G$ על אז גם $G \circ F$ היא על.
	\item $F, G$ הפיכות אז $G \circ F$ הפיכה גם היא.
	\item $F$ הפיכה אז $Id_A = F^{-1} \circ F$ וגם $Id_B = F \circ F^{-1}$
\end{enumerate}
נחזור לעוצמות: \\*
נראה כי שוויון עוצמות הוא יחס שקילות:
\begin{enumerate}
	\item אם יש $F: A \to B$ הפיכה אז גם יש $F^{-1} : B \to A$ ולכן $|A| = |B| \iff |B| = |A|$. כלומר יחס שוויון עוצמה הוא סימטרי.
	\item לכל $A$ מתקיים $|A| = |A|$ שכן $Id_A : A \to A$ היא הפיכה לעצמה.
	\item אם $|A| = |B|$ וגם $|B| = |C|$ אז גם $|A| = |C|$ בגלל היכולת להרכיב פונקציות הפיכות מתאימות.
\end{enumerate}

\subsection{קבוצות סופיות}
סימון לכל $n \ge 0$ נסמן $[n] = \{0, 1, \hdots, n - 1 \}$. \\*
הגדרה זמנית: הקבוצה $A$ נקראת סופית אם קיים $n \in \NN$ כך שמתקיים $|A| = |[n]|$. \\*
הבחנה: לכל קבוצה סופית $A \ne \emptyset$ אם $A^*$ מתקבלת מ־$A$ על־ידי השמטת איבר אז $|A| \ne |A^*|$. \\*
טענה: קבוצת כל המספרים הטבעיים $\NN = \{0, 1, \hdots\}$ אינה סופית. \\*
הוכחה: נסמן $\NN^* = \NN\setminus\{0\}$ ונגדיר $F : \NN \to \NN^*$ על־ידי $F(n) = n + 1$, בבירור $F$ חד־חד ערכית ועל $\NN^*$ ולכן $|\NN| = |\NN^*|$.

צריך להשלים את הסוף של ההאצאה.

\section{שיעור 2 --- 15.5.2024}
\subsection{תוצאות ראשונות בשוויון עוצמות}
\subsubsection{הקדמה למשפט קנטור}
לכל מספר $x \in \RR$ יש חלק שלם וחלק שברי כך שמתקיים $x = \lfloor x \rfloor + \langle x \rangle$. \\*
במקרה זה $\lfloor x \rfloor = n \in \ZZ$, כאשר $n \le x$. \\*
נובע כי $0 \le x - \lfloor x \rfloor < 1$. נגדיר $\langle x \rangle = x - \lfloor x \rfloor$. \\*
כל מספר $\langle x \rangle$ ניתן להצגה כהצגה בצור
\[
	\langle x \rangle = 0.x_1x_2 \hdots x_k \hdots
\]
נשים לב כי צורת הצגה זו היא יחידה פרט למקרה בודד בו ''הזנב'' של הספרות נגמר ב־$x_k = 0$ או כאשר הזנב נגמר ב־$x_k = 9$. \\*
לדוגמה $0.359999\hdots = 0.360000\hdots$.

\subsubsection{מונח: פיתוח סטנדרטי}
לכל מספר $x$ עבורו ל־$\langle x \rangle$ יש פיתוח יחיד נקרא לו פיתוח \textbf{סטנדרטי}. \\*
אחרת אם ל־$\langle x \rangle$ יש שני פיתוחים, אז נבחר את זה המסתיים ב־$x_k = 0$ להיות הסטנדרטי.

\subsubsection{משפט קנטור}
\begin{proof}
	נראה כי לכל פונקציה $f : \NN \to \RR$ אז $f$ איננה על $\RR$. \\*
	לכל $n \in \NN$ נרשום את הפיתוח הסטנדרטי של $\langle f(n) \rangle$:
	\[
		\langle f(n) \rangle = 0.x_0^n x_1^n x_2^n \hdots
	\]
	\[
		\begin{matrix}
			\langle f(0) \rangle & 0.x_0^0 & x_1^0 & x_2^0 & \hdots \\
			\langle f(1) \rangle & 0.x_0^1 & x_1^1 & x_2^1 & \hdots \\
			\langle f(2) \rangle & 0.x_0^2 & x_1^2 & x_2^2 & \hdots \\
		\end{matrix}
	\]
	ונבחן את האלכסונים, ונבנה מספר כך שלכל ערך אלכסוני נבחר ספרה שונה מהערך האלכסוני. לכן נוכל לבנות מספר שלא מופיע בכלל ברשימה הזו. \\*
	נתבונן כעת במספר $y \in \RR$ המוגדר על־ידי הפיתוח $y = 0.y_1 y_2 \hdots$ כאשר לכל $n \in \NN$ אנו מגדירים
	\[
		y_n = \begin{cases}
			2, & x_n^n \ne 2 \\
			7, & x_n^n = 2
		\end{cases}
	\]
	מכיוון שכל הספרות בפיתוח הנתון הן 2 או 7 אז פיתוח זה הוא הפיתוח הסטנדרטי של $y$. \\*
	לכל $n \in \NN$ לא יתכן ש־$y = f(n)$ שכן אחרת $\langle y \rangle = \langle f(n) \rangle$ ומכאן של־$\langle y \rangle$ ול־$\langle f(n) \rangle$ אותו פיתוח סטנדרטי בסתירה לכך ש־$y_n \le x_n^n$. \\*
	מסיקים $\forall n \in \NN : y \ne f(n)$ ולכן $y \not\in rng(f)$ ובהתאם $f$ איננה על $\RR$.
\end{proof}

הגדרות נוספות:
\subsubsection{אי־שוויון עוצמות}
עבור קבוצות $A, B$ נאמר שעוצמת $A$ קטנה מעוצמת $B$ או $|A| \le |B|$ כאשר יש פונקציה חד־חד ערכית $f : A \to B$. \\*
נאמר שעוצמת $A$ קטנה ממש מעוצמת $B$ אם $|A| \le |B| \land |A| \ne |B|$.

\textbf{מסקנה:} $|\NN| < |\RR|$ \\*
זאת משום ש־$\NN \subseteq \RR$ ולכן $|\NN| \le |\RR|$ והוכחנו במשפט קנטור ש־$|\NN| \ne |\RR|$.

\subsection{שאלות המשך}
\subsubsection{שאלה 1}
\[
	\NN \subseteq \ZZ \subseteq \QQ \subseteq \text{Alg}_\RR \subseteq \RR
\]
מהן עוצמות קבוצות הביניים בין $\NN$ ל־$\RR$?

\subsubsection{שאלה 2}
האם יש גודל אינסופי מירבי?

\subsubsection{קבוצה בת־מנייה}
קבוצה $A$ ששוות עוצמה ל־$\NN$ תיקרא \textbf{בת־מנייה}.

\subsubsection{קבוצה מעוצמת הרצף}
קבוצה $A$ ששוות עוצמה ל־$\RR$ תיקרא בעוצמת הרצף.

\subsubsection{משפט קנטור־שרדר־ברנשטיין}
תהינה שתי קבוצות $A, B$, אם $|A| \le |B|$ וגם $|B| \le |A|$ אז $|A| = |B|$.
\begin{proof}
	נדחה לסוף הפרק, יושלם בהמשך
\end{proof}

\subsubsection{טענה: עוצמת הטבעיים ומכפלת הטבעיים בעצמם}
\[
	|\NN| = |\NN \times \NN|
\]
נתאהר שתי הוכחות שונות למשפט.
\begin{proof}[בניית הנחש]
	\[
		\begin{matrix}
			(0, 0) & (0, 1) & (0, 2) & \hdots & (0, n) & \hdots \\
			(1, 0) & (1, 1) & (1, 2) & \hdots & (1, n) & \hdots \\
			(2, 0) & (2, 1) & (2, 2) & \hdots & (2, n) & \hdots \\
			\vdots \\
			(m, 0) & (m, 1) & (m, 2) & \hdots & (m, n) & \hdots
		\end{matrix}
	\]
	ונעבור על המטריצה הזאת באופן אלכסוני. \\*
	נגדיר $f : \NN \times \NN \to \NN$ על־ידי
	\[
		f(i, j) = \frac{(i + j)(i + j + 1)}{2} + i
	\]
\end{proof}
\begin{proof}[שימוש במשפט קנטור־שרדר־ברנשטיין]
	נמצא שתי פונקציות
	\[
		f : \NN \times \NN \to \NN,
		g : \NN \to \NN \times
	\]
	את $f$ נגדיר על־ידי $f(n) = (0, n)$. \\*
	ונגדיר $g(i, j) = 2^i 3^j$. שתי הפונקציות כמובן חד־חד ערכיות. \\*
	נובע מיחידות הצגת מספרים טבעיים כמכפלת ראשוניים.
\end{proof}

\subsubsection{טענה: מכפלת קבוצות בנות מנייה}
אם $A, B$ קבוצות בנות מנייה, אז גם $A \times B$ בת מנייה.
\begin{proof}
	נתון $A, B$ בנות מנייה אז ניקח פונקציה $h_B : \NN \to B$ חד־חד ערכית על $B$, \\*
	וניקח $h_A : \NN \to A$
	נקבע פונקציה חד־חד ערכית ועל $f : \NN \to \NN \times \NN$ (מטענה קודמת), $f(n) = (i_n, j_n)$, ונגדיר
	\[
		H : \NN \to A \times B, H(n) = (h_A(i_n), h_B(j_n)) \in A \times B
	\]
	נראה כי $H$ חד־חד ערכית, נניח $n \ne m$ אז $f(n) \ne f(m)$ (כי $f$ חד־חד ערכית). \\*
	אז או $i_n \ne i_m$ או $j_n \ne j_m$ ונקבל $H(n) \ne H(m)$. \\*
	$H$ גם על: $a \in A, b \in B$ וקיימים $i, j \in \NN$ כך ש־$a = h_A(i), b = h_B(j)$, נובע מזה שהן על. \\*
	ידוע כי יש $n \in \NN$ כך ש־$f(n) = (i, j)$ ולכן מחיבור הטענות נקבל כי $H(n) = (a, b)$.
\end{proof}

\subsubsection{הגדרה: חזקה קרטזית}
לכל קבוצה $A$ ו־$k \in \NN$ נגדיר $A^k$ באופן הבא: \\*
אילו $k = 1$ אז $A^k = A$ ובמקרה ש־$k > 1$ אז $A^{k + 1} = A^k \times A$. \\*
סימון: נסמן את אברי $A^k$ על־ידי $(a_1, a_2, \hdots, a_k)$, זאת למרות שבמציאות הקבוצה מוגדרת כ־$(((a_1, a_2), \hdots), a_k)$

\subsubsection{טענה: חזקה קרטזית בת מנייה}
לכל קבוצה $A$ בת־מנייה ו־$k \ge 1$ טבעי נובע $A^k$ בת־מנייה.
\begin{proof}
	באינדוקציה על $k$ ושימוש בטענה האחרונה.
\end{proof}

\subsubsection{קבוצת הרציונליים היא בת־מנייה}
$\QQ$ היא בת־מנייה.
\begin{proof}
	נשתמש במשפט קנטור־שרדר־ברנשטיין
	\[
		\NN \subseteq \QQ \implies |\NN| \le |\QQ|
	\]
	כדי להראות ש־$|\QQ| \le |\NN|$ מספיק לבנות פונקציה חד־חד ערכית לקבוצה בת מנייה כלשהי. \\*
	נגדיר $f : \QQ \to A$. לכל מספר רציונלי $z \ne 0$ יש הצגה יחידה בצורה $z = \pm \frac{p}{q}$ כאשר $p, q > 0$ טבעיים וזרים. \\*
	נגדיר $f : \QQ \to \NN \times \NN \times \NN$ על־ידי
	\[
		f(z) = \begin{cases}
			(0, 0, 0), & z = 0 \\
			(1, p, q), & z > 0 \\
			(2, p, q), & z < 0
		\end{cases}
	\]
	נובע מהגדרתה כי $f$ היא חד־חד ערכית ולכן $|\QQ| \le |\NN \times \NN \times \NN| = |\NN^3| = |\NN|$.
\end{proof}

\end{document}
