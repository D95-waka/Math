\documentclass[a4paper]{article}

% packages
\usepackage{inputenc, amsmath, amsthm, thmtools, amsfonts, amssymb, luacode, catchfile, tikzducks, hyperref}
\usepackage[a4paper, margin=50pt, includeheadfoot]{geometry} % set page margins
\usepackage[shortlabels]{enumitem}
\usepackage[skip=3pt, indent=0pt]{parskip}

% language
\usepackage[bidi=basic, layout=tabular, provide=*]{babel}
\babelprovide[main, import]{hebrew}
\babelprovide{rl}
\babelfont{rm}{Libertinus Serif}
\babelfont{sf}{Libertinus Sans}
\babelfont{tt}{Libertinus Mono}

% style
\AddToHook{cmd/section/before}{\clearpage}	% Add line break before section
\linespread{1.3}
\setcounter{secnumdepth}{0}		% Remove default number tags from sections, this won't do well with theorems
\AtBeginDocument{\setlength{\belowdisplayskip}{3pt}}
\AtBeginDocument{\setlength{\abovedisplayskip}{3pt}}

% operators
\DeclareMathOperator\cis{cis}
\DeclareMathOperator\Sp{Sp}
\DeclareMathOperator\tr{tr}
\DeclareMathOperator\im{Im}
\DeclareMathOperator\re{Re}
\DeclareMathOperator\diag{diag}
\DeclareMathOperator*\lowlim{\underline{lim}}
\DeclareMathOperator*\uplim{\overline{lim}}
\DeclareMathOperator\rng{rng}
\DeclareMathOperator\Sym{Sym}
\DeclareMathOperator\Arg{Arg}
\DeclareMathOperator\Log{Log}
\DeclareMathOperator\dom{dom}

% commands
%\renewcommand\qedsymbol{\textbf{מש''ל}}
%\renewcommand\qedsymbol{\fbox{\emoji{lizard}}}
\newcommand{\NN}[0]{\mathbb{N}}
\newcommand{\ZZ}[0]{\mathbb{Z}}
\newcommand{\QQ}[0]{\mathbb{Q}}
\newcommand{\RR}[0]{\mathbb{R}}
\newcommand{\CC}[0]{\mathbb{C}}
\newcommand{\FF}[0]{\mathbb{F}}
\newcommand{\PP}[0]{\mathbb{P}}
\newcommand{\TT}[0]{\mathbb{T}}
\newcommand{\acts}[0]{\circlearrowright}
\newcommand{\explain}[2] {
	\begin{flalign*}
		 && \text{#2} && \text{#1}
	\end{flalign*}
}
\newcommand{\maketitleprint}[0]{ \begin{center}
	\begin{tikzpicture}[scale=3]
		\duck[graduate=gray!20!black, tassel=red!70!black]
	\end{tikzpicture}	
\end{center}
}

% theorem commands
\newtheoremstyle{c_remark}
	{}	% Space above
	{}	% Space below
	{}% Body font
	{}	% Indent amount
	{\bfseries}	% Theorem head font
	{}	% Punctuation after theorem head
	{.5em}	% Space after theorem head
	{\thmname{#1}\thmnumber{ #2}\thmnote{ \normalfont{\text{(#3)}}}}	% head content
\newtheoremstyle{c_definition}
	{3pt}	% Space above
	{3pt}	% Space below
	{}% Body font
	{}	% Indent amount
	{\bfseries}	% Theorem head font
	{}	% Punctuation after theorem head
	{.5em}	% Space after theorem head
	{\thmname{#1}\thmnumber{ #2}\thmnote{ \normalfont{\text{(#3)}}}}	% head content
\newtheoremstyle{c_plain}
	{3pt}	% Space above
	{3pt}	% Space below
	{\itshape}% Body font
	{}	% Indent amount
	{\bfseries}	% Theorem head font
	{}	% Punctuation after theorem head
	{.5em}	% Space after theorem head
	{\thmname{#1}\thmnumber{ #2}\thmnote{ \text{(#3)}}}	% head content

\theoremstyle{c_plain}
\newtheorem{theorem}{משפט}[section]
\newtheorem{lemma}[theorem]{למה}
\newtheorem{proposition}[theorem]{טענה}
\newtheorem*{proposition*}{טענה}
%\newtheorem{corollary}[theorem]{אין חלופה עברית}

\theoremstyle{c_definition}
\newtheorem{definition}[theorem]{הגדרה}
\newtheorem*{definition*}{הגדרה}
\newtheorem{example}{דוגמה}[section]
\newtheorem{exercise}{תרגיל}[section]

\theoremstyle{c_remark}
\newtheorem*{remark}{הערה}
\newtheorem*{solution}{פתרון}
\newtheorem{conclusion}[theorem]{מסקנה}
\newtheorem{notation}[theorem]{סימון}

% Questions related commands
\newcounter{question}
\setcounter{question}{1}
\newcounter{sub_question}
\setcounter{sub_question}{1}

\newcommand{\question}[1][0]{
	\ifthenelse{#1 = 0}{}{\setcounter{question}{#1}}
	\subsection{שאלה \arabic{question}}
	\addtocounter{question}{1}
	\setcounter{sub_question}{1}
}

\newcommand{\subquestion}[1][0]{
	\ifthenelse{#1 = 0}{}{\setcounter{sub_question}{#1}}
	\subsubsection{סעיף \localecounter{letters.gershayim}{sub_question}}
	\addtocounter{sub_question}{1}
}

% import lua and start of document
\directlua{common = require ('../common')}

\GetEnv{AUTHOR}

% headers
\author{\AUTHOR}
\date\today

\title{תורת הקבוצות}

\begin{document}
\maketitle
\maketitleprint{}

\section{שיעור 1 --- 8.5.2024}

מרצה: עומר בן־נריה, מייל: omer.bn@mail.huji.ac.il

\subsection{מבוא}
הקורס בנוי מחצי של תורת הקבוצות הנאיבית, בה מתעסקים בקבוצה באופן כללי ולא ריגורזי, ומחצי של תורת הקבוצות האקסיומטית, בה יש הגדרה חזקה להכול. \\*
הסיבה למעבר לתורה אקסיומטית נעוצה בפרדוקסים הנוצרים ממתמטיקה לא מוסדרת, לדוגמה הפרדוקס של בנך־טרסקי. \\*
עוד דוגמה היא פרדוקס ראסל, אם במתמטקיה שואלים אילו קבוצות קיימות, אינטואיטיבית אפשר להניח שכל קבוצה קיימת, הפרדוקס מתאר שזה לא ממש אופציונלי. נניח שכל קבוצה קיימת, אז ניקח את הקבוצה $y = \{x \mid x \not\in x\}$.
מה אפשר להגיד על $y \in y$ ועל $y \not\in y$, אז נראה כי $y \in y \implies y \not\in y, y\not\in y \implies y \in y$ ואלו הן סתירות מן הסתם. \\*
התוכנית של הילברט, היא ניסיון להגדיר אקסיומטית בסיס רוחבי למתמטיקה, אבל ניתן להוכיח שגם זה לא עובד בלא מעט מקרים.
מומלצת קריאה נוספת על Zermelo Frankel ZF בהקשר לסט האקסיומות הבסיסי המקובל היום.

\subsection{עוצמות}
העוצמה של קבוצה $A$ היא הגודל של $A$. \\*
שאלות: איך משווים בין גדלים של קבוצות $A$ ו־$B$? \\*
הגדרה: נאמר כי זוג קבוצות $A$ ו־$B$ הן שוות עוצמה ונסמן $|A| = |B|$, אם ורק אם יש פונקציה הפיכה $F: A \to B$.

\subsection{תזכורת על פונקציות}
סימון: הזוג הסדור של אובייקטים $x, y$ יסומן $\langle x, y \rangle$. \\*
הערה: אם $x \ne y$ אז $\langle x, y \rangle \ne \langle y, x\rangle$ \\*
המכפלה הקרטזית של קבוצות $A, B$ היא הקבוצה
\[
	A \times B = \{ \langle a, b \rangle \mid a \in A, b \in B \}
\]
הגדרה: יחס בין $A$ ל־$B$ קבוצות, הוא תת־קבוצה $R$ של המכפלה הקרטזית, $R \subseteq A \times B$. \\*
הגדרה: פונקציה $F : A \to B$ היא יחס $F \subseteq A \times B$ המקיים כי $\forall a \in A \exists ! b \in B : \langle a, b \rangle \in F$. \\*
הערה חשובה: $\exists! $ קיים מקרה אחד בלבד כך שמתקיימת טענה. \\*
דוגמה 1: $A = \{0, 1\}, B = \{3, \pi \}, R_1 = \{\langle 0, 3\rangle\}$ לא פונקציה. \\*
דוגמה 2: אותן קבוצות, אבל $R_2 = \{ \langle 0, \pi\rangle, \langle 1, \pi\rangle\}$, היא אכן פונקציה. \\*
דוגמה 3: לכל קבוצה $X$ נסמן $Id_X = \{ \langle a, a \rangle \mid a \in X \}$ מתקיים $Id_X : X \to X$ והיא פונקציית הזהות. \\*
הגדרה: יהי יחס $R \subseteq A \times B$ נגדיר $dom(R) = \{ a \in A \mid \exists b \in B \langle a, b\rangle \in R\}$. \\*
נגדיר $rng(R) = \{ b \in A \mid \exists a \in A \langle a, b\rangle \in R\}$, נקרא לזה גם תמונה של $R$. \\*
הבחנה: אם $R \subseteq A \times B$ הוא פונקציה מ־$A$ ל־$B$ אז $dom(R) = A$ ועוד נראה כי $rng(R) \subseteq B$. \\*
הגדרות בסיסיות נוספות:
\begin{enumerate}
	\item בהינתן $F : A \to B$ אז נסמן לכל $a \in A$ את $F(a)$ להיות $b \in B$ היחיד עבורו מתקיים $\langle a, b \rangle \in F$.
	\item פונקציה $F : A \to B$ היא חד־חד ערכית אם לכל $a_1 \ne a_2$ איברים $a_1, a_2 \in A$ אז מתקיים $F(a_1) \ne F(a_2)$.
	\item פונקציה $F: A \to B$ תיקרא על אם לכל $b \in B$ קיים $a \in A$ כך ש־$\langle a, b \rangle \in R$, או גם $rng(F) = B$.
	\item בהינתן יחס $R$ נגדיר את היחס ההופכי $R^{-1} \subseteq B \times A$ להיות $R^{-1} = \{ \langle b, a\rangle \mid \langle a, b \rangle \in R\}$.
	\item פונקציה $F : A \to B$ נקראת הפיכה אם היחס ההופכי $F^{-1}$ הוא פונקציה מ־$B$ ל־$A$ ונרשום $F^{-1} : B \to A$.
\end{enumerate}
תרגיל: $F : A \to B$ היא הפיכה, אם ורק אם היא חד־חד ערכית ועל $B$. \\*
מסקנה: אם $F : A \to B$ היא פונקציה חד־חד ערכית ועל אז גם הפונקציה ההופכית שלה $F^{-1} : B \to A$ היא חד־חד ערכית ועל. \\*
הוכחה: נתון $F : A \to B$ ונתון כי היא חד־חד ערכית ועל, נסיק כי $F$ הפיכה גם כן ולכן הגדרת ההפיכה מעידה כי $F^{-1} : B \to A$ היא פונקציה. \\*
לכן ${(F^{-1})}^{-1}$ היא פונקציה ולכן $F^{-1}$ היא הפיכה על־פי הגדרה ובהתאם גם חח''ע ועל. \\*
הגדרה: הרכבת יחסים. נניח כי קיימים שני יחסים $R \subseteq A \times B, S \subseteq B \times C$ אז נגדיר $S \circ R \subseteq A \times C$ על־ידי
\[
	S \circ R = \{ \langle a, c \rangle \mid a \in A, c \in C, \exists b \in B : \langle a, b \rangle \in R \land \langle b, c \rangle \in S \}
\]
תרגיל: אם $F : A \to B$ ו־$G : B \to C$ אז $G \circ F \subseteq A \times C$ הוא יחס שהוא גם פונקציה. \\*
הבחנות שהן גם תרגיל: בהינתן פונקציות כמו שהגדרנו השנייה אז מתקיימים המצבים הבאים:
\begin{enumerate}
	\item אם $F, G$ הן חד־חד ערכיות, אז גם $G \circ F$ היא חד־חד ערכית.
	\item אם $F, G$ על אז גם $G \circ F$ היא על.
	\item $F, G$ הפיכות אז $G \circ F$ הפיכה גם היא.
	\item $F$ הפיכה אז $Id_A = F^{-1} \circ F$ וגם $Id_B = F \circ F^{-1}$
\end{enumerate}
נחזור לעוצמות: \\*
נראה כי שוויון עוצמות הוא יחס שקילות:
\begin{enumerate}
	\item אם יש $F: A \to B$ הפיכה אז גם יש $F^{-1} : B \to A$ ולכן $|A| = |B| \iff |B| = |A|$. כלומר יחס שוויון עוצמה הוא סימטרי.
	\item לכל $A$ מתקיים $|A| = |A|$ שכן $Id_A : A \to A$ היא הפיכה לעצמה.
	\item אם $|A| = |B|$ וגם $|B| = |C|$ אז גם $|A| = |C|$ בגלל היכולת להרכיב פונקציות הפיכות מתאימות.
\end{enumerate}

\subsection{קבוצות סופיות}
סימון לכל $n \ge 0$ נסמן $[n] = \{0, 1, \hdots, n - 1 \}$. \\*
הגדרה זמנית: הקבוצה $A$ נקראת סופית אם קיים $n \in \NN$ כך שמתקיים $|A| = |[n]|$. \\*
הבחנה: לכל קבוצה סופית $A \ne \emptyset$ אם $A^*$ מתקבלת מ־$A$ על־ידי השמטת איבר אז $|A| \ne |A^*|$. \\*
טענה: קבוצת כל המספרים הטבעיים $\NN = \{0, 1, \hdots\}$ אינה סופית. \\*
הוכחה: נסמן $\NN^* = \NN\setminus\{0\}$ ונגדיר $F : \NN \to \NN^*$ על־ידי $F(n) = n + 1$, בבירור $F$ חד־חד ערכית ועל $\NN^*$ ולכן $|\NN| = |\NN^*|$.

צריך להשלים את הסוף של ההאצאה.

\end{document}


