\documentclass[a4paper]{article}

% packages
\usepackage{inputenc, fontspec, amsmath, amsthm, amsfonts, polyglossia, catchfile}
\usepackage[a4paper, margin=50pt, includeheadfoot]{geometry} % set page margins

% style
\AddToHook{cmd/section/before}{\clearpage}	% Add line break before section
\linespread{1.5}
\setcounter{secnumdepth}{0}		% Remove default number tags from sections
\setmainfont{Libertinus Serif}
\setsansfont{Libertinus Sans}
\setmonofont{Libertinus Mono}
\setdefaultlanguage{hebrew}
\setotherlanguage{english}

% operators
\DeclareMathOperator\cis{cis}
\DeclareMathOperator\Sp{Sp}
\DeclareMathOperator\tr{tr}
\DeclareMathOperator\im{Im}
\DeclareMathOperator\diag{diag}
\DeclareMathOperator*\lowlim{\underline{lim}}
\DeclareMathOperator*\uplim{\overline{lim}}

% commands
\renewcommand\qedsymbol{\textbf{משל}}
\newcommand{\NN}[0]{\mathbb{N}}
\newcommand{\ZZ}[0]{\mathbb{Z}}
\newcommand{\QQ}[0]{\mathbb{Q}}
\newcommand{\RR}[0]{\mathbb{R}}
\newcommand{\CC}[0]{\mathbb{C}}
\newcommand{\getenv}[2][] {
  \CatchFileEdef{\temp}{"|kpsewhich --var-value #2"}{\endlinechar=-1}
  \if\relax\detokenize{#1}\relax\temp\else\let#1\temp\fi
}
\newcommand{\explain}[2] {
	\begin{flalign*}
		 && \text{#2} && \text{#1}
	\end{flalign*}
}

% headers
\getenv[\AUTHOR]{AUTHOR}
\author{\AUTHOR}
\date\today

\title{פתרון מטלה 10 --- תורת הקבוצות (80200)}

\DeclareMathOperator\dom{dom}
\DeclareMathOperator\add{Add}
\DeclareMathOperator\mult{Mult}
\begin{document}
\maketitle
\maketitleprint{}

\Question{}
נוכיח שלכל סודר $\alpha$ מתקיים $\alpha \le \omega_\alpha$.
\begin{proof}
	נתחיל בבדיקת הטבעיים, מהגדרה נקבל כי $n \in \NN = \omega \in \omega_n$ ולכן $n \le \omega_n$ לכל $n \in \NN$. \\*
	עבור $\omega \le \alpha \le \omega^+$ נקבל $\omega^+ \le \omega_\alpha$ כמובן.
	נוכל כמובן באופן דומה לחסום כל סודר $\alpha$ במונים ולקבל שהמונה $\omega_\alpha$ גדול מהם על־פי הגדרה.
\end{proof}

\Question{}
נוכיח שלכל סודר $\alpha$ קיים סודר $\beta \ge \alpha$ כך ש־$\beta = \omega_\beta$.
\begin{proof}
	נגדיר מחלקת פונקציה $\xi \to \omega_\xi$ ולכן ממשפט הרקורסיה עבור $\alpha$ נקבל כי קיימת הקבוצה $\delta = \{ \alpha, \omega_\alpha, \omega_{\omega_\alpha}, \dots \}$. \\*
	מההגדרה של סודר גבולי נקבל ש־$\omega_\delta = \bigcup_{\beta \in \delta} \beta = \delta$.
\end{proof}

\Question{}
נוכיח שמחלקת המונים היא מחלקה נאותה.
\begin{proof}
	נניח בשלילה כי מחלקת המונים היא קבוצה ונסמנה $\Omega$, ולכן ממשפט הסדר הטוב נסיק כי קיים מונה יחיד $\omega$ כך ש־$|\omega| = |\Omega|$. \\*
	ידוע כי $\Omega$ קבוצת סודרים ולכן $A = \bigcup \Omega$ הוא סודר, ו־$\omega \in A$ ולכן $\omega \subseteq A$ ובהתאם $|\omega| \le |A|$, אבל $\omega^+ \in \Omega$ ולכן גם $\omega^+ \subseteq A$ ונסיק $|A| = |\omega^+|$.
	מצאנו אם כן ש־$|A| \ne |\omega|$ וזו סתירה, לכן מחלקת המונים היא לא קבוצה.
\end{proof}

\directlua{Q_number = 5}
\Question{}
נוכיח שהתנאים הבאים שקולים
\begin{enumerate}
	\item אקסיומת הבחירה: לכל קבוצה $A$ אם $\emptyset \notin A$ אז קיימת $f : A \to \bigcup A$ כך שלכל $a \in A$ מתקיים $f(a) \in a$.
	\item אקסיומת הבחירה לקבוצות חזקה: לכל קבוצה $A$ קיימת $f : \mathcal{P}(A) \setminus \{ \emptyset \} \to A$ כך שלכל $B \subseteq A$ מתקיים $f(B) \in B$.
	\item לכל יחס שקילות יש קבוצת נציגים: אם $E \subseteq X \times X$ יחס שקילות, אז קיימת $Y \subseteq X$ כך שלכל $x \in X$ קיים ויחיד $y \in Y$ כך ש־$\langle y, x \rangle \in E$.
	\item לכל פונקציה יש חתך: תהי $f : A \to B$ כאשר $B = \rng f$, אז קיימת $g : B \to A$ כך שלכל $b \in B$ מתקיים $f(g(b)) = b$.
\end{enumerate}
\begin{proof}
	$1 \to 2$:
	נניח את אקסיומת הבחירה.
	תהי קבוצה $A$ כך ש־$\emptyset \notin A$ ותהי פונקציית בחירה $f : A \to \bigcup A$.
	עתה נגדיר ש־$A = \mathcal{P}(C)$ עבור קבוצה $C$ כלשהי, ולכן $C = \bigcup A$.
	נקבל אם כן כי לכל $B \in A \iff B \subseteq C$ מתקיים $f(B) \in B$.

	$2 \to 3$:
	נניח את אקסיומת הבחירה לקבוצות חזקה.
	יהי $E \subseteq X \times X$ יחס שקילות, ולכן כלל מחלקות השקילות מוכלות בקבוצה $\mathcal{P}(X)$ וידוע כי לא יכולה להיות מחלקת שקילות ריקה, ולכן מחלקות השקילות מוכלות ב־$\mathcal{P}(X) \setminus \{ \emptyset \}$.
	לכן מאקסיומת הבחירה לקבוצות חזקה קיימת פונקציה $f : \mathcal{P}(X) \setminus \{ \emptyset \} \to X$. \\*
	תהי מחלקת שקילות $B \subseteq A$, אז כמובן $f(B) \in B$. אנו יודעים כי צמצום של פונקציה הוא פונקציה ולכן נצמצם את $f$ לחול רק על מחלקות השקילות, ונבחין כי $Y = \rng f$ מקיימת את התנאי שרצינו להוכיח.

	$3 \to 4$:
	נניח כי לכל יחס שקילות יש קבוצת נציגים ונוכיח שלכל פונקציה יש חתך. \\*
	תהי $f : A \to B$, ונגדיר $C = A \cup B$, ואת יחס השקילות $E \subseteq C \times C$ על־ידי $\langle a, b \rangle \in C \iff f(a) = b \lor f(b) = a$.
	לכן קיימת $Y \subseteq C$ קבוצת נציגים, ומיחידות המקור של פונקציות נסיק כי אם $D$ מחלקת שקילות, אז $\forall d, d' \in D, f(d) = f(d')$, ולכן גם קיים ויחיד $y \in Y$ כך ש־$f(y) = f(d)$. \\*
	עתה נגדיר פונקציה חדשה $g : B \to A$ כך ש־$g(b)$ הוא הנציג מ־$Y$ של מחלקת השקילות שהיא המקורות של $b$.
	לכן מתקיים $f(g(b)) = b$ לכל $b \in \rng f = B$.

	$4 \to 1$:
	נניח כי לכל פונקציה יש חתך ונוכיח את אקסיומת הבחירה. \\*
	תהי $A$ קבוצה ותהי $f : \bigcup A \to A$ פונקציה כלשהי על (נוכל לבנות על־ידי מיפוי ערכים לקבוצות המקור שלהם).
	לכן יש לפונקציה חתך $g : A \to \bigcup A$ כך ש־$f(g(b)) = b$ לכל $b \in A$, וזוהי פונקציית בחירה.
\end{proof}

\end{document}
