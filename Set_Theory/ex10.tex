\documentclass[a4paper]{article}

% packages
\usepackage{inputenc, fontspec, amsmath, amsthm, amsfonts, polyglossia, catchfile}
\usepackage[a4paper, margin=50pt, includeheadfoot]{geometry} % set page margins

% style
\AddToHook{cmd/section/before}{\clearpage}	% Add line break before section
\linespread{1.5}
\setcounter{secnumdepth}{0}		% Remove default number tags from sections
\setmainfont{Libertinus Serif}
\setsansfont{Libertinus Sans}
\setmonofont{Libertinus Mono}
\setdefaultlanguage{hebrew}
\setotherlanguage{english}

% operators
\DeclareMathOperator\cis{cis}
\DeclareMathOperator\Sp{Sp}
\DeclareMathOperator\tr{tr}
\DeclareMathOperator\im{Im}
\DeclareMathOperator\diag{diag}
\DeclareMathOperator*\lowlim{\underline{lim}}
\DeclareMathOperator*\uplim{\overline{lim}}

% commands
\renewcommand\qedsymbol{\textbf{משל}}
\newcommand{\NN}[0]{\mathbb{N}}
\newcommand{\ZZ}[0]{\mathbb{Z}}
\newcommand{\QQ}[0]{\mathbb{Q}}
\newcommand{\RR}[0]{\mathbb{R}}
\newcommand{\CC}[0]{\mathbb{C}}
\newcommand{\getenv}[2][] {
  \CatchFileEdef{\temp}{"|kpsewhich --var-value #2"}{\endlinechar=-1}
  \if\relax\detokenize{#1}\relax\temp\else\let#1\temp\fi
}
\newcommand{\explain}[2] {
	\begin{flalign*}
		 && \text{#2} && \text{#1}
	\end{flalign*}
}

% headers
\getenv[\AUTHOR]{AUTHOR}
\author{\AUTHOR}
\date\today

\title{פתרון מטלה 10 --- תורת הקבוצות (80200)}

\DeclareMathOperator\dom{dom}
\DeclareMathOperator\add{Add}
\DeclareMathOperator\mult{Mult}
\begin{document}
\maketitle
\maketitleprint{}

\Question{}
נוכיח שלכל סודר $\alpha$ מתקיים $\alpha \le \omega_\alpha$.
\begin{proof}
	נתחיל בבדיקת הטבעיים, מהגדרה נקבל כי $n \in \NN = \omega \in \omega_n$ ולכן $n \le \omega_n$ לכל $n \in \NN$. \\*
	עבור $\omega \le \alpha \le \omega^+$ נקבל $\omega^+ \le \omega_\alpha$ כמובן.
	נוכל כמובן באופן דומה לחסום כל סודר $\alpha$ במונים ולקבל שהמונה $\omega_\alpha$ גדול מהם על־פי הגדרה.
\end{proof}

\Question{}
נוכיח שלכל סודר $\alpha$ קיים סודר $\beta \ge \alpha$ כך ש־$\beta = \omega_\beta$.
\begin{proof}
	נגדיר מחלקת פונקציה $\xi \to \omega_\xi$ ולכן ממשפט הרקורסיה עבור $\alpha$ נקבל כי קיימת הקבוצה $\delta = \{ \alpha, \omega_\alpha, \omega_{\omega_\alpha}, \dots \}$. \\*
	מההגדרה של סודר גבולי נקבל ש־$\omega_\delta = \bigcup_{\beta \in \delta} \beta = \delta$.
\end{proof}

\Question{}
נוכיח שמחלקת המונים היא מחלקה נאותה.
\begin{proof}
	
\end{proof}

\end{document}
