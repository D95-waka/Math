\documentclass[a4paper]{article}

% packages
\usepackage{inputenc, fontspec, amsmath, amsthm, amsfonts, polyglossia, catchfile}
\usepackage[a4paper, margin=50pt, includeheadfoot]{geometry} % set page margins

% style
\AddToHook{cmd/section/before}{\clearpage}	% Add line break before section
\linespread{1.5}
\setcounter{secnumdepth}{0}		% Remove default number tags from sections
\setmainfont{Libertinus Serif}
\setsansfont{Libertinus Sans}
\setmonofont{Libertinus Mono}
\setdefaultlanguage{hebrew}
\setotherlanguage{english}

% operators
\DeclareMathOperator\cis{cis}
\DeclareMathOperator\Sp{Sp}
\DeclareMathOperator\tr{tr}
\DeclareMathOperator\im{Im}
\DeclareMathOperator\diag{diag}
\DeclareMathOperator*\lowlim{\underline{lim}}
\DeclareMathOperator*\uplim{\overline{lim}}

% commands
\renewcommand\qedsymbol{\textbf{משל}}
\newcommand{\NN}[0]{\mathbb{N}}
\newcommand{\ZZ}[0]{\mathbb{Z}}
\newcommand{\QQ}[0]{\mathbb{Q}}
\newcommand{\RR}[0]{\mathbb{R}}
\newcommand{\CC}[0]{\mathbb{C}}
\newcommand{\getenv}[2][] {
  \CatchFileEdef{\temp}{"|kpsewhich --var-value #2"}{\endlinechar=-1}
  \if\relax\detokenize{#1}\relax\temp\else\let#1\temp\fi
}
\newcommand{\explain}[2] {
	\begin{flalign*}
		 && \text{#2} && \text{#1}
	\end{flalign*}
}

% headers
\getenv[\AUTHOR]{AUTHOR}
\author{\AUTHOR}
\date\today

\title{פתרון ממ''ן 16 – אלגברה לינארית 2 (20229)}

\begin{document}
\maketitle
\maketitleprint{}

\section{שאלה 1}
תהי מטריצה
\[
	A = \begin{pmatrix}
		6 & -9 \\
		1 & 0
	\end{pmatrix}
\]
\subsection{סעיף א'}
נמצא צורת ז'ורדן $G$ של $A$ ומטריצה הפיכה $P$ כך שיתקיים $P^{-1} A P = G$. \\*
נמצא את הפולינום האופייני של $A$:
\[
	|A| = (t - 6) t + 9 = t^2 - 6t + 9 = {(t - 3)}^2
\]
אז $P(t) = {(t - 3)}^2$ ול־$A$ יש ערך עצמי יחיד $3$, ולכן ממשפט 11.9.2 נובע כי $A - 3I$ היא מטריצה נילפוטנטית.
חישוב ישיר מראה כי
\[
	{(A - 3I)}^2
	= \begin{pmatrix}
		3 & -9 \\
		1 & -3
	\end{pmatrix}^2
	= \begin{pmatrix}
		0 & 0 \\
		0 & 0 \\
	\end{pmatrix}
\]
ולכן אינדקס הנילפוטנטיות שלה הוא $2$. \\*
נשתמש באלגוריתם החישוב אשר מופיע בסעיף 11.7 על $A - 3I$ כדי למצוא בסיס אשר בו $A$ היא בצורת ז'ורדן. \\*
נגדיר $B_1 = \ker A - 3I = \Sp\{ (3, 1) \}$. נשלים את $B_1$ לבסיס של $\RR^3$ על־ידי הקבוצה $E_2 = \{ (1, 0) \}$, נקבע גם $D_2 = E_2$. 
נגדיר $D_1 = \{ (A - 3I) D_2 \} = \{ (3, 1) \}$ ולכן הביסיס $D = D_1 \cup D_2 = \{ (3, 1), (1, 0) \}$ בסיס בו $A - 3I$ מטריצה בעלת צורת ז'ורדן. \\*
ממשפט 11.9.2 נובע כי $D$ בסיס בו גם $A$ מקבלת צורת ז'ורדן, נחשב:
\[
	P = \begin{pmatrix}
		3 & 1 \\
		1 & 0
	\end{pmatrix},
	P^{-1} = \begin{pmatrix}
		0 & 1 \\
		1 & -3
	\end{pmatrix}
	J = P^{-1} A P
	= \begin{pmatrix}
		3 & 1 \\
		0 & 3
	\end{pmatrix}
\]

\subsection{סעיף ב'}
נחשב את $A^{100}$ ואת $G^{100}$.
תחילה נחשב את $G^{100}$. ממסקנה 10.1.7 ומטענה 11.3.6 תוך שימוש בהערה 11.3.7 נובע כי
\[
	J^{100} = {J_2 (3)}^{100} = \sum_{k = 0}^1 \binom{100}{k} 3^{100 - k} {J_2(0)}^k
	= \binom{100}{0} 3^{100} \begin{pmatrix}
		1 & 0 \\
		0 & 1
	\end{pmatrix}
	+ \binom{100}{1} 3^{99} \begin{pmatrix}
		0 & 1 \\
		0 & 0
	\end{pmatrix}
	= \begin{pmatrix}
		3^{100} & 100 \cdot 3^{99} \\
		0 & 3^{100}
	\end{pmatrix}
\]
משאלה 8.2.3 א' נובע כי
\[
	A^{100} = P^{-1} J^{100} P
	= \begin{pmatrix}
		3 & 1 \\
		1 & 0
	\end{pmatrix}
	\begin{pmatrix}
		3^{100} & 100 \cdot 3^{99} \\
		0 & 3^{100}
	\end{pmatrix}
	\begin{pmatrix}
		0 & 1 \\
		1 & -3
	\end{pmatrix}
\]

\subsection{סעיף ג'}
נמצא נוסחה עבור $a_n$, כאשר נתון
\[
	a_n = \begin{cases}
		a & n = 0 \\
		b & n = 1 \\
		6a_{n + 1} - 9a_n & n > 1
	\end{cases}
\]
מחישוב ישיר ניתן לראות כי מתקיים
\[
	A \begin{pmatrix}
		a_{n + 1} \\
		a_n
	\end{pmatrix}
	=
	\begin{pmatrix}
		6a_{n + 1} - 9a_n \\
		a_{n + 1}
	\end{pmatrix}
	= \begin{pmatrix}
		a_{n + 2} \\
		a_{n + 1}
	\end{pmatrix}
\]
לכן נוכל להוכיח באינדוקציה כי
\begin{align*}
	\begin{pmatrix}
		a_{n + 2} \\
		a_{n + 1}
	\end{pmatrix}
	= A^n \begin{pmatrix}
		b \\
		a
	\end{pmatrix}
	& = \begin{pmatrix}
		3 & 1 \\
		1 & 0
	\end{pmatrix}
	\begin{pmatrix}
		3^{n} & n \cdot 3^{n - 1} \\
		0 & 3^{n}
	\end{pmatrix}
	\begin{pmatrix}
		0 & 1 \\
		1 & -3
	\end{pmatrix}
	\begin{pmatrix}
		b \\
		a
	\end{pmatrix} \\
	& = \begin{pmatrix}
		3^n & n 3^n + 3^n \\
		3^n & n 3^{n - 1}
	\end{pmatrix}
	\begin{pmatrix}
		0 & 1 \\
		1 & -3
	\end{pmatrix}
	\begin{pmatrix}
		b \\
		a
	\end{pmatrix} \\
	& = \begin{pmatrix}
		(1 + n)3^n & -n 3^n \\
		n 3^{n - 1} & (1 - n) 3^n
	\end{pmatrix}
	\begin{pmatrix}
		b \\
		a
	\end{pmatrix} \\
	\rightarrow a_n & = b(1 + n)3^n - n a 3^n
\end{align*}

\section{שאלה 2}
יהי $V = \CC^n$ מרחב אוניטרי מממד סופי ותהי העתקה לינארית $T : V \to V$. \\*
ידוע כי כל וקטור עצמי של $T$ הוא גם וקטור עצמי של $T^*$. \\*
נוכיח כי $T$ העתקה נורמלית.
\begin{proof}
	יהיו $\lambda_1, \lambda_2, \hdots, \lambda_n$ הערכים העצמיים של $T$. \\*
	יהי $i$ מספר כך ש־$1 \le i \le n$, ונגדיר את $V_i$ להיות המרחב העצמי של $\lambda_i$ לכל $i$. \\*
	אנו יודעים כי לכל $u \in V_i$ מתקיים $T u = \lambda_i u \in V_i$ ולכן נגדיר $T_i : V_i \to V_i$ צמצום של $T$ ל־$V_i$ בהתאם לשאלה 8.4.3 א'. \\*
	מהגדרתה נובע ש־$T_i$ היא העתקה סקלרית ולכן מטריצת יצוגה דומה ל־$\lambda_i I_n$, מהגדרה 3.1.1 נובע כי היא לכסינה אוניטרית ונורמלית. \\*
	מסיבה זו נוכל גם לקבוע כי קיים בסיס אורתונורמלי $B_i \subseteq V_i$ אשר מלכסן אוניטרית את $T_i$. \\*
	מנורמליות $T_i$ על־פי למה 3.2.5 נובע גם כי לכל $u \in V_i$
	\[
		T_i^* u = \overline{\lambda_i} u \tag{1}
	\]
	יהיו $i, j$ כך ש־$1 \le i, j \le n, i \ne j$, ויהיו וקטורים $v_i \in V_i, v_j \in V_j$. \\*
	אנו יודעים כי כל וקטור עצמי של $T$ הוא גם וקטור עצמי של $T^*$, אז
	\[
		(T v_i, v_j)
		= \lambda_i (v_i, v_j) = (v_i, T^* v_j)
		\overset{\text{8.4.8}}{=} (v_i, T_j^* v_j)
		\overset{(1)}{=} (v_i, \overline{\lambda_j} v_j)
		\overset{\text{1.2.3}}{=} \lambda_j (v_i, v_j)
	\]
	ולכן בהתאם
	\[
		(\lambda_i - \lambda_j) (v_i, v_j) = 0
	\]
	ידוע כי $\lambda_i \ne \lambda_j$ ולכן בהכרח $(v_i, v_j) = 0$ ובהתאם כל שני וקטורים עצמיים לערכים עצמיים שונים הם אורתוגונליים. \\*
	אנו יודעים כי $B_i$ ו־$B_j$ בלתי תלויות לינארית, ועתה נובע גם כי $B_i \perp B_j$, לכן גם
	\[
		B = \bigcup_{i = 1}^n B_i
	\]
	הוא אורתונורמלי, בלתי תלוי לינארית, ומהגדרת $B_i$ מהווה קבוצת יוצרים ל־$V$. \\*
	אנו יודעים כי לכל $b \in B$ מתקיים $T b = \alpha b$ ולכן מהגדרת האלכסוניות $T$ תיוצג כמטריצה אלכסונית בבסיס $B$. \\*
	לכן מהגדרה 3.1.1 נובע כי $T$ לכסינה אוניטרית ולכן גם נורמלית.
\end{proof}

\section{שאלה 3}
\subsection{סעיף א'}
נמצא צורת ז'ורדן למטריצה הבאה
\[
	A = \begin{pmatrix}
		1 & -3 & 0 & 3 \\
		-2 & -6 & 0 & 13 \\
		0 & -3 & 1 & 3 \\
		-1 & -4 & 0 & 8
	\end{pmatrix}
\]
ננתחיל בחישוב ערכיה העצמיים בעזרת פולינום אופייני:
\begin{align*}
	P(t)
	& = \begin{vmatrix}
		t - 1 & 3 & 0 & -3 \\
		2 & t + 6 & 0 & -13 \\
		0 & 3 & t - 1 & -3 \\
		1 & 4 & 0 & t - 8
	\end{vmatrix}
	= (t - 1) \begin{vmatrix}
		t - 1 & 3 & -3 \\
		2 & t + 6 & -13 \\
		1 & 4 & t - 8
	\end{vmatrix} \\
	& = (t - 1) ((t -1) (t + 6) (t - 8) - 39 - 24 + 3 (t + 6) - 6 (t - 8) + 52 (t - 1)) = {(t - 1)}^4
\end{align*}
ל־$A$ ערך עצמי יחיד $1$ אשר ריבויו האלגברי הוא $4$. \\*
נחשב את הריבוי הגאומטרי של $1$ עבור $A$ על־ידי חישוב ממד מרחב הפתרונות של $(A - I) u = 0$:
\[
	\begin{pmatrix}
		0 & -3 & 0 & 3 \\
		-2 & -7 & 0 & 13 \\
		0 & -3 & 0 & 3 \\
		-1 & -4 & 0 & 7
	\end{pmatrix}
	\rightarrow
	\begin{pmatrix}
		1 & 4 & 0 & -7 \\
		0 & 1 & 0 & 1 \\
		0 & -3 & 0 & 3 \\
		0 & -3 & 0 & 3
	\end{pmatrix}
	\rightarrow \rho (A - I) = 2
\]
אז הריבוי הגאומטרי של $1$ הוא $2$. \\*
מהגדרה 9.10.1 נובע ש־$A - I$ נילפוטנטית, נחשב את אינקס הנילפוטנטיות שלה על־ידי חישוב ישיר של $A^k$. \\*
מחישוב ישיר מוצאים כי ${(A - I)}^2 \ne 0, {(A - I)}^3 = 0$, לכן אינדקס הנילפוטנטיות של $A - I$ הוא $3$. \\*
מטענה 11.8.1 אנו מסיקים כי מספר מטריצות הז'ורדן היסודיות המופיעות במטריצת הז'ורדן הדומה ל־$A$ היא $2$ ומטריצת הז'ורדן הגדולה ביותר היא בגודל $3$.
לכן $A$ דומה למטריצה הבאה
\[
	J = \begin{pmatrix}
		1 & 1 & 0 & 0 \\
		0 & 1 & 1 & 0 \\
		0 & 0 & 1 & 0 \\
		0 & 0 & 0 & 1
	\end{pmatrix}
\]

\subsection{סעיף ב'}
נגדיר מטריצה
\[
	B = \begin{pmatrix}
		1 & -3 & 3 \\
		-2 & -6 & 13 \\
		-1 & -4 & 8
	\end{pmatrix}
\]
נמצא צורת ז'ורדן $J$ של המטריצה $B$ ומטריצה הפיכה $P$ המקיימת $B = P^{-1} J P$. \\*
מתהליך חישוב הפולינום האופייני של $A$ אנו למדים כי $P_B(t) = {(t - 1)}^3$. \\*
מחישוב ישיר אנו מקבלים כי $\rho(B - I) = 2$, וגם כי אינדקס הנילפוטנטיות של $A - I$ הוא $3$. \\*
בשל נתונים אלה אנו יכולים להסיק מטענה 11.8.1
\[
	J = \begin{pmatrix}
		1 & 1 & 0 \\
		0 & 1 & 1 \\
		0 & 0 & 1
	\end{pmatrix}
\]
עתה נמצא את המטריצה $P$: \\*
המטריצה $P$ המקיימת את התנאי היא מטריצת בין $B - I$ לבין $J_3(0)$. \\*
נגדיר בסיס $W = (w_1, w_2, w_3)$ בסיס הז'ורדן של $B - I$ וכמובן גם של $B$. \\*
ידוע כי
\[
	J_3(0) = \begin{pmatrix}
		0 & 1 & 0 \\
		0 & 0 & 1 \\
		0 & 0 & 0 \\
	\end{pmatrix}
\]
ולכן על־פי הגדרת הבסיס $W$ צריך להתקיים $(B - I) w_3 = w_2$.
נגדיר $w_3 = (1, 0, 0)$ מטעמי פשטות, ולכן
\[
	(B - I) w_3
	= \begin{pmatrix}
		0 & -3 & 3 \\
		-2 & -7 & 13 \\
		-1 & -4 & 7
	\end{pmatrix}
	\begin{pmatrix}
		1 \\ 0 \\ 0
	\end{pmatrix}
	= \begin{pmatrix}
		0 \\
		-2 \\
		-1
	\end{pmatrix}
\]
בהתאם להגדרת $w_3$ נובע כי $w_2 = (0, -2, -1)$. \\*
על־פי הגדרת $W$ מתקיים גם $(B - I) w_2 = w_1$, נחשב
\[
	(A - I) w_2 = w_1 = (3, 3, 1)
\]
נשים לב כי אכן מתקיים $(B - I) w_1 = 0$ על־פי חישוב ישיר, ואכן $W$ בסיס ז'ורדן של $B$. \\*
מהגדרת מטריצת מעבר נקבע
\[
	P = \begin{pmatrix}
		3 & 0 & 1 \\
		3 & -2 & 0 \\
		1 & -1 & 0
	\end{pmatrix}
\]

\section{שאלה 4}
תהי $A$ מטריצה ריבועית מסדר $7$ בעלת ערך עצמי יחיד $\lambda \in \CC$. \\*
ידוע כי $\rho{(A - \lambda I)}^2 = 1$ וכי $\rho(A - \lambda I) = 2$. \\*
נמצא את צורת ז'ורדן ואת הפולינום המינימלי של $A$. \\*
מטענה 9.12.1 נובע כי $B = A - \lambda I$ מטריצה נילפוטנטית, ומטענה 9.11.5 אנו מסיקים כי אינדקס הנילפוטנטיות של $B$ הוא $3$. \\*
על־פי טענה 11.8.1 מספר מטריצות הז'ורדן היסודיות המופיעות בצורת ז'ורדן של $A$ הוא $5$, וגם כי מטריצת ז'ורדן היסודית הגדולה ביותר היא מסדר $3$.
נוכל להסיק מכל האמור לעיל כי $B$ דומה למטריצת ז'ורדן הבאה
\[
	\begin{pmatrix}
		 0 & 1 & 0 & 0 & 0 & 0 & 0 \\
		 0 & 0 & 1 & 0 & 0 & 0 & 0 \\
		 0 & 0 & 0 & 0 & 0 & 0 & 0 \\
		 0 & 0 & 0 & 0 & 0 & 0 & 0 \\
		 0 & 0 & 0 & 0 & 0 & 0 & 0 \\
		 0 & 0 & 0 & 0 & 0 & 0 & 0 \\
		 0 & 0 & 0 & 0 & 0 & 0 & 0 \\
	\end{pmatrix}
\]
ובהתאם ממשפט 11.9.2 נובע כי $A$ דומה למטריצת ז'ורדן הבאה
\[
	J = \begin{pmatrix}
		 \lambda & 1 & 0 & 0 & 0 & 0 & 0 \\
		 0 & \lambda & 1 & 0 & 0 & 0 & 0 \\
		 0 & 0 & \lambda & 0 & 0 & 0 & 0 \\
		 0 & 0 & 0 & \lambda & 0 & 0 & 0 \\
		 0 & 0 & 0 & 0 & \lambda & 0 & 0 \\
		 0 & 0 & 0 & 0 & 0 & \lambda & 0 \\
		 0 & 0 & 0 & 0 & 0 & 0 & \lambda \\
	\end{pmatrix}
\]
על־פי טענה 9.12.1 על־כן נובע גם כי $M_A(x) = {(x - \lambda)}^3$.

\section{שאלה 5}
תהי $A \in M_3(\CC)$ בעלת ערכים עצמיים ממשיים בלבד. \\*
ידוע כי צורת ז'ורדן של $A^3$ היא
\[
	\begin{pmatrix}
		8 & 1 & 0 \\
		0 & 8 & 0 \\
		0 & 0 & 1
	\end{pmatrix}
\]
נמצא את הפולינום האופייני ואת הפולינום המינימלי של $A$ ונציג צורת ז'ורדן שלה. \\*
חישוב ישיר מניב כי הפולינום האופייני של $A^3$ הוא $p_3(t) = {(t - 8)}^2 (t - 1)$. \\*
על־פי שאלה 11.3.2 מהקורס הקודם והעובדה כי כלל ערכיה העצמיים של $A$ הם ממשיים נובע כי ערכיה העצמיים של $A$ הם $1, 2$. \\*
על־ידי שימוש בדרך חישוב החזקה בשאלה 1 סעיף ב' ושילוב זהויות דטרמיננטה, אנו יכולים להסיק כי הפולינום האופייני של $A$ הוא
\[
	p(t) = {(t - 2)}^2 (t - 1)
\]
מטענה 9.8.8 א' נובע כי הפולינום המינימלי של $A$ הוא ${(t - 2)}^2(t-1)$ או $(t - 2)(t - 1)$. \\*
אילו הוא היה $(t - 2)(t - 1)$ אז מצב זה היה אפשרי על־פי טענה 9.8.2 רק אם $A$ הייתה לכסינה, אך במצב זה $A^3$ הייתה לכסינה אף היא, בסתירה לנתון. לכן
\[
	M_A(t) = {(t - 2)}^2 (t - 1) 
\]
לבסוף נובע ממשפט 11.10.1 כי למטריצה $A$ צורת ז'ורדן ומהפולינום המינימלי אנו יכולים להסיק כי מטריצת ז'ורדן זו היא
\[
	\begin{pmatrix}
		2 & 1 & 0 \\
		0 & 2 & 1 \\
		0 & 0 & 1
	\end{pmatrix}
\]

\end{document}
