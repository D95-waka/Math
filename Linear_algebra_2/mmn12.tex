\documentclass[a4paper]{article}

% packages
\usepackage{inputenc, fontspec, amsmath, amsthm, amsfonts, polyglossia, catchfile}
\usepackage[a4paper, margin=50pt, includeheadfoot]{geometry} % set page margins

% style
\AddToHook{cmd/section/before}{\clearpage}	% Add line break before section
\linespread{1.5}
\setcounter{secnumdepth}{0}		% Remove default number tags from sections
\setmainfont{Libertinus Serif}
\setsansfont{Libertinus Sans}
\setmonofont{Libertinus Mono}
\setdefaultlanguage{hebrew}
\setotherlanguage{english}

% operators
\DeclareMathOperator\cis{cis}
\DeclareMathOperator\Sp{Sp}
\DeclareMathOperator\tr{tr}
\DeclareMathOperator\im{Im}
\DeclareMathOperator\diag{diag}
\DeclareMathOperator*\lowlim{\underline{lim}}
\DeclareMathOperator*\uplim{\overline{lim}}

% commands
\renewcommand\qedsymbol{\textbf{משל}}
\newcommand{\NN}[0]{\mathbb{N}}
\newcommand{\ZZ}[0]{\mathbb{Z}}
\newcommand{\QQ}[0]{\mathbb{Q}}
\newcommand{\RR}[0]{\mathbb{R}}
\newcommand{\CC}[0]{\mathbb{C}}
\newcommand{\getenv}[2][] {
  \CatchFileEdef{\temp}{"|kpsewhich --var-value #2"}{\endlinechar=-1}
  \if\relax\detokenize{#1}\relax\temp\else\let#1\temp\fi
}
\newcommand{\explain}[2] {
	\begin{flalign*}
		 && \text{#2} && \text{#1}
	\end{flalign*}
}

% headers
\getenv[\AUTHOR]{AUTHOR}
\author{\AUTHOR}
\date\today


\title{פתרון ממ"ן 12 – אלגברה לינארית 2 (20229)}

\begin{document}
\maketitle

\section{שאלה 1}
\subsection{סעיף א'}
נגדיר
\[
	A_1 = \begin{pmatrix}
		0 & i \\
		-i & 0
	\end{pmatrix},
	A_2 = \begin{pmatrix}
		i & 2i \\
		0 & 1
	\end{pmatrix},
	A_3 = \begin{pmatrix}
		1 & i \\
		1 & 2 + i
	\end{pmatrix}
\]
נבדוק אם כל אחת מן המטריצות נורמליות ואם כן נמצא מטריצה אוניטרית המלכסנת אותן. \\*
נבדוק אם $A_1$ נורמלית:
\[
	A_1^* = \begin{pmatrix}
		0 & i \\
		-i & 0
	\end{pmatrix} = A_1
\]
לכן בהכרח $A_1 A_1^* = A_1^* A_1$ וכן $A_1$ נורמלית.
נחשב את הפולינום האופייני של $A_1$:
\[
	p(t) = \begin{vmatrix}
		t & -i \\
		i & t
	\end{vmatrix}
	= t^2 - (-i)i = t^2 - 1 = (t - 1)(t + 1)
\]
אז למטריצה שני ערכים עצמיים $-1, 1$.
נמצא את $V_{-1}$ על־ידי פתרון המערכת:
\[
	A_1 u = -u \rightarrow (A_1 + I) u = 0 \rightarrow \begin{pmatrix}
		1 & i \\
		-i & 1
	\end{pmatrix}
	\xrightarrow{R_2 \to R_2 + iR_1}
	\begin{pmatrix}
		1 & i \\
		0 & 0
	\end{pmatrix}
	\rightarrow
	V_{-1} = \Sp\left\{ \begin{pmatrix} -i \\ 1 \end{pmatrix} \right\}
\]
לכן בבסיס האורתוגונלי של $V_{-1}$ ישנו רק הווקטור $\begin{pmatrix} \frac{-i}{\sqrt{2}} \\ \frac{1}{\sqrt{2}} \end{pmatrix}$. \\*
נמצא את המרחב $V_1$ באופן דומה:
\[
	(A_1 - I) = 0
	\rightarrow
	\begin{pmatrix}
		-1 & i \\
		-i & -1
	\end{pmatrix}
	\xrightarrow[R_2 \to i R_2, R_2 \to R_2 - R_1]{R_1 \to -R_1}
	\begin{pmatrix}
		1 & -i \\
		0 & 0
	\end{pmatrix}
	\rightarrow
	V_1 = \Sp\left\{ \begin{pmatrix} i \\ 1 \end{pmatrix}\right\}
\]
בבסיס האורתוגונלי של $V_{-1}$ ישנו הווקטור $\begin{pmatrix} \frac{i}{\sqrt{2}} \\ \frac{1}{\sqrt{2}} \end{pmatrix}$. \\*
בשל כך המטריצה $P$ המוגדרת להלן מלכסנת אוניטרית את $A_1$:
\[
	P = \begin{pmatrix}
		\frac{-i}{\sqrt{2}} & \frac{i}{\sqrt{2}} \\
		\frac{1}{\sqrt{2}} & \frac{1}{\sqrt{2}}
	\end{pmatrix}
\]
נחשב את $A_2^*$:
\[
	A_2^* = \begin{pmatrix}
		-i & 0 \\
		-2i & 1
	\end{pmatrix}
\]
נבדוק אם $A_2$ נורמלית:
\begin{align*}
	A_2 A_2^* = 
	\begin{pmatrix}
		i & 2i \\
		0 & 1
	\end{pmatrix} \begin{pmatrix}
		-i & 0 \\
		-2i & 1
	\end{pmatrix}
	= \begin{pmatrix}
		3 & 2i \\
		-2i & 1
	\end{pmatrix} \\
	A_2^* A_2 = 
	\begin{pmatrix}
		-i & 0 \\
		-2i & 1
	\end{pmatrix}
	\begin{pmatrix}
		i & 2i \\
		0 & 1
	\end{pmatrix}
	= \begin{pmatrix}
		1 & 2 \\
		2 & 5
	\end{pmatrix} \\
\end{align*}
אנו רואים כי $A_2 A_2^* \ne A_2^* A_2$ ולכן $A_2$ איננה נורמלית. \\*
נבדוק אם $A_3$ נורמלית, תחילה נבחין כי
\[
	A_3^* = \begin{pmatrix}
		1 & 1 \\
		-i & 2 - i
	\end{pmatrix}
\]
נבדוק אם היא נורמלית:
\begin{align*}
	A_3 A_3^* = 
	\begin{pmatrix}
		1 & i \\
		1 & 2 + i
	\end{pmatrix}
	\begin{pmatrix}
		1 & 1 \\
		-i & 2 - i
	\end{pmatrix}
	= \begin{pmatrix}
		2 & 2 + 2i \\
		2 - 2i & 6
	\end{pmatrix} \\
	A_3^* A_3 = 
	\begin{pmatrix}
		1 & 1 \\
		-i & 2 - i
	\end{pmatrix}
	\begin{pmatrix}
		1 & i \\
		1 & 2 + i
	\end{pmatrix}
	= \begin{pmatrix}
		2 & 2 + 2i \\
		2 - 2i & 6
	\end{pmatrix}
\end{align*}
אנו רואים כי המטריצה $A_3$ אכן נורמלית. \\*
נחשב את הפולינום האופייני של $A_3$:
\[
	p(t) = \begin{vmatrix}
		t - 1 & -i \\
		-1 & t - 2 - i
	\end{vmatrix}
	= (t - 1)(t - 2 - i) + i
	= t^2 + (-3 - i)t + 2
	\rightarrow
	t = \frac{3 + i \pm \sqrt{8 + 6i - 8}}{2}
	= \frac{3 + i \pm (\sqrt{3} + \sqrt{3}i)}{2}
\]
נשתמש בחישוב שמבוצע בתשובה 3.2.2 עבור המטריצה ולכן המטריצה האוניטרית היא
\[
	\begin{pmatrix}
		\frac{1}{\sqrt{3 + \sqrt{3}}} & \frac{1}{\sqrt{3 - \sqrt{3}}} \\
		\frac{1 + \sqrt{3}}{2\sqrt{3 + \sqrt{3}}}(1 - i) & \frac{1 - \sqrt{3}}{2\sqrt{3 - \sqrt{3}}}(1 - i)
	\end{pmatrix}
\]

\subsection{סעיף ב'}
נמצא אילו מבין המטריצות המוגדרות להלן חיוביות. \\*
נגדיר
\[
	C_1 = \begin{pmatrix}
		1 & 1 \\
		1 & 1
	\end{pmatrix}
\]
המטריצה ממשית וסימטרית, ולכן גם נורמלית. נחשב את ערכיה העצמיים:
\[
	p(t) = (t - 1)^2 - 1 = t^2 - 2t = t(t - 2)
\]
לכן ערכיה העצמיים הם $0, 2$ והיא חיובית אך לא חיובית לחלוטין. \\*
נגדיר
\[
	C_2 = \begin{pmatrix}
		0 & i \\
		-i & 0
	\end{pmatrix}
\]
קל לראות כי המטריצה צמודה לעצמה, נחשב את ערכיה העצמיים:
\[
	p(t) = t^2 - 1 = (t - 1)(t + 1)
\]
למטריצה ערך עצמי שלילי ולכן היא לא חיובית כלל לפי משפט 3.2.2.
נגדיר
\[
	C_3 = \begin{pmatrix}
		0 & 1 \\
		-1 & 0
	\end{pmatrix}
\]
מטריצה זו לא צמודה לעצמה, ולכן לא יכולה להיות מטריצה חיובית כלל. \\*
נגדיר
\[
	C_4 = \begin{pmatrix}
		1 & 1 \\
		0 & 1
	\end{pmatrix}
\]
גם מטריצה זו לא סימטרית ולכן לא עומדת בהגדרה 1.2.5 כלל. \\*
נגדיר
\[
	C_5 = \begin{pmatrix}
		2 & 1 \\
		1 & 2
	\end{pmatrix}
\]
מטריצה זו סימטרית, נחשב את ערכיה העצמיים:
\[
	p(t) = (t - 2)^2 - 1 = t^2 - 4t + 4 - 1 = t^2 - 4t + 3 = (t - 3)(t - 1)
\]
לכן לפי משפט 3.2.2 המטריצה חיובית לחלוטין. \\*
נגדיר
\[
	C_6 = \begin{pmatrix}
		1 & 2 \\
		3 & 1
	\end{pmatrix}
\]
גם מטריצה זו לא סימטרית ולכן לא חיובית.

\section{שאלה 2}
יהי $V$ מרחב מכפלה פנימית מממד סופי ותהי $T: V \to V$ העתקה לינארית נורמלית.
\subsection{i}
נוכיח כי $\ker T = \ker T^*$. \\*
יהי $u \in \ker T$, אז מתקיים $T u = 0$ ובשל כך גם $\lVert u \rVert = 0$. \\*
על־פי הגדרת הנורמה מתקיים גם $(T u, T u) = 0$. אז
\begin{align*}
	\lVert T u \rVert & = (T u, T u) \\
					  & = (u, T^* T u) && \text{מהגדרת הצמוד} \\
					  & = (u, T T^* u) && \text{ידוע כי $T$ נורמלית} \\
					  & = (T^* u, T^* u) && \text{שוב על־פי צמוד} \\
					  & = \lVert T^* u \rVert = 0 && \text{הגדרת הנורמה} \\
					  & \rightarrow T^* u = 0 \rightarrow u \in \ker T^* && \text{נורמה חיובית לחלוטין}
\end{align*}
בשל הגדרת המכפלה הפנימית נובע כי $T^* u = 0$. \\*
נשים לב כי בשל סימטריות המטריצות הנורמליות נוכל להוכיח באופן דומה גם כי אם $T^* u = 0$ אז $T u = 0$,
ולכן מתקיים $\ker T = \ker T^*$.

\subsection{ii}
נוכיח כי $\im T = (\ker T)^\perp$. \\*
על־פי משפט 3.2.1 ההעתקה $T$ היא לכסינה אוניטרית בפרט ולכן לכסינה בכלל.
מסיבה זו כל וקטור ב־$V$ הוא וקטור עצמי לאיזשהו ערך עצמי ב־$T$. \\*
יהיו $u \in \im T, v \in \ker T$. נשים לב כי $v$ הוא וקטור עצמי של $0$, ואילו $u$ וקטור עצמי לערך עצמי $\lambda \ne 0$. \\*
בשל היותם ערכים עצמיים שונים, לפי משפט 3.2.6 הוקטורים אורתוגונליים זה לזה. \\*
נכליל את הטענה הזו ונראה שלכל $u$ כזה התנאי מתקיים לכל $v$, לכן $u \in (\ker T)^\perp$. \\*
באופן דומה נוכל להכליל את הטענה על התמונה, ולכן $\im T = (\ker T)^\perp$.

\subsection{iii}
נוכיח כי $\im T = \im T^*$. \\*
על־פי סעיף ii מתקיים
\[
	\im T^* = (\ker T^*)^\perp
\]
וידוע כי $\ker T = \ker T^*$ על־פי סעיף i, לכן
\[
	\im T^* = (\ker T)^\perp = \im T
\]

\section{שאלה 3}
יהי $V$ מרחב מכפלה עצמית מממד סופי ו־$T: V \to V$ העתקה לינארית המקיימת
\[
	T^2 = \frac{1}{2}(T + T^*)
\]
נוכיח כי $T$ נורמלית ומתקיים $T^2 = T$.
\[
	T T^* = T (2T^2 - T) = T (2T - I) T = (2T^2 - T) T = T^* T
\]
אנו רואים כי $T$ נורמלית. \\*
נוכיח גם כי $T^2 = T$. \\*
יהי $\lambda$ ערך עצמי של $T$ ו־$u$ וקטור עצמי שלו. \\*
לפי למה 3.2.5 $\overline{\lambda}$ ערך עצמי של $T^*$ ו־$u$ וקטור עצמי שלו,
דהינו $T u = \lambda u, T^* u = \overline{\lambda} u$. \\*
עוד אנו יודעים כי $T^2 u = T \lambda u = \lambda^2 u$. מתקיים
\[
	T^2 u = \lambda^2 u = \frac{1}{2}(\lambda u + \overline{\lambda} u) = \frac{1}{2}(T u + T^* u)
\]
אז נוכל להניח גם כי
\[
	2 \lambda^2 = \lambda + \overline{\lambda}
\]
נניח כי המרחב מוגדר מעל $\CC$, כך שכל מרחב מכפלה פנימית הלכה למעשה מוכל בו.
נגדיר $\lambda = a + bi$, אז
\[
	2a^2 + 4abi - 2b^2 = a + bi + a - bi
\]
דהינו מתקיים
\[
	\begin{cases}
		2ab = 0 \\
		2a^2 - 2b^2 = 2a
	\end{cases}
\]
אילו $b \ne 0$, אז נובע מהמשוואה הראשונה כי $a = 0$ ומהשמוואה השנייה כי $b = 0$ בסתירה לטענה, אז מהמשוואה הראשונה אנו למדים כי $a \ne 0$.
לכן גם $b = 0$ ומהמשוואה השנייה נובע ש־$a(a - 1) = 0$, ידוע כי $a \ne 0$ ולכן $a = 1$.
מצאנו כי כלל הערכים העצמיים של $T, T^2, T^*$ הם $1$ בלבד, לכן לכל $v \in V$ מתקיים $T v = T^2 v = v$ ובהתאם $T^2 = T$.

\section{שאלה 4}
תהי $H$ מטריצה סימטרית ממשית מסדר $n \times n$ ויהי $\lambda$ הערך העצמי המקסימלי של $H$. \\*
נוכיח כי לכל $v \in \RR^n$ כאשר $\lVert v \rVert = 1$ מתקיים $v^t H v \le \lambda$. \\*
על־פי משפט 3.2.1 המטריצה $H$ לכסינה, ולכן בכלל כל וקטור הוא ערך עצמי לערך עצמי כלשהו. \\*
יהי $\mu$ ערך עצמי של $H$ כך ש־$v \in V_\mu$ , אז מתקיים $\lambda \ge \mu$ ובנוסף $H v = \mu v$.
\[
	v^t H v = \mu v^t v = \mu \lVert v \rVert^2 \mu = \mu \le \lambda \rightarrow v^t H v \le \lambda
\]

\section{שאלה 5}
נוכיח כי המטריצה $A$ המוגדרת להלן היא נורמלית.
\[
	A = \begin{pmatrix}
		2 - i & -1 & 0 \\
		-1 & 1 - i & 1 \\
		0 & 1 & 2 - i
	\end{pmatrix}
\]
לפי הגדרת המשלים
\[
	A^* = \begin{pmatrix}
		2 + i & -1 & 0 \\
		-1 & 1 + i & 1 \\
		0 & 1 & 2 + i
	\end{pmatrix}
\]
נחשב
\begin{align*}
	A A^* = 
	\begin{pmatrix}
		2 - i & -1 & 0 \\
		-1 & 1 - i & 1 \\
		0 & 1 & 2 - i
	\end{pmatrix}
	\begin{pmatrix}
		2 + i & -1 & 0 \\
		-1 & 1 + i & 1 \\
		0 & 1 & 2 + i
	\end{pmatrix}
	= \begin{pmatrix}
		6 & -3 & -1 \\
		-3 & 4 & 3 \\
		-1 & 3 & 6
	\end{pmatrix} \\
	A^* A = 
	\begin{pmatrix}
		2 + i & -1 & 0 \\
		-1 & 1 + i & 1 \\
		0 & 1 & 2 + i
	\end{pmatrix}
	\begin{pmatrix}
		2 - i & -1 & 0 \\
		-1 & 1 - i & 1 \\
		0 & 1 & 2 - i
	\end{pmatrix}
	= \begin{pmatrix}
		6 & -3 & -1 \\
		-3 & 4 & 3 \\
		-1 & 3 & 6
	\end{pmatrix}
\end{align*}
על־פי החיושב $A A^* = A^* A$ ולכן המטריצה נורמלית. \\*
נמצא את הפירוק
\[
	A = \sum_{i} \lambda_i P_i
\]
כאשר $P_i$ הן מטריצות של ההטלות האורתוגונליות שמופיעות בפירוק הספקטראלי של $T_A$. \\*
נחשב את הפולינום האופייני של $A$:
\begin{align*}
	p(t) 
	& = 
	\begin{vmatrix}
		t - 2 + i & 1 & 0 \\
		1 & t - 1 + i & -1 \\
		0 & -1 & t - 2 + i
	\end{vmatrix}
	\xrightarrow{R_3 \to R_3 + R_1} 
	\begin{vmatrix}
		t - 2 + i & 1 & 0 \\
		1 & t - 1 + i & -1 \\
		t - 2 + i & 0 & t - 2 + i
	\end{vmatrix} \\
	& = 
	(t - 2 + i)
	\begin{vmatrix}
		t - 2 + i & 1 & 0 \\
		1 & t - 1 + i & -1 \\
		1 & 0 & 1
	\end{vmatrix}
	\xrightarrow{R_2 \to R_2 + R_3}
	(t - 2 + i)
	\begin{vmatrix}
		t - 2 + i & 1 & 0 \\
		2 & t - 1 + i & 0 \\
		1 & 0 & 1
	\end{vmatrix} \\
	& =
	(t - 2 + i)
	\begin{vmatrix}
		t - 2 + i & 1 \\
		2 & t - 1 + i \\
	\end{vmatrix}
	= (t - 2 + i) ((t - 2 + i) (t - 1 + i) - 2) \\
	& = (t - 2 + i)(t^2 + (-3 + 2i) t + (-1 - 3i))
	= (t - 2 + i)(t^2 + (-1 + 2i) t + 1 - 3i) \\
	& = (t - 2 + i)(t + i)(t - 3 + i)
\end{align*}
אז הערכים העצמיים של $T_A$ הם $2 - i, -i, 3 - i$. נחשב את המרחב העצמי שלהם על־ידי פתרון מערכת המשוואות $(A - \lambda I)u = 0$:
\begin{align*}
	& (A - (2 - i) I) u = 0
	\rightarrow
	\begin{pmatrix}
		0 & -1 & 0 \\
		-1 & -1 & 1 \\
		0 & 1 & 0
	\end{pmatrix}
	\xrightarrow[R_3 \to R_3 + R_1]{R_2 \to -R_2 + R_1}
	\begin{pmatrix}
		0 & -1 & 0 \\
		1 & 0 & -1 \\
		0 & 0 & 0
	\end{pmatrix} \\
	& \rightarrow
	V_{2 -i} = \Sp \{ (1, 0, 1) \}, v_1 = (1, 0, 1) \\
	& (A + i I) u = 0
	\rightarrow
	\begin{pmatrix}
		2 & -1 & 0 \\
		-1 & 1 & 1 \\
		0 & 1 & 2
	\end{pmatrix}
	\xrightarrow[R_1 \to -R_1]{R_2 \leftrightarrow R_1}
	\begin{pmatrix}
		1 & -1 & -1 \\
		2 & -1 & 0 \\
		0 & 1 & 2
	\end{pmatrix}
	\xrightarrow[R_3 \to R_3 - R_2 + 2R_1]{R_2 \to R_2 - 2R_1}
	\begin{pmatrix}
		1 & -1 & -1 \\
		0 & 1 & 2 \\
		0 & 0 & 0
	\end{pmatrix} \\
	& \rightarrow
	V_{-i} = \Sp\{ (1, 2, -1)^t \}, v_2 = (1, 2, -1) \\
	& (A - (3 - i) I) u = 0
	\rightarrow
	\begin{pmatrix}
		-1 & -1 & 0 \\
		-1 & -2 & 1 \\
		0 & 1 & -1
	\end{pmatrix}
	\xrightarrow[R_2 \to R_2 + R_1]{R_1 \to - R_1}
	\begin{pmatrix}
		1 & 1 & 0 \\
		0 & -1 & 1 \\
		0 & 1 & -1
	\end{pmatrix}
	\rightarrow
	\begin{pmatrix}
		1 & 1 & 0 \\
		0 & 1 & -1 \\
		0 & 0 & 0
	\end{pmatrix} \\
	& \rightarrow V_{3 - i} = \Sp\{ (1, -1, -1)^t \}, v_3 = (1, -1, -1)
\end{align*}
נגדיר $B = (v_1, v_2, v_3)$ בסיס אורתוגונלי ו־$E$ הבסיס הסטנדרטי, אז מטריצת המעבר מ־$E$ ל־$B$ היא
\[
	M = \begin{pmatrix}
		1 & 1 & 1 \\
		0 & 2 & -1 \\
		1 & -1 & -1
	\end{pmatrix}
\]
לפי חישוב גם
\[
	M^{-1} = \begin{pmatrix}
		\frac{1}{2} & 0 & \frac{1}{2} \\
		-\frac{1}{2} & 1 & \frac{1}{2} \\
		1 & -1 & -1
	\end{pmatrix}
\]
נגדיר $P_i$ ההטלה האורתוגונלית על האיבר ה־$i$, אז לפי הקורס הקודם
\[
	[P_i]_E = M [P_i]_B M^{-1}
\]
\begin{align*}
	& [P_1]_E = 
	\begin{pmatrix}
		1 & 1 & 1 \\
		0 & 2 & -1 \\
		1 & -1 & -1
	\end{pmatrix}
	\begin{pmatrix}
		1 & 0 & 0 \\ 
		0 & 0 & 0 \\ 
		0 & 0 & 0
	\end{pmatrix}
	\begin{pmatrix}
		\frac{1}{2} & 0 & \frac{1}{2} \\
		-\frac{1}{2} & 1 & \frac{1}{2} \\
		1 & -1 & -1
	\end{pmatrix}
	= \begin{pmatrix}
		\frac{1}{2} & 0 & 0 \\
		0 & 0 & 0 \\
		1 & 0 & 0
	\end{pmatrix} \\
	& [P_2]_E = 
	\begin{pmatrix}
		1 & 1 & 1 \\
		0 & 2 & -1 \\
		1 & -1 & -1
	\end{pmatrix}
	\begin{pmatrix}
		0 & 0 & 0 \\ 
		0 & 1 & 0 \\ 
		0 & 0 & 0
	\end{pmatrix}
	\begin{pmatrix}
		\frac{1}{2} & 0 & \frac{1}{2} \\
		-\frac{1}{2} & 1 & \frac{1}{2} \\
		1 & -1 & -1
	\end{pmatrix}
	= \begin{pmatrix}
		0 & 0 & 0 \\
		0 & 2 & 0 \\
		0 & 1 & 0
	\end{pmatrix} \\
	& [P_3]_E = 
	\begin{pmatrix}
		1 & 1 & 1 \\
		0 & 2 & -1 \\
		1 & -1 & -1
	\end{pmatrix}
	\begin{pmatrix}
		0 & 0 & 0 \\ 
		0 & 0 & 0 \\ 
		0 & 0 & 1
	\end{pmatrix}
	\begin{pmatrix}
		\frac{1}{2} & 0 & \frac{1}{2} \\
		-\frac{1}{2} & 1 & \frac{1}{2} \\
		1 & -1 & -1
	\end{pmatrix}
	= \begin{pmatrix}
		0 & 0 & \frac{1}{2} \\
		0 & 0 & \frac{1}{2} \\
		0 & 0 & 1
	\end{pmatrix} \\
\end{align*}
ולכן $A = (2 - i) [P_1]_E - i [P_2]_E + (3 - i) [P_3]_E$

\end{document}
