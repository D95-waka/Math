\documentclass[a4paper]{article}

% packages
\usepackage{inputenc, fontspec, amsmath, amsthm, amsfonts, polyglossia, catchfile}
\usepackage[a4paper, margin=50pt, includeheadfoot]{geometry} % set page margins

% style
\AddToHook{cmd/section/before}{\clearpage}	% Add line break before section
\linespread{1.5}
\setcounter{secnumdepth}{0}		% Remove default number tags from sections
\setmainfont{Libertinus Serif}
\setsansfont{Libertinus Sans}
\setmonofont{Libertinus Mono}
\setdefaultlanguage{hebrew}
\setotherlanguage{english}

% operators
\DeclareMathOperator\cis{cis}
\DeclareMathOperator\Sp{Sp}
\DeclareMathOperator\tr{tr}
\DeclareMathOperator\im{Im}
\DeclareMathOperator\diag{diag}
\DeclareMathOperator*\lowlim{\underline{lim}}
\DeclareMathOperator*\uplim{\overline{lim}}

% commands
\renewcommand\qedsymbol{\textbf{משל}}
\newcommand{\NN}[0]{\mathbb{N}}
\newcommand{\ZZ}[0]{\mathbb{Z}}
\newcommand{\QQ}[0]{\mathbb{Q}}
\newcommand{\RR}[0]{\mathbb{R}}
\newcommand{\CC}[0]{\mathbb{C}}
\newcommand{\getenv}[2][] {
  \CatchFileEdef{\temp}{"|kpsewhich --var-value #2"}{\endlinechar=-1}
  \if\relax\detokenize{#1}\relax\temp\else\let#1\temp\fi
}
\newcommand{\explain}[2] {
	\begin{flalign*}
		 && \text{#2} && \text{#1}
	\end{flalign*}
}

% headers
\getenv[\AUTHOR]{AUTHOR}
\author{\AUTHOR}
\date\today


\title{פתרון ממ"ן 13 – אלגברה לינארית 2 (20229)}

\begin{document}
\maketitle

\section{שאלה 1}
יהי $V = M_{n \times n}^\RR$ ותהי $f: V \times V \to \RR$ אשר מוגדרת $f(A, B) = \tr(A^t M B)$ לכל $A, B \in V$.

\subsection{סעיף א'}
נמצא תנאי מספיק והכרחי על $M$ כך ש־$f$ תהיה תבנית סימטרית. \\*
על־פי שאלה 4.1.2 ההעתקה $f$ היא בילינארית. נראה כי
\[
	f(A, B) = \tr(A^t M B) = \tr( (A^t M B)^t ) = \tr( (M B)^t A) = \tr( B^t M^t A)
\]
אילו הייתה $M$ סימטרית אז $M = M^t$ ובהתאם גם
\[
	\tr(B^t M^t A) = \tr( B^t M A) = f(A, B) = f(B, A)
\]
דהינו, תנאי מספק והכרחי כדי ש־$f$ תהיה תבנית סימטרית הוא ש־$M$ תהיה מטריצה סימטרית.

\subsection{סעיף ב'}
נגדיר $n = 2$, $E$ הבסיס הסטנדרטי של $V$,
\[
	M = \begin{pmatrix}
		1 & 2 \\
		3 & 5
	\end{pmatrix}
\]
נמצא את $[f]_E$: \\*
על־פי שאלה 4.1.13 מתקיים
\[
	[f]_E = \begin{pmatrix}
		1 & 0 & 2 & 0 \\
		0 & 1 & 0 & 2 \\
		3 & 0 & 5 & 0 \\
		0 & 3 & 0 & 5 \\
	\end{pmatrix}
\]

\subsection{סעיף ג'}
נמצא הצגה של $f$ כסכום של תבנית בילינארית סימטרית ותבנית בילינארית אנטיסימטרית עבור הנתונים אשר הוגדרו בסעיף הקודם. \\*
נגדיר
\[
	g_1(A, B) = \frac{1}{2} (f(A, B) + f(B, A)), g_2(B, A) = \frac{1}{2}(f(A, B) - f(B, A))
\]
על־פי שאלה 4.2.3 $g_1, g_2$ הן תבניות בילינאריות סימטרית ואנטיסימטרית בהתאמה, ומחישוב ישיר מתקבל כי
\[
	f = g_1 + g_2
\]

\section{שאלה 2}
נוכיח כי תבנית בילינארית $f \ne 0$ ניתנת להצגה כמכפלה של שתי תבניות לינאריות:
\[
	f(x, y) = \left( \sum_{i = 1}^n b_i x_i \right) \left(\sum_{j = 1}^n c_j y_j \right)
\]
אם ורק אם הדרגה של $f$ היא $1$. \\*
נניח כי $f$ ניתנת להצגה כמכפלת שתי תבניות לינאריות כפי כמתואר לעיל בבסיס נתון ונוכיח כי דרגת $\rho f = 1$. \\*
נגדיר את הבסיס $W = (w_1, w_2, \hdots, w_n)$, אז המטריצה המייצגת את $f$ היא
\[
	[f]_W = \begin{pmatrix}
		b_1 c_1 & b_1 c_2 & \cdots & b_1 c_n \\
		b_2 c_1 & b_2 c_2 & \cdots & b_2 c_n \\
		\vdots & & \ddots & \vdots \\
		b_n c_1 & b_n c_2 & \cdots & b_n c_n \\
	\end{pmatrix}
\]
ניתן לראות כי כלל השורות תלויות לינארית, ואנו יודעים כי חייבת להיות לפחות שורה אחת שונה מאפס (שאם לא כן $f = 0$) ולכן $\rho f = 1$. \\*
נניח כי $\rho f = 1$ ונוכיח כי קיימות שתי תבניות לינאריות עבורן $f$ היא מכפלתן. \\*
מדרגה זו נובע כי קיים בסיס $W$ עבורו $[f]_W$ היא מטריצה בה כלל השורות תלויות לינארית בווקטור יחיד
נגדיר $b$ להיות הווקטור היחיד, ו־$c$ סדרת סקלרים אשר מהווים המקדמים של הווקטור במטריצת הייצוג
\[
	[f]_W = \begin{pmatrix}
		b_1 c_1 & \cdots & b_1 c_n \\
		\vdots & \ddots & \vdots \\
		b_n c_1 & \cdots & b_n c_n \\
	\end{pmatrix}
\]
ובהתאם לפי מסקנה 4.1.6 מתקיים
\[
	f(x, y)
	= [x]_W [f]_W [y]_W
	= \sum_{i = 1}^n \sum_{j = 1}^n b_i x_i c_j y_j
	= \left( \sum_{i = 1}^n b_i x_i \right) \left(\sum_{j = 1}^n c_j y_j \right)
\]

\section{שאלה 3}
תהי תבנית על $\RR^2$ המוגדרת על־ידי:
\[
	f\left( (x_1, x_2), (y_1, y_2) \right) = x_1 y_1 + 4 x_2 y_2 + 2 x_1 y_2 + 2x_2 y_1
\]

\subsection{סעיף א'}
נוכיח כי $f$ תבנית בילינארית. \\*
מטענה 4.1.4 נובע ש־$f$ תבנית בילינארית, ונבנה לה מטריצת ייצוג:
\[
	A = \begin{pmatrix}
		1 & 2 \\
		2 & 4
	\end{pmatrix}
\]
מלמה 4.2.2 נובע ש־$f$ תבנית סימטרית.
נמצא בסיס בו $f$ מיוצגת על־ידי מטריצה אלכסונית. \\*
התבנית הריבועית המסומכת ל־$f$ היא
\[
	q(x) = f(x, x) = f((x_1, x_2), (x_1, x_2)) = x_1^2 + 4x_1 x_2 + 4 x_2^2 = (x_1 + 2x_2)^2
\]
נגדיר משתנה חדש $z \in \RR^2$, על־פי ערך $q$ נגדיר $z_1 = x_1 + 2x_2$, ולכן בהתאם $q(z) = z_1^2 + 0z_2$.
נבחר $z_2 = x_2$ ונראה כי
\[
	f(z, z') = z_1 z'_1
\]
מטריצת הייצוג לפי $z$ של $f$ היא
\[
	B = \begin{pmatrix}
		1 & 0 \\
		0 & 0
	\end{pmatrix}
\]
נגדיר $W$ הבסיס שמקיים $[f]_W = B$. אז מטריצת המעבר מהבסיס הסטנדרטי ל־$W$ היא לפי שיטת לגרנז':
\[
	M = \begin{pmatrix}
		1 & 2 \\
		0 & 1
	\end{pmatrix}
\]
מחישוב עולה כי $x_2 = z_2$ וכי $x_1 = z_1 - 2z_2$ ולכן בהתאם
\[
	M^{-1} = \begin{pmatrix}
		1 & -2 \\
		0 & 1
	\end{pmatrix}
\]
ובהתאם להגדרת מטריצת המעבר גם $W = ((1, 0), (-2, 1))$.

\subsection{סעיף ב'}
נבדוק את נכונות נוסחת המעבר מן הבסיס הסטנדרטי של $\RR^2$ ל־$W$. \\*
על־פי משפט 4.5.1 התבנית $f$ מקיימת
\[
	A = [f]_E = M^t [f]_W M = M^t B M^t
\]
נציב:
\[
	\begin{pmatrix}
		1 & 2 \\
		2 & 4
	\end{pmatrix}
	=
	\begin{pmatrix}
		1 & 0 \\
		2 & 1
	\end{pmatrix}
	\begin{pmatrix}
		1 & 0 \\
		0 & 0
	\end{pmatrix}
	\begin{pmatrix}
		1 & 2 \\
		0 & 1
	\end{pmatrix}
\]
שוויון זה אכן מתקיים בחישוב ישיר, ולכן בהתאם נוסחת המעבר על־ידי שימוש במטריצת המעבר $M$ אכן נכונה.

\section{שאלה 4}
\subsection{סעיף א'}
תהי תבנית ריבועית $q: \RR^n \to \RR$ כאשר $n \in \NN$ המוגדרת על־ידי
\[
	q(x_1, x_2, \ldots, x_n) = \sum_{i = 1}^n x_i^2 + \sum_{1 \le i < j \le n} x_i x_j
\]
נמצא ל־$q$ תבנית אלכסונית. \\*
נגדיר $[q]_E = A$ מטריצת ייצוג סימטרית לפי הבסיס הסטנדרטי של $\RR^n$, מהגדרת $q$ נובע:
\[
	A = \begin{pmatrix}
		1  & \frac{1}{2} & \frac{1}{2} & \hdots \\
		\frac{1}{2} & 1 & \frac{1}{2} & \hdots \\
		\frac{1}{2} & \frac{1}{2} & 1 & \hdots \\
		\vdots & & \ddots & \ddots
	\end{pmatrix}
\]
בשל היות המטריצה $A$ סימטרית אז אנו יודעים כי היא חופפת למטריצה אלכסונית, וחפיפה בממשיים מתלכדת עם דמיון אורתוגונלי, נחשב את ערכיה העצמיים של $A$:
\begin{align*}
	& \begin{vmatrix}
		t - 1  & -\frac{1}{2} & -\frac{1}{2} & \hdots \\
		-\frac{1}{2} & t - 1 & -\frac{1}{2} & \hdots \\
		-\frac{1}{2} & -\frac{1}{2} & t - 1 & \hdots \\
		\vdots & & \ddots & \ddots
	\end{vmatrix}
	\xrightarrow{R_1 \to R_1 + \sum_{i = 2}^n R_i}
	\begin{vmatrix}
		t - \frac{n + 1}{2} & t - \frac{n + 1}{2} & t - \frac{n + 1}{2} & \hdots \\
		-\frac{1}{2} & t - 1 & -\frac{1}{2} & \hdots \\
		-\frac{1}{2} & -\frac{1}{2} & t - 1 & \hdots \\
		\vdots & & \ddots & \ddots
	\end{vmatrix} \\
	& \rightarrow
	\left(t - \frac{n + 1}{2}\right)
	\begin{vmatrix}
		1 & 1 & 1 & \cdots \\
		-\frac{1}{2} & t - 1 & -\frac{1}{2} & \hdots \\
		-\frac{1}{2} & -\frac{1}{2} & t - 1 & \hdots \\
		\vdots & & \ddots & \ddots
	\end{vmatrix}
	\xrightarrow{R_i \to R_i + R_1/2 \mid 1 < i \le n}
	\left(t - \frac{n + 1}{2}\right)
	\begin{vmatrix}
		1 & 1 & 1 & \cdots \\
		0 & t - \frac{1}{2} & 0 & \hdots \\
		0 & 0 & t - \frac{1}{2} & \hdots \\
		\vdots & & \ddots & \ddots
	\end{vmatrix} \\
	& \rightarrow
	\left(t - \frac{n + 1}{2}\right)
	\left(t - \frac{1}{2}\right)^{n - 1}
\end{align*}
אז כלל ערכיה העצמיים הם $\frac{n + 1}{2}, \frac{1}{2}$.
נוכל בדרך דומה לתהליך מציאת הערך העצמי $\frac{n + 1}{2}$ להגיע למסקנה כי $(1, 1, \hdots)$ וקטור עצמי של הערך. \\*
נחשב את המרחב העצמי של $\frac{1}{2}$:
\[
	A x = \frac{1}{2} x
\]
ובהמרה למערכת משוואות הומוגנית
\[
	\begin{pmatrix}
		\frac{1}{2} & \frac{1}{2} & \frac{1}{2} & \hdots \\
		\frac{1}{2} & \frac{1}{2} & \frac{1}{2} & \hdots \\
		\frac{1}{2} & \frac{1}{2} & \frac{1}{2} & \hdots \\
		\vdots & & \ddots & \ddots
	\end{pmatrix}
	\rightarrow
	\begin{pmatrix}
		\frac{1}{2} & \frac{1}{2} & \frac{1}{2} & \hdots \\
		0 & 0 & 0 & \hdots \\
		0 & 0 & 0 & \hdots \\
		\vdots & & \ddots & \ddots
	\end{pmatrix}
	\rightarrow
	\Sp\{ (1, 0, 0, \hdots, -1), (0, 1, 0, \hdots, -1), \hdots, (0, 0, 0, \hdots, 1, -1) \}
\]
קל לבדוק כי מצאנו $n$ וקטורים בלתי תלויים, ולכן $A$ חופפת למטריצה האלכסונית
\[
	\begin{pmatrix}
		\frac{n + 1}{2}  & 0 & 0 & \hdots \\
		0 & \frac{1}{2} & 0 & \hdots \\
		0 & 0 & \frac{1}{2} & \hdots \\
		\vdots & & \ddots & \ddots
	\end{pmatrix}
\]
נמצא תבנית בילינארית סימטרית הקוטבית ל־$q$. \\*
למעשה, המטריצה $A$ כבר מייצגת העתקה כזו:
\[
	p(x, y) = \sum_{i = 1}^n x_i y_i + \sum_{j \ne i} \frac{1}{2} x_j y_j
\]

\subsection{סעיף ב'}
נמצא בסיס שבו התבנית הריבועית $q$ היא בעלת צורה אלכסונית. \\*
בסעיף הקודם מצאנו מטריצה אלכסונית כמו גם ערכים עצמיים ווקטורים עצמיים שלהם, של מטריצת הייצוג של התבנית,
כדי למצוא מטריצת לכסון אוניטרית נצטרך למצוא גם בסיס אורתונורמלי מתאים, אותו נוכל לבנות מהבסיס הקיים:
\[
	B = \left( (1, 1, \hdots), (1, 0, 0, \hdots, -1) \hdots \right)
\]
נגדיר $u_i$ הווקטור הנורמלי ה־$i$.
\[
	u_1 = \frac{b_1}{\lVert b_1 \rVert} = \frac{1}{\sqrt{n}} (1, 1, \hdots)
\]
נשים לב כי הווקטור הראשון אורתוגונלי לכל שאר הווקטורים, לכן כדי לנרמל את הבסיס נוכל לבצע הליך גרם־שמידט רק לווקטורים העצמיים של $\frac{1}{2}$.
\[
	u_2 = \frac{b_2}{\lVert b_2 \rVert} = \frac{1}{\sqrt{2}} \left(1, 0, 0, \hdots, -1\right)
	= \left(\frac{1}{\sqrt{2}}, 0, 0, \hdots, \frac{-1}{\sqrt{2}}\right)
\]
נשים לב כי מכפלת כל שני ווקטורים עצמיים שונים של $\frac{1}{2}$ היא $1$ בשל האיבר המשותף האחרון שלהם וחוסר התלות ללא התיחסות אליו.
\begin{align*}
	& v_3 = (0, 1, 0, \hdots, -1) - \frac{1}{\sqrt{2}} u_2 = (-\frac{1}{2}, 1, 0, \hdots, -\frac{1}{2}) \\
	& u_3 = \frac{v_3}{\lVert v_3 \rVert} = \frac{\sqrt{2}}{\sqrt{3}} u_2 = \frac{1}{\sqrt{6}} (-1, 2, 0, \hdots, -1)
\end{align*}
ניתן להוכיח באינדוקציה שהביטוי $(b_n, u_m) u_m$ כאשר $m < n$ שווה ל־$(0, \hdots, -1, 0, \hdots, 1)$. \\*
בהתאם, לכל $i$ כאשר $2 \le i \le n$ מתקיים $v_i = (-1, \hdots, i - 1, 0, \hdots, -1)$. \\*
לאחר נרמול נקבל כי $u_i = \frac{1}{\sqrt{i^2 - i + 1}} (-1, \hdots, i - 1, 0, \hdots, -1)$. \\*
נגדיר בסיס $(u_1, u_2, \hdots)$. בסיס זה הוא אורתונורמלי אשר בו לתבנית הריבועית $q$ צורה אלכסונית כפי שמופיעה בסעיף א'.

\section{שאלה 5}
\subsection{סעיף א'}
יהי $V$ מרחב וקטורי מעל $\CC$ כך ש־$\dim V \ge 2$. \\*
נוכיח שאם $q: V \to \CC$ תבנית ריבועית, אז קיים $v \ne 0$ כך ש־$q(v) = 0$.
\begin{proof}
	אין לי מושג
\end{proof}
\end{document}
