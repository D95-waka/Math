\documentclass[a4paper]{article}

% packages
\usepackage{inputenc, fontspec, amsmath, amsthm, amsfonts, polyglossia, catchfile}
\usepackage[a4paper, margin=50pt, includeheadfoot]{geometry} % set page margins

% style
\AddToHook{cmd/section/before}{\clearpage}	% Add line break before section
\linespread{1.5}
\setcounter{secnumdepth}{0}		% Remove default number tags from sections
\setmainfont{Libertinus Serif}
\setsansfont{Libertinus Sans}
\setmonofont{Libertinus Mono}
\setdefaultlanguage{hebrew}
\setotherlanguage{english}

% operators
\DeclareMathOperator\cis{cis}
\DeclareMathOperator\Sp{Sp}
\DeclareMathOperator\tr{tr}
\DeclareMathOperator\im{Im}
\DeclareMathOperator\diag{diag}
\DeclareMathOperator*\lowlim{\underline{lim}}
\DeclareMathOperator*\uplim{\overline{lim}}

% commands
\renewcommand\qedsymbol{\textbf{משל}}
\newcommand{\NN}[0]{\mathbb{N}}
\newcommand{\ZZ}[0]{\mathbb{Z}}
\newcommand{\QQ}[0]{\mathbb{Q}}
\newcommand{\RR}[0]{\mathbb{R}}
\newcommand{\CC}[0]{\mathbb{C}}
\newcommand{\getenv}[2][] {
  \CatchFileEdef{\temp}{"|kpsewhich --var-value #2"}{\endlinechar=-1}
  \if\relax\detokenize{#1}\relax\temp\else\let#1\temp\fi
}
\newcommand{\explain}[2] {
	\begin{flalign*}
		 && \text{#2} && \text{#1}
	\end{flalign*}
}

% headers
\getenv[\AUTHOR]{AUTHOR}
\author{\AUTHOR}
\date\today

\title{פתרון ממ''ן 15 – אלגברה לינארית 2 (20229)}

\begin{document}
\maketitle

\section{שאלה 1}
\subsection{סעיף א'}
תהי העתקה לינארית $T : V \to V$ אשר בבסיס הסטנדרטי מיוצגת על־ידי המטריצה
\[
	A = \begin{pmatrix}
		1 & 5 \\
		-10 & -1
	\end{pmatrix}
\]
נמצא את כל תת־המרחבים ה־$T$ שמורים של $V$ כאשר $V = \RR^2, \CC^2$: \\*
\textbf{(1)}
נגדיר $V = \RR^2$: \\*
תחילה, נמצא את ערכיה העצמיים של $T$ על־ידי חישוב פולינום אופייני:
\[
	p(t) = \begin{vmatrix}
		t - 1 & -5 \\
		10 & t + 1
	\end{vmatrix}
	= (t - 1)(t + 1) + 50
	= t^2 + 49
\]
אין להעתקה $T$ אם־כן ערכים עצמיים כלל, ולכן משאלה 8.4.3 א' נובע כי אין ל־$T$ תת־מרחב $T$ שמור שאיננו טריוויאלי. \\*
בהתאם כלל התת־מרחבים ה־$T$ שמורים הם, על־פי דוגמה 8.4.2, הם מרחב האפס ו־$V$ עצמו. \\*
\textbf{(2)}
נגדיר $V = \CC^2$: \\*
מעל שדה המרוכבים $p(t) = t^2 + 49 = (t - 7i)(t + 7i)$ ובהתאם $-7i, 7i$ הם ערכיה העצמיים של $T$ והיא אף לכסינה. \\*
נמצא וקטורים עצמיים של $T$ על־ידי חישוב המרחב העצמי לכל ערך עצמי:
\begin{align*}
	& (7i I - A) {(x, y)}^t = 0 \rightarrow
	\begin{pmatrix}
		-1 + 7i & -5 \\
		10 & 1 + 7i
	\end{pmatrix}
	\rightarrow
	\begin{pmatrix}
		-50 & -5(1 + 7i) \\
		10 & 1 + 7i
	\end{pmatrix}
	\rightarrow
	10x - (1 + 7i)y = 0
	\rightarrow
	\Sp\{ (1 + 7i, 10) \} \\
	& (-7i - A) {(x, y)}^t = 0
	\rightarrow
	\begin{pmatrix}
		-1 -7i & -5 \\
		10 & 1 - 7i
	\end{pmatrix}
	\rightarrow
	\begin{pmatrix}
		-50 & -5(1 - 7i) \\
		10 & 1 - 7i
	\end{pmatrix}
	\rightarrow
	10x + (1 - 7i)y = 0
	\rightarrow
	\Sp\{ (7i - 1, 10) \}
\end{align*}
אז מצאנו כי כל תת־המרחבים ה־$T$ שמורים שאינם טריוויאליים הם
\[
	\Sp\{ (7i - 1, 10) \},
	\Sp\{ (7i + 1, 10) \}
\]

\subsection{סעיף ב'}
יהי $V$ מרחב לינארי מעל $F$ ותהי $T : V \to V$ העתקה לינארית כך שכל תת־מרחב של $V$ הוא $T$ שמור. \\*
נוכיח שקיים $\alpha \in F$ כך ש־$T = \alpha I$.
\begin{proof}
	ידוע כי כל תת־מרחב של $V$ הוא $T$ שמור ולכן נובע כי גם עבור תת־מרחבים מממד $1$ הם $T$ שמורים, לכן
	\[
		\forall v \in V: T v = \alpha v
	\]
	דהינו ש־$T v$ הוא צ''ל של $v$. מהגדרת ההעתקה הלינארית נובע כי $\alpha$ קבוע לכל $v$ שנבחר.
\end{proof}

\section{שאלה 2}
\subsection{סעיף א'}
תהי $T$ העתקה לינארית במרחב לינארי $V$ שממדו סופי ויהי $W$ תת־מרחב $T$ שמור של $V$ ו־$T_W$ הצמצום של $T$ ל־$W$.

\textbf{(1)}
נוכיח כי הפולינום המינימלי של $T_W$ מחלק את הפולינום המינימלי של $T$.
\begin{proof}
	נגדיר $M_1(x)$ הפולינום המינימלי של $T$ ו־$M_2(x)$ הפולינום המינימלי של $T_W$. \\*
	ידוע כי $M_1(T) = M_2(T_W) = 0$. נניח בשלילה כי $M_1(T_W) \ne 0$, ולכן קיים $u \in W$ כך ש־$M_1(T_W) u \ne 0$.
	אבל $W \subseteq V$ ולכן $u \in V$ ובהתאם $M_1(T) u = 0$, וידוע כי מעל $W$ מתקיים $T = T_W$ ולכן $M_1(T_w) = 0$ בסטירה לטענה. \\*
	משאלה 9.9.1 סעיף א' נובע ישירות כי $M_2$ מחלק את $M_1$.
\end{proof}
\textbf{(2)}
נוכיח כי אם $T$ לכסינה אז גם $T_W$ לכסינה.
\begin{proof}
	ממשפט 10.2.11 נובע כי $M_1$ מתפרקת למכפלת גורמים לינאריים שונים וידוע לנו כי $M_2$ מחלקת אותה,
	לכן גם $M_2$ מורכבת ממכפלת גורמים לינאריים שונים. ממשפט 10.2.11 נובע מסיבה זו שגם $M_2$ לכסינה.
\end{proof}

\subsection{סעיף ב'}
נניח כי $T : \RR^3 \to \RR^3$ היא בעלת ערכים עצמיים $1, 2, 3$ ווקטורים עצמיים $v_1, v_2, v_3$ בהתאמה. \\*
נמצא את כל תת־המרחבים ה־$T$ שמורים של $\RR^3$: \\*
ההעתקה $T$ היא בעלת $3$ ערכים עצמיים שונים ולכן לכסינה. לכל וקטור עצמי $u$ שנבחר קיים ערך עצמי $\lambda$ המקיים $T u = \lambda u$,
ולכן הצמצום של $T$ ל־$\Sp\{ u \}$ הוא $T$ שמור.
כמובן שגם חיבור שני תתי־מרחב $T$ שמורים יובילו לתת־מרחב $T$ שמור, ולכן כלל התת־מרחבים ה־$T$ שמורים על $\RR^3$ הם:
\[
	0,
	\Sp\{ v_1 \},
	\Sp\{ v_2 \},
	\Sp\{ v_3 \},
	\Sp\{ v_1, v_2 \},
	\Sp\{ v_1, v_3 \},
	\Sp\{ v_2, v_3 \},
	\RR^3
\]

\section{שאלה 3}
תהי $T : \RR^3 \to \RR^3$ העתקה לינארית אשר מיוצגת בבסיס הסטנדרטי על־ידי המטריצה
\[
	A = \begin{pmatrix}
		3 & 1 & 0 \\
		0 & 3 & 0 \\
		0 & 0 & 2
	\end{pmatrix}
\]
\subsection{סעיף א'}
נמצא תתי־מרחב $T$ שמורים לא טריוויאליים על $\RR^3$. \\*
$A$ היא מטריצה משולשית ולכן $T$ ניתן לשילוש ומטענה 8.1.1 נובע כי ערכיה הפנימיים של $T$ הם $2, 3$.
מחישוב המרחב העצמי אנו למדים כי $T(0, 0, 1) = 2(0, 0, 1)$ וגם כי $T(1, 0, 0) = 3(1, 0, 0)$ ולכן משאלה 8.4.1 סעיף ג' נובע כי המרחבים
\[
	\Sp\{ (1, 0, 0) \},
	\Sp\{ (0, 0, 1) \},
\]
הם מרחבים לא טריוויאליים $T$ שמורים מעל $\RR^3$.

\subsection{סעיף ב'}
יהי $W = \ker(T - 3I)$, נוכיח כי לא קיים תת־מרחב $U \subseteq \RR^3$ שהוא $T$ שמור ומקיים
\[
	\RR^3 = W \oplus U
\]
\begin{proof}
	מחישוב מערכת המשוואות אנו מקבלים כי
	\[
		W = \Sp\{ (1, 0, 0) \}
	\]
	לכן על $U$ לקיים
	\[
		U = \Sp\{ (0, 1, 0), (0, 0, 1) \} \tag{1}
	\]
	כדי שהטענה תתקיים. \\*
	כבר מצאנו כי $(0, 0, 1)$ הוא וקטור עצמי של $2$ והצמצום על תת־המרחב שהוא יוצר הוא $T$ שמור, לכן עלינו לבדוק את $(0, 1, 0)$ בלבד. \\*
	מחישוב נובע כי $A {(0, 1, 0)}^t = {(1, 3, 0)}^t$ ולכן וקטור זה איננו יוצר תת־מרחב $T$ שמור, ואף מעל $U$ על־פי (1) הוא לא $T$ שמור. \\*
	אז לא קיים אף $U$ אשר יכול לקיים את התנאים.
\end{proof}

\section{שאלה 4}
יהי $V$ מרחב לינארי מממד סופי ו־$T : V \to V$ העתקה לינארית. \\*
יהי $M(x) = M_1(x) M_2(x) \cdots M_k(x)$ הפולינום המינימלי של $T$ כאשר $M_i(x)$ פולינומים מתוקנים זרים בזוגות. \\*
נגדיר $V = W_1 \oplus \cdots \oplus W_k$ הפירוק הפרימרי המתאים ל־$T$ כאשר $W_i = \ker M_i(T)$. \\*
יהי $W$ תת־מרחב $T$ שמור של $V$. \\*
נוכיח כי
\[
	W = (W \cap W_1) \oplus \cdots \oplus (W \cap W_k)
\]
\begin{proof}
	יהי $1 \le i, j \le k$, על־פי הפירוק הפרימרי $W_i \cap W_j = \{0\}$ ולכן
	\[
		W_i \cap W_j \cap W = (W_i \cap W) \cap (W_j \cap W) = \{ 0 \} \tag{1}
	\]
	כמו־כן
	\begin{align*}
		(W_i \cap W) + \left(W_j \cap W\right)
		& = \{ u_i + u_j \mid u_i \in (W_i \cap W), u_j \in (W_j \cap W) \} \\
		& = \{ u_i + u_j \mid u_i \in W_i \cap W, u_j \in W_j \cap W \} \cap W & \text{$W$ מרחב לינארי} \\
		& = \{ u_i + u_j \mid u_i \in W_i, u_j \in W_j \} \cap W & \text{כנסיעה מתכונות המרחב} \\
		& = (W_i + W_j) \cap W
	\end{align*}
	אז בתהליך דומה נוכל לקבוע גם כי
	\[
		(W \cap W_1) + \cdots + (W \cap W_k) = W \cap (W_1 + \cdots + W_k) = W \cap V = W
	\]
	וכנביעה מ־$(1)$ גם
	\[
		W = (W \cap W_1) \oplus \cdots \oplus (W \cap W_k)
	\]
\end{proof}

\section{שאלה 5}
יהי $V$ מרחב אוניטרי מממד סופי ו־$T : V \to V$ העתקה נורמלית. \\*
נוכיח שכל תת־מרחב $T$ שמור הוא גם $T^*$ שמור.
\begin{proof}
	יהי $W$ תת־מרחב $T$ שמור של $V$ ונגדיר $T_W$ הצמצום של $T$ על $W$. \\*
	אנו יודעים כי עבור וקטורים ב־$W$ ההעתקות $T$ ו־$T_W$ זהות ולכן גם $T_W$ נורמלית. \\*
	בשל כך ועל־פי משפט 3.2.1 ההעתקה $T_W$ לכסינה אוניטרית. \\*
	ממשפט 3.4.2 נובע כי אם $\lambda_i$ כאשר $1 \le i \le k$ ערכים עצמיים של $T_W$ ו־$P_i$ ההיטלים האורתוגונליים של $V_{\lambda_i}$, אז לכל $w \in W$
	\[
		T_W w = \lambda_1 P_1 w + \cdots + \lambda_k P_k w \in W
	\]
	ובהתאם גם לכל $i$
	\[
		\lambda_i P_i w \in W
	\]
	מלמה 3.2.5 וממשפט 3.4.2 סעיף ה' נובע כי
	\[
		\overline{\lambda_i} P_i \in W
	\]
	ולכן
	\[
		T^* w = \overline{\lambda_1} P_1 w + \cdots + \overline{\lambda_k} P_k w \in W
	\]
	ומכאן נובע כי $W$ הוא תת־מרחב $T^*$ שמור.
\end{proof}

\end{document}
