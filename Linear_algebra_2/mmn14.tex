\documentclass[a4paper]{article}

% packages
\usepackage{inputenc, amsmath, amsthm, thmtools, amsfonts, amssymb, luacode, catchfile, tikzducks, hyperref}
\usepackage[a4paper, margin=50pt, includeheadfoot]{geometry} % set page margins
\usepackage[shortlabels]{enumitem}
\usepackage[skip=3pt, indent=0pt]{parskip}

% language
\usepackage[bidi=basic, layout=tabular, provide=*]{babel}
\babelprovide[main, import]{hebrew}
\babelprovide{rl}
\babelfont{rm}{Libertinus Serif}
\babelfont{sf}{Libertinus Sans}
\babelfont{tt}{Libertinus Mono}

% style
\AddToHook{cmd/section/before}{\clearpage}	% Add line break before section
\linespread{1.3}
\setcounter{secnumdepth}{0}		% Remove default number tags from sections, this won't do well with theorems
\AtBeginDocument{\setlength{\belowdisplayskip}{3pt}}
\AtBeginDocument{\setlength{\abovedisplayskip}{3pt}}

% operators
\DeclareMathOperator\cis{cis}
\DeclareMathOperator\Sp{Sp}
\DeclareMathOperator\tr{tr}
\DeclareMathOperator\im{Im}
\DeclareMathOperator\re{Re}
\DeclareMathOperator\diag{diag}
\DeclareMathOperator*\lowlim{\underline{lim}}
\DeclareMathOperator*\uplim{\overline{lim}}
\DeclareMathOperator\rng{rng}
\DeclareMathOperator\Sym{Sym}
\DeclareMathOperator\Arg{Arg}
\DeclareMathOperator\Log{Log}
\DeclareMathOperator\dom{dom}

% commands
%\renewcommand\qedsymbol{\textbf{מש''ל}}
%\renewcommand\qedsymbol{\fbox{\emoji{lizard}}}
\newcommand{\NN}[0]{\mathbb{N}}
\newcommand{\ZZ}[0]{\mathbb{Z}}
\newcommand{\QQ}[0]{\mathbb{Q}}
\newcommand{\RR}[0]{\mathbb{R}}
\newcommand{\CC}[0]{\mathbb{C}}
\newcommand{\FF}[0]{\mathbb{F}}
\newcommand{\PP}[0]{\mathbb{P}}
\newcommand{\TT}[0]{\mathbb{T}}
\newcommand{\acts}[0]{\circlearrowright}
\newcommand{\explain}[2] {
	\begin{flalign*}
		 && \text{#2} && \text{#1}
	\end{flalign*}
}
\newcommand{\maketitleprint}[0]{ \begin{center}
	\begin{tikzpicture}[scale=3]
		\duck[graduate=gray!20!black, tassel=red!70!black]
	\end{tikzpicture}	
\end{center}
}

% theorem commands
\newtheoremstyle{c_remark}
	{}	% Space above
	{}	% Space below
	{}% Body font
	{}	% Indent amount
	{\bfseries}	% Theorem head font
	{}	% Punctuation after theorem head
	{.5em}	% Space after theorem head
	{\thmname{#1}\thmnumber{ #2}\thmnote{ \normalfont{\text{(#3)}}}}	% head content
\newtheoremstyle{c_definition}
	{3pt}	% Space above
	{3pt}	% Space below
	{}% Body font
	{}	% Indent amount
	{\bfseries}	% Theorem head font
	{}	% Punctuation after theorem head
	{.5em}	% Space after theorem head
	{\thmname{#1}\thmnumber{ #2}\thmnote{ \normalfont{\text{(#3)}}}}	% head content
\newtheoremstyle{c_plain}
	{3pt}	% Space above
	{3pt}	% Space below
	{\itshape}% Body font
	{}	% Indent amount
	{\bfseries}	% Theorem head font
	{}	% Punctuation after theorem head
	{.5em}	% Space after theorem head
	{\thmname{#1}\thmnumber{ #2}\thmnote{ \text{(#3)}}}	% head content

\theoremstyle{c_plain}
\newtheorem{theorem}{משפט}[section]
\newtheorem{lemma}[theorem]{למה}
\newtheorem{proposition}[theorem]{טענה}
\newtheorem*{proposition*}{טענה}
%\newtheorem{corollary}[theorem]{אין חלופה עברית}

\theoremstyle{c_definition}
\newtheorem{definition}[theorem]{הגדרה}
\newtheorem*{definition*}{הגדרה}
\newtheorem{example}{דוגמה}[section]
\newtheorem{exercise}{תרגיל}[section]

\theoremstyle{c_remark}
\newtheorem*{remark}{הערה}
\newtheorem*{solution}{פתרון}
\newtheorem{conclusion}[theorem]{מסקנה}
\newtheorem{notation}[theorem]{סימון}

% Questions related commands
\newcounter{question}
\setcounter{question}{1}
\newcounter{sub_question}
\setcounter{sub_question}{1}

\newcommand{\question}[1][0]{
	\ifthenelse{#1 = 0}{}{\setcounter{question}{#1}}
	\subsection{שאלה \arabic{question}}
	\addtocounter{question}{1}
	\setcounter{sub_question}{1}
}

\newcommand{\subquestion}[1][0]{
	\ifthenelse{#1 = 0}{}{\setcounter{sub_question}{#1}}
	\subsubsection{סעיף \localecounter{letters.gershayim}{sub_question}}
	\addtocounter{sub_question}{1}
}

% import lua and start of document
\directlua{common = require ('../common')}

\GetEnv{AUTHOR}

% headers
\author{\AUTHOR}
\date\today

\title{פתרון ממ''ן 14 – אלגברה לינארית 2 (20229)}

\begin{document}
\maketitle

\section{שאלה 1}
נמצא את הדרגה ואת הסימנית של התבנית הריבועית $q : \RR^4 \to \RR$ המוגדרת
\[
	q(x_1, x_2, x_3, x_4) = x_1 x_2 + 2x_1 x_3 + 3x_1 x_4 + x_2 x_3 + 2x_2 x_4 + x_3 x_4
\]
מסימטריית המטריצה המייצגת של $q$, נגדיר את המטריצה המייצגת
\[
	A = \begin{pmatrix}
		0 & \frac{1}{2} & 1 & \frac{3}{2} \\
		\frac{1}{2} & 0 & \frac{1}{2} & 1 \\
		1 & \frac{1}{2} & 0 & \frac{1}{2} \\
		\frac{3}{2} & 1 & \frac{1}{2} & 0
	\end{pmatrix}
\]
נחפוף את $A$ אלמנטרית:
\begin{align*}
	& \begin{pmatrix}
		0 & \frac{1}{2} & 1 & \frac{3}{2} \\
		\frac{1}{2} & 0 & \frac{1}{2} & 1 \\
		1 & \frac{1}{2} & 0 & \frac{1}{2} \\
		\frac{3}{2} & 1 & \frac{1}{2} & 0
	\end{pmatrix}
	\xrightarrow[C_i \to \sqrt{2} C_i]{R_i \to \sqrt{2} R_i \mid 1 \le i \le 4}
	\begin{pmatrix}
		0 & 1 & 2 & 3 \\
		1 & 0 & 1 & 2 \\
		2 & 1 & 0 & 1 \\
		3 & 2 & 1 & 0
	\end{pmatrix}
	\xrightarrow[C_3 \to C_3 - 2C_2]{R_3 \to R_3 - 2R_2}
	\begin{pmatrix}
		0 & 1 & 0 & 3 \\
		1 & 0 & 1 & 2 \\
		0 & 1 & -4 & -3 \\
		3 & 2 & -3 & 0
	\end{pmatrix}
	\xrightarrow[C_4 \to C_4 - 3C_2]{R_4 \to R_4 - 3R_2} \\
	& \begin{pmatrix}
		0 & 1 & 0 & 0 \\
		1 & 0 & 1 & 2 \\
		0 & 1 & -4 & -6 \\
		0 & 2 & -6 & -12
	\end{pmatrix}
	\xrightarrow[C_4 \to C_4 - 2C_3]{R_4 \to R_4 - 2R_3}
	\begin{pmatrix}
		0 & 1 & 0 & 0 \\
		1 & 0 & 1 & 0 \\
		0 & 1 & -4 & 2 \\
		0 & 0 & 2 & -4
	\end{pmatrix}
	\xrightarrow[C_3 \to C_3 - C_1]{R_3 \to R_3 - R_1}
	\begin{pmatrix}
		0 & 1 & 0 & 0 \\
		1 & 0 & 0 & 0 \\
		0 & 0 & -4 & 2 \\
		0 & 0 & 2 & -4
	\end{pmatrix}
	\xrightarrow[C_1 \to C_1 + \frac{1}{2}C_2]{R_1 \to R_1 + \frac{1}{2}R_2} \\
	& \begin{pmatrix}
		1 & 1 & 0 & 0 \\
		1 & 0 & 0 & 0 \\
		0 & 0 & -4 & 2 \\
		0 & 0 & 2 & -4
	\end{pmatrix}
	\xrightarrow[C_2 \to C_2 - C_1]{R_2 \to R_2 - R_1}
	\begin{pmatrix}
		1 & 0 & 0 & 0 \\
		0 & -1 & 0 & 0 \\
		0 & 0 & -4 & 2 \\
		0 & 0 & 2 & -4
	\end{pmatrix}
	\xrightarrow[C_4 \to C_4 + \frac{1}{2}C_3]{R_4 \to R_4 + \frac{1}{2}R_2}
	\begin{pmatrix}
		1 & 0 & 0 & 0 \\
		0 & -1 & 0 & 0 \\
		0 & 0 & -4 & 0 \\
		0 & 0 & 0 & -3
	\end{pmatrix}
\end{align*}
מצאנו מטריצה אלכסונית וממנה נובע $\rho(q) = 4$ ומהגדרת החתימה נובע כי $\sigma = -2$.

\subsection{סעיף ב'}
נמצא תת־מרחב מממד מקסימלי של $\RR^4$ שעליו $q$ היא תבנית חיובית לחלוטין. \\*
בסעיף הקודם מצאנו מטריצה מייצגת אלכסונית של $q$, נוכל לבנות לפיו בסיס עבורו $q$ אלכסונית. נגדיר בסיס זה להיות $W$. \\*
אז בבסיס זה
\[
	q(x_1, x_2, x_3, x_4) = x_1^2 - x_2^2 - 4x_3^2 - 3x_4^2 \tag{1}
\]
עבור תת־המרחב $\Sp\{ w_1 \}$ התבנית $q$ חיובית לחלוטין.
אילו היה תת־המרחב המקסימלי המקיים תנאי זה גדול בממדו מ־$1$ אז היה קיים וקטור $u$ בלתי תלוי ב־$w_1$ אשר מקיים
\[
	q(u) \ge 0
\]
בשל היותו בלתי תלוי ב־$w_1$ כאשר $u = \lambda_1 w_1 + \lambda_2 w_2 + \lambda_3 w_3 + \lambda_4 w_4$
אנו יודעים כי לפחות אחד הערכים $\lambda_2, \lambda_3, \lambda_3 \ne 0$. \\*
נוכל ללא פגיעה בכלליות להניח ש־$\lambda_1 = 0$ שאם לא כן נוכל להגדיר מחדש את הווקטור כך. \\*
מ־$(1)$ נובע במצב זה ש־$q(v) < 0$ בסתירה להנחה ולכן ממד המרחב המקסימלי אשר עבורו $q$ חיובית לחלוטין הוא $1$. \\*
נעבור לחישוב $w_1$. אנו יודעים כי וקטור זה הוא ווקטור העמודה הראשון במטריצת המעבר מהבסיס הסטנדרטי ל־$W$,
ומטריצה זו מתקבלת מביצוע פעולות השורה שביצענו בחפיפה האלמנטרית בסעיף הקודם, נפעיל אותם על מטריצת היחידה ונקבל
\[
	w_1 = (2, 1, 0, 0)
\]
ובהתאם תת־המרחב המקסימלי הוא
\[
	\Sp\{ (2, 1, 0, 0) \}
\]

\section{שאלה 2}
יהי $V$ מרחב וקטורי אשר ממדו $n$, ותהי $q$ תבנית ריבועית חיובית למחצה מעל המרחב $V$. נגדיר $\rho$ דרגת $q$.\\*
נוכיח כי
\[
	L_0 = \left\{ v \in V \mid q(v) = 0 \right\}
\]
הוא תת־מרחב של $V$ ושמתקיים $\dim L_0 = n - \rho$.
\begin{proof}
	% .םהלש ל"צה םג תינבתה לש תויראנילה ללגבו ,םיספאמש םייטרדנטס םירוטקו םימייק ןסכלמ סיסבבש תוארהל איה הרטמה
	יהי $W$ בסיס אורתונורמלי מלכסן קנונית של $q$, לכן קיימים $\lambda_i, 1 \le i \le n$ עבורם לפי טענה 6.3.2
	\[
		0 \le q(x_1, x_2, \hdots, x_n) = \lambda_1 x_1^2 + \cdots + \lambda_{\rho} x_{\pi}^2 + 0 x_{\pi + 1}^2 + \cdots + 0 x_n^2 \tag{1}
	\]
	בשל התאפסות המקדמים לכל $\rho < i \le n$ מתקיים $q(w_i) = 0$, $q$ תבנית בילינארית ולכן נובע גם כי לכל סקלר $\lambda$ מתקיים
	\[
		q(\lambda w_i) = \lambda^2 q(w_i) = 0
	\]
	יהי $j \in \NN$ כך ש־$\rho < j \le n$, מ־$(1)$ נובע גם כי $q(w_i + w_j) = 0$. \\*
	נגדיר $L_0 = \{ v \in v \mid q(v) = 0 \}$, קבוצת הווקטורים שמאפסים את $q$, משתי הטענות האחרונות נובע כי קבוצה זו היא מרחב ווקטורי. \\*
	עוד אנו יודעים כי $L_0 \subseteq V$ ולכן $L_0$ תת־מרחב של $V$. \\*
	ראינו כי תת־המרחב נוצר על־ידי $\{ w_{\rho + 1}, \hdots, w_n \}$, וידוע לנו כי קבוצה זו בלתי תלויה מינימלית, ולכן מהווה בסיס ל־$L_0$,
	אז ממד $L_0$ הוא $n - \rho$.
\end{proof}

\section{שאלה 3}
יהי $V$ מרחב וקטורי מממד סופי מעל $\RR$ ותהי $q : V \to \RR$ תבנית ריבועית. \\*
נוכיח כי אם הקבוצה $L = \{ v \mid q(v) \ge 0 \}$ היא תת־מרחב של $V$ אז $q$ שומרת סימן.
\begin{proof}
	תהי $L$ קבוצה המוגדרת כלעיל, ונניח כי היא תת־מרחב של $V$. \\*
	נגדיר בסיס מלכסן קנונית $W$ עבור $q$, ונגדיר $\rho$ דרגת $q$, $\pi$ מספר האיברים החיוביים בצורה הקנונית של $q$. \\*
	נניח כי קיים $w_i$ כך ש־$q(w_i) \ge 0$ ובהתאם $w_i \in L$.
	נניח בשלילה כי קיים גם וקטור $w_j$ עבורו $q(w_j) < 0$. \\*
	מהגדרת התבנית הריבועית הקנונית נובע ישירות כי קיימים סקלרים $\lambda, \mu > 0$ כך ש־$q(\lambda w_i + \mu w_j) \ge 0$.
	ולכן גם $\lambda w_i + \mu w_j \in L$, וכמובן בשל היות $L$ מרחב וקטורי כך ש־$w_i \in L$ נובע גם כי $w_j \in L$.
	אם $w_j \in L$ אז $q(w_j) \ge 0$ בסתירה להנחה כי $q(w_j) < 0$ ולכן הנחת הבסיס היא סתירה. \\*
	אז לא קיימים וקטורים $w_i, w_j$ עבורם $q(w_i) \ge 0 > q(w_j)$ ומכאן נובע כי $q$ שומרת סימן.
\end{proof}

\section{שאלה 4}
\subsection{סעיף א'}
נמצא את כל הערכים הממשיים של $\lambda$ עבורם התבנית הריבועית $q : \RR^3 \to \RR$ המוגדרת על־ידי
\[
	q(x_1, x_2, x_3) = x_1^2 + 4x_2^2 + x_3^2 + 2\lambda x_1 x_2 + 10x_1x_3 + 6x_2x_3
\]
חיובית לחלוטין. \\*
נחשב את המטריצה המייצגת של $q$:
\[
	A = \begin{pmatrix}
		1 & \lambda & 5 \\
		\lambda & 4 & 3 \\
		5 & 3 & 1
	\end{pmatrix}
\]
לפי מסקנה 6.4.3 $q$ תבנית חיובית לחלוטין אם ורק אם כלל המינורים הראשיים של $A$ חיוביים. נבדוק את חיוביותם:
\begin{align*}
	& \begin{vmatrix}
		1
	\end{vmatrix} = 1 > 0 \\
	& \begin{vmatrix}
		1 & \lambda \\
		\lambda & 4
	\end{vmatrix} = 4 - \lambda^2 > 0 \rightarrow
	& -2 < \lambda < 2 \\
	& \begin{vmatrix}
		1 & \lambda & 5 \\
		\lambda & 4 & 3 \\
		5 & 3 & 1
	\end{vmatrix}
	= \begin{vmatrix}
		-24 & \lambda - 15 & 5 \\
		\lambda - 15 & -5 & 3 \\
		0 & 0 & 1
	\end{vmatrix}
	= \begin{vmatrix}
		-24 & \lambda - 15 \\
		\lambda - 15 & -5
	\end{vmatrix} \\
	& = 120 - {(\lambda - 15)}^2 > 0
	\rightarrow 120 - \lambda^2 + 30 \lambda - 225 > 0
	\rightarrow & 15 - \sqrt{120} < \lambda < 15 + \sqrt{120}
\end{align*}
לא קיים ערך $\lambda$ אשר עבורו כלל המינורים הם חיוביים ובהתאם אין ערך $\lambda$ עבורו התבנית $q$ חיובית לחלוטין.

\subsection{סעיף ב'}
תהיינה התבניות הבאות על $\RR^3$:
\begin{align*}
	& q_1(x_1, x_2, x_3) = x_1^2 + 2x_2^2 + 3x_3^2 + 2x_1x_2 - 2x_1x_3 \\
	& q_2(x_1, x_2, x_3) = 2x_1^2 + 8x_2^2 + 3x_3^2 + 8x_1x_2 + 2x_1x_3 + 4x_2x_3
\end{align*}
נמצא בסיס של $\RR^3$ עבורו
\[
	q_1 = y_1^2 + y_2^2 + y_3^3 \tag{1}
\]
נבנה את המטריצה המייצגת של $q_1$ ונחפוף אותה אלמנטרית תוך כדי ביצוע פעולות השורה על מטריצת היחידה $I_3$:
\begin{align*}
	& \begin{pmatrix}
		1 & 1 & -1 & \vline & 1 & 0 & 0 \\
		1 & 2 & 0 & \vline & 0 & 1 & 0 \\
		-1 & 0 & 3 & \vline & 0 & 0 & 1
	\end{pmatrix}
	\xrightarrow[C_2 \to C_2 - C_1]{R_2 \to R_2 - R_1}
	\begin{pmatrix}
		1 & 0 & -1 & \vline & 1 & 0 & 0 \\
		0 & 1 & 1 & \vline & -1 & 1 & 0 \\
		-1 & 1 & 3 & \vline & 0 & 0 & 1
	\end{pmatrix} \\
	& \xrightarrow[C_3 \to C_3 + C_1]{R_3 \to R_3 + R_1}
	\begin{pmatrix}
		1 & 0 & 0 & \vline & 1 & 0 & 0 \\
		0 & 1 & 1 & \vline & -1 & 1 & 0 \\
		0 & 1 & 2 & \vline & 1 & 0 & 1
	\end{pmatrix}
	\xrightarrow[C_3 \to C_3 - C_2]{R_3 \to R_3 - R_2}
	\begin{pmatrix}
		1 & 0 & 0 & \vline & 1 & 0 & 0 \\
		0 & 1 & 0 & \vline & -1 & 1 & 0 \\
		0 & 0 & 1 & \vline & 2 & -1 & 1
	\end{pmatrix}
\end{align*}
אז המטריצה המלכסנת של $q_1$ היא
\[
	M = \begin{pmatrix} 1 & -1 & 2 \\
		0 & 1 & -1 \\
		0 & 0 & 1
	\end{pmatrix}
\]
מצאנו על־ידי חפיפה אלמנטרית בסיס בו מתקבלת צורה $(1)$. בסיס זה הוא
\[
	B = ((1, 0, 0), (-1, 1, 0), (2, -1, 1))
\]
נחשב את המטריצה האלכסונית של $q_2$ על־ידי מציאת המטריצה המייצגת שלה ולכסון על־ידי $M$.
\[
	M^t {[q_2]}_E M
	= \begin{pmatrix}
		1 & 0 & 0 \\
		-1 & 1 & 0 \\
		2 & -1 & 1
	\end{pmatrix}
	\begin{pmatrix}
		2 & 4 & 1 \\
		4 & 8 & 2 \\
		1 & 2 & 3
	\end{pmatrix}
	\begin{pmatrix}
		1 & -1 & 2 \\
		0 & 1 & -1 \\
		0 & 0 & 1
	\end{pmatrix}
	= \begin{pmatrix}
		2 & 4 & 1 \\
		2 & 4 & 1 \\
		1 & 2 & 3
	\end{pmatrix}
	\begin{pmatrix}
		1 & -1 & 2 \\
		0 & 1 & -1 \\
		0 & 0 & 1
	\end{pmatrix}
	= \begin{pmatrix}
		2 & 2 & 1 \\
		2 & 2 & 1 \\
		1 & 1 & 3
	\end{pmatrix}
	= A
\]
אנו יודעים כי $M$ מלכסנת את $q_1$ למטריצת היחידה ואת $q_2$ ל־$A$, אילו נמצא מטריצה $P$ המלכסנת את $A$ בשל החפיפה למטריצת היחידה החפיפה של $q_1$ לא תיפגע.
נחפוף אלמנטרית את $A$:
\[
	\begin{pmatrix}
		2 & 2 & 1 & \vline & 1 & 0 & 0 \\
		2 & 2 & 1 & \vline & 0 & 1 & 0 \\
		1 & 1 & 3 & \vline & 0 & 0 & 1
	\end{pmatrix}
	\xrightarrow[C_2 \to C_2 - C_1]{R_2 \to R_2 - R_1}
	\begin{pmatrix}
		2 & 0 & 1 & \vline & 1 & 0 & 0 \\
		0 & 0 & 0 & \vline & -1 & 1 & 0 \\
		1 & 0 & 3 & \vline & 0 & 0 & 1
	\end{pmatrix}
	\xrightarrow[C_3 \to C_3 - \frac{1}{2}C_1]{R_3 \to R_3 - \frac{1}{2}R_1}
	\begin{pmatrix}
		2 & 0 & 0 & \vline & 1 & 0 & 0 \\
		0 & 0 & 0 & \vline & -1 & 1 & 0 \\
		0 & 0 & \frac{5}{2} & \vline & -\frac{1}{2} & 0 & 1
	\end{pmatrix}
\]
אז מטריצת המעבר מ־$B$ לבסיס בו $q_2$ אלכסונית היא
\[
	P = \begin{pmatrix}
		1 & -1 & -\frac{1}{2} \\
		0 & 1 & 0 \\
		0 & 0 & 1
	\end{pmatrix}
\]
ומטריצת המעבר השלמה היא $MP$, נחשבה
\[
	M' = MP = 
	\begin{pmatrix} 1 & -1 & 2 \\
		0 & 1 & -1 \\
		0 & 0 & 1
	\end{pmatrix}
	\begin{pmatrix}
		1 & -1 & -\frac{1}{2} \\
		0 & 1 & 0 \\
		0 & 0 & 1
	\end{pmatrix}
	= \begin{pmatrix}
		1 & -2 & \frac{3}{2} \\
		0 & 1 & -1 \\
		0 & 0 & 1
	\end{pmatrix}
\]
ובהתאם הבסיס של $\RR^3$ הוא
\[
	((1, 0, 0), (-2, 1, 0), (\frac{3}{2}, -1, 1))
\]
ומתקיים בו
\[
	q_2(y) = 2y_1^2 + 0y_2^2 + \frac{5}{2}y_3^2
\]

\section{שאלה 5}
\subsection{סעיף א'}
נוכיח כי אם $q$ תבנית ריבועית חיובית למחצה אז המטריצה המייצגת אותה היא מטריצה סינגולרית.
\begin{proof}
	תהי $q$ תבנית ריבועית, $A_E$ מטריצת הייצוג שלה בבסיס הסטנדרטי, בסיס מלכסן אורתונורמלי $W$ ומטריצת הייצוג לפיו $A_W$. \\*
	ידוע כי $q$ חיובית למחצה, לכן קיים $i$ כך ש־$q(w_i) = 0$, ובהתאם במטריצה האלכסונית $A_W$ האיבר ה־$ii$ הוא $0$. \\*
	מסיבה זו $A_W$ כמובן בלתי הפיכה, ולכן נותר רק לראות כי $A_E$ ו־$A_W$ שקולות־שורה. \\*
	אנו יודעים כי קיימת מטריצה הפיכה $P$ עבורה מתקיים $A_E = P^t A_W P$, וממשפט 2.3.7 נובע כי $P$ אורתונורמלית ומקיימת $P^t = P^{-1}$.
	אז מתקיים $A_E = P^{-1} A_W P$ והמטריצה $A_E$ סינגולרית.
\end{proof}

\subsection{סעיף ב'}
תהי מטריצה סימטרית $A$ ותהי תבנית ריבועית $q : \RR^n \to \RR$ המוגדרת על־ידי $q(x) = x^t A x$, ידוע כי $q$ חיובית לחלוטין. \\*
נוכיח כי $A$ מטריצה אורתוגונלית אם ורק אם $A = I_n$.
\begin{proof}
	$q$ חיובית לחלוטין ולכן על־פי טענה 6.3.2 $A$ חופפת אלמנטרית ל־$I$. $(1)$ \\*
	נניח כי $A$ אורתוגונלית ונוכיח כי $A = I$. \\*
	ידוע כי $A$ סימטרית ולכן $A = A^t$. עוד אנו יודעים כי $A$ אורתוגונלית ולכן גם $A A^t = I$, אז נובע כי $A^2 = I$.
	משוויון זה נובע כי $(A - I)(A + I) = O$, ובהתאם $A = \pm I$ וממסקנה 6.2.1 ו־$(1)$ נובע כי $A = I$ בלבד. \\*
	נניח כי $A = I$ ונוכיח כי $A$ אורתוגונלית. \\*
	$I = I^t$ על־פי סימטריית $I$, וידוע גם כי $II = I$, אז $A A^t = I$.
\end{proof}

\end{document}
