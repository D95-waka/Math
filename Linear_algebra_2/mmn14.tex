\documentclass[a4paper]{article}

% packages
\usepackage{inputenc, fontspec, amsmath, amsthm, amsfonts, polyglossia, catchfile}
\usepackage[a4paper, margin=50pt, includeheadfoot]{geometry} % set page margins

% style
\AddToHook{cmd/section/before}{\clearpage}	% Add line break before section
\linespread{1.5}
\setcounter{secnumdepth}{0}		% Remove default number tags from sections
\setmainfont{Libertinus Serif}
\setsansfont{Libertinus Sans}
\setmonofont{Libertinus Mono}
\setdefaultlanguage{hebrew}
\setotherlanguage{english}

% operators
\DeclareMathOperator\cis{cis}
\DeclareMathOperator\Sp{Sp}
\DeclareMathOperator\tr{tr}
\DeclareMathOperator\im{Im}
\DeclareMathOperator\diag{diag}
\DeclareMathOperator*\lowlim{\underline{lim}}
\DeclareMathOperator*\uplim{\overline{lim}}

% commands
\renewcommand\qedsymbol{\textbf{משל}}
\newcommand{\NN}[0]{\mathbb{N}}
\newcommand{\ZZ}[0]{\mathbb{Z}}
\newcommand{\QQ}[0]{\mathbb{Q}}
\newcommand{\RR}[0]{\mathbb{R}}
\newcommand{\CC}[0]{\mathbb{C}}
\newcommand{\getenv}[2][] {
  \CatchFileEdef{\temp}{"|kpsewhich --var-value #2"}{\endlinechar=-1}
  \if\relax\detokenize{#1}\relax\temp\else\let#1\temp\fi
}
\newcommand{\explain}[2] {
	\begin{flalign*}
		 && \text{#2} && \text{#1}
	\end{flalign*}
}

% headers
\getenv[\AUTHOR]{AUTHOR}
\author{\AUTHOR}
\date\today


\title{פתרון ממ"ן 14 – אלגברה לינארית 2 (20229)}

\begin{document}
\maketitle

\section{שאלה 1}
נמצא את הדרגה ואת הסימנית של התבנית הריבועית $q : \RR^4 \to \RR$ המוגדרת
\[
	q(x_1, x_2, x_3, x_4) = x_1 x_2 + 2x_1 x_3 + 3x_1 x_4 + x_2 x_3 + 2x_2 x_4 + x_3 x_4
\]
מסימטריית המטריצה המייצגת של $q$, נגדיר את המטריצה המייצגת
\[
	A = \begin{pmatrix}
		0 & \frac{1}{2} & 1 & \frac{3}{2} \\
		\frac{1}{2} & 0 & \frac{1}{2} & 1 \\
		1 & \frac{1}{2} & 0 & \frac{1}{2} \\
		\frac{3}{2} & 1 & \frac{1}{2} & 0
	\end{pmatrix}
\]
נחפוף את $A$ אלמנטרית:
\begin{align*}
	& \begin{pmatrix}
		0 & \frac{1}{2} & 1 & \frac{3}{2} \\
		\frac{1}{2} & 0 & \frac{1}{2} & 1 \\
		1 & \frac{1}{2} & 0 & \frac{1}{2} \\
		\frac{3}{2} & 1 & \frac{1}{2} & 0
	\end{pmatrix}
	\xrightarrow[C_i \to \sqrt{2} C_i]{R_i \to \sqrt{2} R_i \mid 1 \le i \le 4}
	\begin{pmatrix}
		0 & 1 & 2 & 3 \\
		1 & 0 & 1 & 2 \\
		2 & 1 & 0 & 1 \\
		3 & 2 & 1 & 0
	\end{pmatrix}
	\xrightarrow[C_3 \to C_3 - 2C_2]{R_3 \to R_3 - 2R_2}
	\begin{pmatrix}
		0 & 1 & 0 & 3 \\
		1 & 0 & 1 & 2 \\
		0 & 1 & -4 & -3 \\
		3 & 2 & -3 & 0
	\end{pmatrix}
	\xrightarrow[C_4 \to C_4 - 3C_2]{R_4 \to R_4 - 3R_2} \\
	& \begin{pmatrix}
		0 & 1 & 0 & 0 \\
		1 & 0 & 1 & 2 \\
		0 & 1 & -4 & -6 \\
		0 & 2 & -6 & -12
	\end{pmatrix}
	\xrightarrow[C_4 \to C_4 - 2C_3]{R_4 \to R_4 - 2R_3}
	\begin{pmatrix}
		0 & 1 & 0 & 0 \\
		1 & 0 & 1 & 0 \\
		0 & 1 & -4 & 2 \\
		0 & 0 & 2 & -4
	\end{pmatrix}
	\xrightarrow[C_3 \to C_3 - C_1]{R_3 \to R_3 - R_1}
	\begin{pmatrix}
		0 & 1 & 0 & 0 \\
		1 & 0 & 0 & 0 \\
		0 & 0 & -4 & 2 \\
		0 & 0 & 2 & -4
	\end{pmatrix}
	\xrightarrow[C_1 \to C_1 + \frac{1}{2}C_2]{R_1 \to R_1 + \frac{1}{2}R_2} \\
	& \begin{pmatrix}
		1 & 1 & 0 & 0 \\
		1 & 0 & 0 & 0 \\
		0 & 0 & -4 & 2 \\
		0 & 0 & 2 & -4
	\end{pmatrix}
	\xrightarrow[C_2 \to C_2 - C_1]{R_2 \to R_2 - R_1}
	\begin{pmatrix}
		1 & 0 & 0 & 0 \\
		0 & -1 & 0 & 0 \\
		0 & 0 & -4 & 2 \\
		0 & 0 & 2 & -4
	\end{pmatrix}
	\xrightarrow[C_4 \to C_4 + \frac{1}{2}C_3]{R_4 \to R_4 + \frac{1}{2}R_2}
	\begin{pmatrix}
		1 & 0 & 0 & 0 \\
		0 & -1 & 0 & 0 \\
		0 & 0 & -4 & 0 \\
		0 & 0 & 0 & -3
	\end{pmatrix}
\end{align*}
מצאנו מטריצה אלכסונית וממנה נובע $\rho(q) = 4$ ומהגדרת החתימה נובע כי $\sigma = -2$.

\subsection{סעיף ב'}
נמצא תת־מרחב מממד מקסימלי של $\RR^4$ שעליו $q$ היא תבנית חיובית לחלוטין. \\*
בסעיף הקודם מצאנו מטריצה מייצגת אלכסונית של $q$, נוכל לבנות לפיו בסיס עבורו $q$ אלכסונית. נגדיר בסיס זה להיות $W$. \\*
אז בבסיס זה
\[
	q(x_1, x_2, x_3, x_4) = x_1^2 - x_2^2 - 4x_3^2 - 3x_4^2 \tag{1}
\]
עבור תת־המרחב $\Sp\{ w_1 \}$ התבנית $q$ חיובית לחלוטין.
אילו היה תת־המרחב המקסימלי המקיים תנאי זה גדול בממדו מ־$1$ אז היה קיים וקטור $u$ בלתי תלוי ב־$w_1$ אשר מקיים
\[
	q(u) \ge 0
\]
בשל היותו בלתי תלוי ב־$w_1$ כאשר $u = \lambda_1 w_1 + \lambda_2 w_2 + \lambda_3 w_3 + \lambda_4 w_4$
אנו יודעים כי לפחות אחד הערכים $\lambda_2, \lambda_3, \lambda_3 \ne 0$. \\*
נוכל ללא פגיעה בכלליות להניח ש־$\lambda_1 = 0$ שאם לא כן נוכל להגדיר מחדש את הווקטור כך. \\*
מ־$(1)$ נובע במצב זה ש־$q(v) < 0$ בסתירה להנחה ולכן ממד המרחב המקסימלי אשר עבורו $q$ חיובית לחלוטין הוא $1$. \\*
נעבור לחישוב $w_1$. אנו יודעים כי וקטור זה הוא ווקטור העמודה הראשון במטריצת המעבר מהבסיס הסטנדרטי ל־$W$,
ומטריצה זו מתקבלת מביצוע פעולות השורה שביצענו בחפיפה האלמנטרית בסעיף הקודם, נפעיל אותם על מטריצת היחידה ונקבל
\[
	w_1 = (2, 1, 0, 0)
\]
ובהתאם תת־המרחב המקסימלי הוא
\[
	\Sp\{ (2, 1, 0, 0) \}
\]

\section{שאלה 2}
יהי $V$ מרחב וקטורי אשר ממדו $n$, ותהי $q$ תבנית ריבועית חיובית למחצה מעל המרחב $V$. נגדיר $\rho$ דרגת $q$.\\*
נוכיח כי
\[
	L_0 = \left\{ v \in V \mid q(v) = 0 \right\}
\]
הוא תת־מרחב של $V$ ושמתקיים $\dim L_0 = n - \rho$.
\begin{proof}
	משהו
\end{proof}

\end{document}
