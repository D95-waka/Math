\documentclass[a4paper]{article}

% packages
\usepackage{inputenc, fontspec, amsmath, amsthm, amsfonts, polyglossia, catchfile}
\usepackage[a4paper, margin=50pt, includeheadfoot]{geometry} % set page margins

% style
\AddToHook{cmd/section/before}{\clearpage}	% Add line break before section
\linespread{1.5}
\setcounter{secnumdepth}{0}		% Remove default number tags from sections
\setmainfont{Libertinus Serif}
\setsansfont{Libertinus Sans}
\setmonofont{Libertinus Mono}
\setdefaultlanguage{hebrew}
\setotherlanguage{english}

% operators
\DeclareMathOperator\cis{cis}
\DeclareMathOperator\Sp{Sp}
\DeclareMathOperator\tr{tr}
\DeclareMathOperator\im{Im}
\DeclareMathOperator\diag{diag}
\DeclareMathOperator*\lowlim{\underline{lim}}
\DeclareMathOperator*\uplim{\overline{lim}}

% commands
\renewcommand\qedsymbol{\textbf{משל}}
\newcommand{\NN}[0]{\mathbb{N}}
\newcommand{\ZZ}[0]{\mathbb{Z}}
\newcommand{\QQ}[0]{\mathbb{Q}}
\newcommand{\RR}[0]{\mathbb{R}}
\newcommand{\CC}[0]{\mathbb{C}}
\newcommand{\getenv}[2][] {
  \CatchFileEdef{\temp}{"|kpsewhich --var-value #2"}{\endlinechar=-1}
  \if\relax\detokenize{#1}\relax\temp\else\let#1\temp\fi
}
\newcommand{\explain}[2] {
	\begin{flalign*}
		 && \text{#2} && \text{#1}
	\end{flalign*}
}

% headers
\getenv[\AUTHOR]{AUTHOR}
\author{\AUTHOR}
\date\today


\title{פתרון ממ"ן 11 – אלגברה לינארית 2 (20229)}

\begin{document}
\maketitle

\section{שאלה 1}
\subsection{סעיף א'}
יהיה $V = M_{n \times n}^\CC$ עם המכפלה הפנימית הסטנדרטית, תהי $P \in V$ מטריצה הפיכה,
ותהי העתקה לינארית $T_P: V \to V$ המוגדרת
\[
	T_P X = P^{-1} X P
\]
נוכיח שמתקיים $T_P^* = T_{P^*}$. \\*
נגדיר $A, B \in V$. נשים לב כי לפי משפט 2.1.4(ו') מתקיים
\[
	(P^{-1})^* = (P^*)^{-1} \tag{1}
\]
נראה כי
\begin{align*}
	(A, T_{P^*} B)
	& = (A, (P^*)^{-1} B P^*) \\
	& = \tr( ((P^*)^{-1} B P^*)^* A ) \\
	& = \tr( (P^*)^* ((P^*)^{-1} B)^* A ) & \text{2.1.4 ו'} \\
	& = \tr( (P^*)^* B^* ((P^*)^{-1})^* A ) \\
	& = \tr( (P^*)^* B^* ((P^*)^{-1})^* A ) \\
	& = \tr( P B^* P^{-1} A ) & (1) \\
	& = \tr( B^* P^{-1} A P ) & \text{מכפלה ב־tr} \\
	& = (P^{-1} A P, B ) \\
	& = (T_P A, B ) \\
\end{align*}
על־פי הגדרת העתקה צמודה מתקיים
\[
	(T_P)^* = (T_{P^*})
\]

\subsection{סעיף ב'}
נגדיר $V = M^\CC_{2 \times 2}$.
תהי $T_P : V \to V$ המוגדרת על־ידי $T_P X = P^{-1} X P$ כאשר
\[
	P = \begin{pmatrix}
		i & 1 \\
		-1 & -i
	\end{pmatrix}
\]
נמצא את המטריצה המייצגת את $(T_P)^*$ בבסיס הסטנדרטי של $V$. \\*
על־פי הסעיף הקודם מתקיים $(T_P)^* = T_{P^*}$, ולפי חישוב
\[
	P^* = \begin{pmatrix}
		-i & -1 \\
		1 & i
	\end{pmatrix},
	(P^*)^{-1} = \frac{1}{2}\begin{pmatrix}
		i & 1 \\
		-1 & -i
	\end{pmatrix}
\]
נגדיר $B = (E_1, E_2, E_3, E_4)$ הבסיס הסטנדרטי של $V$ ונחשב:
\[
	T_{P^*} E_1 = (P^*)^{-1} E_1 P^*
	= \frac{1}{2}
	\begin{pmatrix} -i & -1 \\ 1 & i \end{pmatrix}
	\begin{pmatrix} 1 & 0 \\ 0 & 0 \end{pmatrix}
	\begin{pmatrix} i & 1 \\ -1 & -i \end{pmatrix}
	= \frac{1}{2}
	\begin{pmatrix} -i & 0 \\ 1 & 0 \end{pmatrix}
	\begin{pmatrix} i & 1 \\ -1 & -i \end{pmatrix}
	= \frac{1}{2} \begin{pmatrix} 1 & -i \\ i & 1 \end{pmatrix}
\]
\[
	T_{P^*} E_2 = \frac{1}{2}
	\begin{pmatrix} -i & -1 \\ 1 & i \end{pmatrix}
	\begin{pmatrix} 0 & 1 \\ 0 & 0 \end{pmatrix}
	\begin{pmatrix} i & 1 \\ -1 & -i \end{pmatrix}
	= \frac{1}{2}
	\begin{pmatrix} 0 & -i \\ 0 & 1 \end{pmatrix}
	\begin{pmatrix} i & 1 \\ -1 & -i \end{pmatrix}
	= \frac{1}{2} \begin{pmatrix} i & -1 \\ -1 & -i \end{pmatrix}
\]
\[
	T_{P^*} E_3 = \frac{1}{2}
	\begin{pmatrix} -i & -1 \\ 1 & i \end{pmatrix}
	\begin{pmatrix} 0 & 0 \\ 1 & 0 \end{pmatrix}
	\begin{pmatrix} i & 1 \\ -1 & -i \end{pmatrix}
	= \frac{1}{2}
	\begin{pmatrix} -1 & 0 \\ i & 0 \end{pmatrix}
	\begin{pmatrix} i & 1 \\ -1 & -i \end{pmatrix}
	= \frac{1}{2} \begin{pmatrix} -i & -1 \\ -1 & i \end{pmatrix}
\]
\[
	T_{P^*} E_4 = \frac{1}{2}
	\begin{pmatrix} -i & -1 \\ 1 & i \end{pmatrix}
	\begin{pmatrix} 0 & 0 \\ 0 & 1 \end{pmatrix}
	\begin{pmatrix} i & 1 \\ -1 & -i \end{pmatrix}
	= \frac{1}{2}
	\begin{pmatrix} 0 & -1 \\ 0 & i \end{pmatrix}
	\begin{pmatrix} i & 1 \\ -1 & -i \end{pmatrix}
	= \frac{1}{2} \begin{pmatrix} 1 & i \\ -i & 1 \end{pmatrix}
\]
לכן 
\[
	[T_{P^*}]_B = \begin{pmatrix}
		1 & -i & i & 1 \\
		i & -1 & -1 & -i \\
		-i & -1 & -1 & i \\
		1 & i & -i & 1
	\end{pmatrix}
\]

\section{שאלה 2}
\subsection{סעיף א'}
יהיו $P, Q \in M^\RR_{n \times n}$ ותהי $U = P + iQ$. נסמן
\[
	D = \begin{pmatrix}
		P & -Q \\
		Q & P
	\end{pmatrix}
\]
נוכיח שאם $U$ מטריצה הרמיטית אז $D$ מטריצה סימטרית. \\*
נניח כי $U$ הרמיטית, לכן לפי 2.1.4:
\[
	U^* = (P + iQ)^* = P^* - i Q^* = P + i Q
\]
בשל היות $P, Q$ ממשיות מתקיים:
\[
	P^* = P^t = P, -Q^* = -Q^t = Q
\]
לכן $P$ מטריצה סימטרית ו־$Q$ מטריצה אינטי־סימטרית, ובהתאם:
\[
	D^t = \begin{pmatrix}
		P^t & Q^t \\
		(-Q)^t & P^t
	\end{pmatrix}
	= \begin{pmatrix}
		P & -Q \\
		Q & P
	\end{pmatrix}
	= D
\]
המטריצה $D$ היא סימטרית.

\subsection{סעיף ב'}
נוכיח כי אם $U$ מטריצה אוניטרית אז $D$ מטריצה אורתוגונלית. \\*
נניח כי $U$ אוניטרית, לכן
\[
	U U^* = I \rightarrow (P + i Q)(P^t - i Q^t) = P P^t + i Q P^t - i P Q^t + Q Q^t = I \rightarrow PP^t + QQ^t = I, QP^t = PQ^t
\]
נחשב
\[
	D D^t
	= \begin{pmatrix}
		P & -Q \\
		Q & P
	\end{pmatrix} \begin{pmatrix}
		P^t & Q^t \\
		-Q^t & P^t
	\end{pmatrix}
	= \begin{pmatrix}
		PP^t + QQ^t & PQ^t - QP^t \\
		QP^t - PQ^t & QQ^t + PP^t
	\end{pmatrix}
	= \begin{pmatrix}
		I_n & 0 \\
		0 & I_n
	\end{pmatrix}
	= I_{2n}
\]
לכן $D$ מטריצה אורתוגונלית.

\section{שאלה 3}
תהינה $A, B$ מטריצות חיוביות לחלוטין ו־$Q$ מטריצה אוניטרית. \\*
נוכיח כי אם $A = BQ$ אז $A = B$. \\*
יהי $\lambda$ ערך עצמי של $Q$ ו־$u$ וקטור עצמי עבור $\lambda$, לכן $A u = B Q u = \lambda B u$. \\*
נשים לב כי
\[
	(A u, u) = (B Q u, u) = \lambda (B u, u)
\]
ידוע כי $A, B$ חיוביות לחלוטין, לכן $(Au, u), (Bu, u) > 0$, ובהתאם גם
\[
	\lambda = \frac{(A u, u)}{(B u, u)} > 0
\]
לכן בכלל גם $\lambda \in \RR$, אז לפי טענה 2.4.3 ואי־השוויון מתקיים $\lambda = 1$ וזהו הערך העצמי היחיד של המטריצה. \\*
לפי שאלה 2.3.5 ב' אנו יכולים להסיק כי קיימת מטריצה אלכסונית שדומה ל־$Q$, ואנו יודעים כי כלל ערכיה העצמיים הם $1$, לכן $Q$ דומה ל־$I$,
לכן קיימת מטריצה הפיכה $P$ כך שמתקיים $Q = P^{-1} I P = P^{-1} P = I$ לכן $Q = I$. \\*
אז גם $A = B Q = B I = B$.

\section{שאלה 4}
יהי $0 \ne w \in \CC^n$ וקטור עמודה. נמצא תנאי הכרחי ומספיק עבור $w$ כדי שהמטריצה $H = I - 2 w w^*$ תהיה אוניטרית. \\*
נשים לב כי $H^* = (I - 2 ww^*)^* = I^* - (2ww^*)^* = I - 2(ww^*)^*$. \\*
ידוע גם כי $w^* = \overline{w}^t$, וגם $\overline{(w w^*)}^t = (\overline{w} w^t)^t = w w^*$ לפי חוקי שחלוף מהקורס הקודם. \\*
בסך־הכול מתקיים $H^* = H$. נחשב:
\begin{align*}
	H H^* & = H^* H = H^2 = (I - 2w w^*)(I - 2w w^*) \\
	& = I^2 - 2 \cdot I \cdot 2w w^* + 4(ww^*)^2 = I - 4ww^* + 4w \lVert w \rVert w^* \\
	& = I - 4 ww^* (1 - \lVert w \rVert)
\end{align*}
אנו יודעים כי $w \ne 0$ ולכן גם $ww^* \ne 0$, בהתאם מתקיים $H H^* = I$ כאשר $\lVert w \rVert = 1$. \\*
נשים לב כי במקרה זה $H$ היא מטריצת שיקוף ביחס ל־$\{ w \}^\perp$:
\[
	H w = I w - 2w w^* w = w - 2 \lVert w \rVert w = w - 2w = -w
\]
\[
	\forall v \in \{ w \}^\perp: H v = I v - 2 w w^* v = v - 2w 0 = v
\]

\section{שאלה 5}
יהי $V$ מרחב מכפלה פנימית, $w_1, w_2 \in V$ וקטורים המקיימים
\[
	(w_1, w_2) = 0, \lVert w_1 \rVert = \lVert w_2 \rVert = 1
\]
נגדיר העתקה לינארית $T: V \to V$ כך שמתקיים
\[
	Tv = v - 2(v, w_1) w_1 - 2(v, w_2) w_2
\]
\subsection{סעיף א'}
נוכיח כי $T$ צמודה לעצמה ואוניטרית. \\*
על־פי למה 1.2.3:
\begin{align*}
	(T v, u)
	& = (v - 2(v, w_1) w_1 - 2(2, w_2) w_2, u) \\
	& = (v, u) - 2 (v, w_1) (w_1, u) - 2(v, w_2) (w_2, u) \\
	& = (v, u) - (v, 2 (u, w_1) w_1) - (v, (u, w_2) 2w_2) \\
	& = (v, u) - (v, 2 (u, w_1) w_1) - (v, (u, w_2) 2w_2) \\
	& = (v, u - 2 (u, w_1) w_1 - (u, w_2) 2w_2) \\
	(T v, u) & = (v, T u)
\end{align*}
לכן $T$ צמודה לעצמה. \*
נחשב
\begin{align*}
	(T v, T v)
	= & (v - 2(v, w_1) w_1 - 2(2, w_2) w_2, v - 2(v, w_1) w_1 - 2(2, w_2) w_2) \\
	= & (v, v - 2(v, w_1) w_1 - 2(2, w_2) w_2) \\
	& - ((v, w_1) w_1, v - 2(v, w_1) w_1 - 2(2, w_2) w_2) \\
	& - 2((v, w_2) w_2, v - 2(v, w_1) w_1 - 2(2, w_2) w_2) \\
	= & (v, v) - 2 (v, w_1)^2 - 2(v, w_2)^2 \\
	& - 2(v, w_1)^2 + 4(v, w_1)^2  + 4(v, w_2)^2 \\
	& - 2(v, w_2)^2 + 4(v, w_1)^2 + 4(2, w_2)^2 \\
	= & \lVert v\rVert^2
\end{align*}
ובהתאם להגדרת הנורמה $\lVert T v \rVert = \lVert v \rVert$ ולכן לפי משפט 2.3.2 $T$ אוניטרית.

\subsection{סעיף ב'}
נבדוק אם $T$ אי־שלילית. \\*
נחשב את $(T w_1, w_1)$:
\[
	(T w_1, w_1)
	= (w_1, w_1) - 2 (w_1, w_1)^2 - 2(w_1, w_2)^2
	= \lVert w_1 \rVert - 2\lVert w_1 \rVert^2 - 0
	= -1
\]
לכן $T$ איננה אי־שלילית.

\end{document}
