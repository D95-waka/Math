\documentclass[a4paper]{article}

% packages
\usepackage{inputenc, fontspec, amsmath, amsfonts, polyglossia, catchfile}
\usepackage[a4paper, margin=50pt, includeheadfoot]{geometry} % set page margins

% style
\AddToHook{cmd/section/before}{\clearpage}	% Add line break before section
\setdefaultlanguage{hebrew}
\setotherlanguage{english}
\setmainfont{Libertinus Serif}
\linespread{1.5}
\setcounter{secnumdepth}{0}		% Remove default number tags from sections

% custom operators
\newcommand{\getenv}[2][]{%
  \CatchFileEdef{\temp}{"|kpsewhich --var-value #2"}{\endlinechar=-1}%
  \if\relax\detokenize{#1}\relax\temp\else\let#1\temp\fi}
\getenv[\AUTHOR]{AUTHOR}
\DeclareMathOperator\cis{cis}
\DeclareMathOperator\Sp{Sp}
\DeclareMathOperator\tr{tr}
\DeclareMathOperator\im{Im}
\DeclareMathOperator\diag{diag}
\DeclareMathOperator*\lowlim{\underline{lim}}
\DeclareMathOperator*\uplim{\overline{lim}}
\def\NN{\mathbb{N}}
\def\RR{\mathbb{R}}
\def\CC{\mathbb{C}}

\title{פתרון ממ''ן 14 – אלגברה לינארית 1 (20109)}
\author{\AUTHOR}
\date\today

\begin{document}
\maketitle
\section{שאלה 1}
\subsection{סעיף א'}
יהיו $U, W_1, W_2$ תת־מרחבים של מרחב לינארי $V$.
נוכיח שמתקיים:
\[
	(U \cap W_1) +
	(U \cap W_2) \subseteq
	U \cap (W_1 + W_2)
	\tag{*}
\]
לפי הגדרת חיבור הקבוצות
$W_1 + W_2$ היא כלל הווקטורים המשמשים כצירוף לינארי של וקטורי $W_1, W_2$.
הקבוצות $U \cap W_1$ ו־$U \cap W_2$ שתיהן מורכבות מווקטורים שמוכלים ב־$W_1$ ו־$W_2$ בהתאמה, בהתאם להגדרת חיתוך קבוצות.
לכן גם חיבור הקבוצות $(U \cap W_1) + (U \cap W_2)$
יכיל צירופים לינאריים של וקטורים הקיימים ב־$W_1, W_2$, אז מתקיים
\[
	(U \cap W_1) +
	(U \cap W_2) \subseteq
	W_1 + W_2
\]
באופן דומה אנו רואים כי גם
$(U \cap W_1) + (U \cap W_2) \subseteq U$,
שהרי כל וקטור ב־$(U \cap W_1) + (U \cap W_2)$
מוכל גם ב־$U$ וגם ב־$W_1 + W_2$,
לכן בוודאי שהוא מוכל גם ב־$U \cap (W_1 + W_2)$,
דהינו $(U \cap W_1) + (U \cap W_2) \subseteq U \cap (W_1 + W_2)$.

\subsection{סעיף ב'}
נמצא שלושה תת־מרחבים $U, W_1, W_2 \subseteq \RR^2$
כך שיתקיים (*) ביחס הכלה ממש. \\*
נגדיר
\begin{align*}
	& U = \Sp\{(1, 1)\} \\
	& W_1 = \Sp\{(1, 0)\} \\
	& W_2 = \Sp\{(0, 1)\}
\end{align*}
אז מתקיים:
\[
	\begin{aligned}
		& (U \cap W_1) = (U \cap W_2) = O \\
		& (U \cap W_1) + (U \cap W_2) = O \\
		& W_1 + W_2 = \RR^2 \\
		& U \cap (W_1 + W_2) = \Sp\{(1, 1)\} \\
		& O \subset \Sp\{(1, 1)\}
	\end{aligned}
\]
ולכן
\[
	(U \cap W_1) + (U \cap W_2) \subset U \cap (W_1 + W_2)
\]

\section{שאלה 2}
\subsection{סעיף א'}
יהיה מרחב $V$ ושני תתי־מרחבים $U, W \subseteq V$ כך ש־$U \ne W$.
נתונים $\{u_1, u_2\}$ בסיס ל־$U$ ו־$\{w_1, w_2\}$ בסיס ל־$W$.
נוכיח כי אם הקבוצה $\{u_1, u_2, w_1\}$ תלויה לינארית אז $w_1 \in U \cap W$: \\*
ידוע כי $\{u_1, u_2, w_1\}$ תלויה לינארית,
אך הקבוצה $\{u_1, u_2\}$ בלתי תלויה לינארית,
לכן נוכל להניח כי הווקטור $w_1$ תלוי לינארית ב־$u_1, u_2$
ללא חשש כי התלות היא בין $u_1$ ל־$u_2$ בלבד.
לכן גם $w_1 \in U$ על־פי הגדרת הבסיס.
ידוע כי $w_1 \in W$ מעצם היותו מרכיב בבסיס ל־$W$.
ראינו כי $w_1 \in U, W$, לכן בוודאי שגם $w_1 \in U \cap W$.
\subsection{סעיף ב'}
נמצא את $\dim(U + W)$. \\*
לפי הגדרת חיבור מרחבים והגדרת קבוצת יוצרים:
\[
	U + W
	= \Sp\{ u_1, u_2 \} + \Sp\{ w_1, w_2 \}
	= \Sp\{ u_1, u_2, w_1, w_2 \}
\]
ידוע כי $w_1$ תלוי לינארית ב־$u_1, u_2$,
לכן נוכל להסיר את $w_1$ מהקבוצה הפורשת ללא שינוי במרחב הנוצר:
\[
	U + W
	= \Sp\{ u_1, u_2, w_1, w_2 \}
	= \Sp\{ u_1, u_2, w_2 \}
\]
ידוע כי $U \ne W$ וכי $w_1 \in U$, אילו היה מתקיים גם $w_2 \in U$,
ניתן היה לתאר את שני המרחבים בעזרת אותה קבוצת יוצרים ו־$U = W$
בסתירה לטענה, לכן $w_2$ איננו תלוי לינארית ב־$u_1, u_2$.
לפיכך $\{u_1, u_2, w_2\}$ היא קבוצה בלתי תלויה לינארית המהווה בסיס ל־$U + W$.
על־פי הגדרת הממד, אנו יודעים שערך ממד מרחב הוא ככמות הווקטורים בבסיסו,
לכן $\dim(U + W) = 3$.

\section{שאלה 3}
\subsection{סעיף א'}
יהיו $U, W$ תת־המרחבים הבאים של $\RR_4[x]$:
\begin{align*}
	& U = \Sp\{ x^3 + 4x^2 - x + 3,
	x^3 + 5x^2 + 5,
	3x^3 + 10x^2 + 5 \} \\
	& W = \Sp\{ x^3 + 4x^2 + 6,
	x^3 + 2x^2 - x + 5,
	2x^3 + 2x^2 - 3x + 9\}
\end{align*}
נמצא את הבסיס והממד עבור $U, W, U + W$: \\*
תחילה נגדיר את הבסיס הסדור הסטנדרטי ל־$\RR_4[x]$:
\[
	E = \begin{bmatrix}
		1 \\
		x \\
		x^2 \\
		x^3
	\end{bmatrix}
\]
נמצא את הבסיס ל־$U$: \\*
נמצא את הבסיס לתת־המרחב על־ידי המרה למרחב $\RR^4$ על־ידי
קורדינטה לפי $E$. נגדיר:
\begin{align*}
	U' = {[U]}_E
	& = \Sp\{ {[x^3 + 4x^2 - x + 3]}_E,
	{[x^3 + 5x^2 + 5]}_E,
	{[3x^3 + 10x^2 + 5]}_E \} \\
	& = \Sp\{ (1, 4, -1, 3),
	(1, 5, 0, 5),
	(3, 10, 0, 5) \}
\end{align*}
נמצא בסיס ל־$U'$ לפי שאלה 7.5.12, נבנה את מטריצת השורות המייצגת את הקבוצה
הפורשת של $U'$, כל מטריצה שקולת שורות למטריצה זו מייצגת קבוצה פורשת שקולה
לקבוצת היוצרים המקורית:
\begin{align*}
	& \begin{bmatrix}
		1 & 4 & -1 & 3 \\
		1 & 5 & 0 & 5 \\
		3 & 10 & 0 & 5
	\end{bmatrix}
	\xrightarrow[R_3 \to R_3 - 3R_1]{R_2 \to R_2 - R_1}
	\begin{bmatrix}
		1 & 4 & -1 & 3 \\
		0 & 1 & 1 & 2 \\
		0 & -2 & 3 & -4
	\end{bmatrix}
	\xrightarrow{R_3 \to R_3 + 2R_2}
	\begin{bmatrix}
		1 & 4 & -1 & 3 \\
		0 & 1 & 1 & 2 \\
		0 & 0 & 5 & 0
	\end{bmatrix}
	\xrightarrow{R_3 \to R_3 / 5} \\
	& \begin{bmatrix}
		1 & 4 & -1 & 3 \\
		0 & 1 & 1 & 2 \\
		0 & 0 & 1 & 0
	\end{bmatrix}
	\xrightarrow[R_2 \to R_2 - R_3]{R_1 \to R_1 + R_3}
	\begin{bmatrix}
		1 & 4 & 0 & 3 \\
		0 & 1 & 0 & 2 \\
		0 & 0 & 1 & 0
	\end{bmatrix}
	\xrightarrow{R_1 \to R_1 - 4R_2}
	\begin{bmatrix}
		1 & 0 & 0 & -5 \\
		0 & 1 & 0 & 2 \\
		0 & 0 & 1 & 0
	\end{bmatrix}
\end{align*}
אנו רואים כי המטריצה לא שקולה למטריצה עם שורת אפסים,
לכן הקבוצה הפורשת של $U'$ בלתי תלויה לינארית ומתקיים:
\[
	U' = \Sp\{ (1, 0, 0, -5), (0, 1, 0, 2), (0, 0, 1, 0) \}
\]
בשל החד־חד ערכיות במעבר בין $U$ ל־$U'$ לפי משפט 8.4.2
מתקיים גם:
\[
	U = \Sp\{ x^3 - 5, x^2 + 2, x\} \tag{*}
\]
שכן קבוצה פורשת זו במעבר קורדינטות שווה ל־$(*)$.
ראינו כי הקבוצה הפורשת של $U'$ בלתי תלויה לינארית,
לכן לפי משפט 8.4.4 גם הקבוצה הפורשת את $U$ היא בלתי תלויה לינארית
ומהווה בסיס ל־$U$. בשל כך מתקיים גם $\dim U = 3$. \\*
נמצא את הבסיס ל־$W$: \\*
נפעל בדומה לדרך המציאה עבור $U$ ונגדיר:
\[
	W' = \Sp\{ {[x^3 + 4x^2 + 6]}_E,
	{[x^3 + 2x^2 - x + 5]}_E,
	{[2x^3 + 2x^2 - 3x + 9]}_E\}
	= \Sp\{ (1, 4, 0, 6),
	(1, 2, -1, 5),
	(2, 2, -3, 9)\}
\]
נבדוק אם קבוצה פורשת זו תלויה לינארית על־ידי המרה למטריצת שורות ודירוגה:
\[
	\begin{bmatrix}
		1 & 4 & 0 & 6 \\
		1 & 2 & -1 & 5 \\
		2 & 2 & -3 & 9
	\end{bmatrix}
	\xrightarrow[R_3 \to R_3 - 2R_1]{R_2 \to R_2 - R_1}
	\begin{bmatrix}
		1 & 4 & 0 & 6 \\
		0 & -2 & -1 & -1 \\
		0 & -6 & -3 & -3
	\end{bmatrix}
	\xrightarrow[R_3 \to -R_3]{R_2 \to -R_2}
	\begin{bmatrix}
		1 & 4 & 0 & 6 \\
		0 & 2 & 1 & 1 \\
		0 & 6 & 3 & 3
	\end{bmatrix}
	\xrightarrow[R_3 \to R_3 - 3R_2]{R_1 \to R_1 - 2R_2}
	\begin{bmatrix}
		1 & 0 & -2 & 4 \\
		0 & 2 & 1 & 1 \\
		0 & 0 & 0 & 0
	\end{bmatrix}
\]
ניתן לראות כי המטריצה שקולה למטריצה עם שורת אפס,
לכן הקבוצה תלויה לינארית ומתקיים:
\[
	W' = \Sp\{ (1, 0, -2, 4), (0, 2, 1, 1) \}
\]
על־פי החד־חד ערכיות של מעבר הקורדינטה:
\[
	W = \Sp\{x^3 - 2x + 4, 2x^2 + x + 1\}
\]
לכן הקבוצה $\{x^3 - 2x + 4, 2x^2 + x + 1\}$ מהווה בסיס ל־$W$,
ומהגדרה 8.3.3 נובע ש־$\dim W = 2$. \\*
נמצא את הבסיס ל־$U + W$: \\*
תחילה נשתמש בהגדרת חיבור המרחבים ונראה כי:
\[
	U + W
	= \Sp\{x^3 - 5, x^2 + 2, x \} + \Sp\{x^3 - 2x + 4, 2x^2 + x + 1\}
\]
על־פי שאלה 7.6.8:
\[
	U + W
	= \Sp\{x^3 - 5, x^2 + 2, x, x^3 - 2x + 4, 2x^2 + x + 1\}
\]
כמו במציאת הבסיסים ל־$U$ ו־$W$,
גם עתה נגדיר קבוצה חדשה $V = {[U + W]}_E$ ונמצא אם היא בסיס על־ידי
שימוש במטריצת שורות ודירוגה:
\begin{align*}
	& \begin{bmatrix}
		1 & 0 & 0 & -5 \\
		0 & 1 & 0 & 2 \\
		0 & 0 & 1 & 0 \\
		1 & 0 & -2 & 4 \\
		0 & 2 & 1 & 1 \\
	\end{bmatrix}
	\xrightarrow[R_5 \to R_5 - 2R_2 - R_3]{R_4 \to R_4 - R_1 + 2R_3}
	\begin{bmatrix}
		1 & 0 & 0 & -5 \\
		0 & 1 & 0 & 2 \\
		0 & 0 & 1 & 0 \\
		0 & 0 & 0 & 9 \\
		0 & 0 & 0 & -3 \\
	\end{bmatrix}
	\xrightarrow{R_4 \to R_4 / 9} \\
	& \begin{bmatrix}
		1 & 0 & 0 & -5 \\
		0 & 1 & 0 & 2 \\
		0 & 0 & 1 & 0 \\
		0 & 0 & 0 & 1 \\
		0 & 0 & 0 & -3 \\
	\end{bmatrix}
	\xrightarrow[R_2 \to R_2 - 2R_4, R_5 \to R_5 + 3R_1]{R_1 \to R_1 + 5R_4}
	\begin{bmatrix}
		1 & 0 & 0 & 0 \\
		0 & 1 & 0 & 0 \\
		0 & 0 & 1 & 0 \\
		0 & 0 & 0 & 1 \\
		0 & 0 & 0 & 0 \\
	\end{bmatrix}
\end{align*}
הקבוצה הפורשת של $V$ מתלכדת עם הבסיס הסטנדרטי של $\RR^4$,
ולכן בהתאם $U + W = \RR_4[x]$ ו־$\dim(U + W) = 4$.
\subsection{סעיף ב'}
נמצא בסיס לתת־המרחב $V = U \cap W$.
נציב ערכים במשוואה המתקיימת ממשפט 8.3.6:
\[
	\begin{aligned}
		& \dim(U + W) = \dim U + \dim W - \dim V \\
		& 4 = 2 + 3 - \dim V \\
		& \dim V = 1
	\end{aligned}
\]
$V$ מוגדר כחיתוך של שני תתי־מרחב, לכן הוא מכיל את כלל הווקטורים אשר
מופיעים בשני תתי־המרחב.
דהינו קבוצת הסקלרים $\lambda_1, \hdots, \lambda_5$
המקיימים את המשוואה:
\[
	\lambda_1 (x^3 - 5) + \lambda_2 (x^2 + 2) + \lambda_3 (x)
	= \lambda_4 (x^3 - 2x + 4) + \lambda_5 (2x^2 + x + 1)
\]
נעביר אגפים:
\[
	\lambda_1 (x^3 - 5) + \lambda_2 (x^2 + 2) + \lambda_3 (x) +
	\lambda_4 (-x^3 + 2x - 4) + \lambda_5 (-2x^2 - x - 1)
	= 0
\]
זוהי משוואה לינארית הומוגנית,
נמצא את קבוצת הפתרונות שלה על־ידי המרה למטריצת מקדמים מצומצמת ודירוגה:
\[
	\begin{bmatrix}
		1 & 0 & 0 & -1 & 0 \\
		0 & 1 & 0 & 0 & -2 \\
		0 & 0 & 1 & 2 & -1 \\
		-5 & 2 & 0 & -4 & -1
	\end{bmatrix}
	\xrightarrow{R_4 \rightarrow R_4 + 5R_1 - 2R_2}
	\begin{bmatrix}
		1 & 0 & 0 & -1 & 0 \\
		0 & 1 & 0 & 0 & -2 \\
		0 & 0 & 1 & 2 & -1 \\
		0 & 0 & 0 & -9 & 3
	\end{bmatrix}
	\xrightarrow{R_4 \rightarrow R_4 / 3}
	\begin{bmatrix}
		1 & 0 & 0 & -1 & 0 \\
		0 & 1 & 0 & 0 & -2 \\
		0 & 0 & 1 & 2 & -1 \\
		0 & 0 & 0 & -3 & 1
	\end{bmatrix}
\]
נבצע הצבה לאחור:
\[
	\begin{aligned}
		& -3\lambda_4 - \lambda_5 = 0 & \rightarrow 
		& \lambda_5 = 3\lambda_4 \\
		& \lambda_3 + 2 \lambda_4 - \lambda_5 = 0 & \rightarrow
		& \lambda_3 = \lambda_4 \\
		& \lambda_2 - 2 \lambda_5 = 0 & \rightarrow
		& \lambda_2 = 6 \lambda_4 \\
		& \lambda_1 - \lambda_4 = 0 & \rightarrow
		& \lambda_1 = \lambda_4
	\end{aligned}
\]
נגדיר $\lambda_4 = t$, הווקטור על־פי הצבת ערכי הסקלרים
\[
	t (x^3 - 5) + 6t (x^2 + 2) + t (x)
\]
נקבץ ערכים ונבצע כפל בסקלר:
\[
	t (x^3 + 6x^2 + x + 7)
\]
לכן $V = \Sp\{ (x^3 + 6x^2 + x + 7) \}$

\subsection{סעיף ג'}
נמצא תת־מרחב $T \subseteq \RR_4[x]$
כך שמתקיים $W \oplus T = \RR_4[x]$. \\*
על־פי משפט 8.3.6 והגדרת החיבור הישר:
\[
	\begin{aligned}
		& \dim(W + T) = \dim W + \dim T - \dim (W \cap T) \\
		& 4 = 2 + \dim T - 0 \\
		& \dim T = 2
	\end{aligned}
\]
נזכיר
\[
	W = \Sp\{x^3 - 2x + 4, 2x^2 + x + 1\}
\]
בקבוצת היוצרים של $W$ מופיעים המקדמים $x^3$ ו־$x^2$,
נגדיר $T = \Sp\{1, x\}$ ונבדוק אם $T$ עומד בדרישות. \\*
ניתן לראות כי ארבעת הווקטורים ב־$W + T$ אינם תלויים לינארית,
שכן בכל וקטור מופיע מקדם שאיננו קיים בווקטורים האחרים,
לכן $W + T$ הוא בסיס ל־$\RR_4[x]$ ומתקיים $W + T = \RR_4[x]$. \\*
כבר ראינו כי הקבוצה $W + T$ בלתי תלויה לינארית,
לכן לא קיים וקטור מלבד וקטור האפס הקיים בשני תתי־המרחב,
אז $W \cap T = \{0\}$. \\*
לכן לפי משפט 7.7.2 $W \oplus T = \RR_4[x]$.

\section{שאלה 4}
יהיו תת־מרחבים $U, W \subseteq \RR^4$ כך ש־$\dim W < \dim U (**)$.
ידוע כי $U \cap W = \Sp\{(1, 2, 3, 4), (1, 1, 1, 1), (-1, 0, 1, 2)\}$
וגם כי $(0, 0, 1, 0) \not\in U + W$.
נמצא את ממדו של $U + W$. \\*
נמצא בסיס ל־$U \cap W$ על־ידי דירוג מטריצת השורות של קבוצת היוצרים
הפורשים הנתונה לפי שאלה 7.5.12:
\[
	\begin{bmatrix}
		1 & 1 & 1 & 1 \\
		1 & 2 & 3 & 4 \\
		-1 & 0 & 1 & 2
	\end{bmatrix}
	\xrightarrow[R_3 \to R_3 + R_1]{R_2 \to R_2 - R_3}
	\begin{bmatrix}
		1 & 1 & 1 & 1 \\
		0 & 1 & 2 & 3 \\
		0 & 1 & 2 & 3
	\end{bmatrix}
	\xrightarrow{R_3 \to R_3 - R_2}
	\begin{bmatrix}
		1 & 1 & 1 & 1 \\
		0 & 1 & 2 & 3 \\
		0 & 0 & 0 & 0
	\end{bmatrix}
\]
אנו רואים כי $U \cap W = \Sp\{(1, 1, 1, 1), (0, 1, 2, 3)\}$,
קבוצת היוצרים בלתי תלויה כפי שאנו למדים מהדירוג למטריצת מדרגות,
לכן מתקיים $\dim(U \cap W) = 2$.
מהגדרת החיתוך נובע $U \cap W \subseteq U, W$,
לכן לפי משפט 8.3.4 מתקיים $2 \le \dim U, \dim W (\#)$.
ידוע כי קיים  וקטור ב־$\RR^4$ שאיננו מוכל ב־$U + W$,
לכן לא יתכן כי ממד החיבור הוא 4, אז $\dim(U + W) < 4 (*)$.
לפי משפט 8.3.6 מתקיים:
\[
	\begin{aligned}
		& \dim(U + W) = \dim U + \dim W - \dim(U \cap W) & (***) \\
		& \dim(U + W) = \dim U + \dim W - 2 \\
		& \dim U + \dim W - 2 < 4 & (*) \\
		& \dim U + \dim W < 6 \\
		& 4 \le \dim U + \dim W < 6 & (\#) \\
		& \dim U = 3, \dim W = 2 & (**) \\
	\end{aligned}
\]
על־ידי הצבת הערכים המתקבלים ב־$(***)$ מתקבל כי $\dim(U + W) = 3$. \\*
נמצא בסיס ל־$W$. ידוע כי $\Sp\{(1,1,1,1), (0, 1, 2, 3)\} \subseteq W$
וכי $\dim W = 2$, לכן קבוצה זו מהווה בסיס ל־$W$ ומתקיים
\[
	W = Sp\{(1,1,1,1), (0, 1, 2, 3)\}
\]

\section{שאלה 5}
\subsection{סעיף א'}
תהיה $A = [a_{ij}]$ מטריצה ריבועית מסדר $n$ שדרגתה $1$. נוכיח $A^2 = \tr(A)A$. \\*
מהגדרה 8.5.4 אנו למדים כי מטריצה שדרגתה $1$ היא מטריצה בה יש
שורה אחת בלבד שאיננה שורת אפסים, נגדיר $0 \le k \le n$,
וגם כי ${[ a ]}^r_k = (b_0, b_1, \hdots, b_n)$.
עבור כל $j \ne k$ אנו יודעים כי ${[ a ]}^r_j = (0, \hdots, 0)$.
בעזרת מסקנה 3.4.4 מתקבל כי גם ${[ a^2 ]}^r_j = (0, \hdots, 0)$,
דהינו שמלבד השורה $k$ כלל השורות ב־$A^2$ הן שורות אפסים.
בעזרת חישוב מתקבל כי 
\[
	{[ a^2 ]}^r_k = (b_k b_0, b_k b_1, \hdots, b_k b_n)
	= b_k (b_0, b_1, \hdots, b_n) \tag{*}
\]
לכן גם $A^2 = b_k \cdot A$.
האלכסון של המטריצה $A$ מורכב מ־$0$ מלבד האיבר $a_{kk} = b_k$,
לכן גם $\tr(A) = b_k$.
נציב ונקבל שמתקיים $A^2 = \tr(A) A$.

\subsection{סעיף ב'}
אם מתקיים $\tr(A) = 0$ אז $A^2 = 0 A$, לכן $A^2 = 0$.
נראה כי עבור כל $k \in \mathbb{N} \backslash \{0\}$,
אם $\tr(A) \ne 0$ אז גם $A^k \ne 0$.
את פעולה $(*)$ ניתן להחיל שוב ושוב על־ידי הכפלת מטריצה שמהווה חזקה של $A$
ב־$A$ שוב, לכן תמיד תהיה שורה שאיננה ריקה, ומכאן נובע $A^k \ne 0$.

\subsection{סעיף ג'}
נניח שעבור $k \in \NN$ נתון $A^k = 0$ ונוכיח כי אז גם $A^2 = 0$.
אם $k = 1$ אז המטריצה היא מטריצת אפס, וכל מכפלה שלה תוביל למטריצת אפס,
לכן בוודאי שגם $A^2 = 0$.
אם $k = 2$ אז כמובן ש־$A^2 = 0$.
במקרה שבו $k \ge 2$ נשתמש בחוקי מכפלת מטריצות ונכתוב מחדש,
$A^2 A^{k - 2} = 0$.
הראינו בסעיף ב' כי $A^2 = 0$ במקרה זה, לכן $A^k = 0 A^{k - 2} = 0$.

\section{שאלה 6}
יהיה מרחב $V$ הנפרש על־ידי הפונקציות $f, g: \RR \to \RR$
אשר מוגדרות על־ידי $f(x) = \sin x$ ו־$g(x) = \cos x$.
מגדירים את הפונקציות $h, k: \RR \to \RR$ על־ידי
$h(x) = 2 \sin x + \cos x$ ו־$k(x) = 3 \cos x$.

\subsection{סעיף א'}
נוכיח כי הקבוצות $B = \{f, g\}, C = \{h, k\}$ בסיסים של $V$. \\*
ידוע כי המרחב $V$ נפרש על־ידי $f$ ו־$g$,
כדי להוכיח ש־$B$ בסיס עלינו גם לוודא ששתי הפונקציות בלתי תלויות לינארית.
במילים אחרות, עלינו לבדוק ש־$f$ לא פורפורציולית ל־$g$.
על־פי הגדרות הפונקציות הטריגונומטריות $f$ ו־$g$ אכן לא פורפורציונליות,
ולכן $B$ בלתי תלויה לינארית ומהווה בסיס ל־$V$. \\*
נשתמש בשאלה 7.5.11 כדי להראות $\Sp B = \Sp C$:
\[
	\Sp\{\sin x, \cos x\}
	= \Sp\{2 \sin x, \cos x\}
	= \Sp\{2 \sin x + \cos x, \cos x\}
	= \Sp\{2 \sin x + \cos x, 3\cos x\}
\]
לכן גם $C$ פורשת את $V$.
בקבוצה $C$ שני וקטורים בלתי תלויים לינארית, לכן ממדה כממד המרחב $V$ והיא בסיס עבורו.

\subsection{סעיף ב'}
נמצא את מטריצת המעבר מ־$B$ ל־$C$. \\*
נחשב את ${[h]}_B$:
\[
	h(x) = 2 \sin x + \cos x = 2 f(x) + g(x) \rightarrow 
	{[h]}_B = \begin{bmatrix} 2 \\ 1 \end{bmatrix}
\]
נחשב את ${[k]}_B$:
\[
	k(x) = 0 \sin x + 3 \cos x = 0 f(x) + 3 g(x) \rightarrow
	{[k]}_B = \begin{bmatrix} 0 \\ 3 \end{bmatrix}
\]
בעזרת הגדרה 8.4.6 נבנה את מטריצת המעבר $M$:
\[
	M =
	\begin{bmatrix}
		2 & 0 \\
		1 & 3
	\end{bmatrix}
\]
נחשב את המטריצה ההופכית $M^{-1}$ אשר מהווה מטריצת המעבר מ־$C$ ל־$B$
לפי משפט 8.4.9:
\[
	\begin{bmatrix}
		2 & 0 & \vline & 1 & 0 \\
		1 & 3 & \vline & 0 & 1
	\end{bmatrix}
	\xrightarrow{R_1 \to R_1 / 2}
	\begin{bmatrix}
		1 & 0 & \vline & {\frac{1}{2}} & 0 \\
		1 & 3 & \vline & 0 & 1
	\end{bmatrix}
	\xrightarrow{R_2 \to R_2 - R_1}
	\begin{bmatrix}
		1 & 0 & \vline & {\frac{1}{2}} & 0 \\
		0 & 3 & \vline & -{\frac{1}{2}} & 1
	\end{bmatrix}
	\xrightarrow{R_2 \to R_2 / 3}
	\begin{bmatrix}
		1 & 0 & \vline & {\frac{1}{2}} & 0 \\
		0 & 1 & \vline & -{\frac{1}{6}} & {\frac{1}{3}}
	\end{bmatrix}
\]
מטריצת המעבר מ־$C$ ל־$B$ היא
\[
	\begin{bmatrix}
		{\frac{1}{2}} & 0 \\
		-{\frac{1}{6}} & {\frac{1}{3}}
	\end{bmatrix}
\]

\subsection{סעיף ג'}
נתונה פונקציה $l: \RR \to \RR$ המוגדרת על־ידי
$l(x) = 5 \sin x - 2 \cos x$.
נחשב את ${[l]}_C$ בעזרת מטריצת מעבר. \\*
תחילה נחשב את ${[l]}_B$:
\[
	l(x) = 5 \sin x - 2 \cos x = 5 f(x) - 2 g(x) =
	\begin{bmatrix} 5 \\ -2 \end{bmatrix}
\]
נחשב על־ידי מטריצת מעבר לפי משפט 8.4.7:
\begin{align*}
	& {[l]}_C = M^{-1} {[l]}_B \\
	& {[l]}_C =
	\begin{bmatrix}
		{\frac{1}{2}} & 0 \\
		-{\frac{1}{6}} & {\frac{1}{3}}
	\end{bmatrix}
	\begin{bmatrix} 5 \\ -2 \end{bmatrix} \\
	& {[l]}_C =
	\begin{bmatrix}
		{\frac{5}{2}} \\
		-{\frac{3}{2}}
	\end{bmatrix}
\end{align*}

\end{document}
