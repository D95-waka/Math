\documentclass[a4paper,10pt]{article}

% packages
\usepackage{inputenc}
\usepackage{fontspec}
\usepackage{amsmath}
\usepackage{amsfonts}
\usepackage{polyglossia}
\usepackage{geometry}
\usepackage{catchfile}

% style
\newcommand{\getenv}[2][]{%
  \CatchFileEdef{\temp}{"|kpsewhich --var-value #2"}{\endlinechar=-1}%
  \if\relax\detokenize{#1}\relax\temp\else\let#1\temp\fi}
\getenv[\AUTHOR]{AUTHOR}
\setotherlanguage{hebrew}
\setmainfont{Libertinus Serif}
\newfontfamily\hebrewfont{Libertinus Serif}[Script=Hebrew]
\linespread{1.5}
\setcounter{secnumdepth}{0}
\DeclareMathOperator\cis{cis}
\DeclareMathOperator\Sp{Sp}
\DeclareMathOperator\tr{tr}
\DeclareMathOperator\im{Im}
\geometry{paper=a4paper, margin=54pt, includeheadfoot}

\title{פתרון ממ''ן 15 – אלגברה לינארית 1 (20109)}
\author{\AUTHOR}
\date{\today}

\begin{document}
\begin{hebrew}
	\maketitle
	\section{שאלה 1}
	\subsection{סעיף א'}
	נבדוק אם ההעתקה $T_1 : \mathbb{R}^2 \to \mathbb{R}^2$
	המוגדרת על־ידי $T_1(x, y) = (\sin y, x)$ היא העתקה לינארית. \\*
	נעשה זאת על־ידי בדיקה ישירה של תנאי הגדרה של העתקה לינארית,
	לפיהם עבור כל העתקה $T$ מעל מרחב $U$ ושדה $F$ מקיימת:
	\[
		\begin{aligned}
			& u_1, u_2 \in U, \lambda \in F \\
			& T(\lambda u_1) = \lambda T(u_1) \\
			& T(u_1) + T(u_2) = T(u_1 + u_2)
		\end{aligned}
	\]
	ניתן לראות כי עבור $T_1$ לא מתקיים $\lambda T(u) = T(\lambda u)$
	כאשר $u = (0, \frac{\pi}{2}), \lambda = 2$,
	שכן מתקיים:
	\[
		\begin{aligned}
			& 2 T_1(0, \frac{\pi}{2}) = 2(1, 0) = (2, 0) \\
			& T_1(2(0, \frac{\pi}{2})) = T_1(0, \pi) = (0, 0) \\
			& (2, 0) \ne (0, 0)
		\end{aligned}
	\]
	לכן ההעתקה $T_1$ היא לא העתקה לינארית.

	\subsection{סעיף ב'}
	נבדוק אם ההעתקה $T_2 : \mathbb{R}[x] \to \mathbb{R}[x]$
	המוגדרת על־ידי $T_2(p(x)) = (x + 1)p'(x) - p(x)$
	היא העתקה לינארית. \\*
	תחילה נבדוק אם עבור כל $\lambda \in \mathbb{R}$
	ו־$p(x) \in \mathbb{R}[x]$ מתקיים $\lambda T_2(p(x)) = T_2(\lambda p(x))$.
	נסתמך על התכונות של פולינומים:
	\[
		\lambda T_2(p(x))
		= \lambda ((x + 1) p'(x) - p(x))
		= (x + 1) \lambda p'(x) - \lambda p(x)
		= (x + 1) p'(\lambda x) - p(\lambda x)
		= T_2(\lambda p(x))
	\]
	אנו רואים כי תנאי זה אכן מתקיים.
	נבדוק אם עבור כל $p_1(x), p_2(x) \in \mathbb{R}[x]$
	מתקיים $T_2(p_1(x)) + T_2(p_2(x)) = T_2(p_1(x) + p_2(x))$:
	\[
		\begin{aligned}
			T_2(p_1(x)) + T_2(p_2(x))
			& = (x + 1) p_1'(x) - p_1(x) + (x + 1) p_2'(x) - p_2(x) \\
			& = (x + 1) (p_1'(x) + p_2'(x)) - (p_1(x) + p_2(x)) \\
			& = (x + 1) (p_1 + p_2)'(x) - (p_1(x) + p_2(x)) \\
			& = T_2(p_1(x) + p_2(x)) \\
		\end{aligned}
	\]
	ראינו כי גם תנאי זה מתקיים, לכן ההעתקה $T_2$ היא אכן העתקה לינארית.

	\section{שאלה 2}
	\subsection{סעיף א'}
	נראה כי לא קיימת העתקה לינארית $T : \mathbb{R}^3 \to \mathbb{R}^3$
	השונה מאפס המקיימת:
	\[
		\lambda = T(1, 0, 1) = T(1, 2, 1) = T(0, 1, 1) = T(2, 3, 3)
	\]
	ידוע כי $T$ היא העתקה לינארית, לכן מתקיים:
	\[
		\lambda
		= T(2, 3, 3)
		= T((1, 0, 1) + (0, 1, 1) + (0, 1, 1))
		= T(1, 0, 1) + T(0, 1, 1) + T(0, 1, 1)
		= 3 \lambda
	\]
	מכך נובע כי $\lambda = 0$.
	נבדוק אם הווקטורים $(1, 0, 1), (1, 2, 1), (0, 1, 1)$
	פורשים את $\mathbb{R}^3$ לפי שאלה 7.5.12 בעזרת דירוג מטריצת השורות שלהם:
	\[
		\begin{bmatrix}
			1 & 0 & 1 \\
			1 & 2 & 1 \\
			0 & 1 & 1
		\end{bmatrix}
		\xrightarrow{R_2 = R_2 - R_1}
		\begin{bmatrix}
			1 & 0 & 1 \\
			0 & 2 & 0 \\
			0 & 1 & 1
		\end{bmatrix}
		\xrightarrow{R_2 = R_2 / 2}
		\begin{bmatrix}
			1 & 0 & 1 \\
			0 & 1 & 0 \\
			0 & 1 & 1
		\end{bmatrix}
		\xrightarrow{R_3 = R_3 - R_2}
		\begin{bmatrix}
			1 & 0 & 1 \\
			0 & 1 & 0 \\
			0 & 0 & 1
		\end{bmatrix}
	\]
	הגענו למטריצת מדרגות ללא שורת אפסים, לכן הווקטורים אכן פורשים את
	המרחב $\mathbb{R}^3$, ישנם שלושה וקטורים ולכן לפי משפט 8.3.2(ד')
	קבוצה זו היא בסיס למרחב. כל וקטור במרחב הוא צירוף לינארי של שלושת 
	הווקטורים הנתונים, אך ידוע שכל צירוף לינארי שלהם בהצבה ב־$T$ שווה ל־$0$,
	לכן גם ההתעקה הלינארית $T$ היא העתקת האפס,
	לכן לא קיימת העתקה כזו המקיימת את תנאי השאלה.

	\subsection{סעיף ב'}
	יהיה $V$ מרחב לינארי ממד סופי ו־$U$ תת־מרחב שלו. \\*
	נוכיח שקיימת העתקה לינארית $T: V \to V$,
	כך שמתקיים $\ker T = U$ ו־$\ker T \cap \im T = \{0\}$. \\*
	ידוע כי ממדו של $V$ סופי, אז נגדיר $V = \Sp\{ v_1, \hdots v_n \}$
	כך ש־$n$ מספר טבעי והקבוצה הפורשת היא בסיס ל־$V$. ידוע כי $U \subseteq V$,
	לכן נגדיר $U = \Sp\{v_1,\hdots, v_k\}$ כך ש־$k \le n$
	והקבוצה הפורשת היא בסיס ל־$U$.
	נגדיר העתקה לינארית $T: V \to V$ כך שמתקיים:
	\[
		T(v_1) = T(v_2) = \hdots = T(v_k) = 0
	\]
	בנוסף נגדיר כי לכל $k < i \le n$ מתקיים:
	\[
		T(v_i) = v_i
	\]
	ההעתקה $T$ מוגדרת עבור כלל המרחב $V$ על־ידי הרחבה לינארית מהגדרה
	על בסיס שלה. על־פי הגדרת ההעתקה מתקיים:
	\[
		\ker T = \Sp\{v_1, \hdots, v_k\} = U
	\]
	על־פי ההגדרה התמונה של $T$, היא מכילה כל וקטור ב־$V$ שאיננו ב־$U$, דהינו:
	\[
		\im T = \Sp\{v_{k + 1}, \hdots v_n\}
	\]
	לכן מתקיים:
	\[
		\ker T \cap \im T
		= U \cap \Sp\{v_{k + 1}, \hdots, v_n\}
		= \Sp\{v_1,\hdots, v_k\} \cap \Sp\{v_{k + 1}, \hdots, v_n\}
		= \{0\}
	\]
	הוכחנו כי קיימת העתקה לינארית $T$ המקיימת את התנאים על־ידי בנייתה.

	\subsection{סעיף ג'}
	יהיה $V$ מרחב לינארי ו־$S: V \to V$ העתקה לינארית הפיכה. \\*
	נוכיח כי לא קיימת העתקה לינארית $T: V \to V$ שמקיימת
	$\ker TS = \{0\}$ ו־$\ker T \ne \{0\}$. \\*
	על־פי הגדרת איזומורפיזם והעתקה הפיכה, אנו למדים כי $S$ איזומורפיזם.
	על־פי למה 9.5.2 $\ker S = \{0\}$.
	על־פי אותה הלמה אנו רואים כי על $TS$ להיות איזומורפיזם בפרט והפיכה בכלל,
	וכי על $T$ עצמה להיות לא הפיכה.
	אם $T$ לא הפיכה, אז קיימים $v, u \in V, u \ne v$
	שעבורם מתקיים $T(u) = T(v)$.
	נגדיר כי $v', u' \in V, S(v') = v, S(u') = u$,
	נראה כי $u' \ne v'$, שכן $S$ הפיכה ולכן גם חד־חד ערכית.
	אנו רואים כי $TS(v') = TS(u')$, אבל $v' \ne u'$ ולכן על־פי
	ההפיכות של $TS$ מתקיים גם $TS(u') \ne TS(v')$.
	ההגדרה של $T$ מובילה לסתירה, ולכן לא קיימת העתקה המקיימת את התנאים.

	\section{שאלה 3}
	\subsection{סעיף א'}
	תהיה $T: V \to V$ העתקה לינארית כאשר $V$ מרחב לינארי מעל השדה $F$. \\*
	קיים מספר טבעי $k \ge 2$ כך ש־$T^{k - 1} \ne 0$ ו־$T^k = 0$.
	יהי $u \in V$ כך ש־$T^{k - 1}(u) \ne 0$. \\*
	נוכיח שהקבוצה $L = \{u, T(u), T^2(u), \hdots, T^{k-1}(u)\}$ 
	בלתי תלויה לינארית. \\*
	על־פי הגדרת התלות הלינארית, קיימת תלות רק אם למשוואה:
	\[
		\lambda_1 u + \lambda_2 T(u) + \cdots + \lambda_{k} T^{k - 1}(u) = 0
	\]
	יש רק פתרון טריוויאלי.
	נבצע את הפעולה $T$ על שני אגפי המשוואה:
	\begin{align*}
		& T(\lambda_1 u + \lambda_2 T(u) +
		\cdots + \lambda_{k} T^{k - 1}(u)) = T(0) \\
		& \lambda_1 T(u) + \lambda_2 T^2(u) +
		\cdots + \lambda_{k} T^k(u) = 0
		& T^k(v) = 0 \\
		& \lambda_1 T(u) + \lambda_2 T^2(u) +
		\cdots + \lambda_{k} 0 = 0 \\
	\end{align*}
	אם נבצע על אגפי המשוואה השמה ב־$T$ $k-1$ פעמים יתקבל כי:
	\[
		\lambda_1 T^{k - 1}(u) = 0
	\]
	ולכן $\lambda_1 = 0$. נוכל לבצע את התהליך הזה שוב ושוב ולהשתמש בהצבה,
	ונראה כי מתקיים $\lambda_1 = \lambda_2 = \cdots = \lambda_{k - 1} = 0$.
	דהינו, יש רק פתרון אחד למשוואה והוא הפתרון הטריוויאלי,
	לכן הקבוצה בלתי תלויה לינארית.

	\subsection{סעיף ב'}
	נוכיח כי לא קיימת מטריצה $A \in M_{2 \times 2}(F)$
	המקיימת $A^2 \ne 0$ ו־$A^3 = 0$. \\*
	נגדיר העתקה לינארית $T: F^2 \to F^2$ כך ש־$A$ המטריצה המייצגת שלה.
	אז מתקיים $T^2 \ne 0$ ו־$T^3 = 0$. על־פי הסעיף הקודם קיים $u \in F^2$
	המקיים $T^2(u) \ne 0$ עבורו $\{u, T(u), T^2(u)\}$ קבוצה בלתי תלויה לינארית.
	לכן גם הקבוצה השקולה לה $\{u, Au, A^2 u\}$ בלתי תלויה לינארית.
	ידוע כי $\dim F^2 = 2$, לכן מספר האיברים בבסיס של $F^2$ הוא $2$.
	לפי משפט 8.3.2 קבוצה גדולה מ־$2$ מ־$F^2$ היא תלויה לינארית,
	אבל הקבוצה בעלת שלושת האיברים $\{u, Au, A^2u\}$ בלתי תלויה לינארית
	וזוהי סתירה, לכן אין מטריצה $A$ שיכולה לקיים את התנאים הנתונים.

	\section{שאלה 4}
	\subsection{סעיף א'}
	תהיה העתקה לינארית $T: \mathbb{R}^2 \to \mathbb{R}^2$
	השונה מהעתקת האפס ואשר מקיימת $T^2 = 2T$, וידוע כי $T$ לא הפיכה. \\*
	נמצא את ממד הגרעין והתמונה של $T$. \\*
	ידוע כי $T$ איננה הפיכה, לכן לפי למה 9.5.2 $\ker T \ne \{0\}$.
	לפיכך אנו יודעים כי $T$ נפרש על־ידי לפחות איבר אחד, לכן $0 < \dim \ker T$.
	ההעתקה $T$ שונה מהעתקת האפס, לכן בתמונתה קיימים איברים שונים מאפס,
	וכך גם ממדה לא יכול להיות 0, $0 < \dim \im T$.
	ידוע כי ממד המרחב הוא $2$, אז לפי משפט 9.6.1:
	\[
		\underset{0 < }{\dim \ker T} + \underset{0 <}{\dim \im T} = 2
	\]
	הערכים האפשריים של ממדים אלה הם 0 עד 2,
	וניתן לראות כי השוויון מתקיים רק כאשר:
	\[
		\dim \ker T = \dim \im T = 1
	\]

	\subsection{סעיף ב'}
	נוכיח כי קיים בסיס $B$ של $\mathbb{R}^2$
	כך שמטריצת הייצוג של $T$ לפי $B$ היא:
	\[
		{[T]}_B =
		\begin{bmatrix}
			0 & 0 \\
			0 & 2
		\end{bmatrix}
	\]
	ננסה לבנות בסיס מתאים על־פי הגרעין והתמונה של $T$.
	נגדיר $u_1 \in \ker T, u_2 \in \im T$ ובהתאם $B = (u_1, u_2)$.
	הווקטור $u_1$ מוכל בגרעין ההעתקה, לכן מתקיים $T(u_1) = 0$,
	לכן גם:
	\[
		{[T(u_1)]}_B = \begin{bmatrix} 0 \\ 0 \end{bmatrix}
	\]
	נגדיר $u_2 \ne 0$, הווקטור מוכל בתמונת ההעתקה,
	לכן בהתאם קיים וקטור $u_2' \in \mathbb{R}^2, u_2' \ne 0$
	אשר מקיים $T(u_2') = u_2$. ידוע כי $T^2(u_2') = 2T(u_2')$,
	לכן מתקיים גם $T(u_2) = 2u_2 = 0u_1 + 2u_2$, לכן גם:
	\[
		{[T(u_2)]}_B = \begin{bmatrix} 0 \\ 2 \end{bmatrix}
	\]
	נבנה מטריצה מייצגת על־פי הגדרה 10.1.1:
	\[
		{[T]}_B =
		\begin{bmatrix}
			0 & 0 \\
			0 & 2
		\end{bmatrix}
	\]
	הוכחנו כי קיים בסיס $B$ כזה על־ידי בנייתו.

	\section{שאלה 5}
	\subsection{סעיף א'}
	תהיה $T: \mathbb{R}^3 \to \mathbb{R}^3$ העתקה לינארית
	ויהיה $B = ((1, 1, 1), (1, 0, 1), (1, 1, 0))$ בסיס ל־$\mathbb{R}^3$.
	ידוע כי:
	\[
		{[T]}_B =
		\begin{bmatrix}
			a & 0 & b \\
			3 & 2a & 1 \\
			2c & b & a \\
		\end{bmatrix},
		(1, 0, 0) \in \ker T
	\]
	נמצא את ערכי $a, b, c$: \\*
	נגדיר $B = (b_1, b_2, b_3)$.
	תחילה נראה כי $(1, 0, 0) = -b_1 + b_2 + b_3$,
	לכן ${[ (1, 0, 0) ]}_B = {[-1, 1, 1]}^t$.
	נראה כי $T(1, 0, 0) = 0$, שכן נתון כי הווקטור בגרעין ההעתקה.
	לפי משפט 10.2.1 מתקיים השוויון הבא:
	\[
		\begin{aligned}
			& {[T(1, 0, 0)]}_B = {[T]}_B {[(1, 0, 0)]}_B \\
			& \begin{bmatrix} 0 \\ 0 \\ 0 \end{bmatrix} =
				\begin{bmatrix}
					a & 0 & b \\
					3 & 2a & 1 \\
					2c & b & a \\
				\end{bmatrix}
				\begin{bmatrix} -1 \\ 1 \\ 1 \end{bmatrix} \\
			& \begin{bmatrix} 0 \\ 0 \\ 0 \end{bmatrix} =
			\begin{bmatrix}
					-a + b \\
					2a - 2 \\
					a + b - 2c
				\end{bmatrix}
		\end{aligned}
	\]
	את שיוויון המטריצות הזה נוכל לפרק למערכת המשוואות:
	\[
		\begin{cases}
			-a + b = 0 \\
			2a - 2 = 0 \\
			a + b - 2c = 0
		\end{cases}
	\]
	פתרונה מניב כי $a = b = c = 1$.

	\subsection{סעיף ב'}
	נמצא בסיס לתמונת וגרעין ההעתקה $T$. \\*
	בסעיף הקודם גילינו כי:
	\[
		{[T]}_B =
		\begin{bmatrix}
			1 & 0 & 1 \\
			3 & 2 & 1 \\
			2 & 1 & 1
		\end{bmatrix}
	\]
	נמצא את כלל הערכים עבורם ההעתקה מקבלת $0$, דהינו את גרעינה,
	על־ידי פתרון מערכת המשוואות $T(u) = 0$ בעזרת משפט 10.2.1:
	\[
		\begin{bmatrix}
			1 & 0 & 1 \\
			3 & 2 & 1 \\
			2 & 1 & 1
		\end{bmatrix}
		\begin{bmatrix} x \\ y \\ z \end{bmatrix}
		=
		\begin{bmatrix} 0 \\ 0 \\ 0 \end{bmatrix}
	\]
	זוהי כמובן מערכת משוואות הומוגנית, נדרגה כדי למצוא את קבוצת הפתרונות:
	\[
		\begin{bmatrix}
			1 & 0 & 1 \\
			3 & 2 & 1 \\
			2 & 1 & 1
		\end{bmatrix}
		\xrightarrow[R_3 \rightarrow R_3 - 2R_1]{R_2 \rightarrow R_2 - 3R_1}
		\begin{bmatrix}
			1 & 0 & 1 \\
			0 & 2 & -2 \\
			0 & 1 & -1
		\end{bmatrix}
		\xrightarrow{R_2 \rightarrow R_2 / 2}
		\begin{bmatrix}
			1 & 0 & 1 \\
			0 & 1 & -1 \\
			0 & 1 & -1
		\end{bmatrix}
		\xrightarrow{R_3 \rightarrow R_3 - R_2}
		\begin{bmatrix}
			1 & 0 & 1 \\
			0 & 1 & -1 \\
			0 & 0 & 0
		\end{bmatrix}
	\]
	אנו רואים כי ישנה שורת אפסים אחת,
	לכן יש משתנה חופשי אחד ובהתאם ממד מרחב הפתרונות הוא 1,
	ידוע כבר כי $(1, 0, 0) \in \ker T$,
	לכן וקטור זה מהווה בסיס לגרעין ההעתקה. \\*
	נמצא את תמונת $T$. על־פי הגדרת מטריצת מעבר, וקטורי העמודה המופיעים בה הם קורדינטות וקטורים בתמונת $T$.
	נבחר את:
	\[
		\begin{bmatrix}
			1 \\ 3 \\ 2
		\end{bmatrix},
		\begin{bmatrix}
			1 \\ 1 \\ 1
		\end{bmatrix}
	\]
	שכן אינם בלתי תלויים ונחשב את ערכם לפי בסיס $B$:
	\begin{align*}
		1 \cdot (1, 1, 1) + 3 \cdot (1, 0, 1) + 2 \cdot (1, 1, 0) = (6, 3, 4) \\
		1 \cdot (1, 1, 1) + 1 \cdot (1, 0, 1) + 1 \cdot (1, 1, 0) = (3, 2, 2)
	\end{align*}
	לכן
	\[
		\ker T = \Sp\{(1, 0, 0)\},
		\im T = \Sp\{(6, 3, 4), (3, 2, 2)\}
	\]

	\subsection{סעיף ג'}
	נמצא ביטוי עבור $T(x, y, z)$ לכל $x, y, z \in \mathbb{R}^3$. \\*
	אנו יודעים כי $(1, 0, 0) \in \ker T$, לכן $T(1, 0, 0) = (0, 0, 0)$. \\*
	נחשב את ערכי הווקטורים ששומשו לייצוג בסיס התמונה בסעיף הקודם בעזרת 
	הגדרת הקורדינטה:
	\[
		{[T(1, 1, 1)]}_B
		= \begin{bmatrix} 1 \\ 3 \\ 2 \end{bmatrix}
		\rightarrow T(1, 1, 1)
		= 1(1, 1, 1) + 3(1, 0, 1) + 2(1, 1, 0)
		= (6, 3, 4)
	\]
	\[
		{[T(1, 1, 0)]}_B
		= \begin{bmatrix} 1 \\ 1 \\ 1 \end{bmatrix}
		\rightarrow T(1, 1, 0)
		= 1(1, 1, 1) + 1(1, 0, 1) + 1(1, 1, 0)
		= (3, 2, 2)
	\]
	בשל היות $T$ העתקה לינארית, נראה כי:
	\[
		T(1, 1, 0) - T(1, 0, 0) = T(0, 1, 0) = (3, 2, 2)
	\]
	\[
		T(1, 1, 1) - T(1, 0, 0) - T(0, 1, 0) = T(0, 0, 1) = (3, 1, 2)
	\]
	לכן:
	\[
		\begin{aligned}
			T(x, y, z)
			& = T(x, 0, 0) + T(0, y, 0)  + T(0, 0, z) \\
			& = (0, 0, 0) + (3y, 2y, 2y) + (3z, 1z, 2z) \\
			& = (3y + 3z, 2y + z, 2y + 2z) \\
		\end{aligned}
	\]
	\end{hebrew}
\end{document}
