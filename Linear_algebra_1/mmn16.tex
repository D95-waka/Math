\documentclass[a4paper,10pt]{article}

% packages
\usepackage{inputenc, fontspec, amsmath, amsfonts, polyglossia, catchfile}
\usepackage[a4paper, margin=50pt, includeheadfoot]{geometry} % set page margins

% style
\AddToHook{cmd/section/before}{\clearpage}	% Add line break before section
\setdefaultlanguage{hebrew}
\setotherlanguage{english}
\setmainfont{Libertinus Serif}
\linespread{1.5}
\setcounter{secnumdepth}{0}		% Remove default number tags from sections

% custom operators
\newcommand{\getenv}[2][]{%
  \CatchFileEdef{\temp}{"|kpsewhich --var-value #2"}{\endlinechar=-1}%
  \if\relax\detokenize{#1}\relax\temp\else\let#1\temp\fi}
\getenv[\AUTHOR]{AUTHOR}
\DeclareMathOperator\cis{cis}
\DeclareMathOperator\Sp{Sp}
\DeclareMathOperator\tr{tr}
\DeclareMathOperator\im{Im}
\DeclareMathOperator\diag{diag}
\def\RR{\mathbb{R}}
\def\CC{\mathbb{C}}

\title{פתרון ממ''ן 16 – אלגברה לינארית 1 (20109)}
\author{\AUTHOR}
\date\today

\begin{document}
\maketitle
\section{שאלה 1}
\subsection{סעיף א'}
תהיה המטריצה:
\[
	A =
	\begin{bmatrix}
		0 & a & 1 \\
		a & 0 & -1 \\
		0 & 0 & a
	\end{bmatrix}
\]
כך ש־$a$ מספר ממשי. נמצא עבור אילו ערכי $a$ המטריצה לכסינה. \\*
תחילה נחשב את הפולינום האופייני של המטריצה ונמצא בעזרתו
את הערכים העצמיים של $A$:
\begin{align*}
	& \det(\lambda I - A) = 0 \\
	& \begin{vmatrix}
		\lambda & -a & -1 \\
		-a & \lambda & 1 \\
		0 & 0 & \lambda - a
	\end{vmatrix} = 0 \\
	& (\lambda - a) \begin{vmatrix}
		\lambda & -a \\
		-a & \lambda \\
	\end{vmatrix} = 0 \\
	& (\lambda - a) (\lambda^2 - {(-a)}^2) = 0 \\
	& {(\lambda - a)}^2 (\lambda + a) = 0 \\
	& \lambda = -a, a
\end{align*}
למטריצה שני ערכים עצמיים, $-a, a$.
מצאנו אפוא גם כי הריבוי האלגברי של $-a$ הוא $1$
וריבויו האלגברי של $a$ הוא 2. \\*
נחשב את הריבוי הגיאומטרי של הערכים העצמיים.
נמצא את ממד מרחב הפתרונות של $A_{-a}$ על־ידי מציאת מרחב הפתרונות של מערכת המשוואות:
\begin{align*}
	& (-aI - A) \begin{bmatrix} x \\ y \\ z \end{bmatrix} = \underline{0} \\
	& (aI + A) \begin{bmatrix} x \\ y \\ z \end{bmatrix} = \underline{0} \\
	& \begin{bmatrix}
		a & a & 1 \\
		a & a & -1 \\
		0 & 0 & 2a
	\end{bmatrix}
	\begin{bmatrix} x \\ y \\ z \end{bmatrix} = \underline{0} \\
\end{align*}
זוהי כמובן מערכת משוואות הומוגנית, לכן נוכל לדרגה ולמצוא את פתרונותיה:
\[
	\begin{bmatrix}
		a & a & 1 \\
		a & a & -1 \\
		0 & 0 & 2a
	\end{bmatrix}
	\xrightarrow[R_3 \rightarrow R_3 / 2a]{R_2 \rightarrow R_2 - R_1}
	\begin{bmatrix}
		a & a & 1 \\
		0 & 0 & -2 \\
		0 & 0 & 1
	\end{bmatrix}
	\xrightarrow[R_2 \rightarrow R_2 / -2]{R_1 \rightarrow R_1 - R_3}
	\begin{bmatrix}
		a & a & 0 \\
		0 & 0 & 1 \\
		0 & 0 & 1
	\end{bmatrix}
	\xrightarrow{R_3 \rightarrow R_2 - R_2}
	\begin{bmatrix}
		a & a & 0 \\
		0 & 0 & 1 \\
		0 & 0 & 0
	\end{bmatrix}
	\tag{2}
\]
ניתן לראות כי דרגת המטריצה היא 2,
לכן לפי משפט 8.6.1 ממד מרחב פתרונות המטריצה הוא 1,
ובהתאם זהו ערכו של הריבוי הגיאומטרי עבור $-a$.
נפתור את מערכת המשוואות ההמומוגנית המייצגת את המרחב העצמי של $a$:
\[
	\begin{bmatrix}
		a & -a & -1 \\
		-a & a & 1 \\
		0 & 0 & 0
	\end{bmatrix}
	\xrightarrow{R_2 \rightarrow R_2 + R_1}
	\begin{bmatrix}
		a & -a & -1 \\
		0 & 0 & 0 \\
		0 & 0 & 0
	\end{bmatrix} \tag{1}
\]
דרגת מטריצת המערכת היא 1, ולכן בהתאם ממד המרחב העצמי הוא 2.
אנו רואים כי הריבוי האלגברי והריבוי הגיאומטרי שווים עבור לכלל הערכים
הפנימיים של המטריצה, לכן לפי משפט 11.5.4 המטריצה לכסינה.
מסיבה זו המטריצה $A$ לכסינה עבור כל ערך של $a$ מלבד 0.
אילו היה מתקיים $a = 0$ אז הערך העצמי היחיד היה $0$, ריבויו האלגברי היה $3$ וחישוב ריבויו הגיאומטרי כבדרך החישוב קודם:
\[
	\begin{bmatrix}
		0 & 0 & 1 \\
		0 & 0 & -1 \\
		0 & 0 & 0
	\end{bmatrix}
	\xrightarrow{R_2 \to R_2 + R_1}
	\begin{bmatrix}
		0 & 0 & 1 \\
		0 & 0 & 0 \\
		0 & 0 & 0
	\end{bmatrix}
\]
דרגת המטריצה היא $1$ ובהתאם הריבוי הגיאומטרי הוא 2, קטן מהריבוי האלגברי, לכן $A$ לכסינה כאשר $a \in \RR\backslash\{0\}$.

\subsection{סעיף ב'}
נקבע $a = -1$, לכן:
\[
	A=
	\begin{bmatrix}
		0 & -1 & 1 \\
		-1 & 0 & -1 \\
		0 & 0 & -1
	\end{bmatrix}
\]
נמצא מטריצה $D$ ומטריצה הפיכה $P$ כך ש־$D = P^{-1} A P$. \\*
על־פי משפט 11.2.3 נגדיר:
\[
	A \cong D =
	\begin{bmatrix}
		-1 & 0 & 0 \\
		0 & -1 & 0 \\
		0 & 0 & 1
	\end{bmatrix}
\]
נמצא מטריצה הפיכה $P$ המקיימת $D = P^{-1} A P$.
מטריצה זו היא מטריצת הייצוג לפי בסיס וקטורים עצמיים של המטריצה,
נמצא שלושה וקטורים כאלה. למעשה,
כבר בנינו את מערכת המשוואות המתאימה למציאת וקטורים כאלה,
נחזור למערכת $(1)$, בהצבת הערך של $a$ ונציב לאחור:
\[
	x - y + z = 0 \rightarrow y = x + z
\]
מצאנו כי $A_{-1} = \{(t, t + s, s) \mid t, s \in \mathbb{R} \}$.
נמצא שני וקטורים בלתי תלויים מהמרחב, נבחר את $(1, 1, 0), (0, 1, 1)$. \\*
נבצע הליך דומה על מערכת המשוואות $(2)$ ונראה כי עבורה מתקיים:
\[
	z = 0, -x -y = 0 \rightarrow y = -x
\]
ולכן מתקיים $A_1 = \{(t, -t, 0) \mid t \in \mathbb{R} \}$.
נבחר מהמרחב את הווקטור $(1, -1, 0)$.
על־פי משפט 11.2.3:
\[
	P =
	\begin{bmatrix}
		1 & 0 & 1 \\
		1 & 1 & -1 \\
		0 & 1 & 0
	\end{bmatrix}
\]
נחשב את $P^{-1}$ בעזרת שיטת הדירוג למטריצת היחידה של $P$:
\[
	\begin{bmatrix}
		1 & 0 & 1 & \vline & 1 & 0 & 0 \\
		1 & 1 & -1 & \vline & 0 & 1 & 0 \\
		0 & 1 & 0 & \vline & 0 & 0 & 1
	\end{bmatrix}
	\xrightarrow{R_2 \leftrightarrow R_3}
	\begin{bmatrix}
		1 & 0 & 1 & \vline & 1 & 0 & 0 \\
		0 & 1 & 0 & \vline & 0 & 0 & 1 \\
		1 & 1 & -1 & \vline & 0 & 1 & 0
	\end{bmatrix}
	\xrightarrow{R_3 \rightarrow R_3 - R_1 - R_2}
	\begin{bmatrix}
		1 & 0 & 1 & \vline & 1 & 0 & 0 \\
		0 & 1 & 0 & \vline & 0 & 0 & 1 \\
		0 & 0 & -2 & \vline & -1 & 1 & -1
	\end{bmatrix}
\]
\[
	\xrightarrow{R_3 \rightarrow - \frac{1}{2} R_3}
	\begin{bmatrix}
		1 & 0 & 1 & \vline & 1 & 0 & 0 \\
		0 & 1 & 0 & \vline & 0 & 0 & 1 \\
		0 & 0 & 1 & \vline & {\frac{1}{2}} & -{\frac{1}{2}} & {\frac{1}{2}}
	\end{bmatrix}
	\xrightarrow{R_1 \rightarrow R_1 - R_3}
	\begin{bmatrix}
		1 & 0 & 0 & \vline & {\frac{1}{2}} & {\frac{1}{2}} & -{\frac{1}{2}} \\
		0 & 1 & 0 & \vline & 0 & 0 & 1 \\
		0 & 0 & 1 & \vline & {\frac{1}{2}} & -{\frac{1}{2}} & {\frac{1}{2}}
	\end{bmatrix}
\]
מצאנו על־ידי דירוג כי:
\[
	P^{-1} =
	\begin{bmatrix}
		{\frac{1}{2}} & {\frac{1}{2}} & -{\frac{1}{2}} \\
		0 & 0 & 1 \\
		{\frac{1}{2}} & -{\frac{1}{2}} & {\frac{1}{2}}
	\end{bmatrix}
\]
נחשב את $A^{2023}$. ידוע כי:
\[
	D = P^{-1} A P
\]
נכפול משמאל ב־$P$ ומימין ב־$P^{-1}$:
\[
	A = P D P^{-1}
\]
לכן בהתאם נעלה בחזקה ונקבץ בהתאם לקיבוציות מטריצות בכפל:
\[
	A^{2023} = {(P D P^{-1})}^{2023} = P D^{2023} P^{-1}
\]
ידוע כי $D$ מטריצה אלכסונית ולכן לפי טענה 3.6.8:
\[
	A^{2023} = P
	\begin{bmatrix}
		{-1}^{2023} & 0 & 0 \\
		0 & {-1}^{2023} & 0 \\
		0 & 0 & 1^{2023}
	\end{bmatrix}
	P^{-1}
	= P D P^{-1}
	= A
\]
אנו רואים כי $A^{2023} = A$.

\section{שאלה 2}
\subsection{סעיף א'}
נוכיח כי לא קיימת מטריצה מדרגה 3 עם פולינום אופייני
$p(x) = x^7 - x^5 + x^3$. \\*
נניח בשלילה כי ישנה מטריצה $A$ אשר $p(x)$ הפולינום האופייני שלה.
על־פי שאלה 11.4.5 סדר המטריצה הוא גם מעלת הפולינום האופייני,
לכן $A$ מטריצה מסדר 7. לפולינום האופייני הגורם הלינארי 0,
ריבויו האלגברי הוא 3.
נחשב את הריבוי הגיאומטרי על ידי מציאת המרחב $A_0$.
הריבוי הגיאומטרי שווה לממד מרחב הפתרונות של מערכת המשוואות
$(0I_7 - A) \underline{x} = 0$ כאשר $x$ מייצג וקטור עמודה של משתנים.
מערכת משוואות זו היא מערכת הומוגנית כך ש־$A$
היא מטריצת המקדמים המצומצמת שלה.
לפי משפט 8.6.1 ממד מרחב הפתרונות הוא מספר המשתנים, 7,
פחות דרגת המטריצה, שידוע ששווה ל־3. לכן הריבוי הגיאומטרי של 0 הוא 4.
אנו רואים כי הריבוי הגיאומטרי גדול מהריבוי האלגברי,
אבל לפי משפט 11.5.3 מצב זה לא יתכן, לכן פולינום אופייני זה לא יכול לייצג אף מטריצה.

\subsection{סעיף ב'}
תהיה $T: \mathbb{R}^2 \to \mathbb{R}^2$ העתקה לינארית כך שהפולינום
האופייני שלה הוא $p(x) = x^2 + 2x - 3$.
נוכיח שההעתקה הלינארית $3T + I$ היא איזומורפיזם. \\*
נבחין כי מתקיים:
\[
	p(x) = x^2 + 2x - 3 = (x + 3)(x - 1)
\]
לכן שורשיה האופייניים של ההעתקה הם $-3, 1$.
על־פי משפט 11.4.1 אלו הם גם הערכים העצמיים של ההעתקה.
נראה כי להעתקה מספר ערכים עצמיים השווה לממדה,
לכן לפי משפט 11.2.5 ההעתקה לכסינה.
לפי משפט 11.2.3 קיים בסיס $B$ ל־$\mathbb{R}^2$ כך שמתקיים:
\[
	{[T]}_B =
	\begin{bmatrix}
		-3 & 0 \\
		0 & 1
	\end{bmatrix}
\]
לפי משפט 8.4.2 מתקיים:
\[
	{[3T + I]}_B =
3{[T]}_B + {[I]}_B =
	\begin{bmatrix}
		-8 & 0 \\
		0 & 4
	\end{bmatrix}
\]
לפי משפט 10.3.2 המעבר מההעתקה למטריצה המייצגת על־פי בסיס משמרת את תכונת
האיזומורפיזם, לכן עלינו רק להוכיח שההעתקה ${[3T + I]}_B$ היא איזומורפיזם.
נמצא נוסחה כללית להעתקה לפי $x, y$, כאשר הם מייצגים ערכי קורדינטה $B$.
\[
	{[3T + I]}_B \begin{bmatrix} x \\ y \end{bmatrix} =
	-8x + 4y
\]
ניתן לראות כי ניתן להגיע לכל ערך ב־$\mathbb{R}^2$ בשל היות
$x, y$ בסיס למרחב. בשל כך ההעתקה היא על.
ידוע כי זהו צירוף לינארי מעל בסיס, לכן ישנה חד־חד ערכיות ל־$\mathbb{R}^2$.
לכן ההעתקה היא איזומורפיזם, ובשל כך גם $3T + I$ איזומורפיזם. \\
נמצא את הפולינום האופייני של $T^3$: \\*
נראה בעזרת משפט 10.4.1 כי
\[
	{[T^3]}_B = {[T]}_B^3 =
	\begin{bmatrix}
		-27 & 0 \\
		0 & 1
	\end{bmatrix}
\]
הפולינום האופייני של המטריצה אם כך הוא:
\[
	\det(xI - {[T]}_B^3) =
	\begin{vmatrix}
		x + 27 & 0 \\
		0 & x - 1
	\end{vmatrix}
	= (x + 27)(x - 1)
\]

\subsection{סעיף ג'}
תהיה מטריצה $A$ מסדר $4 \times 4$ בלתי הפיכה.
ידוע כי מתקיים $\rho(A + 2I) = 2, \det(A - 2I) = \underline{0}$.
נמצא את הפולינום האופייני של $A$. \\*
המטריצה $A$ לא הפיכה, לכן ערך הדטרמיננטה שלה מתאפס, 
לכן לפי שאלה 11.4.7 הפולינום לאופייני של $A$ מקדם חופשי $0$ והפולינום הוא כפולה של $x$. \\*
נראה כי:
\[
	\det(A - 2I) = 0 \xrightarrow{4.3.2} \det(2I - A) = 0
\]
לכן לפי משפט 11.4.2 2 הוא שורש אופייני של $A$
ולכן $A$ מכפלה של $(x - 2)$. \\*
דרגה מוגדרת כממד מרחב פתרונות, לכן מכפלת מטריצה בסקלר לא תשנה את דרגתה.
נוכל לטעון כי $\rho(A + 2I) = \rho(-2I - A)$.
לפי משפט 8.6.1 מתקיים $4 - \rho(A + 2I) = 2$ הוא ממד מרחב השורות או העמודות של מערכת
המשוואות $(-2I - A)x = 0$, נשים לב כי זוהי המערכת של המרחב העצמי $A_{-2}$,
דהינו הפולינום האופייני של $A$ הוא מכפלה של ${(x + 2)}^n$.
כאשר $2 \le n$.
לפי שאלה 11.4.5 המקדם של $x^4$ בפולינום האופייני הוא 1,
מצאנו עד כה כי ערך הפולינום הוא:
\[
	x(x-2){(x+2)}^n
\]
ניתן לראות כי תנאי זה מתקיים רק כאשר $n = 2$,
לכן הפולינום האופייני של $A$ הוא $x(x - 2){(x + 2)}^2$. \\*
המטריצה $A$ היא לכסינה, שכן על־פי משפט 11.5.3 נוכל להניח כי
הריבוי הגיאומטרי והאלגברי של השורשים האופיניים $0, 2$ זהים,
וידוע כי התנאי הזה מתקיים עבור השורש $-2$,
לכן לפי משפט 11.5.4 המטריצה לכסינה.

\section{שאלה 3}
תהי $A$ מטריצה לכסינה מסדר $n \times n$.
נסמן $p(t) = \sum_{i = 0}^{n} a_i t^i$ הפולינום האופייני של $A$.
נגדיר $p(A) = \sum_{i = 0}^{n} a_i A^i$.
נוכיח כי $p(A) = 0$. \\*
בשל היותה מטריצה לכסינה, $A$ דומה למטריצה אלכסונית כלשהי.
נגדיר $A \cong \diag(b_0, \hdots b_n) = B$.
הפולינום האופייני של מטריצות דומות שווה לפי משפט 11.4.3,
לכן $p(t) = \prod_{i = 0}^{n} (t - b_i)$.
בהתאם מתקיים גם $p(A) = \prod_{i = 0}^n (b_i I - A)$ בהתאם להגדרת $p(A)$. \\*
נגדיר $P$ מטריצה הפיכה כך ש־$P^{-1} A P = B$.
נראה כי עבור כל $i$ כך ש־$0 \le i \le n$ מתקיים:
\[
	b_i I - A
	= b_i I - P B P^{-1}
	= P b_i I P^{-1} - P B P^{-1}
	= P (b_i I - B) P^{-1}
\]
במטריצה $b_i I - B$ העמודה ה־$i$ היא עמודת אפסים לפי הגדרתה.
לפי מסקנה 3.4.4(ב') גם במטריצה $P (b_i I - B) P^{-1}$
העמודה ה־$i$ היא עמודת אפסים.
לכן במטריצה $b_i I - A$ יש עמודת אפסים במקום ה־$i$. \\*
נבחן את הביטוי $p(t) = \prod_{i = 0}^{n} (t - b_i)$.
בביטוי זה יש מכפלה של $n$ מטריצות כך שלכל אחת עמודת אפסים במקום ה־$i$
בסדר המכפלה. בסך־הכול בשרשרת המכפלה יש עמודת אפסים לפחות פעם אחת עבור
כל העמודות במטריצה המתקבלת. לפי משפט 3.4.4 תוצאת המכפלה הסופית היא
מטריצה שבה כל עמודה היא עמודת אפס, לכן המטריצה היא מטריצת האפס,
אז $p(A) = 0$.

\section{שאלה 4}
\subsection{סעיף א'}
נראה כי לא בהכרח מתקיים שאם לשתי מטריצות ישנו אותו פולינום אופייני אז יש להן אותה דרגה על־ידי דוגמה נגדית. \\*
נגדיר
\[
	A = \begin{bmatrix}
		0 & 0 & 0 \\
		0 & 0 & 0 \\
		0 & 0 & 0
	\end{bmatrix},
	B = \begin{bmatrix}
		0 & 0 & 1 \\
		0 & 0 & 0 \\
		0 & 0 & 0
	\end{bmatrix}
\]
קל לראות כי הפולינום האופייני $p_A(x)$ של המטריצה $A$ הוא
\[
	p_A(x) = x^3
\]
נחשב את הפולינום האופייני של $B$
\[
	p_B(x) = |Ix - B|
	= \begin{vmatrix}
		x & 0 & -1 \\
		0 & x & 0 \\
		0 & 0 & x
	\end{vmatrix}
	\overset{\text{פיתוח לפי $C_1$}}{=}
	{(-1)}^{1+1} x \begin{vmatrix}
		x & 0 \\
		0 & x
	\end{vmatrix}
	= x^3
\]
אנו רואים כי $\rho(A) = 0 \ne 1 = \rho(B)$ אבל $p_A(x) = p_B(x)$ בניגוד לטענה ולכן היא איננה נכונה.

\subsection{סעיף ב'}
נוכיח כי המטריצות הנתונות דומות:
\[
	A =
	\begin{bmatrix}
		2 & -\sqrt{3} \\
		\sqrt{3} & -2
	\end{bmatrix},
	B =
	\begin{bmatrix}
		-1 & 0 \\
		0 & 1
	\end{bmatrix}
\]
נחשב את הפולינום האופייני של $A$ בעזרת המערכת $\det(tI - A) = 0$:
\[
	\begin{vmatrix}
		t - 2 & \sqrt{3} \\
		- \sqrt{3} & t + 2
	\end{vmatrix} = 0
	\rightarrow
	(t - 2)(t + 2) + 3 = 0
	\rightarrow
	(t - 1)(t + 1) = 0
\]
למטריצה $A$ יש שני ערכים עצמיים $-1, 1$.
לפי משפט 11.3.6 מטריצה $A$ לכסינה,
וניתן לראות כי $B$ מהווה מטריצה אלכסונית של $A$.

\subsection{סעיף ג'}
נראה כי המטריצות הבאות אינן דומות:
\[
	A =
	\begin{bmatrix}
		-3 & 1 & 1 \\
		1 & -3 & 1 \\
		1 & 1 & -3
	\end{bmatrix},
	B = 
	\begin{bmatrix}
		-4 & 0 & 0 \\
		0 & -1 & 0 \\
		0 & 1 & -1
	\end{bmatrix}
\]
אילו היו המטריצות $A$ ו־$B$ דומות אז הייתה להן אותה העקבה, לפי טענה 10.7.5.
מחישוב ישיר אנו רואים כי $\tr(A) = -9 \ne -6 = \tr(B)$,
לכן לא יתכן שהמטריצות דומות.

\section{שאלה 5}
יהיו $u$ ו־$v$ שני וקטורים שונים מווקטור האפס ב־$\RR^n$.
נתון כי $(*) \Vert u \Vert = \Vert v \Vert$.
נמצא את כל הערכים של המספר הממשי $a$ כך שמתקיים
ש־$u + av$ אורתוגונלי ל־$u - av$. \\*
ידוע כי הווקטורים אורתוגונליים ולכן על־פי הגדרה 11.2.1 מתקיים:
\[
	(u - av)(u + av) = 0
\]
על־פי משפט 12.1.2:
\[
	u^2 - a^2 v^2 = 0
\]
לפי הגדרת הנורמה:
\[
	{\Vert u \Vert}^2
	-
	a^2 {\Vert v \Vert}^2
	= 0
\]
נציב $(*)$:
\[
	{\Vert u \Vert}^2 (a^2 - 1) = 0
\]
ידוע כי $u$ איננו וקטור $0$, לכן המשוואה מתאפסת רק כאשר:
\[
	a^2 - 1 = (a - 1)(a + 1) = 0
\]
לכן הווקטורים אורתוגונליים רק כאשר $a = -1, 1$.

\section{שאלה 6}
יהיו תת־מרחבים $U_1, U_2$ של המרחב $\mathbb{R}^n$.

\subsection{סעיף א'}
נוכיח כי אם $\mathbb{R}^n = U_1 \oplus U_2$ אז $U_1^\perp \cap U_2^\perp = \{ 0 \}$. \\*
על־פי משפט 7.7.2 מתקיים
\[
	\RR^n = U_1 \oplus U_1^\perp = U_1 \oplus U_2
\]
אז לפי הגדרת החיבור הישר קיים בסיס $B$ ככה שמתקיים
\[
	U_1 \oplus \Sp B = \RR^n
\]
והבסיס $B$ הוא בסיס גם עבור $U_1^\perp$ וגם עבור $U_2$.
אז כמובן $U_1^\perp = \Sp B = U_2$. באופן דומה נוכל להוכיח כי גם $U_2^\perp = U_1$.
על־פי משפט 7.7.2
\[
	U_1 \cap U_2 = \{0\} = U_2^\perp \cap U_1^\perp
\]
כמובן גם מתקיים
\[
	\RR^n = U_1 \oplus U_2 = U_1^\perp \oplus U_2^\perp
\]

\subsection{סעיף ב'}
נניח כי $\mathbb{R}^n = U_1 + U_2$.
הטענה כי $\mathbb{R}^n = U_1^\perp + U_2^\perp$ איננה בהכרח נכונה. \\*
נבחין בדוגמה הנגדית בה $U_1 = U_2 = \mathbb{R}^n$.
קל לראות כי במצב זה
\[
	U_1 + U_2 = \mathbb{R}^n + \mathbb{R}^n = \mathbb{R}^n 
\]
אבל המשלים האורתוגונלי של $U_1$ ושל $U_2$ הוא מרחב האפס,
והוא כמובן איננו פורש את המרחב $\mathbb{R}^n$.

\end{document}
