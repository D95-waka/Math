\documentclass[a4paper]{article}

% packages
\usepackage{inputenc, fontspec, amsmath, amsthm, amsfonts, polyglossia, catchfile}
\usepackage[a4paper, margin=50pt, includeheadfoot]{geometry} % set page margins

% style
\AddToHook{cmd/section/before}{\clearpage}	% Add line break before section
\linespread{1.5}
\setcounter{secnumdepth}{0}		% Remove default number tags from sections
\setmainfont{Libertinus Serif}
\setsansfont{Libertinus Sans}
\setmonofont{Libertinus Mono}
\setdefaultlanguage{hebrew}
\setotherlanguage{english}

% operators
\DeclareMathOperator\cis{cis}
\DeclareMathOperator\Sp{Sp}
\DeclareMathOperator\tr{tr}
\DeclareMathOperator\im{Im}
\DeclareMathOperator\diag{diag}
\DeclareMathOperator*\lowlim{\underline{lim}}
\DeclareMathOperator*\uplim{\overline{lim}}

% commands
\renewcommand\qedsymbol{\textbf{משל}}
\newcommand{\NN}[0]{\mathbb{N}}
\newcommand{\ZZ}[0]{\mathbb{Z}}
\newcommand{\QQ}[0]{\mathbb{Q}}
\newcommand{\RR}[0]{\mathbb{R}}
\newcommand{\CC}[0]{\mathbb{C}}
\newcommand{\getenv}[2][] {
  \CatchFileEdef{\temp}{"|kpsewhich --var-value #2"}{\endlinechar=-1}
  \if\relax\detokenize{#1}\relax\temp\else\let#1\temp\fi
}
\newcommand{\explain}[2] {
	\begin{flalign*}
		 && \text{#2} && \text{#1}
	\end{flalign*}
}

% headers
\getenv[\AUTHOR]{AUTHOR}
\author{\AUTHOR}
\date\today

\title{פתרון מטלה 03 --- מבוא ללוגיקה, 80423}

\begin{document}
\maketitle
\maketitleprint{}

\Question{}
תהי $L = \{ p_i \mid i < n \}$ שפה לתחשיב פסוקים, ותהי $L'$ שפה נוספת כזו. בנוסף יהיו הפסוקים $\alpha \in sent_L, \beta_0, \dots, \beta_{n - 1} \in sent_{L'}$. \\*
נגדיר $\alpha_{\beta_0, \dots, \beta_{n - 1}}^{p_0, \dots, p_{n - 1}} \in sent_{L'}$ כפסוק המתקבל מהחלפת כל מופע של $p_i$ ב־$\beta_i$.

\Subquestion{}
נוכיח כי הפונקציה ניתנת להגדרה ברקורסיה.
\begin{proof}
	נגדיר פונקציה $g : L \to sent_{L'}$ על־ידי $g(p_i) = \beta_i$ לכל $i < n$. \\*
	נגדיר $\epsilon_\lnot : sent_{L'} \to sent_{L'}$ על־ידי $\epsilon_\lnot(\varphi) = (\lnot \varphi)$.
	נבחין כי זהו סימון בלבד, קיצרנו את הכתיב $\epsilon_\lnot(\varphi) = \langle ( \lnot \rangle \frown \varphi \frown \langle ) \rangle$, וכך נמשיך ונעשה מטעמי קריאות. \\*
	עבור כל $\square \in \mathcal{B}$ נגדיר $\epsilon_\square : sent_{L'}^2 \to sent_{L'}$ על־ידי $\epsilon_\square(\varphi, \phi) = (\varphi \square \phi)$. \\*
	ממשפט ההגדרה ברקורסיה נובע כי קיימת $\bar{g} : sent_L \to sent_{L'}$ יחידה המקיימת את דרישות ההחלפה, ומתקיים $\bar{g}(\alpha) = \alpha_{\beta_0, \dots, \beta_{n - 1}}^{p_0, \dots, p_{n - 1}}$.
\end{proof}

\Subquestion{}
תהי פונקציית הערכת אמת $v : L' \to \{ \TT, \FF \}$.
אילו $u : L \to \{ \TT, \FF \}$ כך ש־$u(p_i) = \bar{v}(\beta_i)$ אז נוכיח כי מתקיים $\bar{u}(\alpha) = \bar{v}(\bar{g}(\alpha))$.
\begin{proof}
	הפונקציה $u$ משרה ממשפט הרקורסיה להערכות אמת פונקציה יחידה $\bar{u} : sent_L \to \{ \TT, \FF \}$, ועלינו להראות כי השוויון המבוקש מתקיים, נעשה זאת באינדוקציה על מבנה הפסוק. \\*
	נתחיל מבסיס האינדוקציה ונניח כי הפסוק הוא פסוק יסודי, אז מתקיים $\bar{u}(p_i) = u(p_i) = \bar{v}(\beta_i) = \bar{v}(\bar{g}(\alpha))$ וקיבלנו כי השוויון מתקיים מהגדרת $g$. \\*
	נניח עתה כי השוויון מתקיים עבור $\varphi$ ונוכיח כי הוא מתקיים גם עבור $(\lnot \varphi)$.
	מהגדרת הפונקציות הרקורסיביות $\bar{u}, \bar{v}$ נסיק כי
	\[
		\bar{u}(\lnot \varphi) = V_\lnot(\bar{u}(\varphi)) = V_\lnot(\bar{v}(\bar{g}(\varphi))) = \bar{v}(\lnot \bar{g}(\varphi))
	\]
	ומצאנו כי השוויון מתקיים, לכן נותר לנו רק להניח כי הוא מתקיים עבור $\psi$ ולהראות כי לכל $\square \in \mathcal{B}$ השוויון מתקיים עבור $(\varphi \square \phi)$:
	\[
		\bar{u}(\varphi \square \psi) = V_\square(\bar{u}(\varphi), \bar{u}(\psi)) = V_\square(\bar{v}(\bar{g}(\varphi)), \bar{v}(\bar{g}(\psi))) = \bar{v}(\bar{g}(\varphi) \square \bar{v}(\bar{g}(\psi)))
	\]
	ולכן השוויון מתקיים והוכחנו את מהלך האינדוקציה, אז נסיק כי השוויון אכן חל.
\end{proof}

\Subquestion{}
נסיק שאם $\alpha$ טאוטולוגיה אז גם $\bar{g}(\alpha)$ טאוטולוגיה, וכן אם $\alpha \equiv_{tau} \beta$ אז גם $\bar{g}(\alpha) \equiv_{tau} \bar{g}(\beta)$.
\begin{proof}
	נניח ש־$\alpha$ טאוטולוגיה, אז לכל $u$ הערכת אמת מתקיים $\bar{u}(\alpha) = \TT$, ותהי $v$ הערכת אמת ל־$L'$. \\*
	מהסעיף הקודם נסיק כי קיימת $u$ הערכת אמת כך ש־$\bar{u}(\alpha) = \bar{v}(\bar{g}(\alpha))$, אבל $\bar{u}(\alpha) = \TT$, לכן גם $\bar{v}(\bar{g}(\alpha)) = \TT$, דהינו היא אכן טאוטולוגיה.

	נניח עתה שמתקיים $\alpha \equiv_{tau} \beta$, אז לכל הערכת אמת $u$ של $L$ מתקיים $\bar{u}(\alpha) = \bar{u}(\beta)$. \\*
	תהי $v$ הערכת אמת של $L'$, אז עבור $u$ מתאימה של $v$ נקבל
	\[
		\bar{v}(\bar{g}(\alpha)) = \bar{u}(\alpha) = \bar{u}(\beta) = \bar{v}(\bar{g}(\beta))
	\]
	ולכן $\bar{g}(\alpha) \equiv_{tau} \bar{g}(\beta)$.
\end{proof}

\Subquestion{}
נסתור את הטענה כי יתכן ש־$\bar{g}(\alpha)$ טאוטולוגיה ו־$\alpha$ אינה טאוטולוגיה על־ידי דוגמה נגדית.
\begin{solution}
	נניח $n = 1$ וכן $L = L'$, עוד נניח $\beta_0 = (p_0 \to p_0)$, לכן $\beta_0$ טאוטולוגיה, נגדיר $\alpha = p_0$ ונקבל ש־$\alpha$ יכולה לקבל אמת ושקר, אבל $\bar{g}(\alpha) = (p_0 \to p_0)$ וזו כמובן טאוטולוגיה.
\end{solution}

\Question{}
יהי $G = (V, E)$ גרף. נאמר ש־$G$ גרף סופי מקומית אם לכל $v \in V$ מתקיים שקבוצת השכנים $N(v) = \{ w \in V \mid \{ v, w \} \in E \}$ היא סופית.
נאמר ש־$G$ הוא דו־צדדי אם ישנה חלוקה $V = V_0 \uplus V_1$ של הקודקודים כך שאם $u, w \in V$ מקיימים $vEw$ אז הם לא שניהם ב־$V_0$ או ב־$V_1$.
עבור גרף דו־צדדי $G = (V_0 \uplus V_1, E)$ נגדיר צימוד מושלם ב־$G$ להיות פונקציה חד־חד ערכית $f : V_0 \to V_1$ כך שלכל $v \in V_0$ מתקיים $vEf(v)$. \\*
תהי השפה $L = \{ p_{v, w} : v \in V_0, w \in V_1, vEw \}$ שפה לתחשיב פסוקים, ונסמן
\begin{align*}
	\Sigma_0 & = \{ (\lnot (p_{v, w_1} \cap  p_{v, w_2}) ) \mid v \in V, w_1 \ne w_2 \} \\
	\Sigma_1 & = \{ (\lnot (p_{v_1, w} \cap  p_{v_2, w}) ) \mid w \in V, v_1 \ne v_2 \} \\
	\Sigma_2 & = \left\{ \left( \bigvee_{k = 0}^{n_v - 1} p_{v, w_k} \right) \mid v \in V, N(v) = \{ w_0, \dots, w_{n_v - 1} \} \right\}
\end{align*}
וכן $\Sigma = \Sigma_0 \cup \Sigma_1 \cup \Sigma_2$.

\Subquestion{}
נוכיח שאם הקבוצה $\Sigma$ ספיקה אז יש לגרף $G$ צימוד מושלם.

\end{document}
