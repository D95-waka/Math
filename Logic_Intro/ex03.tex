\documentclass[a4paper]{article}

% packages
\usepackage{inputenc, amsmath, amsthm, thmtools, amsfonts, amssymb, luacode, catchfile, tikzducks, hyperref}
\usepackage[a4paper, margin=50pt, includeheadfoot]{geometry} % set page margins
\usepackage[shortlabels]{enumitem}
\usepackage[skip=3pt, indent=0pt]{parskip}

% language
\usepackage[bidi=basic, layout=tabular, provide=*]{babel}
\babelprovide[main, import]{hebrew}
\babelprovide{rl}
\babelfont{rm}{Libertinus Serif}
\babelfont{sf}{Libertinus Sans}
\babelfont{tt}{Libertinus Mono}

% style
\AddToHook{cmd/section/before}{\clearpage}	% Add line break before section
\linespread{1.3}
\setcounter{secnumdepth}{0}		% Remove default number tags from sections, this won't do well with theorems
\AtBeginDocument{\setlength{\belowdisplayskip}{3pt}}
\AtBeginDocument{\setlength{\abovedisplayskip}{3pt}}

% operators
\DeclareMathOperator\cis{cis}
\DeclareMathOperator\Sp{Sp}
\DeclareMathOperator\tr{tr}
\DeclareMathOperator\im{Im}
\DeclareMathOperator\re{Re}
\DeclareMathOperator\diag{diag}
\DeclareMathOperator*\lowlim{\underline{lim}}
\DeclareMathOperator*\uplim{\overline{lim}}
\DeclareMathOperator\rng{rng}
\DeclareMathOperator\Sym{Sym}
\DeclareMathOperator\Arg{Arg}
\DeclareMathOperator\Log{Log}
\DeclareMathOperator\dom{dom}

% commands
%\renewcommand\qedsymbol{\textbf{מש''ל}}
%\renewcommand\qedsymbol{\fbox{\emoji{lizard}}}
\newcommand{\NN}[0]{\mathbb{N}}
\newcommand{\ZZ}[0]{\mathbb{Z}}
\newcommand{\QQ}[0]{\mathbb{Q}}
\newcommand{\RR}[0]{\mathbb{R}}
\newcommand{\CC}[0]{\mathbb{C}}
\newcommand{\FF}[0]{\mathbb{F}}
\newcommand{\PP}[0]{\mathbb{P}}
\newcommand{\TT}[0]{\mathbb{T}}
\newcommand{\acts}[0]{\circlearrowright}
\newcommand{\explain}[2] {
	\begin{flalign*}
		 && \text{#2} && \text{#1}
	\end{flalign*}
}
\newcommand{\maketitleprint}[0]{ \begin{center}
	\begin{tikzpicture}[scale=3]
		\duck[graduate=gray!20!black, tassel=red!70!black]
	\end{tikzpicture}	
\end{center}
}

% theorem commands
\newtheoremstyle{c_remark}
	{}	% Space above
	{}	% Space below
	{}% Body font
	{}	% Indent amount
	{\bfseries}	% Theorem head font
	{}	% Punctuation after theorem head
	{.5em}	% Space after theorem head
	{\thmname{#1}\thmnumber{ #2}\thmnote{ \normalfont{\text{(#3)}}}}	% head content
\newtheoremstyle{c_definition}
	{3pt}	% Space above
	{3pt}	% Space below
	{}% Body font
	{}	% Indent amount
	{\bfseries}	% Theorem head font
	{}	% Punctuation after theorem head
	{.5em}	% Space after theorem head
	{\thmname{#1}\thmnumber{ #2}\thmnote{ \normalfont{\text{(#3)}}}}	% head content
\newtheoremstyle{c_plain}
	{3pt}	% Space above
	{3pt}	% Space below
	{\itshape}% Body font
	{}	% Indent amount
	{\bfseries}	% Theorem head font
	{}	% Punctuation after theorem head
	{.5em}	% Space after theorem head
	{\thmname{#1}\thmnumber{ #2}\thmnote{ \text{(#3)}}}	% head content

\theoremstyle{c_plain}
\newtheorem{theorem}{משפט}[section]
\newtheorem{lemma}[theorem]{למה}
\newtheorem{proposition}[theorem]{טענה}
\newtheorem*{proposition*}{טענה}
%\newtheorem{corollary}[theorem]{אין חלופה עברית}

\theoremstyle{c_definition}
\newtheorem{definition}[theorem]{הגדרה}
\newtheorem*{definition*}{הגדרה}
\newtheorem{example}{דוגמה}[section]
\newtheorem{exercise}{תרגיל}[section]

\theoremstyle{c_remark}
\newtheorem*{remark}{הערה}
\newtheorem*{solution}{פתרון}
\newtheorem{conclusion}[theorem]{מסקנה}
\newtheorem{notation}[theorem]{סימון}

% Questions related commands
\newcounter{question}
\setcounter{question}{1}
\newcounter{sub_question}
\setcounter{sub_question}{1}

\newcommand{\question}[1][0]{
	\ifthenelse{#1 = 0}{}{\setcounter{question}{#1}}
	\subsection{שאלה \arabic{question}}
	\addtocounter{question}{1}
	\setcounter{sub_question}{1}
}

\newcommand{\subquestion}[1][0]{
	\ifthenelse{#1 = 0}{}{\setcounter{sub_question}{#1}}
	\subsubsection{סעיף \localecounter{letters.gershayim}{sub_question}}
	\addtocounter{sub_question}{1}
}

% import lua and start of document
\directlua{common = require ('../common')}

\GetEnv{AUTHOR}

% headers
\author{\AUTHOR}
\date\today

\title{פתרון מטלה 03 --- מבוא ללוגיקה, 80423}

\begin{document}
\maketitle
\maketitleprint{}

\Question{}
תהי $L = \{ p_i \mid i < n \}$ שפה לתחשיב פסוקים, ותהי $L'$ שפה נוספת כזו. בנוסף יהיו הפסוקים $\alpha \in sent_L, \beta_0, \dots, \beta_{n - 1} \in sent_{L'}$. \\*
נגדיר $\alpha_{\beta_0, \dots, \beta_{n - 1}}^{p_0, \dots, p_{n - 1}} \in sent_{L'}$ כפסוק המתקבל מהחלפת כל מופע של $p_i$ ב־$\beta_i$.

\Subquestion{}
נוכיח כי הפונקציה ניתנת להגדרה ברקורסיה.
\begin{proof}
	נגדיר פונקציה $g : L \to sent_{L'}$ על־ידי $g(p_i) = \beta_i$ לכל $i < n$. \\*
	נגדיר $\epsilon_\lnot : sent_{L'} \to sent_{L'}$ על־ידי $\epsilon_\lnot(\varphi) = (\lnot \varphi)$.
	נבחין כי זהו סימון בלבד, קיצרנו את הכתיב $\epsilon_\lnot(\varphi) = \langle ( \lnot \rangle \frown \varphi \frown \langle ) \rangle$, וכך נמשיך ונעשה מטעמי קריאות. \\*
	עבור כל $\square \in \mathcal{B}$ נגדיר $\epsilon_\square : sent_{L'}^2 \to sent_{L'}$ על־ידי $\epsilon_\square(\varphi, \phi) = (\varphi \square \phi)$. \\*
	ממשפט ההגדרה ברקורסיה נובע כי קיימת $\bar{g} : sent_L \to sent_{L'}$ יחידה המקיימת את דרישות ההחלפה, ומתקיים $\bar{g}(\alpha) = \alpha_{\beta_0, \dots, \beta_{n - 1}}^{p_0, \dots, p_{n - 1}}$.
\end{proof}

\Subquestion{}
תהי פונקציית הערכת אמת $v : L' \to \{ \TT, \FF \}$.
אילו $u : L \to \{ \TT, \FF \}$ כך ש־$u(p_i) = \bar{v}(\beta_i)$ אז נוכיח כי מתקיים $\bar{u}(\alpha) = \bar{v}(\bar{g}(\alpha))$.
\begin{proof}
	הפונקציה $u$ משרה ממשפט הרקורסיה להערכות אמת פונקציה יחידה $\bar{u} : sent_L \to \{ \TT, \FF \}$, ועלינו להראות כי השוויון המבוקש מתקיים, נעשה זאת באינדוקציה על מבנה הפסוק. \\*
	נתחיל מבסיס האינדוקציה ונניח כי הפסוק הוא פסוק יסודי, אז מתקיים $\bar{u}(p_i) = u(p_i) = \bar{v}(\beta_i) = \bar{v}(\bar{g}(\alpha))$ וקיבלנו כי השוויון מתקיים מהגדרת $g$. \\*
	נניח עתה כי השוויון מתקיים עבור $\varphi$ ונוכיח כי הוא מתקיים גם עבור $(\lnot \varphi)$.
	מהגדרת הפונקציות הרקורסיביות $\bar{u}, \bar{v}$ נסיק כי
	\[
		\bar{u}(\lnot \varphi) = V_\lnot(\bar{u}(\varphi)) = V_\lnot(\bar{v}(\bar{g}(\varphi))) = \bar{v}(\lnot \bar{g}(\varphi))
	\]
	ומצאנו כי השוויון מתקיים, לכן נותר לנו רק להניח כי הוא מתקיים עבור $\psi$ ולהראות כי לכל $\square \in \mathcal{B}$ השוויון מתקיים עבור $(\varphi \square \phi)$:
	\[
		\bar{u}(\varphi \square \psi) = V_\square(\bar{u}(\varphi), \bar{u}(\psi)) = V_\square(\bar{v}(\bar{g}(\varphi)), \bar{v}(\bar{g}(\psi))) = \bar{v}(\bar{g}(\varphi) \square \bar{v}(\bar{g}(\psi)))
	\]
	ולכן השוויון מתקיים והוכחנו את מהלך האינדוקציה, אז נסיק כי השוויון אכן חל.
\end{proof}

\Subquestion{}
נסיק שאם $\alpha$ טאוטולוגיה אז גם $\bar{g}(\alpha)$ טאוטולוגיה, וכן אם $\alpha \equiv_{tau} \beta$ אז גם $\bar{g}(\alpha) \equiv_{tau} \bar{g}(\beta)$.
\begin{proof}
	נניח ש־$\alpha$ טאוטולוגיה, אז לכל $u$ הערכת אמת מתקיים $\bar{u}(\alpha) = \TT$, ותהי $v$ הערכת אמת ל־$L'$. \\*
	מהסעיף הקודם נסיק כי קיימת $u$ הערכת אמת כך ש־$\bar{u}(\alpha) = \bar{v}(\bar{g}(\alpha))$, אבל $\bar{u}(\alpha) = \TT$, לכן גם $\bar{v}(\bar{g}(\alpha)) = \TT$, דהינו היא אכן טאוטולוגיה.

	נניח עתה שמתקיים $\alpha \equiv_{tau} \beta$, אז לכל הערכת אמת $u$ של $L$ מתקיים $\bar{u}(\alpha) = \bar{u}(\beta)$. \\*
	תהי $v$ הערכת אמת של $L'$, אז עבור $u$ מתאימה של $v$ נקבל
	\[
		\bar{v}(\bar{g}(\alpha)) = \bar{u}(\alpha) = \bar{u}(\beta) = \bar{v}(\bar{g}(\beta))
	\]
	ולכן $\bar{g}(\alpha) \equiv_{tau} \bar{g}(\beta)$.
\end{proof}

\Subquestion{}
נסתור את הטענה כי יתכן ש־$\bar{g}(\alpha)$ טאוטולוגיה ו־$\alpha$ אינה טאוטולוגיה על־ידי דוגמה נגדית.
\begin{solution}
	נניח $n = 1$ וכן $L = L'$, עוד נניח $\beta_0 = (p_0 \to p_0)$, לכן $\beta_0$ טאוטולוגיה, נגדיר $\alpha = p_0$ ונקבל ש־$\alpha$ יכולה לקבל אמת ושקר, אבל $\bar{g}(\alpha) = (p_0 \to p_0)$ וזו כמובן טאוטולוגיה.
\end{solution}

\Question{}
יהי $G = (V, E)$ גרף. נאמר ש־$G$ גרף סופי מקומית אם לכל $v \in V$ מתקיים שקבוצת השכנים $N(v) = \{ w \in V \mid \{ v, w \} \in E \}$ היא סופית.
נאמר ש־$G$ הוא דו־צדדי אם ישנה חלוקה $V = V_0 \uplus V_1$ של הקודקודים כך שאם $u, w \in V$ מקיימים $vEw$ אז הם לא שניהם ב־$V_0$ או ב־$V_1$.
עבור גרף דו־צדדי $G = (V_0 \uplus V_1, E)$ נגדיר צימוד מושלם ב־$G$ להיות פונקציה חד־חד ערכית $f : V_0 \to V_1$ כך שלכל $v \in V_0$ מתקיים $vEf(v)$. \\*
תהי השפה $L = \{ p_{v, w} : v \in V_0, w \in V_1, vEw \}$ שפה לתחשיב פסוקים, ונסמן
\begin{align*}
	\Sigma_0 & = \{ (\lnot (p_{v, w_1} \cap  p_{v, w_2}) ) \mid v \in V, w_1 \ne w_2 \} \\
	\Sigma_1 & = \{ (\lnot (p_{v_1, w} \cap  p_{v_2, w}) ) \mid w \in V, v_1 \ne v_2 \} \\
	\Sigma_2 & = \left\{ \left( \bigvee_{k = 0}^{n_v - 1} p_{v, w_k} \right) \mid v \in V, N(v) = \{ w_0, \dots, w_{n_v - 1} \} \right\}
\end{align*}
וכן $\Sigma = \Sigma_0 \cup \Sigma_1 \cup \Sigma_2$.

\Subquestion{}
נוכיח שאם הקבוצה $\Sigma$ ספיקה אז יש לגרף $G$ צימוד מושלם.
\begin{proof}
	נתון כי $\Sigma$ ספיקה ולכן קיימת הערכת אמת $u : L \to \{ \TT, \FF \}$ כך ש־$\forall \varphi \in \Sigma, \bar{u}(\varphi) = \TT$. \\*
	מהנתון על מבנה $\Sigma$ נסיק כי $v$ מקיימת גם את $\Sigma_0, \Sigma_1, \Sigma_2$. \\*
	עתה נגדיר $f = \{ \langle v, w \rangle \mid v \in V_0, w \in W_0, u(p_{v, w}) = \TT \}$, זוהי כמובן קבוצה מאקסיומת הפרדה, ואף יחס כך ש־$\text{dom}(f) \subseteq V_0, \text{rng}(f) \subseteq V_1$,
	אנו רוצים להוכיח כי $f$ מקיימת את תכונת הפונקציה וכי $\text{dom}(f) = V_0$. \\*
	נניח בשלילה ש־$f$ לא מקיימת את תכונת הפונקציה, ולכן קיימים $\langle v, w \rangle, \langle v, w' \rangle \in f$ עבור $w \ne w'$. \\*
	הפסוק המתאים הוא $p_{v, w} \cap p_{v, w'}$, אבל נבחין כי $\Sigma_0 \models \lnot (p_{v, w} \cap p_{v, w'})$ ולכן ההנחה כי הפסוק נובע מ־$\Sigma$ מובילה לסתירה, לכן $f$ היא פונקציה. \\*
	נעבור להוכיח כי תחומה הוא $V_0$. לכל $v \in V_0$ קיים פסוק $\varphi \in \Sigma_2$ כך ש־$\bar{u}(\varphi) = \TT$ מתכונות של איווי ומההנחה שלנו, ולכן $v \in \text{dom}(f)$, לכן בהכרח $\text{dom}(f) = V_0$. \\*
	מצאנו אם כך ש־$f : V_0 \to V_1$, נותר להוכיח כי היא חד־חד ערכית.
	למעשה הוכחה זו זהה להוכחה כי $f$ פונקציה ובהתבסס על $\Sigma_1$, נניח בשלילה כי היא לא חד־חד ערכית ונקבל שני זוגות סדורים שמוכלים ב־$f$, אבל הם יהוו סתירה להגדרת $f$. \\*
	מצאנו אם כך ש־$f$ עומדת בכל התנאים שהצגנו להוות צימוד מושלם.
\end{proof}

\Subquestion{}
נוכיח כי אם ל־$G$ יש צימוד מושלם אז $\Sigma$ ספיקה.
\begin{proof}
	נניח $f$ הצימוד המושלם של $G$ שהנחנו שקיים. \\*
	נגדיר $u : L \to \{ \TT, \FF$ על־ידי
		\[
			u(p_{v, w}) = \begin{cases}
				\TT, & \langle v, w \rangle \in f \\
				\FF, & \langle v, w \rangle \notin f
		\end{cases}
	\]
	אנו יודעים כי $f$ מקיימת את תכונת הפונקציה וזו שקולה לקבוצת הפסוקים $\Sigma_0$, כמו־כן היא חד־חד ערכית ותכונה זו שקולה ל־$\Sigma_1$. \\*
	לבסוף $\text{dom}(f) = V_0$ ולכן לכל $v \in V_0$ מתקיים $\langle v, f(v) \rangle \in f$ ומתקיים גם $\Sigma_2$, לכן $v$ מספק את שלוש הקבוצות ובהתאם גם את איחודן,
	דהינו $\Sigma = \Sigma_0 \cup \Sigma_1 \cup \Sigma_2$ היא ספיקה, וספיקה בפרט על־ידי $u$.
\end{proof}

\Question{}
תהי $L$ שפה כך ש־$|L| = k$.

\Subquestion{}
נוכיח שלכל $\psi \in sent_L$ קיים $\varphi \in sent_L$ כך ש־$\varphi = \left( \bigwedge_{i < m} \left( \bigvee_{j < n} \varphi_{i, j} \right)\right)$
כך ש־$\forall i, j (\exists p \in L (\varphi_{i, j} = p \lor \varphi_{i, j} = (\lnot p)))$,
ומתקיים $\varphi \equiv_{tau} \psi$.
\begin{proof}
	ממשפט שהוכח בהרצאה נובע כי קיים ל־$\psi \equiv_{tau} \phi \in sent_L$ עבור $\phi = \left( \bigvee_{i < n} \left( \bigwedge_{j < n} \phi_{i, j} \right) \right)$,
	כאשר $\phi_{i, j} = p \in L \lor (\lnot p)$. \\*
	עוד אנו יודעים כי $\phi \equiv (\lnot (\lnot \phi))$ וכי מחוקי דה־מורגן
	\[
		\phi
		\equiv \left( \lnot \left( \bigwedge_{i < n} \lnot \left( \bigwedge_{j < n} \phi_{i, j} \right) \right) \right)
		\equiv \left( \lnot \left( \bigwedge_{i < n} \left( \bigvee_{j < n} (\lnot \phi_{i, j}) \right) \right) \right)
		\equiv \left( \lnot \left( \bigwedge_{i < n} \left( \bigvee_{j < n} \phi_{i, j} \right) \right) \right)
	\]
	כאשר במעבר האחרון הגדרנו את $\phi_{i, j}$ מחדש כך שהוא עדיין עומד בהגדרתו המקורית. \\*
	נבחין כי מצאנו פסוק שקול בצורה של שלילה של פסוק גימום נורמלי, ולכן נוכל לבצע את התהליך מחדש עבור $(\lnot \varphi)$ ולקבל כי קיים $\phi \equiv \psi$ כך שגם
	\[
		\phi \equiv \left( \bigwedge_{i < n} \left( \bigvee_{j < n} \phi_{i, j} \right) \right)
	\]
\end{proof}

\Subquestion{}
נסמן ב־$X$ את קבוצת הערכות האמת על $L$. עבור פסוק $\psi$ נתון נסמן $T_\psi \subseteq X$ את קבוצת הערכות האמת $v : L \to \{ \TT, \FF \}$ כך שמתקיים $\bar{v}(\psi) = \TT$. \\*
נוכיח שלכל שני פסוקים $\varphi_1, \varphi_2 \in sent_L$ מתקיים $\varphi_1 \models \varphi_2$ אם ורק אם $T_{\varphi_1} \subseteq T_{\varphi_2}$.
\begin{proof}
	נניח כי $\varphi_1 \models \varphi_2$. \\*
	תהי הערכת אמת $v \in T_{\varphi_1}$, אז נקבל $\bar{v}(\varphi_1) = \TT$.
	אבל מהגדרת הגרירה הטאוטולוגית נסיק גם $\bar{v}(\varphi_2) = \TT$, לכן $v \in T_{\varphi_2}$, אז $T_{\varphi_1} \subseteq T_{\varphi_2}$.

	נניח עתה כי $T_{\varphi_1} \subseteq T_{\varphi_2}$. \\*
	לכל הערכת אמת $v \in T_{\varphi_1}$ נקבל $v \in T_{\varphi_2}$. עתה נבחין כי ההגדרה הזו לכל $v$ מתלכדת עם ההגדרה ''לכל $v$ כך ש־$\bar{v}(\varphi_1) = \TT$'' מתלכדות מהגדרת $T_{\varphi_1}$,
	ולכן $\bar{v}(\varphi_1) = \TT$ וגם $\bar{v}(\varphi_2) = \TT$ בשל ההכלה של $T_{\varphi_1}$, לכן לפי הגדרת הגרירה הטאוטולוגית מתקיים $\varphi_1 \models \varphi_2$.
\end{proof}

\Subquestion{}
נגדיר שרשרת גרירה באורך $l$ להיות סדרת פסוקים ב־$L$, $\langle \psi_0, \dots, \psi_{l - 1} \rangle$ כך שלכל $i < l - 1$ מתקיים $\psi_{i} \models \psi_{i + 1}$ וגם $\psi_{i + 1} \not\models \psi_i$. \\*
נמצא את האורך המקסימלי של שרשרת ב־$L$.
\begin{solution}
	מטענת סעיף ב' נבחין כי כל שרשרת גרירה משרה שרשרת קבוצות $\langle T_{\psi_0}, \dots, T_{\psi_{l - 1}} \rangle$ כך ש־$T_{\psi_i} \subsetneq T_{i + 1}$. \\*
	אנו גם יודעים כי הפסוק בעל הקבוצה כזו המינימלית היא זו המושרה מהקבוצה הריקה, זוהי כמובן סתירה, לכן נגדיר $\psi_0 = \perp$. \\*
	באופן דומה הקבוצה $T$ כזו המקסימלית היא הקבוצה $T = X$, אשר מקושרת לפסוק שערכו $\TT$ תמיד, דהינו $\psi_{l - 1} = (p \to p)$ עבור $p \in L$ כלשהו. \\*
	עתה נאמר כי במטרה להשיג את השרשרת הארוכה ביותר, נרצה בכל שלב להקטין את גודל $|T_{\psi_i}|$ במינימום, הוא כמובן $1$, לכן נבחר איבר כלשהו $v \in T_{\psi_i}$ ונגדיר $T_{\psi_{i + 1}} = T_{\psi_i} \setminus v$. \\*
	בהתאם קיים $\psi_{i + 1}$ שמקיים קבוצה זו, ואותו נגדיר כאיבר הבא בשרשרת באופן רקורסיבי. \\*
	נותר אם כך להראות שבבנייה זו האיבר האחרון הוא אכן סתירה, ואכן כל תנאי ש־$|T_{\psi_i}| > 0$ נוכל להמשיך את התהליך, וכאשר $|T_{\psi_i}| = 0$ כבר ראינו כי הוא סתירה. \\*
	עתה נעבור לחישוב גודל השרשרת הזו, למעשה גודלה הוא $|X|$ מדרך הבנייה של הסדרה, ואנו יודעים כי $|X| = |\{ v \mid v : L \to \{\TT, \FF\}\}| = 2^k$.
\end{solution}

\Subquestion{}
נוכיח שקיימים $2^{2^k}$ פסוקים ב־$L$ עד כדי שקילות טאוטולוגית.
\begin{proof}
	בסעיף הקודם ראינו כי אנו יכולים לקבוע פסוק בהתאם להגדרה האם הוא מתקיים או לא עבור $v$ הערכת אחת כלשהי, אנו גם מצאנו כי ישנן $2^k$ הערכות אמת שונות ל־$L$. \\*
	לכל הערכת אמת כזו קיים ערך יחיד של אמת או שקר שפסוק יקבל, ובהתאם למידע זה ומההוכחה לצורת איווי נורמלית נוכל להסיק כי אם אנו יודעים עבור פסוק לכל הערכת אמת מה התוצאה שלו, אז הוא שקול לפסוק נורממלי יחיד. \\*
	נסיק אם כן שקיימים $2^{2^k}$ פסוקים נורמליים כאלה, וכל פסוק ב־$sent_L$ שקול לאחד מהם.
\end{proof}

\end{document}
