\documentclass[a4paper]{article}

% packages
\usepackage{inputenc, amsmath, amsthm, thmtools, amsfonts, amssymb, luacode, catchfile, tikzducks, hyperref}
\usepackage[a4paper, margin=50pt, includeheadfoot]{geometry} % set page margins
\usepackage[shortlabels]{enumitem}
\usepackage[skip=3pt, indent=0pt]{parskip}

% language
\usepackage[bidi=basic, layout=tabular, provide=*]{babel}
\babelprovide[main, import]{hebrew}
\babelprovide{rl}
\babelfont{rm}{Libertinus Serif}
\babelfont{sf}{Libertinus Sans}
\babelfont{tt}{Libertinus Mono}

% style
\AddToHook{cmd/section/before}{\clearpage}	% Add line break before section
\linespread{1.3}
\setcounter{secnumdepth}{0}		% Remove default number tags from sections, this won't do well with theorems
\AtBeginDocument{\setlength{\belowdisplayskip}{3pt}}
\AtBeginDocument{\setlength{\abovedisplayskip}{3pt}}

% operators
\DeclareMathOperator\cis{cis}
\DeclareMathOperator\Sp{Sp}
\DeclareMathOperator\tr{tr}
\DeclareMathOperator\im{Im}
\DeclareMathOperator\re{Re}
\DeclareMathOperator\diag{diag}
\DeclareMathOperator*\lowlim{\underline{lim}}
\DeclareMathOperator*\uplim{\overline{lim}}
\DeclareMathOperator\rng{rng}
\DeclareMathOperator\Sym{Sym}
\DeclareMathOperator\Arg{Arg}
\DeclareMathOperator\Log{Log}
\DeclareMathOperator\dom{dom}

% commands
%\renewcommand\qedsymbol{\textbf{מש''ל}}
%\renewcommand\qedsymbol{\fbox{\emoji{lizard}}}
\newcommand{\NN}[0]{\mathbb{N}}
\newcommand{\ZZ}[0]{\mathbb{Z}}
\newcommand{\QQ}[0]{\mathbb{Q}}
\newcommand{\RR}[0]{\mathbb{R}}
\newcommand{\CC}[0]{\mathbb{C}}
\newcommand{\FF}[0]{\mathbb{F}}
\newcommand{\PP}[0]{\mathbb{P}}
\newcommand{\TT}[0]{\mathbb{T}}
\newcommand{\acts}[0]{\circlearrowright}
\newcommand{\explain}[2] {
	\begin{flalign*}
		 && \text{#2} && \text{#1}
	\end{flalign*}
}
\newcommand{\maketitleprint}[0]{ \begin{center}
	\begin{tikzpicture}[scale=3]
		\duck[graduate=gray!20!black, tassel=red!70!black]
	\end{tikzpicture}	
\end{center}
}

% theorem commands
\newtheoremstyle{c_remark}
	{}	% Space above
	{}	% Space below
	{}% Body font
	{}	% Indent amount
	{\bfseries}	% Theorem head font
	{}	% Punctuation after theorem head
	{.5em}	% Space after theorem head
	{\thmname{#1}\thmnumber{ #2}\thmnote{ \normalfont{\text{(#3)}}}}	% head content
\newtheoremstyle{c_definition}
	{3pt}	% Space above
	{3pt}	% Space below
	{}% Body font
	{}	% Indent amount
	{\bfseries}	% Theorem head font
	{}	% Punctuation after theorem head
	{.5em}	% Space after theorem head
	{\thmname{#1}\thmnumber{ #2}\thmnote{ \normalfont{\text{(#3)}}}}	% head content
\newtheoremstyle{c_plain}
	{3pt}	% Space above
	{3pt}	% Space below
	{\itshape}% Body font
	{}	% Indent amount
	{\bfseries}	% Theorem head font
	{}	% Punctuation after theorem head
	{.5em}	% Space after theorem head
	{\thmname{#1}\thmnumber{ #2}\thmnote{ \text{(#3)}}}	% head content

\theoremstyle{c_plain}
\newtheorem{theorem}{משפט}[section]
\newtheorem{lemma}[theorem]{למה}
\newtheorem{proposition}[theorem]{טענה}
\newtheorem*{proposition*}{טענה}
%\newtheorem{corollary}[theorem]{אין חלופה עברית}

\theoremstyle{c_definition}
\newtheorem{definition}[theorem]{הגדרה}
\newtheorem*{definition*}{הגדרה}
\newtheorem{example}{דוגמה}[section]
\newtheorem{exercise}{תרגיל}[section]

\theoremstyle{c_remark}
\newtheorem*{remark}{הערה}
\newtheorem*{solution}{פתרון}
\newtheorem{conclusion}[theorem]{מסקנה}
\newtheorem{notation}[theorem]{סימון}

% Questions related commands
\newcounter{question}
\setcounter{question}{1}
\newcounter{sub_question}
\setcounter{sub_question}{1}

\newcommand{\question}[1][0]{
	\ifthenelse{#1 = 0}{}{\setcounter{question}{#1}}
	\subsection{שאלה \arabic{question}}
	\addtocounter{question}{1}
	\setcounter{sub_question}{1}
}

\newcommand{\subquestion}[1][0]{
	\ifthenelse{#1 = 0}{}{\setcounter{sub_question}{#1}}
	\subsubsection{סעיף \localecounter{letters.gershayim}{sub_question}}
	\addtocounter{sub_question}{1}
}

% import lua and start of document
\directlua{common = require ('../common')}

\GetEnv{AUTHOR}

% headers
\author{\AUTHOR}
\date\today

\title{פתרון מטלה 05 --- מבוא ללוגיקה, 80423}

\begin{document}
\maketitle
\maketitleprint{}

\question{}
\subquestion{}
נגדיר פעולות וקבועים על $sent_L$ על־ידי
\[
	\varphi + \psi = (\varphi \lor \psi),
	\quad
	\varphi \cdot \psi = (\varphi \land \psi),
	\quad
	- \varphi = (\lnot \varphi),
	\quad
	0 = \perp,
	\quad
	1 = (\lnot \perp)
\]
נראה שהקבוצה $sent_L$ יחד עם פעולות אלה היא לא אלגברה בוליאנית.
\begin{solution}
	תהי שפה $L = \{ P \}$, אז מהשלמה נובע
	\[
		(P \land (\lnot P))
		= P \cdot (- P)
		= 0
		= (\lnot (P \to P))
	\]
	ואלה הן כמובן סדרות שונות של סימנים, לכן קיבלנו סתירה להנחה שספיגה מתקיימת, ובהתאם גם סתירה להנחה כי זו היא אלגברה בוליאנית.
\end{solution}

\subquestion{}
נראה כי כל אחת מהפעולות שהגדרנו זה עתה מכבדת את יחס השקילות הטאוטולוגית.
\begin{proof}
	למען הקריאות נסמן $\equiv$ כשכוונתנו $\equiv_{tau}$ לשאלה זו. \\*
	נניח ש־$\varphi, \varphi' \in sent_L$ ונניח גם $\varphi \equiv \varphi'$, לכן
	\[
		(-\varphi) = (\lnot \varphi) \equiv (\lnot \varphi') = (-\varphi')
	\]
	כאשר המעבר השני נובע ממהלכים שראינו עבור ההגדרה הרקורסיבית לחישוב ערך אמת. \\*
	באותו אופן אם גם $\psi, \psi' \in sent_L$ ו־$\psi \equiv \psi'$ אז מתקיים לכל
	\[
		\varphi \cdot \psi
		= (\varphi \land \psi)
		\equiv (\varphi' \land \psi')
		= \varphi' \cdot \psi'
	\]
	ולבסוף גם
	\[
		\varphi + \psi
		= (\varphi \lor \psi)
		\equiv (\varphi' \lor \psi')
		= \varphi' + \psi'
	\]
	ולכן מצאנו שהפעולות המוגדרות אכן מכבדות את יחס שקילות זה.
\end{proof}

\subquestion{}
נגדיר $X = sent_L / \equiv$ ונגדיר את הפעולות הבאות
\[
	{[\varphi]}_\equiv \tilde{+} {[\psi]}_\equiv = {[\varphi + \psi]}_\equiv
	\quad
	{[\varphi]}_\equiv \tilde{\cdot} {[\psi]}_\equiv = {[\varphi \cdot \psi]}_\equiv
	\quad
	\tilde{-} {[\varphi]}_\equiv = {[-\varphi]}_\equiv
\]
ונוכיח ש־$X$ יחד עם הפעולות החדשות היא אלגברה בוליאנית.
\begin{proof}
	נבחין כי יש שתי דרכים להוכיח טענה זו, הדרך הראשונה נובעת מהגדרת האלגברה הבוליאנית, לדוגמה ההוכחה שהפעולות הדו־מקומיות קומוטטיביות ואסוציאטיביות. \\*
	נבחר נציגים $\varphi, \psi$, אז $\varphi + \psi \in {[\varphi + \psi]}_\equiv$ וכן $\varphi + \psi = (\varphi \lor \psi)$, אך איווי אסוציאטיבי וקומוטטיבי כפי שראינו בהרצאות הקודמות. \\*
	באופן דומה נוכל להראות את כל התנאים המתאימים.

	ההוכחה השנייה מתבססת על סעיף ג' של תרגיל 4. \\*
	הגדרה שקולה לחלוקה שראינו היא $\{ \Sigma \subseteq sent_L \mid \forall v : P \to \{ \TT, \FF \}, \forall \varphi, \psi \in \Sigma, \overline{v}(\varphi) = \overline{v}(\psi)\}$. \\*
	תוצאת הסעיף היא ש־$B_X$ היא אלגברה בוליאנית, אבל מההגדרה הזו יחד עם ההגדרה השקולה שראינו זה עתה, נובע כי הפעולות שהגדרנו על מחלוקה הזו יוצרים אלגברה בוליאנית, והאחת שאנו רוצים למצוא בדיוק.
\end{proof}

\question{}
תהי $X \subseteq \NN$ ויהיה $n_0 \in \NN$ קבוע. \\*
נגדיר $L = \{ P_{n, m, i} \mid n, m \in X, n < m, i \in \{0, 1\}\}$ שפה לתחשיב פסוקים ונגדיר גם
\[
	\Sigma_0^X = \{ (P_{n, m, 0} \leftrightarrow (\lnot P_{n, m, 1})) \mid n, m \in X, n < m \}
\]
וכן גם
\[
	\Sigma_1^{X, n_0} = \left\{\left(\lnot\left(\bigwedge_{\{n, m \in A \mid n < m\}} P_{n, m, i}\right)\right) \middle| i \in \{0, 1\}, A \subseteq X, |A| = n_0\right\}
\]
והאיחוד $\Sigma^{X, n_0} = \Sigma_0^X \cup \Sigma_1^{X, n_0}$. \\*
נוכיח שהקבוצה $\Sigma^{X, n_0}$ ספיקה אם ורק אם קיימת צביעה $c : {[X]}^2 \to \{0, 1\}$ כך שאין קבוצה מונוכרומטית $A \subseteq X$ בגודל $n_0$.
\begin{proof}
	נניח ש־$\Sigma^{X, n_0}$ ספיקה, קיימת הערכת אמת $v$ המספקת את $\Sigma_0^X$ ואת $\Sigma_1^{X, n_0}$. \\*
	נגדיר קבוצה חדשה $c = \{ \langle \{n, m\}, i \rangle \mid v(P_{n, m, i}) = \TT \}$,
	נבחין כי $c$ יחס וגם ש־$\dom c = {[X]}^2$, זאת שכן $\Sigma_0^X$ מסופקת על־ידי $v$ ולכן לכל $\{n, m\}$ מתקיים $v(P_{n, m, 0}) = \TT$ או $v(P_{n, m, 1}) = \TT$, כאשר $n < m$.
	מאותה טענה גם נובע ש־$c$ מקיימת את תנאי הפונקציה, זאת בשל הגרירה הדו־כיוונית לשלילה, לכן $c : {[X]}^2 \to \{0, 1\}$. \\*
	תהי $A \subseteq X$ כך ש־$|X| = n_0$ ויהי $i \in \{0, 1\}$, אז קיימים $n, m \in X$ כך ש־$\Sigma_1^{X, n_0} \models P_{n, m, j}$ עבור $j \ne i$, כלומר $c(\{n, m\}) \ne i$.
	זוהי כמובן ההגדרה לפונקציית צביעה המקיימת את התנאי על אי־מונוכרומטיות קבוצות מגודל $n_0$.

	בכיוון ההפוך נניח כי קיימת $c : {[X]}^2 \to \{0, 1\}$ המקיימת את התנאים ונוכיח ספיקות. \\*
	נגדיר את הערכת האמת
	\[
		v(P_{n, m, i}) = \begin{cases}
			\TT & c(\{n, m\}) = i \\
			\FF & \text{else}
		\end{cases}
	\]
	ונוכיח כי היא מספקת את $\Sigma^{X, n_0}$. \\*
	עבור $\varphi \in \Sigma_0^X$ נבחין כי $\{n, m\} \in \dom c$ וגם מתנאי הפונקציה לא יתכן שיש שתי צביעות שונות לצימוד זה, ולכן נוכל להסיק $\overline{v}(\varphi) = \TT$. \\*
	עבור $\varphi \in \Sigma_1^{X, n_0}$ נניח $A \subseteq X, |A| = n_0$ הקבוצה שמגדירה את $\varphi$, ויהי צבע $i$, אז קיים $n, m \in X$ כך ש־$c(\{n, m\}) \ne i$ ולכן $v(P_{n, m, i}) = \FF$ ונקבל $\overline{v}(\varphi) = \TT$. \\*
	מצאנו אם כן ש־$v$ מספקת את $\Sigma_0^X$ ואת $\Sigma_1^{X, n_0}$ ולכן היא מספקת את $\Sigma^{X, n_0}$.
\end{proof}

\question{}
תהי $L$ שפה לתחשיב יחסים.

\subquestion{}
נגדיר קבוצה $X$, פונקציה $f : const_L \cup Var \to X$ ולכל $n \in \NN$ נגדיר $F \in Func_{L, n}$ פונקציה $\epsilon_F : X^n \to X$,
כך שהפונקציה $\overline{f} : term_L \to X$ מתקבלת ממשפט ההגדרה ברקורסיה עבור שמות עצם, ומחזירה לכל שם עצם את קבוצת המשתנים המופיעים בו. \\*
נוכיח שהפונקציה הזו מקיימת את הדרישות שלנו.
\begin{proof}
	נוכיח באינדוקציה על שמות עצם שהפונקציה אכן מקיימת את הנדרש.

	במקרה הבסיס, $t \in const_L \cup Var$, אם $t \in const_L$ אז $f(t) = \emptyset$, ואילו $t \in Var$ אז $f(t) = \{ t \}$ והפונקציה מקיימת את הנדרש.

	יהי $t = F(t_1, \dots, t_n)$ עבור $F \in Func_{L, n}$, אז נגדיר $\epsilon_F(X_1, \dots, X_n) = \bigcup_{i \in [n]} X_i$ ולכן עבור ההנחה כי $X_i$ מייצגים את הערך הנכון נקבל את איחוד המשתנים וזהו אכן הביטוי שאנו מחפשים.
\end{proof}

\subquestion{}
נגדיר קבוצה $Y$, פונקציה $g : atom_L \cup Var \to X$ ולכל $n \in \NN$ פונקציות $\epsilon_\forall, \epsilon_\exists : Var \times Y \to Y$, פונקציה $\epsilon_\lnot : Y \to Y$,
ולכל $\square \in B$ גם $\epsilon_\square : Y^2 \to Y$ כך שהפונקציה $\overline{g} : form_L \to Y$ המתקבלת ממשפט הרקורסיה לנוסחות תתאים את מספר המשתנים המופיעים ב־$\varphi$. \\*
נוכיח כי פונקציה זו מקיימת את הנדרש.
\begin{proof}
	נגדיר את $g$ כך שלכל $E$ יחס $n$ מקומי כך ש־$\varphi = E(t_1, \dots, t_n)$ יתקיים $g(\varphi) = \bigcup_{i \in [n]} \overline{f}(t_i)$. \\*
	עוד נגדיר $\epsilon_\forall(v, Y) = \epsilon_\exists(v, Y) = Y \cup \{v\}$, וכן $\epsilon_\lnot(Y) = Y$ ו־$\epsilon_\square(Y_1, Y_2) = Y_1 \cup Y_2$. \\*
	ממשפט ההגדרה ברקורסיה על יחסים נקבל ש־$\overline{g}$ קיימת ויחידה.

	נוכיח באינדוקציה שהיא מקיימת את הטענות. \\*
	כבסיס נניח $\varphi \in atom_L$ ולכן $\varphi = E(t_1, \dots, t_n)$ עבור $E$ יחס $n$ מקומי ו־$t_i$ שמות עצם, אז מההגדרה נובע ישירות. \\*
	נניח כי הטענה נכונה עבור $\varphi$ ונבחן את $\forall v \varphi$, מתקיים $\overline{g}(\forall v \varphi) = \epsilon_\forall(v, \overline{g}(\varphi))$ ופונקציה זו מקיימת את הטענה, ההוכחה ל־$\exists$ זהה. \\*
	עבור $\varphi = \lnot \psi$ מתקיים מהנחת האינדוקציה וההגדרה $\overline{g}(\varphi) = \overline{g}(\psi)$ וזה אכן נכון. \\*
	עבור $\square \in B$ אם $\varphi = (\psi_1 \square \psi_2)$ אז מהנחת האינדוקציה וההגדרה מתקיים $\overline{g}(\varphi) = \overline{g}(\psi_1) \cup \overline{g}(\psi_2)$ והטענה עדיין תקפה.

	השלמנו את מהלך האינדוקציה ולכן הפונקציה אכן מקיימת את הרצוי.
\end{proof}

\question{}
תהי $L$ שפת תורת הקבוצות.

\subquestion{}
נבדוק כמה מבנים יש ל־$L$ שעולמם הוא $A = \{1\}$. \\*
מטעמי קריאות נגדיר את שפת תורת הקבוצות $(E)$ במקום $(\in)$.
\begin{solution}
	בשפת תורת הקבוצות יחס סדר דו־מקומי יחיד $E$, ולכן $E \subseteq A \times A \iff E \in \mathcal{P}(A \times A)$. \\*
	בהתאם אם $X_A = \{ E \mid E \subseteq A \times A \}$ אז $|X| = |\mathcal{P}(A \times A)| = 2^{|A \times A|} = 2^{{|A|}^2}$. \\*
	נותר אם כך להציב וקבל שאם העולם $A$ יש בדיוק $2$ מודלים ל־$\langle A, E \rangle$.
\end{solution}

\subquestion{}
נחשב כמה מבנים יש ל־$L$ שעולמם הוא הקבוצה $\{1, 2\}$ ועבור קבוצה $[n]$.
\begin{solution}
	כפי שראינו בסעיף הקודם מתקיים $|X_{[2]}| = 2^{4^2} = 16$ וכן $|X_{[n]}| = 2^{n^2}$.
\end{solution}


\end{document}
