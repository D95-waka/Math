\documentclass[a4paper]{article}

% packages
\usepackage{inputenc, fontspec, amsmath, amsthm, amsfonts, polyglossia, catchfile}
\usepackage[a4paper, margin=50pt, includeheadfoot]{geometry} % set page margins

% style
\AddToHook{cmd/section/before}{\clearpage}	% Add line break before section
\linespread{1.5}
\setcounter{secnumdepth}{0}		% Remove default number tags from sections
\setmainfont{Libertinus Serif}
\setsansfont{Libertinus Sans}
\setmonofont{Libertinus Mono}
\setdefaultlanguage{hebrew}
\setotherlanguage{english}

% operators
\DeclareMathOperator\cis{cis}
\DeclareMathOperator\Sp{Sp}
\DeclareMathOperator\tr{tr}
\DeclareMathOperator\im{Im}
\DeclareMathOperator\diag{diag}
\DeclareMathOperator*\lowlim{\underline{lim}}
\DeclareMathOperator*\uplim{\overline{lim}}

% commands
\renewcommand\qedsymbol{\textbf{משל}}
\newcommand{\NN}[0]{\mathbb{N}}
\newcommand{\ZZ}[0]{\mathbb{Z}}
\newcommand{\QQ}[0]{\mathbb{Q}}
\newcommand{\RR}[0]{\mathbb{R}}
\newcommand{\CC}[0]{\mathbb{C}}
\newcommand{\getenv}[2][] {
  \CatchFileEdef{\temp}{"|kpsewhich --var-value #2"}{\endlinechar=-1}
  \if\relax\detokenize{#1}\relax\temp\else\let#1\temp\fi
}
\newcommand{\explain}[2] {
	\begin{flalign*}
		 && \text{#2} && \text{#1}
	\end{flalign*}
}

% headers
\getenv[\AUTHOR]{AUTHOR}
\author{\AUTHOR}
\date\today

\title{פתרון מטלה 07 --- מבוא ללוגיקה, 80423}

\begin{document}
\maketitle
\maketitleprint{}

\question{}
יהיו $mathcal{A, B}$ מבנים ל־$L$ ויהי איזומורפיזם $f : \mathcal{A} \to \mathcal{B}$ ושם עצם $t \in term_L$. \\*
נוכיח שלכל השמה $\sigma : Var \to A$ מתקיים
\[
	f(t^\mathcal{A}(\sigma))
	= t^\mathcal{B}(f \circ \sigma)
\]
\begin{proof}
	נוכיח את הטענה באינדוקציה על שמות עצם. \\*
	נניח כי $t \in const_L$, ולכן מהגדרה של איזומורפיזם ושל השמה על קבועים
	\[
		f(t^\mathcal{A}(\sigma))
		= f(t^\mathcal{A})
		= t^\mathcal{B}
		= t^\mathcal{B}(f \circ \sigma)
	\]
	נניח ש־$t \in Var$ ולכן מאותן ההגדרות נובע
	\[
		f(t^\mathcal{A}(\sigma))
		= f(\sigma(t))
		= t^\mathcal{B}(f \circ \sigma)
	\]
	והשלמנו את בסיס האינדוקציה, נותר לבדוק את המהלך. \\*
	יהי $n \in \NN$ ויהי $F \in Func_{L, n}$ סימן פונקציה $n$־מקומי, ונניח $t_0, \dots, t_{n - 1} \in term_L$ כך שהם מקיימים את טענת האינדוקציה, אז. \\*
	נגדיר $t = F(t_0, \dots, t_{n - 1})$ ולכן מהגדרת איזומורפיזם, השמה עבור סימני פונקציה ויחד עם הנחת האינדוקציה נובע
	\[
		f(t^\mathcal{A}(\sigma))
		= f(F^\mathcal{A}(t_0^\mathcal{A}(\sigma), \dots, t_{n - 1}^\mathcal{B}(\sigma)))
		= F^\mathcal{B}(f(t_0^\mathcal{A}(\sigma)), \dots, f(t_{n - 1}^\mathcal{B}(\sigma)))
		= F^\mathcal{B}(t_0^\mathcal{B}(f \circ \sigma), \dots, t_{n - 1}^\mathcal{B}(f \circ \sigma))
		= t^\mathcal{B}(f \circ \sigma)
	\]
	והשלמנו את מהלך האינדוקציה.
\end{proof}

\question{}
נניח ש־$L$ מכילה אינסוף סימני יחס חד־מקומיים ${\{P_n\}}_{n \in \NN}$. \\*
יהי מבנה $\mathcal{A} = \langle A, I \rangle$ כך ש־$A = \{0\}$ ו־$P_n^\mathcal{A} = \emptyset$ לכל $n \in \NN$. \\*
נוכיח שלכל פסוק $\varphi \in sent_L$ כך ש־$\mathcal{A} \models \varphi$ קיים מבנה $\mathcal{B}$ ל־$L$ כך ש־$\mathcal{B} \models \varphi$ וגם $\mathcal{A} \not\cong \mathcal{B}$.
\begin{proof}
	יהי פסוק $\varphi$ כזה, ונגדיר $X_p = \{P_n \mid P_n \in \varphi\}$, קבוצת סימני היחס אשר מופיעים ב־$\varphi$ (בסימון זה התייחסנו ל־$\varphi$ כסדרה). \\*
	יהי $k \in \NN \setminus \{i \in \NN \mid P_i \in X_p\}$ כלשהו (הגדרה זו לא מצריכה בחירה). \\*
	נגדיר מבנה חדש $\mathcal{B} = \langle A, J\rangle$ כך ש־$I = J$ מלבד $P_k^\mathcal{B} = A \times A$. \\*
	נוכל להוכיח באינדוקציה על יחסים שמתקיים $\mathcal \models \varphi$, אבל עבור הפסוק $\phi = \forall x\ P_k(x)$ נקבל $\mathcal{A} \not\models \phi$ בעוד $\mathcal{B} \models \phi$ ולכן בפרט $\mathcal{A} \not\cong \mathcal{B}$.
\end{proof}

\question{}
יהיו $\mathcal{A} \subseteq \mathcal{B}$ מבנים ל־$L$ ויהי $\psi$ פסוק ללא כמתים ו־$\varphi$ פסוק כך שעבור המשתנים $x_0, \dots, x_{k - 1} \in Var$ מתקיים $\varphi = \forall x_0, \dots, \forall x_{k - 1} \psi$.

\subquestion{}
נפריך את הטענה שאם $\mathcal{A} \models \varphi$ אז $\mathcal{B} \models \varphi$.
\begin{solution}
	נגדיר $L$ שפת השוויון ו־$A = \{0\}, B = \{0, 1\}$, ונגדיר גם $\psi(x, y) = x = y$, לכן $\varphi$ מתלכדת עם $\varphi_{\le 1}$ מהמטלה הקודמת. \\*
	בהתאם נבחין כי $\mathcal{A} \models \varphi$ אבל $\mathcal{B} \not\models \varphi$, זאת שכן $\psi(0, 1)$ לא מתקיים.
\end{solution}

\subquestion{}
נוכיח שאם $\mathcal{B} \models \varphi$ אז גם $\mathcal{A} \models \varphi$.
\begin{proof}
	לכל השמה $\sigma : Var \to A$, נובע מהשיכון הנתון ומהעובדה ש־$\psi$ חסר כמתים כי
	\[
		\mathcal{A} \models \psi^\mathcal{A}(\sigma)
		\iff \mathcal{B} \models \psi^\mathcal{B}(\sigma)
		\impliedby \mathcal{B} \models \varphi
	\]
	כאשר הגדרנו את $\sigma$ מחדש כהרחבת הטווח, וכאשר הגרירה האחרונה נובעת מבדיקת הצבה ישירה והגדרת הקיום. \\*
	בהתאם מצאנו כי $\mathcal{A} \models \varphi$.
\end{proof}

\question{}
תהי $S$ מחלקה של מבנים ל־$L$.

\subquestion{}
נניח ש־$\mathcal{A}$ מבנה ל־$L$. \\*
נוכיח שמתקיים $\mathcal{A} \in Mod(Th(S))$ אם ורק אם $\forall \varphi \in Th(\mathcal{A}), \exists \mathcal{B} \in S, \mathcal{B} \models \varphi$.
\begin{proof}
	
\end{proof}

\end{document}
