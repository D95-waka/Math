\documentclass[a4paper]{article}

% packages
\usepackage{inputenc, amsmath, amsthm, thmtools, amsfonts, amssymb, luacode, catchfile, tikzducks, hyperref}
\usepackage[a4paper, margin=50pt, includeheadfoot]{geometry} % set page margins
\usepackage[shortlabels]{enumitem}
\usepackage[skip=3pt, indent=0pt]{parskip}

% language
\usepackage[bidi=basic, layout=tabular, provide=*]{babel}
\babelprovide[main, import]{hebrew}
\babelprovide{rl}
\babelfont{rm}{Libertinus Serif}
\babelfont{sf}{Libertinus Sans}
\babelfont{tt}{Libertinus Mono}

% style
\AddToHook{cmd/section/before}{\clearpage}	% Add line break before section
\linespread{1.3}
\setcounter{secnumdepth}{0}		% Remove default number tags from sections, this won't do well with theorems
\AtBeginDocument{\setlength{\belowdisplayskip}{3pt}}
\AtBeginDocument{\setlength{\abovedisplayskip}{3pt}}

% operators
\DeclareMathOperator\cis{cis}
\DeclareMathOperator\Sp{Sp}
\DeclareMathOperator\tr{tr}
\DeclareMathOperator\im{Im}
\DeclareMathOperator\re{Re}
\DeclareMathOperator\diag{diag}
\DeclareMathOperator*\lowlim{\underline{lim}}
\DeclareMathOperator*\uplim{\overline{lim}}
\DeclareMathOperator\rng{rng}
\DeclareMathOperator\Sym{Sym}
\DeclareMathOperator\Arg{Arg}
\DeclareMathOperator\Log{Log}
\DeclareMathOperator\dom{dom}

% commands
%\renewcommand\qedsymbol{\textbf{מש''ל}}
%\renewcommand\qedsymbol{\fbox{\emoji{lizard}}}
\newcommand{\NN}[0]{\mathbb{N}}
\newcommand{\ZZ}[0]{\mathbb{Z}}
\newcommand{\QQ}[0]{\mathbb{Q}}
\newcommand{\RR}[0]{\mathbb{R}}
\newcommand{\CC}[0]{\mathbb{C}}
\newcommand{\FF}[0]{\mathbb{F}}
\newcommand{\PP}[0]{\mathbb{P}}
\newcommand{\TT}[0]{\mathbb{T}}
\newcommand{\acts}[0]{\circlearrowright}
\newcommand{\explain}[2] {
	\begin{flalign*}
		 && \text{#2} && \text{#1}
	\end{flalign*}
}
\newcommand{\maketitleprint}[0]{ \begin{center}
	\begin{tikzpicture}[scale=3]
		\duck[graduate=gray!20!black, tassel=red!70!black]
	\end{tikzpicture}	
\end{center}
}

% theorem commands
\newtheoremstyle{c_remark}
	{}	% Space above
	{}	% Space below
	{}% Body font
	{}	% Indent amount
	{\bfseries}	% Theorem head font
	{}	% Punctuation after theorem head
	{.5em}	% Space after theorem head
	{\thmname{#1}\thmnumber{ #2}\thmnote{ \normalfont{\text{(#3)}}}}	% head content
\newtheoremstyle{c_definition}
	{3pt}	% Space above
	{3pt}	% Space below
	{}% Body font
	{}	% Indent amount
	{\bfseries}	% Theorem head font
	{}	% Punctuation after theorem head
	{.5em}	% Space after theorem head
	{\thmname{#1}\thmnumber{ #2}\thmnote{ \normalfont{\text{(#3)}}}}	% head content
\newtheoremstyle{c_plain}
	{3pt}	% Space above
	{3pt}	% Space below
	{\itshape}% Body font
	{}	% Indent amount
	{\bfseries}	% Theorem head font
	{}	% Punctuation after theorem head
	{.5em}	% Space after theorem head
	{\thmname{#1}\thmnumber{ #2}\thmnote{ \text{(#3)}}}	% head content

\theoremstyle{c_plain}
\newtheorem{theorem}{משפט}[section]
\newtheorem{lemma}[theorem]{למה}
\newtheorem{proposition}[theorem]{טענה}
\newtheorem*{proposition*}{טענה}
%\newtheorem{corollary}[theorem]{אין חלופה עברית}

\theoremstyle{c_definition}
\newtheorem{definition}[theorem]{הגדרה}
\newtheorem*{definition*}{הגדרה}
\newtheorem{example}{דוגמה}[section]
\newtheorem{exercise}{תרגיל}[section]

\theoremstyle{c_remark}
\newtheorem*{remark}{הערה}
\newtheorem*{solution}{פתרון}
\newtheorem{conclusion}[theorem]{מסקנה}
\newtheorem{notation}[theorem]{סימון}

% Questions related commands
\newcounter{question}
\setcounter{question}{1}
\newcounter{sub_question}
\setcounter{sub_question}{1}

\newcommand{\question}[1][0]{
	\ifthenelse{#1 = 0}{}{\setcounter{question}{#1}}
	\subsection{שאלה \arabic{question}}
	\addtocounter{question}{1}
	\setcounter{sub_question}{1}
}

\newcommand{\subquestion}[1][0]{
	\ifthenelse{#1 = 0}{}{\setcounter{sub_question}{#1}}
	\subsubsection{סעיף \localecounter{letters.gershayim}{sub_question}}
	\addtocounter{sub_question}{1}
}

% import lua and start of document
\directlua{common = require ('../common')}

\GetEnv{AUTHOR}

% headers
\author{\AUTHOR}
\date\today

\title{פתרון מטלה 01 --- מבוא ללוגיקה, 80423}

\begin{document}
\maketitle
\maketitleprint{}

\Question{}
יהי $T = (V, E)$ עץ כאשר $V$ סופית.

\Subquestion{}
נוכיח כי קיים מסלול ללא חזרות בין $v, u \in E$.
\begin{proof}
	יהי ${(v_n)}_{n = 1}^l$ מסלול סופי שמובטח שקיים מקשירות העץ כך ש־$v_1 = v, v_l = u$. \\*
	אילו אין חזרות סיימנו, לכן נניח שישנן חזרות, נבחר $0 \le i < j \le l$ כך ש־$v_i = v_j$.
	נבנה סדרה חדשה ${(v_n')}_{n = 1}^m$ על־ידי $v_k = v_k'$ לכל $0 \le k \le i$ ו־$v_k' = v_{k + j - i}$,
	זהו מסלול חדש בו אין את החזרה על $v_i$, והיא חוקית שכן ידוע כי $(v_{i - 1}, v_i), (v_j, v_{j + 1}) \in E$.
	נחזור על תהליך זה על $(v_n')$ שוב ושוב עד שנקבל מסלול ללא חזרות.
	נבחין כי אכן נקבל מסלול כזה, שכן כל מסלול הוא סופי, ולכן כמות החזרות אף היא סופית, וכמות החזרות לאחר התהליך קטנה ממש מכמות החזרות המקורית.
\end{proof}

\Subquestion{}
נוכיח כי אם $(v, u) \in E$ אז המסלול היחיד ללא חזרות ביניהם הוא $\langle v, w \rangle$.
\begin{proof}
	ידוע לנו כי קיים מסלול ללא חזרות בין שני הקודקודים, וברור כי $\langle v, u \rangle$ הוא מסלול כזה, עתה נוכיח את יחידותו. \\*
	יהי $\langle v_1, \dots, v_l \rangle$ מסלול נוסף ללא חזרות כך ש־$v_1 = v, v_l = u$.
	אילו $v_2 = u$ אז נקבל את המסלול שהגדרנו קודם, ולכן נניח כי $v_2 \ne u$.
	נבחין עתה כי ידוע שבמסלול אין מעגלים, לכן לא קיים $2 < i \le l$ כך ש־$v_i = v$, אך בעץ אין מעגלים ולכן נקבל ש־$G' = (V \setminus \{ v \}, E \setminus \{ v \} \times V)$ הוא גרף בעל שני רכיבי קשירות.
	יהי $G_1'$ רכיב הקשירות אשר מכיל את $v_2$, נוכל להסיק כי $u \notin G_1'$ אחרת נקבל סתירה להנחה ש־$v_2 \ne u$, ולכן נוכל לקבוע כי $\forall i, 0 \le i \le l \implies v_i \ne u$ וזו סתירה להגדרת מסלול זה.
	נסיק אם כן שקיים מסלול ללא חזרות יחיד והוא זה אשר ציינו.

	הוכחה נוספת והרבה יותר פשוטה:
	ידוע כי $\langle v, u \rangle$ מסלול ללא חזרות, נניח כי $\langle v = v_1, \dots, v_l = u \rangle$ מסלול נוסף כזה, אז $\langle v = v_1, \dots, v_l = u, v \rangle$ הוא מסלול ומעגל פשוט וזו סתירה להגדרת העץ,
	לכן מצאנו כי המסלול היחיד הוא אכן $\langle v, u \rangle$.
\end{proof}

\Subquestion{}
יהי מסלול ללא חזרות $\alpha = \langle v = v_0, \dots, v_l = u \rangle$ ונניח באינדוקציה תוך שימוש בסעיף הקודם כבסיס שלכל $2 \le k' < k$ יש מסלול ללא חזרות יחיד, ונוכיח כי גם המסלול הנתון הוא יחיד.
\begin{proof}
	נניח בשלילה כי קיים מסלול נוסף $\beta = \langle v = u_0, \dots, u_k = u \rangle$. \\*
	אילו קיים $0 < i < k$ כך ש־$v_i \in \beta$ אז מהנחת האינדוקציה נקבל כי קיים מסלול יחיד בין $v$ ל־$v_i$ ובין $v_i$ לבין $u$ ולכן $\alpha = \beta$ וקיבלנו סתירה לקיום מסלול נוסף. \\*
	נניח אם כן שאין $i$ כזה, דהינו המסלולים זרים מלבד בקצוות.
	נבנה מסלול חדש $\langle v = v_0, \dots, v_l = u = u_m, u_{m - 1}, \dots, u_0 = v \rangle$, נבחין כי במסלול זה אין חזרות מלבד בקצוות, והוא מתחיל ונגמר ב־$v$, דהינו מסלול זה הוא מעגל פשוט. \\*
	ידוע כי ב־$T$ אין מעגלים מהגדרתו כעץ ולכן קיבלנו סתירה, ואין מסלול נוסף המקיים את התנאים.
\end{proof}

\Subquestion{}
נוכיח כי $\le_T$ הוא יחס סדר חלקי.
\begin{proof}
	נגדיר $e \in V$ כגזע העץ, ונגדיר את $\le_T$ ביחס אליו, נוכיח כי הוא אכן סדר חלקי:
	\begin{itemize}
		\item רפלקסיביות: אם $\langle e = v_0, \dots, v_l = v \rangle$ מסלול, אז $v \in \langle v_i \rangle$ ולכן $v \le_T v$.
		\item אנטי־סימטריה: נניח כי $v \le_T u$ וגם $u \le_T v$,
			לכן קיים מסלול $\langle e = v_0, \dots, v_l = v \rangle$ וקיים מסלול $\langle e = u_0, \dots, v_m = u \rangle$ כך ש־$v_i = u, u_j = v$ עבור $0 \le i \le l, 0 \le j \le m$.
			אבל נקבל שכל מסלול מוכל בשני מהטענה שהוכחנו בסעיפים הקודמים, ולכן בפרט $i = j$ ו־$u = v$.
		\item טרנזיטיביות: נניח $u \le_T v, v \le_T w$, ולכן קיים מסלול בין $e$ לבין $w$ כך ש־$v$ מוכל בו, אבל יש מסלול יחיד גם בין $e$ ל־$v$ וידוע כי $u$ מוכל בו, לכן בפרט $u$ מוכל גם במסלול של $e$ ל־$w$ ומתקיים $u \le_T w$.
	\end{itemize}
\end{proof}

\Question{}


\end{document}
