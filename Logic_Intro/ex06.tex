\documentclass[a4paper]{article}

% packages
\usepackage{inputenc, fontspec, amsmath, amsthm, amsfonts, polyglossia, catchfile}
\usepackage[a4paper, margin=50pt, includeheadfoot]{geometry} % set page margins

% style
\AddToHook{cmd/section/before}{\clearpage}	% Add line break before section
\linespread{1.5}
\setcounter{secnumdepth}{0}		% Remove default number tags from sections
\setmainfont{Libertinus Serif}
\setsansfont{Libertinus Sans}
\setmonofont{Libertinus Mono}
\setdefaultlanguage{hebrew}
\setotherlanguage{english}

% operators
\DeclareMathOperator\cis{cis}
\DeclareMathOperator\Sp{Sp}
\DeclareMathOperator\tr{tr}
\DeclareMathOperator\im{Im}
\DeclareMathOperator\diag{diag}
\DeclareMathOperator*\lowlim{\underline{lim}}
\DeclareMathOperator*\uplim{\overline{lim}}

% commands
\renewcommand\qedsymbol{\textbf{משל}}
\newcommand{\NN}[0]{\mathbb{N}}
\newcommand{\ZZ}[0]{\mathbb{Z}}
\newcommand{\QQ}[0]{\mathbb{Q}}
\newcommand{\RR}[0]{\mathbb{R}}
\newcommand{\CC}[0]{\mathbb{C}}
\newcommand{\getenv}[2][] {
  \CatchFileEdef{\temp}{"|kpsewhich --var-value #2"}{\endlinechar=-1}
  \if\relax\detokenize{#1}\relax\temp\else\let#1\temp\fi
}
\newcommand{\explain}[2] {
	\begin{flalign*}
		 && \text{#2} && \text{#1}
	\end{flalign*}
}

% headers
\getenv[\AUTHOR]{AUTHOR}
\author{\AUTHOR}
\date\today

\title{פתרון מטלה 06 --- מבוא ללוגיקה, 80423}

\begin{document}
\maketitle
\maketitleprint{}

\question{}
תהי $L$ שפה סופית לתחשיב פסוקים, ונסמן $\mathcal{F} = {\{\TT, \FF\}}^L$. \\*
נגדיר $B : sent_L / \equiv_{tau} \to \mathcal{F}$ על־ידי $B({[\varphi]}_{\equiv_{tau}}) = B_\varphi$.

\subquestion{}
נוכיח ש־$B$ מוגדרת היטב.
\begin{proof}
	החל מעתה נסמן $\equiv$ כ־$\equiv_{tau}$ מטעמי קריאות. \\*
	יהיו $\varphi, \psi \in {[\varphi]}_\equiv$ שני פסוקים. נראה כי
	\[
		B_\varphi = B_\psi
		\iff \forall v \in \mathcal{F}, B_\varphi(v) = B_\psi(v)
		\iff \forall v \in \mathcal{F}, \overline{v}(\varphi) = \overline{v}(\psi)
		\iff \varphi \equiv \psi
	\]
	אבל מהגדרתם $\varphi \equiv \psi$ ולכן $B_\varphi = B_\psi$ וכן הפונקציה $B$ לא תלויה בבחירת נציג, קרי היא מוגדרת היטב.
\end{proof}

\subquestion{}
נוכיח ש־$B$ חד־חד ערכית ועל.
\begin{proof}
	נניח ${[\varphi]}_\equiv \ne {[\psi]}_\equiv$ ולכן קיים $v \in \mathcal{F}$ כך שמתקיים
	\[
		\overline{v}(\varphi) \ne \overline{v}(\psi)
		\implies B_\varphi \ne B_\psi
		\iff B({[\varphi]}_\equiv) \ne B({[\psi]}_\equiv)
	\]
	ומצאנו חד־חד ערכיות. נעבור להוכחת על.
	תהי $f \in \mathcal{F}$, אז מטענה מההרצאה קיים $\varphi$ כך ש־$f = B_\varphi$, ולכן ${[\varphi]}_\equiv \in \dom B$, וגם $B(f) = B_\varphi$ ומצאנו ש־$B$ על.
\end{proof}

\subquestion{}
נוכיח שלכל $\varphi, \phi \in sent_L$ מתקיים
\[
	B({[\varphi]}_\equiv \tilde{\cdot} {[\phi]}_\equiv) = B({[\varphi]}_\equiv) \cdot' B({[\phi]}_\equiv),
	\qquad
	B({[\varphi]}_\equiv \tilde{+} {[\phi]}_\equiv) = B({[\varphi]}_\equiv) +' B({[\phi]}_\equiv),
	\qquad
	B(\tilde{-} {[\varphi]}_\equiv) = -' B({[\varphi]}_\equiv)
\]
\begin{proof}
	יהי $v \in \mathcal{F}$, אז
	\begin{align*}
		B({[\varphi]}_\equiv \tilde{\cdot} {[\phi]}_\equiv)(\varphi)
		& = B({[\varphi \cdot \phi]}_\equiv)(\varphi) \\
		& = B_{\varphi \cap \phi}(v) \\
		& = \overline{v}(\varphi \cap \phi) \\
		& = V_\land(\overline{v}(\varphi), \overline{v}(\phi)) \\
		& = \overline{v}(\varphi) \cdot \overline{v}(\phi) \\
		& = B_\varphi(v) \cdot B_\phi(v) \\
		& = (B_\varphi \cdot' B_\phi)(v) \\
		& = (B({[\varphi]}_\equiv) \cdot' B({[\phi]}_\equiv))(v)
	\end{align*}
	ולכן $B({[\varphi]}_\equiv \tilde{\cdot} {[\phi]}_\equiv) = B({[\varphi]}_\equiv) \cdot' B({[\phi]}_\equiv)$. \\*
	המהלך עבור $B({[\varphi]}_\equiv \tilde{+} {[\phi]}_\equiv) = B({[\varphi]}_\equiv) +' B({[\phi]}_\equiv)$ זהה. \\*
	נבחן את השלילה:
	\[
		B(\tilde{-} {[\varphi]}_\equiv)(v)
		= B({[- \varphi]}_\equiv)(v)
		= B_{(\lnot \varphi)}(v)
		= V_\lnot(\overline{v}(\varphi))
		= - B_\varphi(v)
		= (-' B({[\varphi]}_\equiv))(v)
	\]
	ולכן $B(\tilde{-} {[\varphi]}_\equiv) = -' B({[\varphi]}_\equiv)$.
\end{proof}

\question{}
תהי $L$ שפה לתחשיב יחסים.

\subquestion{}
יהי $\mathcal{A} = (A; I)$ מבנה ל־$L$.
תהי $\sigma : Var \to A$ השמה ונבנה ברקורסיה פונקציה כך ש־$\bar{f}_\sigma(t) = t^\mathcal{A}(\sigma)$.
\begin{proof}
	נגדיר $f_\sigma : Var \cup const_L \to A$ על־ידי $f_\sigma(v) = \sigma(v)$ עבור $v \in Var$ ו־$f_\sigma(c) = c^\mathcal{A}$ עבור $c \in const_L$. \\*
	לכל $n \in \NN$ ולכל $F \in Func_{L, n}$ נגדיר $\epsilon_F : A^n \to A$ על־ידי $\epsilon_F(a_0, \dots, a_{n - 1}) = F^\mathcal{A}(a_0, \dots, a_{n - 1})$. \\*
	ממשפט ההגדרה ברקורסיה נובע שקיימת ויחידה $\bar{f}_\sigma : term_L \to A$ המקיימת לפי ההגדרה $\bar{f}_\sigma(t) = t^\mathcal{A}(\sigma)$, ניתן כמובן להוכיח טענה זו באינדוקציה, אך לא התבקשנו לעשות כן.
\end{proof}

\subquestion{}
נבנה פונקציה $form_L \to Y$ המחזירה עבור נוסחה את קבוצת המשתנים החופשיים בה.
\begin{solution}
	נגדיר $Y = \mathcal{P}(Var)$, וכן $f : atom_L \to Y$ כך שלכל $\varphi \in atom_L, f(\varphi) = g(\varphi)$, כאשר $g$ הפונקציה המתקבלת משאלה 3 ב' במטלה 5. \\*
	עוד נגדיר $\epsilon_\lnot : Y \to Y$ על־ידי $\epsilon_\lnot(x) = x$ וכמו־כן לכל $\square \in \mathcal{B}$ גם $\epsilon_\square : Y^2 \to Y$ על־ידי $\epsilon_\square(x, y) = x \cup y$. \\*
	נגדיר גם $\epsilon_\exists, \epsilon_\forall : Var \times Y$ על־ידי $\epsilon_\forall(v, x) = \epsilon_\exists(v, x) = x \setminus \{v\}$. \\*
	כל זאת בנינו בהתאם להגדרה של משתנים חופשיים, ולכן גם $\overline{f}$ המתקיימת מהפעלת משפט הרקורסיה ליחסים קיימת ומתאימה קבוצות משתנים חופשיים לנוסחות.
\end{solution}

\question{}
תהי $L_=$ שפת השוויון ויהי $n \in \NN$.

\subquestion{}
נוכיח שקיים $\varphi_{\ge n} \in sent_L$ כך שלכל $\mathcal{A} = (A; I)$ מבנה ל־$L_=$ מתקיים $\mathcal{A} \models \varphi_{\ge n}$ אם ורק אם $|A| \ge n$.
\begin{proof}
	נסמן $(\lnot =(x, y))$ על־ידי $x \ne y$, נבחין כי זהו סימון בלבד.
	נגדיר נוסחה
	\[
		\varphi_{\ge n} = \exists x_0 (\exists x_1 (x_1 \ne x_0 \land \exists x_2 (x_2 \ne x_1 \land x_2 \ne x_0 \land \dots \exists x_{n - 1} (x_{n - 1} \ne x_0 \land \dots \land x_{n - 1} \ne x_{n - 2}))))
	\]
	נניח ש־$\mathcal{A} \models \varphi_{\ge n}$ ולכן ישירות מהגדרת הנוסחה נקבל שקיימים $x_0, \dots, x_{n - 1} \in A$ שונים בזוגות, ולכן $|A| \ge |\{x_0, \dots, x_{n - 1}\}| = n$. \\*
	נניח ש־$|A| \ge n$ ולכן קיימת תת־קבוצה $X \subseteq A$ כך ש־$|X| = n$ (ואין צורך באקסיומת הבחירה מסופיות $X$) כך שכל האיברים שונים בזוגות,
	דהינו $\varphi_{\ge n}$ מתקיים ב־$\mathcal{A}$ ולכן $\mathcal{A} \models \varphi_{\ge n}$.
\end{proof}

\subquestion{}
נסיק כי יש נוסחה $\varphi_{=n}$ כך שלכל $\mathcal{A} = (A; I)$ מבנה ל־$L_=$ מתקיים $\mathcal{A} \models \varphi_{= n}$ אם ורק אם $|A| = n$.
\begin{proof}
	נגדיר
	\[
		\varphi_{\le n} = \exists x_0 \exists x_1 \dots \exists x_{n - 1} (\forall x (x = x_0 \lor \dots \dots x = x_{n - 1}))
	\]
	נוסחה שמתארת שקיימים לכל היותר $n$ איברים בעולם, נבחין כי בנוסחה אין התייחסות למקרה שבו $x_i = x_j$ עבור $i \ne j$, נוכל כמובן לתקן נוסחה זו באופן דומה לבניית הנוסחה $\varphi_{\ge n}$, אך אין בכך צורך. \\*
	ניתן להוכיח באופן דומה לסעיף א' כי מתקיים $\mathcal{A} \models \varphi_{\le n} \iff |A| \le n$. \\*
	נבחן אם כן את $\varphi_{=n} \overset{def}{=} (\varphi_{\le n} \land \varphi_{\ge n})$ ונקבל מהטענות שמצאנו ומזהויות שמתקיים $\mathcal{A} \models \varphi_{= n} \iff |A| = n$ כמבוקש, זאת שכן
	\[
		A \models \varphi_{= n} \iff A \models \varphi_{\ge n}, \varphi_{\le n} \iff n \le |A| \le n \iff |A| = n
	\]
\end{proof}

\subquestion{}
נסיק שאם $L$ שפה כלשהי לתחשיב יחסים ו־$\mathcal{A, B}$ מבנים סופיים שלה כך ש־$|A| \ne |B|$ אז $\mathcal{A, B}$ אינם שקולים אלמנטרית.
\begin{proof}
	נניח $|A| = n \ne m = |B|$ בהתאם לנתון. \\*
	לכן גם $\mathcal{A} \models \varphi_{= n}$ וגם $\mathcal{B} \models \varphi_{= m}$.
	אבל מאותו הנתון גם $\mathcal{A} \not\models \varphi_{= m}$ ו־$\mathcal{B} \not\models \varphi_{= n}$ ולכן $\mathcal{A}$ לא שקול אלמנטרית ל־$\mathcal{B}$.
\end{proof}

\end{document}
