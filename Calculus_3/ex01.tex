\documentclass[a4paper]{article}

% packages
\usepackage{inputenc, amsmath, amsthm, thmtools, amsfonts, amssymb, luacode, catchfile, tikzducks, hyperref}
\usepackage[a4paper, margin=50pt, includeheadfoot]{geometry} % set page margins
\usepackage[shortlabels]{enumitem}
\usepackage[skip=3pt, indent=0pt]{parskip}

% language
\usepackage[bidi=basic, layout=tabular, provide=*]{babel}
\babelprovide[main, import]{hebrew}
\babelprovide{rl}
\babelfont{rm}{Libertinus Serif}
\babelfont{sf}{Libertinus Sans}
\babelfont{tt}{Libertinus Mono}

% style
\AddToHook{cmd/section/before}{\clearpage}	% Add line break before section
\linespread{1.3}
\setcounter{secnumdepth}{0}		% Remove default number tags from sections, this won't do well with theorems
\AtBeginDocument{\setlength{\belowdisplayskip}{3pt}}
\AtBeginDocument{\setlength{\abovedisplayskip}{3pt}}

% operators
\DeclareMathOperator\cis{cis}
\DeclareMathOperator\Sp{Sp}
\DeclareMathOperator\tr{tr}
\DeclareMathOperator\im{Im}
\DeclareMathOperator\re{Re}
\DeclareMathOperator\diag{diag}
\DeclareMathOperator*\lowlim{\underline{lim}}
\DeclareMathOperator*\uplim{\overline{lim}}
\DeclareMathOperator\rng{rng}
\DeclareMathOperator\Sym{Sym}
\DeclareMathOperator\Arg{Arg}
\DeclareMathOperator\Log{Log}
\DeclareMathOperator\dom{dom}

% commands
%\renewcommand\qedsymbol{\textbf{מש''ל}}
%\renewcommand\qedsymbol{\fbox{\emoji{lizard}}}
\newcommand{\NN}[0]{\mathbb{N}}
\newcommand{\ZZ}[0]{\mathbb{Z}}
\newcommand{\QQ}[0]{\mathbb{Q}}
\newcommand{\RR}[0]{\mathbb{R}}
\newcommand{\CC}[0]{\mathbb{C}}
\newcommand{\FF}[0]{\mathbb{F}}
\newcommand{\PP}[0]{\mathbb{P}}
\newcommand{\TT}[0]{\mathbb{T}}
\newcommand{\acts}[0]{\circlearrowright}
\newcommand{\explain}[2] {
	\begin{flalign*}
		 && \text{#2} && \text{#1}
	\end{flalign*}
}
\newcommand{\maketitleprint}[0]{ \begin{center}
	\begin{tikzpicture}[scale=3]
		\duck[graduate=gray!20!black, tassel=red!70!black]
	\end{tikzpicture}	
\end{center}
}

% theorem commands
\newtheoremstyle{c_remark}
	{}	% Space above
	{}	% Space below
	{}% Body font
	{}	% Indent amount
	{\bfseries}	% Theorem head font
	{}	% Punctuation after theorem head
	{.5em}	% Space after theorem head
	{\thmname{#1}\thmnumber{ #2}\thmnote{ \normalfont{\text{(#3)}}}}	% head content
\newtheoremstyle{c_definition}
	{3pt}	% Space above
	{3pt}	% Space below
	{}% Body font
	{}	% Indent amount
	{\bfseries}	% Theorem head font
	{}	% Punctuation after theorem head
	{.5em}	% Space after theorem head
	{\thmname{#1}\thmnumber{ #2}\thmnote{ \normalfont{\text{(#3)}}}}	% head content
\newtheoremstyle{c_plain}
	{3pt}	% Space above
	{3pt}	% Space below
	{\itshape}% Body font
	{}	% Indent amount
	{\bfseries}	% Theorem head font
	{}	% Punctuation after theorem head
	{.5em}	% Space after theorem head
	{\thmname{#1}\thmnumber{ #2}\thmnote{ \text{(#3)}}}	% head content

\theoremstyle{c_plain}
\newtheorem{theorem}{משפט}[section]
\newtheorem{lemma}[theorem]{למה}
\newtheorem{proposition}[theorem]{טענה}
\newtheorem*{proposition*}{טענה}
%\newtheorem{corollary}[theorem]{אין חלופה עברית}

\theoremstyle{c_definition}
\newtheorem{definition}[theorem]{הגדרה}
\newtheorem*{definition*}{הגדרה}
\newtheorem{example}{דוגמה}[section]
\newtheorem{exercise}{תרגיל}[section]

\theoremstyle{c_remark}
\newtheorem*{remark}{הערה}
\newtheorem*{solution}{פתרון}
\newtheorem{conclusion}[theorem]{מסקנה}
\newtheorem{notation}[theorem]{סימון}

% Questions related commands
\newcounter{question}
\setcounter{question}{1}
\newcounter{sub_question}
\setcounter{sub_question}{1}

\newcommand{\question}[1][0]{
	\ifthenelse{#1 = 0}{}{\setcounter{question}{#1}}
	\subsection{שאלה \arabic{question}}
	\addtocounter{question}{1}
	\setcounter{sub_question}{1}
}

\newcommand{\subquestion}[1][0]{
	\ifthenelse{#1 = 0}{}{\setcounter{sub_question}{#1}}
	\subsubsection{סעיף \localecounter{letters.gershayim}{sub_question}}
	\addtocounter{sub_question}{1}
}

% import lua and start of document
\directlua{common = require ('../common')}

\GetEnv{AUTHOR}

% headers
\author{\AUTHOR}
\date\today

\title{פתרון מטלה 01 --- חשבון אינפינטסמלי 3 (80415)}

\begin{document}
\maketitle
\maketitleprint{}

\Question{}
\Subquestion{}
נקבע לכל ביטוי האם הוא מגדיר נורמה מעל $\RR^2$:
\begin{enumerate}
	\item $\lVert (x, y) \rVert = 2|x| + 3|y|$: לא משמרת כפל בסקלר: $2|\lambda x| + 3|\lambda y| \ne \lambda(2|x| + 3|y|)$, ולכן לא מהווה נורמה.
	\item $\lVert (x, y) \rVert = x^2 + y^2$ לא משמרת כפל בסקלר באותו אופן ולכן איננה נורמה.
	\item $\lVert (x, y) \rVert = {(\sqrt{|x|} + \sqrt{|y|})}^2$ זוהי אכן נורמה, נראה:
		\begin{itemize}
			\item הפעולה חיובית לכל ערך בשל חיוביות הריבוע, ומתאפסת רק כאשר $x = y = 0$.
			\item $\lVert \lambda(x, y) \rVert = {(\sqrt{|\lambda x|} + \sqrt{|\lambda y|})}^2 = {\sqrt{|\lambda|}}^2{(\sqrt{|x|} + \sqrt{|y|})}^2 = |\lambda| \lVert (x, y) \rVert$.
			\item $\lVert (x_1, y_1) + (x_2, y_2) \rVert = {(\sqrt{|x_1 + x_2|} + \sqrt{|y_1 + y_2|})}^2 = |x_1 + x_2| + |y_1 + y_2| + 2\sqrt{|x_1 + x_2|} \sqrt{|y_1 + y_2|}$ \\*
				$\le |x_1| + |x_2| + |y_1| + |y_2| + 2\sqrt{|x_1 + x_2|} \sqrt{|y_1 + y_2|} = \lVert (x_1, y_1)\rVert + \lVert (x_2, y_2)\rVert + 2\sqrt{|x_1 + x_2|} \sqrt{|y_1 + y_2|} - 2 \sqrt{|x_1y_1|} - 2 \sqrt{|x_2y_2|}
				 \le \lVert (x_1, y_1)\rVert + \lVert (x_2, y_2)\rVert$
		\end{itemize}
	\item $\lVert (x, y) \rVert = |x| + |y| + \sqrt{x^2 + y^2}$ היא אכן נורמה, ניתן להסיק זאת ישירות מהרכבה כחיבור של שתי פונקציות נורמה.
\end{enumerate}

\Subquestion{}
תהי $A$ קבוצה ויהי $X$ אוסף תתי־הקבוצות הסופיות של $A$. \\*
לכל $x, y \in X$ נסמן $\rho(x, y) = |x \triangle y|$. נוכיח ש־$\rho$ מטריקה על $X$.
\begin{proof}
	נוכיח שתכונות המטריקה מתקיימות:
	\begin{enumerate}
		\item $x = y \iff \rho(x, y) = 0$ נובע ישירות מהגדרת ההפרש הסימטרי.
		\item סימטריה נובעת מסימטריית ההפרש הסימטרי.
		\item אי־שוויון המשולש: $x \triangle z = (x \cup z) \setminus (x \cap z)
		\subseteq (x \cup z \cup y) \setminus (x \cap y \cap z)
		= (x \cup y) \setminus (x \cap y \cap z) \cup (z \cup y) \setminus (x \cap y \cap z)
		\subseteq (x \cup y) \setminus (x \cap y) \cup (z \cup y) \setminus (y \cap z)
		= x \triangle y \cup y \triangle z$
		ולכן נוכל להסיק את אי־שוויון המשולש.
	\end{enumerate}
	מצאנו כי כלל התכונות למטריקה חלות על הפונקציה $\rho$ ולכן היא מהווה מטריקה על קבוצה זו.
\end{proof}

\Question{}
יהיו $d \in \NN, 1 \le p \le \infty$, נוכיח כי קיים $C > 0$ כך שלכל $x \in \RR^d$ מתקיים
\[
	\lVert x \rVert_\infty \le \lVert x \rVert_p \le C\lVert x \rVert_\infty
\]
\begin{proof}
	נגדיר $x_m = \max_{i \in d} |x_i|$ ונשים לב כי
	\[
		{|x_m|}^p
		\le \sum_{i = 1}^{d} {|x_i|}^p
		\le \sum_{i = 1}^{d} {|x_m|}^p
		= d {|x_m|}^p
	\]
	דהינו
	\[
		\lVert x \rVert_\infty^p \le \lVert x \rVert_p^p \le d \lVert x \rVert_\infty^p
	\]
	ולכן גם נובע
	\[
		\lVert x \rVert_\infty \le \lVert x \rVert_p \le \sqrt[p]{d} \lVert x \rVert_\infty
	\]
	ומצאנו מספר $C = \sqrt[p]{d}$ אשר מקיים את אי־השוויון.
\end{proof}

\Question{}
יהיו ${\{(X_i, \rho_i)\}}_{i = 1}^d$ מרחבים מטריים ותהיה $f : \RR_+^d \to \RR$ פונקציה חיובית, מקיימת את אי־שוויון המשולש ומונוטונית לכל רכיב. \\*
נסמן $X = X_1 \times \cdots \times X_d$ ונגדיר $\rho_f : X \to \RR$ על־ידי
\[
	\rho_f(\{x_1, \hdots, x_d\}, \{y_1, \hdots, y_d\}) = f(\rho_1(x_1, y_1, \hdots, \rho_d(x_d, y_d)))
\]

\Subquestion{}
נוכיח כי $\rho_f$ היא מטריקה על $X$.
\begin{proof}
	נבדוק את תכונות המטריקה:
	\begin{enumerate}
		\item חיוביות: נתון לנו כי לכל רכיב הוא מתאפס אם ורק אם $x_i = y_i$, ונתונה לנו חיוביות ב־$f$, לכן $\rho_f$ יתאפס אם ורק אם $x = y = 0$.
		\item סימטריה: נראה כי
			\[
				\rho_f(x, y)
				= f(\rho_1(x_1, y_1, \hdots, \rho_d(x_d, y_d)))
				= f(\rho_1(y_1, x_1, \hdots, \rho_d(y_d, x_d)))
				= \rho_f(y, x)
			\]
		\item אי שוויון המשולש: נתון כי $f$ מקיימת את אי־שוויון המשולש וכלל הרכיבים הם מטריקות ולכן מקיימים את אי־שוויון המשולש ולכן נובע.
	\end{enumerate}
\end{proof}

\Subquestion{}
נראה כי הפונקציה $f(x) = x_1 + \hdots + x_d$ מקיימת את תנאי השאלה. \\*
חיוביות: נתון $x_i \ge 0$ לכל $i \in [d]$ ולכן גם חיבור כלל הרכיבים הוא חיובי. \\*
תת־חיבוריות: $f(x + y) = \sum_{i = 1}^{d} x_i + y_i = f(x) + f(y)$. \\*
מונוטוניות בכל רכיב נובעת מהמעבר האחרון.

\Subquestion{}
נראה שעבור $f$ הנתונה התכנסות ב־$\rho_f$ שקולה להתכנסות איבר־איבר.
\begin{proof}
	\begin{align*}
		& \lim_{n \to \infty} x_n = y \\
		& \iff \forall \epsilon > 0 \exists N > 0 \forall n > N, x \in X \implies \rho_f(x, x_n) < \epsilon \\
		& \iff \forall \epsilon > 0 \exists N > 0 \forall n > N, x \in X \implies \sum_{i = 1}^{d} x_i < \epsilon \\
		& \iff \forall \epsilon > 0 \exists N > 0 \forall n > N, x \in X, i \in [d] \implies x_i < \frac{\epsilon}{i} \\
		& \iff \forall i \in [d] \forall \epsilon > 0 \exists N > 0 \forall n > N, x \in X \implies x_i < \epsilon \\
		& \iff \forall i \in [d] \lim_{n \to \infty} x_i = y_i, \sum_{i = 1}^{d} y_i = d
	\end{align*}
	ומצאנו כי ההגדרות שקולות מחיוביות החיבור.
\end{proof}

\Question{}
נבדוק לכל אחת מהקבוצות הבאות אם היא פתוחה ואם היא סגורה:

\Subquestion{}
נבחן את $C^1([0, 1])$ כתת־קבוצה של $C([0, 1])$. \\*
נראה כי הקבוצה איננה פתוחה, שכן לכל $f \in C^1([0, 1])$ נוכל לבנות פונקציה חדשה בהינתן $h \in (0, 1)$ על־ידי
\[
	f^*(x) = \begin{cases}
		f(x), & 0 \le x \le h \\
		f(h) - f(x), & h < x \le 1
	\end{cases}
\]
פונקציה זו רציפה ונראה כי $\lim_{h \to 1} f^*(x) = f(x)$ ולכן גם $\lim_{h \to 1} \rho(f, f^*) = 0$. \\*
לעומת זאת הפונקציה $f^* \not\in C^1([0, 1])$ ו־$f^* \in C([0, 1])$ כל עוד $f$ איננה פונקציה קבועה, ולכל כדור סביב $f$ נוכל לבחור $h$ כך ש־$f^*$ תהיה בכדור. \\*
נסיק שהקבוצה איננה פתוחה. \\*
נניח בשלילה שהקבוצה הנתונה סגורה, לכן כל סדרה ${(f_n)}_{n = 1}^\infty \in C^1([0, 1])$ מקיימת $\lim_{n \to \infty} f_n \in C^1([0, 1])$. \\*
נגדיר
\[
	f_n(x) = \sqrt{{(x - \frac{1}{2})}^2 + \frac{1}{n}}
\]
פונקציה זו היא רציפה וגזירה בתחום $[0, 1]$ ולכן מקיימת את התנאים אבל
\[
	\lim_{n \to \infty} f_n = \sqrt{{(x - \frac{1}{2})}^2 + 0} = |x - \frac{1}{2}| = f
\]
וידוע כי $f \not\in C^1([0, 1])$, בסתירה להנחה, לכן הקבוצה איננה סגורה. \\*
מצאנו כי הקבוצה הנתונה איננה פתוחה ואיננה סגורה.

\Subquestion{}
נבדוק את הקבוצה $A = \{ (x, y, z) \in \RR^3 \mid x + y + z \le 1\}$ כתת־קבוצה של $\RR^3$. \\*
נבחר $a = (1, 0, 0)$ ונראה שלכל $r > 0$ מתקיים $B(a, r) = \{(x, y, z) \mid {(x - 1)}^2 + y^2 + z^2 < r^2 \} \not\subseteq A$. \\*
אם נבחר $x = 1, y = 0, z = r / 2$ אז כמובן הנקודה המוגדרת מוכלת בכדור, אבל $(x, y, z) \not\in A$. \\*
לכן הקבוצה איננה פתוחה. \\*
נבדוק אם הקבוצה סגורה. נגדיר ${(a_n)}_{n = 1}^\infty \in A$ סדרת נקודות מתכנסת ל־$a$. \\*
משאלה 3 נובע כי הגבול מתכנס אם ורק אם כל סדרת נקודות $(a_n^i)$ כאשר $1 \le i \le 3$ מתכנסת אף היא, ולכן $\lim_{n \to \infty} a_n^1 + a_n^2 + a_n^3 = a_1 + a_2 + a_3$. \\*
ואנו יודעים כי לכל $n$ מתקיים $a_n^1 + a_n^2 + a_n^3 \le 1$ ולכן ממשפט הפרוסה נקבל גם $a_1 + a_2 + a_3 \le 1$ ומצאנו כי $A$ מכילה את כל הנקודות הגבוליות שלה. \\*
לסיכום הקבוצה $A$ היא סגורה ואיננה פתוחה.

\end{document}
