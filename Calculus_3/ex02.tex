\documentclass[a4paper]{article}

% packages
\usepackage{inputenc, amsmath, amsthm, thmtools, amsfonts, amssymb, luacode, catchfile, tikzducks, hyperref}
\usepackage[a4paper, margin=50pt, includeheadfoot]{geometry} % set page margins
\usepackage[shortlabels]{enumitem}
\usepackage[skip=3pt, indent=0pt]{parskip}

% language
\usepackage[bidi=basic, layout=tabular, provide=*]{babel}
\babelprovide[main, import]{hebrew}
\babelprovide{rl}
\babelfont{rm}{Libertinus Serif}
\babelfont{sf}{Libertinus Sans}
\babelfont{tt}{Libertinus Mono}

% style
\AddToHook{cmd/section/before}{\clearpage}	% Add line break before section
\linespread{1.3}
\setcounter{secnumdepth}{0}		% Remove default number tags from sections, this won't do well with theorems
\AtBeginDocument{\setlength{\belowdisplayskip}{3pt}}
\AtBeginDocument{\setlength{\abovedisplayskip}{3pt}}

% operators
\DeclareMathOperator\cis{cis}
\DeclareMathOperator\Sp{Sp}
\DeclareMathOperator\tr{tr}
\DeclareMathOperator\im{Im}
\DeclareMathOperator\re{Re}
\DeclareMathOperator\diag{diag}
\DeclareMathOperator*\lowlim{\underline{lim}}
\DeclareMathOperator*\uplim{\overline{lim}}
\DeclareMathOperator\rng{rng}
\DeclareMathOperator\Sym{Sym}
\DeclareMathOperator\Arg{Arg}
\DeclareMathOperator\Log{Log}
\DeclareMathOperator\dom{dom}

% commands
%\renewcommand\qedsymbol{\textbf{מש''ל}}
%\renewcommand\qedsymbol{\fbox{\emoji{lizard}}}
\newcommand{\NN}[0]{\mathbb{N}}
\newcommand{\ZZ}[0]{\mathbb{Z}}
\newcommand{\QQ}[0]{\mathbb{Q}}
\newcommand{\RR}[0]{\mathbb{R}}
\newcommand{\CC}[0]{\mathbb{C}}
\newcommand{\FF}[0]{\mathbb{F}}
\newcommand{\PP}[0]{\mathbb{P}}
\newcommand{\TT}[0]{\mathbb{T}}
\newcommand{\acts}[0]{\circlearrowright}
\newcommand{\explain}[2] {
	\begin{flalign*}
		 && \text{#2} && \text{#1}
	\end{flalign*}
}
\newcommand{\maketitleprint}[0]{ \begin{center}
	\begin{tikzpicture}[scale=3]
		\duck[graduate=gray!20!black, tassel=red!70!black]
	\end{tikzpicture}	
\end{center}
}

% theorem commands
\newtheoremstyle{c_remark}
	{}	% Space above
	{}	% Space below
	{}% Body font
	{}	% Indent amount
	{\bfseries}	% Theorem head font
	{}	% Punctuation after theorem head
	{.5em}	% Space after theorem head
	{\thmname{#1}\thmnumber{ #2}\thmnote{ \normalfont{\text{(#3)}}}}	% head content
\newtheoremstyle{c_definition}
	{3pt}	% Space above
	{3pt}	% Space below
	{}% Body font
	{}	% Indent amount
	{\bfseries}	% Theorem head font
	{}	% Punctuation after theorem head
	{.5em}	% Space after theorem head
	{\thmname{#1}\thmnumber{ #2}\thmnote{ \normalfont{\text{(#3)}}}}	% head content
\newtheoremstyle{c_plain}
	{3pt}	% Space above
	{3pt}	% Space below
	{\itshape}% Body font
	{}	% Indent amount
	{\bfseries}	% Theorem head font
	{}	% Punctuation after theorem head
	{.5em}	% Space after theorem head
	{\thmname{#1}\thmnumber{ #2}\thmnote{ \text{(#3)}}}	% head content

\theoremstyle{c_plain}
\newtheorem{theorem}{משפט}[section]
\newtheorem{lemma}[theorem]{למה}
\newtheorem{proposition}[theorem]{טענה}
\newtheorem*{proposition*}{טענה}
%\newtheorem{corollary}[theorem]{אין חלופה עברית}

\theoremstyle{c_definition}
\newtheorem{definition}[theorem]{הגדרה}
\newtheorem*{definition*}{הגדרה}
\newtheorem{example}{דוגמה}[section]
\newtheorem{exercise}{תרגיל}[section]

\theoremstyle{c_remark}
\newtheorem*{remark}{הערה}
\newtheorem*{solution}{פתרון}
\newtheorem{conclusion}[theorem]{מסקנה}
\newtheorem{notation}[theorem]{סימון}

% Questions related commands
\newcounter{question}
\setcounter{question}{1}
\newcounter{sub_question}
\setcounter{sub_question}{1}

\newcommand{\question}[1][0]{
	\ifthenelse{#1 = 0}{}{\setcounter{question}{#1}}
	\subsection{שאלה \arabic{question}}
	\addtocounter{question}{1}
	\setcounter{sub_question}{1}
}

\newcommand{\subquestion}[1][0]{
	\ifthenelse{#1 = 0}{}{\setcounter{sub_question}{#1}}
	\subsubsection{סעיף \localecounter{letters.gershayim}{sub_question}}
	\addtocounter{sub_question}{1}
}

% import lua and start of document
\directlua{common = require ('../common')}

\GetEnv{AUTHOR}

% headers
\author{\AUTHOR}
\date\today

\title{פתרון מטלה 02 --- חשבון אינפינטסמלי 3 (80415)}

\begin{document}
\maketitle
\maketitleprint{}

\Question{}
יהי $X$ מרחב מטרי ו־$A \subseteq X$.

\Subquestion{}
נוכיח כי $A^\circ = A$ אם ורק אם היא פתוחה.
\begin{proof}
	\textbf{כיוון ראשון:}
	נניח כי $A^\circ = A$. נובע כי כל נקודה ב־$A$ היא גם נקודה פנימית ב־$A$ ולכן ניתן ליצור כדור סביבה המוכל ב־$A$, ועל־כן היא עומדת בהגדרה של קבוצה פתוחה. \\*
	\textbf{כיוון שני:}
	נניח כי $A$ קבוצה פתוחה. נניח בשלילה כי $A \ne A^\circ$ ולכן קיימת נקודה ב־$A$ שהיא לא נקודה פנים, ולכן אי אפשר ליצור כדור סביבה המוכל ב־$A$. אבל זאת סתירה להיותה של $A$ קבוצה פתוחה, ולכן $A = A^\circ$.
\end{proof}

\Subquestion{}
נוכיח כי $\overline{A} = A$ אם ורק אם $A$ קבוצה סגורה.
\begin{proof}
	\textbf{כיוון ראשון:}
	נניח כי מתקיים $\overline{A} = A$.
	תהי $\{ x_n \} \in A$ סדרת נקודות מתכנסת ל־$x$, מהשוויון נובע כי $x \in A$, ולכן נסיק כי $A$ סגורה. \\*
	\textbf{כיוון שני:}
	נניח כי $A$ קבוצה סגורה. באופן דומה לסעיף הקודם נניח כי $\overline{A} \ne A$ ולכן ישנה סדרת נקודות $\{x_n\} \in A$ שמתכנסת לנקודה $x \not\in A$, בסתירה לסגירות של $A$, ולכן נובע $A = \overline{A}$.
\end{proof}

\Subquestion{}
נוכיח את השוויון ${(A^\circ)}^C = \overline{A^C}$.
\begin{proof}
	נשים לב כי מההגדרה המשלים לקבוצה פתוחה הוא קבוצה סגורה, וידוע כי $A^\circ$ היא קבוצה פתוחה, על־כן המשלים שלה הוא קבוצה סגורה. \\*
	הקבוצה $A^C$ מכילה את אוסף כל הנקודות שאינן ב־$A$, לרבות נקודות קצה שלה, וידוע כי היא קבוצה סגורה ולכן היא שקולה לקבוצה $\overline{A^C}$.
\end{proof}

\Subquestion{}
נוכיח כי $\partial A = \partial (A^C)$.
\begin{proof}
	נראה כי
	\[
		\partial A
		= \overline{A} \setminus A^\circ
		= \overline{A} \cap {(A^\circ)}^C
		= \overline{A} \cap \overline{A^C}
		= \overline{A^C} \cap {({(A^C)}^\circ)}^C
		= \overline{A^C} \setminus {(A^C)}^\circ
		= \partial (A^C)
	\]
\end{proof}

\Question{}
יהיו $(X, \rho)$ מרחב נורמי, $x_0 \in X$ ו־$r > 0$, ונתון $D = \{ x \in X \mid \rho(x, x_0) \le r \}$ כדור סגור מתאים.

\Subquestion{}
נוכיח כי $D^\circ = B(x_0, r)$.
\begin{proof}
	תהי $x \in D$ נקודה פנימית, ונניח בשלילה כי $\rho(x, x_0) = r$. לכל $r_0 > 0$ שנבחר $B(x, r_0) \not\subseteq D$ שכן $\rho(y, x) + \rho(x_0, x) > r$. \\*
	לכן $x$ לא נקודה פנימית של $D$ ובהתאם $D^\circ = \{ x \in X \mid \rho(x, x_0) < r\} = B(x_0, r)$.
\end{proof}

\Subquestion{}
אנו יודעים כי $\partial D = \overline{D} \setminus D^\circ$. ידוע לנו כי $D$ הוא כדור סגור ולכן גם $D = \overline{D}$ ומצאנו כי $D^\circ = B(x_0, r)$ ולכן נובע
\[
	D \setminus D^\circ = \{ x \in X \mid x \le r \land \lnot x < r \}
	= \{ x \in X \mid x \le r \land x \ge r \}
	= \{ x \in X \mid x = r \}
	= \partial D = S(x_0, r)
\]

\Question{}
נגדיר את הפונקציות $f, g : \RR^2 \to \RR$ על־ידי
\[
	f(x, y) = xy, g(x, y) = x + y
\]
נוכיח על־פי הגדרה כי $f$ רציפה על־פי הגדרה.
\begin{proof}
	יהי $\epsilon > 0$ ונקודות $p_0 = (x_0, y_0) \in \RR^2$. \\*
	נראה כי $\rho(f(p_0), f(p_0 + h)) < \epsilon \iff |(x_0 + h)(y_0 + h) - x_0y_0| = |h| \cdot |x_0 + y_0 + h| < \epsilon$, \\*
	אז נגדיר $\delta_0 = \frac{\epsilon}{|x_0 + y_0 + h|}$ ונקבל $\rho(p_0, p_0 + h) < \delta_0 \iff \sqrt{{(x_0 + h - x_0)}^2 + {(y_0 + h - y_0)}^2} = \sqrt{2} |h||x_0 + y_0 + h| < \epsilon$ \\*
	ונקבל כי הטענה מתקיימת ולכן $f$ רציפה בכל נקודה $(x_0, y_0) \in \RR^2$.
\end{proof}
נוכיח כי $g$ אף היא רציפה על־פי הגדרה.
\begin{proof}
	נראה כי
	\[
		\rho(f(p_0), f(p_0 + h)) < \epsilon \iff |x_0 + h + y_0 + h - x_0 - y_0| = 2|h| < \epsilon
	\]
	לכן נגדיר $\delta_1 = \sqrt{2} \epsilon$ ונקבל בדומה להוכחה הקודמת כי $2 |h| < \epsilon$ כפי שהיה עלינו למצוא.
\end{proof}

\Question{}
יהי $(X, \rho)$ מרחב מטרי ונגדיר $f : X \times X \to \RR^+, f(x, y) = \rho(x, y)$ פונקציית המרחק במרחב לממשיים האי־שליליים. נוכיח כי $f$ רציפה.
\begin{proof}
	נבחר קטע פתוח $(a, b) \in \RR^+$ ונבחין כי $f^{-1}((a, b)) = \{ x, y \in X \times X \mid a < \rho(x, y) < b \}$ ונשים לב כי זוהי קבוצה לא סופית של איחוד קבוצות פתוחות. \\*
	זאת שכן ההגדרה מתלכדת עם כדורים פתוחים סביב $x$ וחיתוך קבוצה סגורה מהם, כך שניתן ליצור כדור פתוח סביב כל נקודה בקבוצה. \\*
	כל קבוצה פתוחה ב־$\RR$ היא איחוד של קטעים פתוחים ולכן גם התמונה ההפוכה של כל קבוצה פתוחה היא איחוד קבוצות פתוחות ולכן מהווה קבוצה פתוחה. \\*
	לכן נובע כי $f$ היא פונקציה רציפה.
\end{proof}

\Question{}
נגדיר $X_1 = (C[0, 1], \lVert \cdot \rVert_1), X_2 = (C[0, 1], \lVert \cdot \rVert_\infty)$ ותהי פונקציה $\Lambda : C[0, 1] \to \RR$ המוגדרת על־ידי $\Lambda(f) = f(1)$.

\Subquestion{}
נפריך את הטענה כי $\Lambda$ רציפה כפונקציה $X_1 \to \RR$ על־ידי דוגמה נגדית:
\begin{proof}[דוגמה נגדית]
	נגדיר סדרת פונקציות ${(f_n)}_{n = 1}^\infty \subseteq C[0, 1]$ על־ידי
	\[
		f_n(x) = \begin{cases}
			0, & 0 \le x < 1 - \frac{2}{n} \\
			\frac{n - 0}{1 - (1 - \frac{1}{n^2})}(x - 1 + \frac{2}{n}), & 1 - \frac{2}{n} \le x \le 1
		\end{cases}
	\]
	הפונקציה מוגדרת להיות $0$ ואז משולש שבסיסו הולך וקטן ושמסתיים ב־$f_n(1) = n$ לכל $n \in \NN$, כך ששטחו הוא $\frac{1}{n}$ ובעקבות כך גם האינגרל שלו. \\*
	אילו נגדיר $f(x) = 0$ אז מתקיים $f_n \xrightarrow{n \to \infty} f$. \\*
	על־פי ההגדרה $\lVert f - f_n \rVert_1 = \frac{1}{n}$ ולכן הסדרה $(f_n)$ מקיימת את התנאים להיות סדרת היינה, אבל
	\[
		\lim_{n \to \infty} | \Lambda(f) - \Lambda(f_n) |
		= \lim_{n \to \infty} | 0 - n |
		= \infty
	\]
	ומצאנו סדרה הסותרת את ההתכנסות של הפונקציה.
\end{proof}

\Subquestion{}
נוכיח כי $\Lambda$ היא רציפה כפונקציה $X_2 \to \RR$.
\begin{proof}
	יהי $\epsilon > 0$ ופונקציה $f_0 \in C[0, 1]$. נראה כי על־פי הגדרת הרציפות $| \Lambda(f) - \Lambda(f_0) | < \epsilon \iff |f(1) - f_0(1)| < \epsilon$. \\*
	נגדיר $\delta = \epsilon$ נבחן $\lVert f - f_0 \rVert_\infty = \max_{x \in [0, 1]}|f(x) - f_0(x)| < \delta \implies |f(1) - f_0(1)| < \epsilon$. \\*
	מצאנו כי הגדרת הרציפות מתקיימת ולכן $\Lambda$ רציפה ב־$X_2$.
\end{proof}

\Question{}
נגדיר
\[
	X = [0, 2 \pi) \subseteq \RR, \quad Y = S_1(0) = \{ x \in \RR^2 \mid \lVert x \rVert = 1 \} \subseteq \RR^2 % chktex 9
\]
ונגדיר $f : X \to Y$ על־ידי
\[
	f(t) = (\cos t, \sin t)
\]

\Subquestion{}
נוכיח כי $f$ רציפה.
\begin{proof}
	על־פי טענה מהתרגול הפונצקיה $f$ רציפה אם ורק אם היא רציפה באגפיה השונים. \\*
	נגדיר $f_i = \pi_i \circ f$ פירוק לאגפים, ונראה כי $f_1(t) = \cos t$ פונקציה רציפה וגם $f_2(t) = \sin t$ רציפה אף היא, ולכן $f$ רציפה בכללותה.
\end{proof}

\Subquestion{}
נוכיח כי $f$ פונקציה פתוחה.
\begin{proof}
	יהי $(a, b) \subseteq X$ כדור פתוח. \\*
	מהגדרת $\cos$ אנו יכולים להסיק כי $\cos^{-1} b < f_1(t) < \cos^{-1} a$ וכמובן זהו כדור פתוח. נוכל אם כן לחבר כל כמות של כדורים פתוחים ב־$X$ ולקבל חיבור כדורים ב־$Y$. 
	כמובן שהטענה הזו מתקיימת גם עבור האגף השני שכן $\sin$ אף היא רציפה ומקיימת את הטענה. \\*
	בהתאם לכל קבוצה $U \subseteq X$ קבוצה פתוחה גם $f(U)$ פתוחה.
\end{proof}

\Subquestion{}
נוכיח כי $f$ פונקציה סגורה.
\begin{proof}
	נובע ישירות ממשפט ויירשטראס לתמונות פונקציות רציפות על האגפים השונים של $f$.
\end{proof}

\Subquestion{}
נוכיח כי $f$ היא חד־חד ערכית ועל $Y$.
\begin{proof}
	נניח בשלילה כי $f$ איננה חד־חד ערכית, לכן קיים $x, y \in X$ כך ש־$x < y$ אבל $f(x) = f(y)$. \\*
	מסיבה זו נקבל כי $\cos x = \cos y$ וגם $\sin x = \sin y$. מכפל השוויונות נקבל $\cos x \cos y - \sin x \sin y = \cos(x + y) = 0$. \\*
	בהתאם $x + y = \frac{\pi}{2}, \frac{3 \pi}{2}$. באופן דומה נקבל גם $\sin(x + y) = 0$ ולכן $x + y = 0, \pi$ וקיבלנו סתירה, לכן $f$ חד־חד ערכית. \\*
	נבחר נקודה $(x, y) \in Y$. גאומטרית נוכל לבחור את הזווית שהקרן לנקודה הזאת מהמרכז יוצרת עם הכיוון החיובי של ציר ה־$x$, ועל־פי הגדרה זוהי הזווית $t$ עבורה $(x, y) = (\cos x, \sin y)$.
	מצאנו כי $f$ חד־חד ערכית ועל.
\end{proof}

\Subquestion{}
נוכיח כי הפונקציה ההופכית $f^{-1}$ היא רציפה אף היא.
\begin{proof}
	למעשה, כבר הוכחנו טענה זו בסעיף ב', ראינו שכל קבוצה פתוחה $U \subseteq X$ גם ${(f^{-1})}^{-1}(U)$ פתוחה, ולכן נובע כי $f^{-1}$ רציפה.
\end{proof}

\Question{}
יהי $K$ מרחב מטרי קומפקטי ו־$f : K \to K$ העתקה מכווצת.

\Subquestion{}
נוכיח כי ל־$f$ יש נקודת שבת.
\begin{proof}
	יהי $x \in K$, ונגדיר $f_n : K \to K$ על־ידי
	\[
		f_n(x) = (\circ_{n} f)(x) = (f \circ f_{n - 1})(x)
	\]
	נגדיר סדרה חדשה ${(a_n)}_{n = 1}^\infty$ על־ידי
	\[
		a_n = f_n(x)
	\]
	על־פי הגדרת הכיווץ מתקיים לכל $n \in \NN$
	\[
		\rho(f_2(x), f_1(x)) < \rho(f_1(x), x), \qquad
		\rho(f_{n + 1}(x), f_n(x)) < \rho(f_{n}(x), f_{n - 1}(x))
	\]
	מהקומפקטיות נסיק כי ל־$(a)$ יש תת־סדרה מתכנסת, נגדיר ${(n_k)}_{k = 1}^\infty$ ונקבל כי $a_{n_k} \xrightarrow{n \to \infty} L$ ערך סופי כלשהו. \\*
	נוכל כמובן להסיק עבור ההגדרה של $a$ כי $L = f(L)$, ולכן גם $0 \le \rho(L, f(L)) < \rho(f_{n + 1}(x), f_n(x))$ לכל $n$ טבעי, ונסיק $\rho(L, f(L))$ והגדרת הכיווץ נשמרה.
\end{proof}

\Subquestion{}
נוכיח כי נקודת השבת של $f$ היא יחידה.
\begin{proof}
	נבחר שני ערכים $x, y \in X$. נגדיר $(a)$ כמו בסעיף הקודם וסדרה $(b)$ שקולה המקיימת $b_n = f_n(y)$. \\*
	נסיק מיידית מהכיווץ כי לכל $n \in \NN$
	\[
		0 \le \rho(a_{n + 1}, b_{n + 1}) < \rho(a_n, b_n)
	\]
	מאי־שוויון זה נסיק מונוטוניות יורדת וחסימות ולכן אם הסדרות מתכנסות אז $\lim_{n \to \infty} a_n = \lim_{n \to \infty} b_n$. \\*
	המרחב $X$ קומפקטי ולכן נוכל למצוא תת־סדרה מתכנסת של $(a)$ ותת־סדרה מתכנסת לתת־הסדרה של $(b)$ המוגדרת על־פי האינדקסים לתת־הסדרה הראשונה, לצורך הפשטות נגדיר $(n_k)$ סדרת אינדקסים עבורה בהתאם לשוויון הקודם
	\[
		\lim_{k \to \infty} a_{n_k}
		= \lim_{k \to \infty} b_{n_k}
		= L
	\]
	ומצאנו כי נקודת השבת $L \in X$ איננה תלויה בבחירת $x \in X$ ולכן היא יחידה.
\end{proof}

\end{document} % chktex 17
