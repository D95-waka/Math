\documentclass[a4paper]{article}

% packages
\usepackage{inputenc, amsmath, amsthm, thmtools, amsfonts, amssymb, luacode, catchfile, tikzducks, hyperref}
\usepackage[a4paper, margin=50pt, includeheadfoot]{geometry} % set page margins
\usepackage[shortlabels]{enumitem}
\usepackage[skip=3pt, indent=0pt]{parskip}

% language
\usepackage[bidi=basic, layout=tabular, provide=*]{babel}
\babelprovide[main, import]{hebrew}
\babelprovide{rl}
\babelfont{rm}{Libertinus Serif}
\babelfont{sf}{Libertinus Sans}
\babelfont{tt}{Libertinus Mono}

% style
\AddToHook{cmd/section/before}{\clearpage}	% Add line break before section
\linespread{1.3}
\setcounter{secnumdepth}{0}		% Remove default number tags from sections, this won't do well with theorems
\AtBeginDocument{\setlength{\belowdisplayskip}{3pt}}
\AtBeginDocument{\setlength{\abovedisplayskip}{3pt}}

% operators
\DeclareMathOperator\cis{cis}
\DeclareMathOperator\Sp{Sp}
\DeclareMathOperator\tr{tr}
\DeclareMathOperator\im{Im}
\DeclareMathOperator\re{Re}
\DeclareMathOperator\diag{diag}
\DeclareMathOperator*\lowlim{\underline{lim}}
\DeclareMathOperator*\uplim{\overline{lim}}
\DeclareMathOperator\rng{rng}
\DeclareMathOperator\Sym{Sym}
\DeclareMathOperator\Arg{Arg}
\DeclareMathOperator\Log{Log}
\DeclareMathOperator\dom{dom}

% commands
%\renewcommand\qedsymbol{\textbf{מש''ל}}
%\renewcommand\qedsymbol{\fbox{\emoji{lizard}}}
\newcommand{\NN}[0]{\mathbb{N}}
\newcommand{\ZZ}[0]{\mathbb{Z}}
\newcommand{\QQ}[0]{\mathbb{Q}}
\newcommand{\RR}[0]{\mathbb{R}}
\newcommand{\CC}[0]{\mathbb{C}}
\newcommand{\FF}[0]{\mathbb{F}}
\newcommand{\PP}[0]{\mathbb{P}}
\newcommand{\TT}[0]{\mathbb{T}}
\newcommand{\acts}[0]{\circlearrowright}
\newcommand{\explain}[2] {
	\begin{flalign*}
		 && \text{#2} && \text{#1}
	\end{flalign*}
}
\newcommand{\maketitleprint}[0]{ \begin{center}
	\begin{tikzpicture}[scale=3]
		\duck[graduate=gray!20!black, tassel=red!70!black]
	\end{tikzpicture}	
\end{center}
}

% theorem commands
\newtheoremstyle{c_remark}
	{}	% Space above
	{}	% Space below
	{}% Body font
	{}	% Indent amount
	{\bfseries}	% Theorem head font
	{}	% Punctuation after theorem head
	{.5em}	% Space after theorem head
	{\thmname{#1}\thmnumber{ #2}\thmnote{ \normalfont{\text{(#3)}}}}	% head content
\newtheoremstyle{c_definition}
	{3pt}	% Space above
	{3pt}	% Space below
	{}% Body font
	{}	% Indent amount
	{\bfseries}	% Theorem head font
	{}	% Punctuation after theorem head
	{.5em}	% Space after theorem head
	{\thmname{#1}\thmnumber{ #2}\thmnote{ \normalfont{\text{(#3)}}}}	% head content
\newtheoremstyle{c_plain}
	{3pt}	% Space above
	{3pt}	% Space below
	{\itshape}% Body font
	{}	% Indent amount
	{\bfseries}	% Theorem head font
	{}	% Punctuation after theorem head
	{.5em}	% Space after theorem head
	{\thmname{#1}\thmnumber{ #2}\thmnote{ \text{(#3)}}}	% head content

\theoremstyle{c_plain}
\newtheorem{theorem}{משפט}[section]
\newtheorem{lemma}[theorem]{למה}
\newtheorem{proposition}[theorem]{טענה}
\newtheorem*{proposition*}{טענה}
%\newtheorem{corollary}[theorem]{אין חלופה עברית}

\theoremstyle{c_definition}
\newtheorem{definition}[theorem]{הגדרה}
\newtheorem*{definition*}{הגדרה}
\newtheorem{example}{דוגמה}[section]
\newtheorem{exercise}{תרגיל}[section]

\theoremstyle{c_remark}
\newtheorem*{remark}{הערה}
\newtheorem*{solution}{פתרון}
\newtheorem{conclusion}[theorem]{מסקנה}
\newtheorem{notation}[theorem]{סימון}

% Questions related commands
\newcounter{question}
\setcounter{question}{1}
\newcounter{sub_question}
\setcounter{sub_question}{1}

\newcommand{\question}[1][0]{
	\ifthenelse{#1 = 0}{}{\setcounter{question}{#1}}
	\subsection{שאלה \arabic{question}}
	\addtocounter{question}{1}
	\setcounter{sub_question}{1}
}

\newcommand{\subquestion}[1][0]{
	\ifthenelse{#1 = 0}{}{\setcounter{sub_question}{#1}}
	\subsubsection{סעיף \localecounter{letters.gershayim}{sub_question}}
	\addtocounter{sub_question}{1}
}

% import lua and start of document
\directlua{common = require ('../common')}

\GetEnv{AUTHOR}

% headers
\author{\AUTHOR}
\date\today

\title{פתרון מטלה 02 --- חשבון אינפינטסמלי 3 (80415)}

\begin{document}
\maketitle
\maketitleprint{}

\Question{}
נמצא שתי קבוצות סגורות ולא ריקות $A, B \subseteq \RR^2$ כך שהמרחק ביניהן לא מתקבל. \\*
נבחר את הקבוצות $A = \{ (x, y) \in \RR^2 \mid y = 0\}, B = \{ (x, y) \in \RR^2 \mid x y = 1 \}$. \\*
שתי הקבוצות ניתנות לתיאור על־ידי פונקציות $\RR \to \RR$ רציפות בכל תחומן ולכן הקבוצות מכילות את כל הנקודות הגבוליות שלהן, ובהתאם סגורות. \\*
לעומת זאת, הקבוצות זרות, ונראה כי לכל $x \in \RR$ מתקיים $\rho((x, 0), (x, \frac{1}{x})) = \sqrt{0^2 + \frac{1}{x^2}} = \frac{1}{|x|}$ ולכן $\inf \text{dist}(A, B) = 0$

\Question{}
יהי $V$ מרחב וקטורי ו־$\lVert \cdot \rVert, \lVert \cdot \rVert'$ שתי נורמות עליו. \\*
תהי $S_1 = \{ v \in V \mid \lVert v \rVert = 1 \}$ ונגדיר $A = \{ \lVert v\rVert' \mid v \in S_1\}$. \\*
נוכיח כי הנורמות שקולות אם ורק אם $\inf A > 0, \sup A < \infty$.
\begin{proof}
	\textbf{כיוון ראשון:}
	נניח כי הנורמות שקולות. \\*
	לכן קיימים $0 < C_1 \le C_2$ כך ש־$\forall v \in V : C_1 \lVert v \rVert \le \lVert v \rVert' \le C_2 \lVert v \rVert$. \\*
	בפרט
	\[
		\forall v \in S_1 : C_1 \le \lVert v \rVert' \le C_2
		\implies \forall a \in A : C_1 \le a \le C_2
	\]
	ובהתאם $\inf A \ge C_1 \ge 0$ וגם $\sup A \le C_2 < \infty$.

	\textbf{כיוון שני:}
	נניח כי $\inf A = a > 0$ ו־$\sup A = b < \infty$. \\*
	לכן נובע ישירות $\forall x \in A : a \le x \le b$ ובהתאם $\forall v \in S_1 : a \lVert v \rVert \le \lVert v \rVert' \le b \lVert v \rVert$. \\*
	יהי $u \in V$, אז אנו יודעים כי קיים $u^* \in V$ כך ש־$\lVert u^* \rVert = 1$ ו־$\lambda \in \RR^+$ כך ש־$u = \lambda u^*$ ולכן גם $\lVert u \rVert = \lambda \lVert u^* \rVert = \lambda$. \\*
	נשים לב עתה כי $u^* \in S_1$ ולכן $a \lVert u^* \rVert \le \lVert u^* \rVert' \le b \lVert u^* \rVert$, נכפיל את שלושת האגפים ב־$\lambda$ ונקבל
	\[
		a \lVert u \rVert \le \lVert u \rVert' \le b \lVert u \rVert
	\]
	ולכן הנורמות שקולות.
\end{proof}

\Question{}
יהי $(V, \lVert \cdot \rVert)$ מרחב נורמי ממימד $d$ מעל $\RR$.

\Subquestion{}
תהי $T : \RR^d \to V$ העתקה לינארית חד־חד ערכית ועל. \\*
נוכיח שהפונקציה $f : \RR^d \to \RR$ המוגדרת על־ידי $f(u) = \lVert T(u) \rVert$ משרה נורמה על $\RR^d$.
\begin{proof}
	נראה כי שלוש התכונות של נורמה מתקיימות.
	\begin{enumerate}
		\item חיוביות: ידוע כי $T$ הפיכה ולכן גרעינה הוא וקטור האפס בלבד, ובהתאם $f(u) = \lVert Tu \rVert = 0 \iff \lVert u \rVert = 0$.
		\item כפל בסקלר: $\forall u \in \RR^2, \lambda \in \RR, f(\lambda u) = \lVert T (\lambda u) \rVert  = \lVert \lambda \cdot Tu \rVert = |\lambda| \lVert Tu \rVert = |\lambda| f(u)$.
		\item אי־שוויון המשולש: $\forall u, v \in \RR^2, f(u + v) = \lVert T(u + v) \rVert = \lVert Tu + Tv \rVert \le \lVert T u \rVert + \lVert T v \rVert = f(u) + f(v)$.
	\end{enumerate}
	ומצאנו כי $f$ מקיימת את התנאים לנורמה ולכן מהווה נורמה ל־$\RR^d$.
\end{proof}
יהי $B = (v_1, \dots, v_d)$ בסיס ל־$V$, ונגדיר נורמה על־ידי
\[
	\lVert \alpha v_1 + \cdots + \alpha_d v_d \rVert_1 = \sum_{i = 1}^{d} |\alpha_1| = d |\alpha_1|
\]
נוכיח שהנורמות $\lVert \cdot \rVert, \lVert \cdot \rVert_1$ שקולות.
\begin{proof}
	בסעיף הקודם הוכחנו ששימוש ב־$T$ מאפשר להשרות נורמה מ־$V$ ל־$\RR^2$, ולכן אפשר להגדיר פונצקיה משרה לנורמה לשתי הנורמות הנתונות. \\*
	נשתמש בטענה כי כל הנורמות ב־$\RR^d$ הן שקולות ונקבל כי גם $\lVert \cdot \rVert$ שקולה ל־$\lVert \cdot \rVert_1$.
\end{proof}

\Question{}
יהיו $X, Y$ מרחביים מטריים ו־$f : X \to Y$ פונקציה חד־חד ערכית ועל. \\*
נוכיח כי שלושת התנאים הבאים שקולים:
\begin{enumerate}
	\item $f$ היא הומיאומורפיזם.
	\item כל קבוצה $U \subseteq X$ היא פתוחה אם ורק אם $f(U)$ פתוחה.
	\item כל קבוצה $U \subseteq X$ היא סגורה אם ורק אם $f(U)$ סגורה.
\end{enumerate}

\begin{proof}
	\textbf{1 \leftarrow{} 2:}
	נניח כי $f$ היא הומיאומורפיזם. \\*
	לכן $f$ רציפה וגם $f^{-1}$ רציפה. לכן נסיק מהתנאי המספיק לרציפות כי $U \subseteq X$ פתוחה אז $f(U)$ פתוחה ולכל $U \subseteq Y$ גם $f^{-1}(U) \subseteq X$ פתוחה. \\*
	במסתכם $U \subseteq X$ פתוחה אם ורק אם $f(U) \subseteq Y$ פתוחה.

	\textbf{2 \leftarrow{} 3:}
	נניח כי $U \subseteq X$ פתוחה אם ורק אם $f(U) \subseteq Y$ פתוחה. \\*
	תהי קבוצה סגורה $A \subseteq X$, אז $A^C$ קבוצה פתוחה ולכן $f(A^C)$ קבוצה פתוחה ב־$Y$. \\*
	באופן דומה $f(U)$ קבוצה פתוחה. אילו נניח בשלילה כי $f(U) \cap f(U^C) \ne \emptyset$ נקבל כי ישנם $x \ne y \in X$ כך ש־$f(x) = f(y)$ בסתירה לחד־חד ערכיות, ולכן $f(U) \cap f(U^C) = \emptyset$. \\*
	נראה גם $f(U) \cup f(U^C) = Y$ אחרת תכונת העל נסתרת עבור $f$, ולכן נסיק $f(U^C) = {(f(U))}^C$. \\*
	מטענה זו נוכל להסיק ישירות ש־$U \subseteq X$ סגורה אם ורק אם $f(U) \subseteq Y$ סגורה.

	\textbf{2 \leftarrow{} 3:}
	נניח כי כל קבוצה $U \subseteq X$ סגורה אם ורק אם $f(U) \subseteq Y$ סגורה אף היא. \\*
	נוכל להשתמש בחד־חד ערכיות באופן זהה לסעיף הקודם וכך גם בעל ונקבל כי $f(U^C) = {(f(U))}^C$ לכל $U \subseteq X$. \\*
	נשים לב כי התנאי המספיק לרציפות, שמקור כל קבוצה פתוחה הוא קבוצה פתוחה מתקיים לשני הכיוונים, לכן $f, f^{-1}$ רציפות ובהתאם $f$ הומיאומורפיזם.
\end{proof}

\Question{}
יהיו $X, Y$ מרחבים מטריים ו־$f : X \to Y$ פונקציה חד־חד ערכית, על ורציפה.

\Subquestion{}
נוכיח שאם $X$ קומפקטי אז $f$ היא הומיאומורפיזם.

\end{document}
