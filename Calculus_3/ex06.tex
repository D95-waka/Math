\documentclass[a4paper]{article}

% packages
\usepackage{inputenc, amsmath, amsthm, thmtools, amsfonts, amssymb, luacode, catchfile, tikzducks, hyperref}
\usepackage[a4paper, margin=50pt, includeheadfoot]{geometry} % set page margins
\usepackage[shortlabels]{enumitem}
\usepackage[skip=3pt, indent=0pt]{parskip}

% language
\usepackage[bidi=basic, layout=tabular, provide=*]{babel}
\babelprovide[main, import]{hebrew}
\babelprovide{rl}
\babelfont{rm}{Libertinus Serif}
\babelfont{sf}{Libertinus Sans}
\babelfont{tt}{Libertinus Mono}

% style
\AddToHook{cmd/section/before}{\clearpage}	% Add line break before section
\linespread{1.3}
\setcounter{secnumdepth}{0}		% Remove default number tags from sections, this won't do well with theorems
\AtBeginDocument{\setlength{\belowdisplayskip}{3pt}}
\AtBeginDocument{\setlength{\abovedisplayskip}{3pt}}

% operators
\DeclareMathOperator\cis{cis}
\DeclareMathOperator\Sp{Sp}
\DeclareMathOperator\tr{tr}
\DeclareMathOperator\im{Im}
\DeclareMathOperator\re{Re}
\DeclareMathOperator\diag{diag}
\DeclareMathOperator*\lowlim{\underline{lim}}
\DeclareMathOperator*\uplim{\overline{lim}}
\DeclareMathOperator\rng{rng}
\DeclareMathOperator\Sym{Sym}
\DeclareMathOperator\Arg{Arg}
\DeclareMathOperator\Log{Log}
\DeclareMathOperator\dom{dom}

% commands
%\renewcommand\qedsymbol{\textbf{מש''ל}}
%\renewcommand\qedsymbol{\fbox{\emoji{lizard}}}
\newcommand{\NN}[0]{\mathbb{N}}
\newcommand{\ZZ}[0]{\mathbb{Z}}
\newcommand{\QQ}[0]{\mathbb{Q}}
\newcommand{\RR}[0]{\mathbb{R}}
\newcommand{\CC}[0]{\mathbb{C}}
\newcommand{\FF}[0]{\mathbb{F}}
\newcommand{\PP}[0]{\mathbb{P}}
\newcommand{\TT}[0]{\mathbb{T}}
\newcommand{\acts}[0]{\circlearrowright}
\newcommand{\explain}[2] {
	\begin{flalign*}
		 && \text{#2} && \text{#1}
	\end{flalign*}
}
\newcommand{\maketitleprint}[0]{ \begin{center}
	\begin{tikzpicture}[scale=3]
		\duck[graduate=gray!20!black, tassel=red!70!black]
	\end{tikzpicture}	
\end{center}
}

% theorem commands
\newtheoremstyle{c_remark}
	{}	% Space above
	{}	% Space below
	{}% Body font
	{}	% Indent amount
	{\bfseries}	% Theorem head font
	{}	% Punctuation after theorem head
	{.5em}	% Space after theorem head
	{\thmname{#1}\thmnumber{ #2}\thmnote{ \normalfont{\text{(#3)}}}}	% head content
\newtheoremstyle{c_definition}
	{3pt}	% Space above
	{3pt}	% Space below
	{}% Body font
	{}	% Indent amount
	{\bfseries}	% Theorem head font
	{}	% Punctuation after theorem head
	{.5em}	% Space after theorem head
	{\thmname{#1}\thmnumber{ #2}\thmnote{ \normalfont{\text{(#3)}}}}	% head content
\newtheoremstyle{c_plain}
	{3pt}	% Space above
	{3pt}	% Space below
	{\itshape}% Body font
	{}	% Indent amount
	{\bfseries}	% Theorem head font
	{}	% Punctuation after theorem head
	{.5em}	% Space after theorem head
	{\thmname{#1}\thmnumber{ #2}\thmnote{ \text{(#3)}}}	% head content

\theoremstyle{c_plain}
\newtheorem{theorem}{משפט}[section]
\newtheorem{lemma}[theorem]{למה}
\newtheorem{proposition}[theorem]{טענה}
\newtheorem*{proposition*}{טענה}
%\newtheorem{corollary}[theorem]{אין חלופה עברית}

\theoremstyle{c_definition}
\newtheorem{definition}[theorem]{הגדרה}
\newtheorem*{definition*}{הגדרה}
\newtheorem{example}{דוגמה}[section]
\newtheorem{exercise}{תרגיל}[section]

\theoremstyle{c_remark}
\newtheorem*{remark}{הערה}
\newtheorem*{solution}{פתרון}
\newtheorem{conclusion}[theorem]{מסקנה}
\newtheorem{notation}[theorem]{סימון}

% Questions related commands
\newcounter{question}
\setcounter{question}{1}
\newcounter{sub_question}
\setcounter{sub_question}{1}

\newcommand{\question}[1][0]{
	\ifthenelse{#1 = 0}{}{\setcounter{question}{#1}}
	\subsection{שאלה \arabic{question}}
	\addtocounter{question}{1}
	\setcounter{sub_question}{1}
}

\newcommand{\subquestion}[1][0]{
	\ifthenelse{#1 = 0}{}{\setcounter{sub_question}{#1}}
	\subsubsection{סעיף \localecounter{letters.gershayim}{sub_question}}
	\addtocounter{sub_question}{1}
}

% import lua and start of document
\directlua{common = require ('../common')}

\GetEnv{AUTHOR}

% headers
\author{\AUTHOR}
\date\today

\title{פתרון מטלה 06 --- חשבון אינפינטסמלי 3 (80415)}

\begin{document}
\maketitle
\maketitleprint{}

\Question{}
תהי $A \subseteq \RR^2$ פתוחה ותהי $f : A \to \RR$ כך שכל הנגזרות החלקיות של $f$ מסדר ראשון ושני קיימות ורציפות ב־$A$.

\Subquestion{}
נוכיח כי $f$ גזירה פעמיים.
\begin{proof}
	תהי $x_0 \in A$, נתון כי כל הנגזרות החלקיות שלה מוגדרות, ולכן נובע כי $f$ גזירה ונתון כי היא גם רציפה, דהינו $Df|_{x_0} : A \to \hom(A, \RR)$ רציפה,
	ונתון כי היא גם גזירה ברציפות ולכן מאותו משפט נובע גם כי $D^2 f\mid_{x_0}$ גם היא מוגדרת ורציפה.
\end{proof}

\Subquestion{}
יהיו $v = (v_1, v_2)$ ו־$w = (w_1, w_2)$ וקטורים ב־$\RR^2$, נכתוב את $D^2 f \mid_p(v, w)$.

נבחין כי
\[
	Df\mid_v x = \nabla f_v(x) = (\partial_x f(v) x, \partial_y f(v) x)
\]
כאשר $Df\mid_v : \RR^2 \to \RR^2$. \\*
עוד אנו יודעים כי $D^2 f : \RR^2 \to \RR^{2 \times 2}$, ובהתאם נקבל
\[
	D^f(v, w) = \begin{pmatrix}
		\partial_{xx}f(v, w) & \partial_{yx}f(v, w) \\
		\partial_{xy}f(v, w) & \partial_{yy}f(v, w)
	\end{pmatrix}
\]

\Question{}
נגדיר $f : \RR^2 \to \RR$ על־ידי
\[
	f(x, y) = \begin{cases}
		\frac{x^2y^2}{x^2 + y^2} & x^2 + y^2 \ne 0 \\
		0 & x^2 + y^2 = 0
	\end{cases}
\]

\Subquestion{}
נוכיח כי $f$ גזירה בכל נקודה ונחשב את $\nabla f$.
\begin{proof}
	נראה כי בנקודות $(x, y) \ne (0, 0)$ זוהי הרכבת פונקציות רציפות וגזירות ונוכל לחשב ולקבל
	\[
		\nabla f (x, y) = (\frac{2xy^4}{{(x^2 + y^2)}^2}, \frac{2x^4y}{{(x^2 + y^2)}^2})
	\]
	נבדוק את הגבול ב־$(0, 0)$:
	\[
		\lim_{(x, y) \to (0, 0)} \frac{f(x, y) - f(0, 0)}{\sqrt{x^2 + y^2}}
		= \lim_{(x, y) \to (0, 0)} \frac{x^2y^2}{\lVert (x, y) \rVert^3}
		= \lim_{(x, y) \to (0, 0)} \frac{\lVert (x, y) \rVert^4}{\lVert (x, y) \rVert^3}
		= 0
	\]
	ומצאנו כי $\nabla f(0, 0) = (0, 0)$, וכי הפונקציה גזירה בכל נקודה.
\end{proof}

\Subquestion{}
נוכיח כי כל הנגזרות החלקיות מסדר שני של $f$ קיימות ב־$(0, 0)$.
\begin{proof}
	נבדוק
	\[
		\partial_{xx} f(0, 0)
		= \lim_{x \to 0} \frac{2x \cdot 0^4}{\lVert (x, 0) \rVert^3}
		= 0
	\]
	ונקבל באופן דומה גם כי $\partial_{yy} f (0, 0) = 0$. \\*
	נותר לבדוק את $\partial_{xy} f(0, 0)$, נראה
	\[
		\partial_{xy} f(0, 0)
		= \lim_{x \to 0} \frac{2 x^4 \cdot 0}{\lVert (x, 0) \rVert^3}
		= 0
	\]
	וקיבלנו כי כל הנגזרות החלקיות מתאפסות בנקודה.
\end{proof}

\Subquestion{}
נראה כי $f$ לא גזירה פעמיים ב־$(0, 0)$.
\begin{proof}
	נבחן את $g(x, y) = \frac{2xy^4}{{(x^2 + y^2)}^2}$, הרכיב הראשון בנגזרת של $f$, ונבדוק את נגזרתו הכיוונית עבור $v = (1, 1)$:
	\[
		\lim_{t \to 0} \frac{g(t, t) - g(0, 0)}{t} = \frac{2t^5}{\sqrt{2} t^5} = \frac{\sqrt{2}}{2}
	\]
	דהינו, נגזרת מכוונת זו היא לא אפס בסתירה למה שמצאנו בסעיף ההקודם.
\end{proof}

\Question{}
תהי $f : \RR^d \to \RR^m$ פונקציה גזירה שלוש פעמים ב־$p \in \RR^d$. יהי $\{e_1, \dots, e_d\}$ הבסיס הסטנדרטי של $\RR^3$, נכתוב במפורש את $D^3 f\mid_p (e_i, e_j, e_k)$.

נבחן תחילה את $Df\mid_p(e_i)$, ונראה כי מתקיים
\[
	Df\mid_p =
	\begin{pmatrix}
		\partial_1 f_1 \mid_p & \hdots & \partial_d f_1 \mid_p \\
		\vdots & & \vdots \\
		\partial_1 f_m \mid_p & \hdots & \partial_d f_m \mid_p
	\end{pmatrix}
\]
ולכן נוכל להציב ולקבל
\[
	Df\mid_p(e_i) =
	\begin{pmatrix}
		\partial_1 f_1 \mid_p & \hdots & \partial_d f_1 \mid_p \\
		\vdots & & \vdots \\
		\partial_1 f_m \mid_p & \hdots & \partial_d f_m \mid_p
	\end{pmatrix}
	\cdot
	e_i
	=
	\begin{pmatrix}
		\partial_i f_1 \mid_p \\
		\vdots \\
		\partial_i f_m \mid_p
	\end{pmatrix}
	= \partial_i f\mid_p
\]
מחזרה על חישוב זה נקבל
\[
	D^3f \mid_p (e_i, e_j, e_k) = \partial_k \partial_j \partial_i f \mid_p
\]

\Question{}
\Subquestion{}
נחשב את פולינום טיילור של $f(x, y) = e^{-(x^2 + y^2)} \cos(xy)$ עד סדר $4$ סביב $(0, 0)$.

נבחין כי $e^{-(x^2 + y^2)} = e^{-t^2}$ כאשר $t = \lVert (x, y) \rVert$, מפיתוח טיילור של האקספוננט נקבל
\[
	e^{-t^2} = 1 - 0 t + \frac{1}{2} \cdot (-2) \cdot t^2 + \frac{1}{3!} \cdot 0 \cdot t^3 + \frac{1}{4!} \cdot (-8) t^4 + o(t^4)
	= 1 - t^2 - \frac{1}{2} t^4 + o(t^4)
\]
ולכן נסיק גם
\[
	e^{-(x^2+y^2)} = 1 - (x^2 + y^2) - \frac{1}{2} {(x^2 + y^2)}^2 + o(\lVert (x, y) \rVert^4)
\]
אנו יודעים כי
\[
	\cos(t) = 1 - \frac{1}{2} t^2 + \frac{1}{4!} t^4 + o(t^4),
	\qquad
	xy = xy + o(\lVert (x, y) \rVert^4)
\]
פיתוח טיילור סטנדרטי ופיתוח טיילור של פולינום.
נשתמש בהרכבת פולינומי טיילור ונקבל כי
\[
	\cos(xy) = 1 - \frac{x^2 y^2}{2} + o(\lVert (x, y) \rVert^4)
\]
ועתה נכפיל ונקבל
\begin{align*}
	f(x, y)
	& = (1 - (x^2 + y^2) - \frac{1}{2} {(x^2 + y^2)}^2)(1 - \frac{1}{2}(x^2y^2)) + o(\lVert (x, y) \rVert^4) \\
	& = 1 - (x^2 + y^2) - \frac{1}{2} {(x^2 + y^2)}^2 - \frac{1}{2}(x^2y^2) + o(\lVert (x, y) \rVert^4)
\end{align*}

\Subquestion{}
נמצא פיתוח טיילור של $f(x, y) = e^y \tan x$ עד סדר $3$ סביב $(0, \frac{1}{2})$.

נתחיל ונבחין כי
\[
	e^y = \sqrt{e} + \sqrt{e} (y - \frac{1}{2}) + \frac{\sqrt{e}}{2} {(y - \frac{1}{2})}^2 + \frac{\sqrt{e}}{3!} {(y - \frac{1}{2})}^3 + o(\lVert (x, y) \rVert^3)
\]
ונחשב ונקבל גם
\[
	\tan x = x + \frac{1}{3!} x^3 + o(\lVert (x, y) \rVert^3)
\]
לכן נוכל להסיק
\begin{align*}
	f(x, y)
	& = (\sqrt{e} + \sqrt{e} (y - \frac{1}{2}) + \frac{\sqrt{e}}{2} {(y - \frac{1}{2})}^2 + \frac{\sqrt{e}}{3!} {(y - \frac{1}{2})}^3)(x + \frac{1}{3!} x^3) + o(\lVert (x, y) \rVert^3) \\
	& = \sqrt{e} x + \frac{\sqrt{e}}{3!} x^3 + \sqrt{e} x (y - \frac{1}{2}) + o(\lVert (x, y) \rVert^3)
\end{align*}

\Subquestion{}
נחשב את פולינום טיילור של $f(x, y, z) = x^3 + 2z^2 - 3yz + 5z$ סביב $(0, 1, 2)$ עד סדר $3$.

נתחיל בסיבה לבחירת הסדר, היא כמובן על־פי מעלת הפולינום. \\*
נבחין כי פולינום טיילור של פולינום מתכנס לפולינום עצמו, ולכן נוכל לקבוע כי
\[
	f(x, y, z) = x^3 + 2z^2 - 3yz + 5z + o(\lVert (x, y, z) \rVert^3)
\]

\Question{}
יהיו $a, b, d \ne 0$, נמצא את הנקודה הקרובה ביותר לראשית שעל המישור $ax + by + z = d$.

נגדיר $f : \RR^2 \to \RR$ על־ידי
\[
	f(x, y) = d - ax - by
\]
לכן הנקודה הקרובה ביותר על המישור היא גם המינימום של $\lVert (x, y, f(x, y)) \rVert$, מינימום של פונקציה זו זהה למינימום של
\[
	g(x, y) = \lVert (x, y, f(x, y)) \rVert^2
\]
ולכן מספיק לחקור אותה.
תחילה נראה כי
\[
	g(x, y) = x^2 + y^2 + {(d - ax - by)}^2
\]
נחפש את נקודות הקיצון של $g$, אלה תופיענה בנקודות בהן הנגזרות החלקיות מתאפסות, לכן נחשב אותן
\[
	\nabla g (x, y) = (2x - 2a(d - ax - by), 2y - 2b(d - ax - by) )
\]
נבדוק התאפסות
\begin{align*}
	& (1 + a^2) x + ab y - ad = 0
	\iff x = \frac{ad - ab y}{1 + a^2}, \\
	& (1 + b^2) y - bd + ab \frac{ad - aby}{1 + a^2} = 0 \\
	& \iff (1 + b^2 - \frac{a^2b^2}{1 + a^2}) y = bd - \frac{a^2bd}{1 + a^2} \\
	& \iff y = \frac{bd - \frac{a^2bd}{1 + a^2}}{1 + b^2 - \frac{a^2b^2}{1 + a^2}} \\
	& \iff y = \frac{bd}{1 + b^2a + a^2b}
\end{align*}
ומתהליך מקביל והפוך נוכל לקבל גם
\[
	x = \frac{ad}{1 + a^2b + ab^2}
\]
ומצאנו נקודה יחידה חשודה להיות המינימום. \\*
נחשב את $D^2 f \mid_{(x, y)}$:
\[
	D^2 f = 
	\begin{pmatrix}
		2 + 2a^2 & 2ab \\
		2ab & 2 + 2b^2
	\end{pmatrix}
\]
נשתמש במשפט סילבסטר ונראה כי
\[
	2 + 2a^2 \ge 0,
	\quad
	(2 + 2a^2)(2 + 2b^2) - 4a^2b^2 = 4 + 4a^2 + 4b^2 \ge 0
\]
ולכן נקבל כי המטריצה חיובית לחלוטין ונסיק כי הנקודה שמצאנו אכן מינימום מקומי ויחיד ולכן מינימום.

\Question{}
נמצא ונסווג את הנקודות הקריטיות של הפונקציות הנתונות.

\Subquestion{}
\[
	f(x, y) = (x - 1)(x - 3)(y - 1)(y - 3)
\]
נחשב את הגרדיאנט ונקבל
\[
	\nabla f(x, y) = (2(x - 2)(y - 1)(y - 3), 2(x - 1)(x - 3)(y - 2))
\]
נבדוק איפוס ונקבל
\[
	\nabla f(x, y) = 0 \iff (x = 2, y \in \{1, 3\}) \cap (y = 2, x \in \{1, 3\}) = \emptyset
\]
לכן הנקודות החשודות בקיצון הן $\{ (2, 2), (1, 1), (1, 3), (3, 1), (3, 3) \}$. \\*
נחשב עתה את הנגזרת השנייה ונקבל
\[
	D^2 f \mid_{(x, y)}
	= \begin{pmatrix}
		2(y - 1)(y - 3) & 4(x - 2)(y - 2) \\
		4(x - 2)(y - 2) & 2(x - 1)(x - 3)
	\end{pmatrix}
\]
נציב ונשתמש במשפט סילבסטר ונקבל כי $(2, 2)$ מינימום, $(1, 3), (3, 1), (1, 1), (3, 3)$ נקודות אוכף.

\Subquestion{}
\[
	f(x, y) = xy \cdot \exp(-8(x^2 + y^2))
\]
נחשב
\[
	\nabla f(x, y) = (\exp(-8(x^2 + y^2))(y - 16x^2y), \exp(-8(x^2 + y^2)) (x - 16xy^2))
\]
נשווה לאפס ונקבל נקודות חשודות לקיצון הן $(0, 0), (\pm \frac{1}{4}, \pm \frac{1}{4})$. \\*
נחשב נגזרת שנייה ונקבל
\[
	D^2 f \mid_{(x, y)}
	= \begin{pmatrix}
		e^{-8(x^2 + y^2)} ( -32xy + {(y - 16x^2y)}^2) & e^{-8(x^2 + y^2)} (1 - 16x^2 + (y - 16x^2y)(x - 16xy^2))   \\
		e^{-8(x^2 + y^2)} (1 - 16x^2 + (y - 16x^2y)(x - 16xy^2))  & e^{-8(x^2 + y^2)} ( -32xy + {(x - 16y^2x)}^2)
	\end{pmatrix}
\]
נבחין כי רכיבי האקספוננט לא משפיעים על החיוביות ולכן נשתמש בבחינת התבנית הבילינארית
\[
	T
	= \begin{pmatrix}
		-32xy + {(y - 16x^2y)}^2 & 1 - 16x^2 + (y - 16x^2y)(x - 16xy^2)   \\
		1 - 16x^2 + (y - 16x^2y)(x - 16xy^2)  & -32xy + {(x - 16y^2x)}^2
	\end{pmatrix}
\]
על־ידי הצבה וחישוב נקבל כי $(0, 0)$ אוכף, $\pm(\frac{1}{4}, \frac{1}{4})$ מקסימום מקומי וכי $\pm(\frac{1}{4}, -\frac{1}{4})$ מינימום מקומי.

\Subquestion{}
\[
	f(x, y, z) = x^2 + y^2 + z^2 + xy + yz - 6x - 7y - 8z
\]
נחשב את הגרדיאנט
\[
	\nabla f(x, y, z) = (2x + y - 6, 2y + x + z - 7, 2z + y - 8)
\]
מפתרון מערכת המשוואות לאפס נקבל נקודה יחידה $(3, 0, 4)$. \\*
נמצא את הנגזרת בנקודה על־ידי גזירה כפולה ונקבל
\[
	D^2 f = \begin{pmatrix}
		2 & 1 & 0 \\
		1 & 2 & 1 \\
		0 & 1 & 2
	\end{pmatrix}
\]
מצאנו מטריצה המייצגת את הנגזרת השנייה בכל נקודה, ועל־ידי משפט סילבסטר נקבל ישירות כי המטריצה חיובית לחלוטין, ולכן $(3, 0, 4)$ מינימום מקומי ויחיד ועל־כן מינימום.

\end{document}
