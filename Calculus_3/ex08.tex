\documentclass[a4paper]{article}

% packages
\usepackage{inputenc, fontspec, amsmath, amsthm, amsfonts, polyglossia, catchfile}
\usepackage[a4paper, margin=50pt, includeheadfoot]{geometry} % set page margins

% style
\AddToHook{cmd/section/before}{\clearpage}	% Add line break before section
\linespread{1.5}
\setcounter{secnumdepth}{0}		% Remove default number tags from sections
\setmainfont{Libertinus Serif}
\setsansfont{Libertinus Sans}
\setmonofont{Libertinus Mono}
\setdefaultlanguage{hebrew}
\setotherlanguage{english}

% operators
\DeclareMathOperator\cis{cis}
\DeclareMathOperator\Sp{Sp}
\DeclareMathOperator\tr{tr}
\DeclareMathOperator\im{Im}
\DeclareMathOperator\diag{diag}
\DeclareMathOperator*\lowlim{\underline{lim}}
\DeclareMathOperator*\uplim{\overline{lim}}

% commands
\renewcommand\qedsymbol{\textbf{משל}}
\newcommand{\NN}[0]{\mathbb{N}}
\newcommand{\ZZ}[0]{\mathbb{Z}}
\newcommand{\QQ}[0]{\mathbb{Q}}
\newcommand{\RR}[0]{\mathbb{R}}
\newcommand{\CC}[0]{\mathbb{C}}
\newcommand{\getenv}[2][] {
  \CatchFileEdef{\temp}{"|kpsewhich --var-value #2"}{\endlinechar=-1}
  \if\relax\detokenize{#1}\relax\temp\else\let#1\temp\fi
}
\newcommand{\explain}[2] {
	\begin{flalign*}
		 && \text{#2} && \text{#1}
	\end{flalign*}
}

% headers
\getenv[\AUTHOR]{AUTHOR}
\author{\AUTHOR}
\date\today

\title{פתרון מטלה 08 --- חשבון אינפינטסמלי 3 (80415)}

\begin{document}
\maketitle
\maketitleprint{}

\Question{}
תהי $D$ קבוצת החמישיות הסדורות $(a, b, c, d, e) \in \RR^5$ שעבורן יש למשוואה $ax^4 + bx^3 + cx^2 + dx^1 + e = 0$ יש פתרון ממשי.

\Subquestion{}
נוכיח ש־$(1, 2, -4, 3, -2)$ נקודה פנימית של $D$.
\begin{proof}
	נגדיר $f : \RR^5 \to \RR$ על־ידי
	\[
		f(a, b, c, d, e, x) = ax^4 + bx^3 + cx^2 + dx^1 + e
	\]
	נראה כי $f(1, 2, -4, 3, -2, 1) = 0$ ונבדוק את הנגזרת בנקודה לפי $x$:
	\[
		\frac{\partial f}{\partial x} = 4ax^3 + 3bx^2 + 2cx + d
	\]
	ונקבל כי היעקוביאן (בגודל 1) בנקודה הוא $J(1, 2, -4, 3, -2, 1) = 4 + 6 - 8 + 6 = 8 \ne 0$. \\*
	לכן משפט הפונקציה הסתומה מתקיים וניתן להגדיר את $x$ על־ידי $(a, b, c, d, e)$ ונוכל להסיק כי נקודה זו פנימית ב־$D$.
\end{proof}

\Subquestion{}
נמצא נקודה ב־$D$ שאיננה פנימית.

אם נקודה לא פנימית אז בהכרח היעקוביאן בה מתאפס, ולכן נקבל
\[
	J(a, b, c, d, e, x) = 0
	\implies
	4x^3 + 3bx^2 + 2cx + d = 0
\]
נבנה נקודה כזו, נגדיר $x = 1, a = 1, b = 1, c = 1, d = -9$.
נקבל מהשוויון $f(1, 1, 1, -9, e, 1) = 0$ כי $e = 6$ וקיבלנו ערך קצה.

\Question{}
נגדיר
\[
	f(x, y, z, w) = (xz - yw + e^x, \sin(xy) + x^2 - y^2 + z^3 - w^3)
\]
ונגדיר את מערכת המשוואות
\[
	f_1(x, y, z, w) = 0,
	\qquad
	f_2(x, y, z, w) = 6
\]
נוכיח כי המערכת מגדירה את $z, w$ כפונקציות של $x, y$ בסביבה כלשהי של
\[
	(x_0, y_0, z_0, w_0) = (0, 1, 2, 1)
\]
ונחשב את הנגזרות $\frac{\partial w}{\partial x}, \frac{\partial w}{\partial y}, \frac{\partial z}{\partial x}$ בנקודה $(x_0, y_0) = (0, 1)$.
\begin{proof}
	נתחיל ונחשב ישירות כי $f(x_0, y_0, z_0, w_0) = (0, -6)$. \\*
	נבחין כי הפונקציה מוגדרת על־ידי רכיבים גזירים ולכן אף היא גזירה ונקבל
	\[
		\frac{\partial f}{\partial (z, w)}
		= \begin{pmatrix}
			x & -y \\
			3z^2 & -3w^2
		\end{pmatrix}
	\]
	נחשב את היעקוביאן ונקבל
	\[
		J = 3yz^2 - 3xw^2
	\]
	נציב את הנקודה ונקבל $J(0, 1, 2, 1) = 3(1 \cdot 2^2 - 0 \cdot 1^2) = 12 \ne 0$, ולכן משפט הפונקציה הסתומה חל ונקבל כי המערכת אכן מגדירה פונקציה של $z, w$ כתלות ב־$x, y$. \\*
	נעבור עתה לחישוב הנגזרות.
	\[
		\frac{\partial f_1}{\partial x} (x, y) = 0,
		\qquad
		\frac{\partial f_2}{\partial x} (x, y) = 0
	\]
	ולכן נקבל
	\[
		z + x \frac{\partial z}{\partial x} - y \frac{\partial w}{\partial x} + e^x = 0,
		\qquad
		y \sin(xy) + 2x + 3z^2 \frac{\partial z}{\partial x} - 3w^2 \frac{\partial w}{\partial x} = 0
	\]
	נציב $(0, 1)$ ונקבל כמובן
	\[
		2 - \frac{\partial w}{\partial x} + 1 = 0,
		\qquad
		12 \frac{\partial z}{\partial x} - 3 \frac{\partial w}{\partial x} = 0
	\]
	ולכן נקבל כי $\frac{\partial z}{\partial x} = \frac{3}{4}, \frac{\partial w}{\partial x} = 3$. \\*
	נגזור עתה על־פי $y$ ונקבל
	\[
		x \frac{\partial z}{\partial y} - w - y \frac{\partial w}{\partial y} = 0
		\implies
		0 - 1  - 1\frac{\partial w}{\partial y} = 0
		\implies
		\frac{\partial w}{\partial y} = -1
	\]
\end{proof}

\Question{}
תהי $f : \RR^3 \to \RR$ פונקציה גזירה ברציפות, ונניח ש־$f$ מתאפסת בראשית, ושכל הנזגרות החלקיות הראשונות בראשית לא מתאפסות.

\Subquestion{}
נוכיח שאפשר לחלץ כל אחד מהמשתנים $x, y, z$ כפונקציה של שני המשתנים האחרים בסביבת הראשית.
\begin{proof}
	ידוע לנו כי $\frac{\partial f}{\partial z}$ לא אפס, ולכן יחד עם כל ההנחות שעשינו משפט הפונקציה הסתומה חל (יחד עם היעקוביאן של איבר אחד של נגזרת זו) ונקבל כי $z$ ניתן לתיאור כפונקציה של $x, y$.
	נוכל כמובן לבצע אותו תהליך עבור שני המשתנים הנוספים וקיבלנו כי כל משתנה ניתן לביטוי על־ידי שני האחרים.
\end{proof}

\Subquestion{}
נוכיח שהחילוצים הללו מקיימים בראשית
\[
	\frac{\partial z}{\partial x}
	= - \frac{\partial z}{\partial y} \cdot \frac{\partial y}{\partial x}
\]
\begin{proof}
	אם נגדיר $f(x, y, z(x, y))$ אז בגזירה על־ידי כלל ההצבה יחד עם $g(x, y) = (x, y, z(x, y))$ נקבל
	\[
		\nabla f(x, y, z) = (\frac{\partial f}{\partial x}, \frac{\partial f}{\partial y}, \frac{\partial f}{\partial z}),
		Dg\mid_{(x, y)} = \begin{pmatrix}
			1 & 0 \\
			0 & 1 \\
			\frac{\partial z}{\partial x} & \frac{\partial z}{\partial y}
		\end{pmatrix}
	\]
	ולכן נקבל מהמשפט כי
	\[
		\nabla f \circ g \mid_{(x, y)}
		= (\frac{\partial f}{\partial x}, \frac{\partial f}{\partial y}, \frac{\partial f}{\partial z})
		\begin{pmatrix}
			1 & 0 \\
			0 & 1 \\
			\frac{\partial z}{\partial x} & \frac{\partial z}{\partial y}
		\end{pmatrix}
		= (\frac{\partial f}{\partial x} + \frac{\partial f}{\partial z} \cdot \frac{\partial z}{\partial x}, \frac{\partial f}{\partial y} + \frac{\partial f}{\partial z} \cdot \frac{\partial z}{\partial y})
	\]
	נוכל אם כן לקבל באופן דומה כי גם
	\[
		\nabla f(x, y(x, z), z)
		= (\frac{\partial f}{\partial x} + \frac{\partial f}{\partial y} \cdot \frac{\partial y}{\partial x}, \frac{\partial f}{\partial z} + \frac{\partial f}{\partial y} \cdot \frac{\partial y}{\partial z})
	\]
	וכמובן גם
	\[
		\nabla f(x(y, z), y, z)
		= (\frac{\partial f}{\partial y} + \frac{\partial f}{\partial x} \cdot \frac{\partial x}{\partial y}, \frac{\partial f}{\partial z} + \frac{\partial f}{\partial x} \cdot \frac{\partial x}{\partial z})
	\]
	ועל־ידי השוואת האגפים הימניים והשמאליים נוכל לקבל את הנוסחה שהתבקשנו למצוא.
\end{proof}

\Question{}
תהי $f : \RR^{d + 1} \to \RR$ פונקציה $C^2$ בסביבת $(a, b)$ כאשר $a \in \RR^d, b \in \RR$, וידוע כי $f(a, b) = 0$ ו־$\partial_{d + 1} f(a, b) \ne 0$. \\*
נוכיח כי החילוץ $g$ שמתקבל ממשפט הפונקציה הסתומה הוא $C^2$.
\begin{proof}
	נתחיל ונראה שעל־פי המשפט, $g$ היא $C^1$, ולכן עלינו להראות רק שנגזרותיה הן גם $C^1$. \\*
	נניח בשלילה כי $g$ לא $C^2$, לכן נוכל להסיק כי נגזרת שנייה של $g$ לא מתקיימת לפחות עבור נגזרת חלקית אחת. \\*
	אם כן, נבחר את הכיוון הזה ונקבל כי גם הנגזרת של הפונקציה $f$ בכיוון זה לא גזירה, שכן מצאנו כי $f$ ניתנת לכתיבה על־ידי $g$ זו באגפה ה־$d + 1$. \\*
	קיבלנו סתירה כמובן ל־$f \in C^2$ ולכן נוכל להסיק כי $g$ עצמה היא $C^2$.
\end{proof}

\Question{}
תהי $A = [a_1, b_1] \times \cdots \times [a_d, b_d] \subseteq \RR^d$ תיבה, ו־$f : A \to \RR$ פונקציה חסומה.

\Subquestion{}
נניח ש־$f$ אינטגרבילית. יהי $\varepsilon > 0$. \\*
נוכיח שקיים $\delta > 0$ כך שלכל חלוקה מתאימה $P = (P_1, \dots, P_d)$ שעבורה הפרמטר של כל $P_i$ קטן מ־$\delta$, \\*
ושאם נסמן $C_1, \dots, C_k$ את התיבות המתקבלות אז לכל בחירת נקודות $p_1 \in C_1, \dots, p_k \in C_k$ מתקיים
\[
	\left\lvert \sum_{i = 1}^{k} f(p_i) \cdot \text{vol}(C_i) - \int_A f \right\rvert < \varepsilon
\]
\begin{proof}
	\[
		\forall \delta > 0, \forall P, \lambda(P) < \delta \implies \forall p_i \in P_i : f(p_i) \le \sup_{p \in P_i} (p), \text{vol}(C_i) \le \delta \text{vol}(C_\lambda)
		\implies \sum_{i = 1}^{k} f(p_i) \text{vol}(C_i) \le \delta \sum_{i = 1}^{k} M_i |C_i|
	\]
	נקבל אם כן כי
	\[
		\left\lvert \sum_{i = 1}^{k} f(p_i) \cdot \text{vol}(C_i) - \int_A f \right\rvert \le
		\left\lvert \delta \overline{S}(f, P) - \int_A f \right\rvert < \varepsilon
	\]
	וכמובן מהגדרת האינטגרל קיים $\delta > 0$ עבורו אי־שוויון זה מתקיים.
\end{proof}

\Subquestion{}
יהיו $\varepsilon > 0$ ו־$I \in \RR$. נניח שיש חלוקה $P$ של $A$ עם תיבות $C_1, \dots, C_k$ כך שלכל בחירת נקודות $p_i \in C_i$ לכל $0 < i \le k$ מתקיים
\[
	\left\lvert \sum_{i = 1}^{k} f(p_i) \cdot \text{vol}(C_i) - I \right\rvert < \varepsilon
\]
נוכיח כי $I - \varepsilon \le \underline{\int_A} f \le \overline{\int_A} f \le I + \varepsilon$.
\begin{proof}
	מצאנו קודם כי
	\[
		\left\lvert \sum_{i = 1}^{k} f(p_i) \cdot \text{vol}(C_i) - I \right\rvert \le
		\left\lvert \overline{S}(f, P) - I \right\rvert < \varepsilon
	\]
	ולכן נוכל להסיק כי גם
	\[
		|\underline{S}(f, P) - I| < \varepsilon
	\]
	וכמובן על־ידי חיבור המשוואות וחוקי ערך מוחלט נוכל לקבל כי הטענה נכונה.
\end{proof}

\end{document}
