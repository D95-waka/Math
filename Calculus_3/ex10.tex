\documentclass[a4paper]{article}

% packages
\usepackage{inputenc, fontspec, amsmath, amsthm, amsfonts, polyglossia, catchfile}
\usepackage[a4paper, margin=50pt, includeheadfoot]{geometry} % set page margins

% style
\AddToHook{cmd/section/before}{\clearpage}	% Add line break before section
\linespread{1.5}
\setcounter{secnumdepth}{0}		% Remove default number tags from sections
\setmainfont{Libertinus Serif}
\setsansfont{Libertinus Sans}
\setmonofont{Libertinus Mono}
\setdefaultlanguage{hebrew}
\setotherlanguage{english}

% operators
\DeclareMathOperator\cis{cis}
\DeclareMathOperator\Sp{Sp}
\DeclareMathOperator\tr{tr}
\DeclareMathOperator\im{Im}
\DeclareMathOperator\diag{diag}
\DeclareMathOperator*\lowlim{\underline{lim}}
\DeclareMathOperator*\uplim{\overline{lim}}

% commands
\renewcommand\qedsymbol{\textbf{משל}}
\newcommand{\NN}[0]{\mathbb{N}}
\newcommand{\ZZ}[0]{\mathbb{Z}}
\newcommand{\QQ}[0]{\mathbb{Q}}
\newcommand{\RR}[0]{\mathbb{R}}
\newcommand{\CC}[0]{\mathbb{C}}
\newcommand{\getenv}[2][] {
  \CatchFileEdef{\temp}{"|kpsewhich --var-value #2"}{\endlinechar=-1}
  \if\relax\detokenize{#1}\relax\temp\else\let#1\temp\fi
}
\newcommand{\explain}[2] {
	\begin{flalign*}
		 && \text{#2} && \text{#1}
	\end{flalign*}
}

% headers
\getenv[\AUTHOR]{AUTHOR}
\author{\AUTHOR}
\date\today

\title{פתרון מטלה 10 --- חשבון אינפינטסמלי 3 (80415)}

\begin{document}
\maketitle
\maketitleprint{}

\Question{}
יהי $A \subset \RR^2$ התחום החסום על־ידי העקומים $x^2 - y^2 = 1, y = 0, y = x, xy = 1$ ברביע הראשון. \\*
נחשב את $\int_A (x^2 + y^2)\ dx\ dy$ על־ידי החלפת המשתנים $u = xy, v = x^2 - y^2$.

נחשב את היעקוביאן עבור החלפת המשתנים ונקבל
\[
	J = \frac{\partial(u, v)}{\partial(x, y)}
	= \begin{vmatrix}
		y & x \\
		2x & -2y
	\end{vmatrix}
	= -2(x + y)
\]
ולכן $|J^{-1}| = 2{(x + y)}^2$.
נקבל אם כן $f(x, y) = x^2 + y^2 = \frac{1}{2} |J|$. \\*
עוד נבחין כי $x^2 - y^2 = 1 \iff v = 1, y = 0 \iff u = 0, y = x = 0 \iff v = 0, xy = u = 1$ ולכן נקבל ממשפט החלפת משתנים
\[
	\int_A (x^2, y^2)\ dx\ dy
	= \int_0^1 \left( \int_0^1 \frac{1}{2}\ du \right)\ dv
	= \frac{1}{2}
\]

\Question{}
נגדיר
\[
	f(x, y) = \frac{y^2 - x^2}{{(x^2 + y^2)}^2}
\]

\Subquestion{}
נראה ש־$f = \frac{\partial}{\partial x}( \frac{x}{x^2 + y^2})$.
\begin{proof}
	נחשב
	\[
		\frac{\partial}{\partial x}( \frac{x}{x^2 + y^2})
		= \frac{1 \cdot (x^2 + y^2) - x (2x + 0)}{{(x^2 + y^2)}^2}
		= \frac{-x^2 + y^2}{{(x^2 + y^2)}^2}
		= f(x, y)
	\]
\end{proof}

\Subquestion{}
נחשב את $\int_0^1( \int_0^1 f(x, y)\ dx)\ dy$ ואת $\int_0^1( \int_0^1 f(x, y)\ dy)\ dx$.

נשתמש בתוצאת הסעיף הקודם ונקבל
\[
	\int_0^1 \left( \int_0^1 f(x, y)\ dx\right)\ dy
	= \int_0^1 \left( \left. \frac{x}{x^2 + y^2} \right|_{x = 0}^{x = 1} \right)\ dy
	= \int_0^1 \left( \frac{1}{1 + y^2} - 0 \right)\ dy
	= \arctan(y) \big|_0^1
	= \frac{\pi}{4}
\]
ניתן לראות כי
\[
	f = \frac{\partial}{\partial y}( \frac{-y}{x^2 + y^2})
\]
באותו האופן בו ראינו בסעיף א' את השוויון הדומה עבור $x$, ונקבל
\[
	\int_0^1 \left( \int_0^1 f(x, y)\ dy\right)\ dx
	= \int_0^1 \left( \left. \frac{-y}{x^2 + y^2} \right|_{y = 0}^{y = 1} \right)\ dx
	= \int_0^1 \left( \frac{-1}{1 + x^2} - 0 \right)\ dx
	= -\arctan(x) \big|_0^1
	= -\frac{\pi}{4}
\]

\Subquestion{}
נשים לב כי תוצאת שני האינטגרלים היא שונה, לכאורה בסתירה למשפט פוביני, נסביר את מקור הבעיה.

הפתרון הוא ש־$f$ כלל לא חסומה בתחום הנתון, כאשר $f(0, 0)$ לא מוגדר והפונקציה לא שואפת לערך קבוע בנקודה זו, ולכן תנאי המשפט כלל לא חלים.

\Question{}
נחשב את האינטגרל של $f(x, y, z, w) = (x^2 + y^2) {(z^2 + w^2)}^3$ בתחום $\overline{B}(0, 3) \subseteq \RR^4$

נגדיר $r = \sqrt{x^2 + y^2}, \theta = \arctan(\frac{y}{x})$ קורדינטות פולריות ו־$s = \sqrt{z^2 + w^2}, \phi = \arctan(\frac{w}{z})$ קורדינטות קוטביות בלתי תלויות. \\*
ראינו בתרגול כי החלפת משתנים זו עומדת במבחן החלפת משתנים, ולכן נחשב אותה
\begin{align*}
	J
	& = \left\lvert \frac{\partial(r, \theta, s, \varphi)}{\partial(x, y, z, w)} \right\rvert \\
	& = \begin{vmatrix}
		\frac{x}{\sqrt{x^2 + y^2}} & \frac{y}{\sqrt{x^2 + y^2}} & 0 & 0 \\
		\frac{-y}{x^2 + y^2} & \frac{x}{x^2 + y^2} & 0 & 0 \\
		0 & 0 & \frac{z}{\sqrt{z^2 + w^2}} & \frac{w}{\sqrt{z^2 + w^2}} \\
		0 & 0 & \frac{-w}{z^2 + w^2} & \frac{z}{z^2 + w^2}
	\end{vmatrix} \\
	& = \frac{x^2 + y^2}{(x^2 + y^2) \sqrt{x^2 + y^2}} \cdot \frac{z^2 + w^2}{(z^2 + w^2) \sqrt{z^2 + w^2}} \\
	& = \frac{1}{\sqrt{x^2 + y^2} \sqrt{z^2 + w^2}}
\end{align*}
ולכן $|J^{-1}| = J^{-1} = \sqrt{(x^2 + y^2)(z^2 + w^2)}$. \\*
עתה נקבל
\[
	f(x, y, z, w) = J^{-1} \cdot r s^5
\]
ולכן גם
\begin{align*}
	\int_{\overline{B}(0, 3)} f
	& = \int_{0}^{2 \pi} \left( \int_{0}^{3} \left( \int_0^{2 \pi} \left( \int_0^3 r s^5\ ds \right)\ d\varphi \right)\ dr \right)\ d\theta \\
	& = \left( \int_{0}^{2 \pi} 1\ d\theta\right) \cdot \left( \int_{0}^{2 \pi} 1\ d\varphi \right) \cdot \int_{0}^{3} \left( \int_0^3 r s^5\ ds \right)\ dr \\
	& = 4\pi^2 \cdot \int_0^3 \frac{1}{6} (3^6 r - 0)\ dr \\
	& = 2 \cdot 3^5 \pi^2 \cdot \int_0^3 r dr \\
	& = 3^7 \pi^2
\end{align*}

\Question{}
תהי $A \in \RR^{3 \times 3}$ מטריצה, ויהי $B \subset \RR^3$ כדור היחידה.
לכל $v = {(x, y, z)}^t$ נחשב
\[
	\iiint_B \langle A v, v \rangle\ dx\ dy\ dz
\]

נגדיר את פונקציית ההמרה לקורדינטה כדורית
\[
	g(r, \theta, \varphi) = (r \cos \theta \sin \varphi, r \sin \theta \sin \varphi, r \cos \varphi)
\]
ומצאנו בתרגול כי $J_g(r, \theta, \varphi) = r^2 \sin \varphi$ ונבחין כי ביטוי זה תמיד חיובי עבור תחומי המשתנים. \\*
אנו יודעים כי
\[
	f(x, y, z) := \langle A v, v \rangle = a_{11} x^2 + a_{22} y^2 + a_{33} z^2 + (a_{21} + a_{12}) xy + (a_{31} + a_{13}) xz + (a_{23} + a_{32}) yz
\]
ולכן
\[
	\iiint_B f(x, y, z)\ dx\ dy\ dz
	= \iiint_{g^{-1}(B)} f(g(r, \theta, \varphi)) J_g\ dr\ d\theta\ d\varphi
\]
נציב ונקבל את הביטוי
\begin{align*}
	\iiint_{\substack{0 \le r \le 1, \\ 0 \le \theta \le 2\pi, \\ 0 \le \varphi \le \pi}}
	& (a_{11} {(r \cos(\theta) \sin(\varphi))}^2 + a_{22} {(r \sin(\theta) \sin(\varphi))}^2 + a_{33} {(r \cos(\varphi))}^2 \\
	& + (a_{21} + a_{12}) (r \cos(\theta) \sin(\varphi))(r \sin(\theta) \sin(\varphi)) + (a_{31} + a_{13}) (r \cos(\theta) \sin(\varphi))(r \cos(\varphi)) \\
	& + (a_{23} + a_{32}) (r \sin(\theta) \sin(\varphi))(r \cos(\varphi))) r^2 \sin(\varphi)\ dr\ d\theta\ d\varphi
\end{align*}
נכנס איברים ונקבל
\begin{align*}
	\iiint_{\substack{0 \le r \le 1, \\ 0 \le \theta \le 2\pi, \\ 0 \le \varphi \le \pi}}
	& a_{11} (r^4 \cos^2(\theta)\sin^3(\varphi)) + a_{22} (r^4 \sin^2(\theta)\sin^3(\varphi)) + a_{33} (r^4 \cos^2(\varphi) \sin(\varphi)) \\
	& + \frac{1}{2} (a_{21} + a_{12}) (r^4 \sin(2\theta) \sin^3(\varphi)) + (a_{31} + a_{13}) (r^4 \cos(\theta) \sin^2(\varphi) \cos(\varphi)) \\
	& + (a_{23} + a_{32}) (r^4 \sin(\theta) \sin^2(\varphi) \cos(\varphi))\ dr\ d\theta\ d\varphi
\end{align*}
עתה נבחין כי $\int_0^1 r^4\ dr = \frac{1}{5}$ ולכן האינטגרל הנתון שווה לביטוי
\begin{align*}
	\frac{1}{5} \iint_{\substack{0 \le \theta \le 2\pi, \\ 0 \le \varphi \le \pi}}
	& a_{11} (\cos^2(\theta)\sin^3(\varphi)) + a_{22} (\sin^2(\theta)\sin^3(\varphi)) + a_{33} (\cos^2(\varphi) \sin(\varphi)) \\
	& + \frac{1}{2} (a_{21} + a_{12}) (\sin(2\theta) \sin^3(\varphi)) + (a_{31} + a_{13}) (\cos(\theta) \sin^2(\varphi) \cos(\varphi)) \\
	& + (a_{23} + a_{32}) (\sin(\theta) \sin^2(\varphi) \cos(\varphi))\ d\theta\ d\varphi
\end{align*}
נחשב את המכפלות שקשורות ל־$\theta$, נחשב כל מקרה לגופו ונבחין לפני זה כי
\[
	\int \cos^2(\theta)\ \theta = \sin(2\theta) + \frac{x}{2},
	\qquad
	\int \sin^2(\theta)\ \theta = -\frac{1}{4} \sin(2\theta) + \frac{x}{2}
\]
ולכן גם
\[
	\int_0^{2\pi} \cos^2(\theta)\ \theta = \pi,
	\qquad
	\int_0^{2\pi} \sin^2(\theta)\ \theta = \pi
\]
ונקבל
\[
	\frac{1}{5} \int_{0 \le \varphi \le \pi} (\pi a_{11} (\sin^3(\varphi)) + \pi a_{22} (\sin^3(\varphi)) + 2\pi a_{33} (\cos^2(\varphi) \sin(\varphi)) + 0)\ d\varphi
\]
עתה נבחין כי
\[
	\int_0^\pi \sin^3(\varphi)\ d\varphi = \frac{4}{3}
\]
ולכן נקבל כי ערך האינטגרל הוא
\[
	\frac{\pi}{5} (\frac{4}{3} a_{11} + \frac{4}{3} a_{22} - \frac{2}{3} a_{33})
\]

\Question{}
מומנט אינרציה מוגדר לגוף בתחום $V \subseteq \RR^3$ ושצפיפותו היא $\rho : V \to \RR^+$ על־ידי
\[
	I_z = \iiint_V (x^2 + y^2) \rho(x, y, z)\ dx\ dy\ dz
\]
נוכיח שעבור גוף $V = \{ (x, y, z) \in \RR^3 \mid R_1^2 \le x^2 + y^2 + z^2 \le R_2^2 \}$ בעל צפיפות אחידה, מתקיים $I_z = \frac{2M}{5} \cdot \frac{R_2^5 - R_1^5}{R_2^3 - R_1^3}$.
\begin{proof}
	נעבור לקורדינטות כדוריות ונקבל
	\begin{align*}
		I_z & = \iiint_{\substack{R_1 \le r \le R_2, \\ 0 \le \theta \le 2\pi, \\ 0 \le \varphi \le \pi}} (r^4 \cos^2(\theta)\sin^3(\varphi) + r^4 \sin^2(\theta) \sin^3(\varphi))\ dr\ d\theta\ d\varphi \\
			& = \left( \int_{R_1}^{R_2} r^4\ dr \right) \left( \int_0^{2\pi} (\cos^2 \theta + \sin^2 \theta)\ d\theta \right) \left( \int_0^\pi \sin^3(\varphi)\ d\varphi \right) \\
			& = \frac{1}{5} \left( R_2^5 - R_1^5 \right) \left( 2\pi \right) \left( \frac{4}{3} \right) \\
			& = \frac{4 \pi}{15} \left( R_2^5 - R_1^5 \right)
	\end{align*}
	נעבור עתה לחישוב המסה של הקליפה הכדורית
	\[
		\iiint_V \rho(x, y, z)\ dx\ dy\ dz
		= \left( \int_{R_1}^{R_2} r^2\ dr \right) \left( \int_0^{2\pi} 1\ d\theta \right) \left( \int_0^\pi \sin \varphi\ d\varphi \right)
		= \frac{R_2^3 - R_1^3}{3} \cdot 2\pi \cdot 1
	\]
	ולכן נקבל
	\[
		I_z
		= \frac{2 \pi}{15} \left( R_2^5 - R_1^5 \right) \cdot M \frac{3}{2\pi(R_2^3 - R_1^3)}
		= \frac{2M}{5} \cdot \frac{R_2^5 - R_1^5}{(R_2^3 - R_1^3)}
	\]
\end{proof}

\end{document}
