\documentclass[a4paper]{article}

% packages
\usepackage{inputenc, fontspec, amsmath, amsthm, amsfonts, polyglossia, catchfile}
\usepackage[a4paper, margin=50pt, includeheadfoot]{geometry} % set page margins

% style
\AddToHook{cmd/section/before}{\clearpage}	% Add line break before section
\linespread{1.5}
\setcounter{secnumdepth}{0}		% Remove default number tags from sections
\setmainfont{Libertinus Serif}
\setsansfont{Libertinus Sans}
\setmonofont{Libertinus Mono}
\setdefaultlanguage{hebrew}
\setotherlanguage{english}

% operators
\DeclareMathOperator\cis{cis}
\DeclareMathOperator\Sp{Sp}
\DeclareMathOperator\tr{tr}
\DeclareMathOperator\im{Im}
\DeclareMathOperator\diag{diag}
\DeclareMathOperator*\lowlim{\underline{lim}}
\DeclareMathOperator*\uplim{\overline{lim}}

% commands
\renewcommand\qedsymbol{\textbf{משל}}
\newcommand{\NN}[0]{\mathbb{N}}
\newcommand{\ZZ}[0]{\mathbb{Z}}
\newcommand{\QQ}[0]{\mathbb{Q}}
\newcommand{\RR}[0]{\mathbb{R}}
\newcommand{\CC}[0]{\mathbb{C}}
\newcommand{\getenv}[2][] {
  \CatchFileEdef{\temp}{"|kpsewhich --var-value #2"}{\endlinechar=-1}
  \if\relax\detokenize{#1}\relax\temp\else\let#1\temp\fi
}
\newcommand{\explain}[2] {
	\begin{flalign*}
		 && \text{#2} && \text{#1}
	\end{flalign*}
}

% headers
\getenv[\AUTHOR]{AUTHOR}
\author{\AUTHOR}
\date\today

\title{פתרון מטלה 09 --- חשבון אינפינטסמלי 3 (80415)}

\begin{document}
\maketitle
\maketitleprint{}

\Question{}
תהי $S \subseteq \RR^d$ קבוצה בעלת נפח, ו־$f : S \to \RR$ חסומה כך ש־$f(x) = 0$ לכל $x \in S^\circ$. נוכיח ש־$\int_S f = 0$.
\begin{proof}
	ממשפט מההרצאה נסיק כי לשפה יש מידת לבג 0 ולכן נוכל לבנות חלוקה $P$ כך ש־$A = \{ A_1, \dots, A_k, \dots, A_n \}$ כך ש־$k$ התיבות הראשונות מכילות את השפה ושאר התיבות מכילות את כל הפנים (על־פי הלמה שנלמדה על השפה).
	עתה נקבל מהלמה כי לכל $\varepsilon > 0$ שנבחר $\sum_{i = 1}^{k} |A_i| < \varepsilon$, וכמובן $f(A_i) = 0$ לכל $i > k$ ומכאן נוכל להסיק ישירות כי הפונקציה אינטגרבילית ונפחה מתכנס לאפס.
\end{proof}

\Question{}
תהי $S \subseteq \RR^d$ קבוצה בעלת נפח ו־$f : S \to \RR$ פונקציה רציפה וחסומה. נראה ש־$f$ אינטגרבילית ב־$S$.
\begin{proof}
	נבחר תיבה $A \supseteq S$, ונגדיר עליה חלוקה $P$ על־פי הפונקציה המאפיינת $1_S$. \\*
	לכן נוכל להגדיר קבוצת תיבות $C = \{ C_1, \dots, C_k, \dots, C_m \}$ כך ש־$C_i \subseteq S$ לכל $1 \le i \le k$, ונוכל להסיק כי $1_S(C_i) = 0$ לכל $i > k$. \\*
	עבור התיבות החיצוניות עידון לא ישנה את העובדה שהפונקציה מקבל אפס עבורן, ולכן נבחן את העידון של התיבות $\{ C_1, \dots, C_k \}$ בלבד. \\*
	מהרציפות בתחום נוכל להסיק כי התמונה ההפוכה של כל נקודה היא קבוצה פתוחה, נוכל לחסום כל קבוצה כזו בתיבה ולהשתמש בקבוצת התיבות הזו כדי לעדן את החלוקה ונקבל התכנסות.
\end{proof}

\directlua{ Q_number = 4; }
\Question{}
נחשב את נפח הקבוצות הבאות.

\Subquestion{}
\[
	V = \left\{ (x, y, z) \in \RR^3 \mid \sqrt{\frac{x^2}{2} + \frac{y^2}{3}} \le z \le 6 - \frac{x^2}{2} - \frac{y^2}{3} \right\}
\]

נקבל
\[
	\text{Vol}(V) = \int_S 1_S\ dx\ dy\ dz
	= \int_{\sqrt{\frac{x^2}{2} + \frac{y^2}{3}} \le 6 - \frac{x^2}{2} - \frac{y^2}{3}} \left( \int_{\sqrt{\frac{x^2}{2} + \frac{y^2}{3}}}^{6 - \frac{x^2}{2} - \frac{y^2}{3}} 1 dz \right) dx\ dy
	= \int_0^6 \left(\int_{\sqrt{\frac{x^2}{2} + \frac{y^2}{3}} \le 6 - \frac{x^2}{2} - \frac{y^2}{3}} 1_S\ dx\ dy\right)\ dz
\]
נקבל באינטגרל הפנימי שטח אליפסה (הסכום המחוסר מתאפס) ולכן נסיק כי האינטגרל שווה לביטוי
\[
	\int_0^6 \pi \cdot 2 \cdot 3\ dz = 36 \pi
\]

\Subquestion{}
\[
	D = \{(x_1, \dots, x_d) \mid 0 \le x_1 \le x_2 \le \cdots \le x_d \le M \}
\]
פירמידה כאשר $M$ קבוע חיובי.

נחשב ונקבל
\[
	S_D
	= \int_D 1\ dx
	= \int_{0 \le x_1 \le \dots \le x_{d - 1} \le M} \left( \int_{x_{d - 1}}^M 1\ dx_d \right)\ dx_1 \dots dx_{d - 1}
	= \int_{0 \le x_1 \le \dots \le x_{d - 1} \le M} \left(M - x_{d - 1}\right)\ dx_1 \dots dx_{d - 1}
\]
נוכל אם כן לבצע את אותו התהליך שוב ולקבל
\[
	\int_{0 \le x_1 \le \dots \le x_{d - 1} \le M} \left(M - x_{d - 1}\right)\ dx_1 \dots dx_{d - 1}
	= \int_{0 \le x_1 \le \dots \le x_{d - 2} \le M} \left(\int_{x_{d - 2}}^M (M - x_{d - 1})\ dx_{d - 1} \right)\ dx_1 \dots dx_{d - 2}
\]
ערך האינטגרל הפנימי הוא $M x^{d - 1} - \frac{1}{2} x_{d - 1}^2 \mid_{x_{d - 2}}^M = \frac{M^2}{2} - M x_{d - 2} + \frac{x_{d - 2}^2}{2} = \frac{1}{2} {(M - x_{d - 2})}^2$ ונקבל
\[
	\int_{0 \le x_1 \le \dots \le x_{d - 2} \le M} \left( \frac{1}{2} {(M - x_{d - 2})}^2 \right)\ dx_1 \dots dx_{d - 2}
\]
ומהפעלת התהליך הזה בדיוק $d - 2$ פעמים נוספות נקבל
\[
	S_D = \frac{1}{d} M^d
\]

\Question{}
\Subquestion{}
תהי $f : \RR^3 \to \RR$ רציפה, נשנה את סדר האינטגרציה של
\[
	\int_{0}^{1} \left( \int_{0}^{1 - x} \left( \int_{0}^{x + y} f(x, y, z)\ dz \right)\ dy\right)\ dx
\]
להיות על־פי הסדר $dy\ dx\ dz$.

נתחיל בהחלפה של סדר שני האינטגרלים הפנימיים, נבדוק את התחומים ונקבל $0 \le z \le x + y, 0 \le y \le 1 - x$.
משרשור אי־השוויונות נקבל $0 \le z \le x + y \le x + 1 - x = 1$ ונסיק $0 \le z \le 1$, וכמו־כן $0 \le z \le x + y \implies 0 \le z + x \le y \le 1 - x$ ונקבל את האינטגרל
\[
	\int_{0}^{1} \left( \int_{0}^{1} \left( \int_{\max(0, z - x)}^{1 - x} f(x, y, z)\ dy \right)\ dz\right)\ dx
\]
מצאנו כי התחום של $z$ ושל $x$ לא תלויים ולכן נסיק כי האינטגרל שקול לביטוי
\[
	\int_{0}^{1} \left( \int_{0}^{1} \left( \int_{\max(0, z - x)}^{1 - x} f(x, y, z)\ dy \right)\ dx\right)\ dz
\]

\Subquestion{}
תהי $g : \RR \to \RR$ רציפה, ונוכיח את נוסחת קושי לאינטגרל משנה:
\[
	\int_{0}^{M}  \int_{0}^{z_1} \int_{0}^{z_2} \dots \int_{0}^{z_{n - 1}} g(z_n)\ dz_n \dots dz_1 = \frac{1}{(n - 1)!} \int_{0}^{M} {(M - t)}^{n - 1} g(t)\ dt
\]
\begin{proof}
	מהחלפת משתנים נקבל עבור שני האינטגרלים הפנימיים
	\[
		\int_{0}^{z_{n - 2}} \int_{0}^{z_{n - 1}} g(z_n)\ dz_n\ d_{z - 1}
		= \int_{0}^{z_{n - 1}} \int_{0}^{z_{n - 2}} g(z_n)\ d_{z - 1}\ dz_n
		= \int_{0}^{z_{n - 1}} (z_{n - 2} - 0) g(z_n)\ dz_n
	\]
	אם נבחן את תוצאת אינטגרל זה תחת האינטגרל בו הם מוכלים נקבל באותו הליך
	\[
		\int_{0}^{z_{n - 3}} \int_{0}^{z_{n - 1}} z_{n - 2} g(z_n)\ dz_n\ dz_{n - 2}
		= \int_{0}^{z_{n - 1}} \int_{0}^{z_{n - 3}} z_{n - 2} g(z_n)\ dz_{n - 2}\ dz_n
		= \frac{1}{2} \int_{0}^{z_{n - 1}} z_{n - 2}^2 g(z_n)\ dz_n
	\]
	נבצע את התהליך הזה שוב ושוב עד שנגיע לאינטגרל החיצוני ונקבל את הביטוי
	\begin{align*}
		\frac{1}{(n - 2)!} \int_{0}^{M} \int_{0}^{z_{n - 1}} z_1^{n - 2} g(z_n)\ dz_n\ dz_1
		& = \frac{1}{(n - 2)!} \int_{0}^{M} \int_{0}^{M - z_n} z_1^{n - 2} g(z_n)\ dz_1\ dz_n \\
		& = \frac{1}{(n - 1)!} \int_{0}^{M} {(M - z_n)}^{n - 1} g(z_n)\ dz_n
	\end{align*}
\end{proof}

\Question{}
נמצא באילו פרוסות אבטיח הנפרס לגובהו יש יותר קליפה. \\*
נניח כי נפח הקליפה הוא $S = \{ p \in \RR^3 \mid R_1 \le \lVert p \rVert \le R_2 \}$ עבור $0 < R_1, R_2$ כלשהם,
והפרוסות הן מהצורה $\{ (x, y, z) \in S \mid a \le z \le b \}$ כאשר $[a, b] \subseteq [-R_1, R_1]$ ועובי הפרוסות $b - a$ אחיד.

נבחין כי פרוסה אחת של האבטיח היא הנפח של $a \le z \le b$ של $S$ עצמו, ולכן נסיק
\[
	\text{Vol}(C_{\text{rust}}) = \int_{a}^{b} \int_S 1\ dx\ dy\ dz
\]
נשים לב כי עבור $z$ קבוע $S$ היא שני עיגולים נחתכים מהצורה
\[
	x^2 + y^2 = R_i^2 - z^2
\]
ואנו יודעים כי שטח עיגול הוא $\pi r^2$ ולכן נסיק כי שטח הפרוסה לפי ציר $z$ הוא $\pi(R_2^2 - z^2 - R_1^2 + z^2) = \pi(R_2^2 - R_1^2)$. \\*
נקבל אם כן שהנפח הוא
\[
	\int_{a}^{b} \pi(R_2^2 - R_1^2)\ dz = (b - a)(R_2^2 - R_1^2)
\]
ולכן נסיק שנפח הקליפה הוא זהה לכל פרוסה של אבטיח, תוצאה לא הגיונית שאני לא מקבל.

\end{document}
