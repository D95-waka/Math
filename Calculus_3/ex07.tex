\documentclass[a4paper]{article}

% packages
\usepackage{inputenc, amsmath, amsthm, thmtools, amsfonts, amssymb, luacode, catchfile, tikzducks, hyperref}
\usepackage[a4paper, margin=50pt, includeheadfoot]{geometry} % set page margins
\usepackage[shortlabels]{enumitem}
\usepackage[skip=3pt, indent=0pt]{parskip}

% language
\usepackage[bidi=basic, layout=tabular, provide=*]{babel}
\babelprovide[main, import]{hebrew}
\babelprovide{rl}
\babelfont{rm}{Libertinus Serif}
\babelfont{sf}{Libertinus Sans}
\babelfont{tt}{Libertinus Mono}

% style
\AddToHook{cmd/section/before}{\clearpage}	% Add line break before section
\linespread{1.3}
\setcounter{secnumdepth}{0}		% Remove default number tags from sections, this won't do well with theorems
\AtBeginDocument{\setlength{\belowdisplayskip}{3pt}}
\AtBeginDocument{\setlength{\abovedisplayskip}{3pt}}

% operators
\DeclareMathOperator\cis{cis}
\DeclareMathOperator\Sp{Sp}
\DeclareMathOperator\tr{tr}
\DeclareMathOperator\im{Im}
\DeclareMathOperator\re{Re}
\DeclareMathOperator\diag{diag}
\DeclareMathOperator*\lowlim{\underline{lim}}
\DeclareMathOperator*\uplim{\overline{lim}}
\DeclareMathOperator\rng{rng}
\DeclareMathOperator\Sym{Sym}
\DeclareMathOperator\Arg{Arg}
\DeclareMathOperator\Log{Log}
\DeclareMathOperator\dom{dom}

% commands
%\renewcommand\qedsymbol{\textbf{מש''ל}}
%\renewcommand\qedsymbol{\fbox{\emoji{lizard}}}
\newcommand{\NN}[0]{\mathbb{N}}
\newcommand{\ZZ}[0]{\mathbb{Z}}
\newcommand{\QQ}[0]{\mathbb{Q}}
\newcommand{\RR}[0]{\mathbb{R}}
\newcommand{\CC}[0]{\mathbb{C}}
\newcommand{\FF}[0]{\mathbb{F}}
\newcommand{\PP}[0]{\mathbb{P}}
\newcommand{\TT}[0]{\mathbb{T}}
\newcommand{\acts}[0]{\circlearrowright}
\newcommand{\explain}[2] {
	\begin{flalign*}
		 && \text{#2} && \text{#1}
	\end{flalign*}
}
\newcommand{\maketitleprint}[0]{ \begin{center}
	\begin{tikzpicture}[scale=3]
		\duck[graduate=gray!20!black, tassel=red!70!black]
	\end{tikzpicture}	
\end{center}
}

% theorem commands
\newtheoremstyle{c_remark}
	{}	% Space above
	{}	% Space below
	{}% Body font
	{}	% Indent amount
	{\bfseries}	% Theorem head font
	{}	% Punctuation after theorem head
	{.5em}	% Space after theorem head
	{\thmname{#1}\thmnumber{ #2}\thmnote{ \normalfont{\text{(#3)}}}}	% head content
\newtheoremstyle{c_definition}
	{3pt}	% Space above
	{3pt}	% Space below
	{}% Body font
	{}	% Indent amount
	{\bfseries}	% Theorem head font
	{}	% Punctuation after theorem head
	{.5em}	% Space after theorem head
	{\thmname{#1}\thmnumber{ #2}\thmnote{ \normalfont{\text{(#3)}}}}	% head content
\newtheoremstyle{c_plain}
	{3pt}	% Space above
	{3pt}	% Space below
	{\itshape}% Body font
	{}	% Indent amount
	{\bfseries}	% Theorem head font
	{}	% Punctuation after theorem head
	{.5em}	% Space after theorem head
	{\thmname{#1}\thmnumber{ #2}\thmnote{ \text{(#3)}}}	% head content

\theoremstyle{c_plain}
\newtheorem{theorem}{משפט}[section]
\newtheorem{lemma}[theorem]{למה}
\newtheorem{proposition}[theorem]{טענה}
\newtheorem*{proposition*}{טענה}
%\newtheorem{corollary}[theorem]{אין חלופה עברית}

\theoremstyle{c_definition}
\newtheorem{definition}[theorem]{הגדרה}
\newtheorem*{definition*}{הגדרה}
\newtheorem{example}{דוגמה}[section]
\newtheorem{exercise}{תרגיל}[section]

\theoremstyle{c_remark}
\newtheorem*{remark}{הערה}
\newtheorem*{solution}{פתרון}
\newtheorem{conclusion}[theorem]{מסקנה}
\newtheorem{notation}[theorem]{סימון}

% Questions related commands
\newcounter{question}
\setcounter{question}{1}
\newcounter{sub_question}
\setcounter{sub_question}{1}

\newcommand{\question}[1][0]{
	\ifthenelse{#1 = 0}{}{\setcounter{question}{#1}}
	\subsection{שאלה \arabic{question}}
	\addtocounter{question}{1}
	\setcounter{sub_question}{1}
}

\newcommand{\subquestion}[1][0]{
	\ifthenelse{#1 = 0}{}{\setcounter{sub_question}{#1}}
	\subsubsection{סעיף \localecounter{letters.gershayim}{sub_question}}
	\addtocounter{sub_question}{1}
}

% import lua and start of document
\directlua{common = require ('../common')}

\GetEnv{AUTHOR}

% headers
\author{\AUTHOR}
\date\today

\title{פתרון מטלה 07 --- חשבון אינפינטסמלי 3 (80415)}

\begin{document}
\maketitle
\maketitleprint{}

\Question{}
נמצא ונסווג את כל הנקודות הקריטיות של הפונקציה
\[
	f(x, y, z) = x^2 + y^2 + z^2 + xy + 3z - x
\]

נחשב
\[
	\nabla f(x, y, z) = (2x + y - 1, 2y + x, 2z + 3)
\]
נבדוק את התאפסות הגרדיאנט
\[
	\nabla f = 0 \iff z = \frac{-3}{2}, y = \frac{-1}{3}, x = \frac{2}{3}
\]
ומצאנו כי ישנה נקודה יחידה חשודה והיא $p = (\frac{2}{3}, -\frac{1}{3}, -\frac{3}{2})$.
נחשב את הנגזרת השנייה ונקבל
\[
	D^2f = \begin{pmatrix}
		2 & 1 & 0 \\
		1 & 2 & 0 \\
		0 & 0 & 2
	\end{pmatrix}
\]
זוהי כמובן מטריצה חיובית לחלוטין לפי משפט סילבסטר ולכן $p$ מינימום מקומי ויחיד ולכן גם מינימום מוחלט.

\Question{}
יהי $V = \RR^{d \times d}$ מרחב מטריצות. \\*
נוכיח שקיימת סביבה $U$ של $I$ כך שכל המטריצות ב־$U$ הן ריבועים.
\begin{proof}
	נגדיר $f : V \to V$ על־ידי $f(A) = A^2$. \\*
	נראה כי $f(A + H) = A^2 + A H + H A + H^2$ ולכן נוכל לקבוע כי $f$ גזירה בכל נקודה וכי $(D f \mid_A)(X) = AX + XA$. \\*
	זוכי כמובן פונקציה רציפה ולכן מצאנו כי $f$ גזירה ברציפות בסביבת $A \in V$ ובפרט ב־$I$. \\*
	נבחין כי $(Df\mid_I)(X) = 2X$ ובהתאם $J_f(I) = \det(2I) \ge 0$ ולכן קיימת סביבה פתוחה $I \ni U \subseteq V$ כך ש־$f^{-1} : U \to U$ מוגדרת וגזירה ברציפות בה.
	מהגדרת $f$ ישירות נובע כי כל $A \in U$ היא ריבוע.
\end{proof}

\Question{}
נמצא את המינימום והמקסימום המוחלטים של הפונקציות הבאות.

\Subquestion{}
הפונקציה $f(x, y) = x^2 + y^2 - 12x + 16y$ בתחום $A = \{(x, y) \mid x^2 + y^2 \le 1, 3x \ge -y \}$.

נתחיל בבדיקת $A^\circ$, זוהי קבוצה פתוחה ולכן נבדוק על־פי גזירה ומציאת נקודות קריטיות:
\[
	\nabla f(x, y) = (2x - 12, 2y + 16)
\]
הנקודה היחידה החשודה לקיצון היא $(6, -8) \notin A$. \\*
עתה נבדוק את $\partial A = \{(x, y) \mid x^2 + y^2 = 1, 3x = -y \}$. \\*
נגדיר $g_1(x, y) = x^2 + y^2 - 1, g_2(x, y) = 3x + y$ ונראה כי $(x, y) \in \partial A \iff g_1(x, y) = g_2(x, y) = 0$. \\*
נראה גם כי $\nabla g_1 = (2x, 2y), \nabla g_2 = (3, 1)$. \\*
נתחיל בחיפוש נקודות קיצון תחת האילוץ $g_1$ ונקבל
\[
	\nabla f = \lambda \nabla g_1
	\implies
	2x (1 - \lambda) = 12, 2y (1 - \lambda) = -16
\]
אם $\lambda = 1$ נקבל סתירה ולכן נניח $\lambda \ne 1$, וידוע כי $x, y \ne 0$ מהתחום ולכן נוכל להסיק
\[
	x \cdot \frac{-8}{y} = 6 \implies x = -\frac{3}{4} y
\]
נשתמש בשוויון $g_1(x, y) = 0$ ונקבל $\frac{9}{16}y^2 + y^2 = 1$ ולכן $y = \pm \frac{4}{5}$ ונקבל $\pm(\frac{3}{5}, -\frac{4}{5})$, בתחום נמצאת רק $(\frac{3}{5}, -\frac{4}{5})$, ובה $f(\frac{3}{5}, -\frac{4}{5}) = -19$. \\*
נבדוק את האילוץ השני ונקבל את השוויון
\[
	2x - 12 = 3 \lambda, 2y + 16 = \lambda
\]
אילו $\lambda = 0$ אז נקבל נקודה יחידה שאיננה בתחום $g_1$ ולכן נניח $\lambda \ne 0$ ונקבל
\[
	2x - 12 = 3(2y + 16) \implies x = 3y + 30
\]
ונסיק מ־$g_2 = 0$ כי
\[
	3x + y = 0
	\implies 3x + 3x + 30 = 0
	\implies x = -5
\]
ונסיק כי הנקודה לא בתחום. \\*
נבדוק את המקרה בו שני האילוצים מתקיימים, במקרה זה יש שתי נקודות יחידות כאשר
\[
	9y^2 + y^2 = 1 \implies y = \pm \frac{1}{\sqrt{10}}
\]
וקיבלנו את הנקודות $\pm(\frac{3}{\sqrt{10}}, -\frac{1}{\sqrt{10}})$,
ונקבל $f(\pm(\frac{3}{\sqrt{10}}, -\frac{1}{\sqrt{10}})) = 1 \mp \frac{36}{\sqrt{10}} \mp \frac{16}{\sqrt{10}} = 1 \mp \frac{52}{\sqrt{10}}$. \\*
מצאנו את כל הנקודות.

\Subquestion{}
\[
	f(x, y) = 8x - 2y,
	\quad
	A = \{ (x, y) \in \RR^2 \mid 2x - y^3 \le 1, 0 \le x \le 1, 0 \le y \le 1 \}
\]
נתחיל מחישוב קיצון פניומיות:
\[
	\nabla f = (8, -2)
\]
ולכן אין קיצון פנימיות. \\*
נעבור על חמשת האילוצים: \\*
נגדיר $g_1(x, y) = 2x - y^3 - 1$ ונקבל גם
\[
	\nabla g_1 = (2, 3y^2)
\]
מלגרנז' נקבל
\[
	(8, -2) = \lambda(2, 3y^2)
\]
נקבל סתירה ישירות מחיוביות השוויון השני. \\*
נגדיר $g_2(x, y) = x$ ונבחן את האילוץ $g_2(x, y) = 0$:
\[
	\nabla g_2 = (1, 0)
\]
ולכן
\[
	(8, -2) = \lambda(1, 0)
\]
ונראה כי אנחנו מקבלים סתירות, נוכל לעשות תהליך זה לכל האילוצים הישירים על הצירים ולקבל סתירה דומה, נבחן עתה את חיתוכי האילוצים. \\*
כמובן נקבל ישירות את הנקודות $(0, 0), (0, 1), (1, 0), (1, 1)$ ובהתאמה
\[
	f(0, 0) = 0,
	f(0, 1) = -2,
	f(1, 0) = 8,
	f(1, 1) = 6
\]
ונשאר לבדוק את $g_1$ עם הגבלות על הצירים, נקבל
\[
	(0, -1), (1, 1), (\frac{1}{2}, 0)
\]
ישנה נקודה יחידה שבתחום ולא בדקנו, ונקבל
\[
	f(\frac{1}{2}, 0) = 4
\]
ולכן המקסימום הוא $(1, 0)$ ו־$(0, 1)$ המינימום.

\Subquestion{}
\[
	f(x, y, z) = x^2 + y^2 + z^2, A = \{ (x, y, z) \in \RR^3 \mid 2x^2 + y^2 + z^2 \le 1, 5x + 4y + 3z = 0 \}
\]
לכן נגדיר
\[
	g_1(x, y, z) = 2x^2 + y^2 + z^2 - 1,
	\qquad
	g_2(x, y, z) = 5x + 4y + 3z
\]
ונחשב
\[
	\nabla f = (2x, 2y, 2z),
	\qquad
	\nabla g_1 = (4x, 2y, 2z),
	\qquad
	\nabla g_2 = (5, 4, 3)
\]
נבדוק קיצון פנימי ונקבל
\[
	\nabla f = 0 \implies (0, 0, 0)
\]
ואכן $(0, 0, 0) \in A$, אנו כבר יודעים כי $f$ פרבולואיד ולכן זהו מינימום מקומי. \\*
נבדוק את האילוץ $g_1 = 0$ על־ידי לגרנז' ונקבל
\[
	(2x, 2y, 2z) = \lambda (4x, 2y, 2z)
\]
נקבל מהשוויון השלישי ישירות כי $\lambda = 1$ ומהשוויון הראשון כי $x = 0$ ולכן $g_1(0, y, z) = y^2 + z^2 - 1 = 0$, נצמצם עוד על־ידי האילוץ השני ונקבל $4y + 3z = 0$ ולכן $16y^2 = 9z^2 = 9 - 9y^2 \implies y = \pm \frac{3}{5}$. \\*
נקבל $\pm(0, \frac{3}{5}, -\frac{4}{5})$ בלבד וערך $f$ בנקודות אלה הוא
\[
	f(\pm(\frac{3}{5}, -\frac{4}{5})) = 1
\]
נבדוק את האילוץ השני ונקבל
\[
	(2x, 2y, 2z) = \lambda (5, 4, 3)
\]
נקבל מערכת הומוגנית ונפתור אותה
\[
	\begin{pmatrix}
		2 & 0 & 0 & -5 \\
		0 & 2 & 0 & -4 \\
		0 & 0 & 2 & -3 \\
		5 & 4 & 3 & 0 \\
	\end{pmatrix}
\]
זו כמובן מטריצה הפיכה ונסיק כי $(0, 0, 0)$ פתרון יחיד.

נבדוק את שני האילוצים יחד ונקבל את המשוואה
\[
	(2x, 2y, 2z) = \lambda_1 (4x, 2y, 2z) + \lambda_2 (5, 4, 3)
\]
והיא תניב לנו תוצאות שלא בתחום ונסיק כי $(0, 0, 0)$ מינימום ו־$\pm(\frac{3}{5}, -\frac{4}{5}, 1)$ מקסימום.

\Subquestion{}
\[
	f(x, y, z) = x - y + 3z, A = \{(x, y, z) \in \RR^3 \mid \frac{1}{4}x^2 + \frac{1}{4}y^2 + z^2 \le 1, z \ge 0 \}
\]
ולכן נגדיר
\[
	g_1(x, y, z) = \frac{1}{4}x^2 + \frac{1}{4}y^2 + z^2 - 1,
	\qquad
	g_2(x, y, z) = z
\]
ונחשב
\[
	\nabla f = (1, -1, 3),
	\qquad
	\nabla g_1 = (\frac{1}{2}x, \frac{1}{2}y, 2z),
	\qquad
	\nabla g_2 = (0, 0, 1)
\]
הפונקציה לינארית ולכן אין נקודות קיצון פנימיות. \\*
נבדוק את האילוץ הראשון על־ידי לגרנז'
\[
	(1, -1, 3) = \lambda (\frac{1}{2}x, \frac{1}{2}y, 2z)
\]
ולכן כמובן נקבל $x = -y = 4z$ על־ידי שימוש באגפים הראשונים. נקבל מ־$g_1(x, y, z) = 0$ גם
\[
	\frac{1}{4} 16z^2 + \frac{1}{4} 16z^2 + z^2 = 1 \implies z = \frac{1}{3}
\]
ולכן נקבל את הנקודה $(\frac{4}{3}, -\frac{4}{3}, \frac{1}{3})$, עבורה מתקיים $f(\frac{4}{3}, -\frac{4}{3}, \frac{1}{3}) = \frac{11}{3}$. \\*
האילוץ השני הוא לינארי והחיתוך שלו על הפונקציה לא מניב נקודות קריטיות. \\*
האילוץ השני גורר $z = 0$ ולכן כדי לבדוק את שניהם מספיק שנחשב את
\[
	(1, -1) = \lambda (\frac{1}{2} x, \frac{1}{2} y)
\]
כמובן נקבל $x = -y$ וגם $\frac{1}{2} x^2 = 1 \implies x = \pm \sqrt{2}$, ולכן הנקודות הקריטיות הן $\pm(\sqrt{2}, -\sqrt{2}, 0)$. \\*
נקבל גם $f(\pm(\sqrt{2}, -\sqrt{2}, 0)) = \pm 2 \sqrt{2}$.

\Question{}
נחשב את הנפח המקסימלי של תיבה מקבילה לצירים שאפשר לבנות בתוך האליפסואיד
\[
	\frac{x^2}{a^2} + \frac{y^2}{b^2} + \frac{z^2}{c^2} \le 1
\]

נבחין כי מסימטריה של האליפסואיד סביב הצירים כל נקודה יחידה פנימית שלו מגדירה היטב תיבה על־ידי בחירת קודקוד נגדי, ונפח תיבה זו הוא $8xyz$, ולכן מטעמי נוחות נגדיר
\[
	f(x, y, z) = xyz
\]
פונקציית נפח שמייצגת נאמנה את התנהגות נפח התיבה הסופית, על־ידי $8|f(p)|$. \\*
נגדיר את החסימה על־ידי האליפסואיד כאילוץ על־ידי הפונקציה
\[
	g(x, y, z) = \frac{x^2}{a^2} + \frac{y^2}{b^2} + \frac{z^2}{c^2}
\]
נחשב את הגרדיאנטים להמשך החישוב
\[
	\nabla f = (yz, xz, xy),
	\qquad
	\nabla g = (\frac{2x}{a^2}, \frac{2y}{b^2}, \frac{2z}{c^2})
\]
נתחיל במציאת קיצון פנימי ונקבל מאיפוס הגרדיאנט ישירות את הנקודה $(0, 0, 0)$, בה כמובן הנפח עצמו הוא מינימלי.

נעבור לבדיקת קיצון על האליפסואיד, נשתמש בשיטת לגרנז' ונקבל את השוויון
\[
	(yz, xz, xy) = \lambda (\frac{x}{a^2}, \frac{y}{b^2}, \frac{z}{c^2})
\]
לכן נקבל $xyz = \frac{x^2}{a^2} = \frac{y^2}{b^2} = \frac{z^2}{c^2}$ והשוויון $g(x, y, z) = 1$ נקבל גם $\frac{3}{a^2} x^2 = 1$, ולכן $x = \frac{a}{\sqrt{3}}$, נתעלם מהתוצאה השלילית מסימטריית התיבה המתקבלת.
נקבל מהתליך הזה את הנקודה $(\frac{a}{\sqrt{3}},\frac{b}{\sqrt{3}}, \frac{c}{\sqrt{3}})$, ו־$f(\frac{a}{\sqrt{3}},\frac{b}{\sqrt{3}}, \frac{c}{\sqrt{3}}) = \frac{abc}{3 \sqrt{3}}$. \\*
ולכן נפח התיבה המקסימלית יהיה $\frac{8abc}{3 \sqrt{3}}$.

\Question{}
יהיו $p, q > 0$ כך ש־$\frac{1}{p} + \frac{1}{q} = 1$.

\Subquestion{}
נבדוק אם לפונקציה
\[
	f(x, y) = \frac{x^p}{p} + \frac{y^q}{q}
\]
יש מינימום ומקסימום על חצי ההיפרבולה $\{(x, y) \in \RR^2 \mid x > 0, y > 0, xy = 1 \}$.

נתחיל כמובן מלבדוק קיצון כללי בפונקציה, נחשב
\[
	\nabla f = (x^{p - 1}, y^{q - 1})
\]
ולכן כמובן נקבל נקודה יחידה $(0, 0)$ אשר איננה בתחום. \\*
נגדיר $g(x, y) = xy$ ונבדוק את ההגבלה $g(x, y) = 1$ על־ידי שימוש במשפט כופלי לגרנז', נקבל
\[
	(x^{p - 1}, y^{q - 1}) = \lambda (y, x)
\]
אנו יודעים כי $xy = 1$ ומחיוביות נוכל לקבוע $y = \frac{1}{y}$ ויחד עם השוויון $x^{p - 1} \frac{1}{y} = \lambda = y^{q - 1} \frac{1}{x}$ הנובע מהשורה הקודמת נקבל
\[
	x^p = y^q \implies x^{pq} = 1 \implies x = 1 \implies y = 1
\]
ומצאנו כי הנקודה $(1, 1)$ היא קיצון בתחום, ומתקיים $f(1, 1) = \frac{1}{p} + \frac{1}{q} = 1$. מבדיקת נקודה נוספת בתחום והעובדה שזהו הקיצון היחיד נסיק כי זהו מינימום.

\Subquestion{}
נוכיח את אי־שוויון יאנג, לכל $x, y > 0$ מתקיים
\[
	xy \le \frac{x^p}{p} + \frac{y^q}{q}
\]
\begin{proof}
	בסעיף הקודם מצאנו כי אם $xy = 1$ אז בהיפרבולה זו המינימום של $f(x, y)$ הוא $1$ ולכן $xy \le f(x, y)$. \\*
	נחפש באופן דומה נקודות קיצון עבור $xy = k$ ל־$k \in \RR_+$ כלשהו, ונקבל מהתהליך את הנקודה $(\sqrt[p]{k}, \sqrt[q]{k})$ כנקודת מינימום, ומתקיים $f(\sqrt[p]{k}, \sqrt[q]{k}) = k$, ומצאנו כי אי־השוויון נכון.
\end{proof}

\Question{}
רוצים לתכנן פחית משקה גלילית עם נפח $V$ ושטח פנים מינימלי. \\*
נחשב את מידות הפחית.

נגדיר פונקציות $f, g$ המסמלות את שטח הפנים והנפח בהתאמה:
\[
	f(r, h) = 2\pi r^2 + 2\pi hr
	\qquad
	g(r, h) = \pi hr^2
\]
כמובן אנו מנסים למצוא את המינימום של $f$ עם האילוץ $g = V$, ולכן נחשב גרדיאנטים
\[
	\nabla f = (4\pi r + 2\pi h, 2\pi r)
	\qquad
	\nabla g = (2\pi rh, \pi r^2)
\]
לכן ממשפט כופלי לגרנז' נקבל
\[
	(4\pi r + 2\pi h, 2\pi r) = \lambda (2\pi rh, \pi r^2)
\]
יחד עם האילוץ$\pi h r^2 = V$. נקבל $2r + h = \lambda rh$ וכן $\lambda = \frac{2r + h}{rh}$, ומהשוויון השני $2r = \lambda r^2$ ולכן $\lambda = \frac{2}{r}$ ונקבל
\[
	\frac{2r + h}{rh} = \frac{2}{r} \implies 2r + h = 2h \implies h = 2r
\]
נציב ונקבל $V = 2\pi r^3$ ולכן $r = \sqrt[3]{\frac{V}{2\pi}}$, ונקבל גם $h = \sqrt[3]{\frac{4V}{\pi}}$, ומבדיקת נקודות אחרות בתחום נקבל כי זהו מינימום. \\*
נסכם ונאמר כי על גובה הפחית להיות $\sqrt[3]{\frac{4V}{\pi}}$ ועל רדיוסה להיות $\sqrt[3]{\frac{V}{2\pi}}$.

\end{document}
