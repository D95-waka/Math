\documentclass[a4paper]{article}

% packages
\usepackage{inputenc, amsmath, amsthm, thmtools, amsfonts, amssymb, luacode, catchfile, tikzducks, hyperref}
\usepackage[a4paper, margin=50pt, includeheadfoot]{geometry} % set page margins
\usepackage[shortlabels]{enumitem}
\usepackage[skip=3pt, indent=0pt]{parskip}

% language
\usepackage[bidi=basic, layout=tabular, provide=*]{babel}
\babelprovide[main, import]{hebrew}
\babelprovide{rl}
\babelfont{rm}{Libertinus Serif}
\babelfont{sf}{Libertinus Sans}
\babelfont{tt}{Libertinus Mono}

% style
\AddToHook{cmd/section/before}{\clearpage}	% Add line break before section
\linespread{1.3}
\setcounter{secnumdepth}{0}		% Remove default number tags from sections, this won't do well with theorems
\AtBeginDocument{\setlength{\belowdisplayskip}{3pt}}
\AtBeginDocument{\setlength{\abovedisplayskip}{3pt}}

% operators
\DeclareMathOperator\cis{cis}
\DeclareMathOperator\Sp{Sp}
\DeclareMathOperator\tr{tr}
\DeclareMathOperator\im{Im}
\DeclareMathOperator\re{Re}
\DeclareMathOperator\diag{diag}
\DeclareMathOperator*\lowlim{\underline{lim}}
\DeclareMathOperator*\uplim{\overline{lim}}
\DeclareMathOperator\rng{rng}
\DeclareMathOperator\Sym{Sym}
\DeclareMathOperator\Arg{Arg}
\DeclareMathOperator\Log{Log}
\DeclareMathOperator\dom{dom}

% commands
%\renewcommand\qedsymbol{\textbf{מש''ל}}
%\renewcommand\qedsymbol{\fbox{\emoji{lizard}}}
\newcommand{\NN}[0]{\mathbb{N}}
\newcommand{\ZZ}[0]{\mathbb{Z}}
\newcommand{\QQ}[0]{\mathbb{Q}}
\newcommand{\RR}[0]{\mathbb{R}}
\newcommand{\CC}[0]{\mathbb{C}}
\newcommand{\FF}[0]{\mathbb{F}}
\newcommand{\PP}[0]{\mathbb{P}}
\newcommand{\TT}[0]{\mathbb{T}}
\newcommand{\acts}[0]{\circlearrowright}
\newcommand{\explain}[2] {
	\begin{flalign*}
		 && \text{#2} && \text{#1}
	\end{flalign*}
}
\newcommand{\maketitleprint}[0]{ \begin{center}
	\begin{tikzpicture}[scale=3]
		\duck[graduate=gray!20!black, tassel=red!70!black]
	\end{tikzpicture}	
\end{center}
}

% theorem commands
\newtheoremstyle{c_remark}
	{}	% Space above
	{}	% Space below
	{}% Body font
	{}	% Indent amount
	{\bfseries}	% Theorem head font
	{}	% Punctuation after theorem head
	{.5em}	% Space after theorem head
	{\thmname{#1}\thmnumber{ #2}\thmnote{ \normalfont{\text{(#3)}}}}	% head content
\newtheoremstyle{c_definition}
	{3pt}	% Space above
	{3pt}	% Space below
	{}% Body font
	{}	% Indent amount
	{\bfseries}	% Theorem head font
	{}	% Punctuation after theorem head
	{.5em}	% Space after theorem head
	{\thmname{#1}\thmnumber{ #2}\thmnote{ \normalfont{\text{(#3)}}}}	% head content
\newtheoremstyle{c_plain}
	{3pt}	% Space above
	{3pt}	% Space below
	{\itshape}% Body font
	{}	% Indent amount
	{\bfseries}	% Theorem head font
	{}	% Punctuation after theorem head
	{.5em}	% Space after theorem head
	{\thmname{#1}\thmnumber{ #2}\thmnote{ \text{(#3)}}}	% head content

\theoremstyle{c_plain}
\newtheorem{theorem}{משפט}[section]
\newtheorem{lemma}[theorem]{למה}
\newtheorem{proposition}[theorem]{טענה}
\newtheorem*{proposition*}{טענה}
%\newtheorem{corollary}[theorem]{אין חלופה עברית}

\theoremstyle{c_definition}
\newtheorem{definition}[theorem]{הגדרה}
\newtheorem*{definition*}{הגדרה}
\newtheorem{example}{דוגמה}[section]
\newtheorem{exercise}{תרגיל}[section]

\theoremstyle{c_remark}
\newtheorem*{remark}{הערה}
\newtheorem*{solution}{פתרון}
\newtheorem{conclusion}[theorem]{מסקנה}
\newtheorem{notation}[theorem]{סימון}

% Questions related commands
\newcounter{question}
\setcounter{question}{1}
\newcounter{sub_question}
\setcounter{sub_question}{1}

\newcommand{\question}[1][0]{
	\ifthenelse{#1 = 0}{}{\setcounter{question}{#1}}
	\subsection{שאלה \arabic{question}}
	\addtocounter{question}{1}
	\setcounter{sub_question}{1}
}

\newcommand{\subquestion}[1][0]{
	\ifthenelse{#1 = 0}{}{\setcounter{sub_question}{#1}}
	\subsubsection{סעיף \localecounter{letters.gershayim}{sub_question}}
	\addtocounter{sub_question}{1}
}

% import lua and start of document
\directlua{common = require ('../common')}

\GetEnv{AUTHOR}

% headers
\author{\AUTHOR}
\date\today

\title{פתרון מטלה 07 --- חשבון אינפינטסמלי 3 (80415)}

\begin{document}
\maketitle
\maketitleprint{}

\Question{}
נמצא ונסווג את כל הנקודות הקריטיות של הפונקציה
\[
	f(x, y, z) = x^2 + y^2 + z^2 + xy + 3z - x
\]

נחשב
\[
	\nabla f(x, y, z) = (2x + y - 1, 2y + x, 2z + 3)
\]
נבדוק את התאפסות הגרדיאנט
\[
	\nabla f = 0 \iff z = \frac{-3}{2}, y = \frac{-1}{3}, x = \frac{2}{3}
\]
ומצאנו כי ישנה נקודה יחידה חשודה והיא $p = (\frac{2}{3}, -\frac{1}{3}, -\frac{3}{2})$.
נחשב את הנגזרת השנייה ונקבל
\[
	D^2f = \begin{pmatrix}
		2 & 1 & 0 \\
		1 & 2 & 0 \\
		0 & 0 & 2
	\end{pmatrix}
\]
זוהי כמובן מטריצה חיובית לחלוטין לפי משפט סילבסטר ולכן $p$ מינימום מקומי ויחיד ולכן גם מינימום מוחלט.

\Question{}
יהי $V = \RR^{d \times d}$ מרחב מטריצות. \\*
נוכיח שקיימת סביבה $U$ של $I$ כך שכל המטריצות ב־$U$ הן ריבועים.
\begin{proof}
	נגדיר $f : V \to V$ על־ידי $f(A) = A^2$. \\*
	נראה כי $f(A + H) = A^2 + A H + H A + H^2$ ולכן נוכל לקבוע כי $f$ גזירה בכל נקודה וכי $(D f \mid_A)(X) = AX + XA$. \\*
	זוכי כמובן פונקציה רציפה ולכן מצאנו כי $f$ גזירה ברציפות בסביבת $A \in V$ ובפרט ב־$I$. \\*
	נבחין כי $(Df\mid_I)(X) = 2X$ ובהתאם $J_f(I) = \det(2I) \ge 0$ ולכן קיימת סביבה פתוחה $I \ni U \subseteq V$ כך ש־$f^{-1} : U \to U$ מוגדרת וגזירה ברציפות בה.
	מהגדרת $f$ ישירות נובע כי כל $A \in U$ היא ריבוע.
\end{proof}

\Question{}
נמצא את המינימום והמקסימום המוחלטים של הפונקציות הבאות.

\Subquestion{}
הפונקציה $f(x, y) = x^2 + y^2 - 12x + 16y$ בתחום $A = \{(x, y) \mid x^2 + y^2 \le 1, 3x \ge -y \}$.

נתחיל בבדיקת $A^\circ$, זוהי קבוצה פתוחה ולכן נבדוק על־פי גזירה ומציאת נקודות קריטיות:
\[
	\nabla f(x, y) = (2x - 12, 2y + 16)
\]
הנקודה היחידה החשודה לקיצון היא $(6, -8) \notin A$. \\*
עתה נבדוק את $\partial A = \{(x, y) \mid x^2 + y^2 = 1, 3x = -y \}$. \\*
נגדיר $g_1(x, y) = x^2 + y^2 - 1, g_2(x, y) = 3x + y$ ונראה כי $(x, y) \in \partial A \iff g_1(x, y) = g_2(x, y) = 0$. \\*
נראה גם כי $\nabla g_1 = (2x, 2y), \nabla g_2 = (3, 1)$, ולכן אוסף הנקודות החשוד הוא הפתרון של המשוואות
\[
	(2x - 12, 2y + 16) = \lambda (2x, 2y),
	\qquad
	(2x - 12, 2y + 16) = \lambda (3, 1)
\]
הפתרון לשוויון השני הוא
\[
	2x - 12 = 3\lambda, 2y + 16 = \lambda \iff 2x - 12 = 3(2(-3x) + 16) = -20x + 60 = 0 \iff x = \frac{1}{3}, y = -1, \lambda = 14
\]
דהינו $(\frac{1}{3}, -1)$ והוא לא בתחום. \\*
נבדוק את השוויון הראשון
\begin{align*}
	& 2x(1 - \lambda) + 16 = 0, 2y(1 - \lambda) - 12 = 0 \iff 2x \frac{6}{y} + 16 = 0 \iff 3x + 4y = 0, \\
	& (2x + 2y)(1 - \lambda) + 4 = 0 \iff (-\frac{3}{2}x + 2x)(1 - \lambda) + 4 = 0 \iff \frac{1}{2}x = -4 \iff x = -8
\end{align*}
ונקבל $(

\end{document}
