\documentclass[a4paper]{article}

% packages
\usepackage{inputenc, amsmath, amsthm, thmtools, amsfonts, amssymb, luacode, catchfile, tikzducks, hyperref}
\usepackage[a4paper, margin=50pt, includeheadfoot]{geometry} % set page margins
\usepackage[shortlabels]{enumitem}
\usepackage[skip=3pt, indent=0pt]{parskip}

% language
\usepackage[bidi=basic, layout=tabular, provide=*]{babel}
\babelprovide[main, import]{hebrew}
\babelprovide{rl}
\babelfont{rm}{Libertinus Serif}
\babelfont{sf}{Libertinus Sans}
\babelfont{tt}{Libertinus Mono}

% style
\AddToHook{cmd/section/before}{\clearpage}	% Add line break before section
\linespread{1.3}
\setcounter{secnumdepth}{0}		% Remove default number tags from sections, this won't do well with theorems
\AtBeginDocument{\setlength{\belowdisplayskip}{3pt}}
\AtBeginDocument{\setlength{\abovedisplayskip}{3pt}}

% operators
\DeclareMathOperator\cis{cis}
\DeclareMathOperator\Sp{Sp}
\DeclareMathOperator\tr{tr}
\DeclareMathOperator\im{Im}
\DeclareMathOperator\re{Re}
\DeclareMathOperator\diag{diag}
\DeclareMathOperator*\lowlim{\underline{lim}}
\DeclareMathOperator*\uplim{\overline{lim}}
\DeclareMathOperator\rng{rng}
\DeclareMathOperator\Sym{Sym}
\DeclareMathOperator\Arg{Arg}
\DeclareMathOperator\Log{Log}
\DeclareMathOperator\dom{dom}

% commands
%\renewcommand\qedsymbol{\textbf{מש''ל}}
%\renewcommand\qedsymbol{\fbox{\emoji{lizard}}}
\newcommand{\NN}[0]{\mathbb{N}}
\newcommand{\ZZ}[0]{\mathbb{Z}}
\newcommand{\QQ}[0]{\mathbb{Q}}
\newcommand{\RR}[0]{\mathbb{R}}
\newcommand{\CC}[0]{\mathbb{C}}
\newcommand{\FF}[0]{\mathbb{F}}
\newcommand{\PP}[0]{\mathbb{P}}
\newcommand{\TT}[0]{\mathbb{T}}
\newcommand{\acts}[0]{\circlearrowright}
\newcommand{\explain}[2] {
	\begin{flalign*}
		 && \text{#2} && \text{#1}
	\end{flalign*}
}
\newcommand{\maketitleprint}[0]{ \begin{center}
	\begin{tikzpicture}[scale=3]
		\duck[graduate=gray!20!black, tassel=red!70!black]
	\end{tikzpicture}	
\end{center}
}

% theorem commands
\newtheoremstyle{c_remark}
	{}	% Space above
	{}	% Space below
	{}% Body font
	{}	% Indent amount
	{\bfseries}	% Theorem head font
	{}	% Punctuation after theorem head
	{.5em}	% Space after theorem head
	{\thmname{#1}\thmnumber{ #2}\thmnote{ \normalfont{\text{(#3)}}}}	% head content
\newtheoremstyle{c_definition}
	{3pt}	% Space above
	{3pt}	% Space below
	{}% Body font
	{}	% Indent amount
	{\bfseries}	% Theorem head font
	{}	% Punctuation after theorem head
	{.5em}	% Space after theorem head
	{\thmname{#1}\thmnumber{ #2}\thmnote{ \normalfont{\text{(#3)}}}}	% head content
\newtheoremstyle{c_plain}
	{3pt}	% Space above
	{3pt}	% Space below
	{\itshape}% Body font
	{}	% Indent amount
	{\bfseries}	% Theorem head font
	{}	% Punctuation after theorem head
	{.5em}	% Space after theorem head
	{\thmname{#1}\thmnumber{ #2}\thmnote{ \text{(#3)}}}	% head content

\theoremstyle{c_plain}
\newtheorem{theorem}{משפט}[section]
\newtheorem{lemma}[theorem]{למה}
\newtheorem{proposition}[theorem]{טענה}
\newtheorem*{proposition*}{טענה}
%\newtheorem{corollary}[theorem]{אין חלופה עברית}

\theoremstyle{c_definition}
\newtheorem{definition}[theorem]{הגדרה}
\newtheorem*{definition*}{הגדרה}
\newtheorem{example}{דוגמה}[section]
\newtheorem{exercise}{תרגיל}[section]

\theoremstyle{c_remark}
\newtheorem*{remark}{הערה}
\newtheorem*{solution}{פתרון}
\newtheorem{conclusion}[theorem]{מסקנה}
\newtheorem{notation}[theorem]{סימון}

% Questions related commands
\newcounter{question}
\setcounter{question}{1}
\newcounter{sub_question}
\setcounter{sub_question}{1}

\newcommand{\question}[1][0]{
	\ifthenelse{#1 = 0}{}{\setcounter{question}{#1}}
	\subsection{שאלה \arabic{question}}
	\addtocounter{question}{1}
	\setcounter{sub_question}{1}
}

\newcommand{\subquestion}[1][0]{
	\ifthenelse{#1 = 0}{}{\setcounter{sub_question}{#1}}
	\subsubsection{סעיף \localecounter{letters.gershayim}{sub_question}}
	\addtocounter{sub_question}{1}
}

% import lua and start of document
\directlua{common = require ('../common')}

\GetEnv{AUTHOR}

% headers
\author{\AUTHOR}
\date\today

\title{פתרון מטלה 05 --- חשבון אינפינטסמלי 3 (80415)}

\begin{document}
\maketitle
\maketitleprint{}

\Question{}
תהי $\gamma : \RR \to \RR^d$ מסילה גזירה.

\Subquestion{}
נחשב את הנגזרת של $f : \RR \to \RR$ המוגדרת על־ידי $f(t) = {\lVert \gamma(t)\rVert}^2$.

נגדיר $g : \RR^d \to \RR$ על־ידי $g(v) = \lVert v \rVert^2$.
נשים לב כי מתקיים $f = g \circ \gamma$ ונחשב את נגזרתה של $f$ על־ידי שימוש בכלל הנגזרת, ונקבל
\[
	f'(t) = Dg\mid_{\gamma(t)} \circ \gamma'(t)
\]
ולכן גם
\[
	f'(t) = \langle \nabla g \mid_{\gamma(t)}, \gamma'(t) \rangle
\]
ומחישוב על־ידי הגדרת הנורמה האוקלידית ונגדיר $v = (v_1, \dots, v_d)$ פירוק לקורדינטה ונקבל
\[
	\nabla g_v = (2v_1, \dots, 2v_d)
\]
ולכן גם
\[
	f'(t) = \langle (2 \gamma_1(t), \dots, 2 \gamma_d(t)), (\gamma_1'(t), \dots, \gamma_d'(t)) \rangle
\]
ולכן
\[
	f'(t) = 2\gamma_1(t) \cdot \gamma_1'(t) + \cdots + 2\gamma_d(t) \cdot \gamma_d'(t)
\]

\Subquestion{}
נניח כי קיימים $a < b$ כך ש־$\gamma(a) = \gamma(b)$.
נוכיח שקיים $t_0 \in \RR$ כך ש־$\gamma(t_0)$ מאונך ל־$\gamma'(t_0)$.
\begin{proof}
	נבחין כי $\gamma(t_0)$ מאונך ל־$\gamma'(t_0)$ על־פי הגדרה כאשר $<\langle \gamma(t_0), \gamma'(g_0) \rangle = 0$, דהינו כאשר $f'(t_0) = 0$, ולכן מספיק להוכיח כי קיים $t_0$ המאפס את $f'$. \\*
	עתה נשים לב גם ש־$f(a) = \lVert \gamma(a) \rangle^2 = \lVert \gamma(b) \rangle^2 = f(b)$. \\*
	הפונקציה $f$ גזירה בכל תחומה ולכן גם רציפה ומצאנו כי $f(a) = f(b)$ בעוד $a < b$ ולכן ממשפט רול נקבל כי קיים $t_0 \in (a, b)$ עבורו $f'(t_0) = 0$.
\end{proof}

\Question{}
נגדיר את $k : \RR^+ \times \RR \to \RR$ על־ידי $k(x, y) = x^y$. \\*
נוכיח כי $k$ גזירה בכל התחום ונחשב את הגרדיאנט שלה.
\begin{proof}
	נבדוק את כלל נגזרותיה החלקיות של $k$:
	\[
		\frac{\partial k}{\partial x} = y x^{y - 1},
		\quad
		\frac{\partial k}{\partial y} = x^y \ln(x)
	\]
	שתי הפונקציות שקיבלנו מוגדרות עבור כל $(x, y)$ בתחום של $k$ ולכן נסיק ממשפט הנלמד בהרצאה כי $k$ גזירה בכל תחום הגדרתה. \\*
	נשתמש בנגזרות החלקיות ונקבע גם שמתקיים
	\[
		\nabla k(x, y) = (yx^{y - 1}, x^y \ln(x))
	\]
\end{proof}

\Subquestion{}
תהניה $f : \RR \to \RR^+$ ו־$g : \RR \to \RR$ פונקציות גזירות בנקודה $t_0$, ונחשב את נגזרת הפונקציה $h(t) = {(f(t))}^{f(t)}$ ב־$t_0$.

נראה כי $h(t) = k(f(t), g(t))$, ולכן נגדיר $m : \RR \to \RR^+ \times \RR$ על־ידי $m(t) = (f(t), g(t))$ ולכן $h(t) = (k \circ m)(t)$ ועלינו לגזור את ההרכבה $k \circ m$.
נקבל מגזירה אגף אגף כי $m'(t) = (f'(t), g'(t))$ ולכן מתקיים
\begin{align*}
	h'(t_0) = \nabla k(m(t_0)) \circ m'(t_0)
	& = \langle (g(t_0) {(f(t_0))}^{g(t_0) - 1}, {(f(t_0))}^{g(t_0)} \ln(f(t_0))), (f'(t_0), g'(t_0)) \rangle \\
	& =  (g(t_0) {(f(t_0))}^{g(t_0) - 1} f'(t_0) + {(f(t_0))}^{g(t_0)} \ln(f(t_0))) g'(t_0)
\end{align*}

\Question{}
\Subquestion{}
תהי $f : \RR^d \to \RR$ פונקציה גזירה ב־$p \in \RR^d$. \\*
נסמן $A = f^{-1}(\{f(p)\})$ \\*
נוכיח כי אם וקטור $v \in \RR^d$ משיק ל־$A$ בנקודה $p$ או הוא מאונך ל־$\nabla f |_p$.
\begin{proof}
	נגדיר $\gamma : (-\epsilon, \epsilon) \to A$ כך ש־$\gamma(0) = p$ ו־$\gamma'(0) = v$. \\*
	נגדיר $y = f(p)$ ולכן מהגדרת $A$ נובע $f(\gamma(t)) = y$ לכל $t$ בתחום של העקומה. \\*
	נגזור את שני צדדי הביטוי ונקבל
	\[
		\langle \nabla f |_{\gamma(0)}, \gamma'(0) \rangle = \langle \nabla f |_p, v \rangle = 0
	\]
	דהינו, $v$ מאונך ל־$\nabla f |_p$.
\end{proof}

\Subquestion{}
נמצא את אוסף הווקטורים המשיקים לאליפסואיד
\[
	\left\{ (x, y, z) \in \RR^3 \middle| \frac{x^2}{3} + \frac{y^2}{12} + \frac{z^2}{27} = 1 \right\}
\]
בנקודה $(1, 2, 3)$.

נגדיר פונקציה $f : \RR^2 \to \RR$ על־ידי
\[
	f(x, y) = 3 \sqrt{3 - \frac{y^2}{4} - x^2}
\]
נשים לב כי אכן $f(1, 2) = 3$ והנקודה מוכלת בגרף הפונקציה.
נמצא את הגרדיאנט של $f$:
\[
	\nabla f = \left( \frac{-3x}{\sqrt{3 - \frac{y^2}{4} - x^2}}, \frac{-3y}{2\sqrt{3 - \frac{y^2}{4} - x^2}} \right)
\]
ולכן נציב ונקבל
\[
	\nabla f |_{(1, 2)}
	= \left( \frac{-3}{\sqrt{3 - \frac{2^2}{4} - 1^2}}, \frac{-3 \cdot 2}{4\sqrt{3 - \frac{2^2}{4} - 1^2}} \right)
	= (-3, -\frac{3}{2})
\]
אנו יודעים כי כל נגזרת כיוונית היא מכפלת וקטור כיוון בגרדיאנט ולכן נקבל כי קבוצת כל הווקטורים היא $(-3, \frac{-3}{2}) u$.

\Subquestion{}
נגדיר $f(x, y) = x^2 - y^2$. נראה כי יש וקטור המאונך ל־$\nabla f|_{(0, 0)}$ אבל לא משיק ל־$f^{-1}(\{0\})$ בנקודה $(0, 0)$.

נחשב את הגרדיאנט ונקבל $\nabla f = (2x, -2y)$ ולכן גם $\nabla f |_{(0, 0)} = (0, 0)$, לכן נוכל להסיק שכל וקטור מאונך לגרדיאנט. \\*
עתה נבחין כי $f^{-1}(\{0\}) = (t, t, 0) \cup (t, -t, 0)$, דהינו קבוצה זו ניתנת לפירוק לשני ישרים ומהגדרת המשיק נקבל כי משיקיהם ב־$(0, 0)$ הם וטורים מהצורה $(1, 1), (1, -1)$.
לכן נבחר בווקטור $(1, 0)$ ונראה שהוא מקיים את כלל התנאים לשאלה.

\Question{}
תהי $f : \RR^n \to \RR^m$ פונקציה $L$־ליפשיצית. \\*
נוכיח כי אם $f$ גזירה ב־$p \in \RR^n$ אז $\lVert Df |_p \rVert \le L$.
\begin{proof}
	אנו יודעים כי
	\[
		\forall p, v \in \RR^n \lVert f(p + v) - f(p) \rVert \le L \lVert p + v - p \rVert = L \lVert v \rVert
	\]
	אם נבחר $v = e_i$ כאשר $e_i$ הבסיס ל־$\RR^n$ ($1 \le i \le n$) נקבל מהגזירות של $f$ גם כי
	\[
		\frac{\lVert f(p + te_i) - f(p) \rVert}{\lVert te_i \rVert}
		= \left\lVert \frac{f(p + te_i) - f(p)}{|t|} \right\rVert
		\xrightarrow[t \to 0]{}
		\left\lVert \sum_{j = 1}^{m} \frac{\partial f_j}{\partial x_i}|_p \right\rVert
		\le L
	\]
	אנו יודעים כי $(Df|_p)$ היא מטריצה כך שמתקיים
	\[
		Df|_p \cdot e_i = {(\frac{\partial f_1}{\partial x_i}, \dots, \frac{\partial f_m}{\partial x_i})}^t
	\]
	ולכן נוכל להסיק לכל $1 \le i \le n$:
	\[
		\lVert Df|_p e_i \rVert \le L
	\]
	נבחר $\lVert v \rVert = 1$ עבורו $\lVert Df |_p \cdot v \rVert$ מקסימלי ונקבל
	\[
		\lVert Df |_p \rVert = \lVert Df|_p \cdot v \rVert
	\]
	נקבע $v = \sum_{k = 1}^{n} e_k v_k$ ולכן
	\[
		\lVert Df |_p \rVert
		= \lVert \sum_{k = 1}^{n} Df|_p \cdot e_k v_k \rVert
		= \lVert \sum_{k = 1}^{n} v_k Df|_p \cdot e_k \rVert
		\le \sum_{k = 1}^{n} v_k L
		= L
	\]
\end{proof}

\end{document}
