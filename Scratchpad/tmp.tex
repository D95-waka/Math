\documentclass[a4paper]{article}

% packages
\usepackage{inputenc, fontspec, amsmath, amsthm, amsfonts, polyglossia, catchfile}
\usepackage[a4paper, margin=50pt, includeheadfoot]{geometry} % set page margins

% style
\AddToHook{cmd/section/before}{\clearpage}	% Add line break before section
\linespread{1.5}
\setcounter{secnumdepth}{0}		% Remove default number tags from sections
\setmainfont{Libertinus Serif}
\setsansfont{Libertinus Sans}
\setmonofont{Libertinus Mono}
\setdefaultlanguage{hebrew}
\setotherlanguage{english}

% operators
\DeclareMathOperator\cis{cis}
\DeclareMathOperator\Sp{Sp}
\DeclareMathOperator\tr{tr}
\DeclareMathOperator\im{Im}
\DeclareMathOperator\diag{diag}
\DeclareMathOperator*\lowlim{\underline{lim}}
\DeclareMathOperator*\uplim{\overline{lim}}

% commands
\renewcommand\qedsymbol{\textbf{משל}}
\newcommand{\NN}[0]{\mathbb{N}}
\newcommand{\ZZ}[0]{\mathbb{Z}}
\newcommand{\QQ}[0]{\mathbb{Q}}
\newcommand{\RR}[0]{\mathbb{R}}
\newcommand{\CC}[0]{\mathbb{C}}
\newcommand{\getenv}[2][] {
  \CatchFileEdef{\temp}{"|kpsewhich --var-value #2"}{\endlinechar=-1}
  \if\relax\detokenize{#1}\relax\temp\else\let#1\temp\fi
}
\newcommand{\explain}[2] {
	\begin{flalign*}
		 && \text{#2} && \text{#1}
	\end{flalign*}
}

% headers
\getenv[\AUTHOR]{AUTHOR}
\author{\AUTHOR}
\date\today


\begin{document}

נגדיר
\[
	f(x) = \frac{\ln x}{\sqrt{x^3 + x^2 - x + 1}}
\]
ונמצא את
\[
	\int_{1}^{\infty} f(x)\ dx
\]

נגדיר $g(x) = \frac{\ln x}{\sqrt{x^3}}$ ונקבל
\[
	\lim_{x \to \infty} \frac{f(x)}{g(x)} = 1
\]
ולכן ממבחן ההשוואה הגבולי האינטגרל שאנו מחפשים מתכנס אם ורק אם $\int_{1}^{\infty} g(x)\ dx$ מתכנס אף הוא. \\*
אנו יודעים כי $\forall x > 2, \ln x < \sqrt[3]{x}$ ולכן נקבל גם
\[
	\frac{\ln x}{\sqrt{x^3}} < \frac{\sqrt[3]{x}}{\sqrt{x^3}}
\]
ולכן האינטגרל הנתון מתכנס אם ורק אם מתכנס האינטגרל
\[
	\int_{1}^{\infty} \frac{\sqrt[3]{x}}{\sqrt{x^3}}\ dx
	= \int_{1}^{\infty} x^{-7/6}\ dx
	= \frac{-6}{13} x^{-1/6} \mid_1^\infty = \frac{-6}{13}
\]
ולכן ממבחן ההשוואה נקבל כי גם $\int_{1}^{\infty} f(x)\ dx$ מתכנס.

\end{document}
