\documentclass[a4paper]{article}

% packages
\usepackage{inputenc, fontspec, amsmath, amsthm, amsfonts, polyglossia, catchfile}
\usepackage[a4paper, margin=50pt, includeheadfoot]{geometry} % set page margins

% style
\AddToHook{cmd/section/before}{\clearpage}	% Add line break before section
\linespread{1.5}
\setcounter{secnumdepth}{0}		% Remove default number tags from sections
\setmainfont{Libertinus Serif}
\setsansfont{Libertinus Sans}
\setmonofont{Libertinus Mono}
\setdefaultlanguage{hebrew}
\setotherlanguage{english}

% operators
\DeclareMathOperator\cis{cis}
\DeclareMathOperator\Sp{Sp}
\DeclareMathOperator\tr{tr}
\DeclareMathOperator\im{Im}
\DeclareMathOperator\diag{diag}
\DeclareMathOperator*\lowlim{\underline{lim}}
\DeclareMathOperator*\uplim{\overline{lim}}

% commands
\renewcommand\qedsymbol{\textbf{משל}}
\newcommand{\NN}[0]{\mathbb{N}}
\newcommand{\ZZ}[0]{\mathbb{Z}}
\newcommand{\QQ}[0]{\mathbb{Q}}
\newcommand{\RR}[0]{\mathbb{R}}
\newcommand{\CC}[0]{\mathbb{C}}
\newcommand{\getenv}[2][] {
  \CatchFileEdef{\temp}{"|kpsewhich --var-value #2"}{\endlinechar=-1}
  \if\relax\detokenize{#1}\relax\temp\else\let#1\temp\fi
}
\newcommand{\explain}[2] {
	\begin{flalign*}
		 && \text{#2} && \text{#1}
	\end{flalign*}
}

% headers
\getenv[\AUTHOR]{AUTHOR}
\author{\AUTHOR}
\date\today


\begin{document}

\begin{proposition}
	תהינה פונקציות גזירות $f, g : [a, b] \to \RR$,
	מתקיים $f(a) = g(a)$ ו־$\forall x \in [a, b], f'(x) < g'(x)$,
	אז מתקיים $\forall x \in (a, b], f(x) < g(x)$.
\end{proposition}
\begin{remark}
	הטענה נכונה גם באופן מוכלל, דהינו אם $b = \infty$.
\end{remark}
\begin{proof}
	נגדיר פונקציה חדשה $h(x) = g(x) - f(x)$, ולכן נקבל $h(a) = 0$. \\*
	נשים לב כי $h'(x) = g'(x) - f'(x) > 0$ ולכן נובע כי הפונקציה עולה בכל התחום. \\*
	נסיק אם כן ש־$h(x) > 0$ בתחום הפתוח, דהינו $g(x) - f(x) > 0$.
\end{proof}
\begin{conclusion}
	אם $f'(x) > M$ כאשר $0 < M \in \RR$ אז נקבל
	\[
		f(x) > M(x - a) + f(a)
	\]
\end{conclusion}
\begin{proof}
	נגדיר $g(x) = M(x - a) + f(a)$, לכן $g(a) = f(a)$ ותנאי הטענה מתקיימים. \\*
	לכן נקבל לכל $x > a$ שגם $f(x) > M(x - a) + f(a)$.
\end{proof}
\begin{proposition}
	תהי פונקציה $f : [a, \infty) \to \RR$ כך ש־$\lim_{x \to \infty} f(x) = L$ אז $\lim_{x \to \infty} f'(x) = 0$.
\end{proposition}
\begin{proof}
	ההוכחה הזאת לא מלאה כי היא לא פוסלת מקרים שהנגזרת לא מתכנסת. \\*
	נניח בשלילה ש־$\lim_{x \to \infty} f'(x) = L' > 0$,
	אז קיים $N$ כך ש־$\forall x > M, f'(x) > \frac{L'}{2}$. \\*
	נסתכל על המשיק בנקודה כלשהי $t > N$,
	\[
		g(x) = f'(t)(x - t) + f(t) > \frac{L'}{2}(x - t) + f(t)
	\]
	ונקבל מהטענה הקודמת ש־$f(x) > g(x)$ לכל $x > N$, אבל $\lim_{x \to \infty} g(x) = \infty$,
	אז ממשפט הפרוסה נקבל גם $\lim_{x \to \infty} f(x) = \infty$ וזו סתירה.
\end{proof}


\end{document}
