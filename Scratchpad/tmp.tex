\documentclass[a4paper]{article}

% packages
\usepackage{inputenc, fontspec, amsmath, amsthm, amsfonts, polyglossia, catchfile}
\usepackage[a4paper, margin=50pt, includeheadfoot]{geometry} % set page margins

% style
\AddToHook{cmd/section/before}{\clearpage}	% Add line break before section
\linespread{1.5}
\setcounter{secnumdepth}{0}		% Remove default number tags from sections
\setmainfont{Libertinus Serif}
\setsansfont{Libertinus Sans}
\setmonofont{Libertinus Mono}
\setdefaultlanguage{hebrew}
\setotherlanguage{english}

% operators
\DeclareMathOperator\cis{cis}
\DeclareMathOperator\Sp{Sp}
\DeclareMathOperator\tr{tr}
\DeclareMathOperator\im{Im}
\DeclareMathOperator\diag{diag}
\DeclareMathOperator*\lowlim{\underline{lim}}
\DeclareMathOperator*\uplim{\overline{lim}}

% commands
\renewcommand\qedsymbol{\textbf{משל}}
\newcommand{\NN}[0]{\mathbb{N}}
\newcommand{\ZZ}[0]{\mathbb{Z}}
\newcommand{\QQ}[0]{\mathbb{Q}}
\newcommand{\RR}[0]{\mathbb{R}}
\newcommand{\CC}[0]{\mathbb{C}}
\newcommand{\getenv}[2][] {
  \CatchFileEdef{\temp}{"|kpsewhich --var-value #2"}{\endlinechar=-1}
  \if\relax\detokenize{#1}\relax\temp\else\let#1\temp\fi
}
\newcommand{\explain}[2] {
	\begin{flalign*}
		 && \text{#2} && \text{#1}
	\end{flalign*}
}

% headers
\getenv[\AUTHOR]{AUTHOR}
\author{\AUTHOR}
\date\today


\begin{document}

\question{}
\subquestion[3]{}
נניח כי ישנן $X$ דבורים, כאשר $X_a$ הן ג'ו והשאר, קרי $X_b = X - X_b$, הן בטהובן.
עוד ידוע לנו שכמות הדבש הנוצרת ביום היא $\frac{X_a}{2} + \ln(X_b^2 + 1)$.
נגדיר אם כן $g : \RR_{\ge 0}^2 \to \RR_{\ge 0}$ על־ידי,
\[
	g(x, y)
	= \frac{x}{2} + \ln(y^2 + 1)
\]
פונקציה המייצגת את כמות הדבש הנוצרת ביום כתלות בכמות סוגי הדבורים.
נתון כי $g(x, y) = C$ עבור $C = 200000$.
נבחין כי גם,
\[
	\nabla g = (\frac{1}{2}, \frac{2y}{y^2 + 1})
\]
נגדיר $f(x, y) = x + y$ כמות הדבורים הכוללת, ונרצה למצוא את ערכי הקיצון שלה תחת האילוץ $g = C$, נבחין כי גם,
\[
	\nabla f = (1, 1)
\]
לכן משיטת כופלי לגרנז' נובע,
\[
	(1, 1) = \lambda (\frac{1}{2}, \frac{2y}{y^2 + 1})
\]
בפרט $1 = \lambda \cdot \frac{1}{2} \iff \lambda = 2$ ומשוויון האגף השני,
\[
	1 = 2 \cdot \frac{2y}{y^2 + 1}
	\iff y^2 - 4y + 1 = 0
	\iff y = 2 \pm \sqrt{3}
\]
ומהשוויון $g = C$ נסיק,
\[
	x = 2(C - \ln(7 \pm 4 \sqrt{3}))
\]
כלומר מצאנו שהנקודות $(2(C - \ln(7 \pm 4 \sqrt{3})), 2 \pm \sqrt{3})$ הן נקודות קיצון בתחום.
נחשב ונקבל שגם
\[
	f(2(C - \ln(7 + 4 \sqrt{3})), 2 + \sqrt{3}) 
	\approx 399998.32,
	\qquad
	f(2(C - \ln(7 - 4 \sqrt{3})), 2 - \sqrt{3})
	\approx 400000.12
\]
ולכן $(2(C - \ln(7 + 4 \sqrt{3})), 2 + \sqrt{3})$ מינימום ונובע $X_a \approx 399999.86, X_b \approx 0.26$.

\end{document}
