\newcounter{english}
\documentclass[a4paper]{article}

% packages
\usepackage{inputenc, fontspec, amsmath, amsthm, amsfonts, polyglossia, catchfile}
\usepackage[a4paper, margin=50pt, includeheadfoot]{geometry} % set page margins

% style
\AddToHook{cmd/section/before}{\clearpage}	% Add line break before section
\linespread{1.5}
\setcounter{secnumdepth}{0}		% Remove default number tags from sections
\setmainfont{Libertinus Serif}
\setsansfont{Libertinus Sans}
\setmonofont{Libertinus Mono}
\setdefaultlanguage{hebrew}
\setotherlanguage{english}

% operators
\DeclareMathOperator\cis{cis}
\DeclareMathOperator\Sp{Sp}
\DeclareMathOperator\tr{tr}
\DeclareMathOperator\im{Im}
\DeclareMathOperator\diag{diag}
\DeclareMathOperator*\lowlim{\underline{lim}}
\DeclareMathOperator*\uplim{\overline{lim}}

% commands
\renewcommand\qedsymbol{\textbf{משל}}
\newcommand{\NN}[0]{\mathbb{N}}
\newcommand{\ZZ}[0]{\mathbb{Z}}
\newcommand{\QQ}[0]{\mathbb{Q}}
\newcommand{\RR}[0]{\mathbb{R}}
\newcommand{\CC}[0]{\mathbb{C}}
\newcommand{\getenv}[2][] {
  \CatchFileEdef{\temp}{"|kpsewhich --var-value #2"}{\endlinechar=-1}
  \if\relax\detokenize{#1}\relax\temp\else\let#1\temp\fi
}
\newcommand{\explain}[2] {
	\begin{flalign*}
		 && \text{#2} && \text{#1}
	\end{flalign*}
}

% headers
\getenv[\AUTHOR]{AUTHOR}
\author{\AUTHOR}
\date\today


\usepackage{ifthen}

\begin{document}

\question{}
\subquestion{}
\begin{theorem}
	שלום לכם
\end{theorem}
\begin{proof}
	Nope
\end{proof}
\begin{definition}
	Definition
\end{definition}
\begin{solution}
	asdsad
\end{solution}
נגדיר $\Omega = [6]$ עם הסתברות אחידה $\PP$.
נגדיר $X(\omega) = \omega$ ו־$Y(\omega) = \omega$
\[
	Supp X = [6]
	\qquad
	Supp Y = [6]
\]
נגדיר $Z = X + Y$.
\[
	Supp Z = 2 \cdot [6]
\]
עכשיו נגדיר $X$ הטלת קובייה ו־$Y$ הטלת קובייה שנייה. עכשיו $\Omega = {[6]}^2$.
\[
	X((a, b)) = a,
	\qquad
	Y((a, b)) = b,
	\qquad
	Z((a, b)) = a + b
	\qquad
	Supp Z = [12]
\]
נחשב
\[
	\PP(Z = 4) = \PP(\{(1, 3), (3, 1), (2, 2)\}) = \frac{3}{36}
	= \PP(\{\omega \in \Omega \mid Z(\omega) = 4\})
\]

$Supp\ X = Supp\ Y = \{0, \dots, n - 1\}$ וגם $Z = X + Y \mod n$.
\[
	Supp\ Z = Supp\ X = \{0, \dots, n - 1\}
\]
נניח ש־$X, Z$ בלתי־תלויים. נראה ש־$\PP(Y = \omega) = \frac{1}{n}$.
\begin{align*}
	\PP(X = k) \cdot \PP(Z = l)
	& = \PP(X = k, Z = l) \\
	& = \PP(X = k, X + Y \mod n = l) \\
	& = \PP(X = k, X + Y = l + a n) \\
	& = \PP(X = k, X + Y \in \{l, l + n\}) \\
	& = \PP(X = k, k + Y \in \{l, l + n\}) \\
	& = \PP(X = k, Y \in \{l - k, l - k + n\}) \\
	& = \PP(X = k) \PP(Y \in \{l - k, l - k + n\}) \\
\end{align*}
ולכן
\[
	\PP(Z = l) = \PP(Y \in \{l - k, l - k + n\})
\]
נחפש את
\[
	\PP(Y = m)
\]
נקבל $m = l - k$ או $m = l - k + n$.
\[
	\PP(Y = m) = \PP(Z = l)
\]
נכתוב $l = m + k$ או $l = m + k - n$.
\[
	\PP(Y = m) = \PP(Z = m + k)
\]
נגדיר $k = i, j$
אז
\[
	\PP(Y = m) = \PP(Z = m + i) = \PP(Z = m + j)
\]
לדוגמה אם $m = 0$ אז
\[
	\PP(Z = i) = \PP(Z = j)
\]
לכל $0 \le i, j \le n - 1$.
לכן
\[
	\PP(Z = l) = \frac{1}{n}
\]
אז נובע
\[
	\PP(Y = m) = \frac{1}{n}
\]

\end{document}
