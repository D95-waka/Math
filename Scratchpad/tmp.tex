\documentclass[a4paper]{article}

% packages
\usepackage{inputenc, amsmath, amsthm, thmtools, amsfonts, amssymb, luacode, catchfile, tikzducks, hyperref}
\usepackage[a4paper, margin=50pt, includeheadfoot]{geometry} % set page margins
\usepackage[shortlabels]{enumitem}
\usepackage[skip=3pt, indent=0pt]{parskip}

% language
\usepackage[bidi=basic, layout=tabular, provide=*]{babel}
\babelprovide[main, import]{hebrew}
\babelprovide{rl}
\babelfont{rm}{Libertinus Serif}
\babelfont{sf}{Libertinus Sans}
\babelfont{tt}{Libertinus Mono}

% style
\AddToHook{cmd/section/before}{\clearpage}	% Add line break before section
\linespread{1.3}
\setcounter{secnumdepth}{0}		% Remove default number tags from sections, this won't do well with theorems
\AtBeginDocument{\setlength{\belowdisplayskip}{3pt}}
\AtBeginDocument{\setlength{\abovedisplayskip}{3pt}}

% operators
\DeclareMathOperator\cis{cis}
\DeclareMathOperator\Sp{Sp}
\DeclareMathOperator\tr{tr}
\DeclareMathOperator\im{Im}
\DeclareMathOperator\re{Re}
\DeclareMathOperator\diag{diag}
\DeclareMathOperator*\lowlim{\underline{lim}}
\DeclareMathOperator*\uplim{\overline{lim}}
\DeclareMathOperator\rng{rng}
\DeclareMathOperator\Sym{Sym}
\DeclareMathOperator\Arg{Arg}
\DeclareMathOperator\Log{Log}
\DeclareMathOperator\dom{dom}

% commands
%\renewcommand\qedsymbol{\textbf{מש''ל}}
%\renewcommand\qedsymbol{\fbox{\emoji{lizard}}}
\newcommand{\NN}[0]{\mathbb{N}}
\newcommand{\ZZ}[0]{\mathbb{Z}}
\newcommand{\QQ}[0]{\mathbb{Q}}
\newcommand{\RR}[0]{\mathbb{R}}
\newcommand{\CC}[0]{\mathbb{C}}
\newcommand{\FF}[0]{\mathbb{F}}
\newcommand{\PP}[0]{\mathbb{P}}
\newcommand{\TT}[0]{\mathbb{T}}
\newcommand{\acts}[0]{\circlearrowright}
\newcommand{\explain}[2] {
	\begin{flalign*}
		 && \text{#2} && \text{#1}
	\end{flalign*}
}
\newcommand{\maketitleprint}[0]{ \begin{center}
	\begin{tikzpicture}[scale=3]
		\duck[graduate=gray!20!black, tassel=red!70!black]
	\end{tikzpicture}	
\end{center}
}

% theorem commands
\newtheoremstyle{c_remark}
	{}	% Space above
	{}	% Space below
	{}% Body font
	{}	% Indent amount
	{\bfseries}	% Theorem head font
	{}	% Punctuation after theorem head
	{.5em}	% Space after theorem head
	{\thmname{#1}\thmnumber{ #2}\thmnote{ \normalfont{\text{(#3)}}}}	% head content
\newtheoremstyle{c_definition}
	{3pt}	% Space above
	{3pt}	% Space below
	{}% Body font
	{}	% Indent amount
	{\bfseries}	% Theorem head font
	{}	% Punctuation after theorem head
	{.5em}	% Space after theorem head
	{\thmname{#1}\thmnumber{ #2}\thmnote{ \normalfont{\text{(#3)}}}}	% head content
\newtheoremstyle{c_plain}
	{3pt}	% Space above
	{3pt}	% Space below
	{\itshape}% Body font
	{}	% Indent amount
	{\bfseries}	% Theorem head font
	{}	% Punctuation after theorem head
	{.5em}	% Space after theorem head
	{\thmname{#1}\thmnumber{ #2}\thmnote{ \text{(#3)}}}	% head content

\theoremstyle{c_plain}
\newtheorem{theorem}{משפט}[section]
\newtheorem{lemma}[theorem]{למה}
\newtheorem{proposition}[theorem]{טענה}
\newtheorem*{proposition*}{טענה}
%\newtheorem{corollary}[theorem]{אין חלופה עברית}

\theoremstyle{c_definition}
\newtheorem{definition}[theorem]{הגדרה}
\newtheorem*{definition*}{הגדרה}
\newtheorem{example}{דוגמה}[section]
\newtheorem{exercise}{תרגיל}[section]

\theoremstyle{c_remark}
\newtheorem*{remark}{הערה}
\newtheorem*{solution}{פתרון}
\newtheorem{conclusion}[theorem]{מסקנה}
\newtheorem{notation}[theorem]{סימון}

% Questions related commands
\newcounter{question}
\setcounter{question}{1}
\newcounter{sub_question}
\setcounter{sub_question}{1}

\newcommand{\question}[1][0]{
	\ifthenelse{#1 = 0}{}{\setcounter{question}{#1}}
	\subsection{שאלה \arabic{question}}
	\addtocounter{question}{1}
	\setcounter{sub_question}{1}
}

\newcommand{\subquestion}[1][0]{
	\ifthenelse{#1 = 0}{}{\setcounter{sub_question}{#1}}
	\subsubsection{סעיף \localecounter{letters.gershayim}{sub_question}}
	\addtocounter{sub_question}{1}
}

% import lua and start of document
\directlua{common = require ('../common')}

\GetEnv{AUTHOR}

% headers
\author{\AUTHOR}
\date\today

\title{tmp}

\begin{document}

\Question{}
\Subquestion{}
נגדיר סדרה
\[
	a_n = \begin{cases}
		\frac{n}{2} & n \% 2 = 0 \\
		0 & n \% 2 = 1 \\
	\end{cases}
\]
תחילה נגדיר $n_k = 2k$ סדרת האינדקסים הזוגית, אז כמובן נובע כי תת־הסדרה $a_{n_k} = \frac{n}{2}$. \\*
אנו יכולים להסיק בנקל כי $\lim_{k \to \infty} a_{n_k} = \lim_{n \to \infty} \frac{n}{2} = \infty$. \\*
נגדיר עתה גם כי $m_k = 2k - 1$ סדרת האינדקסים האי־זוגית, ומהגדרת $(a_n)$ נובע כי $a_{m_k} = 0$, ובהתאם גבול תת־הסדרה הוא $0$ גם כן. \\*
מצאנו שתי תת־סדרות אשר מכסות את הסדרה $(a_n)$ ולכן שני הגבולות היחידים של הסדרה הם $0$ ו־$\infty$.

\Subquestion{}
נגדיר את הסדרה $(a_n)$ בצורה הבאה:
\[
	a_n = (1, 1, \frac{1}{2}, 1, \frac{1}{2}, \frac{1}{3}, 1 \dots)
\]
סדרה חזרתית של הסדרה $\frac{1}{n}$ עבור ערכים עולים. \\*
יהי $m \in \NN$ ונגדיר $\lambda = \frac{1}{m}$. אנו יודעים מהגדרת הסדרה $(a_n)$ כי קיים $n \in \NN$ עבורו מתקיים $a_n = \lambda$. \\*
למעשה, מהגדרת החזרתיות של $(a_n)$ נובע כי ישנם אינסוף אינדקסים עבורם מתקיים $a_n = \lambda$,
ולכן נוכל להגדיר תת־סדרה $n_k$ כך שמתקיים $a_{n_k} = \lambda$. בהתאם מתקיים $\lim_{k \to \infty} a_{n_k} = \lim_{k \to \infty} \lambda = \lambda$.
מצאנו אם כך כי כל $\lambda$ כזה הוא גבול חלקי של הסדרה $(a_n)$. \\*
עתה נשים לב כי גם $0$ גבול חלקי של הסדרה, זאת נוכל לקבל על־ידי הגדרת סדרת אינדקסים עבורה תת־הסדרה $n_k$ מקיימת $a_{n_k} = \frac{1}{n}$. \\*
אנו יכולים להסיק כי קיימת סדרה כזו מהגדרת הסדרה, הסדרה מורכבת מתבנית הולכת ונמשכת, ובכל פעם ''נוסף'' איבר חדש בדמות $\frac{1}{n}$. כמובן גם
\[
	\lim_{k \to \infty} a_{n_k} = \lim_{n \to \infty} \frac{1}{n} = 0
\]
וקיבלנו כי קבוצת הגבולות החלקיים של $(a_n)$ היא $\{ 0, \frac{1}{n} \mid n \in \NN \}$. \\*
אנו יודעים כי לכל $n \in \NN$ מתקיים $0 < a_n \le 1$ על־פי הגדרת הסדרה, לכן כמובן $\sup a_n = 1$ וגם $\inf a_n = 0$. \\*
מחסימות זו נסיק כי לכל $L \in \RR$ אשר לא מקיים $0 \le L \le 1$ בהכרח $L$ איננו גבול חלקי של הסדרה, 
שאם לא כן נקבל כי קיימים בסדרה ערכים שחוצים את החסמים העליונים או התחתונים שלה. \\*
לדוגמה, אם נבחר $L = 2$, אז מהגדרת הגבול על־פי היינה ו־$\epsilon = \frac{1}{2}$ נקבל כי קיימים אינסוף ערכים של $(a_n)$ אשר מקיימים
\[
	| a_n - 2 | < \frac{1}{2} \rightarrow -\frac{1}{2} < a_n - 2 < \frac{1}{2} \rightarrow 1 \frac{1}{2} < a_n < 2 \frac{1}{2}
\]
וזאת בסתירה לטענה כי $a_n \le 1$ לכל $n$ טבעי. \\*
לסיכום מצאנו כי קבוצת הגבולות החלקיים של $(a_n)$ היא $\{ 0 \} \cup \{ \frac{1}{n} \mid n \in \NN \}$.

\end{document}

