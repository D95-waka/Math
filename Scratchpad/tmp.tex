\documentclass[a4paper]{article}

% packages
\usepackage{inputenc, fontspec, amsmath, amsthm, amsfonts, polyglossia, catchfile}
\usepackage[a4paper, margin=50pt, includeheadfoot]{geometry} % set page margins

% style
\AddToHook{cmd/section/before}{\clearpage}	% Add line break before section
\linespread{1.5}
\setcounter{secnumdepth}{0}		% Remove default number tags from sections
\setmainfont{Libertinus Serif}
\setsansfont{Libertinus Sans}
\setmonofont{Libertinus Mono}
\setdefaultlanguage{hebrew}
\setotherlanguage{english}

% operators
\DeclareMathOperator\cis{cis}
\DeclareMathOperator\Sp{Sp}
\DeclareMathOperator\tr{tr}
\DeclareMathOperator\im{Im}
\DeclareMathOperator\diag{diag}
\DeclareMathOperator*\lowlim{\underline{lim}}
\DeclareMathOperator*\uplim{\overline{lim}}

% commands
\renewcommand\qedsymbol{\textbf{משל}}
\newcommand{\NN}[0]{\mathbb{N}}
\newcommand{\ZZ}[0]{\mathbb{Z}}
\newcommand{\QQ}[0]{\mathbb{Q}}
\newcommand{\RR}[0]{\mathbb{R}}
\newcommand{\CC}[0]{\mathbb{C}}
\newcommand{\getenv}[2][] {
  \CatchFileEdef{\temp}{"|kpsewhich --var-value #2"}{\endlinechar=-1}
  \if\relax\detokenize{#1}\relax\temp\else\let#1\temp\fi
}
\newcommand{\explain}[2] {
	\begin{flalign*}
		 && \text{#2} && \text{#1}
	\end{flalign*}
}

% headers
\getenv[\AUTHOR]{AUTHOR}
\author{\AUTHOR}
\date\today


\begin{document}

ענו על השאלות הבאות לפי הסדר.
\begin{enumerate}
	\item צבעים הם תחום מורכב ומשונה, ואין בכוונתנו לאפיין את הדרך שבה הם עובדים.
		אבל צבע מונוכרומטי, קרי צבע שהוא בין שחור ללבן, הוא פשוט הרבה יותר.
		אפיינו דרך לשמור מידע על צבע בין שחור ללבן בעזרת מחשב (ספציפית בקוד של שפת c).
	\item עתה כשיש לנו את היכולת להגדיר צבע בודד, נרצה להשתמש ביכולת זו כדי לייצג תמונה שלמה במחשב.
		אפיינו דרך לשמור תמונה שלמה על־ידי שימוש בסעיף הקודם.
	\item איך נוכל לנצל את הדרך שמצאנו לייצוג תמונה כדי לייצג וידאו?
	\item יש לנו תמונה, עתה נרצה לשמור אותה לקובץ, איך נוכל לעשות זאת?
		אפיינו את מבנה הקובץ, האם הוא יהיה בינארי או טקסטואלי?
	\item קיבלנו תמונה הפוכה (מסובבת ב־180 מעלות), השתמשו במבנה התמונה שקיבלתם כדי לאפיין מנגנון שמסובב תמונות.
	\item כל גרפיקאי ששווה את לחמו יודע לשנות את החדות של תמונה,
		אפיינו בבקשה אלגוריתם שמשתמש בתמונה שלכם כדי לשפר את חדות התמונה. \\
		\textbf{רמז:} נבחין שככל שצבע חזק יותר, ככה הוא צריך להיות חזק לאחר האלגוריתם אף יותר.
	\item נרצה להיות מסוגלים לטשטש את התמונה, מצאו אלגוריתם שעושה זאת, מה שם הפעולה המתמטית המתאימה?
	\item איך נוכל לייצג תמונה עם צבע? \\
		\textbf{רמז:} קראו על RGB\@.
\end{enumerate}

\end{document}
