\documentclass[a4paper]{article}

% packages
\usepackage{inputenc, fontspec, amsmath, amsthm, amsfonts, polyglossia, catchfile}
\usepackage[a4paper, margin=50pt, includeheadfoot]{geometry} % set page margins

% style
\AddToHook{cmd/section/before}{\clearpage}	% Add line break before section
\linespread{1.5}
\setcounter{secnumdepth}{0}		% Remove default number tags from sections
\setmainfont{Libertinus Serif}
\setsansfont{Libertinus Sans}
\setmonofont{Libertinus Mono}
\setdefaultlanguage{hebrew}
\setotherlanguage{english}

% operators
\DeclareMathOperator\cis{cis}
\DeclareMathOperator\Sp{Sp}
\DeclareMathOperator\tr{tr}
\DeclareMathOperator\im{Im}
\DeclareMathOperator\diag{diag}
\DeclareMathOperator*\lowlim{\underline{lim}}
\DeclareMathOperator*\uplim{\overline{lim}}

% commands
\renewcommand\qedsymbol{\textbf{משל}}
\newcommand{\NN}[0]{\mathbb{N}}
\newcommand{\ZZ}[0]{\mathbb{Z}}
\newcommand{\QQ}[0]{\mathbb{Q}}
\newcommand{\RR}[0]{\mathbb{R}}
\newcommand{\CC}[0]{\mathbb{C}}
\newcommand{\getenv}[2][] {
  \CatchFileEdef{\temp}{"|kpsewhich --var-value #2"}{\endlinechar=-1}
  \if\relax\detokenize{#1}\relax\temp\else\let#1\temp\fi
}
\newcommand{\explain}[2] {
	\begin{flalign*}
		 && \text{#2} && \text{#1}
	\end{flalign*}
}

% headers
\getenv[\AUTHOR]{AUTHOR}
\author{\AUTHOR}
\date\today

\title{tmp}
\usepackage{cancel}
% chktex-file 8
% chktex-file 36
% chktex-file 12
% chktex-file 26
% chktex-file 38

\begin{document}
\directlua{
	function map(tbl, f)
		local t = {}
		for k,v in pairs(tbl) do
			t[k] = f(v)
		end
		return t
	end

	function equals(o1, o2, ignore_mt)
		if o1 == o2 then return true end
		local o1Type = type(o1)
		local o2Type = type(o2)
		if not (o1Type == o2Type) then return false end
		if not (o1Type == 'table') then return false end

		if not ignore_mt then
			local mt1 = getmetatable(o1)
			if mt1 and mt1.__eq then
				return o1 == o2
			end
		end

		local keySet = {}

		for key1, value1 in pairs(o1) do
			local value2 = o2[key1]
			if value2 == nil or equals(value1, value2, ignore_mt) == false then
				return false
			end
			keySet[key1] = true
		end

		for key2, _ in pairs(o2) do
			if not keySet[key2] then return false end
		end
		return true
	end

	function indexOf(array, value)
		for i, v in ipairs(array) do
			if equals(v, value, false) then
				return i
			end
		end

		return nil
	end

	function point_text(point)
		return '(' .. point.x .. ',' .. point.y .. ')'
	end

	function compile_permutation(ord)
		return function(set, item)
			local id = indexOf(set, item)
			if id == nil then
				return nil
			end

			return set[ord[id]]
		end
	end

	points = {
		{x = 0, y = 0},
		{x = 1, y = 0},
		{x = 1, y = 1},
		{x = 0, y = 1}
	}

	group_definitions = {
		{1, 2, 3, 4},
		{2, 3, 4, 1},
		{1, 3, 2, 4},
	}

	group = map(group_definitions, function(def)
		local compiled = compile_permutation(def)
		return function(point)
			return compiled(points, point)
		end
	end)

	%text = table.concat(map(map(points, function(p) return group[3](points, p) end), point_text), " ")
	%text = table.concat(map(points, function(p) return indexOf(points, p) end), " ")
}

\newcommand{\drawRect}[1] {
	\begin{center}
		\begin{tikzpicture}[scale=3]
			\draw \directlua{
				c_points = map(points, group[#1])
				tex.print(table.concat(map(map(c_points, function(p) return {x = p.x - 1, y = p.y} end), point_text), ' -- '))
			} -- cycle;

			\directlua{ for key, point in ipairs(c_points) do
				tex.print('\\node at ' .. point_text(point) .. ' [above] {' .. key .. ': ' .. point_text(point) .. '};' )
				end
			}
		\end{tikzpicture}
	\end{center}
}

נתחיל בהסבר.
המסמך הזה מדגים את הקונספט של ההבדל בין קבוצה, חבורה, והפעולה של החבורה על הקבוצה. 
נגדיר את $X$ להיות קבוצה כלשהי, ואת $G$ להיות ''חבורת'' תמורות. היא לא באמת חבורה כי לא הוספנו אליה את כלל האיברים כדי שהיא תקיים את תכונות החבורה, ולכן היא לא חבורה אמיתית. 
החבורה עצמה, כפי שאפשר לראות בקוד של המסמך, היא מערך של פונקציות (מטעמי נוחות הן מקבלות רק ארגומנט אחד ולא שניים) ו־$G$ עצמה היא בסך הכול הסימון שמהם הפונקציות נוצרות.
הרעיון הוא שהקבוצה מכילה נקודות, משהו שאפשר לצייר, והחבורה מכילה פונקציות שמשנות את הסדר של רשימה. 
לכן בכל פעם ניתן את רשימת הנקודות לפונקציה מסוימת שמייצגת תמורה, ונקבל רשימה חדשה של נקודות.
לאחר מכן נדפיס את הנקודות האלה באופן גאומטרי, כאשר ליד כל נקודה מופיע מספור של הסדר שלה במערך זה.

נשים לב כי $D_4$ בהגדרה אמור לשמר את המבנה של ריבוע גאומטרי, ואכן כאשר מפעילים $g \in D_4$ על $X$ נקבל ריבוע שוב, גם אם הנקודות לא הודפסו בסדר המקורי שלהן.
לעומת זאת, כאשר ניקח $g \in S_4$ כך ש־$g \notin D_4$ רשימת הנקודות לא בהכרח תשמר את המבנה של ריבוע, כפי שנראה במקרה $g_3$.

ועתה להדגמה, כל מה שמופיע מכאן והלאה מבוסס על חישוב מעל הגדרות הקבוצה והחבורה המופיעות למעלה:
\[
	X = \{
		\directlua{tex.print(table.concat(map(points, function(point) return point_text(point) end), ', '))}
	\},
	G = \{
		\directlua{tex.print(table.concat(map(group_definitions, function(a) return "(" .. table.concat(a, ", ") .. ")" end), ', '))}
	\}
\]

\directlua{tex.print(text)}

\directlua{
	for key, value in pairs(group_definitions) do
		tex.print('עבור התמורה $g_' .. key .. ' = (' .. table.concat(value, ", ") .. ')$ נקבל:')
		tex.print('\\drawRect{' .. key .. '}')
	end
}

\end{document}

