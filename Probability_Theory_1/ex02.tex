\documentclass[a4paper]{article}

% packages
\usepackage{inputenc, amsmath, amsthm, thmtools, amsfonts, amssymb, luacode, catchfile, tikzducks, hyperref}
\usepackage[a4paper, margin=50pt, includeheadfoot]{geometry} % set page margins
\usepackage[shortlabels]{enumitem}
\usepackage[skip=3pt, indent=0pt]{parskip}

% language
\usepackage[bidi=basic, layout=tabular, provide=*]{babel}
\babelprovide[main, import]{hebrew}
\babelprovide{rl}
\babelfont{rm}{Libertinus Serif}
\babelfont{sf}{Libertinus Sans}
\babelfont{tt}{Libertinus Mono}

% style
\AddToHook{cmd/section/before}{\clearpage}	% Add line break before section
\linespread{1.3}
\setcounter{secnumdepth}{0}		% Remove default number tags from sections, this won't do well with theorems
\AtBeginDocument{\setlength{\belowdisplayskip}{3pt}}
\AtBeginDocument{\setlength{\abovedisplayskip}{3pt}}

% operators
\DeclareMathOperator\cis{cis}
\DeclareMathOperator\Sp{Sp}
\DeclareMathOperator\tr{tr}
\DeclareMathOperator\im{Im}
\DeclareMathOperator\re{Re}
\DeclareMathOperator\diag{diag}
\DeclareMathOperator*\lowlim{\underline{lim}}
\DeclareMathOperator*\uplim{\overline{lim}}
\DeclareMathOperator\rng{rng}
\DeclareMathOperator\Sym{Sym}
\DeclareMathOperator\Arg{Arg}
\DeclareMathOperator\Log{Log}
\DeclareMathOperator\dom{dom}

% commands
%\renewcommand\qedsymbol{\textbf{מש''ל}}
%\renewcommand\qedsymbol{\fbox{\emoji{lizard}}}
\newcommand{\NN}[0]{\mathbb{N}}
\newcommand{\ZZ}[0]{\mathbb{Z}}
\newcommand{\QQ}[0]{\mathbb{Q}}
\newcommand{\RR}[0]{\mathbb{R}}
\newcommand{\CC}[0]{\mathbb{C}}
\newcommand{\FF}[0]{\mathbb{F}}
\newcommand{\PP}[0]{\mathbb{P}}
\newcommand{\TT}[0]{\mathbb{T}}
\newcommand{\acts}[0]{\circlearrowright}
\newcommand{\explain}[2] {
	\begin{flalign*}
		 && \text{#2} && \text{#1}
	\end{flalign*}
}
\newcommand{\maketitleprint}[0]{ \begin{center}
	\begin{tikzpicture}[scale=3]
		\duck[graduate=gray!20!black, tassel=red!70!black]
	\end{tikzpicture}	
\end{center}
}

% theorem commands
\newtheoremstyle{c_remark}
	{}	% Space above
	{}	% Space below
	{}% Body font
	{}	% Indent amount
	{\bfseries}	% Theorem head font
	{}	% Punctuation after theorem head
	{.5em}	% Space after theorem head
	{\thmname{#1}\thmnumber{ #2}\thmnote{ \normalfont{\text{(#3)}}}}	% head content
\newtheoremstyle{c_definition}
	{3pt}	% Space above
	{3pt}	% Space below
	{}% Body font
	{}	% Indent amount
	{\bfseries}	% Theorem head font
	{}	% Punctuation after theorem head
	{.5em}	% Space after theorem head
	{\thmname{#1}\thmnumber{ #2}\thmnote{ \normalfont{\text{(#3)}}}}	% head content
\newtheoremstyle{c_plain}
	{3pt}	% Space above
	{3pt}	% Space below
	{\itshape}% Body font
	{}	% Indent amount
	{\bfseries}	% Theorem head font
	{}	% Punctuation after theorem head
	{.5em}	% Space after theorem head
	{\thmname{#1}\thmnumber{ #2}\thmnote{ \text{(#3)}}}	% head content

\theoremstyle{c_plain}
\newtheorem{theorem}{משפט}[section]
\newtheorem{lemma}[theorem]{למה}
\newtheorem{proposition}[theorem]{טענה}
\newtheorem*{proposition*}{טענה}
%\newtheorem{corollary}[theorem]{אין חלופה עברית}

\theoremstyle{c_definition}
\newtheorem{definition}[theorem]{הגדרה}
\newtheorem*{definition*}{הגדרה}
\newtheorem{example}{דוגמה}[section]
\newtheorem{exercise}{תרגיל}[section]

\theoremstyle{c_remark}
\newtheorem*{remark}{הערה}
\newtheorem*{solution}{פתרון}
\newtheorem{conclusion}[theorem]{מסקנה}
\newtheorem{notation}[theorem]{סימון}

% Questions related commands
\newcounter{question}
\setcounter{question}{1}
\newcounter{sub_question}
\setcounter{sub_question}{1}

\newcommand{\question}[1][0]{
	\ifthenelse{#1 = 0}{}{\setcounter{question}{#1}}
	\subsection{שאלה \arabic{question}}
	\addtocounter{question}{1}
	\setcounter{sub_question}{1}
}

\newcommand{\subquestion}[1][0]{
	\ifthenelse{#1 = 0}{}{\setcounter{sub_question}{#1}}
	\subsubsection{סעיף \localecounter{letters.gershayim}{sub_question}}
	\addtocounter{sub_question}{1}
}

% import lua and start of document
\directlua{common = require ('../common')}

\GetEnv{AUTHOR}

% headers
\author{\AUTHOR}
\date\today

\title{פתרון מטלה 02 --- תורת ההסתברות (1), 80420}

\begin{document}
\maketitle
\maketitleprint{}

\Question{}
בוחרים באקראי סדרה של $n$ מספרים $[m]$ עם חזרות. נגדיר $p_m$ את ההסתברות שבסדרה שבחרנו יש מופע של אותו מספר שלוש פעמים לפחות, נניח גם ש־$n(m) = o(m^{2/3})$, נוכיח כי $\lim_{n \to \infty} p_m = 0$.
\begin{proof}
	על־פי הנתון מתקיים $\Omega = {[m]}^n$, וכן נגדיר פונקציית הסבתרות נקודתית אחידה, דהינו $p(\omega) = \frac{1}{m^n}$. \\*
	יהי $A_i$ עבור $i \in [m]$ המאורע ש־$i$ מופיע לפחות שלוש פעמים, אז מתקיים
	\[
		|A_i| = m^n - \binom{n}{2} {(m - 1)}^{n - 2} - \binom{n}{1} {(m - 1)}^{n - 1} - {(m - 1)}^n
	\]
	וכן $|\Omega| = m^n$, ולכן גם $\PP_p(A_i) = \frac{|A_i|}{|\Omega|}$. \\*
	עוד נגדיר $A = \bigcup_{i \in [m]} A_i$ המאורע שיש לפחות שלושה ממספר לאיזשהו מספר, אז מחסם האיחוד
	\[
		p_m
		= \PP(A)
		= \PP(\bigcup_{i \in [m]} A_i)
		\le \sum_{i \in [m]} \PP(A_i)
		= m \cdot \frac{|A_1|}{|\Omega|}
	\]
	ולכן
	\begin{align*}
		p_m
		& \le m \cdot \frac{1}{m^n} \cdot (m^n - \binom{n}{2} {(m - 1)}^{n - 2} - \binom{n}{1} {(m - 1)}^{n - 1} - {(m - 1)}^n) \\
		& = m - \binom{n}{2} \frac{{(m - 1)}^{n - 2}}{m^{n - 1}} - n \frac{{(m - 1)}^{n - 1}}{m^{n - 1}} - \frac{{(m - 1)}^n}{m^{n - 1}}
	\end{align*}
	לבסוף נבחין כי $n(m) = o(m^{2/3})$ ולכן $n \to \infty$ גורר $m \to \infty$, לכן נבחן את
	\[
		0 \le \lim_{m \to \infty} p_m
		\le \lim_{m \to \infty} m - \frac{1}{2} n (n - 1) \frac{{(m - 1)}^{n - 2}}{m^{n - 1}} - n \frac{{(m - 1)}^{n - 1}}{m^{n - 1}} - \frac{{(m - 1)}^n}{m^{n - 1}}
		= 0
	\]
	ולכן גם $\lim_{n \to \infty} p_m = 0$.
\end{proof}

\Question{}
בכל בוקר ילד מקבל מהוריו סכום קבוע לקנות חטיף. בכל חטיף נמצאות אחת מ־22 האותיות של האלפבית העברי בהסתברות שווה, ועל הילד להרכיב את המילה ''קטר''. \\*
נגדיר את האותיות לפי מספרים עד 22, ונגדיר שרירותית את האותיות ''קטר'' להיות 1 עד 3.

\Subquestion{}
עבור $n \in \NN$ נחשב את ההסתברות שביום ה־$n$ לילד לא הייתה האות $a$ עבור $a \in [3]$.
\begin{solution}
	לכל יום $\Omega_d = [22]$ עם פונקציית הסתברות אחידה $p(n) = \frac{1}{22}$, בהתאם לאחר $n$ ימים נקבל $\Omega = \Omega_d^n$, ואנו מחפשים את המאורע $A = \{ \omega \in \Omega \mid a \notin \omega\}$.
	נקבל $|A| = 21^n$, באופן דומה נקבל גם $|\Omega| = 22^n$, לכן
	\[
		\PP(A) = \frac{|A|}{|\Omega|} = \frac{21^n}{22^n}
	\]
\end{solution}

\Subquestion{}
נחשב את ההסתברות שלאחר $n$ ימים הילד עדיין לא הצליח להרכיב את המילה הרצויה על־ידי שימוש בנוסחת הכלה והדחה.
\begin{solution}
	נגדיר $A, B, C$ המאורע שלילד אין את האותיות הראשונה השנייה והשלישית לאחר $n$ ימים, מהכלה והפרדה נקבל
	\[
		\PP(A \cup B \cup C) = \PP(A) + \PP(B) + \PP(C) - \PP(A \cap B) - \PP(A \cap C) - \PP(B \cap C) + \PP(A \cap B \cap C)
	\]
	נבחין כי אנו מחפשים באמת את אחד מהמצבים בהם לפחות אחת מן האותיות חסרה, זהו אכן האיחוד של המאורעות, לעומת זאת מטעמי הסתברות אחידה נוכל כי
	\[
		\PP(A \cup B \cup C) = 3\PP(A) - 3\PP(A \cap B) + \PP(A \cap B \cap C)
	\]
	מצאנו כי $\PP(A) = \frac{21^n}{22^n}$, ובאופן דומה גם נוכל להסיק $\PP(A \cap B) = \frac{20^n}{22^n}$ ואף $\PP(A \cap B \cap C) = \frac{19^n}{22^n}$ ולכן
	\[
		\PP(A \cup B \cup C)
		= 3 \frac{21^n}{22^n} - 3 \frac{20^n}{22^n} + \frac{19^n}{22^n}
		= \frac{3 \cdot 21^n - 3 \cdot 20^n + 19^n}{22^n}
	\]
\end{solution}

\Question{}
בכד $n \ge 2$ כדורים ומתוכם אחד בצבע לבן והאחרים שחורים, מוציאים ללא החזרה שני כדורים ובוחנים את צבעיהם.

\Subquestion{}
נגדיר מרחב הסתברות מתאים.
\begin{solution}
	נגדיר מרחב הסתברות דו־שלבי, נתחיל בהגדרת $\Omega_1 = \{ B, W \}$, נתון כי $p(B) = \frac{n - 1}{n}$ וכי $p(W) = \frac{1}{n}$. \\*
	נעבור לניסוי השני, עבורו מתקיים $\Omega_2 = \Omega_1$, ונגדיר את פונקציית ההסתברות הנקודתית $p_W, p_B$ על־ידי
	\[
		p_W(W) = 0, p_W(B) = 1,
		p_B(W) = \frac{1}{n - 1}, p_B(B) = \frac{n - 2}{n - 1}
	\]
	בהתאם להגדרת הניסוי, לבסוף נגדיר את מרחב הניסוי $(\Omega_1 \times \Omega_2, \mathcal{F}_{1, 2}, \PP_q)$ עבור $q(a, b) = p(a) \cdot p_a(b)$.
\end{solution}

\Subquestion{}
נתון כי ההסתברות להוציא את הכדור הלבן כפולה מההסתברות שלא להוציא אותו, נמצא את $n$.
\begin{solution}
	נבחין כי ההסתברות לא להוציא את הכדור הלבן היא ההסתברות להוציא שני כדורים שחורים $q(B, B) = p(B) \cdot p_B(B) = \frac{n - 1}{n} \cdot \frac{n - 2}{n - 1} = \frac{n - 2}{n}$. \\*
	עוד נבחין כי ההסתברות להוציא כדור לבן היא $\PP(\{ WB, BW, WW \}) = q(W, B) + q(B, W) + q(W, W) = \frac{1}{n} \cdot 1 + \frac{n - 1}{n} \cdot \frac{1}{n - 1} + 0 = \frac{2}{n}$. \\*
	בהתאם נתון גם כי $2 \frac{n - 2}{n} = \frac{2}{n}$, ולכן נובע $n = 3$.
\end{solution}

\Question{}
$2^n$ שחקניות שונות אבל זהות בכישוריהן משתתפות בטורניר שחמט באורך $n$, ונניח שבכל משחק יש מנצחת ומפסידה בלבד, בסיבוב הראשון משחקות $2^n$ שחקניות, בסיבוב השני $2^{n - 1}$ המנצחות וכן הלאה.

\Subquestion{}
עבור $k \in \NN$ זוגי תהי $\Omega_p(k)$ קבוצת האפשרויות לחלק $k$ איברים לזוגות, נחשב את $|\Omega_p(k)|$.
\begin{solution}
	השאלה שקולה למספר התמורות המורכבות ממחזורים זוגיים נפרדים, דהינו נבחר כל פעם 2 ויצור ציוות שלהם, תוך הורדת המספר בהתאם, אם נתחיל ב־$k$ ואפס אפשרויות, נקבל $k - 2$ ו־$\binom{k}{2}$. \\*
	בשלב השני נקבל $k - 4$ נשארו ו־$\binom{k}{2} \binom{k - 2}{2} = \frac{k!}{2^2 (k - 4)!}$, אם נמשיך תהליך זה נקבל $\frac{k!}{2^{k / 2}}$.
\end{solution}

\Subquestion{}
נחשב את ההסתברות ששתי שחקניות נתונות יפגשו בסיבוב הראשון.
\begin{solution}
	נגדיר $1, 2$ הזוג שאנו רוצים שיצוות, נקבל אם כן $A = \{ (x_1, \dots, x_{n / 2}) \in \Omega_p(k) \mid \exists \{ 1, 2 \} = x_i \}$.
\end{solution}

\Question{}
בוחרים באקראי באופן אחיד אחת מהקוביות $D_4, D_6, D_8$ מטילים אותה ומדווחים איזו קוביה נבחרה ומה תוצאת ההטלה, באופן זה מתקבלת הסתברות $\PP$ על $\Omega = \Omega_1 \times \Omega_2$ כאשר $\Omega_1 = \{ 4, 6, 8 \}$ ו־$\Omega = [8]$.

\Subquestion{}
נכתוב מפורשות את הניסוי הדו־שלבי שמגדיר את $\PP$.
\begin{solution}
	נתון כי הסתברות הניסוי הראשון היא אחידה, לכן $p(\omega) = \frac{1}{|\Omega_1|} = \frac{1}{3}$. \\*
	עתה נגדיר את הניסוי השני.
	עבור $\omega_1 = 4$ נקבל
	\[
		p_4(\omega_2) = \begin{cases}
			\frac{1}{4} & 1 \le \omega_2 \le 4 \\
			0 & 4 < \omega_2
		\end{cases}
	\]
	באופן דומה עבור $\omega_1 = 6$ נקבל
	\[
		p_6(\omega_2) = \begin{cases}
			\frac{1}{6} & 1 \le \omega_2 \le 6 \\
			0 & 6 < \omega_2
		\end{cases}
	\]
	ועבור $\omega_1 = 8$ נקבל $p_8(\omega_2) = \frac{1}{8}$. \\*
	לבסוף נגדיר $q(\omega_1, \omega_2) = p(\omega_1) \cdot p_{\omega_1}(\omega_2)$, ובהתאם $\PP = \PP_q$.
\end{solution}

\Subquestion{}
עתה נגדיר את פונקציית ההסתברות במקרה ההפוך, כאשר $\Omega_1' = [8], \Omega_2' = \{4, 6, 8\}$ כך שההסתברות נשארת זהה עבור $\Omega_2' \times \Omega_1'$ זהה ל־$\PP$ בסעיף הקודם.
\begin{solution}
	נתחיל במקרה הפשוט ביותר, אם $5 \le \omega_1 \le 8$ אז $p(\omega_1) = \frac{1}{3 \cdot 8}$ ו־$p_{\omega_1}(\omega_2) = 1$,
	זאת שכן רק קוביה $D_8$ מאפשרת קבלת מספר מטווח זה. \\*
	נעבור עתה למקרה $6 \le \omega_1 \le 7$, במקרה זה עדיין $p(\omega_1) = \frac{1}{8}$ אך $p_{\omega_1}(\omega_2) = \frac{1}{2}$.
\end{solution}

\Question{}
יהי $(\Omega, \PP)$ מרחב הסתברות בדיד, עבור $A, B \subseteq \Omega$ נגדיר $\tilde{\PP}(A \times B) = \PP(A \cap B)$.

\Subquestion{}
נוכיח כי אם $|\Omega| > 1$ אז קיימות תת־קבוצות של $\Omega \times \Omega$ שלא ניתנות להצגה בצורה של $A \times B$ עבור $A, B \subseteq \Omega$.
\begin{proof}
	
\end{proof}

\end{document}
