\documentclass[a4paper]{article}

% packages
\usepackage{inputenc, fontspec, amsmath, amsthm, amsfonts, polyglossia, catchfile}
\usepackage[a4paper, margin=50pt, includeheadfoot]{geometry} % set page margins

% style
\AddToHook{cmd/section/before}{\clearpage}	% Add line break before section
\linespread{1.5}
\setcounter{secnumdepth}{0}		% Remove default number tags from sections
\setmainfont{Libertinus Serif}
\setsansfont{Libertinus Sans}
\setmonofont{Libertinus Mono}
\setdefaultlanguage{hebrew}
\setotherlanguage{english}

% operators
\DeclareMathOperator\cis{cis}
\DeclareMathOperator\Sp{Sp}
\DeclareMathOperator\tr{tr}
\DeclareMathOperator\im{Im}
\DeclareMathOperator\diag{diag}
\DeclareMathOperator*\lowlim{\underline{lim}}
\DeclareMathOperator*\uplim{\overline{lim}}

% commands
\renewcommand\qedsymbol{\textbf{משל}}
\newcommand{\NN}[0]{\mathbb{N}}
\newcommand{\ZZ}[0]{\mathbb{Z}}
\newcommand{\QQ}[0]{\mathbb{Q}}
\newcommand{\RR}[0]{\mathbb{R}}
\newcommand{\CC}[0]{\mathbb{C}}
\newcommand{\getenv}[2][] {
  \CatchFileEdef{\temp}{"|kpsewhich --var-value #2"}{\endlinechar=-1}
  \if\relax\detokenize{#1}\relax\temp\else\let#1\temp\fi
}
\newcommand{\explain}[2] {
	\begin{flalign*}
		 && \text{#2} && \text{#1}
	\end{flalign*}
}

% headers
\getenv[\AUTHOR]{AUTHOR}
\author{\AUTHOR}
\date\today

\title{פתרון מטלה 02 --- תורת ההסתברות (1), 80420}

\begin{document}
\maketitle
\maketitleprint{}

\Question{}
בכל בוקר ילד מקבל מהוריו סכום קבוע לקנות חטיף. בכל חטיף נמצאות אחת מ־22 האותיות של האלפבית העברי בהסתברות שווה, ועל הילד להרכיב את המילה ''קטר''. \\*
נגדיר את האותיות לפי מספרים עד 22, ונגדיר שרירותית את האותיות ''קטר'' להיות 1 עד 3.

\Subquestion{}
עבור $n \in \NN$ נחשב את ההסתברות שביום ה־$n$ לילד לא הייתה האות $a$ עבור $a \in [3]$.
\begin{solution}
	לכל יום $\Omega_d = [22]$ עם פונקציית הסתברות אחידה $p(n) = \frac{1}{22}$, בהתאם לאחר $n$ ימים נקבל $\Omega = \Omega_d^n$, ואנו מחפשים את המאורע $A = \{ \omega \in \Omega \mid a \notin \omega\}$.
	נקבל $|A| = 21^n$, באופן דומה נקבל גם $|\Omega| = 22^n$, לכן
	\[
		\PP(A) = \frac{|A|}{|\Omega|} = \frac{21^n}{22^n}
	\]
\end{solution}

\Subquestion{}
נחשב את ההסתברות שלאחר $n$ ימים הילד עדיין לא הצליח להרכיב את המילה הרצויה על־ידי שימוש בנוסחת הכלה והדחה.
\begin{solution}
	נגדיר $A, B, C$ המאורע שלילד אין את האותיות הראשונה השנייה והשלישית לאחר $n$ ימים, מהכלה והפרדה נקבל
	\[
		\PP(A \cup B \cup C) = \PP(A) + \PP(B) + \PP(C) - \PP(A \cap B) - \PP(A \cap C) - \PP(B \cap C) + \PP(A \cap B \cap C)
	\]
	נבחין כי אנו מחפשים באמת את אחד מהמצבים בהם לפחות אחת מן האותיות חסרה, זהו אכן האיחוד של המאורעות, לעומת זאת מטעמי הסתברות אחידה נוכל כי
	\[
		\PP(A \cup B \cup C) = 3\PP(A) - 3\PP(A \cap B) + \PP(A \cap B \cap C)
	\]
	מצאנו כי $\PP(A) = \frac{21^n}{22^n}$, ובאופן דומה גם נוכל להסיק $\PP(A \cap B) = \frac{20^n}{22^n}$ ואף $\PP(A \cap B \cap C) = \frac{19^n}{22^n}$ ולכן
	\[
		\PP(A \cup B \cup C)
		= 3 \frac{21^n}{22^n} - 3 \frac{20^n}{22^n} + \frac{19^n}{22^n}
		= \frac{3 \cdot 21^n - 3 \cdot 20^n + 19^n}{22^n}
	\]
\end{solution}

\Question{}
בכד $n \ge 2$ כדורים ומתוכם אחד בצבע לבן והאחרים שחורים, מוציאים ללא החזרה שני כדורים ובוחנים את צבעיהם.

\Subquestion{}
נגדיר מרחב הסתברות מתאים.
\begin{solution}
	נגדיר מרחב הסתברות דו־שלבי, נתחיל בהגדרת $\Omega_1 = \{ B, W \}$, נתון כי $p(B) = \frac{n - 1}{n}$ וכי $p(W) = \frac{1}{n}$. \\*
	נעבור לניסוי השני, עבורו מתקיים $\Omega_2 = \Omega_1$, ונגדיר את פונקציית ההסתברות הנקודתית $p_W, p_B$ על־ידי
	\[
		p_W(W) = 0, p_W(B) = 1,
		p_B(W) = \frac{1}{n - 1}, p_B(B) = \frac{n - 2}{n - 1}
	\]
	בהתאם להגדרת הניסוי, לבסוף נגדיר את מרחב הניסוי $(\Omega_1 \times \Omega_2, \mathcal{F}_{1, 2}, \PP_q)$ עבור $q(a, b) = p(a) \cdot p_a(b)$.
\end{solution}

\Subquestion{}
נתון כי ההסתברות להוציא את הכדור הלבן כפולה מההסתברות שלא להוציא אותו, נמצא את $n$.
\begin{solution}
	נבחין כי ההסתברות לא להוציא את הכדור הלבן היא ההסתברות להוציא שני כדורים שחורים $q(B, B) = p(B) \cdot p_B(B) = \frac{n - 1}{n} \cdot \frac{n - 2}{n - 1} = \frac{n - 2}{n}$. \\*
	עוד נבחין כי ההסתברות להוציא כדור לבן היא $\PP(\{ WB, BW, WW \}) = q(W, B) + q(B, W) + q(W, W) = \frac{1}{n} \cdot 1 + \frac{n - 1}{n} \cdot \frac{1}{n - 1} + 0 = \frac{2}{n}$. \\*
	בהתאם נתון גם כי $2 \frac{n - 2}{n} = \frac{2}{n}$, ולכן נובע $n = 3$.
\end{solution}

\Question{}
בוחרים באקראי באופן אחיד אחת מהקוביות $D_4, D_6, D_8$ מטילים אותה ומדווחים איזו קוביה נבחרה ומה תוצאת ההטלה, באופן 

\end{document} % chktex 17
