\documentclass[a4paper]{article}

% packages
\usepackage{inputenc, fontspec, amsmath, amsthm, amsfonts, polyglossia, catchfile}
\usepackage[a4paper, margin=50pt, includeheadfoot]{geometry} % set page margins

% style
\AddToHook{cmd/section/before}{\clearpage}	% Add line break before section
\linespread{1.5}
\setcounter{secnumdepth}{0}		% Remove default number tags from sections
\setmainfont{Libertinus Serif}
\setsansfont{Libertinus Sans}
\setmonofont{Libertinus Mono}
\setdefaultlanguage{hebrew}
\setotherlanguage{english}

% operators
\DeclareMathOperator\cis{cis}
\DeclareMathOperator\Sp{Sp}
\DeclareMathOperator\tr{tr}
\DeclareMathOperator\im{Im}
\DeclareMathOperator\diag{diag}
\DeclareMathOperator*\lowlim{\underline{lim}}
\DeclareMathOperator*\uplim{\overline{lim}}

% commands
\renewcommand\qedsymbol{\textbf{משל}}
\newcommand{\NN}[0]{\mathbb{N}}
\newcommand{\ZZ}[0]{\mathbb{Z}}
\newcommand{\QQ}[0]{\mathbb{Q}}
\newcommand{\RR}[0]{\mathbb{R}}
\newcommand{\CC}[0]{\mathbb{C}}
\newcommand{\getenv}[2][] {
  \CatchFileEdef{\temp}{"|kpsewhich --var-value #2"}{\endlinechar=-1}
  \if\relax\detokenize{#1}\relax\temp\else\let#1\temp\fi
}
\newcommand{\explain}[2] {
	\begin{flalign*}
		 && \text{#2} && \text{#1}
	\end{flalign*}
}

% headers
\getenv[\AUTHOR]{AUTHOR}
\author{\AUTHOR}
\date\today

\title{פתרון מטלה 03 --- תורת ההסתברות (1), 80420}

\begin{document}
\maketitle
\maketitleprint{}

\Question{}
נאמר שמאורע $A$ מחזק את מאורע $B$ אם $\PP(B \mid A) > \PP(B)$. \\*
יהי $(\Omega, \PP)$ מרחב הסתברות, ונוכיח או נפריך את הטענות הבאות.

\Subquestion{}
נסתור את הטענה כי אם $A$ מחזק את $B$ ו־$B$ מחזק את $C$, אז $A$ מחזק את $C$, על־ידי דוגמה נגדית.
\begin{solution}
	נניח $\Omega$ הטלת שתי קוביות הוגנות, נניח גם $A$ המאורע שיצא 2 בקוביה הראשונה, $B$ המאורע שיצא 2 לפחות בכל קוביה, ו־$C$ המאורע שיצא לפחות 2 בקוביה ב'. \\*
	נחשב $\PP(B \mid A) = \frac{5}{6} > \PP(B) = \frac{5^2}{6^2}$, בנוסף $\PP(C \mid B) = 1 > \PP(C) = \frac{5}{6}$ אבל $\PP(C \mid A) = \frac{5}{6} = \PP(C) = \frac{5}{6}$.
\end{solution}

\Subquestion{}
נוכיח כי אם $A$ מחזק את $B$ אז גם $B$ מחזק את $A$.
\begin{proof}
	ישירות מהגדרה
	\[
		\PP(B \mid A) > \PP(B)
		\iff \frac{\PP(A \cap B)}{\PP(A)} > \PP(B)
		\iff \frac{\PP(A \cap B)}{\PP(B)} > \PP(A)
		\iff \PP(A \mid B) > \PP(A)
	\]
\end{proof}

\Subquestion{}
נסתור את הטענה כי אם $A, B$ מאורעות המקיימים $\PP(A \cap B) = 0$ אז $A \cap B = \emptyset$.
\begin{solution}
	יהי $\Omega$ הטלת מטבע טריק, מטבע שבו תמיד צד א' נבחר. \\*
	נגדיר גם $A = \Omega$ ו־$B$ מאורע שיצא צד ב', אז כמובן $\PP(A \cap B) = \PP(B) = 0$, אבל $A \cap B$ הוא המקרה שיצא צד ב', ובפרט איננו מאורע ריק.
\end{solution}

\Subquestion{}
נסתור את הטענה כי אם $A, B, C$ מאורעות כך ש־$\PP(A \cap B \cap C) = 0$ וגם $\PP(A \cap B) = 0$ אז $\PP(A \cap C) = 0$.
\begin{solution}
	נגדיר $\Omega = \{0, 1\}$ עם $\PP$ אחידה, ונגדיר $A = C = \Omega, B = \emptyset$, אז נקבל ש־$A \cap B \cap C = A \cap B = \emptyset$ וגם כי $A \cap C = \Omega$, ולכן הטענה מתקיימת.
\end{solution}

\Subquestion{}
נסתור את הטענה כיאם $(\Omega, \PP)$ מרחב הסתברות אחידה, ונניח גם $B$ מאורע כך ש־$\PP(B) > 0$, אז $(\Omega, \PP_B)$ מרחב הסתברות אחידה.
\begin{solution}
	נבחן מרחב הסתברות של הטלת קוביה הוגנת, הוא עומד בכל התנאים, ואם $B = \{ 1, 2, 3 \}$ אז $\PP_B(\{1\}) = \frac{1}{3} \ne 0 = \PP_B(\{4\})$, דהינו מרחב ההסתברות $(\Omega, \PP_B)$ לא אחיד. \\*
	נבחין כי $(\Omega \cap B, \PP_B)$ הוא כן מרחב הסתברות אחיד.
\end{solution}

\Subquestion{}
נוכיח שאם $(\Omega, \PP)$ מרחב הסתברות לא אחיד ויהי $B$ כך ש־$\PP(B) > 0$, אז $(\Omega, \PP_B)$ מרחב הסתברות לא אחיד.
\begin{proof}
	מהנתון נסיק כי $\Omega$ לא ריק, ונבחין בין שני מקרים. \\*
	אם $\Omega = B$ אז $\PP = \PP_B$ ולכן סיימנו. \\*
	אחרת נגדיר $A = \Omega \setminus B$ ולכן $\PP_B(B) = 1 \ne 0 = \PP_B(A)$, ולכן מרחב ההסתברות הוא לא אחיד.
\end{proof}

\Question{}
\Subquestion{}
בשידה שלוש מגירות, באחת זוג גרביים שחור, בשנייה זוג גרביים לבן, ובשלישית גרב שחור וגרב לבן. \\*
בוחרים מגירה באקראי ובהסתברות אחידה ומוציאים ממנה גרס יחיד באקראי, ונתון כי הוא לבן. \\*
מה ההסתברות שגם הכרב השני במגירה לבן?
\begin{solution}
	השאלה שקולה לשאלה מה הסיכוי להוציא גרב לבן ואז גרב לבן נוסף, נגדיר $\Omega = \{ (w, w), (b, b), (w, b), (b, w) \}$. \\*
	נגדיר $A = \{ (w, w), (w, b) \}$ המאורע שהגרב הראשון שנבחר הוא לבן, עוד נבחין כי $p(w, w) = p(b, b) = p(w, b) + p(b, w)$ ו־$p(w, b) = p(b, w)$. \\*
	נגדיר $B = \{ (w, w) \}$ המאורע ששני הגרביים לבנים, ואנו מחפשים את $\PP(B \mid A)$, לכן
	\[
		\PP(B \mid A) = \frac{\PP(A \cap B)}{\PP(A)} = \frac{\PP(B)}{\PP(A)} = \frac{\frac{1}{3}}{\frac{1}{3} + \frac{1}{6}} = \frac{2}{3}
	\]
\end{solution}

\Subquestion{}
נתון דלי עם $k$ כדורים לבנים ו־$k$ כדורים שחורים.
מוציאים $n < k$ כדורים ללא החזרה ולאחר מכן מוציאים כדור נוסף, כדור $n + 1$.
נחשב מה ההסתברות אם ידוע ש־$n$ הכדורים הראשונים לבנים, מה ההסתברות שהכדור ה־$n + 1$ שחור.
\begin{solution}
	נגדיר $\Omega = {\{b, w\}}^{2k}$ כל הוצאות כל הכדורים ללא החזרה ועם חשיבות לסדר מהדלי. \\*
	נגדיר גם $A = \{ \omega \in \Omega \mid \forall 1 \le i \le n, \omega_i = w \}$ המאורע ש־$n$ הכדורים הראשונים הם לבנים,
	ו־$B = \{ \omega \in \Omega \mid \omega_{n + 1} = b \}$ המאורע שהכדור ה־$n + 1$ שחור. משיקולים קומבינטוריים נוכל להסיק
	\[
		|\Omega| = \binom{2k}{k},
		\qquad
		|A| = \binom{2k - n}{k - n},
		\qquad
		|B| = \binom{2k - 1}{k},
		\qquad
		|A \cap B| = \binom{2k - n - 1}{k - n}
	\]
	נחשב
	\[
		\PP(B \mid A)
		= \frac{\PP(A \cap B)}{\PP(A)}
		= \frac{\frac{|A \cap B|}{|\Omega|}}{\frac{|A|}{|\Omega|}}
		= \frac{|A \cap B|}{|A|}
		= \frac{\binom{2k - n - 1}{k - n}}{\binom{2k - n}{k - n}}
		= \frac{k}{2k - n}
	\]
\end{solution}

\Subquestion{}
נגדיר $\Omega \NN, p(n) = 2^{-n}$.
מגרילים מספר באקראי לפי $p$.
נחשב את ההסתברות שבהינתן שהמספר שהתקבל מתחלק ב־6, ההסתברות שהוא מתחלק ב־7, וההסתברות שהוא מתחלק ב־4.
\begin{solution}
	יהי $k \in \NN$ ונגדיר $l = \text{lcm}(6, k)$, אז אנו יודעים ש־$6 \NN \cap k \NN = l \NN$. \\*
	עוד נוכל לחשב שמתקיים לכל $m \in \NN$
	\[
		\PP(m\NN)
		= \sum_{n = 1}^{\infty} 2^{-mn}
		= \frac{2^{-m}}{1 - 2^{-m}}
		= \frac{1}{2^m - 1}
	\]
	ולכן
	\[
		\PP_6(k) \overset{\text{def}}{=} \PP(k \NN \mid 6 \NN)
		= \frac{\PP(l \NN)}{\PP(6\NN)}
		= \frac{\frac{1}{2^l - 1}}{\frac{1}{2^6 - 1}}
		= \frac{2^6 - 1}{2^{\text{lcm}(6, k)} - 1}
	\]
	לבסוף נציב
	\[
		\PP_6(7) = \frac{2^6 - 1}{2^{42} - 1},
		\qquad
		\PP_6(4) = \frac{2^6 - 1}{2^8 - 1}
	\]
\end{solution}

\Subquestion{}
בוחרים אחד מהמספרים $L = \{ \frac{1}{4}, \frac{1}{2}, \frac{3}{4} \}$ בהסתברות אחידה ואז מטילים מטבע מוטה בהתאם לפרמטר שנבחר פעמיים. \\*
נחשב את ההסתברות לכל אחד מהפרמטרים בהינתן שיצא צד א' ואז צד ב'.
\begin{solution}
	נגדיר $\Omega = L \times {[2]}^2$, וכן $A = L \times (1, 2)$, ונגדיר $B_i = \{ i \} \times {[2]}^2$.
	אנו מחפשים את $\PP(B_i \mid A)$, לכן נתחיל מחישובים הכרחיים:
	\[
		\PP(A) = \frac{1}{3} \frac{1}{4} \frac{3}{4} + \frac{1}{3} \frac{1}{2} \frac{1}{2} + \frac{1}{3} \frac{3}{4} \frac{1}{4} = \frac{5}{24},
		\PP(B_i) = \frac{1}{3}
	\]
	ובהתאם לחישובים האלה
	\[
		\PP(A \cap B_1) = \PP(A \cap B_3) = \frac{1}{16},
		\qquad
		\PP(A \cap B_2) = \frac{1}{12}
	\]
	ולכן
	\[
		\PP(B_1 \mid A) = \PP(B_2 \mid A)
		= \frac{\PP(B_1 \cap A)}{\PP(A)}
		= \frac{\frac{1}{16}}{\frac{5}{24}},
		\qquad
		\PP(B_2 \mid A) = \frac{\frac{1}{12}}{\frac{5}{24}}
	\]
\end{solution}

\Question{}

\Subquestion{}
נוכיח כי לכל שלושה מאורעות $A_1, A_2, A_3$ המקיימים $\PP(A_1 \cap A_2) > 0$ מתקיים
\[
	\PP(A_1 \cap A_2 \cap A_2) = \PP(A_1) \PP(A_2 \mid A_1) \PP(A_3 \mid A_1 \cap A_2)
\]
\begin{proof}
	מהנתון נסיק כי גם $\PP(A_1), \PP(A_2) > 0$ ולכן יש הצדקה לדבר על הסתברות מותנית על מאורעות אלה. \\*
	נובע
	\[
		\PP(A_3 \mid A_1 \cap A_2) = \frac{\PP(A_1 \cap A_2 \cap A_3)}{\PP(A_1 \cap A_2)}
	\]
	מהגדרת הסתברות מותמית נובע גם
	\[
		\PP(A_1 \cap A_2) = \PP(A_2 \mid A_1) \PP(A_1)
	\]
	לכן
	\[
		\PP(A_1 \cap A_2 \cap A_3)
		= \PP(A_2 \mid A_1) \PP(A_1) \PP(A_3 \mid A_1 \cap A_2)
	\]
\end{proof}

\Subquestion{}
נוכיח כי לכל סדרה יורדת של $n$ מאורעות $A_1 \supseteq A_2 \supseteq \cdots \supseteq A_n$ המקיימים $\PP(A_{n - 1}) > 0$ מתקיים השוויון
\[
	\PP(\bigcap_{i \in [n]} A_i) = \PP(A_1) \prod_{i \in [n - 1]} \PP(A_{i + 1} \mid A_i)
\]
\begin{proof}
	נשתמש בתוצאת הסעיף הקודם כבסיס אינדוקציה ונראה עתה את צעד האינדוקציה. \\*
	נניח כי הטענה נכונה עבור $n$ ונראה שהיא נכונה גם עבור $n + 1$, מהגדרת הסתברות מותנית
	\[
		\PP(A_{n + 1} \mid A_n) = \frac{\PP(A_{n + 1} \cap A_n)}{\PP(A_n)}
	\]
	עתה נבחין כי מהגדרת סדרה יורדת מתקיים לכל $1 \le k \le n + 1$
	\[
		\bigcap_{i \in [k]} A_i = A_k
	\]
	ולכן נסיק
	\[
		\PP(A_{n + 1} \mid A_n) \PP(A_n) = \PP(A_{n + 1} \cap A_n) = \PP(A_{n + 1})
	\]
	אז מהנחת האינדוקציה
	\begin{align*}
		\PP(\bigcap_{i \in [n + 1]} A_i)
		& = \PP(A_{n + 1})
		= \PP(A_{n + 1} \mid A_n) \PP(A_n)  
		= \PP(A_{n + 1} \mid A_n) \PP(A_1) \prod_{i \in [n - 1]} \PP(A_{i + 1} \mid A_i) \\
		& = \PP(A_1) \prod_{i \in [n]} \PP(A_{i + 1} \mid A_i)
	\end{align*}
\end{proof}

\Subquestion{}
נראה שהתנאי שהסדרה יורדת הוא הכרחי על־ידי מציאת דוגמה נגדית לטענה כאשר הסדרה לא יורדת.
\begin{solution}
	נגדיר ניסוי של הטלת קוביה הוגנת, ונניח $A_1 = \{1, 2\}, A_2 = \{2, 3\}, A_3 = \{3, 4\}$, אז נקבל
	\[
		0 = \PP(\emptyset) = \PP(\bigcap_{i \in [n]} A_i) \ne \PP(A_1) \prod_{i \in [n - 1]} \PP(A_{i + 1} \mid A_i) = \frac{1}{3} \cdot \frac{1}{2} \cdot \frac{1}{2}
	\]
\end{solution}

\Question{}
לאדם שני ילדים, נתון כי אחד מהם הוא בן ונולד ביום שלישי.
נחשב את ההסתברות ששניהם בנים.
\begin{solution}
	נתחיל ונבחין כי העובדה שהבן הראשון נולד ביום שלישי לא משפיעה על התשובה, לכן נתעלם מעובדה זו. \\*
	נגדיר $\Omega = \{ (b, b), (b, g), (g, b), (b, g) \}$, עוד נגדיר $A = \{ (b, g), (b, b) \}$ המאורע שהילד הראשון בן. \\*
	לבסוף, אנו מחפשים את $\PP(\{ (b, b) \} \mid A) = \frac{\frac{1}{4}}{\frac{1}{2}} = \frac{1}{2}$.

	נבחין כי לחילופין היינו יכולים להגדיר $A = \{(b, b), (b, g)\}, B = \{(g, b), (b, b)\}$ המאורע שהילד הראשון בן ושהילד השני בן, והיינו מקבלים מאורעות בלתי־תלויים:
	\[
		\PP(A) \PP(B) = \frac{1}{2} \frac{1}{2} = \frac{1}{4} = \PP(A \cap B)
	\]
	דהינו מין הילדים איננו תלוי.
\end{solution}

\Question{}
מונטי הול מחביא אוצר באקראי מאחורי אחת משש דלתות, המשתתף בוחר 2 מהן. \\*
מונטי פותח שתי דלתות באקראי מבין הדלתות שלא נבחרו, ואין מאחוריהן אוצר. \\*
המשתתף אז בוחר האם לפתוח את שתי הדלתות שבחר או אחת מהדלתות הנותרות. \\*
נחשב מה עדיף.
\begin{solution}
	נניח שהמשתתף תמיד בוחר את דלתות 1 ו־2, עוד נניח $A_i$ המאורע שהאוצר נמצא מאחורי דלת $i$. \\*
	נניח גם ש־$B_i$ המאורע שהמנחה פותח את הדלת ה־$i$, וכמובן נניח כי היא תמיד ריקה. \\*
	אז המאורע שבו המשתתף זוכה אם הוא לא החליף דלת הוא $A_1 \cup A_2$. \\*
	אנו גם יודעים שההסתברות אחידה לאוצרות, לכן $\PP(A_i) = \frac{1}{6}$ וכן $\PP(A_1 \cup A_2) = \frac{1}{3}$, דהינו לפני שהמנחה מתערב יש סיכוי של $\frac{1}{3}$ לזכות. \\*
	עתה נעבור לבדיקת ההסתברות לאחר פתיחת הדלתות, אם המשתתף בחר את האוצר אז המנחה יכול לפתוח ארבע דלתות בהסתברות שווה, דהינו $\PP(B_i \mid A_1) = \frac{1}{4}$ עבור $3 \le i \le 6$.
	אם לעומת זאת האוצר במיקום אחר, אז למנחה יש רק שלוש דלתות לבחור מהן, דהינו $\PP(B_i \mid A_j) = \frac{1}{3}$ עבור $3 \le j \le 6$ ו־$3 \le i \le 6, i \ne j$. \\*
	נעבור לחישוב ההסתברות שהמשתתף בחר את האוצר ולא החליף דלת, מטעמי אחידות ההסתברות הזו שקולה ל־$\PP(A_1 \cup A_2 \mid B_3 \cup B_4)$.
	\begin{align*}
		\PP(A_1 \cup A_2 \mid B_3 \cup B_4)
		& = \frac{\PP((A_1 \cup A_2) \cap (B_3 \cup B_4))}{\PP(B_3 \cup B_4)} \\
		& = \frac{\PP(A_1 \cap B_3) + \PP(A_3 \cap B_4) + \PP(A_2 \cap B_3) + \PP(A_2 \cap B_4)}{\PP(B_3) + \PP(B_4)} \\
		& = \frac{2 \PP(A_1 \cap B_3)}{\PP(B_3)} \\
		& = 2 \PP(A_1 \mid B_3) \\
		& = 2 \frac{\PP(A_1)}{\PP(B_3)} \PP(B_3 \mid A_1) \\
		& = \frac{1}{12 \PP(B_3)}
	\end{align*}
	אז נחשב את $\PP(B_3)$:
	\[
		\PP(B_3)
		= \sum_{i \in [6]} \PP(A_i) \PP(B_3 \mid A_i)
		= \frac{1}{6} \sum_{i \in [6]} \PP(B_3 \mid A_i)
		= \frac{1}{6}(\frac{1}{4} + \frac{1}{4} + 0 + \frac{1}{3})
		= \frac{5}{36}
	\]
	ולכן
	\[
		\PP(A_1 \cup A_2 \mid B_3 \cup B_4)
		= \frac{3}{5}
	\]
	דהינו אם המשתתף דבק בהחלטתו אז יש לו סיכוי של $\frac{3}{5}$ לזכות בפרס. \\*
	עתה נבחן את המקרה השני, המשתתף ויתר על דלתות 1 ו־2 לטובת דלת 5 לאחר שהמנחה פתח את דלתות 3 ו־4, דהינו נחשב את $\PP(A_5 \mid B_3 \cup B_4)$.
	\[
		\PP(A_5 \mid B_3 \cup B_4)
		= \frac{\PP(A_5 \cap (B_3 \cup B_4))}{\PP(B_3 \cup B_4)}
		= \frac{\PP(A_5 \cap B_3) + \PP(A_5 \cap B_4)}{\PP(B_3) + \PP(B_4)}
		= \frac{\PP(A_5 \cap B_3)}{\PP(B_3)}
		= \frac{\PP(A_5)}{\PP(B_3)} \PP(B_3 \mid A_5)
		= \frac{\frac{1}{6}}{\frac{5}{36}} \frac{1}{3}
		= \frac{2}{5}
	\]
	ולכן למשתתף יהיה סיכוי של $\frac{2}{5}$ לזכות אם הוא יעבור לדלת אחרת לאחר שבחר שתיים ואז המנחה פתח שתיים נוספות. \\*
	נסכם ונאמר שלמשתתף במקרה זה לא משתלם להחליף דלתות, ואם הוא לא יחליף יש סיכוי של 60\% שיצליח לזכות באוצר.
\end{solution}

\Question{}
יהיה $(\Omega, \mathcal{F}, \PP)$ מרחב הסתברות ותהי ${\{A_n\}}_{n = 1}^\infty \subseteq \mathcal{F}$ סדרת מאורעות יורדת. \\*
נוכיח כי מתקיים
\[
	\PP(\bigcap_{n \in \NN} A_n) = \lim_{n \to \infty} \PP(A_n)
\]
\begin{proof}
	נגדיר סדרה חדשה ${\{B_n\}}_{n = 1}^\infty \subseteq \Omega$ על־ידי $B_n = \Omega \setminus A_n$ לכל $n \in \NN$. \\*
	לכל $n$ מתקיים $A_n \supseteq A_{n + 1} \iff \Omega \setminus A_n \subseteq \Omega \setminus A_{n + 1} \iff B_n \subseteq B_{n + 1}$, דהינו $\{B_n\}$ סדרת מאורעות עולה. \\*
	ממשפט רציפות פונקציית ההסתברות נסיק $\PP(\bigcup_{n \in \NN} B_n) = \lim_{n \to \infty} \PP(B_n)$. \\*
	נבחין כי $\bigcap_{n \in \NN} A_n = \Omega \setminus \bigcup_{n \in \NN} (\Omega \setminus A_n) = \Omega \setminus \bigcup_{n \in \NN} B_n$ ולכן
	\begin{align*}
		\PP(\bigcap_{n \in \NN} A_n)
		& = \PP(\Omega \setminus \bigcup_{n \in \NN} B_n) \\
		& = \PP(\Omega) - \PP(\bigcup_{n \in \NN} B_n) \\
		& = \PP(\Omega) - \lim_{n \to \infty} \PP(B_n) \\
		& = \lim_{n \to \infty} \PP(\Omega) - \PP(B_n) \\
		& = \lim_{n \to \infty} \PP(\Omega \setminus B_n) \\
		& = \lim_{n \to \infty} \PP(A_n)
	\end{align*}
\end{proof}

\end{document}
