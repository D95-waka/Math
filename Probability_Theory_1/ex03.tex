\documentclass[a4paper]{article}

% packages
\usepackage{inputenc, amsmath, amsthm, thmtools, amsfonts, amssymb, luacode, catchfile, tikzducks, hyperref}
\usepackage[a4paper, margin=50pt, includeheadfoot]{geometry} % set page margins
\usepackage[shortlabels]{enumitem}
\usepackage[skip=3pt, indent=0pt]{parskip}

% language
\usepackage[bidi=basic, layout=tabular, provide=*]{babel}
\babelprovide[main, import]{hebrew}
\babelprovide{rl}
\babelfont{rm}{Libertinus Serif}
\babelfont{sf}{Libertinus Sans}
\babelfont{tt}{Libertinus Mono}

% style
\AddToHook{cmd/section/before}{\clearpage}	% Add line break before section
\linespread{1.3}
\setcounter{secnumdepth}{0}		% Remove default number tags from sections, this won't do well with theorems
\AtBeginDocument{\setlength{\belowdisplayskip}{3pt}}
\AtBeginDocument{\setlength{\abovedisplayskip}{3pt}}

% operators
\DeclareMathOperator\cis{cis}
\DeclareMathOperator\Sp{Sp}
\DeclareMathOperator\tr{tr}
\DeclareMathOperator\im{Im}
\DeclareMathOperator\re{Re}
\DeclareMathOperator\diag{diag}
\DeclareMathOperator*\lowlim{\underline{lim}}
\DeclareMathOperator*\uplim{\overline{lim}}
\DeclareMathOperator\rng{rng}
\DeclareMathOperator\Sym{Sym}
\DeclareMathOperator\Arg{Arg}
\DeclareMathOperator\Log{Log}
\DeclareMathOperator\dom{dom}

% commands
%\renewcommand\qedsymbol{\textbf{מש''ל}}
%\renewcommand\qedsymbol{\fbox{\emoji{lizard}}}
\newcommand{\NN}[0]{\mathbb{N}}
\newcommand{\ZZ}[0]{\mathbb{Z}}
\newcommand{\QQ}[0]{\mathbb{Q}}
\newcommand{\RR}[0]{\mathbb{R}}
\newcommand{\CC}[0]{\mathbb{C}}
\newcommand{\FF}[0]{\mathbb{F}}
\newcommand{\PP}[0]{\mathbb{P}}
\newcommand{\TT}[0]{\mathbb{T}}
\newcommand{\acts}[0]{\circlearrowright}
\newcommand{\explain}[2] {
	\begin{flalign*}
		 && \text{#2} && \text{#1}
	\end{flalign*}
}
\newcommand{\maketitleprint}[0]{ \begin{center}
	\begin{tikzpicture}[scale=3]
		\duck[graduate=gray!20!black, tassel=red!70!black]
	\end{tikzpicture}	
\end{center}
}

% theorem commands
\newtheoremstyle{c_remark}
	{}	% Space above
	{}	% Space below
	{}% Body font
	{}	% Indent amount
	{\bfseries}	% Theorem head font
	{}	% Punctuation after theorem head
	{.5em}	% Space after theorem head
	{\thmname{#1}\thmnumber{ #2}\thmnote{ \normalfont{\text{(#3)}}}}	% head content
\newtheoremstyle{c_definition}
	{3pt}	% Space above
	{3pt}	% Space below
	{}% Body font
	{}	% Indent amount
	{\bfseries}	% Theorem head font
	{}	% Punctuation after theorem head
	{.5em}	% Space after theorem head
	{\thmname{#1}\thmnumber{ #2}\thmnote{ \normalfont{\text{(#3)}}}}	% head content
\newtheoremstyle{c_plain}
	{3pt}	% Space above
	{3pt}	% Space below
	{\itshape}% Body font
	{}	% Indent amount
	{\bfseries}	% Theorem head font
	{}	% Punctuation after theorem head
	{.5em}	% Space after theorem head
	{\thmname{#1}\thmnumber{ #2}\thmnote{ \text{(#3)}}}	% head content

\theoremstyle{c_plain}
\newtheorem{theorem}{משפט}[section]
\newtheorem{lemma}[theorem]{למה}
\newtheorem{proposition}[theorem]{טענה}
\newtheorem*{proposition*}{טענה}
%\newtheorem{corollary}[theorem]{אין חלופה עברית}

\theoremstyle{c_definition}
\newtheorem{definition}[theorem]{הגדרה}
\newtheorem*{definition*}{הגדרה}
\newtheorem{example}{דוגמה}[section]
\newtheorem{exercise}{תרגיל}[section]

\theoremstyle{c_remark}
\newtheorem*{remark}{הערה}
\newtheorem*{solution}{פתרון}
\newtheorem{conclusion}[theorem]{מסקנה}
\newtheorem{notation}[theorem]{סימון}

% Questions related commands
\newcounter{question}
\setcounter{question}{1}
\newcounter{sub_question}
\setcounter{sub_question}{1}

\newcommand{\question}[1][0]{
	\ifthenelse{#1 = 0}{}{\setcounter{question}{#1}}
	\subsection{שאלה \arabic{question}}
	\addtocounter{question}{1}
	\setcounter{sub_question}{1}
}

\newcommand{\subquestion}[1][0]{
	\ifthenelse{#1 = 0}{}{\setcounter{sub_question}{#1}}
	\subsubsection{סעיף \localecounter{letters.gershayim}{sub_question}}
	\addtocounter{sub_question}{1}
}

% import lua and start of document
\directlua{common = require ('../common')}

\GetEnv{AUTHOR}

% headers
\author{\AUTHOR}
\date\today

\title{פתרון מטלה 03 --- תורת ההסתברות (1), 80420}

\begin{document}
\maketitle
\maketitleprint{}

\Question{}
נאמר שמאורע $A$ מחזק את מאורע $B$ אם $\PP(B \mid A) > \PP(B)$. \\*
יהי $(\Omega, \PP)$ מרחב הסתברות, ונוכיח או נפריך את הטענות הבאות.

\Subquestion{}
נסתור את הטענה כי אם $A$ מחזק את $B$ ו־$B$ מחזק את $C$, אז $A$ מחזק את $C$, על־ידי דוגמה נגדית.
\begin{solution}
	נניח $\Omega$ הטלת שתי קוביות הוגנות, נניח גם $A$ המאורע שיצא 2 בקוביה הראשונה, $B$ המאורע שיצא 2 לפחות בכל קוביה, ו־$C$ המאורע שיצא לפחות 2 בקוביה ב'. \\*
	נחשב $\PP(B \mid A) = \frac{5}{6} > \PP(B) = \frac{5^2}{6^2}$, בנוסף $\PP(C \mid B) = 1 > \PP(C) = \frac{5}{6}$ אבל $\PP(C \mid A) = \frac{5}{6} = \PP(C) = \frac{5}{6}$.
\end{solution}

\Subquestion{}
נוכיח כי אם $A$ מחזק את $B$ אז גם $B$ מחזק את $A$.
\begin{proof}
	ישירות מהגדרה
	\[
		\PP(B \mid A) > \PP(B)
		\iff \frac{\PP(A \cap B)}{\PP(A)} > \PP(B)
		\iff \frac{\PP(A \cap B)}{\PP(B)} > \PP(A)
		\iff \PP(A \mid B) > \PP(A)
	\]
\end{proof}

\Subquestion{}
נסתור את הטענה כי אם $A, B$ מאורעות המקיימים $\PP(A \cap B) = 0$ אז $A \cap B = \emptyset$.
\begin{solution}
	יהי $\Omega$ הטלת מטבע טריק, מטבע שבו תמיד צד א' נבחר. \\*
	נגדיר גם $A = \Omega$ ו־$B$ מאורע שיצא צד ב', אז כמובן $\PP(A \cap B) = \PP(B) = 0$, אבל $A \cap B$ הוא המקרה שיצא צד ב', ובפרט איננו מאורע ריק.
\end{solution}

\Subquestion{}
נסתור את הטענה כיאם $(\Omega, \PP)$ מרחב הסתברות אחידה, ונניח גם $B$ מאורע כך ש־$\PP(B) > 0$, אז $(\Omega, \PP_B)$ מרחב הסתברות אחידה.
\begin{solution}
	נבחן מרחב הסתברות של הטלת קוביה הוגנת, הוא עומד בכל התנאים, ואם $B = \{ 1, 2, 3 \}$ אז $\PP_B(\{1\}) = \frac{1}{3} \ne 0 = \PP_B(\{4\})$, דהינו מרחב ההסתברות $(\Omega, \PP_B)$ לא אחיד. \\*
	נבחין כי $(\Omega \cap B, \PP_B)$ הוא כן מרחב הסתברות אחיד.
\end{solution}

\Subquestion{}
נוכיח שאם $(\Omega, \PP)$ מרחב הסתברות לא אחיד ויהי $B$ כך ש־$\PP(B) > 0$, אז $(\Omega, \PP_B)$ מרחב הסתברות לא אחיד.
\begin{proof}
	מהנתון נסיק כי $\Omega$ לא ריק, ונבחין בין שני מקרים. \\*
	אם $\Omega = B$ אז $\PP = \PP_B$ ולכן סיימנו. \\*
	אחרת נגדיר $A = \Omega \setminus B$ ולכן $\PP_B(B) = 1 \ne 0 = \PP_B(A)$, ולכן מרחב ההסתברות הוא לא אחיד.
\end{proof}

\Question{}
\Subquestion{}
בשידה שלוש מגירות, באחת זוג גרביים שחור, בשנייה זוג גרביים לבן, ובשלישית גרב שחור וגרב לבן. \\*
בוחרים מגירה באקראי ובהסתברות אחידה ומוציאים ממנה גרס יחיד באקראי, ונתון כי הוא לבן. \\*
מה ההסתברות שגם הכרב השני במגירה לבן?
\begin{solution}
	השאלה שקולה לשאלה מה הסיכוי להוציא גרב לבן ואז גרב לבן נוסף, נגדיר $\Omega = \{ (w, w), (b, b), (w, b), (b, w) \}$. \\*
	נגדיר $A = \{ (w, w), (w, b) \}$ המאורע שהגרב הראשון שנבחר הוא לבן, עוד נבחין כי $p(w, w) = p(b, b) = p(w, b) + p(b, w)$ ו־$p(w, b) = p(b, w)$. \\*
	נגדיר $B = \{ (w, w) \}$ המאורע ששני הגרביים לבנים, ואנו מחפשים את $\PP(B \mid A)$, לכן
	\[
		\PP(B \mid A) = \frac{\PP(A \cap B)}{\PP(A)} = \frac{\PP(B)}{\PP(A)} = \frac{\frac{1}{3}}{\frac{1}{3} + \frac{1}{6}} = \frac{2}{3}
	\]
\end{solution}

\Subquestion{}
נתון דלי עם $k$ כדורים לבנים ו־$k$ כדורים שחורים.
מוציאים $n < k$ כדורים ללא החזרה ולאחר מכן מוציאים כדור נוסף, כדור $n + 1$.
נחשב מה ההסתברות אם ידוע ש־$n$ הכדורים הראשונים לבנים, מה ההסתברות שהכדור ה־$n + 1$ שחור.
\begin{solution}
	נגדיר $\Omega = {\{b, w\}}^{2k}$ כל הוצאות כל הכדורים ללא החזרה ועם חשיבות לסדר מהדלי. \\*
	נגדיר גם $A = \{ \omega \in \Omega \mid \forall 1 \le i \le n, \omega_i = w \}$ המאורע ש־$n$ הכדורים הראשונים הם לבנים,
	ו־$B = \{ \omega \in \Omega \mid \omega_{n + 1} = b \}$ המאורע שהכדור ה־$n + 1$ שחור. משיקולים קומבינטוריים נוכל להסיק
	\[
		|\Omega| = \binom{2k}{k},
		\qquad
		|A| = \binom{2k - n}{k - n},
		\qquad
		|B| = \binom{2k - 1}{k},
		\qquad
		|A \cap B| = \binom{2k - n - 1}{k - n}
	\]
	נחשב
	\[
		\PP(B \mid A)
		= \frac{\PP(A \cap B)}{\PP(A)}
		= \frac{\frac{|A \cap B|}{|\Omega|}}{\frac{|A|}{|\Omega|}}
		= \frac{|A \cap B|}{|A|}
		= \frac{\binom{2k - n - 1}{k - n}}{\binom{2k - n}{k - n}}
		= \frac{k}{2k - n}
	\]
\end{solution}

\Subquestion{}
נגדיר $\Omega \NN, p(n) = 2^{-n}$.
מגרילים מספר באקראי לפי $p$.
נחשב את ההסתברות שבהינתן שהמספר שהתקבל מתחלק ב־6, ההסתברות שהוא מתחלק ב־7, וההסתברות שהוא מתחלק ב־4.
\begin{solution}
	יהי $k \in \NN$ ונגדיר $l = \text{lcm}(6, k)$, אז אנו יודעים ש־$6 \NN \cap k \NN = l \NN$. \\*
	עוד נוכל לחשב שמתקיים לכל $m \in \NN$
	\[
		\PP(m\NN)
		= \sum_{n = 1}^{\infty} 2^{-mn}
		= \frac{2^{-m}}{1 - 2^{-m}}
		= \frac{1}{2^m - 1}
	\]
	ולכן
	\[
		\PP_6(k) \overset{\text{def}}{=} \PP(k \NN \mid 6 \NN)
		= \frac{\PP(l \NN)}{\PP(6\NN)}
		= \frac{\frac{1}{2^l - 1}}{\frac{1}{2^6 - 1}}
		= \frac{2^6 - 1}{2^{\text{lcm}(6, k)} - 1}
	\]
	לבסוף נציב
	\[
		\PP_6(7) = \frac{2^6 - 1}{2^{42} - 1},
		\qquad
		\PP_6(4) = \frac{2^6 - 1}{2^8 - 1}
	\]
\end{solution}

\Subquestion{}
בוחרים אחד מהמספרים $\{ \frac{1}{4}, \frac{1}{5}, \frac{3}{4} \}$ בהסתברות אחידה ואז מטילים מטבע מוטה בהתאם לפרמטר שנבחר פעמיים. \\*
נחשב את ההסתברות לכל אחד מהפרמטרים בהינתן שיצא צד א' ואז צד ב'.
\begin{solution}
	
\end{solution}

\end{document}
