\documentclass[a4paper]{article}

% packages
\usepackage{inputenc, amsmath, amsthm, thmtools, amsfonts, amssymb, luacode, catchfile, tikzducks, hyperref}
\usepackage[a4paper, margin=50pt, includeheadfoot]{geometry} % set page margins
\usepackage[shortlabels]{enumitem}
\usepackage[skip=3pt, indent=0pt]{parskip}

% language
\usepackage[bidi=basic, layout=tabular, provide=*]{babel}
\babelprovide[main, import]{hebrew}
\babelprovide{rl}
\babelfont{rm}{Libertinus Serif}
\babelfont{sf}{Libertinus Sans}
\babelfont{tt}{Libertinus Mono}

% style
\AddToHook{cmd/section/before}{\clearpage}	% Add line break before section
\linespread{1.3}
\setcounter{secnumdepth}{0}		% Remove default number tags from sections, this won't do well with theorems
\AtBeginDocument{\setlength{\belowdisplayskip}{3pt}}
\AtBeginDocument{\setlength{\abovedisplayskip}{3pt}}

% operators
\DeclareMathOperator\cis{cis}
\DeclareMathOperator\Sp{Sp}
\DeclareMathOperator\tr{tr}
\DeclareMathOperator\im{Im}
\DeclareMathOperator\re{Re}
\DeclareMathOperator\diag{diag}
\DeclareMathOperator*\lowlim{\underline{lim}}
\DeclareMathOperator*\uplim{\overline{lim}}
\DeclareMathOperator\rng{rng}
\DeclareMathOperator\Sym{Sym}
\DeclareMathOperator\Arg{Arg}
\DeclareMathOperator\Log{Log}
\DeclareMathOperator\dom{dom}

% commands
%\renewcommand\qedsymbol{\textbf{מש''ל}}
%\renewcommand\qedsymbol{\fbox{\emoji{lizard}}}
\newcommand{\NN}[0]{\mathbb{N}}
\newcommand{\ZZ}[0]{\mathbb{Z}}
\newcommand{\QQ}[0]{\mathbb{Q}}
\newcommand{\RR}[0]{\mathbb{R}}
\newcommand{\CC}[0]{\mathbb{C}}
\newcommand{\FF}[0]{\mathbb{F}}
\newcommand{\PP}[0]{\mathbb{P}}
\newcommand{\TT}[0]{\mathbb{T}}
\newcommand{\acts}[0]{\circlearrowright}
\newcommand{\explain}[2] {
	\begin{flalign*}
		 && \text{#2} && \text{#1}
	\end{flalign*}
}
\newcommand{\maketitleprint}[0]{ \begin{center}
	\begin{tikzpicture}[scale=3]
		\duck[graduate=gray!20!black, tassel=red!70!black]
	\end{tikzpicture}	
\end{center}
}

% theorem commands
\newtheoremstyle{c_remark}
	{}	% Space above
	{}	% Space below
	{}% Body font
	{}	% Indent amount
	{\bfseries}	% Theorem head font
	{}	% Punctuation after theorem head
	{.5em}	% Space after theorem head
	{\thmname{#1}\thmnumber{ #2}\thmnote{ \normalfont{\text{(#3)}}}}	% head content
\newtheoremstyle{c_definition}
	{3pt}	% Space above
	{3pt}	% Space below
	{}% Body font
	{}	% Indent amount
	{\bfseries}	% Theorem head font
	{}	% Punctuation after theorem head
	{.5em}	% Space after theorem head
	{\thmname{#1}\thmnumber{ #2}\thmnote{ \normalfont{\text{(#3)}}}}	% head content
\newtheoremstyle{c_plain}
	{3pt}	% Space above
	{3pt}	% Space below
	{\itshape}% Body font
	{}	% Indent amount
	{\bfseries}	% Theorem head font
	{}	% Punctuation after theorem head
	{.5em}	% Space after theorem head
	{\thmname{#1}\thmnumber{ #2}\thmnote{ \text{(#3)}}}	% head content

\theoremstyle{c_plain}
\newtheorem{theorem}{משפט}[section]
\newtheorem{lemma}[theorem]{למה}
\newtheorem{proposition}[theorem]{טענה}
\newtheorem*{proposition*}{טענה}
%\newtheorem{corollary}[theorem]{אין חלופה עברית}

\theoremstyle{c_definition}
\newtheorem{definition}[theorem]{הגדרה}
\newtheorem*{definition*}{הגדרה}
\newtheorem{example}{דוגמה}[section]
\newtheorem{exercise}{תרגיל}[section]

\theoremstyle{c_remark}
\newtheorem*{remark}{הערה}
\newtheorem*{solution}{פתרון}
\newtheorem{conclusion}[theorem]{מסקנה}
\newtheorem{notation}[theorem]{סימון}

% Questions related commands
\newcounter{question}
\setcounter{question}{1}
\newcounter{sub_question}
\setcounter{sub_question}{1}

\newcommand{\question}[1][0]{
	\ifthenelse{#1 = 0}{}{\setcounter{question}{#1}}
	\subsection{שאלה \arabic{question}}
	\addtocounter{question}{1}
	\setcounter{sub_question}{1}
}

\newcommand{\subquestion}[1][0]{
	\ifthenelse{#1 = 0}{}{\setcounter{sub_question}{#1}}
	\subsubsection{סעיף \localecounter{letters.gershayim}{sub_question}}
	\addtocounter{sub_question}{1}
}

% import lua and start of document
\directlua{common = require ('../common')}

\GetEnv{AUTHOR}

% headers
\author{\AUTHOR}
\date\today

\title{פתרון מטלה 05 --- תורת ההסתברות (1), 80420}

\DeclareMathOperator{\Supp}{Supp}

\begin{document}
\maketitle
\maketitleprint{}

\question{}
עבור $i \in \{0, 1\}$ תהי $\Omega_i \subseteq \RR$, ותהי $\PP_i$ פונקציית הסתברות בדידה על $\Omega_i$ ותהי $X_i : \Omega_i \to \RR$ פונקציית הזהות. \\*
נוכיח ש־$X_1 \overset{d}{=} X_2$ אם ורק אם קיימת  $S \subseteq \Omega_1 \cap \Omega_2$ קבוצה בת־מניה במובן הרחב המקיימת $S = \Supp \PP_1 = \Supp \PP_2$ וכן $\forall x \in S, \PP_1(x) = \PP_2(x)$.
\begin{proof}
	נניח ש־$X_1 \overset{d}{=} X_2$, ולכן $\PP_1 = \PP_2$. \\*
	נגדיר $S = \Supp \PP_1 = \Supp \PP_2$ ואנו יודעים כי אלו הן פונקציות בדידות ולכן $|S| \le \aleph_0$, כמובן גם מההגדרה של התומך נסיק $S \subseteq \Omega_1, \Omega_2$ ולכן בפרט $S \subseteq \Omega_1 \cap \Omega_2$. \\*
	מתקיים $\PP_1(X_1 = x) = \PP(\{ y \in S \mid y \in X_1^{-1}(x) \}) = \PP_1(\{ y \in S \mid x = y \}) = \PP_1(x)$ ולכן נסיק ממהלך זהה על $\PP_2$ שגם $\PP_1(x) = \PP_2(x)$.

	נניח את הכיוון השני. \\*
	מצאנו כי $\forall x \in S, \PP_1(X_1 = x) = \PP_1(x)$ ואנו רוצים להראות ש־$\forall x \in \RR, \PP(X_1 = x) = \PP(X_2 = x)$ אבל מהשוויון נובע שעלינו רק להראות ש־$\PP_1(x) = \PP_2(x)$  לכל $x \in \RR$. \\*
	כמובן אם $x \in \RR \setminus S$ אז $\PP_1(x) = 0 = \PP_2(x)$, אחרת $x \in S$, אבל אז ישירות מההנחה מתקבל $\PP_1(x) = \PP_2(x)$ ומצאנו כי $X_1 \overset{d}{=} X_2$.
\end{proof}

\question{}
נוכיח או נפריך את הטענות הבאות.

\subquestion{}
נוכיח שאם תהי $f : \RR \to \RR$ פונקציה חד־חד ערכית, אז $X \overset{d}{=} Y$ אם ורק אם $f(X) \overset{d}{=} f(Y)$.
\begin{proof}
	הכיוון הראשון הוכח כטענה בכיתה, לכן נניח ש־$f(X) \overset{d}{=} f(Y)$. \\*
	יהי $x \in \RR$, אם קיים $y \in \RR$ כך ש־$x = f(y)$ אז נובע $\PP(f(X) = f(y)) = \PP(f(Y) = f(y)) \implies \PP(X = x) = \PP(Y = x)$. \\*
	אם לא קיים $y$ כזה, אז $\PP(f(X) = x) = \PP(\{\omega \in \Omega \mid \omega = X^{-1}(f^{-1}(x))\}) = 0$ ולכן נסיק $\PP(X = x) = 0$.
\end{proof}

\subquestion{}
נסתור את הטענה כי אם $X \overset{d}{=} Y$ אז $\PP(X = Y) > 0$.
\begin{solution}
	עבור $\Omega = [6]$, $\PP$ אחיד, \\*
	עוד נגדיר $X = Id, Y = (1\ 2\ 3\ 4\ 5\ 6)$, אז נקבל $\PP(X = n) = \frac{1}{6} = \PP(Y = n)$ אבל גם
	\[
		\PP(X = Y)
		= \PP(\{n \in [6] \mid X(n) = Y(n)\})
		= 0
	\]
\end{solution}

\subquestion{}
נסתור את הטענה שאם $X \overset{d}{=} Y$ וגם $X, Y$ בלתי־תלויים, אז $\PP(X = Y) > 0$.
\begin{solution}
	נגדיר הטלת שתי קוביות הוגנות וגם $X(n, m) = n, Y(n, m) = m + 6$, אז המשתנים המקריים בלתי־תלויים, וגם $\PP(X = Y) = 0$.
\end{solution}

\subquestion{}
נוכיח שאם $X$ בלתי־תלוי בעצמו, אז קיים $c \in \RR$ שעבורו $\PP(X = c) = 1$.
\begin{proof}
	נתון שמתקיים
	\[
		\PP(X = t, X = s) = \PP(X = t) \PP(X = s)
	\]
	אבל אם $t \ne s$ אז $\{ \omega \in \Omega \mid X(\omega) = s = t \} = \emptyset$ ולכן $\PP(X = s) \PP(X = t) = 0$. \\*
	אם $t = s$ אז נקבל
	\[
		\PP(X = t, X = t) = \PP(X = t) = \PP^2(X = t)
		\iff
		\PP(X = t) = 0, 1
	\]
	ואילו לא קיים $c$ עבורו $\PP(X = c) = 1$ אז נסיק $\PP(X \in \RR) = \PP(\Omega) = 0$ וזו סתירה, לכן $c$ כזה קיים ואף יחיד.
	ולכן 
\end{proof}

\subquestion{}
נוכיח שאם $X \overset{d}{=} X^2$ אז קיים $p \in [0, 1]$ שעבורו $X \sim Ber(p)$.
\begin{proof}
	נבחין כי
	\[
		\PP(X = x) = \PP(X^2 = x)
		\iff
		\PP(\{\omega \in \Omega \mid X(\omega) = x\}) = \PP(\{\omega \in \Omega \mid X^2(\omega) = x\})
	\]
	ועבור $x = 0, 1$ מתקיים $X(\omega) = X^2(\omega)$. \\*
	אילו $x \ne 0, 1$ אז נקבל
	\[
		\PP(X = x) = \PP(X = \sqrt{x}) = \PP(X = \sqrt[4]{x}) = \cdots
	\]
	ואילו $\PP(X = x) \ne 0$ נקבל $\PP(\Omega) = \infty$ בסתירה להגדרת פונקציית הסתברות, ולכן $\PP(X = x) = 0$. \\*
	לכן גם $\PP(X = 0) + \PP(X = 1) = 1$ ובהתאם קיבלנו כי קיים $p \in [0, 1]$ כך ש־$X \sim Ber(p)$.
\end{proof}

\question{}
יהיה $(\Omega, \PP)$ מרחב הסתברות. \\*
נוכיח כי מאורעות $A_1, \dots, A_n$ הם בלתי־תלויים אם ורק אם $1_{A_1}, \dots, 1_{A_n}$ הם משתנים מקריים בלתי־תלויים.
\begin{proof}
	נבחין כי אילו $1_{A_1}, \dots, 1_{A_n}$ בלתי־תלויים ונבחר $S_1 = \cdots = S_n = 1$ אז נקבל את קבוצת המאורעות $A_1, \dots, A_n$.

	נניח אם כך ש־$A_1, \dots, A_n$ בלתי־תלויה.
	תהינה $S_1, \dots, S_n \mathcal{F}_\RR$. נבחין כי אם $1 \in S_i$ אז $\{ 1_{A_i}(\omega) \in S_i \} = A_i$ ובהתאם אם $1 \notin S_i$ אז $\{ 1_{A_i}(\omega) \in S_i \} = \emptyset$. \\*
	לכן נגדיר $I = \{ i \in [n] \mid 1 \in S_i \}$ ונקבל ש־${\{1_{A_i} \in S_i\}}_{i \in [n]} = {\{ A_i \}}_{i \in I}$ וזו כמובן קבוצה בלתי־תלויה מההנחה.
\end{proof}

\question{}
יהיו $X \sim Geo(p), Y \sim Gro(q)$ משתנים מקריים בלתי־תלויים.

\subquestion{}
נחשב את ההסתברות למאורע $\{X \ge n \}$ עבור $n \in \{1, 2, \dots\}$ כלשהו.
\begin{solution}
	\[
		\PP(X \ge n)
		= \sum_{m = n}^\infty \PP(X = m)
		= \sum_{m = n}^\infty {(1 - p)}^{m - 1} p
		= p \sum_{m = n + 1}^\infty {(1 - p)}^m
		= p \cdot \frac{{(1 - p)}^{n - 1}}{1 - (1 - p)}
		= {(1 - p)}^{n - 1}
	\]
\end{solution}

\subquestion{}
נראה שהמשתנה המקרי $Z = \min(X, Y)$ הוא בעל התפלגות $Geo(1 - (1 - p)(1 - q))$.
\begin{proof}
	\begin{align*}
		p_Z(x)
		& = \PP(x = \min(X, Y)) \\
		& = \PP(\{\omega \in \Omega \mid x = \min(X(\omega), Y(\omega))\}) \\
		& = \PP(\{\omega \in \Omega \mid (x = X(\omega) \land x = Y(\omega)) \lor (x = X(\omega) \land x < Y(\omega)) \lor (x = Y(\omega) \land x < X(\omega))\}) \\
		& \overset{(1)}{=} \PP(\{\omega \in \Omega \mid x = X(\omega), x = Y(\omega)\} \uplus \{ \omega \in \Omega \mid x = X(\omega), x < Y(\omega)\} \uplus \{\omega \in \Omega \mid x = Y(\omega), x < X(\omega)\}) \\
		& = \PP(X = x, Y = x) + \PP(X = x) \PP(Y > x) + \PP(Y = x) \PP(X > x) \\
		& \overset{(2)}{=} \PP(X = x) \PP(Y \ge x) + \PP(Y = x) \PP(X \ge x) - \PP(X = x, Y = x) \\
		& = {(1 - p)}^{x - 1} p {(1 - q)}^{x - 1} + {(1 - q)}^{x - 1} q {(1 - p)}^{x - 1} - {(1 - p)}^{x - 1} p {(1 - q)}^{x - 1} q \\
		& = {(1 - p)}^{x - 1} {(1 - q)}^{x - 1} (-p q + p + q) \\
		& = {(1 - (1 - p)(1 - q))}^{x - 1} (1 - (1 - p)(1 - q)) \\
	\end{align*}
	כאשר
	\begin{enumerate}
		\item המאורעות המושרים הם זרים שאם לא כן נקבל סתירה לערך יחיד של $X$ או $Y$ עבור מקרה.
		\item נחליף את המאורעות להיות במקרה של גדול ולא גדול ממש, ועל־ידי שימוש בהכלה והדחה נקבל שיש צורך בחיסור המקרה המשותף.
	\end{enumerate}
	ולכן $\min(X, Y) \sim Geo(1 - (1 - p)(1 - q))$.
\end{proof}

\question{}
יהיו שני משתנים מקריים $X, Y$ בלתי־תלויים שנתמכים על $\{0, \dots, n - 1\}$. נסמן $Z = X + Y \mod n$. \\*
נראה ש־$X$ ו־$Z$ בלתי־תלויים אם ורק אם $Y \sim U(\{0, \dots, n - 1\})$.
\begin{proof}
	נניח ש־$X, Z$ בלתי־תלויים.
	\begin{align*}
		\PP(Z = k, X = l)
		& = \PP(X + Y \in \{k, k + n\}, X = l) \\
		& = \PP(\{\omega \in \Omega \mid X(\omega) + Y(\omega) \in \{k, k + n\}, X(\omega) = l\}) \\
		& = \PP(\{\omega \in \Omega \mid l + Y(\omega) \in \{k, k + n\}, X(\omega) = l\}) \\
		& = \PP(\{\omega \in \Omega \mid Y(\omega) \in \{k - l, k + n - l\}, X(\omega) = l\}) \\
		& = \PP(Y \in \{k - l, k - l + n\}) \PP(X = l) \\
	\end{align*}
	ומצד שני
	\[
		\PP(Z = k, X = l)
		= \PP(Z = k) \PP(X = l)
	\]
	לכן
	\[
		\PP(Y = k - l, k - l + n)
		= \PP(Z = k)
	\]
	נחפש את המקרה $Y = m$ ונקבל $k = m + l \mod n$ ולכן
	\[
		\PP(Y = m) = \PP(Z = m + l)
	\]
	כלומר
	\[
		\PP(Y = m) = \PP(Z = m) = \PP(Z = m + 1) = \cdots
	\]
	ולכן נוכל להסיק שמתקיים $\PP(Y = m) = \frac{1}{n}$, דהינו $Y \sim U(\{0, \dots, n - 1\})$.

	נניח בכיוון ההפוך ש־$Y \sim U$ ונרצה להראות ש־$X, Z$ בלתי־תלויים. \\*
	נבדוק
	\begin{align*}
		\PP(X = k, Z = l)
		& = \PP(X = k, X + Y = l) \\
		& = \PP(X = k, Y \in \{l - k, l - k + n\}) \\
		& = \PP(X = k) \PP(Y = \{l - k, l - k + n\}) \\
		& = \PP(X = k) \frac{1}{n}
	\end{align*}
	כאשר המעבר האחרון נובע מנוסחת ההסתברות השלמה על התומך של $X$.
	ובנוסף
	\[
		\PP(Z = l)
		= \PP(X + Y \in \{l, l + n\})
		= \PP(X + Y \in \{l, l + n\})
		= \sum_{k = 1}^{n} \PP(X = k) \PP(Y = k - l \mod n)
		= \frac{1}{n} \PP(X \in \RR)
		= \frac{1}{n}
	\]
	ולכן נוכל להסיק כי $X, Z$ בלתי־תלויים לפי השוויון הקודם.
\end{proof}

\question{}
תהי $M \in Mat_{2 \times 2}(\RR)$ המטריצה המתארת התפלגות משותפת נקודתית של שני משתני ברנולי $X$ ו־$Y$.

\subquestion{}
נוכיח ש־$X$ ו־$Y$ שווי התפלגות אם ורק אם $M^t = M$.
\begin{proof}
	נניח $X \sim Ber(p), Y \sim Ber(q)$ עבור $0 \le p, q \le 1$. \\*
	ישירות מחישוב בהינתן סימונים אלה
	\[
		M = \begin{pmatrix}
			p q & p (1 - q) \\
			q (1 - p) & (1 - p)(1 - q)
		\end{pmatrix}
	\]
	נבחין כי $\PP(X = Y = 1) = pq$ תמיד נכון, וכך גם $\PP(X = Y = 0) = (1 - p)(1 - q)$, טענות אלה תקפות תמיד. \\*
	נניח ש־$X, Y$ שווי התפלגות ונובע $p(1 - q) = q(1 - p)$ ולכן $M$ סימטרית, נניח ש־$M$ סימטרית ונקבל שהשוויון מתקיים, ולכן בהכרח $p = q$.
\end{proof}

\subquestion{}
נוכיח ש־$X$ ו־$Y$ שווים כמעט תמיד אם ורק אם $M$ היא אלכסונית.
\begin{proof}
	נניח ש־$X \overset{a.s.}{=} Y$ ולכן $\PP(X = Y) = 1$ ובהתאם $\PP(X \ne Y) = 0$, לכן מתקיים $\PP(X = 1, Y = 0) = \PP(X = 0, Y = 1) = 0$, ונקבל ש־$M$ אלכסונית.

	נניח ש־$M$ אלכסונית ונבדוק $\PP(X = Y)$, אבל מהאיפוס נובע $p (1 - q) = q(1 - p) = 0$ ולכן או $p = q = 1$ או $p = q = 0$ ובהתאם מחישוב ישיר $\PP(X = Y) = 1$, לכן $X \overset{a.s.}{=} Y$.
\end{proof}

\subquestion{}
נוכיח $X$ ו־$Y$ בלתי־תלויים אם ורק אם $M$ היא מדרגה 1 מעל הממשיים.
\begin{proof}
	נניח ש־$X$ ו־ $Y$ בלתי־תלויים ולכן $\PP(X = 1, Y = k) = \PP(X = 1) \PP(Y = k)$ וכן $\PP(X = 0, Y = k) = \PP(X = 0) \PP(Y = k)$ ולכן שתי השורות הן תלויות לינארית, והמטריצה מדרגה 1.

	נניח שהמטריצה מדרגה 1 ולכן קיים קבוע $c$ כך ש־$\PP(X = 1, Y = 1) = c \PP(X = 1, Y = 0)$, אבל גם
	\[
		p = \PP(X = 1) = \PP(X = 1, Y = 0) + \PP(X = 1, Y = 1) = (c + 1) \PP(X = 1, Y = 1) + \PP(X = 1, Y = 1)
	\]
	ולכן
	\[
		\PP(X = 1, Y = 1) = \frac{p}{c + 1}
	\]
	מתהליך דומה נקבל גם
	\[
		1 - p = \PP(X = 0) = \PP(X = 0, Y = 1) + \PP(X = 0, Y = 0) = (c + 1) \PP(P = 0, Y = 1)
		\implies \PP(X = 0, Y = 1) = \frac{1 - p}{c + 1}
	\]
	אבל
	\[
		q = \PP(Y = 1) = \PP(X = 1, Y = 1) + \PP(X = 0, Y = 1) = \frac{1 + p - p}{c + 1}
	\]
	ומכאן נסיק ישירות $\PP(X = t, Y = s) = \PP(X = t) \PP(Y = s)$.
\end{proof}


\end{document}
