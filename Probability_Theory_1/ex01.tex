\documentclass[a4paper]{article}

% packages
\usepackage{inputenc, amsmath, amsthm, thmtools, amsfonts, amssymb, luacode, catchfile, tikzducks, hyperref}
\usepackage[a4paper, margin=50pt, includeheadfoot]{geometry} % set page margins
\usepackage[shortlabels]{enumitem}
\usepackage[skip=3pt, indent=0pt]{parskip}

% language
\usepackage[bidi=basic, layout=tabular, provide=*]{babel}
\babelprovide[main, import]{hebrew}
\babelprovide{rl}
\babelfont{rm}{Libertinus Serif}
\babelfont{sf}{Libertinus Sans}
\babelfont{tt}{Libertinus Mono}

% style
\AddToHook{cmd/section/before}{\clearpage}	% Add line break before section
\linespread{1.3}
\setcounter{secnumdepth}{0}		% Remove default number tags from sections, this won't do well with theorems
\AtBeginDocument{\setlength{\belowdisplayskip}{3pt}}
\AtBeginDocument{\setlength{\abovedisplayskip}{3pt}}

% operators
\DeclareMathOperator\cis{cis}
\DeclareMathOperator\Sp{Sp}
\DeclareMathOperator\tr{tr}
\DeclareMathOperator\im{Im}
\DeclareMathOperator\re{Re}
\DeclareMathOperator\diag{diag}
\DeclareMathOperator*\lowlim{\underline{lim}}
\DeclareMathOperator*\uplim{\overline{lim}}
\DeclareMathOperator\rng{rng}
\DeclareMathOperator\Sym{Sym}
\DeclareMathOperator\Arg{Arg}
\DeclareMathOperator\Log{Log}
\DeclareMathOperator\dom{dom}

% commands
%\renewcommand\qedsymbol{\textbf{מש''ל}}
%\renewcommand\qedsymbol{\fbox{\emoji{lizard}}}
\newcommand{\NN}[0]{\mathbb{N}}
\newcommand{\ZZ}[0]{\mathbb{Z}}
\newcommand{\QQ}[0]{\mathbb{Q}}
\newcommand{\RR}[0]{\mathbb{R}}
\newcommand{\CC}[0]{\mathbb{C}}
\newcommand{\FF}[0]{\mathbb{F}}
\newcommand{\PP}[0]{\mathbb{P}}
\newcommand{\TT}[0]{\mathbb{T}}
\newcommand{\acts}[0]{\circlearrowright}
\newcommand{\explain}[2] {
	\begin{flalign*}
		 && \text{#2} && \text{#1}
	\end{flalign*}
}
\newcommand{\maketitleprint}[0]{ \begin{center}
	\begin{tikzpicture}[scale=3]
		\duck[graduate=gray!20!black, tassel=red!70!black]
	\end{tikzpicture}	
\end{center}
}

% theorem commands
\newtheoremstyle{c_remark}
	{}	% Space above
	{}	% Space below
	{}% Body font
	{}	% Indent amount
	{\bfseries}	% Theorem head font
	{}	% Punctuation after theorem head
	{.5em}	% Space after theorem head
	{\thmname{#1}\thmnumber{ #2}\thmnote{ \normalfont{\text{(#3)}}}}	% head content
\newtheoremstyle{c_definition}
	{3pt}	% Space above
	{3pt}	% Space below
	{}% Body font
	{}	% Indent amount
	{\bfseries}	% Theorem head font
	{}	% Punctuation after theorem head
	{.5em}	% Space after theorem head
	{\thmname{#1}\thmnumber{ #2}\thmnote{ \normalfont{\text{(#3)}}}}	% head content
\newtheoremstyle{c_plain}
	{3pt}	% Space above
	{3pt}	% Space below
	{\itshape}% Body font
	{}	% Indent amount
	{\bfseries}	% Theorem head font
	{}	% Punctuation after theorem head
	{.5em}	% Space after theorem head
	{\thmname{#1}\thmnumber{ #2}\thmnote{ \text{(#3)}}}	% head content

\theoremstyle{c_plain}
\newtheorem{theorem}{משפט}[section]
\newtheorem{lemma}[theorem]{למה}
\newtheorem{proposition}[theorem]{טענה}
\newtheorem*{proposition*}{טענה}
%\newtheorem{corollary}[theorem]{אין חלופה עברית}

\theoremstyle{c_definition}
\newtheorem{definition}[theorem]{הגדרה}
\newtheorem*{definition*}{הגדרה}
\newtheorem{example}{דוגמה}[section]
\newtheorem{exercise}{תרגיל}[section]

\theoremstyle{c_remark}
\newtheorem*{remark}{הערה}
\newtheorem*{solution}{פתרון}
\newtheorem{conclusion}[theorem]{מסקנה}
\newtheorem{notation}[theorem]{סימון}

% Questions related commands
\newcounter{question}
\setcounter{question}{1}
\newcounter{sub_question}
\setcounter{sub_question}{1}

\newcommand{\question}[1][0]{
	\ifthenelse{#1 = 0}{}{\setcounter{question}{#1}}
	\subsection{שאלה \arabic{question}}
	\addtocounter{question}{1}
	\setcounter{sub_question}{1}
}

\newcommand{\subquestion}[1][0]{
	\ifthenelse{#1 = 0}{}{\setcounter{sub_question}{#1}}
	\subsubsection{סעיף \localecounter{letters.gershayim}{sub_question}}
	\addtocounter{sub_question}{1}
}

% import lua and start of document
\directlua{common = require ('../common')}

\GetEnv{AUTHOR}

% headers
\author{\AUTHOR}
\date\today

\title{פתרון מטלה 01 --- תורת ההסתברות (1), 80420}

\begin{document}
\maketitle
\maketitleprint{}

\Question{}
יהי $(\Omega, \PP)$ מרחב הסתברות בדיד, נוכיח או נפריך את הטענות הבאות.

\Subquestion{}
נוכיח שאם $A \subseteq \Omega$ מקיימת $\PP(A) = 0$ אז לכל $B \subseteq \Omega$ מתקיים $\PP(A \cup B) = \PP(B)$.
\begin{proof}
	נבחין כי $D = B \setminus A$ היא זרה באיחוד ל־$A \cap B$, לכן נקבל $\PP(B) = \PP(D \cup (A \cap B)) = \PP(D) + \PP(A \cap B)$. \\*
	אבל $A \cap B \subseteq A$ ולכן מתכונות פונקציית הסתברות נקבל $\PP(A \cap B) \le \PP(A) = 0$. \\*
	לכן $\PP(D) = \PP(B)$. נשתמש בטענה זו ונקבל
	\[
		\PP(A \cup B) = \PP(A \cup (B \setminus A)) = \PP(A) + \PP(D) = 0 + \PP(B)
	\]
	וקיבלנו כי השוויון אכן מתקיים.
\end{proof}

\Subquestion{}
נוכיח שאם $A \subseteq B$ אז $\PP(A) \le \PP(B)$.
\begin{proof}
	למעשה תכונה זו הוכחה בהרצאה, נעתיק את ההוכחה:
	\begin{enumerate}
		\item נראה כי $\PP(\emptyset) = \sum_{i = 1}^\infty \PP(\emptyset)$ שכן כל איחוד של קבוצות ריקות הוא זר, לכן אילו $\PP(\emptyset) \ne 0$ נקבל ישר סתירה, נסיק כי $\PP(\emptyset) = 0$ בלבד.
		\item נגדיר $A_i = \emptyset$ לכל $i > n$ ונשתמש בסיגמא־אדיטיביות ונקבל
			\[
				\PP(\bigcup_{i \in I} A_i)
				= \PP(\bigcup_{i \in \NN} A_i)
				= \sum_{i \in \NN} \PP(A_i)
				= \sum_{i \in I} \PP(A_i)
			\]
		\item נשתמש בתכונה 2 על $B, B \setminus A$, אלו הן קבוצות זרות כמובן, אם נגדיר $D = A \cup (B \setminus A)$ נקבל $\PP(D) = \PP(A) + \PP(B \setminus A) \ge \PP(A)$.
	\end{enumerate}
\end{proof}

\Subquestion{}
נסתור את הטענה כי אם $A \subsetneq B$ אז $\PP(A) \lneq \PP(B)$.
\begin{solution}
	נגדיר $\Omega = [3]$ ו־$p(1) = p(2) = \frac{1}{2}, p(3) = 0$, ונבחן את $\PP_p$. \\*
	אם נגדיר $A = \{ 1 \}, B = \{ 1, 3 \}$ נקבל $A \subsetneq B$ אבל $\PP(A) = \PP(B) = \frac{1}{2}$ בסתירה לטענה.
\end{solution}

\Subquestion{}
נוכיח שלכל $A, B \subseteq \Omega$ מתקיים $\PP(A \cap B) \ge \PP(A) + \PP(B) - 1$
\begin{proof}
	מתכונות פונקציית הסתברות נקבל $\PP(A \cup B) = \PP(A) + \PP(B) - \PP(A \cap B)$, \\*
	לאחר החלפת אגפים נקבל
	\[
		\PP(A \cap B) = \PP(A) + \PP(B) - \PP(A \cup B) \ge \PP(A) + \PP(B) - 1
	\]
	שכן מתקיים $1 \ge \PP(X)$ לכל $X \in \mathcal{F}$ ובפרט עבור $A \cup B$.
\end{proof}

\Subquestion{}
נוכיח כי מתקיים $\PP(A \triangle B) = \PP(A) + \PP(B) - 2 \PP(A \cap B)$.
\begin{proof}
	נבחין תחילה כי $A \triangle B = (A \cup B) \setminus (A \cap B) = (A \cup B) \cap {(A \cap B)}^C$, \\*
	נשתמש בשוויון מהסעיף הקודם ונקבל
	\begin{align*}
		\PP(A \triangle B)
		& = \PP((A \cup B) \cap {(A \cap B)}^C) \\
		& = \PP(A \cup B) + \PP({(A \cap B)}^C) - \PP((A \cup B) \cup {(A \cap B)}^C) \\
		& = \PP(A \cup B) + \PP(\Omega \setminus (A \cap B)) - \PP(\Omega) \\ \tag{1}
		& = \PP(A \cup B) + \PP(\Omega) - \PP(A \cap B) - 1 \\
		& = \PP(A) + \PP(B) - 2 \PP(A \cap B)
	\end{align*}
	כאשר המעבר $(1)$ נובע מהשוויון $A \cup B \cup {(A \cap B)}^C = A \cup B \cup A^C \cup B^C = \Omega$ שנובעת מדה־מורגן לקבוצות.
\end{proof}

\Question{}
נוכיח שקיימת פונקציית הסתברות יחידה על $\Omega = \NN$ המקיימת $\PP(\{n\}) = 3 \PP(\{n + 1\})$.
\begin{proof}
	נוכיח שקיימת כזאת פונקציית הסתברות, נגדיר $p : \NN \to [0, 1]$ על־ידי $p(n) = 2 \cdot \frac{1}{3^n}$. \\*
	נקבל מנוסחת סכום סדרה הנדסית ש־$\sum_{n = 1}^{\infty} p(n) = 1$ ולכן זוהי פונקציית הסתברות נקודתית והיא משרה פונקציית הסתברות $\PP_p$ המקיימת את תנאי הטענה.
	אז הוכחנו שקיימת לכל הפחות פונקציה אחת כזו.

	נעבור להוכחת יחידות, נניח ש־$\PP, \PP'$ שתי פונקציות הסתברות המקיימות את הטענה. \\*
	באינדוקציה על $n$ נקבל כי $\PP(\{n\}) = 3^{1 - n} \PP(\{1\})$.
	מסיגמא־אדיטיביות נקבל
	\[
		1
		= \PP(\NN)
		= \PP(\bigcup_{n \in \NN} \{ n \})
		= \sum_{n \in \NN} \PP(\{ n \})
		= \sum_{n = 1}^{\infty} \PP(\{1\}) 3^{1 - n}
		= \PP(\{1\}) \sum_{n = 1}^{\infty} 3^{1 - n}
		= \PP(\{1\}) \cdot \frac{1}{1 - \frac{1}{3}}
	\]
	ולכן נסיק $\PP(\{1\}) = \frac{2}{3}$, וממהלך זה נוכל גם להסיק על־ידי סעיף ב' בתנאים שקולים לפונקציית הסתברות בדידה שקיימות פונקציות הסתברות נקודתיות $p, p'$ כך ש־$\PP = \PP_p, \PP' = \PP_{p'}$.
	אבל קיבלנו אם כך ש־$p(1) = p'(1) = \frac{2}{3}$ ובהתאם להגדרה הרקורסיבית נקבל $p = p'$ ובהתאם $\PP = \PP'$, דהינו קיימת פונקציית הסתברות יחידה המקיימת את תנאי הטענה.
\end{proof}
נחשב את $\PP(\NN)$ ואת $\PP(3\NN)$.
כמובן $\PP(\NN) = 1$, עבור $3\NN$ נקבל
\[
	\PP(3\NN) 
	= \sum_{n \in 3\NN} \PP(\{n\}) 
	= \sum_{n = 1}^{\infty} p(3n) 
	= \sum_{n = 1}^{\infty} 2 \cdot \frac{1}{3^{3n}}
	= \sum_{n = 1}^{\infty} 2 \cdot \frac{1}{9^{n}}
	= \frac{1}{4}
\]

\Question{}
עבור $a : I \to [0, \infty)$ מגדירים % chktex 9
\[
	\sum_{i \in I} a(i) = \sup \left\{ \sum_{i \in J} a(i) \middle| J \subseteq I, |J| < \infty \right\}
\]
נוכיח שאם $\sum_{i \in I} a(i) < \infty$ אז התומך של $a$ בן־מניה.
\begin{proof}
	נניח את תנאי הטענה ושקיים סופרימום כזה $C < \infty$, נגדיר ${\{J_k\}}_{k = 1}^\infty$ קבוצות אינדקסים סופיות כך שלכל $k \in \NN$ מתקיים $C - \sum_{i \in J_k} a(i) < \frac{1}{k}$.
	נבחין כי אכן קיימות כאלה מהגדרת הסופרימום.
	עתה נגדיר $J = \bigcup_{k \in \NN} J_k$ קבוצה אינסופית, משיקולי עוצמות אנו יודעים כי היא בת־מניה, ואף
	\[
		\sum_{i \in I} a(i) = \sum_{j \in J} a(j)
	\]
	אילו נניח בשלילה שקיים $i_0 \in I \setminus J$ כך ש־$a(i_0) > 0$ אז נקבל
	\[
		C = \sum_{j \in J \cup \{i_0\}} a(j) > \sum_{j \in J} a(j) = C
	\]
	וקיבלנו סתירה, לכן $a(i_0) = 0$ לכל $i_0 \in I \setminus J$ ובהתאם $J$ הוא התומך של $a$, אבל אנו יודעים כי $J$ בת־מניה.
\end{proof}

\Question{}
מטעמי קריאות ולאור דרישת שמירת סדר מופע השאלות, שאלות 1 עד 7 בחלק השני של המטלה תאוגדנה כסעיפים א' עד ז' בשאלה זו.

נבנה מרחב הסתברות $(\Omega, \PP)$ לכל סיטואציה בסעיפים הבאים.

\Subquestion{}
מטילים קוביה הוגנת 10 פעמים, לכן נגדיר $\Omega = {[6]}^{10}$, שכן כך אנו מתחשבים בכל תוצאת 10 הטלות ולפי סדר, עוד נגדיר $\PP = \PP_p$ עבור פונקציית הסתברות נקודתית אחידה $p : [6] \to [0, 1]$.
המאורע שקיבלנו לפחות פעם אחת 1 הוא $A = \{ (x_1, \dots, x_{10}) \in \Omega \mid \exists i \in [10], x_i = 1 \}$.
נחשב במקום את המשלים, נגדיר מאורע $B = \Omega \setminus A = \{ (x_i) \mid \forall i, x_i \ne 1 \}$ ולכן נקבל $|B| = 5^{10}$.
נקבל אם כן $\PP(A) = \PP(\Omega \setminus B) = 1 - \frac{5^{10}}{6^{10}}$.
נחשב את ההסתברות שיצא 1 בדיוק פעם אחת, נגדיר מאורע $C = \{ (x_i) \in \Omega \mid \exists ! i \in [10], x_i = 1 \}$.
משיקולים קומבינטוריים נקבל $|C| = 10 \cdot 5^9$ ולכן בהתאם $\PP(C) = \frac{10 \cdot 5^9}{6^{10}}$.

\Subquestion{}
מטילים שתי קוביות הוגנות, לכן $\Omega = {[6]}^2$ ו־$\PP$ פונקציית הסתברות אחידה, ונחשב את ההסתברות שסכום התוצאות הוא לפחות 7.
נגדיר את המאורע $A = \{ (n, m) \in \Omega \mid n + m \ge 7 \}$, ונפרק אותו למקרים, נגדיר $A_i = \{ (i, n) \in \Omega \mid i + n \ge 7 \}$ עבור $1 \le i \le 6$ ונקבל $A = \bigcup_{i \in [6]} A_i$.
לכן
\[
	\PP(A) = \PP(\bigcup_{i \in [6]} A_i)
	= \sum_{i = 1}^{6} \PP(A_i)
	= \sum_{i = 1}^{6} \frac{i}{6^2}
	= \frac{1}{6^2} \cdot \frac{6 \cdot 7}{2}
\]

\Subquestion{}
נתונה קבוצה של עשרה חודשי לידה ולכן נקבע $\Omega = {[12]}^{10}$, ונתון כי ההסתברות היא אחידה.
נחשב את הסבירות ששני אנשים לפחות נולדו באותו חודש.
המאורע הוא $A = \{ (x_i) \in \Omega \mid \exists i, j : 1 \le i < j \le 10, x_i = x_j \}$.
נחשב על־ידי חישוב המשלים, נגדיר $B = \{ (x_i) \in \Omega \mid \forall i, j : i \ne j \implies x_i \ne x_j \}$ המאורע שאין שני אנשים שחוגגים ביחד חודש הולדת.
משיקולים קומבינטוריים נקבל $|B| = \frac{12!}{2!}$ ולכן $\PP(A) = 1 - \PP(B) = 1 - \frac{12!}{2 \cdot 12^{10}}$.

\Subquestion{}
מחלקים 12 תפוזים ל־8 סלסלות, לכן נייצג את הסיטואציה על־ידי מספר התפוזים בכל סלסלה, דהינו נגדיר $\Omega = \{ (x_i) \in {[12]}^8 \mid \sum_{i = 1}^{8} = 12 \}$.
נחשב את הסיכוי שבכל סלסלה יש לפחות תפוז אחד, נגדיר $A = \{ (x_i) \in \Omega \mid \forall 1 \le i \le 8, x_1 \ge 1 \}$.
משיקולים קומבינטוריים נקבל $|\Omega| = \binom{21}{12}$, ובאופן דומה $|A| = \binom{13}{8}$ ולכן $\PP(A) = \frac{|A|}{|\Omega|}$.

\Subquestion{}
שולפים 13 קלפים בזה אחר זה מחפיסת קלפים סטנדרטית ללא חשיבות לסדר, לכן $\Omega = \{ \omega \in \mathcal{P}([52]) \mid |\omega| = 13 \}$.
נבחין כי משיקולים קומבינטוריים $|\Omega| = \binom{52}{13}$.
נרצה לחשב שארבעה קלפים בדיוק הם מהסוג יהלום, נגדיר שרירותית שקבוצת הקלפים ביהלום מיוצגים על־ידי המספרים 1 עד 13, ולכן $A = \{ \omega \in \Omega \mid |\omega \cap [13]| = 4 \}$.
בהתאם נקבל $|A| = \binom{13}{4} \cdot \binom{52 - 13}{9}$ וכמובן בהתאם להסתברות אחידה $\PP(A) = \frac{|A|}{|\Omega|}$.

\Subquestion{}
נתון כי שמונה בנים ושמונה בנות מתיישבים בשורה, נייצג אותם על־ידי קבוצת מיקומי הבנים, כאשר בשאר המקומות יושבות בנות, דהינו $\Omega = \{ X \subseteq [16] \mid |X| = 8 \}$.
נקבל אם כך $|\Omega| = \binom{16}{8}$.
נרצה לחשב את הסיכוי שהבנים והבנות יושבים לסירוגין, למעשה יש רק שתי אפשרויות כאלה, או שיש בן בתחילת השורה או שיש בת, ולכן $A = \{ \{1, 3, 5, 7, 9, 11, 13, 15\}, \{2, 4, 6, 8, 10, 12, 14\}\}$.
בהתאם נקבל $\PP(A) = \frac{2}{\binom{16}{8}}$.

\Subquestion{}
להגרלת לוטו יש $n$ כרטיסים ומתוכם $k$ כרטיסים זכו, ידוע שלאדם מסוים יש $m$ כרטיסים, נגדיר שרירותית שלכל כרטיס מספר בין 1 ל־$n$ וש־$k \le n$ הכרטיסים הראשונים הם זוכים, לכן $\Omega = \{ X \subseteq [n] \mid |X| = m \}$.
כמובן $|\Omega| = \binom{n}{m}$.
נבחן את המרוקע שהאדם זכה, דהינו לפחות אחד הכרטיסים זוכה, אז $A = \{ \omega \in \Omega \mid |\omega \cap [k]| > 0 \}$, מטעמי נוחות נבחן את המאורע המשלים,
נגדיר $B = \{ \omega \in \Omega \mid \omega \cap [k] = \emptyset \}$ ונחשב את גודלוונחשבו.
זהו המקרה שבו כל $m$ הכרטיסים ממוספרים $k + 1 \le i \le n$ ויש $\binom{n - k - 1}{m}$ אפשרויות כאלה, כאשר נגדיר שאם המספר לבחירה שלילי אז תוצאת הביטוי היא 0.
נקבל בהתאם שהסיכוי של האדם לזכות הוא $\PP(A) = 1 - \PP(B) = 1 - \frac{\binom{n - k - 1}{m}}{\binom{n}{m}}$.

\Question{}
תהי $\Omega$ בת־מניה לא סופית, ונגדיר $\mathcal{F} = \{ X \in \mathcal{P}(\Omega) \mid |X| < \infty \lor |X^C| < \infty \}$. \\*
נגדיר $\PP : \mathcal{F} \to [0, 1]$ על־ידי
\[
	\PP(A) = \begin{cases}
		0 & |A| < \infty \\
		1 & |A^C| < \infty
	\end{cases}
\]

\Subquestion{}
נוכיח כי $\mathcal{F}$ סגור למשלים ולאיחוד סופי.
\begin{proof}
	יהיה מאורע $A \in \mathcal{F}$, ונבחן את $A^C$, מהגדרת $\mathcal{F}$ נובע $|A| < \infty \lor |A^C| < \infty$ ובהתאם נקבל $|A^C| < \infty \lor |A| < \infty$ ולכן יש סגירות למשלים ב־$\mathcal{F}$.

	נניח $A, B \in \mathcal{F}$, אם $|A|, |B| < \infty$ אז כמובן $|A \cup B| < \infty$ ולכן $A \cup B \in \mathcal{F}$,
	אם $|A|, |B| = \infty$ אז $A \cup B = {(A^C \cap B^C)}^C$ אבל $A^C$ ו־$B^C$ שתיהן סופיות ובהכרח גם $A^C \cap B^C$ סופית ולכן נקבל גם $A \cup B \in \mathcal{F}$.
	נניח $|A|, |B^C| < \infty$, אז ${(A \cup B)}^C = A^C \cap B^C$ אבל $B^C$ סופית ולכן גם החיתוך $A^C \cap B^C$ ולכן $A \cup B \in \mathcal{F}$ בכל מצב והקבוצה סגורה לאיחוד זוגות. \\*
	נניח ${\{A_i\}}_{i = 1}^l \subseteq \mathcal{F}$ קבוצת מאורעות סופית, ונבדוק האם $\bigcup_{i = 1}^l A_i \in \mathcal{F}$, אכן קיבלנו כי $A_1 \cup A_2 \in \mathcal{F}$ ולכן נוכל להראות באופן אינדוקטיבי שכל איחוד סופי בקבוצה.

	מצאנו ש־$\mathcal{F}$ סגורה להשלמה ולאיחוד סופי.
\end{proof}

\Subquestion{}
נוכיח כי $\PP$ מקיימת $\PP(\Omega) = 1, \PP(\emptyset) = 0$ ושמתקיימת אדיטיביות סופית.
\begin{proof}
	נבחין כי $\emptyset$ היא קבוצה סופית, ולכן מהגדרת $\PP$ מתקיים $\PP(\emptyset) = 0$.
	ידוע גם כי $\Omega^C = \Omega \setminus \Omega = \emptyset$ ולכן $\PP(\Omega) = 1$ מהגדרת הפונקציה.

	נבדוק קיום אדיטיביות סופית.
	תהינה ${\{A_i\}}_{i = 1}^l$ קבוצת מאורעות זרים סופית, תחילה נבחין כי לכל היותר מאורע אחד הוא לא סופי בקבוצה זו, נניח בשלילה שיש $A_i, A_j$ זרים ולא סופיים, אז $A_i^C, A_j^C$ סופיים ולכן נוכל לבחור איבר שלא באיחוד שלהם.
	אם $a$ איבר כזה אז $a \in A_i, A_j$ ולכן קיבלנו סתירה לזרות. נניח $A_i$ סופי לכל $2 \le i \le l$ ו־$A_1$ יכול להיות סופי ויכול להיות אינסופי.
	במקרה בו $A_1$ סופי אז נקבל $\bigcup_{i = 1}^l A_i$ סופי אף הוא ולכן $\PP(\bigcup_{i = 1}^l) = 0 = \sum_{i = 1}^l \PP(A_i)$.
	במקרה בו $A_1$ אינסופית, האיחוד של שאר הקבוצות הוא סופי ולכן נקבל $1 = \PP(\bigcup_{i = 1}^l A_i) = \PP(A_1) + \sum_{i = 2}^l A_i = \sum_{i = 1}^l A_i$.
\end{proof}

\Subquestion{}
נוכיח שקיים מאורע $A \in \mathcal{F}$ כך ש־$\PP(A) = 1$ כך ש־$A = \bigcup_{i = 1}^\infty A_i$ למאועות זרים כך ש־$\PP(A_i) = 0$ לכל $i \in \NN$ ונסיק כי $\PP$ איננה סיגמא אדיטיבית.
\begin{proof}
	ידוע כי $\Omega$ בת־מניה ולכן $|\Omega| = |\NN|$ ובהתאם קיימת $f : \NN \to \Omega$ חד־חד ערכית ועל. \\*
	נגדיר ${\{A_i\}}_{i = 1}^\infty \subseteq \mathcal{F}$ על־ידי $A_i = \{ f(i) \}$ לכל $i \in \NN$, נקבל בהתאם $\bigcup_{i \in \NN} A_i = f(\NN) = \Omega$. \\*
	לכן נוכל להסיק $\PP(\bigcup_{i = 1}^\infty A_i) = \PP(\Omega) = 1$ אבל גם $\PP(A_i) = 0$ שכן $A_i$ יחידון ולכן ככלל סופית לכל $i \in \NN$, וקיבלנו את המבוקש.

	נבחין כי תוצאה זו סותרת באופן ישיר את תכונת הסיגמא־אדיטיביות ולכן $\PP$ איננה פונקציית הסתבות.
\end{proof}

\end{document} % chktex 17
