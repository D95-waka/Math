\documentclass[a4paper]{article}

% packages
\usepackage{inputenc, amsmath, amsthm, thmtools, amsfonts, amssymb, luacode, catchfile, tikzducks, hyperref}
\usepackage[a4paper, margin=50pt, includeheadfoot]{geometry} % set page margins
\usepackage[shortlabels]{enumitem}
\usepackage[skip=3pt, indent=0pt]{parskip}

% language
\usepackage[bidi=basic, layout=tabular, provide=*]{babel}
\babelprovide[main, import]{hebrew}
\babelprovide{rl}
\babelfont{rm}{Libertinus Serif}
\babelfont{sf}{Libertinus Sans}
\babelfont{tt}{Libertinus Mono}

% style
\AddToHook{cmd/section/before}{\clearpage}	% Add line break before section
\linespread{1.3}
\setcounter{secnumdepth}{0}		% Remove default number tags from sections, this won't do well with theorems
\AtBeginDocument{\setlength{\belowdisplayskip}{3pt}}
\AtBeginDocument{\setlength{\abovedisplayskip}{3pt}}

% operators
\DeclareMathOperator\cis{cis}
\DeclareMathOperator\Sp{Sp}
\DeclareMathOperator\tr{tr}
\DeclareMathOperator\im{Im}
\DeclareMathOperator\re{Re}
\DeclareMathOperator\diag{diag}
\DeclareMathOperator*\lowlim{\underline{lim}}
\DeclareMathOperator*\uplim{\overline{lim}}
\DeclareMathOperator\rng{rng}
\DeclareMathOperator\Sym{Sym}
\DeclareMathOperator\Arg{Arg}
\DeclareMathOperator\Log{Log}
\DeclareMathOperator\dom{dom}

% commands
%\renewcommand\qedsymbol{\textbf{מש''ל}}
%\renewcommand\qedsymbol{\fbox{\emoji{lizard}}}
\newcommand{\NN}[0]{\mathbb{N}}
\newcommand{\ZZ}[0]{\mathbb{Z}}
\newcommand{\QQ}[0]{\mathbb{Q}}
\newcommand{\RR}[0]{\mathbb{R}}
\newcommand{\CC}[0]{\mathbb{C}}
\newcommand{\FF}[0]{\mathbb{F}}
\newcommand{\PP}[0]{\mathbb{P}}
\newcommand{\TT}[0]{\mathbb{T}}
\newcommand{\acts}[0]{\circlearrowright}
\newcommand{\explain}[2] {
	\begin{flalign*}
		 && \text{#2} && \text{#1}
	\end{flalign*}
}
\newcommand{\maketitleprint}[0]{ \begin{center}
	\begin{tikzpicture}[scale=3]
		\duck[graduate=gray!20!black, tassel=red!70!black]
	\end{tikzpicture}	
\end{center}
}

% theorem commands
\newtheoremstyle{c_remark}
	{}	% Space above
	{}	% Space below
	{}% Body font
	{}	% Indent amount
	{\bfseries}	% Theorem head font
	{}	% Punctuation after theorem head
	{.5em}	% Space after theorem head
	{\thmname{#1}\thmnumber{ #2}\thmnote{ \normalfont{\text{(#3)}}}}	% head content
\newtheoremstyle{c_definition}
	{3pt}	% Space above
	{3pt}	% Space below
	{}% Body font
	{}	% Indent amount
	{\bfseries}	% Theorem head font
	{}	% Punctuation after theorem head
	{.5em}	% Space after theorem head
	{\thmname{#1}\thmnumber{ #2}\thmnote{ \normalfont{\text{(#3)}}}}	% head content
\newtheoremstyle{c_plain}
	{3pt}	% Space above
	{3pt}	% Space below
	{\itshape}% Body font
	{}	% Indent amount
	{\bfseries}	% Theorem head font
	{}	% Punctuation after theorem head
	{.5em}	% Space after theorem head
	{\thmname{#1}\thmnumber{ #2}\thmnote{ \text{(#3)}}}	% head content

\theoremstyle{c_plain}
\newtheorem{theorem}{משפט}[section]
\newtheorem{lemma}[theorem]{למה}
\newtheorem{proposition}[theorem]{טענה}
\newtheorem*{proposition*}{טענה}
%\newtheorem{corollary}[theorem]{אין חלופה עברית}

\theoremstyle{c_definition}
\newtheorem{definition}[theorem]{הגדרה}
\newtheorem*{definition*}{הגדרה}
\newtheorem{example}{דוגמה}[section]
\newtheorem{exercise}{תרגיל}[section]

\theoremstyle{c_remark}
\newtheorem*{remark}{הערה}
\newtheorem*{solution}{פתרון}
\newtheorem{conclusion}[theorem]{מסקנה}
\newtheorem{notation}[theorem]{סימון}

% Questions related commands
\newcounter{question}
\setcounter{question}{1}
\newcounter{sub_question}
\setcounter{sub_question}{1}

\newcommand{\question}[1][0]{
	\ifthenelse{#1 = 0}{}{\setcounter{question}{#1}}
	\subsection{שאלה \arabic{question}}
	\addtocounter{question}{1}
	\setcounter{sub_question}{1}
}

\newcommand{\subquestion}[1][0]{
	\ifthenelse{#1 = 0}{}{\setcounter{sub_question}{#1}}
	\subsubsection{סעיף \localecounter{letters.gershayim}{sub_question}}
	\addtocounter{sub_question}{1}
}

% import lua and start of document
\directlua{common = require ('../common')}

\GetEnv{AUTHOR}

% headers
\author{\AUTHOR}
\date\today

\title{פתרון מטלה 01 --- תורת ההסתברות (1), 80420}

\begin{document}
\maketitle
\maketitleprint{}

\Question{}
יהי $(\Omega, \PP)$ מרחב הסתברות בדיד, נוכיח או נפריך את הטענות הבאות.

\Subquestion{}
נוכיח שאם $A \subseteq \Omega$ מקיימת $\PP(A) = 0$ אז לכל $B \subseteq \Omega$ מתקיים $\PP(A \cup B) = \PP(B)$.
\begin{proof}
	נבחין כי $D = B \setminus A$ היא זרה באיחוד ל־$A \cap B$, לכן נקבל $\PP(B) = \PP(D \cup (A \cap B)) = \PP(D) + \PP(A \cap B)$. \\*
	אבל $A \cap B \subseteq A$ ולכן מתכונות פונקציית הסתברות נקבל $\PP(A \cap B) \le \PP(A) = 0$. \\*
	לכן $\PP(D) = \PP(B)$. נשתמש בטענה זו ונקבל
	\[
		\PP(A \cup B) = \PP(A \cup (B \setminus A)) = \PP(A) + \PP(D) = 0 + \PP(B)
	\]
	וקיבלנו כי השוויון אכן מתקיים.
\end{proof}

\Subquestion{}
נוכיח שאם $A \subseteq B$ אז $\PP(A) \le \PP(B)$.
\begin{proof}
	למעשה תכונה זו הוכחה בהרצאה, נעתיק את ההוכחה:
	\begin{enumerate}
		\item נראה כי $\PP(\emptyset) = \sum_{i = 1}^\infty \PP(\emptyset)$ שכן כל איחוד של קבוצות ריקות הוא זר, לכן אילו $\PP(\emptyset) \ne 0$ נקבל ישר סתירה, נסיק כי $\PP(\emptyset) = 0$ בלבד.
		\item נגדיר $A_i = \emptyset$ לכל $i > n$ ונשתמש בסיגמא־אדיטיביות ונקבל
			\[
				\PP(\bigcup_{i \in I} A_i)
				= \PP(\bigcup_{i \in \NN} A_i)
				= \sum_{i \in \NN} \PP(A_i)
				= \sum_{i \in I} \PP(A_i)
			\]
		\item נשתמש בתכונה 2 על $B, B \setminus A$, אלו הן קבוצות זרות כמובן, אם נגדיר $D = A \cup (B \setminus A)$ נקבל $\PP(D) = \PP(A) + \PP(B \setminus A) \ge \PP(A)$.
	\end{enumerate}
\end{proof}

\Subquestion{}
נסתור את הטענה כי אם $A \subsetneq B$ אז $\PP(A) \lneq \PP(B)$.
\begin{solution}
	נגדיר $\Omega = [3]$ ו־$p(1) = p(2) = \frac{1}{2}, p(3) = 0$, ונבחן את $\PP_p$. \\*
	אם נגדיר $A = \{ 1 \}, B = \{ 1, 3 \}$ נקבל $A \subsetneq B$ אבל $\PP(A) = \PP(B) = \frac{1}{2}$ בסתירה לטענה.
\end{solution}

\Subquestion{}
נוכיח שלכל $A, B \subseteq \Omega$ מתקיים $\PP(A \cap B) \ge \PP(A) + \PP(B) - 1$
\begin{proof}
	מתכונות פונקציית הסתברות נקבל $\PP(A \cup B) = \PP(A) + \PP(B) - \PP(A \cap B)$, \\*
	לאחר החלפת אגפים נקבל
	\[
		\PP(A \cap B) = \PP(A) + \PP(B) - \PP(A \cup B) \ge \PP(A) + \PP(B) - 1
	\]
	שכן מתקיים $1 \ge \PP(X)$ לכל $X \in \mathcal{F}$ ובפרט עבור $A \cup B$.
\end{proof}

\Subquestion{}
נוכיח כי מתקיים $\PP(A \triangle B) = \PP(A) + \PP(B) - 2 \PP(A \cap B)$.
\begin{proof}
	נבחין תחילה כי $A \triangle B = (A \cup B) \setminus (A \cap B) = (A \cup B) \cap {(A \cap B)}^C$, \\*
	נשתמש בשוויון מהסעיף הקודם ונקבל
	\begin{align*}
		\PP(A \triangle B)
		& = \PP((A \cup B) \cap {(A \cap B)}^C) \\
		& = \PP(A \cup B) + \PP({(A \cap B)}^C) - \PP((A \cup B) \cup {(A \cap B)}^C) \\
		& = \PP(A \cup B) + \PP(\Omega \setminus (A \cap B)) - \PP(\Omega) \\ \tag{1}
		& = \PP(A \cup B) + \PP(\Omega) - \PP(A \cap B) - 1 \\
		& = \PP(A) + \PP(B) - 2 \PP(A \cap B)
	\end{align*}
	כאשר המעבר $(1)$ נובע מהשוויון $A \cup B \cup {(A \cap B)}^C = A \cup B \cup A^C \cup B^C = \Omega$ שנובעת מדה־מורגן לקבוצות.
\end{proof}

\Question{}
נוכיח שקיימת פונקציית הסתברות יחידה על $\Omega = \NN$ המקיימת $\PP(\{n\}) = 3 \PP(\{n + 1\})$.
\begin{proof}
	נוכיח שקיימת כזאת פונקציית הסתברות, נגדיר $p : \NN \to [0, 1]$ על־ידי $p(n) = 2 \cdot \frac{1}{3^n}$. \\*
	נקבל מנוסחת סכום סדרה הנדסית ש־$\sum_{n = 1}^{\infty} p(n) = 1$ ולכן זוהי פונקציית הסתברות נקודתית והיא משרה פונקציית הסתברות $\PP_p$ המקיימת את תנאי הטענה.
	אז הוכחנו שקיימת לכל הפחות פונקציה אחת כזו.

	נעבור להוכחת יחידות, נניח ש־$\PP, \PP'$ שתי פונקציות הסתברות המקיימות את הטענה. \\*
	באינדוקציה על $n$ נקבל כי $\PP(\{n\}) = 3^{1 - n} \PP(\{1\})$.
	מסיגמא־אדיטיביות נקבל
	\[
		1
		= \PP(\NN)
		= \PP(\bigcup_{n \in \NN} \{ n \})
		= \sum_{n \in \NN} \PP(\{ n \})
		= \sum_{n = 1}^{\infty} \PP(\{1\}) 3^{1 - n}
		= \PP(\{1\}) \sum_{n = 1}^{\infty} 3^{1 - n}
		= \PP(\{1\}) \cdot \frac{1}{1 - \frac{1}{3}}
	\]
	ולכן נסיק $\PP(\{1\}) = \frac{2}{3}$, וממהלך זה נוכל גם להסיק על־ידי סעיף ב' בתנאים שקולים לפונקציית הסתברות בדידה שקיימות פונקציות הסתברות נקודתיות $p, p'$ כך ש־$\PP = \PP_p, \PP' = \PP_{p'}$.
	אבל קיבלנו אם כך ש־$p(1) = p'(1) = \frac{2}{3}$ ובהתאם להגדרה הרקורסיבית נקבל $p = p'$ ובהתאם $\PP = \PP'$, דהינו קיימת פונקציית הסתברות יחידה המקיימת את תנאי הטענה.
\end{proof}
נחשב את $\PP(\NN)$ ואת $\PP(3\NN)$.
כמובן $\PP(\NN) = 1$, עבור $3\NN$ נקבל
\[
	\PP(3\NN) 
	= \sum_{n \in 3\NN} \PP(\{n\}) 
	= \sum_{n = 1}^{\infty} p(3n) 
	= \sum_{n = 1}^{\infty} 2 \cdot \frac{1}{3^{3n}}
	= \sum_{n = 1}^{\infty} 2 \cdot \frac{1}{9^{n}}
	= \frac{1}{4}
\]

\Question{}
מטעמי קריאות ולאור דרישת שמירת סדר מופע השאלות, שאלות 1 עד 7 בחלק השני של המטלה תאוגדנה כסעיפים א' עד ז' בשאלה זו.

נבנה מרחב הסתברות $(\Omega, \PP)$ לכל סיטואציה בסעיפים הבאים.

\Subquestion{}
מטילים קוביה הוגנת 10 פעמים, לכן נגדיר $\Omega = {[6]}^{10}$, שכן כך אנו מתחשבים בכל תוצאת 10 הטלות ולפי סדר, עוד נגדיר $\PP = \PP_p$ עבור פונקציית הסתברות נקודתית אחידה $p : [6] \to [0, 1]$.
המאורע שקיבלנו לפחות פעם אחת 1 הוא $A = \{ (x_1, \dots, x_{10}) \in \Omega \mid \exists i \in [10], x_i = 1 \}$.
נחשב במקום את המשלים, נגדיר מאורע $B = \Omega \setminus A = \{ (x_i) \mid \forall i, x_i \ne 1 \}$ ולכן נקבל $|B| = 5^{10}$.
נקבל אם כן $\PP(A) = \PP(\Omega \setminus B) = 1 - \frac{5^{10}}{6^{10}}$.
נחשב את ההסתברות שיצא 1 בדיוק פעם אחת, נגדיר מאורע $C = \{ (x_i) \in \Omega \mid \exists ! i \in [10], x_i = 1 \}$.
משיקולים קומבינטוריים נקבל $|C| = 10 \cdot 5^9$ ולכן בהתאם $\PP(C) = \frac{10 \cdot 5^9}{6^{10}}$.

\Subquestion{}
מטילים שתי קוביות הוגנות, לכן $\Omega = {[6]}^2$ ו־$\PP$ פונקציית הסתברות אחידה, ונחשב את ההסתברות שסכום התוצאות הוא לפחות 7.
נגדיר את המאורע $A = \{ (n, m) \in \Omega \mid n + m \ge 7 \}$, ונפרק אותו למקרים, נגדיר $A_i = \{ (i, n) \in \Omega \mid i + n \ge 7 \}$ עבור $1 \le i \le 6$ ונקבל $A = \bigcup_{i \in [6]} A_i$.
לכן
\[
	\PP(A) = \PP(\bigcup_{i \in [6]} A_i)
	= \sum_{i = 1}^{6} \PP(A_i)
	= \sum_{i = 1}^{6} \frac{i}{6^2}
	= \frac{1}{6^2} \cdot \frac{6 \cdot 7}{2}
\]

\Subquestion{}
נתונה קבוצה של עשרה חודשי לידה ולכן נקבע $\Omega = {[12]}^{10}$, ונתון כי ההסתברות היא אחידה.
נחשב את הסבירות ששני אנשים לפחות נולדו באותו חודש.
המאורע הוא $A = \{ (x_i) \in \Omega \mid \exists i, j : 1 \le i < j \le 10, x_i = x_j \}$.
נחשב על־ידי חישוב המשלים, נגדיר $B = \{ (x_i) \in \Omega \mid \forall i, j : i \ne j \implies x_i \ne x_j \}$ המאורע שאין שני אנשים שחוגגים ביחד חודש הולדת.
משיקולים קומבינטוריים נקבל $|B| = \frac{12!}{2!}$ ולכן $\PP(A) = 1 - \PP(B) = 1 - \frac{12!}{2 \cdot 12^{10}}$.

\Subquestion{}
חידת התפוזים

\end{document}
