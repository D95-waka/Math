\documentclass[a4paper]{article}

% packages
\usepackage{inputenc, amsmath, amsthm, thmtools, amsfonts, amssymb, luacode, catchfile, tikzducks, hyperref}
\usepackage[a4paper, margin=50pt, includeheadfoot]{geometry} % set page margins
\usepackage[shortlabels]{enumitem}
\usepackage[skip=3pt, indent=0pt]{parskip}

% language
\usepackage[bidi=basic, layout=tabular, provide=*]{babel}
\babelprovide[main, import]{hebrew}
\babelprovide{rl}
\babelfont{rm}{Libertinus Serif}
\babelfont{sf}{Libertinus Sans}
\babelfont{tt}{Libertinus Mono}

% style
\AddToHook{cmd/section/before}{\clearpage}	% Add line break before section
\linespread{1.3}
\setcounter{secnumdepth}{0}		% Remove default number tags from sections, this won't do well with theorems
\AtBeginDocument{\setlength{\belowdisplayskip}{3pt}}
\AtBeginDocument{\setlength{\abovedisplayskip}{3pt}}

% operators
\DeclareMathOperator\cis{cis}
\DeclareMathOperator\Sp{Sp}
\DeclareMathOperator\tr{tr}
\DeclareMathOperator\im{Im}
\DeclareMathOperator\re{Re}
\DeclareMathOperator\diag{diag}
\DeclareMathOperator*\lowlim{\underline{lim}}
\DeclareMathOperator*\uplim{\overline{lim}}
\DeclareMathOperator\rng{rng}
\DeclareMathOperator\Sym{Sym}
\DeclareMathOperator\Arg{Arg}
\DeclareMathOperator\Log{Log}
\DeclareMathOperator\dom{dom}

% commands
%\renewcommand\qedsymbol{\textbf{מש''ל}}
%\renewcommand\qedsymbol{\fbox{\emoji{lizard}}}
\newcommand{\NN}[0]{\mathbb{N}}
\newcommand{\ZZ}[0]{\mathbb{Z}}
\newcommand{\QQ}[0]{\mathbb{Q}}
\newcommand{\RR}[0]{\mathbb{R}}
\newcommand{\CC}[0]{\mathbb{C}}
\newcommand{\FF}[0]{\mathbb{F}}
\newcommand{\PP}[0]{\mathbb{P}}
\newcommand{\TT}[0]{\mathbb{T}}
\newcommand{\acts}[0]{\circlearrowright}
\newcommand{\explain}[2] {
	\begin{flalign*}
		 && \text{#2} && \text{#1}
	\end{flalign*}
}
\newcommand{\maketitleprint}[0]{ \begin{center}
	\begin{tikzpicture}[scale=3]
		\duck[graduate=gray!20!black, tassel=red!70!black]
	\end{tikzpicture}	
\end{center}
}

% theorem commands
\newtheoremstyle{c_remark}
	{}	% Space above
	{}	% Space below
	{}% Body font
	{}	% Indent amount
	{\bfseries}	% Theorem head font
	{}	% Punctuation after theorem head
	{.5em}	% Space after theorem head
	{\thmname{#1}\thmnumber{ #2}\thmnote{ \normalfont{\text{(#3)}}}}	% head content
\newtheoremstyle{c_definition}
	{3pt}	% Space above
	{3pt}	% Space below
	{}% Body font
	{}	% Indent amount
	{\bfseries}	% Theorem head font
	{}	% Punctuation after theorem head
	{.5em}	% Space after theorem head
	{\thmname{#1}\thmnumber{ #2}\thmnote{ \normalfont{\text{(#3)}}}}	% head content
\newtheoremstyle{c_plain}
	{3pt}	% Space above
	{3pt}	% Space below
	{\itshape}% Body font
	{}	% Indent amount
	{\bfseries}	% Theorem head font
	{}	% Punctuation after theorem head
	{.5em}	% Space after theorem head
	{\thmname{#1}\thmnumber{ #2}\thmnote{ \text{(#3)}}}	% head content

\theoremstyle{c_plain}
\newtheorem{theorem}{משפט}[section]
\newtheorem{lemma}[theorem]{למה}
\newtheorem{proposition}[theorem]{טענה}
\newtheorem*{proposition*}{טענה}
%\newtheorem{corollary}[theorem]{אין חלופה עברית}

\theoremstyle{c_definition}
\newtheorem{definition}[theorem]{הגדרה}
\newtheorem*{definition*}{הגדרה}
\newtheorem{example}{דוגמה}[section]
\newtheorem{exercise}{תרגיל}[section]

\theoremstyle{c_remark}
\newtheorem*{remark}{הערה}
\newtheorem*{solution}{פתרון}
\newtheorem{conclusion}[theorem]{מסקנה}
\newtheorem{notation}[theorem]{סימון}

% Questions related commands
\newcounter{question}
\setcounter{question}{1}
\newcounter{sub_question}
\setcounter{sub_question}{1}

\newcommand{\question}[1][0]{
	\ifthenelse{#1 = 0}{}{\setcounter{question}{#1}}
	\subsection{שאלה \arabic{question}}
	\addtocounter{question}{1}
	\setcounter{sub_question}{1}
}

\newcommand{\subquestion}[1][0]{
	\ifthenelse{#1 = 0}{}{\setcounter{sub_question}{#1}}
	\subsubsection{סעיף \localecounter{letters.gershayim}{sub_question}}
	\addtocounter{sub_question}{1}
}

% import lua and start of document
\directlua{common = require ('../common')}

\GetEnv{AUTHOR}

% headers
\author{\AUTHOR}
\date\today

\title{תורת ההסתברות 1 --- סיכום}
\setcounter{secnumdepth}{2}
% chktex-file 9
% chktex-file 17

\hypersetup{}
\begin{document}
\maketitle
\maketitleprint{}

\tableofcontents

\section{שיעור 1 --- 29.10.2024}

\subsection{מבוא הקורס}
% אורי גורביץ' הוא המרצה, הקורס הוא מבוא להסתברות. הוא חצי חירש תתמודד עם זה במקרה הצורך.
נלמד לפי ספר שעוד לא יצא לאור שנכתב על־ידי אורי עצמו, הוא עוד לא סופי ויש בו בעיות ואי־דיוקים, תשיג את הספר הזה.
כן יש הבדל בין הקורס והספר אז לא לסמוך על הסדר שלו גם כשאתה משיג אותו, אבל זו תוספת מאוד נוחה.
יש סימון של כוכביות לחומר מוסף, כדאי לעבור עליו לקראת המבחן כי זה יתן לנו עוד אינטואיציה והעמקה של ההבנה.

נשים לב כי ענף ההסתברות הוא ענף חדש יחסית, שהתפתח הרבה אחרי שאר הענפים הקלאסיים של המתמטיקה, למעשה רק לפני 400 שנה נשאלה על־ידי נזיר במהלך חקר של משחק אקראי השאלה הראשית של העולם הזה, מה ההסתברות של הצלחה במשחק.

נעבור לדבר על פילוסופיה של ההסתברות.
מה המשמעות של הטלת מטבע מבחינת הסתברות?
ישנה הגישה של השכיחות, שמציגה הסתברות כתוצאה במקרה של חזרה על ניסוי כמות גדולה מאוד של פעמים.
יש כמה בעיות בזה, לרבות חוסר היכולת להגדיר במדויק אמירה כזו, הטיות שנובעות מפיזיקה, מטבעות הם לא מאוזנים לדוגמה.
הבעיה הראשית היא שלא לכל בעיה אפשר לפנות בצורה כזאת.
ישנה גישה נוספת, היא הגישה האוביקטיבית או המתמטית, הגישה הזו בעצם היא תרגום בעיה מהמציאות לבעיה מתמטית פורמלית.
לדוגמה נשאל את השאלה מה ההסתברות לקבל 6 בהגרלה של כל המספרים מ־1 עד מיליון.
השיטה ההסתברותית קובעת שאם אני רוצה להוכיח קיום של איזשהו אוביקט, לפעמים אפשר לעשות את זה על־ידי הגרלה של אוביקט כזה והוכחה שיש הסתברות חיובית שהוא יוגרל, וזו הוכחה שהוא קיים.
מה התחזיות שינבעו מתורת ההסתברות? לדוגמה אי־אפשר לחזות הטלת מטבע בודדת, אבל היא כן נותנת הבנה כללית של הטלת 1000 מטבעות, הסתברויות קטנות מספיק יכולות להיות זניחות ובמקרה זה נוכל להתעלם מהן.
לפחות בתחילת הקורס נדבר על תרגום של בעיות מהמציאות לבעיות מתמטיות, זה אומנם חלק פחות ריגורזי, אבל הוא כן חשוב ליצירת קישור בין המציאות לבין החומר הנלמד.

דבר אחרון, ישנה השאלה הפילוסופית של האם באמת יש הסתברות שכן לא בטוח שיש אקראיות בטבע, הגישה לנושא מבחינה פיזיקלית קצת השתנתה בעת האחרונה וקשה לענות על השאלה הזאת.
יש לנו תורות פיזיקליות שהן הסתברותיות בעיקרן, כמו תורת הקוונטים, תורה זו לא סתם הסתברותית, אנחנו לא מנסים לפתור בעיות הסתברותיות אלא ממש משתמשים במודלים סטטיסטיים כדי לתאר מצב בעולם.
לדוגמה נוכל להסיק ככה מסקנה פשוטה שאם מיכל גז נפתח בחדר, יהיה ערבוב של הגז הפנימי ושל אוויר החדר, זוהי מסקנה הסתברותית.
החלק המדהים הוא שתורת הקוונטים מניחה חוסר דטרמניזם כתכונה יסודית ועד כמה שאפשר לראות יש ניסויים שמוכיחים שבאמת יש חוסר ודאות בטבע.
דהינו שברמה העקרונית הפשוטה באמת אין תוצאה ודאית בכלל למצבים כאלה במציאות.

\subsection{מרחבי מדגם ופונקציית הסתברות}
\begin{definition}[מרחב מדגם]
	מרחב מדגם הוא קבוצה לא ריקה שמהווה העולם להסתברות. \\*
	נסמנה $\Omega$.
	איבר במרחב המדגם נסמן ב־$\omega \in \Omega$ על־פי רוב.
\end{definition}
נוכל להגיד שמרחב במדגם הוא הקבוצה של האיברים שעליה אנחנו שואלים בכלל שאלות, זהו הייצוג של האיברים או המצבים שמעניינים אותנו.
בהתאם נראה עכשיו מספר דוגמות שמקשרות בין אובייקטים שאנו דנים בהם בהסתברות ובהגדרה פורמלית של מרחבי מדגם עבורם.
\begin{example}[מרחבי הסתברות שונים]
	נראה מספר דוגמות למצבים כאלה:
	\begin{itemize}
		\item הטלת מטבע תוגדר על־ידי $\Omega = \{ H, T \}$.
		\item הטלת שלושה מטבעות תהיה באופן דומה $\Omega = {\{H, T\}}^3$.
		\item הטלת קוביה היא $\Omega = [6] = \{ 1, \dots, 6 \}$.
		\item הטלת מטבע ואז אם יוצא עץ (H) אז מטילים קוביה ואם יוצא פלי (T) אז מטילים קוביה עם 8 פאות. \\*
			במקרה זה נסמן $\Omega = \{ H1, H2, H3, \dots, H6, T1, \dots, T8 \} = \{H, T \} \times \{1, \dots, 8 \}$ כאשר הכוונה פה היא לזוג סדור $\langle H, 1 \rangle$.
		\item ערבוב חפיסת קלפים, במקרה זה מרחב המדגם שלנו יהיה סימון של הקלפים כרשימה מספרית בלבד, דהינו $\Omega = S_{52}$. \\*
			נוכל גם לסמן במקום את $\Omega = {\{1, \dots, 52 \}}^{52}$, זהו סימון זהה.
	\end{itemize}
\end{example}
בדוגמה זו קל במיוחד לראות שכל איבר בקבוצה מתאר מצב סופי כלשהו, ואנו יכולים לשאול שאלות הסתברותיות מהצורה מה הסיכוי שנקבל $\omega$ מסוים מתוך $\Omega$, זאת ללא התחשבות בבעיה שממנה אנו מגיעים.
נבחן עתה גם דוגמות למקרים שבהם אין לנו מספר סופי של אפשרויות, למעשה מקרים אלה דומים מאוד למקרים שראינו עד כה.
\begin{example}[מרחבי מדגם לא סופיים]
	מטילים מטבע עד שיוצא $H$, אז מרחב המדגם הוא $\Omega = \NN \cup \{ \infty \}$. \\*
	באופן דומה נוכל לבחון מדידת זמן התפרקות חלקיק, היא $\Omega = \RR_+ \cup \{ \infty \}$.
\end{example}
\begin{definition}[פונקציית הסתברות נקודתית]
	יהי מרחב מדגם $\Omega$ ותהי $p : \Omega \to [0, \infty)$ פונקציה כך שמתקיים
	\[
		\sum_{\omega \in \Omega} p(\omega) = 1
	\]
	אז פונקציה זו נקראת \textbf{פונקציית הסתברות}.
\end{definition}
למעשה פונקציית הסתברות היא מה שאנחנו נזהה עם הסתברות במובן הפשוט, פונקציה זו מגדירה לנו לכל סיטואציה ממרחב המדגם מה הסיכוי שנגיע אליה, כך לדוגמה אם נאמר שהטלת מטבע תגיע בחצי מהמקרים לעץ ובחצי השני לפלי,
אז זו היא פונקציית ההסתברות עצמה, פונקציה שמחזירה חצי עבור עץ וחצי עבור פלי, נראה מספר דוגמות.
\begin{example}[פונקציית הסתברות להטלת מטבע]
	נגדיר $\Omega = \{ H, T \}$ ויהי $0 \le \alpha \le 1$, נגדיר $p(H) = \alpha, p(T) = 1 - \alpha$.
\end{example}
\begin{example}[פונקציית הסתברות אינסופית]
	נגדיר $\Omega = \NN \cup \{ \infty \}$ ו־$p(\omega) = \begin{cases}
		2^{-\omega} & \omega \in \NN \\
		0 & \omega = \infty
	\end{cases}$.
	בדוגמה זו נקבל $\sum_{n = 1}^{\infty} 2^{-n} = 1$ ולכן זו אכן פונקציית הסתברות.
\end{example}
נבחין כי הדוגמה האחרונה מתארת לנו התפלגות של דעיכה, זאת אומרת שלדוגמה אם קיים חלקיק עם זמן מחצית חיים של יחידה אחת, פונקציית הסתברות זו תניב לנו את הסיכוי שהוא התפרק לאחר כמות יחידות זמן כלשהי.
\begin{example}
	נגדיר $\Omega = \NN$ ו־$p(\omega) = \frac{1}{\omega(\omega + 1)}$, נבחין כי אכן $\sum_{n = 1}^{\infty} \frac{1}{n(n + 1)} = 1$.
\end{example}
\begin{definition}[תומך]
	התומך של $p$ הוא $\text{Supp}(p) = \{ \omega \in \Omega \mid p(\omega) > 0 \}$. \\*
	נבחין כי התומך הוא למעשה קבוצת האיברים שאפשרי לקבל לפי פונקציית ההסתברות, כל שאר המצבים מקבלים 0, משמעו הוא שאין אפשרות להגיע אליו.
\end{definition}
\begin{remark}
	נבחין כי תמיד $\mathcal{F} \subseteq \mathcal{P}(\Omega)$.
\end{remark}
\begin{definition}[מאורע]
	מאורע הוא תת־קבוצה של מרחב המדגם, קבוצת כל המאורעות תסומן $\mathcal{F}$.
	עבור מאורע $A$ המאורע המשלים מסומן ב־$A^C = \Omega \setminus A$.
\end{definition}
\begin{definition}[פונקציית הסתברות]
	נגדיר עתה פונקציית הסתברות שאיננה נקודתית.
	יהי מרחב מדגם $\Omega$ וקבוצת מאורעות $\mathcal{F}$. \\*
	תהי $\PP : \mathcal{F} \to [0, \infty)$ המקיימת את התכונות הבאות:
	\begin{enumerate}
		\item $\PP(\Omega) = 1$
		\item לכל ${\{A_i\}}_{i = 1}^\infty \subseteq \mathcal{F}$ סדרת מאורעות שונים מתקיים
			\[
				\sum_{i \in \NN} \PP(A_i) = \PP(\bigcup_{i \in \NN} A_i)
			\]
			דהינו, הפונקציה סכימה בתת־קבוצות בנות מניה.
	\end{enumerate}
	לפונקציה כזו נקרא \textbf{פונקציית ההסתברות} על $(\Omega, \mathcal{F})$.
\end{definition}
\begin{proposition}
	תהי $p$ פונקציית הסתברות נקודתית על $\Omega$ אז נגדיר פונקציית הסתברות $\PP_p$ על־ידי
	\[
		\PP_p(A) = \sum_{\omega \in A} p(\omega)
	\]
	אז $\PP_p$ היא פונקציית הסתברות.
\end{proposition}
\begin{proof}
	נוכיח ששתי התכונות של פונקציית הסתברות מתקיימות.
	\[
		\PP_p(A) = \sum_{\omega \in A} p(\omega) \ge 0
	\]
	שכן זהו סכום אי־שלילי מהגדרת $p$,
	בנוסף נקבל מההגדרה של $p$ כי
	\[
		\PP_p(\Omega) = \sum_{\omega \in \Omega} p(\omega) = 1
	\]
	וקיבלנו כי התכונה הראשונה מתקיימת. \\*
	תהי ${\{A\}}_{i = 1}^\infty \in \mathcal{F}$, אז נקבל
	\[
		\sum_{i \in \NN} \PP_p(A_i) = \sum_{i \in \NN} \left( \sum_{\omega \in A_i} p(\omega) \right) = \sum_{\omega \in \bigcup_{i \in \NN} A_i} p(\omega) = \PP_p(\bigcup_{i \in \NN} A_i)
	\]
	ולכן גם התכונה השנייה מתקיימת וקיבלנו כי $\PP_p$ היא אכן פונקציית הסתברות.
\end{proof}
נשים לב כי בעוד פונקציית הסתברות נקודתית מאפשרת לנו לדון בהסתברות של איבר בודד בקבוצות בנות מניה, פונקציית הסתברות למעשה מאפשרת לנו לדון בהסתברות של מאורעות, הם קבוצות של כמה מצבים אפשריים, ובכך להגדיל את מושא הדיון שלנו.
מהטענה האחרונה גם נוכל להסיק שבין שתי ההגדרות קיים קשר הדוק, שכן פונקציית הסתברות נקודתית גוררת את קיומה של פונקציית הסתברות כללית.

\end{document}
