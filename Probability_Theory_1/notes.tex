\documentclass[a4paper]{article}

% packages
\usepackage{inputenc, fontspec, amsmath, amsthm, amsfonts, polyglossia, catchfile}
\usepackage[a4paper, margin=50pt, includeheadfoot]{geometry} % set page margins

% style
\AddToHook{cmd/section/before}{\clearpage}	% Add line break before section
\linespread{1.5}
\setcounter{secnumdepth}{0}		% Remove default number tags from sections
\setmainfont{Libertinus Serif}
\setsansfont{Libertinus Sans}
\setmonofont{Libertinus Mono}
\setdefaultlanguage{hebrew}
\setotherlanguage{english}

% operators
\DeclareMathOperator\cis{cis}
\DeclareMathOperator\Sp{Sp}
\DeclareMathOperator\tr{tr}
\DeclareMathOperator\im{Im}
\DeclareMathOperator\diag{diag}
\DeclareMathOperator*\lowlim{\underline{lim}}
\DeclareMathOperator*\uplim{\overline{lim}}

% commands
\renewcommand\qedsymbol{\textbf{משל}}
\newcommand{\NN}[0]{\mathbb{N}}
\newcommand{\ZZ}[0]{\mathbb{Z}}
\newcommand{\QQ}[0]{\mathbb{Q}}
\newcommand{\RR}[0]{\mathbb{R}}
\newcommand{\CC}[0]{\mathbb{C}}
\newcommand{\getenv}[2][] {
  \CatchFileEdef{\temp}{"|kpsewhich --var-value #2"}{\endlinechar=-1}
  \if\relax\detokenize{#1}\relax\temp\else\let#1\temp\fi
}
\newcommand{\explain}[2] {
	\begin{flalign*}
		 && \text{#2} && \text{#1}
	\end{flalign*}
}

% headers
\getenv[\AUTHOR]{AUTHOR}
\author{\AUTHOR}
\date\today

\title{תורת ההסתברות 1 --- סיכום}
\setcounter{secnumdepth}{2}
% chktex-file 9
% chktex-file 17

\hypersetup{}
\begin{document}
\maketitle
\maketitleprint{}

\tableofcontents

\section{שיעור 1 --- 29.10.2024}

\subsection{מבוא הקורס}
% אורי גורביץ' הוא המרצה, הקורס הוא מבוא להסתברות. הוא חצי חירש תתמודד עם זה במקרה הצורך.
נלמד לפי ספר שעוד לא יצא לאור שנכתב על־ידי אורי עצמו, הוא עוד לא סופי ויש בו בעיות ואי־דיוקים, תשיג את הספר הזה.
כן יש הבדל בין הקורס והספר אז לא לסמוך על הסדר שלו גם כשאתה משיג אותו, אבל זו תוספת מאוד נוחה.
יש סימון של כוכביות לחומר מוסף, כדאי לעבור עליו לקראת המבחן כי זה יתן לנו עוד אינטואיציה והעמקה של ההבנה.

נשים לב כי ענף ההסתברות הוא ענף חדש יחסית, שהתפתח הרבה אחרי שאר הענפים הקלאסיים של המתמטיקה, למעשה רק לפני 400 שנה נשאלה על־ידי נזיר במהלך חקר של משחק אקראי השאלה הראשית של העולם הזה, מה ההסתברות של הצלחה במשחק.

נעבור לדבר על פילוסופיה של ההסתברות.
מה המשמעות של הטלת מטבע מבחינת הסתברות?
ישנה הגישה של השכיחות, שמציגה הסתברות כתוצאה במקרה של חזרה על ניסוי כמות גדולה מאוד של פעמים.
יש כמה בעיות בזה, לרבות חוסר היכולת להגדיר במדויק אמירה כזו, הטיות שנובעות מפיזיקה, מטבעות הם לא מאוזנים לדוגמה.
הבעיה הראשית היא שלא לכל בעיה אפשר לפנות בצורה כזאת.
ישנה גישה נוספת, היא הגישה האוביקטיבית או המתמטית, הגישה הזו בעצם היא תרגום בעיה מהמציאות לבעיה מתמטית פורמלית.
לדוגמה נשאל את השאלה מה ההסתברות לקבל 6 בהגרלה של כל המספרים מ־1 עד מיליון.
השיטה ההסתברותית קובעת שאם אני רוצה להוכיח קיום של איזשהו אוביקט, לפעמים אפשר לעשות את זה על־ידי הגרלה של אוביקט כזה והוכחה שיש הסתברות חיובית שהוא יוגרל, וזו הוכחה שהוא קיים.
מה התחזיות שינבעו מתורת ההסתברות? לדוגמה אי־אפשר לחזות הטלת מטבע בודדת, אבל היא כן נותנת הבנה כללית של הטלת 1000 מטבעות, הסתברויות קטנות מספיק יכולות להיות זניחות ובמקרה זה נוכל להתעלם מהן.
לפחות בתחילת הקורס נדבר על תרגום של בעיות מהמציאות לבעיות מתמטיות, זה אומנם חלק פחות ריגורזי, אבל הוא כן חשוב ליצירת קישור בין המציאות לבין החומר הנלמד.

דבר אחרון, ישנה השאלה הפילוסופית של האם באמת יש הסתברות שכן לא בטוח שיש אקראיות בטבע, הגישה לנושא מבחינה פיזיקלית קצת השתנתה בעת האחרונה וקשה לענות על השאלה הזאת.
יש לנו תורות פיזיקליות שהן הסתברותיות בעיקרן, כמו תורת הקוונטים, תורה זו לא סתם הסתברותית, אנחנו לא מנסים לפתור בעיות הסתברותיות אלא ממש משתמשים במודלים סטטיסטיים כדי לתאר מצב בעולם.
לדוגמה נוכל להסיק ככה מסקנה פשוטה שאם מיכל גז נפתח בחדר, יהיה ערבוב של הגז הפנימי ושל אוויר החדר, זוהי מסקנה הסתברותית.
החלק המדהים הוא שתורת הקוונטים מניחה חוסר דטרמניזם כתכונה יסודית ועד כמה שאפשר לראות יש ניסויים שמוכיחים שבאמת יש חוסר ודאות בטבע.
דהינו שברמה העקרונית הפשוטה באמת אין תוצאה ודאית בכלל למצבים כאלה במציאות.

\subsection{מרחבי מדגם ופונקציית הסתברות}
\begin{definition}[מרחב מדגם]
	מרחב מדגם הוא קבוצה לא ריקה שמהווה העולם להסתברות. \\*
	נסמנה $\Omega$.
	איבר במרחב המדגם נסמן ב־$\omega \in \Omega$ על־פי רוב.
\end{definition}
נוכל להגיד שמרחב במדגם הוא הקבוצה של האיברים שעליה אנחנו שואלים בכלל שאלות, זהו הייצוג של האיברים או המצבים שמעניינים אותנו.
בהתאם נראה עכשיו מספר דוגמות שמקשרות בין אובייקטים שאנו דנים בהם בהסתברות ובהגדרה פורמלית של מרחבי מדגם עבורם.
\begin{example}[מרחבי הסתברות שונים]
	נראה מספר דוגמות למצבים כאלה:
	\begin{itemize}
		\item הטלת מטבע תוגדר על־ידי $\Omega = \{ H, T \}$.
		\item הטלת שלושה מטבעות תהיה באופן דומה $\Omega = {\{H, T\}}^3$.
		\item הטלת קוביה היא $\Omega = [6] = \{ 1, \dots, 6 \}$.
		\item הטלת מטבע ואז אם יוצא עץ (H) אז מטילים קוביה ואם יוצא פלי (T) אז מטילים קוביה עם 8 פאות. \\*
			במקרה זה נסמן $\Omega = \{ H1, H2, H3, \dots, H6, T1, \dots, T8 \} = \{H, T \} \times \{1, \dots, 8 \}$ כאשר הכוונה פה היא לזוג סדור $\langle H, 1 \rangle$.
		\item ערבוב חפיסת קלפים, במקרה זה מרחב המדגם שלנו יהיה סימון של הקלפים כרשימה מספרית בלבד, דהינו $\Omega = S_{52}$. \\*
			נוכל גם לסמן במקום את $\Omega = {\{1, \dots, 52 \}}^{52}$, זהו סימון זהה.
	\end{itemize}
\end{example}
בדוגמה זו קל במיוחד לראות שכל איבר בקבוצה מתאר מצב סופי כלשהו, ואנו יכולים לשאול שאלות הסתברותיות מהצורה מה הסיכוי שנקבל $\omega$ מסוים מתוך $\Omega$, זאת ללא התחשבות בבעיה שממנה אנו מגיעים.
נבחן עתה גם דוגמות למקרים שבהם אין לנו מספר סופי של אפשרויות, למעשה מקרים אלה דומים מאוד למקרים שראינו עד כה.
\begin{example}[מרחבי מדגם לא סופיים]
	מטילים מטבע עד שיוצא $H$, אז מרחב המדגם הוא $\Omega = \NN \cup \{ \infty \}$. \\*
	באופן דומה נוכל לבחון מדידת זמן התפרקות חלקיק, היא $\Omega = \RR_+ \cup \{ \infty \}$.
\end{example}
\begin{definition}[פונקציית הסתברות נקודתית]
	יהי מרחב מדגם $\Omega$ ותהי $p : \Omega \to [0, \infty)$ פונקציה כך שמתקיים
	\[
		\sum_{\omega \in \Omega} p(\omega) = 1
	\]
	אז פונקציה זו נקראת \textbf{פונקציית הסתברות}.
\end{definition}
למעשה פונקציית הסתברות היא מה שאנחנו נזהה עם הסתברות במובן הפשוט, פונקציה זו מגדירה לנו לכל סיטואציה ממרחב המדגם מה הסיכוי שנגיע אליה, כך לדוגמה אם נאמר שהטלת מטבע תגיע בחצי מהמקרים לעץ ובחצי השני לפלי,
אז זו היא פונקציית ההסתברות עצמה, פונקציה שמחזירה חצי עבור עץ וחצי עבור פלי, נראה מספר דוגמות.
\begin{example}[פונקציית הסתברות להטלת מטבע]
	נגדיר $\Omega = \{ H, T \}$ ויהי $0 \le \alpha \le 1$, נגדיר $p(H) = \alpha, p(T) = 1 - \alpha$.
\end{example}
\begin{example}[פונקציית הסתברות אינסופית]
	נגדיר $\Omega = \NN \cup \{ \infty \}$ ו־$p(\omega) = \begin{cases}
		2^{-\omega} & \omega \in \NN \\
		0 & \omega = \infty
	\end{cases}$.
	בדוגמה זו נקבל $\sum_{n = 1}^{\infty} 2^{-n} = 1$ ולכן זו אכן פונקציית הסתברות.
\end{example}
נבחין כי הדוגמה האחרונה מתארת לנו התפלגות של דעיכה, זאת אומרת שלדוגמה אם קיים חלקיק עם זמן מחצית חיים של יחידה אחת, פונקציית הסתברות זו תניב לנו את הסיכוי שהוא התפרק לאחר כמות יחידות זמן כלשהי.
\begin{example}
	נגדיר $\Omega = \NN$ ו־$p(\omega) = \frac{1}{\omega(\omega + 1)}$, נבחין כי אכן $\sum_{n = 1}^{\infty} \frac{1}{n(n + 1)} = 1$.
\end{example}
\begin{definition}[תומך]
	התומך של $p$ הוא $\text{Supp}(p) = \{ \omega \in \Omega \mid p(\omega) > 0 \}$. \\*
	נבחין כי התומך הוא למעשה קבוצת האיברים שאפשרי לקבל לפי פונקציית ההסתברות, כל שאר המצבים מקבלים 0, משמעו הוא שאין אפשרות להגיע אליו.
\end{definition}
\begin{remark}
	נבחין כי תמיד $\mathcal{F} \subseteq \mathcal{P}(\Omega)$.
\end{remark}
\begin{definition}[מאורע]
	מאורע הוא תת־קבוצה של מרחב המדגם, קבוצת כל המאורעות תסומן $\mathcal{F}$.
	עבור מאורע $A$ המאורע המשלים מסומן ב־$A^C = \Omega \setminus A$.
\end{definition}
\begin{definition}[פונקציית הסתברות]
	נגדיר עתה פונקציית הסתברות שאיננה נקודתית.
	יהי מרחב מדגם $\Omega$ וקבוצת מאורעות $\mathcal{F}$. \\*
	תהי $\PP : \mathcal{F} \to [0, \infty)$ המקיימת את התכונות הבאות:
	\begin{enumerate}
		\item $\PP(\Omega) = 1$
		\item לכל ${\{A_i\}}_{i = 1}^\infty \subseteq \mathcal{F}$ סדרת מאורעות שונים מתקיים
			\[
				\sum_{i \in \NN} \PP(A_i) = \PP(\bigcup_{i \in \NN} A_i)
			\]
			דהינו, הפונקציה סכימה בתת־קבוצות בנות מניה.
	\end{enumerate}
	לפונקציה כזו נקרא \textbf{פונקציית ההסתברות} על $(\Omega, \mathcal{F})$.
\end{definition}
\begin{proposition}
	תהי $p$ פונקציית הסתברות נקודתית על $\Omega$ אז נגדיר פונקציית הסתברות $\PP_p$ על־ידי
	\[
		\PP_p(A) = \sum_{\omega \in A} p(\omega)
	\]
	אז $\PP_p$ היא פונקציית הסתברות.
\end{proposition}
\begin{proof}
	נוכיח ששתי התכונות של פונקציית הסתברות מתקיימות.
	\[
		\PP_p(A) = \sum_{\omega \in A} p(\omega) \ge 0
	\]
	שכן זהו סכום אי־שלילי מהגדרת $p$,
	בנוסף נקבל מההגדרה של $p$ כי
	\[
		\PP_p(\Omega) = \sum_{\omega \in \Omega} p(\omega) = 1
	\]
	וקיבלנו כי התכונה הראשונה מתקיימת. \\*
	תהי ${\{A\}}_{i = 1}^\infty \in \mathcal{F}$, אז נקבל
	\[
		\sum_{i \in \NN} \PP_p(A_i) = \sum_{i \in \NN} \left( \sum_{\omega \in A_i} p(\omega) \right) = \sum_{\omega \in \bigcup_{i \in \NN} A_i} p(\omega) = \PP_p(\bigcup_{i \in \NN} A_i)
	\]
	ולכן גם התכונה השנייה מתקיימת וקיבלנו כי $\PP_p$ היא אכן פונקציית הסתברות.
\end{proof}
נשים לב כי בעוד פונקציית הסתברות נקודתית מאפשרת לנו לדון בהסתברות של איבר בודד בקבוצות בנות מניה, פונקציית הסתברות למעשה מאפשרת לנו לדון בהסתברות של מאורעות, הם קבוצות של כמה מצבים אפשריים, ובכך להגדיל את מושא הדיון שלנו.
מהטענה האחרונה גם נוכל להסיק שבין שתי ההגדרות קיים קשר הדוק, שכן פונקציית הסתברות נקודתית גוררת את קיומה של פונקציית הסתברות כללית.

\section{תרגול 1 --- 31.10.2024}
המתרגל הוא אמיר, amir.behar@mail.huji.ac.il

\subsection{מרחבי הסתברות סופיים ובני־מניה}
ניזכר בהגדרה למרחב הסתברות, המטרה של הגדרה זו היא לתאר תוצאות אפשריות של מצב נתון.
\begin{definition}[מרחב הסתברות]
	מרחב הסתברות הוא קבוצה $(\Omega, \mathcal{F}, \PP)$ כאשר $\PP : \mathcal{F} \to [0, 1]$, כך שמתקיים
	\begin{enumerate}
		\item חיוביות: $\forall A \in \mathcal{F}, \PP(A) \ge 0$
		\item נרמול: $\PP(\Omega) = 1$
		\item סיגמא־אדיטיביות: $\forall {\{ A_i \}}_{i = 1}^\infty \in \mathcal{F}, (\forall i, j \in \NN, i \ne j \implies A_i \cap A_j = \emptyset) \implies \sum_{i \in I} \PP(A_i) = \PP(\bigcup_{i \in I} A_i )$
	\end{enumerate}
\end{definition}
\begin{exercise}
	יהי $(\Omega, \mathcal{F}, \PP)$ מרחב הסתברות, $A, B \in \mathcal{F}$, הוכיחו
	\[
		\PP(A \cup B) = \PP(A) + \PP(B) - \PP(A \cap B)
	\]
\end{exercise}
\begin{proof}
	נבחין כי $\PP(A) = \PP(A - (A \cap B)) + \PP(A \cap B)$ וגם $\PP(B) = \PP(B - (A \cap B)) + \PP(A \cap B)$. נוכל אם כן לסכום ולקבל
	\[
		\PP(A) + \PP(B) = \PP(A - (A \cap B)) + \PP(A \cap B) + \PP(B - (A \cap B)) + \PP(A \cap B)
		= \PP(A \cup B) + \PP(A \cap B)
	\]
	נבחין כי השוויון האחרון נובע מהזרות של קבוצות אלה.
\end{proof}
לאורך פרק זה נגדיר מעתה שמתקיים $\Omega$ סופית, $\mathcal{F} = 2^\Omega$ ואף נגדיר כי ההסתברות אחידה, דהינו $\forall A \enspace \PP(A) = \frac{|A|}{|\Omega|}$, זה כמובן שקול לטענה
\[
	\forall \omega, \omega' \in \Omega, \PP(\{\omega\}) = \PP(\{\omega'\})
\]
\begin{exercise}
	מטילים קוביה הוגנת, מה ההסתברות שיצא מספר זוגי?
\end{exercise}
\begin{solution}
	נגדיר $\Omega = [6] = \{1, \dots, 6\}$, עם $\PP$ אחידה. \\*
	נרצה לחשב את $A = \{2, 4, 6\}$ ולכן נקבל $\PP(A) = \frac{|A|}{|\Omega|} = \frac{3}{6} = \frac{1}{2}$.
\end{solution}
\begin{exercise}
	מטילים מטבע הוגן שלוש פעמים, מה ההסתברות שיצא עץ בדיוק פעמיים, ומה ההסתברות שיצא עץ לפחות פעמיים?
\end{exercise}
\begin{solution}
	נגדיר $\Omega = \{ TTT, TTP, TPT, PTT, \dots \}$. \\*
	עבור המקרה הראשון נגדיר $A = \{ TTP, TPT, PTT \}$, ולכן נקבל שההסתברות היא $\PP(A) = \frac{3}{8}$. \\*
	במקרה השני נקבל $B = A \cup \{ TTT \}$ ולכן $P(B) = \frac{1}{2}$.
\end{solution}
\begin{exercise}
	מטילים קוביה הוגנת $n$ פעמים.
	\begin{enumerate}
		\item מה ההסתברות שתוצאת ההטלה הראשונה קטנה מ־4?
		\item מה ההסתברות שתוצאת ההטלה הראשונה קטנה שווה מתוצאת ההטלה השנייה?
		\item מה ההסתברות שיצא 1 לפחות פעם אחת?
	\end{enumerate}
\end{exercise}
\begin{solution}
	נגדיר $\Omega = {[6]}^n = \{ (x_1, \dots, x_n) \mid x_i \in [6] \}$.
	\begin{enumerate}
		\item נגדיר $A = \{ (x_1, \dots, x_n) \in \Omega \mid x_1 < 4 \}$ ולכן $\PP(A) = \frac{3 \cdot 6^{n - 1}}{6^n} = \frac{1}{2}$.
		\item נגדיר $B = \{ (x_1, \dots, x_n) \in \Omega \mid x_1 \le x_2 \} = \bigcup_{i = 1}^6 \{ (x_1, i, x_3, \dots, x_n) \in \Omega \mid x_i \le i \}$, ולכן נקבל
			\[
				\PP(B) = \sum \PP(B_i) = \sum \frac{i \cdot 6^{n - 2}}{6^n} = \frac{\sum_{i = 1}^{6} i}{6^2} = \frac{6 \cdot 7}{6^2 \cdot 2} = \frac{7}{12}
			\]
		\item הפעם $C = \{ (x_1, \dots, x_n) \in \Omega \mid \exists i, x_i = 1 \}$, בהתאם $C^C = \{ (x_1, \dots, x_n) \in \Omega \mid \forall i, x_1 \ne 1 \}$. \\*
			לכן נקבל $\PP(C^C) = \frac{5^n}{6^n} \implies \PP(C) = 1 - \frac{5^n}{6^n}$.
	\end{enumerate}
\end{solution}
\begin{exercise}
	חמישה אנשים בריאים וחמישה אנשים חולי שפעת עומדים בשורה. מה ההסתברות שחולי השפעת נמצאים משמאל לאנשים הבריאים?
\end{exercise}
\begin{solution}
	נגדיר $\Omega$ ככל הסידורים של $0, 1$ כשיש חמישה מכל סוג.
	לכן נקבל $|\Omega| = \binom{10}{5}$, שכן $\Omega = \{ X \subset [10] \mid |X| = 5 \}$. \\*
	המאורע הפעם הוא $A = \{ \{ 1, 2, 3, 4, 5 \} \}$ ובהתאם $\PP(A) = \frac{5! 5!}{10!}$.

	נוכל גם להגדיר $\Omega = S_{10}$ כאשר חמשת המספרים הראשונים מייצגים בריאים וחמשת האחרונים מייצגים חולים. \\*
	במקרה זה נקבל $A = \{ \pi \in \Omega \mid \pi(\{1, 2, 3, 4, 5\}) \subseteq \{1, 2, 3, 4, 5\} \}$ ולכן $|A| = 5! 5! $ וכך נקבל $\PP(A) = \frac{5! 5!}{10!}$.
\end{solution}

\section{שיעור 2 --- 31.10.2024}
\subsection{השלמה לטורים דו־מימדיים}
נגדיר הגדרה שדרושה לצורך ההרצאה הקודמת כדי להיות מסוגלים לדון בסכומים אינסופיים בני־מניה.
\begin{definition}
	אם ${\{a_i\}}_{i \in I}$ ו־$a_i \ge 0$ לכל $i \in I$ אז נגדיר
	\[
		\sum_{i \in I} a_i = \sup \left\{ \sum_{i \in J} \mid J \subseteq I, J \text{ is finite} \right\}
	\]
\end{definition}

\subsection{תכונות של פונקציות הסתברות}
נעבור עתה לבחון פונקציות הסתברות ואת תכונותיהן, נתחיל מתרגיל שיוצק תוכן לתומך של פונקציית הסתברות:
\begin{exercise}
	הוכיחו כי אם $\sum_{i \in I} a_i < \infty$ ו־$a_i \ge 0$ לכל $i \in I$ אז $|\{ i \in I \mid a_i < 0 \}| \le \aleph_0$.
	במילים אחרות הוכיחו כי התומך של $a$ הוא בן־מניה.
\end{exercise}
בשיעור הקודם ראינו את ההגדרה והטענה הבאות:
\begin{definition}
	בהינתן פונקציית הסתברות נקודתית $p$ נגדיר
	\[
		\PP_p(A) = \sum_{\omega \in A} p(\omega)
	\]
\end{definition}
\begin{proposition}
	$\PP_p$ היא פונקציית הסתברות.
\end{proposition}
טענה זו בעצם יוצרת קשר בין פונקציות הסתברות לפונקציות הסתברות נקודתיות, ומאפשרת לנו לחקור את פונקציות ההסתברות לעומק באופן פשוט הרבה יותר. נשתמש עתה בכלי זה.
\begin{definition}[מרחב הסתברות בדיד]
	אם $\PP$ פונקציית הסתברות כך שקיימת פונקציית הסתברות נקודתית $p$ כך ש$\PP = \PP_p$, אז נאמר ש־$\PP$ היא בדידה ו־$(\Omega, \mathcal{F}, \PP)$ \textbf{מרחב הסתברות בדיד}.
\end{definition}
\begin{proposition}
	יש פונקציות הסתברות שאינן בדידות.
	בפרט, עבור מדגם ההסתברות $\Omega = [0, 1]$ קיימת פונקציית הסתברות $\PP$ המקיימת
	\[
		\forall a, b \in \RR, 0 \le a \le b \le 1 \implies \PP([a, b]) = b - a
	\]
\end{proposition}
\begin{example}
	ידוע כי $\sum_{n \in \NN} \frac{1}{n^2} = \frac{\pi^2}{6} < \infty$ ולכן נוכל להגדיר $\Omega = \NN$ ו־$p(n) = \frac{1}{\frac{\pi^2}{6} n^2}$, הגדרה זו תניב ש־$\sum_{n \in \NN} p(n) = 1$ ולכן זו פונקציית הסתברות.
	נחשב את $\PP_p(A)$ עבור $A = 2\NN$:
	\[
		\PP_p(A) = \sum_{n \in A} p(n) = \sum_{k \in \NN} p(2k) = \frac{1}{\frac{\pi^2}{6} {(2k)}^2} = \frac{6}{\pi^2} \frac{1}{4} \sum_{k \in \NN} \frac{1}{k^2} = \frac{1}{4}
	\]
	נסביר, הגדרנו פונקציית הסתברות של דעיכה, דהינו שככל שהמספר שאנו מבקשים גדול יותר כך הוא פחות סביר באופן מעריכי (לדוגמה זמן מחצית חיים), ואז שאלנו כמה סביר המאורע שבו נקבל מספר זוגי.
\end{example}
\begin{theorem}[תכונות פונקציית הסתברות]
	$\PP$ פונקציית הסתברות על $(\Omega, \mathcal{F})$, אז
	\begin{enumerate}
		\item $\PP(\emptyset) = 0$
		\item אם $I$ קבוצה סופית ו־${\{A_i\}}_{i \in I}$ מאורעות זרים בזוגות, אז $\PP(\bigcup_{i \in I} A_i) = \sum_{i \in I} \PP(A_i)$
		\item אם $A \subseteq B$ מאורעות אז $\PP(A) \le \PP(B)$
		\item $\PP(A) \le 1$ לכל מאורע $A$
		\item לכל מאורע $A$ מתקיים $\PP(A^C) = 1 - \PP(A)$
	\end{enumerate}
\end{theorem}
\begin{proof}
	נוכיח את התכונות
	\begin{enumerate}
		\item נראה כי $\PP(\emptyset) = \sum_{i = 1}^\infty \PP(\emptyset)$ שכן כל איחוד של קבוצות ריקות הוא זר, לכן אילו $\PP(\emptyset) \ne 0$ נקבל ישר סתירה, נסיק כי $\PP(\emptyset) = 0$ בלבד.
		\item נגדיר $A_i = \emptyset$ לכל $i > n$ ונשתמש בסיגמא־אדיטיביות ונקבל
			\[
				\PP(\bigcup_{i \in I} A_i)
				= \PP(\bigcup_{i \in \NN} A_i)
				= \sum_{i \in \NN} \PP(A_i)
				= \sum_{i \in I} \PP(A_i)
			\]
		\item נשתמש בתכונה 2 על $B, B \setminus A$, אלו הן קבוצות זרות כמובן, אם נגדיר $D = A \cup (B \setminus A)$ נקבל $\PP(D) = \PP(A) + \PP(B \setminus A) \ge \PP(A)$.
		\item נובע ישירות מתכונה 3 ומ־$A \subseteq \Omega$.
		\item ניזכר כי $A^C = \Omega \setminus A$ ולכן $\Omega = A \cup A^C$ ונקבל $1 = \PP(\Omega) = \PP(A) + \PP(A^C)$.
	\end{enumerate}
\end{proof}
נעבור עתה לאפיון של פונקציות הסתברות בדידות, נבין מתי הן כאלה ומתי לא.
\begin{theorem}[תנאים שקולים לפונקציית הסתברות בדידה]
	אם $(\Omega, \mathcal{F}, \PP)$ מרחב הסתברות, התנאים הבאים שקולים:
	\begin{enumerate}
		\item $\PP$ היא פונקציית הסתברות בדידה
		\item $\PP$ נתמכת על קבוצות בנות־מניה, כלומר קיימת קבוצה $A \in \mathcal{F}$ בת־מניה כך ש־$\PP(A) = 1$
		\item $\sum_{\omega \in \Omega} \PP(\{\omega\}) = 1$
		\item לכל מאורע $A \in \mathcal{F}$ מתקיים $\PP(A) = \sum_{\omega \in A} \PP(\{\omega\})$
	\end{enumerate}
\end{theorem}
\begin{proof}
	$1 \implies 2$:
	נניח ש־$\PP = \PP_p$ עבור $p : \Omega \to [0, \infty)$ פונקציית הסתברות נקודתית.
	נסתכל על $\text{Supp}(p) = \{ \omega \in \Omega \mid p(\omega) > 0 \}$, לפי הגדרת הסכום והתרגיל נובע ש־$A = \text{Supp}(p)$ בת־מניה.
	נקבל
	\[
		\PP(A) = \sum_{\omega \in A} p(\omega) = \sum_{\omega \in \Omega} p(\omega) = \PP(\Omega) = 1
	\]

	$2 \implies 4$:
	נניח ש־$\PP(S) = 1$ עבור $S$ בת־מניה. לכן $\PP(S^C) = 0$.
	נראה כי $A$ הוא איחוד זר $A = (A \cap S) \cup (A \cap S^C)$ ולכן נקבל
	\[
		\PP(A) = \PP(A \cap S) + \PP(A \cap S^C) = \PP(A \cap S) + 0 = \sum_{\omega \in A \cap S} \PP(\{\omega\}) = \sum_{\omega \in A} \PP(\{\omega\})
	\]

	$4 \implies 3$:
	אם נבחר $A = \Omega$ נקבל את טענה 3.

	$3 \implies 1$:
	נגדיר $p : \Omega \to [0, \infty)$ על־ידי $p(\omega) = \PP(\{ \omega \})$, נקבל $\sum_{\omega \in \Omega} p(\omega) = 1$ ולכן $p$ היא פונקציית הסתברות נקודתית.
	מהתרגיל והגדרת הסכום נובע ש־$S = \text{Supp}(p)$ היא בת־מניה ומתקיים $\PP(S^C) = 0$, אז לכל $A \in \mathcal{F}$ מתקיים
	\[
		\PP(A) = \PP(A \cap S) + \PP(A \cap S^C) = \PP(A \cap S) = \sum_{\omega \in A \cap S} \PP(\{ \omega \}) = \sum_{\omega \in A} \PP(\{\omega\}) = \sum_{\omega \in A} p(\omega) = \PP_p(A)
	\]
\end{proof}

\subsection{פרדוקס יום ההולדת}
פרדוקס יום ההולדת הוא פרדוקס מוכר הגורס כי גם בקבוצות קטנות יחסית של אנשים, הסיכוי שלשני אנשים שונים יהיה תאריך יום הולדת זהה הוא גבוה במידה משונה.
הפרדוקס נקרא כך שכן לכאורה אין קשר בין מספר הימים בשנה לבין הסיכוי הכל־כך גבוה שמצב זה יקרה, נבחן עתה את הפרדוקס בהיבט הסתברותי.

נניח שכל תאריכי יום ההולדת הם סבירים באותה מידה ונבחן את הפרדוקס.
נגדיר $\Omega = {[365]}^k$ עבור $k$ מספר האנשים בקבוצה נתונה כלשהי.
$p(\omega) = \frac{1}{{365}^k}$ לכל $\omega \in \Omega$.
נקבל $\PP(A) = \PP_p(A) = \frac{|A|}{365^k}$.
נרצה לחשב את $A$ כמאורע שיש לפחות שני אנשים שיש להם יום הולדת באותו יום, דהינו שיש שני ערכים זהים ברשימת המספרים, נגדיר $A = \{ \omega \in \Omega \mid \exists 1 \le i \ne j \le k, \omega_i = \omega_j \}$.
בשל המורכבות נבחן את המשלים $A^C$, נקבל $|A^C| = 365 \cdot 364 \cdots (365 - (k - 1)) = \frac{365!}{(365 - (i - 1))!}$.
נציב ונחשב:
\[
	\PP(A^C) = \frac{|A^C|}{365^k} = \prod_{i = 1}^k \frac{365 - (i - 1)}{365} = \prod_{i = 1}^k (1 - \frac{i - 1}{365})
\]
מהנוסחה שקיבלנו נראה שמההצבה $k = 23$ נקבל שההסתברות היא בערך $\frac{1}{2}$, דהינו בקבוצה של 23 אנשים יש סבירות של חצי שלפחות שניים יחגגו יום הולדת באתו יום.

\section{שיעור 3 --- 5.11.2024}

\subsection{מכפלת מרחבי הסתברות בדידים}
ניזכר תחילה במרחבי הסתברות אחידים
\begin{definition}
	מרחב הסתברות אחיד הוא $(\Omega, \mathcal{F}, \PP_p)$ המקיים $p(\omega_1) = p(\omega_2)$ לכל $\omega_1, \omega_2 \in \Omega$.
\end{definition}
\begin{conclusion}
	$\PP_p(A) = \frac{|A|}{|\Omega|}$
\end{conclusion}
נבחין כי במקרים מסוימים ההסתברות שלנו מורכבת משני מאורעות בלתי תלויים, במקרים אלה נרצה להגדיר מכפלה של מרחבי ההסתברות.
\begin{definition}[מרחב מכפלת הסתברויות]
	אם $(\Omega_1, \mathcal{F}_1, \PP_{p_1})$ ו־$(\Omega_2, \mathcal{F}_2, \PP_{p_2})$ מרחבי הסתברות בדידים
	נגדיר $q : \Omega_1 \times \Omega_2 \to [0, \infty)$ על־ידי $q(\omega_1, \omega_2) = p(\omega_1) \cdot p(\omega_2)$.
\end{definition}
\begin{proposition}
	$q$ פונקציית הסתברות נקודתית.
\end{proposition}
\begin{proof}
	נשתמש ישירות בהגדרה ונחשב
	\[
		\sum_{(\omega_1, \omega_2) \in \Omega_1 \times \Omega_2} q(\omega_1, \omega_2)
		= \sum_{\omega_1 \in \Omega_1, \omega_2 \in \Omega_2} q(\omega_1, \omega_2)
		= \sum_{\omega_1 \in \Omega_1} \left( \sum_{\omega_2 \in \Omega_2} p_1(\omega_1) p_2(\omega_2) \right)
		= \sum_{\omega_1 \in \Omega_1} p_1(\omega_1)
		= 1
	\]
\end{proof}
עתה כשהוכחנו טענה זו, יש לנו הצדקה אמיתית להגדיר את $(\Omega_1 \times \Omega_2, \mathcal{F}_{1,2}, \PP_q)$ כמרחב הסתברות, ונקרא לו מרחב מכפלה.
\begin{proposition}
	אם $(\Omega_1, \mathcal{F}_1, \PP_{p_1})$ ו־$(\Omega_2, \mathcal{F}_2, \PP_{p_2})$ מרחבי הסתברות אחידים,
	אז מרחב המכפלה $(\Omega_1 \times \Omega_2, \mathcal{F}_{1,2}, \PP_q)$ אחיד אף הוא.
\end{proposition}
\begin{proof}
	\[
		q(\omega_1, \omega_2) = p_1(\omega_1) p_2(\omega_2)
		= \frac{1}{|\Omega_1|} \cdot \frac{1}{|\Omega_2|}
		= \frac{1}{|\Omega_1 \times \Omega_2|}
	\]
\end{proof}
\begin{definition}
	במרחב מכפלה המאורעות מהצורה $A \times \Omega_2$ או $\Omega_1 \times A$ נקראים שוליים. \\*
	מאורע מהצורה $A \times B$ נקרא מאורע מכפלה.
\end{definition}
\begin{proposition}
	במרחב מכפלה $\PP_q(A \times B) = \PP_{p_1}(A) \cdot \PP_{p_2}(B)$.
	בפרט $\PP_q(A \times \Omega_2) = \PP_{p_1}(A)$.
\end{proposition}
\begin{proof}
	\[
		\sum_{(\omega_1, \omega_2) \in A \times B} q(\omega_1, \omega_2)
		= \sum_{\omega_1 \in A, \omega_2 \in B} q(\omega_1, \omega_2)
		= \sum_{\omega_1 \in A} \left( \sum_{\omega_2 \in B} p_1(\omega_1) p_2(\omega_2) \right)
		= \sum_{\omega_1 \in A} p_1(\omega_1) \PP_{p_2}(B)
		= \PP_{p_1}(A) \PP_{p_2}(B)
	\]
\end{proof}
\begin{example}
	בהינתן $n$ הטלות מטבע כלשהו, מה ההסתברות שיצאו $k$ עצים?

	עבור ההטלה הראשונה, $\Omega_1 = \{0, 1\}$. עוד נגדיר $p(1) = \alpha, p(0) = 1 - \alpha$ עבור $0 \le \alpha \le 1$ כלשהו. \\*
	בהתאם נקבל $\Omega = {\{0, 1\}}^n$, וכן
	\[
		q(\omega_1, \dots, \omega_n) = \prod_{i = 1}^n p(\omega_i)
		= \prod_{i = 1}^n \alpha^{\omega_i} \cdot {(1 - \alpha)}^{1 - {\omega_i}}
		= \alpha^{\sum_{i = 1}^n \omega_i} {(1 - \alpha)}^{n - \sum_{i = 1}^n \omega_i}
	\]

	נבחין כי היינו יכולים לתאר את המקרה הזה ממש על־ידי $q(\omega) = \alpha^\omega \cdot {(1 - \alpha)}^{1 - \omega}$.

	נעבור עתה לבחינת המאורע
	\[
		A = \{ (\omega_1, \dots, \omega_n) \in \Omega \mid \sum_{i = 1}^{n} \omega_i = k \}
	\]
	נקבל מהביטוי שמצאנו כי
	\[
		\PP_q(A)
		= \sum_{(\omega_1, \dots, \omega_n) \in A} q(\omega_1, \dots, \omega_n)
		\sum_{\sum_{i = 1}^n \omega_i = k} \alpha^{\sum_{i = 1}^n \omega_i} {(1 - \alpha)}^{n - \sum_{i = 1}^n \omega_i}
		= |A| \alpha^k {(1 - \alpha)}^{n - k}
		= \binom{n}{k} \alpha^k {(1 - \alpha)}^{n - k}
	\]
\end{example}
\begin{example}
	נבחן עתה את המקרה של הטלות הוגנות ובחינת המקרה שחצי מההטלות לפחות יצאו עץ,
	זאת־אומרת שנבחן את הדוגמה הקודמת כאשר $n = 2m, k = m$, ו־$\alpha = \frac{1}{2}$.
	מנוסחת סטרלינג שאנחנו לא מכירים $m! \simeq \sqrt{2\pi m} {(\frac{m}{e})}^m$ ואז נוכל להסיק
	\[
		\PP_q(A)
		= \binom{2m}{m} \frac{1}{2^m}
		\simeq \frac{\sqrt{4\pi m} {(\frac{2m}{e})}^{2m}}{{(\sqrt{2\pi m} {(\frac{k}{e})}^m)}^2 2^{2m}}
		= \frac{\sqrt{4\pi m}}{2\pi m}
		= \frac{1}{\sqrt{\pi m}}
	\]
\end{example}

\subsection{ניסויים דו־שלביים}
נניח $(\Omega_1, \mathcal{F}_1, \PP_{p_1})$ מרחב הסתברות בדידה עבור הניסוי הראשון, ונניח שיש מרחב הסתברות בדידה עבור הניסוי השני כך שלכל תוצאה בניסוי הראשון, פונקציית ההסתברות תשתנה בהתאם בניסוי השני.
לכל $\omega_1 \in \Omega_1$ יש פונקציית הסתברות נקודתית $p_{\omega_1} : \Omega_2 \to [0, \infty)$.
נגדיר את מרחב הניסוי הדו־שלבי $(\Omega_1 \times \Omega_2, \mathcal{F}_{1, 2}, \PP_q)$,
כאשר $q(\omega_1, \omega_2) = p_1(\omega_1) \cdot p_{\omega_1}(\omega_2)$.
\begin{proposition}
	$\PP_q$ פונקציית הסתברות.
\end{proposition}
\begin{proof}
	\[
		\sum_{(\omega_1, \omega_2) \in \Omega_1 \times \Omega_2} q(\omega_1, \omega_2)
		= \sum_{\omega_1 \in \Omega_1} \left( \sum_{\omega_2 \in \Omega_2} p_1(\omega_1) p_{\omega_1}(\omega_2) \right)
		= \sum_{\omega_1 \in \Omega_1} p_1(\omega_1) \left( \sum_{\omega_2 \in \Omega_2} p_{\omega_1}(\omega_2) \right)
		= \sum_{\omega_1 \in \Omega_1} p_1(\omega_1)
		= 1
	\]
\end{proof}
\begin{example}
	$\Omega_1 = \{H, T\}$ ו־$\Omega_2 = \{1, \dots, 8\}$, נגדיר $p_1(H) = p_1(T) = \frac{1}{2}$.
	עוד נגדיר
	\[
		p_H(\omega_2) = \begin{cases}
			\frac{1}{6} & 1 \le \omega_2 \le 6 \\
			0 & \text{else}
		\end{cases},
		\qquad
		p_T(\omega_2) = \frac{1}{8}
	\]
	מהגדרה זו נקבל
	\[
		q(\omega_1, \omega_2) = \begin{cases}
			\frac{1}{12} & \omega_1 = H, \omega_2 \in [6] \\
			0 & \omega_1 = H, \omega_2 \in \{7, 8\} \\
			\frac{1}{16} & \omega_1 = T, \omega_2 \in [8]
		\end{cases}
	\]
\end{example}
\begin{theorem}[חסם האיחוד]
	אם $A, B$ מאורעות אז $\PP(A \cup B) \le \PP(A) + \PP(B)$.
\end{theorem}
\begin{proof}
	\[
		\PP(A \cup B)
		= \PP(A \uplus (B \setminus A))
		= \PP(A) + \PP(B \setminus A)
		\le \PP(A) + \PP(B)
	\]
\end{proof}
נוכל להשתמש בחסם האיחוד כדי להוכיח גרסה כללית יותר של המשפט:
\begin{theorem}[אי־שוויון בול]
	אם $A_1, \dots, A_k$ מאורעות, אז $\PP(\bigcup_{i = 1}^k A_i) \le \sum_{i = 1}^k \PP(A_i)$.
\end{theorem}
\begin{example}
	נחזור לבחון את פרדוקס יום ההולדת, הפעם נבחן גרסה כללית יותר של הרעיון.
	נגדיר $\Omega = {[m]}^k$ עם הסתברות אחידה.
	נגדיר גם $A = \{ \omega \in \Omega \mid \exists 1 \le i < j \le k, \omega_i = \omega_j \}$.
	אנו רוצים את ההסתברות $\PP(A) = \frac{|A|}{|\Omega|}$, אז נבחן את המשלים
	\[
		A^C = \{ \omega \in \Omega \mid \forall 1 \le i, j \le k, i \ne j \implies \omega_i \ne \omega_j \}
	\]
	נחשב
	\[
		|A^C| = m (m - 1) \cdots (m - (k - 1))
	\]
	בהתאם
	\[
		\PP(A^C)
		= \frac{\prod_{i = 0}^{k - 1} (m - i)}{m^k}
		= \prod_{i = 0}^{k - 1} \frac{m - i}{m^k}
		= \prod_{i = 0}^{k - 1} (1 - \frac{i}{m})
	\]
	נזכור ש־$\forall x \in \RR, 1 + x \le e^x$, ונוכל לקבל
	\[
		\prod_{i = 0}^{k - 1} (1 - \frac{i}{m})
		\le \prod_{i = 0}^{k - 1} e^{-\frac{i}{m}}
		= \exp(- \frac{1}{m} \sum_{i = 0}^{k - 1} i)
		= e^{- \frac{k(k - 1)}{2m}}
	\]
	כאשר $k$ גדול ביחס ל־$\sqrt{2m}$ מקבלים חסם קרוב ל־$0$.

נגדיר הפעם $A = \bigcup_{\substack{i \ne j \\ i, j \in [k]}} A_{ij}$ עבור $A_{ij} = \{ \omega \in \Omega \mid \omega_i = \omega_j \}$.
	וגם
	\[
		i \ne j \implies \PP(A_{ij}) = \frac{|A_{ij}|}{m^k} = \frac{m \cdot m^{k - 2}}{m^k} = \frac{1}{m}
	\]
	ועתה
	\[
		\PP(A)
		\le \sum_{\substack{i \ne j \\ i, j \in [k]}} \PP(A_{ij})
		= \sum_{\substack{i \ne j \\ i, j \in [k]}} \frac{1}{m}
		= \binom{k}{2} \frac{1}{m}
		= \frac{k (k - 1)}{2m}
	\]
	לכן אם $k$ קטן ביחס ל־$\sqrt{2m}$ אז ההסתברות ליום־הולדת משותף קטנה.
\end{example}

\end{document}
