\documentclass[a4paper]{article}

% packages
\usepackage{inputenc, amsmath, amsthm, thmtools, amsfonts, amssymb, luacode, catchfile, tikzducks, hyperref}
\usepackage[a4paper, margin=50pt, includeheadfoot]{geometry} % set page margins
\usepackage[shortlabels]{enumitem}
\usepackage[skip=3pt, indent=0pt]{parskip}

% language
\usepackage[bidi=basic, layout=tabular, provide=*]{babel}
\babelprovide[main, import]{hebrew}
\babelprovide{rl}
\babelfont{rm}{Libertinus Serif}
\babelfont{sf}{Libertinus Sans}
\babelfont{tt}{Libertinus Mono}

% style
\AddToHook{cmd/section/before}{\clearpage}	% Add line break before section
\linespread{1.3}
\setcounter{secnumdepth}{0}		% Remove default number tags from sections, this won't do well with theorems
\AtBeginDocument{\setlength{\belowdisplayskip}{3pt}}
\AtBeginDocument{\setlength{\abovedisplayskip}{3pt}}

% operators
\DeclareMathOperator\cis{cis}
\DeclareMathOperator\Sp{Sp}
\DeclareMathOperator\tr{tr}
\DeclareMathOperator\im{Im}
\DeclareMathOperator\re{Re}
\DeclareMathOperator\diag{diag}
\DeclareMathOperator*\lowlim{\underline{lim}}
\DeclareMathOperator*\uplim{\overline{lim}}
\DeclareMathOperator\rng{rng}
\DeclareMathOperator\Sym{Sym}
\DeclareMathOperator\Arg{Arg}
\DeclareMathOperator\Log{Log}
\DeclareMathOperator\dom{dom}

% commands
%\renewcommand\qedsymbol{\textbf{מש''ל}}
%\renewcommand\qedsymbol{\fbox{\emoji{lizard}}}
\newcommand{\NN}[0]{\mathbb{N}}
\newcommand{\ZZ}[0]{\mathbb{Z}}
\newcommand{\QQ}[0]{\mathbb{Q}}
\newcommand{\RR}[0]{\mathbb{R}}
\newcommand{\CC}[0]{\mathbb{C}}
\newcommand{\FF}[0]{\mathbb{F}}
\newcommand{\PP}[0]{\mathbb{P}}
\newcommand{\TT}[0]{\mathbb{T}}
\newcommand{\acts}[0]{\circlearrowright}
\newcommand{\explain}[2] {
	\begin{flalign*}
		 && \text{#2} && \text{#1}
	\end{flalign*}
}
\newcommand{\maketitleprint}[0]{ \begin{center}
	\begin{tikzpicture}[scale=3]
		\duck[graduate=gray!20!black, tassel=red!70!black]
	\end{tikzpicture}	
\end{center}
}

% theorem commands
\newtheoremstyle{c_remark}
	{}	% Space above
	{}	% Space below
	{}% Body font
	{}	% Indent amount
	{\bfseries}	% Theorem head font
	{}	% Punctuation after theorem head
	{.5em}	% Space after theorem head
	{\thmname{#1}\thmnumber{ #2}\thmnote{ \normalfont{\text{(#3)}}}}	% head content
\newtheoremstyle{c_definition}
	{3pt}	% Space above
	{3pt}	% Space below
	{}% Body font
	{}	% Indent amount
	{\bfseries}	% Theorem head font
	{}	% Punctuation after theorem head
	{.5em}	% Space after theorem head
	{\thmname{#1}\thmnumber{ #2}\thmnote{ \normalfont{\text{(#3)}}}}	% head content
\newtheoremstyle{c_plain}
	{3pt}	% Space above
	{3pt}	% Space below
	{\itshape}% Body font
	{}	% Indent amount
	{\bfseries}	% Theorem head font
	{}	% Punctuation after theorem head
	{.5em}	% Space after theorem head
	{\thmname{#1}\thmnumber{ #2}\thmnote{ \text{(#3)}}}	% head content

\theoremstyle{c_plain}
\newtheorem{theorem}{משפט}[section]
\newtheorem{lemma}[theorem]{למה}
\newtheorem{proposition}[theorem]{טענה}
\newtheorem*{proposition*}{טענה}
%\newtheorem{corollary}[theorem]{אין חלופה עברית}

\theoremstyle{c_definition}
\newtheorem{definition}[theorem]{הגדרה}
\newtheorem*{definition*}{הגדרה}
\newtheorem{example}{דוגמה}[section]
\newtheorem{exercise}{תרגיל}[section]

\theoremstyle{c_remark}
\newtheorem*{remark}{הערה}
\newtheorem*{solution}{פתרון}
\newtheorem{conclusion}[theorem]{מסקנה}
\newtheorem{notation}[theorem]{סימון}

% Questions related commands
\newcounter{question}
\setcounter{question}{1}
\newcounter{sub_question}
\setcounter{sub_question}{1}

\newcommand{\question}[1][0]{
	\ifthenelse{#1 = 0}{}{\setcounter{question}{#1}}
	\subsection{שאלה \arabic{question}}
	\addtocounter{question}{1}
	\setcounter{sub_question}{1}
}

\newcommand{\subquestion}[1][0]{
	\ifthenelse{#1 = 0}{}{\setcounter{sub_question}{#1}}
	\subsubsection{סעיף \localecounter{letters.gershayim}{sub_question}}
	\addtocounter{sub_question}{1}
}

% import lua and start of document
\directlua{common = require ('../common')}

\GetEnv{AUTHOR}

% headers
\author{\AUTHOR}
\date\today

\title{פתרון מטלה 12 --- תורת ההסתברות (1), 80420}

\DeclareMathOperator{\Supp}{Supp}

\begin{document}
\maketitle
\maketitleprint{}

\question{}
נראה שאם $X \sim Exp(\lambda), Y \sim Exp(\mu)$ משתנים מקריים בלתי־תלויים אז גם $Z = \min\{ X, Y \}$ מתפלג מעריכית.
\begin{proof}
	נבחן את ההתפלגות המצטברת של $Z$,
	\begin{align*}
		\PP(Z \le t)
		& = 1 - \PP(Z > t) \\
		& = 1 - \PP(X > t, Y > t) \\
		& = 1 - \PP(X > t) \PP(Y > t) \\
		& = 1 - (1 - \PP(X \le t))(1 - \PP(Y \le t)) \\
		& = F_X(t) + F_Y(t) - F_X(t) F_Y(t)
	\end{align*}
	אבל נתונה התפלגות מעריכית ולכן $F_X(t) = 1 - e^{-\lambda t}$ ו־$F_Y(t) = 1 - e^{-\mu t}$ ולכן
	\[
		\PP(Z \le t)
		= 1 - e^{-\lambda t} + 1 - e^{-\mu t} - 1 + e^{-\lambda t} + e^{-\mu t} - e^{-(\lambda + \mu)t}
		= 1 - e^{-(\lambda + \mu)t}
	\]
	ובהתאם $F_Z'(t) = 1 - (\lambda + \mu)e^{-(\lambda + \mu)t}$ ולכן $Z \sim Exp(\lambda + \mu)$.
\end{proof}

\question{}
שומר מפטרל לאורך גדר באורך $l$ ומיקומו מתפלג אחיד לאורכה.
גנב מגיע לנקודה בגדר, מיקומה מתפלג אחיד ובלתי־תלוי במיקום השומר.
נחשב את פונקציית הצפיפות של המרחק ביניהם.
\begin{solution}
	נגדיר $X \sim Unif([0, l])$ את מיקום השומר ו־$Y \sim Unif([0, l])$ את מיקום הגנב.
	נגדיר גם $Z = |X - Y|$ המרחק שלהם.
	נבחין כי $\{ Z \le t \} = \{ |X - Y| \le t \} = \{ -t \le X - Y \le t \} = \{X - Y \le t\} \setminus \{X - Y \le -t\}$ כאשר המעבר האחרון נובע מרציפות בהחלט.
	לכן מספיק שנחשב את $\PP(X - Y \le t) = \PP(X \le Y + t)$,
	כאשר $t$ חיובי מתקיים,
	\begin{align*}
		\iint_{x \le y + t} f_{X, Y}(x, y) 1_{[0, l]}(x) 1_{[0, l]}(y) \ dx\ dy
		& = \iint_{x \le y + t} \frac{1}{l^2} 1_{[0, l]}(x) 1_{[0, l]}(y) \ dx\ dy \\
		& = \int_0^l \int_0^{\min\{y + t, l\}} \frac{1}{l^2} \ dx\ dy \\
		& = \int_0^{l - t} \int_0^{y + t} \frac{1}{l^2} \ dx\ dy + \int_{l - t}^l \int_0^{l} \frac{1}{l^2} \ dx\ dy \\
		& = \frac{1}{l^2} \left( \int_0^{l - t} y + t\ dy + \int_{l - t}^l l\ dy \right) \\
		& = \frac{1}{l^2} \left(\frac{1}{2} {(l - t)}^2 + (l - t) t + l t \right) \\
		& = \frac{1}{l^2} (\frac{1}{2} l^2 - \frac{1}{2} t^2 + lt) \\
		& = \frac{1}{2} - \frac{t^2}{2l^2} + \frac{t}{l}
	\end{align*}
	וכאשר נבחן את $-t$ נובע באופן דומה,
	\begin{align*}
		\iint_{x \le y - t} \frac{1}{l^2} 1_{[0, l]}(x) 1_{[0, l]}(y) \ dx\ dy
		& = \int_t^{l} \int_0^{y - t} \frac{1}{l^2} \ dx\ dy + \int_{0}^{l - t} \int_0^{0} \frac{1}{l^2} \ dx\ dy \\
		& = \int_t^{l} \int_0^{y - t} \frac{1}{l^2} \ dx\ dy \\
		& = \frac{1}{l^2} \left( \int_t^{l} y - t\ dy \right) \\
		& = \frac{1}{l^2} \left(\frac{1}{2} l^2 - \frac{1}{2} t^2 - l t + t^2 \right) \\
		& = \frac{1}{2} + \frac{t^2}{2 l^2} - \frac{t}{l}
	\end{align*}
	ונסיק $\PP(|X - Y| \le t) = \PP(X - Y \le t) - \PP(X - Y \le -t) = \frac{t^2}{l^2}$.
	כלומר $F_Z(t) = \frac{t^2}{l^2}$ ולכן $f_Z(t) = \frac{2t}{l^2}$.
\end{solution}

\question{}
יהיו $X, Y$ משתנים מקריים בעלי צפיפות משותפת
\[
	f_{X, Y}(x, y)
	= c e^{-2y - x} 1_{(0, \infty)}(x) 1_{(0, \infty)}(y)
\]

\subquestion{}
נמצא את $c$.
\begin{solution}
	ידוע כי $\int_{\RR^2} f_{X, Y} = 1$, וכן,
	\[
		\int_{-\infty}^{\infty} \int_{-\infty}^{\infty} f_{X, Y}(x, y) \ dx\ dy
		\int_{0}^{\infty} \int_{0}^{\infty} c e^{-2y - x} \ dx\ dy
		= c \left( \int_{0}^{\infty} e^{-2y} \ dy \right) \left( \int_{0}^{\infty} e^{-x} \ dx \right)
		= c (0 - (-\frac{1}{2})) (0 - (-1))
	\]
	ולכן בהכרח $c = 2$.
\end{solution}

\subquestion{}
נבדוק האם $X, Y$ משתנים מקריים בלתי־תלויים.
\begin{solution}
	נרצה לבדוק האם $f_{X, Y}(x, y) = f_X(x) f_Y(y)$,
	לשם כך נשתמש בנוסחה לחישוב צפיפות מצפיפות משותפת,
	\[
		f_X(s)
		= \int_{-\infty}^{\infty} f_{X, Y}(s, t) \ dt
		= 2 e^{-s} \int_{0}^{\infty} e^{-2t} \ dt
		= 2 e^{-s} \left( 0 + \frac{1}{2} \right) 
		= e^{-s}
	\]
	וכן,
	\[
		f_Y(t)
		= \int_{-\infty}^{\infty} f_{X, Y}(s, t) \ ds
		= 2 e^{-2t} \int_{0}^{\infty} e^{-s} \ ds
		= 2 e^{-2t} \left( 0 + 1 \right)
		= 2 e^{-2t}
	\]
	ונעבור לבדיקה,
	\[
		f_X(s) f_Y(t)
		= e^{-s} 2 e^{-2t}
		= f_{X, Y}(s, t)
	\]
	ולכן $X, Y$ משתנים מקריים בלתי־תלויים.
\end{solution}

\subquestion{}
נחשב את ההסתברות למאורע $\{1 < X\} \cap \{Y < 1\}$.
\begin{solution}
	\begin{align*}
		\PP(\{1 < X\} \cap \{Y < 1\})
		& = \PP(1 < X, Y < 1) \\
		& = \PP(1 < X) \PP(Y < 1) \\
		& = \left( \int_{1}^{\infty} e^{-s} \ ds \right) \left( \int_{0}^{1} 2 e^{-2t} \ dt \right) \\
		& = (0 + e^{-1}) (-e^{-2} + 1) \\
		& = e^{-1} - e^{-3}
	\end{align*}
\end{solution}

\subquestion{}
נחשב את ההסתברות למאורע $\{X < Y\}$.
\begin{solution}
	\begin{align*}
		\PP(X < Y)
		& = \iint_{\{ x, y \in \RR^2 \mid x < y \}} f_{X, Y}(x, y)\ dx\ dy \\
		& = \int_{-\infty}^{\infty} \int_{-\infty}^{y} f_{X, Y}(x, y)\ dx\ dy \\
		& = 2 \int_0^{\infty} e^{-2y} \int_0^{y} e^{-x}\ dx\ dy \\
		& = 2 \int_0^{\infty} e^{-2y} \left. (-e^{-x}) \right\rvert_{x = 0}^{x = y}\ dy \\
		& = 2 \int_0^{\infty} e^{-2y} - e^{-3y}\ dy \\
		& = \left. 2 \left(-\frac{1}{2} e^{-2y} + \frac{1}{3} e^{-3y} \right) \right\rvert_{y = 0}^{y = \infty} \\
		& = 2 (0 - (-\frac{1}{2} + \frac{1}{3})) \\
		& = \frac{1}{3}
	\end{align*}
	ולכן נובע $\PP(X < Y) = \frac{1}{3}$.
\end{solution}

\question{}
יהיו $X, Y, Z$ משתנים מקריים בלתי־תלויים שווי־התפלגות בעלי התפלגות $Unif([0, 1])$.
נחשב את $\PP(X \ge YZ)$.
\begin{solution}
	\begin{align*}
		\PP(X \ge YZ)
		& = \iiint_{\{ x, y, z \in \RR^3 \mid x \ge yz \}} f_{X, Y, Z}(x, y, z)\ dx\ dy\ dz \\
		& = \iiint_{\{ x, y, z \in \RR^3 \mid x \ge yz \}} f_X(x) f_Y(y) f_Z(z)\ dx\ dy\ dz \\
		& = \iiint_{\{ x, y, z \in \RR^3 \mid x \ge yz \}} 1_{[0, 1]}(x) 1_{[0, 1]}(y) 1_{[0, 1]}(z)\ dx\ dy\ dz \\
		& = \int_0^1 \int_0^1 \int_{yz}^1 1\ dx\ dy\ dz \\
		& = \int_0^1 \int_0^1 1 - yz\ dy\ dz \\
		& = \int_0^1 1 - \frac{1}{2} z\ dz \\
		& = 1 - \frac{1}{4}
	\end{align*}
	ולכן $\PP(X \ge YZ) = \frac{3}{4}$.
\end{solution}

\question{}
נשתמש במשפט פוביני כדי להראות שאם $X$ משתנה מקרי אי־שלילי בעל צפיפות ובעל תוחלת סופית אז $\EE(X) = \int_0^\infty (1 - F_X(t))\ dt$.
\begin{proof}
	נשחזר את הוכחת נוסחת הזנב לחישוב תוחלת שראינו בהרצאה עבור המקרה הרציף,
	\[
		\EE(X)
		= \int_0^{\infty} t f_X(t)\ dt
		= \int_0^{\infty} \int_{0}^{t} f_X(t)\ dt
		= \iint_{\{ t, s \in \RR^2 \mid 0 \le s \le t \}} f_X(t)\ dt\ ds
		= \int_0^{\infty} \int_{t}^{\infty} f_X(t)\ ds\ dt
		= \int_0^{\infty} 1 - F_X(t)\ dt
	\]
	כאשר הצדקת כלל המעברים זהה להצדקה בהוכחה המקורית כך שמשפט פוביני לטורים מוחלף במשפט פוביני לאינטגרלים.
\end{proof}

\question{}
יהי $n \in \NN$ ויהיו $X_i \sim Exp(1)$ בלתי־תלויים עבור $i \in [n]$.
נסמן $Y_i = X_i X_{i + 1}$ וכן $Y = \sum_{i = 1}^{n - 1} Y_i$.
נחשב את התוחלת והשונות של $Y$.
\begin{solution}
	נתחיל ונבחין שמתקיים $\EE(Y_i) = \EE(X_i X_{i + 1}) = \EE(X_i) \EE(X_{i + 1}) = 1$.
	לכן נובע ישירות ש־$\EE(Y) = \EE(\sum_{i = 1}^{n - 1} Y_i) = \sum_{i = 1}^{n - 1} \EE(Y_i) = n - 1$.
	נעבור לחישוב השונות, נבחין ש־$\cov(Y_i, Y_i) = \EE(Y_i^2) - {\EE(Y_i)}^2 = \EE(X_i^2) \EE(X_{i + 1}^2) - {\EE(X_i)}^2 {\EE(X_{i + 1})}^2 = 4 - 1 = 3$.
	נמשיך ונחשב, $\cov(Y_i, Y_{i + 1}) = \cov(X_i X_{i + 1}, X_{i + 1} X_{i + 2}) = \EE(X_i X_{i + 1}^2 X_{i + 2}) - \EE(X_i X_{i + 1}) \EE(X_{i + 1} X_{i + 2}) = \EE(X_i) \EE(X_{i + 1}^2) \EE(X_{i + 2}) - \EE(X_i) \EE(X_{i + 1}) \EE(X_{i + 1}) \EE(X_{i + 2}) = 1 \cdot 4 \cdot 1 - 1 \cdot 1 \cdot 1 \cdot 1 = 3$.
	לכל $i, j$ כך ש־$i + 1 < j$ המשתנים הם בלתי־תלויים ולכן $\cov(Y_i, Y_j) = 0$.
	לבסוף נובע,
	\[
		\var(Y)
		= \var(\sum_{i = 1}^{n - 1} Y_i)
		= \sum_{i = 1}^{n - 1} \var(Y_i) + 2 \sum_{i = 1}^{n - 2} \cov(Y_i, Y_{i + 1}) + 0
		= 3(n - 1) + 2 \cdot 3(n - 2)
		= 9n - 9
	\]
\end{solution}

\question{}
יהי $C \in \RR$ קבוע ונגדיר $A = \{ (x, y) \in \RR^2 \mid x, y \ge 0, x + y \le 1\}$,
יהיו שני משתנים מקריים $X, Y$ כך שמתקיים
\[
	f_{X, Y}(x, y)
	= x 1_A(x, y) C
\]

\subquestion{}
נחשב את $C$.
\begin{solution}
	נבחין שמתקיים,
	\begin{align*}
		\iint_{\RR^2} f_{X, Y}(x, y) \ dx\ dy
		& = \iint_A x C \ dx\ dy \\
		& = \int_0^1 \int_0^{1 - y} x C \ dx\ dy \\
		& = C \int_0^1 \left. \frac{x^2}{2} \right\rvert_{x = 0}^{x = 1 - y}\ dy \\
		& = C \frac{1}{2} \int_0^1 {(1 - y)}^2\ dy \\
		& = C \frac{1}{2} \cdot \left. \left( -\frac{1}{3} \right) {(1 - y)}^3 \right\rvert_{y = 0}^{y = 1} \\
		& = \frac{C}{6} (0 + 1)
	\end{align*}
	מתכונת פונקציית הציפות נובע $1 = \frac{C}{6}$ ולכן בהכרח $C = 6$.
\end{solution}

\subquestion{}
נחשב את ההתפלגות השולית $f_X(t)$ ואת $f_Y(t)$.
\begin{solution}
	עבור $0 \le t \le 1$ מתקיים,
	\[
		f_X(t)
		= \int_{-\infty}^{\infty} f_{X, Y}(t, y) \ dy
		= 6t \int_{-\infty}^{\infty} 1_A(t, y)\ dy
		= 6t \int_0^{1 - t} 1\ dy
		= 6t (1 - t)
	\]
	ולכן
	\[
		f_X(t)
		= \begin{cases}
			6t (1 - t) & 0 \le t \le 1 \\
			0 & \text{else}
		\end{cases}
	\]
	באופן דומה עבור $0 \le t \le 1$ מתקיים,
	\[
		f_Y(t)
		= \int_{-\infty}^{\infty} f_{X, Y}(x, t) \ dx
		= 6 \int_{-\infty}^{\infty} x 1_A(x, t)\ dx
		= 6 \int_0^{1 - t} x\ dx
		= 6 \cdot \left. \frac{x^2}{2} \right\rvert_{x = 0}^{x = 1 - t}
		= 3 {(1 - t)}^2
	\]
	ולכן,
	\[
		f_Y(t)
		= \begin{cases}
			3 {(1 - t)}^2 & 0 \le t \le 1 \\
			0 & \text{else}
		\end{cases}
	\]
\end{solution}

\subquestion{}
נבדוק אם $X$ ו־$Y$ תלויים.
\begin{solution}
	בתחום ${[0, 1]}^2$ מתקיים,
	\[
		f_{X, Y}(x, y)
		= 6x \cdot 1_A(x, y)
		\ne 18x (1 - x) {(1 - y)}^2
		= f_X(x) f_Y(y)
	\]
	ולכן נסיק ש־$X$ ו־$Y$ אינם בלתי־תלויים.
\end{solution}

\subquestion{}
נחשב את $\PP(X < Y)$.
\begin{solution}
	\begin{align*}
		\PP(X < Y)
		& = \iint_{\{ x, y \in \RR^2 \mid x < y \}} f_{X, Y}(x, y)\ dx\ dy \\
		& = 6 \int_0^{\frac{1}{2}} x \int_x^{1 - x} 1\ dy\ dx \\
		& = 6 \int_0^{\frac{1}{2}} x (1 - x - x)\ dx \\
		& = 6 \int_0^{\frac{1}{2}} x - 2x^2\ dx \\
		& = 6 \left. \left( \frac{x^2}{2} - \frac{2x^3}{3} \right) \right\rvert_{x = 0}^{x = \frac{1}{2}} \\
		& = 6 \left( \frac{1}{8} - \frac{1}{12} \right) \\
	\end{align*}
	ולכן נובע $\PP(X < Y) = \frac{1}{4}$.
\end{solution}

\subquestion{}
נחשב את פונקציית הצפיפות של $Z = X + Y$.
\begin{solution}
	\begin{align*}
		F_Z(t)
		& = \PP(Z \le t) \\
		& = \PP(X + Y \le t) \\
		& = \iint_{\{ x, y \in \RR^2 \mid x + y \le t \}} f_{X, Y}(x, y)\ dx\ dy \\
		& = 6 \int_0^t \int_0^{t - x} x\ dy\ dx \\
		& = 6 \int_0^t tx - x^2\ dx \\
		& = 6 \left. \left( \frac{tx^2}{2} - \frac{x^3}{3} \right) \right\rvert_{x = 0}^{x = t} \\
		& = 6 \left( \frac{t^3}{2} - \frac{t^3}{3} \right) \\
		& = t^3
	\end{align*}
	ולכן $f_Z(t) = F_Z'(t) = 3t^2$ כאשר $0 \le t \le 1$, כאשר אחרת $f_Z(t) = 0$.
\end{solution}

\end{document}
