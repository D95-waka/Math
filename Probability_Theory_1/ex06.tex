\documentclass[a4paper]{article}

% packages
\usepackage{inputenc, fontspec, amsmath, amsthm, amsfonts, polyglossia, catchfile}
\usepackage[a4paper, margin=50pt, includeheadfoot]{geometry} % set page margins

% style
\AddToHook{cmd/section/before}{\clearpage}	% Add line break before section
\linespread{1.5}
\setcounter{secnumdepth}{0}		% Remove default number tags from sections
\setmainfont{Libertinus Serif}
\setsansfont{Libertinus Sans}
\setmonofont{Libertinus Mono}
\setdefaultlanguage{hebrew}
\setotherlanguage{english}

% operators
\DeclareMathOperator\cis{cis}
\DeclareMathOperator\Sp{Sp}
\DeclareMathOperator\tr{tr}
\DeclareMathOperator\im{Im}
\DeclareMathOperator\diag{diag}
\DeclareMathOperator*\lowlim{\underline{lim}}
\DeclareMathOperator*\uplim{\overline{lim}}

% commands
\renewcommand\qedsymbol{\textbf{משל}}
\newcommand{\NN}[0]{\mathbb{N}}
\newcommand{\ZZ}[0]{\mathbb{Z}}
\newcommand{\QQ}[0]{\mathbb{Q}}
\newcommand{\RR}[0]{\mathbb{R}}
\newcommand{\CC}[0]{\mathbb{C}}
\newcommand{\getenv}[2][] {
  \CatchFileEdef{\temp}{"|kpsewhich --var-value #2"}{\endlinechar=-1}
  \if\relax\detokenize{#1}\relax\temp\else\let#1\temp\fi
}
\newcommand{\explain}[2] {
	\begin{flalign*}
		 && \text{#2} && \text{#1}
	\end{flalign*}
}

% headers
\getenv[\AUTHOR]{AUTHOR}
\author{\AUTHOR}
\date\today

\title{פתרון מטלה 06 --- תורת ההסתברות (1), 80420}

\DeclareMathOperator{\Supp}{Supp}

\begin{document}
\maketitle
\maketitleprint{}

\question{}
בקופה $n$ מטבעות. בכל סיבוב שני שחקנים מטילים באופן בלתי תלוי קבויה המתפלגת לפי $\PP(1) = \PP(2) = \PP(3) = \PP(4) = \frac{1}{5}$ ו־$\PP(5) = \PP(6) = \frac{1}{10}$.
מי שמקבל את הערך הגבוה ביותר מקבל מטבע אחד מהקופה, כאשר אם הערכים שווים אף שחקן לא מקבל מטבע.

\subquestion{}
נחשב מהי התפלגות מספר המטבעות שמרוויח כל אחד מהשחקנים בסוף המשחק.
\begin{solution}
	נגדיר $X_i$ שהשחקן הראשון זכה במטבע ה־$i$, ובהתאם $X = \sum_{k = 1}^{n} X_i$, המקרה שהשחקן הראשון זכה במספר מטבעות במשחק. \\*
	נתחיל בחישוב המקרה $\PP(X_i = 1)$, אם $Y, Z$ תוצאות ההטלה של שני השחקנים, מתקבל
	\[
		\PP(X_i = 1) = \PP(Y > Z) = \PP(Y > Z, Z = 1) + \cdots + \PP(Y > Z, Z = 6) = \frac{4}{5} \cdot \frac{1}{5} + \cdots + 0 = \frac{4 + 3 + 2 + 1}{25} + \frac{1}{10} \cdot \frac{1}{10} = \frac{41}{100}
	\]
	נגדיר בהתאם $p = 0.41$.
	נבחין כי מההגדרה $X \sim Bin(n, p)$, לכן
	\[
		\PP(X = k) = \binom{n}{k} p^k {(1 - p)}^{n - k}
	\]
\end{solution}

\subquestion{}
נחשב מהי ההסתברות לקבל תוצאת תיקו בהטלה מסוימת.
\begin{solution}
	למעשה מסימטריה ההסתברות בהטלה כלשהי שהשחקן הראשון יזכה ושהשני שוות, והן $0.41$, לכן ההסתברות לתיקו היא $1 - 2 \cdot 0.41 = 0.09$.
\end{solution}

\subquestion{}
נחשב מהי ההסתברות שמספר הסיבובים הכולל במשחק יהיה לכל היותר $n + 1$.
\begin{solution}
	אנו יודעים שהמשחק מסתיים לאחר $n$ זכיות של אחד המשתתפים, או לחילופין לאחר $n + Z$ זכיות עבור $Z$ משתנה מקרי המייצג את מספר תוצאות התיקו. \\*
	ההסתברות לתוצאת תיקו היא קבועה, ומתקיים $Z \sim Ber(0.09)$. \\*
	נגדיר משתנה מקרי חדש $W$ כך שהוא מייצג את מספר תוצאות התיקו שהיו, 

	נגדיר $W_i$ שסיבוב הסתיים בניצחון כלשהו (ולא בתיקו), לפי הסעיף הקודם מתקיים $W \sim Ber(0.91)$. \\*
	עוד נגדיר $W$ הסיכוי שהיו $k$ סיבובים במשחק, לכן $\PP(W = k) = \PP(\sum_{i = 1}^k W_i = n)$, אבל לפי הנוסחה שראינו בכיתה נובע $W \sim Bin(n, 0.91)$, לכן בהתאם
	\[
		\PP(W \le n + 1)
		= \PP(W < n) + \PP(W = n) + \PP(W = n + 1)
		= 0 + {0.91}^n + \binom{n + 1}{n} {0.91}^n \cdot 0.09
	\]
\end{solution}

\question{}
תהי ${\{X_i\}}_{i \in \NN}$ סדרה של משתנים מקריים המתפלגים $Ber(p)$ בלתי תלויים. \\*
נקבע $Y_0 = 0$ וכן $Y_1 = \min \{i \mid X_i = 1\}$, ובאופן דומה ואינדוקטיבי
\[
	Y_k = \min \{i \in \NN \mid X_i = 1, i > Y_{k - 1}\}
\]
נוכיח ש־${\{Y_k - Y_{k - 1}\}}_{k \in \NN}$ היא סדרה של משתנים מקריים בלתי־תלויים המתפלגים גאומטרית עם סיכוי $p$.
\begin{proof}
	תהי סדרה ${\{l_i\}}_{i = 1}^k \subseteq \NN$, ולכן
	\begin{align*}
		& \PP(\forall 1 \le i \le k, Y_i - Y_{i - 1} = l_i) \\
		= & \PP(Y_0 = 0, Y_1 - 0 = l_1, Y_2 - Y_1 = l_2, \dots, Y_k - Y_{k - 1} = l_k) \\
		= & \PP(Y_0 = 0, Y_1 = l_1, Y_2 = l_2 + l_1, \dots, Y_k = l_k + \cdots + l_1) \\
		= & \PP(X_1 = 0, \dots, X_{l_1} = 1, X_{l_1 + 1} = 0, \dots, X_{l_2 + l_1} = 1, X_{l_2 + 1} = 0, \dots, X_{l_k + \cdots + l_1} = 1) \\
		= & \PP(X_1 = 0) \cdots \PP(X_{l_1} = 1) \PP(X_{l_1 + 1} = 0) \cdots \PP(X_{l_2 + l_1} = 1) \PP(X_{l_2 + 1} = 0) \cdots \PP(X_{l_k + \cdots + l_1} = 1) \\
		= & {(1 - p)}^{l_1 - 1} p {(1 - p)}^{l_2 - 1} p \cdots {(1 - p)}^{l_k - 1} p \\
		= & p^k {(1 - p)}^{l - k}
	\end{align*}
	עבור $l = \sum_{i = 1}^k l_i$.
	נבחין כי מצאנו אי־תלות, שכן אין תלות בבחירת הערכים או $k$, וכן שמתקיים $\PP(Y_k - Y_{k - 1} = l_k) = p {(1 - p)}^{l_k - 1}$, דהינו $Y_k - Y_{k - 1} \sim Geo(p)$.
\end{proof}

\question{}
יהיו $X_1, \dots, X_r \sim Geo(p)$ בלתי־תלויים.

\subquestion{}
נוכיח כי
\[
	\PP(X_1 + X_2 = 2 + k)
	= (k + 1) {(1 - p)}^k p^2
\]
\begin{proof}
	מנוסחת ההסתברות השלמה ושימוש ב־ $\supp(X_1 + X_2)$ נובע
	\begin{align*}
		\PP(X_1 + X_2 = 2 + k)
		& = \sum_{i = 1}^{2 + k} \PP(X_1 + X_2 = 2 + k, X_1 = i) \\
		& = \sum_{i = 1}^{2 + k} \PP(X_2 = 2 + k - i, X_1 = i) \\
		& = \sum_{i = 1}^{2 + k} \PP(X_2 = 2 + k - i) \PP(X_1 = i) \\
		& = \sum_{i = 1}^{2 + k} p {(1 - p)}^{2 + k - i - 1} p {(1 - p)}^{i - 1} \\
		& = \sum_{i = 1}^{2 + k} p^2 {(1 - p)}^{k} \\
		& = (k + 2 - 1) p^2 {(1 - p)}^{k}
	\end{align*}
\end{proof}

\subquestion{}
נוכיח כי מתקיים
\[
	\PP(\sum_{i = 1}^{r} X_i = r + k)
	= \binom{k + r - 1}{r - 1} {(1 - p)}^k p^r
\]
\begin{proof}
	נוכיח את הטענה באינדוקציה על $r$, כאשר את המקרה של $r = 1$ טריוויאלי ו־$r = 2$ הוכח בסעיף א'. \\*
	נניח אם כך שהטענה נכונה עבור $1 \le r' < r$ ולכן
	\begin{align*}
		\PP(\sum_{i = 1}^{r} X_i = r + k)
		& = \sum_{j = 1}^{r + k} \PP(X_1 = j) \PP(\sum_{i = 2}^{r} X_i = (r - 1) + (k - j - 1)) \\
		& = \sum_{j = 1}^{r + k} p {(1 - p)}^{j - 1} \cdot \binom{(k - j) + (r - 1) - 1}{(r - 1) - 1} {(1 - p)}^{k - j} p^{r - 1} \\
		& = \sum_{j = 1}^{r + k} \cdot \binom{k - j + r - 3}{r - 2} {(1 - p)}^k p^r \\
		& = \sum_{j = 1}^{r + k} \cdot \binom{k - j + r - 3}{r - 2} {(1 - p)}^k p^r \\
	\end{align*}
	ולבסוף מזהות ונדרמונדה השוויון המבוקש מתקיים והשלמנו את מהלך האינדוקציה. 
\end{proof}

\question{}
\subquestion{}
נוכיח שאם $X \sim Bin(n, p)$ אז $(n - X) \sim Bin(n, 1 - p)$.
\begin{proof}
	\[
		\PP(n - X = l)
		= \PP(X = n - l)
		= \binom{n}{n - l} p^{n - l} {(1 - p)}^{n - (n - l)}
		= \binom{n}{l} p^{n - l} {(1 - p)}^{l}
	\]
	ולכן מהגדרה $n - X \sim Bin(n, 1 - p)$.
\end{proof}

\subquestion{}
שתי חברות מטילות כל אחת מטבע הוגן $n$ פעמים באופן בלתי־תלוי. \\*
נסמן $N_i$ את מספר תוצאות העץ של החברה ה־$i$. \\*
נראה שמתקיים $N_1 + (n - N_2) \sim Bin(2n, \frac{1}{2})$.
\begin{proof}
	בתרגול ראינו שמתקיים $N_1 + N_2 \sim Bin(2n, \frac{1}{2})$.
	עוד נבחין כי $(n - N_2) \sim Bin(n, 1 - \frac{1}{2}) = Bin(n, \frac{1}{2})$ ולכן תוצאת טענה זו עודנה תקפה ומתקיים $N_1 + (n - N_2) \sim (2n, \frac{1}{2})$.
\end{proof}

\subquestion{}
נראה כי ההסתברות לכך ששתיהן קיבלו את אותו מספר תוצאות $H$ היא $\binom{2n}{n} / 4^n$ ונסיק את הזהות
\[
	\sum_{k = 0}^{n} {\binom{n}{k}}^2 = \binom{2n}{n}
\]
\begin{proof}
	\[
		\PP(N_1 + (n - N_2) = 0)
		= \binom{2n}{n} {(\frac{1}{2})}^n {(1 - \frac{1}{2})}^{2n - n}
		= \binom{2n}{n} \frac{1}{4^n}
	\]
	אבל מתוצאת התרגול נובע גם
	\[
		\PP(N_1 + (n - N_2) = 0)
		= \PP(N_1 + N_2 = n)
		= \sum_{k = 0}^{n} {\binom{n}{k}}^2 \frac{1}{4^n}
	\]
	ולכן נובע
	\[
		\sum_{k = 0}^{n} {\binom{n}{k}}^2 = \binom{2n}{n}
	\]
\end{proof}

\question{}
עבור $i \in [n]$ נגדיר $X_i \sim Ber(p)$ ו־$Z_i \sim Ber(q)$, ובנוסף $Y_i = X_i Z_i, X = \sum X_i, Y = \sum Y_i$.

\subquestion{}
נראה ש־$X \sim Bin(n, p), Y \sim Bin(n, pq)$ ולכל $0 \le m \le n$ גם $Y \mid \{X = m\} \sim Bin(m, q)$.
\begin{proof}
	למעשה כבר ראינו בכיתה כי $X \sim Bin(n, p)$, נראה כי גם $Y \sim Bin(n, pq)$.
	\begin{align*}
		\PP(Y_i = k)
		= \PP(X_i Z_i = k)
		& = \begin{cases}
			\PP(X_i = 1) \PP(Z_i = 1) & k = 1 \\
			\PP(X_i = 1) \PP(Z_i = 0) + \PP(X_i = 0) \PP(Z_i = 1) + \PP(X_i = 0) \PP(Z_i = 0) & k = 0 \\
			\PP(X_i = k) \PP(Z_i = k) & \text{else}
		\end{cases} \\
		& = \begin{cases}
			pq & k = 1 \\
			1 - pq & k = 0 \\
			0 & \text{else}
		\end{cases}
	\end{align*}
	כאשר המעבר האחרון נובע מהשלמה עבור $k = 1, k \notin \{0, 1\}$, לכן $Y_i \sim Ber(pq)$ ובהתאם $Y \sim Bin(n, pq)$. \\*
	נוכיח ש־$Y \mid \{X = m\}$ באינדוקציה על $n$. \\*
	נניח $n = 1$ ולכן
	\begin{align*}
		\PP(Y = k \mid X = m)
		& = \PP(X_1 Z_1 = k \mid X_1 = m)
		&& = \PP(m Z_1 = k, X_1 = m) / \PP(X_1 = m) \\
		& = \PP(m Z_1 = k) \PP(X_1 = m) / \PP(X_1 = m)
		&& = \PP(m Z_1 = k)
	\end{align*}
	אבל $k \in \{0, 1\}$ ולכן $\PP(Z_1 = k) = \binom{m}{k} q^k {(1 - q)}^{m - k}$ מחישוב ישיר ולכן הטענה תקפה עבור המקרה $n = 1$. \\*
	נניח אם כך שהטענה נכונה עבור $1 < n \in \NN$, ונבדוק את המקרה $n$,
	\begin{align*}
		\PP(Y = k \mid X = m)
		& = \PP(Y = k \mid X = m, X_1 = 0) \PP(X_1 = 0) + \PP(Y = k \mid X = m, X_1 = 1) \PP(X_1 = 1) \\
		& = \PP(\sum_{i = 2}^n Y_i = k \mid \sum_{i = 2}^n X_i = m, X_1 = 0) \PP(X_1 = 0) \\
		& \qquad + \PP(Z_1 + \sum_{i = 2}^n Y = k \mid \sum_{i = 2}^n X_i = m - 1, X_1 = 1) \PP(X_1 = 1) \\
		& = \binom{m}{k} q^k {(1 - q)}^{m - k} (1 - p) + \PP(Z_1 = 0) \binom{m - 1}{k} q^k {(1 - q)}^{m - 1 - k} p \\
		& \qquad + \PP(Z_1 = 1) \binom{m - 1}{k - 1} q^{k - 1} {(1 - q)}^{m - 1 - k + 1} p \\
		& = \binom{m}{k} q^k {(1 - q)}^{m - k} (1 - p) + \binom{m}{k - 1} q^k {(1 - q)}^{m - k} p + \binom{m - 1}{k - 1} q^k {(1 - q)}^{m - k} p \\
		& = \binom{m}{k} q^k {(1 - q)}^{m - k} (1 - p + p)
	\end{align*}
	והשלמנו את מהלך האינדוקציה, ולכן $Y \mid \{X = m\} \sim Bin(m, q)$.
\end{proof}

\subquestion{}
נסיק כי אם $X \sim Bin(n, p)$ ו־$Y$ משתנה מקרי כך שלכל ערך $0 \le m \le n$ מתקיים $Y \mid \{X = m\} \sim Bin(m, q)$ אז $Y \sim Bin(n, pq)$.
\begin{proof}
	לא יודע
\end{proof}

\question{}
נניח כי הסתברות של א' לנצח את ב' במשחק היא $\frac{2}{3}$ באופן בלתי תלוי בתוצאת המשחקים הקודמים. \\*
השניים משחקים עד שאחד מהם זוכה בשלושה משחקים בסך־הכול.
נחשב מה ההסתברות שא' יזכה.
\begin{solution}
	אם $X_i$ שא' זכה בסיבוב ה־$i$ אז $X_i \sim U(\frac{2}{3})$, ולכן גם $X^n = \sum_{i = 1}^n X_i$ מספר הזכיות ב־$n$ סיבובים הוא $X^n \sim Bin(n, \frac{1}{3})$. \\*
	אילו $n > 5$ אז בוודאות יש לפחות שלושה נצחונות לאחד מהם, לכן נבחן את $n = 5$ ונבדוק את $\PP(X^n \ge 3)$:
	\begin{align*}
		\PP(X^n \ge 3) 
		& = \PP(X = 3) + \PP(X = 4) + \PP(X = 5) \\
		& = \binom{5}{3} {(\frac{2}{3})}^3 {(1 - \frac{2}{3})}^{5 - 3} + \binom{5}{4} {(\frac{2}{3})}^4 {(1 - \frac{2}{3})}^{5 - 4} + \binom{5}{5} {(\frac{2}{3})}^5 {(1 - \frac{2}{3})}^{5 - 5} \\
		& = \frac{1}{3^5} (10 \cdot 2^3 + 5 \cdot 2^4 + 1 \cdot 2^5) \\
		& = \frac{1}{3^5} (5 \cdot 2^4 + 5 \cdot 2^4 + 2 \cdot 2^4) \\
		& = \frac{2^6}{3^4}
	\end{align*}
\end{solution}

\end{document}
