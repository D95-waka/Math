\documentclass[a4paper]{article}

% packages
\usepackage{inputenc, fontspec, amsmath, amsthm, amsfonts, polyglossia, catchfile}
\usepackage[a4paper, margin=50pt, includeheadfoot]{geometry} % set page margins

% style
\AddToHook{cmd/section/before}{\clearpage}	% Add line break before section
\linespread{1.5}
\setcounter{secnumdepth}{0}		% Remove default number tags from sections
\setmainfont{Libertinus Serif}
\setsansfont{Libertinus Sans}
\setmonofont{Libertinus Mono}
\setdefaultlanguage{hebrew}
\setotherlanguage{english}

% operators
\DeclareMathOperator\cis{cis}
\DeclareMathOperator\Sp{Sp}
\DeclareMathOperator\tr{tr}
\DeclareMathOperator\im{Im}
\DeclareMathOperator\diag{diag}
\DeclareMathOperator*\lowlim{\underline{lim}}
\DeclareMathOperator*\uplim{\overline{lim}}

% commands
\renewcommand\qedsymbol{\textbf{משל}}
\newcommand{\NN}[0]{\mathbb{N}}
\newcommand{\ZZ}[0]{\mathbb{Z}}
\newcommand{\QQ}[0]{\mathbb{Q}}
\newcommand{\RR}[0]{\mathbb{R}}
\newcommand{\CC}[0]{\mathbb{C}}
\newcommand{\getenv}[2][] {
  \CatchFileEdef{\temp}{"|kpsewhich --var-value #2"}{\endlinechar=-1}
  \if\relax\detokenize{#1}\relax\temp\else\let#1\temp\fi
}
\newcommand{\explain}[2] {
	\begin{flalign*}
		 && \text{#2} && \text{#1}
	\end{flalign*}
}

% headers
\getenv[\AUTHOR]{AUTHOR}
\author{\AUTHOR}
\date\today

\title{פתרון מטלה 07 --- תורת ההסתברות (1), 80420}

\DeclareMathOperator{\Supp}{Supp}

\begin{document}
\maketitle
\maketitleprint{}

\question{}
נוכיח או נפריך את הטענות הבאות.

\subquestion{}
נסתור את הטענה שאם יהי $X$ משתנה מקרי המוגדר על מרחב ההסתברות $(\Omega, \PP)$, ונניח ש־$X \sim U(\{1, 2, 3\})$, אז $(\Omega, \PP)$ הוא מרחב הסתברות אחידה.
\begin{solution}
	נניח שמרחב ההסתברות הוא של הטלת קובייה לא אחידה כך שלקבלת המספרים הזוגיים הסתברות של חצי מקבלת המספרים האי־זוגיים, כאשר ההסתברות אחידה בין מספרים עם אותה הזוגיות. \\*
	נגדיר גם $X(1) = X(2) = 1, X(3) = X(4) = 2, X(5) = X(6) = 3$, אז $X \sim U(\{1, 2, 3\})$ בעוד מרחב ההסתברות לא אחיד.
\end{solution}

\subquestion{}
יהיו $X, Y$ משתנים מקריים בלתי־תלויים ובעלי תומך סופי.
נסתור את הטענה שאם $\EE(|X - Y|) = 0$ אז $\PP(X = Y) = 1$.
\begin{solution}
	נניח ש־$X(\omega) = 1$ ו־$Y(\omega) = 2$ לכל $\omega \in \Omega$, אז $|X - Y| = 1$ לכל $\omega$ גם כן, בהתאם $\EE(|X - Y|) = 0$, אבל $\PP(X = Y) = 0$.
\end{solution}

\subquestion{}
נסתור את הטענה שאם נניח ש־$X$ משתנה מקרי בעל תוחלת, אז גם $X^2$ בעל תוחלת.
\begin{solution}
	נניח ש־$\Supp X = \NN$ וכן $\PP(X = n) = \frac{c}{n^3}$, ראינו כי אכן קיים משתנה מקרי כזה עם התפלגות כזו בהרצאות קודמות, ואנו יודעים כי
	\[
		\EE(X) = \sum_{n = 1}^{\infty} n \cdot \frac{c}{n^3}
	\]
	הוא טור מתכנס בהחלט, לעומת זאת
	\[
		\EE(X) = \sum_{n = 1}^{\infty} n^2 \cdot \frac{c}{n^3}
	\]
	הוא טור הרמוני ומתבדר.
\end{solution}

\subquestion{}
נניח ש־$X$ משתנה מקרי כך ש־$X^2$ הוא בעל תוחלת, נוכיח שגם $X$ בעל תוחלת.
\begin{solution}
	נגדיר $S = \supp X$, אז
	\[
		\EE(X^2)
		= \sum_{s \in S} s \PP(X^2 = s)
		= \sum_{s \in S} s \PP(X = \sqrt{s})
		= \sum_{s \in S} |s| \cdot |s \PP(X = s)|
	\]
	הוא טור מתכנס בהחלט, ולכן ממבחן התכנסות גם
	\[
		\sum_{s \in S} |s \PP(X = s)|
	\]
	טור מתכנס, אבל זוהי התכנסות בהחלט של $\EE(X)$ עצמו, קרי יש תוחלת ל־$X$.
\end{solution}

\subquestion{}
יהיו $X, Y$ משתנים מקריים כך ש־$\EE(X) = \EE(Y)$ וכן $\EE(X^2) = \EE(Y^2)$.
נסתור את הטענה שאז $\PP(X = Y) = 1$.
\begin{solution}
	נניח שוב ש־$X, Y$ קבועים כך ש־$X = 1, Y = 2$ כמעט תמיד. \\*
	אז כמובן התוחלת שלהם ושל הריבוע שלהם שווה, אבל $\PP(X = Y) = 0$.
\end{solution}

\subquestion{}
נסתור את הטענה כי קיים משתנה מקרי בדיד $X$ אי־שלילי בעל תוחלת סופית כך שמתקיים
\[
	\forall N \in \NN, \forall n \ge N, \PP(X \ge n) = \frac{\EE(X)}{n}
\]
\begin{proof}
	נניח בשלילה שאכן קיים משתנה מקרי $X$ כזה.
	נובע אם כך עבור ההסתברות שלו
	\[
		\PP(X = n)
		= \PP(X \ge n) - \PP(X \ge n - 1)
		= \frac{\EE(X)}{n} - \frac{\EE(X)}{n - 1}
		= \EE(X) \frac{1}{n(n + 1)}
	\]
	ולכן מהגדרת התוחלת
	\[
		\EE(X)
		= \sum_{n \in \NN} n \PP(n)
		= \sum_{n \in \NN} \EE(X) \frac{1}{n + 1}
	\]
	דהינו
	\[
		1 = \sum_{n = 1}^{\infty} \frac{1}{n + 1}
	\]
	וזו כמובן סתירה אלא אם $\EE(X) = 0$. נעיר שמהנתון התומך הוא לא סופי, אחרת הטענה לא מתקיימת. \\*
	נניח ש־$\EE(X) = 0$, לכן מהנתון התפלגות $X$ קבועה, ובהתאם לאינסופיות התומך היא 0 בלבד, אבל זאת סתירה להגדרת התפלגות, ולכן קיבלנו סתירה.
\end{proof}

\question{}
במשחק מטילים מטבע הוגן עד שמקבלים תוצאה של עץ. אם העץ המתקבל בהטלה ה־$i$ אז מוענקים למטיל $a^i$ נקודות.

\subquestion{}
יהי $X$ המשתנה המקרי המתאר את כמות הנקודות שהוענקה, נחשב את התפלגות $X$
\begin{solution}
	נבחין כי מההגדרה המשתנה $Y$ המתאר את השאלה באיזה סיבוב המשחק הסתיים הוא $Geo(\frac{1}{2})$. \\*
	בהתאם $X = a^Y$, שכן כמות הנקודות המתקבלת היא מספר הסיבוב האחרון כחזקת $a$. \\*
	נעבור אם כן לחישוב ההתפלגות של $X$:
	\[
		\PP(X = n)
		= \PP(a^Y = n)
		= \PP(Y = \log_a n)
		= {(1 - \frac{1}{2})}^{\log_a(n) - 1} \cdot \frac{1}{2}
		= \frac{1}{2^{\log_a n}}
	\]
	נבחין כי התומך הוא $\supp X = \{ a^n \mid n \in \NN \}$.
\end{solution}

\subquestion{}
נחשב את $\EE(X)$ לכל $a$.
\begin{solution}
	\[
		\EE(X)
		= \sum_{s \in \Supp X} s \cdot \PP(X = s)
		= \sum_{n = 1}^\infty a^n \cdot \frac{1}{2^{\log_a a^n}}
		= \sum_{n = 1}^\infty {\left(\frac{a}{2}\right)}^n
		= \frac{\frac{a}{2}}{1 - \frac{a}{2}}
		= \frac{a}{2 - a}
	\]
\end{solution}

\subquestion{}
נחשב את ההסתברות להרוויח יותר מעשר נקודות עבור $a = 2$.
\begin{solution}
	נבחין כי $X = 2^4$ הוא המקרה הראשון שבו מתקבלות מעל 10 נקודות, לכן אנו מחפשים את $\PP(X \ge 4) = 1 - \PP(1 \le X \le 3) = 1 - \PP(X = 1) - \PP(X = 2) - \PP(X = 3)$.
\end{solution}

\question{}
\subquestion{}
ישנן שלוש צנצנות עוגיות, בראשונה 15 עוגיות, בשנייה 18 ובשלישית 9.

\subsubsection{i}
בוחרים עוגייה באופן אקראי ואחיד, נחשב את התוחלת של מספר העוגיות בצנצנת שלה.
\begin{solution}
	אם נמספר את העוגיות נקבל 42 עוגיות, ונגדיר את המשתנה המקרי $X$ כמחזיר לכל מספר עוגייה את מספר העוגיות בצנצנת שלה, כך לדוגמה $X(1) = 15$ וכן הלאה. \\*
	בהתאם נקבל
	\[
		\EE(X)
		= \sum_{i = \{15, 18, 9\}} i \PP(X = i)
		= 15 \cdot \frac{15}{42} + 18 \cdot \frac{18}{42} + 9 \cdot \frac{9}{42}
	\]
\end{solution}

\subsubsection{ii}
בוחרים צנצנת באקראי ובאופן אחיד, נחשב את התוחלת של מספר העוגיות בצנצנת שבחרנו.
\begin{solution}
	הפעם נגדיר $Y$ מחזיר את מספר העוגיות עבור מספר צנצנת, כלומר $Y(1) = 15$ וכדומה, לכן
	\[
		\EE(Y) = \frac{1}{3} \cdot 15 + \frac{1}{3} \cdot 18 + \frac{1}{3} \cdot 9
	\]
\end{solution}

\subquestion{}
מטילים שתי קוביות הוגנות.
נמצא את תוחלת סכום התוצאות של הקוביות בהינתן שהקוביות נפלו על פאות שונות.
\begin{solution}
	נגדיר $X, Y$ תוצאת הטלת שתי הקוביות, ונגדיר גם $Z = X + Y \mid X \ne Y$. \\*
	הסכום יכול לצאת במקרה זה בין 3 ל־11, דהינו $\supp Z = \{3, \dots, 11\}$, ולכן
	\[
		\EE(Z)
		= \sum_{i = 3}^{11} i \PP(Z = i)
		= 3 \cdot \frac{2}{36} + \cdots + 11 \cdot \frac{2}{36}
	\]
\end{solution}

\subquestion{}
מטילים קובייה הוגנת שוב ושוב עד שיוצאת התוצאה 6. \\*
יהי $X$ המשתנה המקרי המייצג את מספר הפעמים שהתקבלה התוצאה 2. \\*
נחשב את התוחלת של $X$ בהינתן שכל ההטלות היו זוגיות.
\begin{solution}
	נגדיר $Y$ שיצא 6 בתוצאה ה־$i$, לכן $Y \sim Geo(\frac{1}{6})$. \\*
	בנוסף נגדיר $Z_n$ המשתנה המקרי המייצג את מספר ה־2 שהתקבלו בהינתן שהיו $n$ זריקות, לכן $Z_n \sim Bin(n - 1, \frac{1}{5})$ מהנתון. \\*
	לבסוף נגדיר את $X$ להיות מספר ה־2 שהתקבלו ללא קשר לכמות הזריקות, לכן מנוסחת ההסתברות השלמה $\PP(X = k) = \sum_{n = k + 1}^\infty \PP(X = k \mid Y = n) \PP(Y = n)$, ולכן
	\begin{align*}
		\PP(X = k)
		& = \sum_{n = k + 1}^{\infty} \PP(Z_k = n) \PP(Y = n) \\
		& = \sum_{n = k + 1}^{\infty} \binom{n - 1}{k} {(\frac{1}{5})}^k \cdot {(\frac{4}{5})}^{n - 1 - k} \cdot {(\frac{5}{6})}^{n - 1} \frac{1}{6} \\
		& = \sum_{n = k + 1}^{\infty} \binom{n - 1}{k} 4^{n - 1 - k} \cdot \frac{1}{6^n}
	\end{align*}
	באופן דומה אם כל ההטלות היו זוגיות אז $Y \sim Geo(\frac{1}{3})$ וכן $Z_n \sim Bin(n - 1, \frac{1}{2})$ ובהתאם
	\[
		\EE(X)
		= \sum_{n = k + 1}^{\infty} n \cdot \PP(X = n)
		= \sum_{k = 1}^{\infty} k \sum_{n = k + 1}^{\infty} \binom{n - 1}{k} 2^{n - 1 - k} \cdot \frac{1}{3^n}
	\]
\end{solution}

\subquestion{}
עשרה שופטים בתחרות מעניקים לקבוצת אנשים ציונים מקריים המתפלגים אחיד ב־$[10]$. \\*
נחשב את תוחלת הציון המינימלי והציון המקסימלי שקיבלה הקבוצה.
\begin{solution}
	צריך להסתכל על ה־$\min\{ X_i \}$ של המשתנה המקרי של $X_i$ הציון של שופט, אז מקבלים את זה.
	השאלה לא מוגדרת היטב. \\*
	נגדיר $X$ המשתנה המקרי המייצג את ציון הקבוצה, ו־$X_i$ הציון שנתן שופט ה־$i$, לכן $X = X_1 + \cdots + X_{10}$, בהתאם $X_i \sim U([10])$ וכן
	\[
		\EE(X) = \sum_{n = 10}^{100} n \cdot \PP(X = n)
	\]
\end{solution}

\question{}
יהי משתנה מקרי בעל תוחלת $X$ הנתמך על $\NN$. \\*
נוכיח כי $X \sim Geo(p)$ עבור $p \in (0, 1)$ כלשהו אם ורק אם לכל $s \in \NN \cup \{0\}$ מתקיים
\[
	\EE(X \mid X > s) = \EE(X) + s
\]
\begin{proof}
	נניח ש־$X \sim Geo(p)$ ולכן תכונת חוסר הזיכרון מתקיימת, כלומר נובע $X \overset{d}{=} X - s \mid X > s$ לכל $s$ בתחום. \\*
	נחשב
	\begin{align*}
		\EE(X \mid X > s)
		& = \sum_{n = 1}^{\infty} n \PP(X - s = n - s \mid X > s) \\
		& = \sum_{n = s + 1}^{\infty} n \PP(X - s = n - s \mid X > s) \\
		& = \sum_{n = s + 1}^{\infty} n \PP(X = n - s) \\
		& = \sum_{n = s + 1}^\infty n \cdot {(1 - p)}^{n - s - 1} p
	\end{align*}
	ומצד שני
	\[
		\EE(X) + s
		= \frac{s p}{1 - (1 - p)} + \sum_{n = 1}^{\infty} n {(1 - p)}^{n - 1} p
		= \sum_{n = 1}^{\infty} {(1 - p)}^{n - 1} s p + \sum_{n = 1}^{\infty} n {(1 - p)}^{n - 1} p
		= \sum_{n = 1}^{\infty} (n + s) {(1 - p)}^{n - 1} p
	\]
	ומצאנו כי הביטויים שווים.

	נניח את הכיוון השני של הטענה, אז הטענה נכונה גם עבור $s = 1$, כלומר
	\[
		\EE(X \mid X > 1) = \EE(X) + 1
	\]
	לכן
	\[
		\EE(X - 1 \mid X > 1)
		= \sum_{n = 1}^{\infty} n \PP(X - 1 = n \mid X > 1)
		= \sum_{n = 2}^{\infty} (n - 1) \PP(X = n \mid X > 1)
		= \EE(X \mid X > 1) - 1
		= \EE(X)
	\]
	ואז $\EE(X - 1 \mid X > 1) - \EE(X) = 0$, ומההתכנסות בהחלט ואינדוקציה על $n$ נקבל משקילות תכונת חוסר הזיכרון את המבוקש.
\end{proof}

\question{}
נמצא דוגמה לשני משתנים מקריים בעלי תוחלת תלויים כך ש־$\EE(X \cdot Y) = \EE(X) \cdot \EE(Y)$.
\begin{solution}
	נגדיר $X = Y \sim Ber(1)$ ולכן
	\[
		\EE(XY)
		= \EE(X^2)
		= \sum_{n \in \{0, 1\}} n^2 \PP(X = n)
		= 0 \cdot (1 - 1) + 1 \cdot 1
		= 1
	\]
	ומצד שני
	\[
		\EE(X) \cdot \EE(Y)
		= {(\EE(X))}^2
		= 1^2
		= 1
	\]
	ומצאנו כי הטענה מתקיימת.
\end{solution}

\end{document}
