\documentclass[a4paper]{article}

% packages
\usepackage{inputenc, fontspec, amsmath, amsthm, amsfonts, polyglossia, catchfile}
\usepackage[a4paper, margin=50pt, includeheadfoot]{geometry} % set page margins

% style
\AddToHook{cmd/section/before}{\clearpage}	% Add line break before section
\linespread{1.5}
\setcounter{secnumdepth}{0}		% Remove default number tags from sections
\setmainfont{Libertinus Serif}
\setsansfont{Libertinus Sans}
\setmonofont{Libertinus Mono}
\setdefaultlanguage{hebrew}
\setotherlanguage{english}

% operators
\DeclareMathOperator\cis{cis}
\DeclareMathOperator\Sp{Sp}
\DeclareMathOperator\tr{tr}
\DeclareMathOperator\im{Im}
\DeclareMathOperator\diag{diag}
\DeclareMathOperator*\lowlim{\underline{lim}}
\DeclareMathOperator*\uplim{\overline{lim}}

% commands
\renewcommand\qedsymbol{\textbf{משל}}
\newcommand{\NN}[0]{\mathbb{N}}
\newcommand{\ZZ}[0]{\mathbb{Z}}
\newcommand{\QQ}[0]{\mathbb{Q}}
\newcommand{\RR}[0]{\mathbb{R}}
\newcommand{\CC}[0]{\mathbb{C}}
\newcommand{\getenv}[2][] {
  \CatchFileEdef{\temp}{"|kpsewhich --var-value #2"}{\endlinechar=-1}
  \if\relax\detokenize{#1}\relax\temp\else\let#1\temp\fi
}
\newcommand{\explain}[2] {
	\begin{flalign*}
		 && \text{#2} && \text{#1}
	\end{flalign*}
}

% headers
\getenv[\AUTHOR]{AUTHOR}
\author{\AUTHOR}
\date\today

\title{פתרון מטלה 10 --- תורת ההסתברות (1), 80420}
% chktex-file 44

\DeclareMathOperator{\Supp}{Supp}

\begin{document}
\maketitle
\maketitleprint{}

\question{}
יהי $X$ משתנה מקרי שהמומנט ה־$k$ שלו קיים וסופי. \\
נוכיח שלכל $c > 0$ מתקיים
\[
	\PP(|X - \EE(X)| \ge c)
	\le \frac{\EE({\left\lvert X - \EE(X)\right\rvert}^k)}{c^k}
\]
\begin{proof}
	נבחין כי הקבוצות הבאות שקולות,
	\[
		\{ |X - \EE(X)| \ge c \}
		= \{ {|X - \EE(X)|}^k \ge c^k \}
	\]
	ולכן מאי־שוויון מרקוב והעובדה שנתון כי המומנט ה־$k$ של $X$ מוגדר,
	\[
		\PP(|X - \EE(X)| \ge c)
		\PP({|X - \EE(X)|}^k \ge c^k)
		\le \frac{\EE({|X - \EE(X)|}^k)}{c^k}
	\]
\end{proof}

\question{}
נחשב את הפונקציה יוצרת המומנטים של $U([k])$ ונקבע את תחום הגדרתה.
\begin{solution}
	נניח ש־$X \sim U([k])$ ולכן עלינו למצוא את $M_X(t)$ לכל $t \in \RR$ או להראות ש־$M_X(t)$ לא מוגדר.
	\begin{align*}
		M_X(t)
		& = \EE(e^{t X}) \\
		& = \sum_{s \in \supp e^{t X}} s \PP(e^{t X} = s) \\
		& = \sum_{s \in \{ e^{tn} \mid n \in [k]\}} s \PP(e^{t X} = s) \\
		& = \sum_{n = 1}^{k} e^{tn} \PP(e^{tX} = e^{tn}) \\
		& = \sum_{n = 1}^{k} e^{tn} \frac{1}{k} \\
		& = \frac{1}{k} \cdot e^t \frac{e^{tk} - 1}{e^t - 1} \\
		& = \frac{e^{t(k + 1)} - e^t}{k(e^t - 1)}
	\end{align*}
	ולכן $M_X(t) = \frac{e^{t(k + 1)} - e^t}{k(e^t - 1)}$, אבל מצאנו בחישוב שלכל $t \ne 0$ הביטוי מוגדר, ולכן $M_X(t) : \ZZ \setminus \{ 0 \} \to \RR$.
\end{solution}

\question{}
יהי $X$ משתנה מקרי עם פונקציה יוצרת מומנטים $M_X(t) = at + b$ עבור $a, b \in \RR$. \\
נמצא את ערכי $a, b$ ואת התפלגות $X$.
\begin{solution}
	נתחיל בהצבה,
	\[
		M_X(0)
		= \EE(e^{0X})
		= \EE(1)
		= 1
		= a \cdot 0 + b
	\]
	ולכן $b = 1$.
	עוד אנו יודעים ש־$M_X'(0) = \EE(X)$ אבל $M_X'(t) = a$ ולכן $\EE(X) = a$, וכן $\EE(X^2) = M_X''(0) = 0$ ולכן
	\[
		\var(X)
		= \EE(X^2) - {\EE(X)}^2
		= 0 - a^2
	\]
	ומאי־שליליות השונות נובע ש־$a = 0$ בלבד, כלומר $M_X(t) = 1$ וכן $\EE(X) = 0, \var(X) = 0$, כלומר $X = 0$.
\end{solution}

\question{}
יהי $X \sim Poi(\lambda)$,
נראה שלכל $c > 1$ מתקיים
\[
	\PP(X \ge c \lambda)
	\le \frac{e^{c\lambda - \lambda}}{c^{c\lambda}}
\]
\begin{proof}
	בהרצאה מצאנו שמתקיים $M_X(t) = e^{\lambda(e^t - 1)}$, לכן מאי־שוויון צ'רנוף לכל $t > 0$,
	\[
		\PP(X \ge c \lambda)
		\le \frac{M_X(t)}{e^{t c \lambda}}
		= \frac{e^{c\lambda (e^t - 1)}}{e^{t c \lambda}}
	\]
	לכן בפרט כאשר $t = \log c$ מתקיים,
	\[
		\PP(X \ge c \lambda)
		\le \frac{e^{\lambda (e^{\log c} - 1)}}{e^{\log(c) c \lambda}}
		= \frac{e^{\lambda c - \lambda}}{c^{c \lambda}}
	\]
	כפי שרצינו.
\end{proof}

\question{}
יהי $n \in \NN$ ונגדיר טבלה בגודל $n \times n$, נעבור על כל צלע בטבלה ונמחק אותה בהסתברות $\frac{1}{2}$ באופן בלתי־תלוי. \\
יהי $X$ מספר הריבועים שלא נמחקה אף צלע שלהם, נוכיח על־ידי שימוש באי־שוויון הופדינג שלכל $c > 0$,
\[
	\PP(X \ge \frac{n^2}{16} + cn) \le 2 e^{-\frac{c^2}{4}}
\]
\begin{proof}
	נגדיר $Y_{i, j} \sim Ber(\frac{1}{2})$ האם הצלע ה־$i, j$ קיימת (מטעמי סימטריה מותר לנו לבחור כן). \\
	נגדיר גם $X_{i, j} = Y_{i, j} \cdot Y_{i + 1, j} \cdot Y_{i, j + 1} \cdot Y_{i + 1, j + 1} \sim Ber(\frac{1}{16})$ המייצג שהריבוע ה־$i, j$ קיים. \\
	לבסוף גם $X = \sum_{1 \le i, j \le n} X_{i, j}$. \\
	נניח שהצלעות של הטבלה הן בלתי־תלויות, דהינו הטבלה מהצורה:
	\begin{center}
		\begin{tabular}{||c || c || c ||}
			\hline\hline
			A & B & C \\
			\hline\hline
			A & B & C \\
			\hline\hline
			A & B & C \\
			\hline\hline
		\end{tabular}
	\end{center}
	נבחין שמתקיים
	\[
		\EE(Y_{i, j}) = \frac{1}{4}
	\]
	ולכן
	\[
		\EE(X_{i, j})
		= \EE(Y_{i, j}) \cdot \EE(Y_{i + 1, j}) \cdot \EE(Y_{i, j + 1}) \cdot \EE(Y_{i + 1, j + 1})
		= \frac{1}{16}
	\]
	ובהתאם
	\[
		\EE(X)
		= \sum_{1 \le i, j \le n} X_{i, j}
		= \frac{n^2}{16}
	\]
	וכמובן גם מההגדרה לכל $1 \le i, j \le n$ גם $|X_{i, j}| \le 1$, אז תנאי אי־שוויון הופדינג חלים ומתקיים
	\[
		\PP(X - \frac{n^2}{16} \ge cn)
		\PP(X \ge cn + \frac{n^2}{16})
		\le e^{-\frac{c^2 n^2}{2 n^2}}
		\le 2e^{-\frac{c^2}{4}}
	\]
	כפי שרצינו להראות.
\end{proof}

\question{}
שיכור עומד על ציר המספרים, בכל יום הוא צועד צעד אחד שמאלה בהסתברות $p$ או שני צעדים ימינה בהסתברות $1 - p$, באופן בלתי־תלוי בימים הקודמים. \\
נוכיח שההסתברות שהשיכור נמצא בראשית ביום ה־$3n$ דועכת מעריכית לכל $p \ne \frac{2}{3}$.
\begin{proof}
	נגדיר $X_n$ המשתנה המקרי המייצג כמה זז השיכור ביום ה־$n$, כלומק
	\[
		\PP(X_n = s)
		= \begin{cases}
			p & s = 1 \\
			1 - p & s = -2
		\end{cases}
	\]
	וכן נגדיר $Y_n = \sum_{i = 1}^{n} X_n$, סך המרחק שזז השיכור לאורך $n$ ימים.
	נבחין ש־$\frac{X_n + 2}{3} \sim Ber(p)$, ולכן $\frac{Y_n + 2n}{3} \sim Bin(n, p)$.
	לכן גם $Y_{3n} + 2n \sim Bin(n, p)$ ונובע
	\[
		\EE(Y_{3n})
		= 3np + 2n
		= n(3p + 2)
	\]
	נחשב
	\[
		\PP(Y_{3n} = 0)
		= 1 - \PP(Y_{3n} \ge 1) - \PP(Y_{3n} \le -1)
	\]
	וכן
	\[
		\PP(Y_{3n} \ge 1)
		= \PP\left(\frac{Y_{3n}}{\EE(Y_{3n})} - 1 \ge \frac{1}{\EE(Y_{3n})} - 1\right)
		\le \exp\left(- \frac{{\left(\frac{1}{n(3p + 2)} - 1\right)}^2}{2n}\right)
		\le \exp\left(- \frac{{\left(n(3p + 2) - 1\right)}^2}{2n {(n(3p + 2))}^2}\right)
		= o(e^{-n})
	\]
	ובאופן דומה,
	\[
		\PP(Y_{3n} \le -1)
		= \PP(-Y_{3n} \ge 1)
		= \PP\left(\frac{-Y_{3n}}{\EE(Y_{3n})} - 1 \ge \frac{1}{\EE(Y_{3n})} - 1\right)
		= o(e^{-n})
	\]
	ומצאנו שאסימפטוטית יש דעיכה מעריכית לעמידה על הראשית, אלא אם $2n = 3$, אז אי־השוויון איננו מוגדר והתוחלת היא 0 תמיד.
\end{proof}

\end{document}
