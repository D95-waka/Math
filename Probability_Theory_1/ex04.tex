\documentclass[a4paper]{article}

% packages
\usepackage{inputenc, fontspec, amsmath, amsthm, amsfonts, polyglossia, catchfile}
\usepackage[a4paper, margin=50pt, includeheadfoot]{geometry} % set page margins

% style
\AddToHook{cmd/section/before}{\clearpage}	% Add line break before section
\linespread{1.5}
\setcounter{secnumdepth}{0}		% Remove default number tags from sections
\setmainfont{Libertinus Serif}
\setsansfont{Libertinus Sans}
\setmonofont{Libertinus Mono}
\setdefaultlanguage{hebrew}
\setotherlanguage{english}

% operators
\DeclareMathOperator\cis{cis}
\DeclareMathOperator\Sp{Sp}
\DeclareMathOperator\tr{tr}
\DeclareMathOperator\im{Im}
\DeclareMathOperator\diag{diag}
\DeclareMathOperator*\lowlim{\underline{lim}}
\DeclareMathOperator*\uplim{\overline{lim}}

% commands
\renewcommand\qedsymbol{\textbf{משל}}
\newcommand{\NN}[0]{\mathbb{N}}
\newcommand{\ZZ}[0]{\mathbb{Z}}
\newcommand{\QQ}[0]{\mathbb{Q}}
\newcommand{\RR}[0]{\mathbb{R}}
\newcommand{\CC}[0]{\mathbb{C}}
\newcommand{\getenv}[2][] {
  \CatchFileEdef{\temp}{"|kpsewhich --var-value #2"}{\endlinechar=-1}
  \if\relax\detokenize{#1}\relax\temp\else\let#1\temp\fi
}
\newcommand{\explain}[2] {
	\begin{flalign*}
		 && \text{#2} && \text{#1}
	\end{flalign*}
}

% headers
\getenv[\AUTHOR]{AUTHOR}
\author{\AUTHOR}
\date\today

\title{פתרון מטלה 04 --- תורת ההסתברות (1), 80420}

\begin{document}
\maketitle
\maketitleprint{}

\Question{}
יהי $(\Omega, \PP)$ מרחב הסתברות.
נוכיח או נפריך את הטענות הבאות.

\Subquestion{}
נסתור את הטענה כי שני מאורעות $A, B$ הם בלתי־תלויים אם ורק אם הם זרים על־ידי דוגמה נגדית.
\begin{solution}
	נגדיר $\Omega$ מאורע של הטלת קוביה, $A = \{ 1, 2 \}, B = \{ 1, 3 \}$ ונגדיר שהקוביה לא הוגנת ו־$1$ לא יכול לצאת (השאר בהסתברות שווה), אז $A \cap B \ne \emptyset$ ולכן הם לא זרים, אבל
	\[
		\PP(A \cap B) = \PP(\{ 1 \}) = 0 \ne \frac{1}{5} \cdot \frac{1}{5} = \PP(A) \PP(B)
	\]
\end{solution}

\Subquestion{}
נסתור את הטענה כי אם $A$ ו־$B$ בלתי־תלויים וגם $B$ ו־$C$ בלתי־תלויים אז $A$ ו־$C$ בלתי־תלויים.
\begin{solution}
	נגדיר $\Omega$ הטלת שתי קוביות, $A$ המאורע שיצא 1 בקוביה א', $B$ המאורע שיצא 1 בקוביה ב' ו־$C = A$, אז
	\[
		\PP(A \cap B) = \frac{1}{36} = \frac{1}{6} \cdot \frac{1}{6} = \PP(A) \cdot \PP(B)
	\]
	לכן $A, B$ בלתי־תלויים, ולכן גם $B, C$ בלתי־תלויים, אבל $A = C$ והם כמובן לא בלתי־תלויים:
	\[
		\PP(A) = \frac{1}{6} \ne \frac{1}{36} = \PP^2(A)
	\]
\end{solution}

\Subquestion{}
נוכיח שאם $A$ בלתי־תלוי בעצמו, אז $\PP(A) = 0$ או $\PP(A) = 1$.
\begin{proof}
	נניח בשלילה ש־$\PP(A) \notin \{0, 1\}$ ולכן
	\[
		\PP(A \cap A) = \PP^2(A)
		\iff
		1 = \PP(A)
	\]
	וקיבלנו סתירה.
\end{proof}

\Subquestion{}
אם $A, B$ בלתי־תלויים אז גם $A^C, B^C$ בלתי־תלויים.
\begin{solution}
	ראינו בכיתה כי גם $A$ ו־$B^C$ בלתי־תלויים, ומאותה טענה גם $A^C$ ו־$B^C$ בלתי־תלויים.
\end{solution}

\Subquestion{}
נסתור את הטענה כי אם $A, B, C$ בלתי־תלויים אז גם $A \cup B$ ו־$C$ בלתי־תלויים.
\begin{solution}
	נניח $A$ קוביה א' יצאה 1, $B$ קוביה ב' יצאה 1 ו־$C = A^C$ אז כמובן $A, B, C$ בלתי־תלויים אבל
	\[
		\PP(C \cap (A \cup B)) = 0 \ne \frac{5}{6} \cdot \frac{1}{36} = \PP(C) \cdot \PP(A \cup B)
	\]
\end{solution}

\Subquestion{}
אם $A, B$ בלתי־תלויים וכן $A, C$ בלתי־תלויים אז $A, B \cup C$ בלתי תלויים.
\begin{solution}
	נניח $\Omega$ הטלת שתי קוביות, נניח גם $A$ שיצא סכום 7, ו־$B$ שיצא 4 בקוביה הראשונה ו־$C$ שיצא 4 בקוביה השניה. \\*
	ראינו בהרצאה כי $A, B$ וגם $A, C$ שני זוגות מאורעות בלתי־תלויים, אבל
	\[
		\PP(A \cap (B \cup C)) = \frac{2}{36} \ne \frac{1}{6} \cdot \frac{6 + 6 - 1}{36} = \PP(A) \PP(B \cup C)
	\]
\end{solution}

\Subquestion{}
נוכיח שאם $A$ ו־$B$ בלתי־תלויים וכן $A$ ו־$C$ בלתי־תלויים וגם $B, C$ זרים אז $A$ ו־$B \cup C$ בלתי־תלויים.
\begin{proof}
	נבחין כי
	\begin{align*}
		\PP(A \cap (B \uplus C))
		& = \PP((A \cap B) \uplus (A \cap C)) \\
		& = \PP(A \cap B) + \PP(A \cap C) \\
		& = \PP(A) \PP(B) + \PP(A) \PP(C) \\
		& = \PP(A)(\PP(B) + \PP(C)) \\
		& = \PP(A) \PP(A \cup C)
	\end{align*}
\end{proof}

\Question{}
נוכיח שהמאורעות $A_1, \dots, A_n$ הם בלתי תלויים אם ורק אם לכל $I \subseteq [n]$ מתקיים
\[
	\PP( (\bigcap_{i \in I} A_i) \cap (\bigcap_{i \in [n] \setminus I} A_i^C))
	= (\prod_{i \in I} \PP(A_i)) \cdot (\prod_{i \in [n] \setminus I} \PP(A_i^C))
\]
\begin{proof}
	נניח ש־$A_1, \dots, A_n$ בלתי תלויים ולכן מטענה מהכיתה גם $A_1, \dots, A_n, A_1^C, \dots, A_n^C$ בלתי־תלויים. \\*
	יהי $I \subseteq [n]$ ונגדיר $J = I \cup \{ n + j \mid j \in [n] \land j \notin I \}$ ונקבל מאי־התלות ישירות
	\[
		\PP( (\bigcap_{i \in I} A_i) \cap (\bigcap_{i \in [n] \setminus I} A_i^C))
		= (\prod_{i \in I} \PP(A_i)) \cdot (\prod_{i \in [n] \setminus I} \PP(A_i^C))
	\]

	נניח את כיוון השני. \\*
	יהי $i \in [n]$ כלשהו ונרצה להראות כי $A_i$ בלתי־תלוי ב־$\{ A_j \mid 1 \le j \le n \} \setminus A_i$.
\end{proof}

\end{document}
