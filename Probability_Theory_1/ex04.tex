\documentclass[a4paper]{article}

% packages
\usepackage{inputenc, fontspec, amsmath, amsthm, amsfonts, polyglossia, catchfile}
\usepackage[a4paper, margin=50pt, includeheadfoot]{geometry} % set page margins

% style
\AddToHook{cmd/section/before}{\clearpage}	% Add line break before section
\linespread{1.5}
\setcounter{secnumdepth}{0}		% Remove default number tags from sections
\setmainfont{Libertinus Serif}
\setsansfont{Libertinus Sans}
\setmonofont{Libertinus Mono}
\setdefaultlanguage{hebrew}
\setotherlanguage{english}

% operators
\DeclareMathOperator\cis{cis}
\DeclareMathOperator\Sp{Sp}
\DeclareMathOperator\tr{tr}
\DeclareMathOperator\im{Im}
\DeclareMathOperator\diag{diag}
\DeclareMathOperator*\lowlim{\underline{lim}}
\DeclareMathOperator*\uplim{\overline{lim}}

% commands
\renewcommand\qedsymbol{\textbf{משל}}
\newcommand{\NN}[0]{\mathbb{N}}
\newcommand{\ZZ}[0]{\mathbb{Z}}
\newcommand{\QQ}[0]{\mathbb{Q}}
\newcommand{\RR}[0]{\mathbb{R}}
\newcommand{\CC}[0]{\mathbb{C}}
\newcommand{\getenv}[2][] {
  \CatchFileEdef{\temp}{"|kpsewhich --var-value #2"}{\endlinechar=-1}
  \if\relax\detokenize{#1}\relax\temp\else\let#1\temp\fi
}
\newcommand{\explain}[2] {
	\begin{flalign*}
		 && \text{#2} && \text{#1}
	\end{flalign*}
}

% headers
\getenv[\AUTHOR]{AUTHOR}
\author{\AUTHOR}
\date\today

\title{פתרון מטלה 04 --- תורת ההסתברות (1), 80420}

\begin{document}
\maketitle
\maketitleprint{}

\Question{}
יהי $(\Omega, \PP)$ מרחב הסתברות.
נוכיח או נפריך את הטענות הבאות.

\Subquestion{}
נסתור את הטענה כי שני מאורעות $A, B$ הם בלתי־תלויים אם ורק אם הם זרים על־ידי דוגמה נגדית.
\begin{solution}
	נגדיר $\Omega$ מאורע של הטלת קוביה, $A = \{ 1, 2 \}, B = \{ 1, 3 \}$ ונגדיר שהקוביה לא הוגנת ו־$1$ לא יכול לצאת (השאר בהסתברות שווה), אז $A \cap B \ne \emptyset$ ולכן הם לא זרים, אבל
	\[
		\PP(A \cap B) = \PP(\{ 1 \}) = 0 \ne \frac{1}{5} \cdot \frac{1}{5} = \PP(A) \PP(B)
	\]
\end{solution}

\Subquestion{}
נסתור את הטענה כי אם $A$ ו־$B$ בלתי־תלויים וגם $B$ ו־$C$ בלתי־תלויים אז $A$ ו־$C$ בלתי־תלויים.
\begin{solution}
	נגדיר $\Omega$ הטלת שתי קוביות, $A$ המאורע שיצא 1 בקוביה א', $B$ המאורע שיצא 1 בקוביה ב' ו־$C = A$, אז
	\[
		\PP(A \cap B) = \frac{1}{36} = \frac{1}{6} \cdot \frac{1}{6} = \PP(A) \cdot \PP(B)
	\]
	לכן $A, B$ בלתי־תלויים, ולכן גם $B, C$ בלתי־תלויים, אבל $A = C$ והם כמובן לא בלתי־תלויים:
	\[
		\PP(A) = \frac{1}{6} \ne \frac{1}{36} = \PP^2(A)
	\]
\end{solution}

\Subquestion{}
נוכיח שאם $A$ בלתי־תלוי בעצמו, אז $\PP(A) = 0$ או $\PP(A) = 1$.
\begin{proof}
	נניח בשלילה ש־$\PP(A) \notin \{0, 1\}$ ולכן
	\[
		\PP(A \cap A) = \PP^2(A)
		\iff
		1 = \PP(A)
	\]
	וקיבלנו סתירה.
\end{proof}

\Subquestion{}
אם $A, B$ בלתי־תלויים אז גם $A^C, B^C$ בלתי־תלויים.
\begin{solution}
	ראינו בכיתה כי גם $A$ ו־$B^C$ בלתי־תלויים, ומאותה טענה גם $A^C$ ו־$B^C$ בלתי־תלויים.
\end{solution}

\Subquestion{}
נסתור את הטענה כי אם $A, B, C$ בלתי־תלויים אז גם $A \cup B$ ו־$C$ בלתי־תלויים.
\begin{solution}
	נניח $A$ קוביה א' יצאה 1, $B$ קוביה ב' יצאה 1 ו־$C = A^C$ אז כמובן $A, B, C$ בלתי־תלויים אבל
	\[
		\PP(C \cap (A \cup B)) = 0 \ne \frac{5}{6} \cdot \frac{1}{36} = \PP(C) \cdot \PP(A \cup B)
	\]
\end{solution}

\Subquestion{}
אם $A, B$ בלתי־תלויים וכן $A, C$ בלתי־תלויים אז $A, B \cup C$ בלתי תלויים.
\begin{solution}
	נניח $\Omega$ הטלת שתי קוביות, נניח גם $A$ שיצא סכום 7, ו־$B$ שיצא 4 בקוביה הראשונה ו־$C$ שיצא 4 בקוביה השניה. \\*
	ראינו בהרצאה כי $A, B$ וגם $A, C$ שני זוגות מאורעות בלתי־תלויים, אבל
	\[
		\PP(A \cap (B \cup C)) = \frac{2}{36} \ne \frac{1}{6} \cdot \frac{6 + 6 - 1}{36} = \PP(A) \PP(B \cup C)
	\]
\end{solution}

\Subquestion{}
נוכיח שאם $A$ ו־$B$ בלתי־תלויים וכן $A$ ו־$C$ בלתי־תלויים וגם $B, C$ זרים אז $A$ ו־$B \cup C$ בלתי־תלויים.
\begin{proof}
	נבחין כי
	\begin{align*}
		\PP(A \cap (B \uplus C))
		& = \PP((A \cap B) \uplus (A \cap C)) \\
		& = \PP(A \cap B) + \PP(A \cap C) \\
		& = \PP(A) \PP(B) + \PP(A) \PP(C) \\
		& = \PP(A)(\PP(B) + \PP(C)) \\
		& = \PP(A) \PP(A \cup C)
	\end{align*}
\end{proof}

\Question{}
נוכיח שהמאורעות $A_1, \dots, A_n$ הם בלתי תלויים אם ורק אם לכל $I \subseteq [n]$ מתקיים
\[
	\PP( (\bigcap_{i \in I} A_i) \cap (\bigcap_{i \in [n] \setminus I} A_i^C))
	= (\prod_{i \in I} \PP(A_i)) \cdot (\prod_{i \in [n] \setminus I} \PP(A_i^C))
\]
\begin{proof}
	נניח ש־$A_1, \dots, A_n$ בלתי תלויים ולכן מטענה מהכיתה גם $A_1, \dots, A_n, A_1^C, \dots, A_n^C$ בלתי־תלויים. \\*
	יהי $I \subseteq [n]$ ונגדיר $J = I \cup \{ n + j \mid j \in [n] \land j \notin I \}$ ונקבל מאי־התלות ישירות
	\[
		\PP( (\bigcap_{i \in I} A_i) \cap (\bigcap_{i \in [n] \setminus I} A_i^C))
		= (\prod_{i \in I} \PP(A_i)) \cdot (\prod_{i \in [n] \setminus I} \PP(A_i^C))
	\]

	נניח את כיוון השני. \\*
	נוכיח את הטענה באינדוקציה על $n$. במקרה $n = 1$ הוא טריוויאלי ולכן נניח שהקבוצה $A_1, \dots, A_{n - 1}$ היא בלתי־תלויה ומקיימת את השוויון ונבדוק את הקבוצה $A_1, \dots, A_n$ אשר מקיימת את השוויון גם כן. \\*
	נבדוק
	\[
		\PP(A_n^C \cap \bigcap_{i \in [n - 1]} A_i)
		= \PP(A_n^C) \cdot \prod_{i \in [n - 1]} \PP(A_i)
	\]
	ולכן הקבוצה $\{A_1, \dots, A_{n - 1}, A_n^C\}$ בלתי־תלויה, ולכן גם הקבוצה $\{A_1, \dots, A_n\}$ והשלמנו את מהלך האינדוקציה.
\end{proof}

\Question{}
שךושה שופטים מכריעים את גורלו של נאשם לפי דעת רוב, שניים מהשופטים צודקים בהחלטתם בסיכוי $0.9$ והשופט השלישי לא מנוסה וצודק בסבירות של $0.51$.
נבדוק את ההסתברות לפסק דין נכון במקרים שונים.

\Subquestion{}
נניח כי החלטת כל שופט היא בלתי־תלויה.
\begin{solution}
	נסמן $A_1, A_2$ המאורעות שהשופטים המנוסים צדקו ו־$A_3$ המאורע שהשופט הלא מנוסה צדק. \\*
	נתון לנו שמתקיים $\PP(A_1) = \PP(A_2) = \frac{9}{10}$ וגם $\PP(A_3) = \frac{51}{100}$. \\*
	אנו מחפשים את הסיכוי לרוב, לכן נבחן את
	\begin{align*}
		& \PP((A_1 \cap A_2 \cap A_3) \cup (A_1 \cap A_2 \cap A_3^C) \cup (A_1^C \cap A_2 \cap A_3) \cup (A_1 \cap A_2^C \cap A_3)) \\
		= & \PP(A_1 \cap A_2 \cap A_3) + \PP(A_1^C \cap A_2 \cap A_3) + \PP(A_1 \cap A_2^C \cap A_3) + \PP(A_1 \cap A_2 \cap A_3^C) \\
		= & \PP(A_1) \PP(A_2) \PP(A_3) + \PP(A_1^C) \PP(A_2) \PP(A_3) + \PP(A_1) \PP(A_2^C) \PP(A_3) + \PP(A_1) \PP(A_2) \PP(A_3^C) \\
		= & \frac{9}{10} \cdot \frac{9}{10} \cdot \frac{51}{100} + 2 \cdot \frac{1}{10} \cdot \frac{9}{10} \cdot \frac{51}{100} + \frac{9}{10} \cdot \frac{9}{10} \cdot \frac{49}{100}
	\end{align*}
\end{solution}

\Subquestion{}
השופטים המנוסים מחליטים באופן בלתי־תלוי והשופט הלא מנוסה בוחר אחד מהם באקראי ומצטרף לדעתו.
\begin{solution}
	נבחין כי הפעם בהינתן המידע החדש מתקיים
	\begin{align*}
		& \PP(A_1 \cap A_2 \cap A_3) + \PP(A_1^C \cap A_2 \cap A_3) + \PP(A_1 \cap A_2^C \cap A_3) + \PP(A_1 \cap A_2 \cap A_3^C) \\
		= & \PP(A_1 \cap A_2) + 2 \PP(A_1^C \cap A_2 \cap A_3) + 0 \\
		= & \PP(A_1) \PP(A_2) + 2 \PP(A_1^C \cap A_2 \cap A_3) \\
	\end{align*}
	במקרה $A_1^C \cap A_2 \cap A_3$ הוא המקרה ששופט אחד צדק והשני טעה מבין המנוסים, והשופט הלא מנוסה בחר אחד מהם בהסתברות אחידה, לכן במקרה זה $\PP(A_3 \mid A_1^C \cap A_2) = \frac{1}{2}$ ולכן נקבל
	\begin{align*}
		\PP(A_1) \PP(A_2) + 2 \PP(A_1^C \cap A_2 \cap A_3)
		& = \frac{9}{10} \cdot \frac{9}{10} + 2 \PP(A_1^C \cap A_2) \cdot \frac{1}{2} \\
		& = \frac{9}{10} \cdot \frac{9}{10} + \PP(A_1^C) \PP(A_2) \\
		& = \frac{9}{10} \cdot \frac{9}{10} + \frac{1}{10} \cdot \frac{9}{10}
	\end{align*}
\end{solution}

\Question{}
נתבונן במרחב המדגם $\Omega = [5]$ עם הסתברות אחידה, \\*
נמצא משתנים מקריים $X, Y$ כך ש־$X, Y, X + Y$ כולם מתפלגים אחיד על $\{-2, -1, 0, 1, 2\}$.
\begin{solution}
	נגדיר $X(\omega) = \omega - 3$. \\*
	עוד נגדיר $Y(1) = 2, Y(2) = -1, Y(3) = 1, Y(4) = -2, Y(5) = 0$. \\*
	בהתאם נקבל $(X + Y)(1) = 0, (X + Y)(2) = -2, (X + Y)(3) = 1, (X + Y)(4) = -1, (X + Y)(5) = 2$. \\*
	נבחין כי גם $X$ גם $Y$ וגם $X + Y$ הן חד־חד ערכיות ועל $[5] \to \{-2, -1, 0, 1, 2\}$ ולכן מההסתברות האחידה של $\Omega$ נסיק התפלגות אחידה.
\end{solution}

\Question{}
נתונים זוג קוביות הוגנות בעלות שש פאות, על פאות קוביה א' כתוב $\{1, 2, 2, 3, 3, 4\}$, ועל פאות קוביה ב' כתוב $\{1, 3, 5, 6, 8\}$. \\*
נוכיח כי התפלגות סכום הקוביות זהה להתפלגות סכומן של שתי קוביות משחק רגילות.
\begin{proof}
	נגדיר $X$ משתנה מקרי המתאר את סכום הקוביות המיוחדות ו־$Y$ סכום הקוביות הרגילות ונבדוק את כלל ערכי ההתפלגות. \\*
	נבחין כי אלו הם משתנים מקריים בדידים ולכן מספיק לבדוק כל ערך בנפרד.

	עבור $n < 2$ נקבל $\PP_X'(n) = \PP_Y(n) = 0$ שכן אין סכום קוביות שמוביל לערך זה. \\*
	באופן דומה נקבל שגם עבור $n > 12$ נקבל ש־$\PP_Y(n) = 0$ ואנו יודעים כי הסכום המקסימלי בקוביות המיוחדות הוא $12$ ולכן נסיק ישירות שגם $\PP_X'(n) = 0$. \\*
	נראה כי $\PP_Y(2) = \PP(\{(a, b) \in \Omega \mid a + b = 2\}) = \PP(\{ (1, 1) \}) = \PP'(\{ (1, 1) \}) = \PP_X(2)$. \\*
	באופן דומה $\PP_Y(3) = \frac{2}{36} = \PP_X(3)$ על־ידי הצירופים $(1, 2)$ ו־$(1, 2)$ נוסף.
	נוכל להמשיך את החישוב הידני ונקבל שההסתברות אכן שווה בכל מקרה, ולכן הקוביות אכן שקולות התפלגות.
\end{proof}

\Question{}
במגירה יש $n$ זוגות גרביים, אדם מוציא גרב מהמגירה שוב ושוב עד שיש ברשותו זוג תואם. \\*
ננתח את ההתפלגות של מספר הגרביים שהוצאו.
\begin{solution}
	נגדיר את המשתנה המקרי $X$ כך שייצג את מספר הגרביים שהוצאו מהמגירה, נבחין כי מתקיים
	\[
		\PP(X = k) = 1 - \frac{2n}{2n} \cdot \frac{n - 1}{2n - 1} \cdots \frac{n - k}{2n - k} = 1 - 2 \frac{n!}{(n - k + 1)!} \cdot \frac{(2n - k + 1)!}{(2n)!}
	\]
	כאשר עבור $X(n) = 1 - 0 = 1$. \\*
	נסביר, ההסתברות עבור $X = k$ היא המשלים להסתברות של המאורע שכולל את כל הסדרים של הוצאת גרביים ב־$k$ המקומות הראשונים, ביחס למאורע שכל $k$ הגרביים הראשונים הם מקבוצה של בודדים מהזוגות (בגודל $n$).
\end{solution}

\Question{}
ניתן דוגמה למרחב הסתברות ולמשתנים מקריים $X, Y$ ופונקציה $f : \RR \to \RR$ כך שיתקיים
\[
	X \overset{a.s.}{\ne} Y,
	\qquad
	f(X) \ne f(Y),
	\qquad
	f(X) \overset{a.s.}{=} f(Y)
\]
\begin{solution}
	נגדיר $\Omega = [6]$ מרחב הסתברות של הטלת קוביה לא הוגנת כך ש־$p(1) = 0, p(n) = \frac{1}{5}$. \\*
	עוד נגדיר $X = \{ (1, 1), (2, 2), (3, 3), (4, 4), (5, 6), (6, 5) \}, Y = \{ (1, -1), (2, 2), (3, 3), (4, 4), (5, 5), (6, 6) \}$ \\*
	נקבל אם כך ש־$\PP(X = Y) = \PP(\{ \omega \in \Omega \mid X(\omega) = Y(\omega) \}) = p(2) + p(3) + p(4) = \frac{3}{5} \ne 1$ ולכן $X \overset{a.s.}{\ne} Y$. \\*
	נגדיר $f$ על־ידי $f(x) = \begin{cases}
		6 & x = 5 \\
		x & x \ne 5
	\end{cases}$.
	עתה נבחין כי $f(X) = \{ (1, 1), (2, 2), (3, 3), (4, 4), (5, 6), (6, 6) \}$, \\*
	וכן $f(Y) = \{ (1, -1), (2, 2), (3, 3), (4, 4), (5, 6), (6, 6) \}$.
	לכן גם $f(X) \ne f(Y)$. \\*
	לבסוף מתקיים $\PP(X = Y) = p(2) + p(3) + p(4) + p(5) + p(6) = \frac{5}{5} = 1$ ולכן $f(X) \overset{a.s.}{=} f(Y)$.
\end{solution}

\end{document}
