\documentclass[a4paper]{article}

% packages
\usepackage{inputenc, fontspec, amsmath, amsthm, amsfonts, polyglossia, catchfile}
\usepackage[a4paper, margin=50pt, includeheadfoot]{geometry} % set page margins

% style
\AddToHook{cmd/section/before}{\clearpage}	% Add line break before section
\linespread{1.5}
\setcounter{secnumdepth}{0}		% Remove default number tags from sections
\setmainfont{Libertinus Serif}
\setsansfont{Libertinus Sans}
\setmonofont{Libertinus Mono}
\setdefaultlanguage{hebrew}
\setotherlanguage{english}

% operators
\DeclareMathOperator\cis{cis}
\DeclareMathOperator\Sp{Sp}
\DeclareMathOperator\tr{tr}
\DeclareMathOperator\im{Im}
\DeclareMathOperator\diag{diag}
\DeclareMathOperator*\lowlim{\underline{lim}}
\DeclareMathOperator*\uplim{\overline{lim}}

% commands
\renewcommand\qedsymbol{\textbf{משל}}
\newcommand{\NN}[0]{\mathbb{N}}
\newcommand{\ZZ}[0]{\mathbb{Z}}
\newcommand{\QQ}[0]{\mathbb{Q}}
\newcommand{\RR}[0]{\mathbb{R}}
\newcommand{\CC}[0]{\mathbb{C}}
\newcommand{\getenv}[2][] {
  \CatchFileEdef{\temp}{"|kpsewhich --var-value #2"}{\endlinechar=-1}
  \if\relax\detokenize{#1}\relax\temp\else\let#1\temp\fi
}
\newcommand{\explain}[2] {
	\begin{flalign*}
		 && \text{#2} && \text{#1}
	\end{flalign*}
}

% headers
\getenv[\AUTHOR]{AUTHOR}
\author{\AUTHOR}
\date\today

\title{פתרון ממ''ן 12 – חשבון אינפיניטסימלי 2 (20475)}

\begin{document}
\maketitle
\maketitleprint{}
\section{שאלה 1}
נחשב את האינטגרלים הבאים

\subsection{סעיף א'}
\subsubsection{דרך א'}
\[
	\int x^3{(1 - 3x^2)}^{10} dx
\]
נשתמש בכלל ההצבה עבור $t = 3x^2, dt = 6x \, dx$:
\[
	\int x^3{(1 - 3x^2)}^{10} dx
	= \int \frac{1}{18} t {(1 - t)}^{10} dt
	= \frac{1}{18} \int t {(t - 1)}^{10} dt
\]
נשתמש שוב בכלל ההצבה עבור $u = t - 1, du = dt$:
\begin{align*}
	\frac{1}{18} \int t {(t - 1)}^{10} dt
	& = \frac{1}{18} \int (u + 1) u^{10} du
	= \frac{1}{18} ( \frac{1}{12} u^{12} + \frac{1}{11} u^{11} ) + C \\
	& = \frac{1}{216} {(t - 1)}^{12} + \frac{1}{198} {(t - 1)}^{11} + C
	= \frac{1}{216} {(t - 1)}^{12} + \frac{1}{198} {(t - 1)}^{11} + C \\
	& = \frac{1}{216} {(3x^2 - 1)}^{12} + \frac{1}{198} {(3x^2 - 1)}^{11} + C
\end{align*}
ומצאנו כי
\[
	\int x^3{(1 - 3x^2)}^{10} dx = \frac{1}{216} {(3x^2 - 1)}^{12} + \frac{1}{198} {(3x^2 - 1)}^{11} + C
\]

\subsubsection{דרך ב'}
דרך נוספת אפשרית שאשמח לקבל עליה ביקורת.
\begin{flalign*}
	&& \int x^3{(1 - 3x^2)}^{10} dx
	& = \int x^3 \left( \sum_{k = 0}^{10} \binom{10}{k} {(-3x^2)}^k \right) dx && & \text{פירוק בינומי} \\
	&& & = \int \left( \sum_{k = 0}^{10} \binom{10}{k} {(-3)}^k x^{2k + 3} \right) dx && & \text{כינוס מקדמים} \\
	&& & = C + \sum_{k = 0}^{10} \binom{10}{k} \frac{1}{2k + 4} {(-3)}^k x^{2k + 4} && & \text{נחשב איבר איבר} \\
	&& & = C + x^4 \sum_{k = 0}^{10} \binom{10}{k} \frac{{(-3)}^k}{2k + 4} x^{2k}
\end{flalign*}

\subsection{סעיף ב'}
\[
	\int \frac{x e^{\arcsin x}}{\sqrt{1 - x^2}}dx
\]
נשים לב כי לכל $x$ בתחום ההגדרה של הפונקציה מתקיים $x = \sin \arcsin x$ ולכן
\[
	\int \frac{x e^{\arcsin x}}{\sqrt{1 - x^2}}dx
	= \int \frac{\sin \arcsin(x) e^{\arcsin x}}{\sqrt{1 - x^2}}dx
\]
נשתמש בכלל ההצבה עבור $t = \arcsin x, dt = \frac{dx}{\sqrt{1 - x^2}}$:
\[
	\int \frac{\sin \arcsin(x) e^{\arcsin x}}{\sqrt{1 - x^2}}dx
	= \int \sin(t) e^t dt
\]
נשתמש באינטגרציה בחלקים עבור $u' = \sin t, v = e^t$:
\[
	\int \sin(t) e^t dt
	= -\cos(t) e^t + \int \cos(t) e^t dt
\]
ושוב אינטגרציה בחלקים עבור $u' = \cos t, v = e^t$:
\[
	-\cos(t) e^t + \int \cos(t) e^t dt
	= -\cos(t) e^t + \sin(x) e^t - \int \sin(t) e^t dt
	= \int \sin(t) e^t dt
\]
ומהשוויון נובע
\[
	\int \sin(t) e^t
	= \frac{1}{2} (\sin(t) - \cos(t)) e^t + C
	= \frac{1}{2} (\sin(\arcsin x) - \cos(\arcsin x)) e^{\arcsin x} + C
\]
ועל־פי זהויות טריגונומטריות
\[
	\int \frac{x e^{\arcsin x}}{\sqrt{1 - x^2}}dx
	= \frac{1}{2} (x - \sqrt{1 - x^2}) e^{\arcsin x} + C
\]

\subsection{סעיף ג'}
\begin{flalign*}
	\int {(x - 1)}^2 e^{2x} dx
	& = \frac{1}{2} e^{2x} {(x - 1)}^2 - \int \frac{1}{2} e^{2x} (2x - 2) dx
	& u' = e^{2x}, v = {(x - 1)}^2
	&& \text{אינטגרציה בחלקים עבור} \\
	& = \frac{1}{2} e^{2x} {(x - 1)}^2 + \int e^{2x} dx - \int e^{2x} x dx \\
	& = \frac{1}{2} e^{2x} {(x - 1)}^2 + \frac{1}{2} e^{2x} - \frac{1}{2} e^{2x} x + \int \frac{1}{2} e^{2x} dx
	& u' = e^{2x}, v = x
	&& \text{אינטגרציה בחלקים עבור} \\
	& = \frac{1}{2} e^{2x} (x^2 - 3x + 2) + \frac{1}{4} e^{2x} + C \\
	& = \frac{1}{2} e^{2x} (x^2 - 3x + 2\frac{1}{2}) + C
\end{flalign*}

\subsection{סעיף ד'}
\[
	\int \frac{4x + 1}{\sqrt{{(x^2 + 4x + 5)}^3}} dx
\]
נשתמש בכלל ההצבה עבור $t = x + 2, dt + dx$
\[
	\int \frac{4x + 1}{\sqrt{{(x^2 + 4x + 5)}^3}} dx
	= \int \frac{4t - 7}{\sqrt{{(t^2 + 1)}^3}} dt \\
	= 2\int \frac{2t}{\sqrt{{(t^2 + 1)}^3}} dt - 7\int \frac{1}{\sqrt{{(t^2 + 1)}^3}} dt \\
\]
נחשב את האינטגרלים המתקבלים בנפרד:
\[
	\int \frac{2t}{\sqrt{{(t^2 + 1)}^3}} dt
\]
נשתמש בכלל ההצבה עבור $u = t^2 + 1, du = 2t \, dt$:
\[
	\int \frac{2t}{\sqrt{{(t^2 + 1)}^3}} dt
	= \int \frac{du}{u^{3/2}}
	= \int u^{-3/2} du
	= -2 u^{-1/2}
	= \frac{-2}{\sqrt{t^2 + 1}}
\]
עבור האינטגרל
\[
	\int \frac{1}{\sqrt{{(t^2 + 1)}^3}} dt
\]
נשתמש בכלל ההצבה עבור $t = \tan u, dt = \frac{du}{\cos^2 u}$ ונקבל
\[
	\int \frac{1}{\sqrt{{(t^2 + 1)}^3}} dt
	= \int \frac{1}{\cos^2 u \sqrt{{(\tan^2 u + 1)}^3}} du
	= \int \frac{1}{\cos^2 u \sqrt{{(\cos^{-2})}^3}} du
	= \sin u = \sin \arctan t + C
	= \frac{t}{\sqrt{t^2 + 1}} + C
\]
ולכן עבור האינטגרל המקורי נקבל
\[
	2\int \frac{2t}{\sqrt{{(t^2 + 1)}^3}} dt - 7\int \frac{1}{\sqrt{{(t^2 + 1)}^3}} dt \\
	= 2 \frac{-2}{\sqrt{t^2 + 1}} - 7 \frac{t}{\sqrt{t^2 + 1}}
	= \frac{-7t - 4}{\sqrt{t^2 + 1}}
	= \frac{-7x - 18}{\sqrt{x^2 + 4x + 5}} + C
\]
נציב ונחשב
\[
	\int_{-2}^{-1} \frac{4x + 1}{\sqrt{{(x^2 + 4x + 5)}^3}} dx
	= \left. \frac{-7x - 18}{\sqrt{x^2 + 4x + 5}} \right|_{-2}^{-1}
	= \frac{-7(-1) - 18}{\sqrt{{(-1)}^2 + 4 (-1) + 5}} - \frac{-7(-2) - 18}{\sqrt{{(-2)}^2 + 4 (-2) + 5}}
	= 4 - \frac{11}{\sqrt{2}}
\]

\subsection{סעיף ה'}
\[
	\int_{-\ln 2}^{\ln 2} e^{|x|} (1 + x^3 + \sin x) dx
	= \int_{-\ln 2}^{\ln 2} e^{|x|} (x^3 + \sin x) dx
	+ \int_{-\ln 2}^{\ln 2} e^{|x|} dx
\]
נגדיר
\[
	f(x) = e^{|x|} (1 + \sin x)
\]
מחישוב עולה כי
\[
	f(-x) = e^{|-x|} (-x^3 - \sin x) = - f(x)
\]
ולכן
\[
	\int_{-\ln 2}^{\ln 2} f(x) dx
	= \int_{-\ln 2}^{0} f(x) dx + \int_{0}^{\ln 2} f(x) dx
	= -\int_{0}^{\ln 2} f(x) dx + \int_{0}^{\ln 2} f(x) dx
	= 0
\]
עוד נראה כי
\[
	\int_{-\ln 2}^{\ln 2} e^{|x|} dx
	= \int_{-\ln 2}^{0} e^{|x|} dx + \int_{0}^{\ln 2} e^{|x|} dx
	= \int_{-\ln 2}^{0} e^{-x} dx + \int_{0}^{\ln 2} e^{x} dx
	= 2\int_{0}^{\ln 2} e^{x} dx
	= 2 (e^{\ln 2} - 1) = 2
\]
ולכן
\[
	\int_{-\ln 2}^{\ln 2} e^{|x|} (1 + x^3 + \sin x) dx = 2
\]

\section{שאלה 2}
עבור כל אחד מהאינטגרלים הבאים נקבע אם הוא מתכנס בהחלט, בתנאי, או כלל לא.

\subsection{סעיף א'}
\[
	\int_0^\infty \frac{\cos^2 x}{\sqrt{x}}dx
\]
ידוע כי $\sin^2 x + \cos^2 x = 1$ ולכן
\[
	\int_0^\infty \frac{\cos^2 x}{\sqrt{x}}dx + \int_0^\infty \frac{\sin^2 x}{\sqrt{x}}dx
	= \int_0^\infty \frac{1}{\sqrt{x}}dx
	= \lim_{x \to \infty} (2\sqrt{x} - 0)
	= \infty
\]
אבל גם $\cos^2x = \sin^2(x + \pi/2)$, לכן שני האינטגרלים המחוברים מתבדרים יחד או מתכנסים יחד, ואנו יודעים כי חיבורם מתבדר, ולכן גם
\[
	\int_0^\infty \frac{\cos^2 x}{\sqrt{x}}dx = \infty
\]
דהינו האינטגרל מתבדר.

\subsection{סעיף ב'}
\[
	\int_1^\infty \frac{x \ln x}{(x^2 - 1){(\ln(x + 1))}^3} dx
\]
נשים לב כי
\[
	\lim_{x \to 1^+} \frac{x \ln x}{(x^2 - 1){(\ln(x + 1))}^3} 
	\overset{(1)}{=} \lim_{x \to 1^+} \frac{\ln x}{(x^2 - 1) \ln^3 2} 
	= \lim_{x \to 1^+} \frac{1/x}{2x \ln^3 2} 
	= \frac{1}{2\ln^3 2} 
\]
$(1)$ רציפות \\*
ולכן כל אינטגרל מהצורה
\[
	\int_1^k \frac{x \ln x}{(x^2 - 1){(\ln(x + 1))}^3} dx
\]
כאשר $k > 1$ הוא אינטגרל מתכנס, שכן זהו אינטגרל מערכים סופיים. \\*
נגדיר $g(x) = \frac{\ln x}{x^2}$ ונראה כי הגבול
\[
	\lim_{x \to \infty} \frac{x \ln x}{(x^2 - 1){(\ln(x + 1))}^3}/g(x)
	= \lim_{x \to \infty} \frac{1}{x(x^2 - 1){(\ln(x + 1))}^3}
	= 0
\]
הוא גבול מתכנס לאפס,
ולכן ממשפט 3.16* נובע כי האינטגרל הנתון מתכנס אם ורק אם האינטגרל של $g(x)$ מתכנס. \\*
נחשב על־פי אינטגרציה בחלקים
\[
	\int_{k}^{\infty} g(x)
	= \frac{-\ln x}{x} - \int_{k}^{\infty} \frac{-1}{x^2}
	= \left. \frac{-\ln x - 1}{x^2} \right|_k^\infty
	= 0 - \frac{-\ln k - 1}{k^2}
\]
מצאנו כי האינטגרל מתכנס בהתאם למשפט ההשוואה הגבולי. \\*
נשים לב כי הפונקציה הנתונה חיובית לכל ערך בתחום, ולכן האינטגרל מתכנס בהחלט.

\subsection{סעיף ג'}
\[
	\int_0^\infty \frac{1}{x} (\sqrt[4]{x} + 1) \sin \left( 2 \sqrt{x} \right) dx
\]
בתחום $x \in (0, 1]$ מהזהות $t \ge \sin t$ והזהות על הגבול $\lim_{t \to 0} \sin t / t$ נובע כי $0 < \sin(2 \sqrt{x}) / \sqrt{x} \le 1$ לכל $x$ בתחום, % chktex 9
ולכן
\[
	0 < 
	\frac{1}{x} (\sqrt[4]{x} + 1) \sin \left( 2 \sqrt{x} \right)
	\le \frac{1}{\sqrt{x}} (\sqrt[4]{x} + 1) \cdot 1
	= \frac{1}{x^{3/4}} + \frac{1}{\sqrt{x}}
\]
מלמה 3.2 נובעת התכנסות האינטגרל
\[
	\int_{0}^{1} \frac{1}{x^{3/4}} + \frac{1}{\sqrt{x}} dx
\]
נשים לב כי כלל תנאי מבחן ההשוואה חלים ונובעת התכנסות האינטגרל
\[
	\int_0^1 \frac{1}{x} (\sqrt[4]{x} + 1) \sin \left( 2 \sqrt{x} \right) dx
\]
נגדיר
\[
	f(x) = \frac{1}{x^{3/4}} + \frac{1}{x^{1/4}}, g(x) = \frac{1}{\sqrt{x}} \sin(2 \sqrt{x})
\]
מאינפי 1 אנו למדים כי הפונקציה $f(x)$ היא מונוטונית יורדת, ומחישוב עולה כי מתקיים הגבול $\lim_{x \to \infty} f(x) = 0$. \\*
נחשב את האינטגרל
\[
	\int_{1}^{x} g(t) dt
\]
נשתמש בכלל ההצבה עבור $t = u^2, dt = 2u \, du$ ונקבל כי
\[
	\int_{1}^{x} g(t) dt
	= \int_{1}^{\sqrt{x}} \frac{1}{u} \sin(2u) \cdot 2u \, du
	= \int_{1}^{\sqrt{x}} 2 \sin(2u) \, du
	= \left. -\cos(2u) \right|_1^{\sqrt{x}}
	= -\cos(2 \sqrt{x}) + \cos(2)
\]
מצאנו כי האינטגרל חסום לכל $x \ge 1$, ולכן הפונקציות $f(x)$ ו־$g(x)$ מקיימות את תנאי מבחן דיריכלה ונובעת התכנסות האינטגרל
\[
	\int_1^\infty f(x) g(x) \, dx
	= \int_1^\infty \frac{1}{x} (\sqrt[4]{x} + 1) \sin \left( 2 \sqrt{x} \right) dx
\]
מאדיטיביות האינטגרל נובע כי שני האינטגרלים המתכנסים מקיימים
\[
	\int_0^1 \frac{1}{x} (\sqrt[4]{x} + 1) \sin \left( 2 \sqrt{x} \right) dx
	+ \int_1^\infty \frac{1}{x} (\sqrt[4]{x} + 1) \sin \left( 2 \sqrt{x} \right) dx
	= \int_0^\infty \frac{1}{x} (\sqrt[4]{x} + 1) \sin \left( 2 \sqrt{x} \right) dx
\]
ולכן האינטגרל מתכנס. \\*
נשים לב כי
\[
	\int_1^\infty \left\lvert \frac{1}{x} (\sqrt[4]{x} + 1) \sin \left( 2 \sqrt{x} \right) \right\rvert dx
	= \int_1^\infty \frac{1}{x} (\sqrt[4]{x} + 1) \left\lvert \sin \left( 2 \sqrt{x} \right) \right\rvert dx
\]
נבחין כי מתקיים בתחום
\[
	\frac{1}{x} (\sqrt[4]{x} + 1) \left\lvert \sin \left( 2 \sqrt{x} \right) \right\rvert
	\ge \frac{1}{\sqrt{x}} \left\lvert \sin \left( 2 \sqrt{x} \right) \right\rvert
\]
ומדוגמה 3.12 נוכל להסיק את התבדרות האינטגרל
\[
	\int_{1}^{\infty} \frac{1}{\sqrt{x}} \left\lvert \sin \left( 2 \sqrt{x} \right) \right\rvert dx
\]
ולכן ממבחן השוואה נובע כי האינטגרל
\[
	\int_1^\infty \frac{1}{x} (\sqrt[4]{x} + 1) \left\lvert \sin \left( 2 \sqrt{x} \right) \right\rvert dx
\]
לא מתכנס ובהתאם האינטגרל המקורי מתכנס בתנאי.

\section{שאלה 3}
תהי פונקציה $f$ אינטגרבילית בקטע $[0, a]$ לכל $a > 0$ ומתקיים $\lvert f(x) \rvert \le \sqrt{x}$ לכל $x \ge 0$. \\*
נגדיר $F(x) = \int_0^x f(t) dt$, ונוכיח כי האינטגרל המוגדר על־ידי
\[
	\int_0^\infty \frac{F(x)}{\pi + x^3} dx \tag{1}
\]
מתכנס.
\begin{proof}
	ידוע כי $\lvert f(x) \rvert \le \sqrt{x}$ לכל $x \ge 0$ ולכן ממבחן ההשוואה נובע אי השוויון
	\[
		F(x) = \int_0^x \lvert f(t) \rvert dt \le \int_0^x \sqrt{t} dt = \left. \frac{2}{3} \sqrt{t^3} \right|_0^x = \frac{2}{3} \sqrt{x^3}
	\]
	נקבל גם
	\[
		\frac{F(x)}{\pi + x^3}
		\le \frac{2 \sqrt{x^3}}{3 \pi + 3 x^3}
		< \frac{2 \sqrt{x^3}}{3 x^3}
		= \frac{2}{3 \sqrt{x^3}}
		\tag{2}
	\]
	מטענה 3.2 נובעת התכנסות האינטגרל
	\[
		\int_{1}^{\infty} \frac{2}{3 \sqrt{x^3}} dx
	\]
	ולכן ממבחן ההשוואה נובעת התכנסות האינטגרל
	\[
		\int_1^\infty \frac{F(x)}{\pi + x^3} dx
	\]
	כמו־כן נבחן את האינטגרל
	\[
		\int_0^1 \frac{F(x)}{\pi + x^3} dx
	\]
	זהו כמובן אינטגרל מסוים של פונקציה רציפה ולכן הוא מתכנס ובעל ערך סופי, ומתקיים
	\[
		\int_1^\infty \frac{F(x)}{\pi + x^3} dx + \int_0^1 \frac{F(x)}{\pi + x^3} dx
		= \int_0^\infty \frac{F(x)}{\pi + x^3} dx
	\]
	דהינו מצאנו כי האינטגרל $(1)$ מתכנס.
\end{proof}

\section{שאלה 4}
\subsection{סעיף א'}
תהי $f(x)$ פונקציה רציפה וחיובית בקטע $[a, \infty)$. \\* % chktex 9
נפריך את הטענה כי אם האינטגרל $\int_a^\infty f(x) dx$ מתכנס אז קיים $0 < c < 1$ כך שהאינטגרל
\[
	\int_a^\infty {(f(x))}^p dx
\]
מתכנס לכל $c \le p \le 1$.
\begin{proof}
	נגדיר
	\[
		f(x) = \frac{1}{x \ln^2 x}
	\]
	מתקיים
	\[
		\int f(x) dx
		= \int \frac{1}{x \ln^2 x} dx
	\]
	נבצע איטנגרציה בחלקים עבור $u' = \frac{1}{x}, v = \ln^{-2}x$:
	\[
		\int \frac{1}{x \ln^2 x} dx
		= \frac{1}{\ln x} - \int \ln(x) (-2) \ln^{-3}(x) \frac{1}{x} dx
		= \frac{1}{\ln x} + 2\int \frac{1}{x \ln^2 x} dx
	\]
	ואחרי העברת אגפים נקבל
	\[
		\int \frac{1}{x \ln^2 x} dx = \int f(x) dx = -\frac{1}{\ln x}
	\]
	ולכן גם
	\[
		\int_{e}^{\infty} f(x) = \left. -\frac{1}{\ln x} \right|_2^\infty = 1
	\]
	עתה נבחן את $f^p(x)$ כאשר $0 < p < 1$. \\*
	נגדיר את הפונקציה $g(x) = \frac{1}{x}$, כידוע
	\[
		\int_{e}^{\infty} g(x) dx = \left. \ln x \right|_e^\infty = \infty
	\]
	נבדוק את גבול חלוקת $f^p$ ב־$g$:
	\[
		\lim_{x \to \infty} \frac{f^p(x)}{g(x)}
		= \lim_{x \to \infty} \frac{x}{x^p \ln^{2p} x}
		= \lim_{x \to \infty} \frac{x^{1 - p}}{\ln^{2p} x}
		= \infty
	\]
	ולכן מהגרסה הגבולית של מבחן ההשוואה נוכל להסיק כי $\int_e^\infty f^p(x) dx = \infty$ לכל $0 < p < 1$ בסתירה לטענה.
\end{proof}

\subsection{סעיף ב'}
נפריך את הטענה כי
אם $f(x)$ רציפה ב־$[a, \infty)$, האינטגרל $\displaystyle\int_a^\infty f(x) dx$ מתכנס וקיים הגבול $\displaystyle\lim_{x \to \infty} f(x)$, \\* % chktex 9
אז האינטגרל $\displaystyle\int_a^\infty f^2(x) dx$ מתכנס.
\begin{proof}
	נגדיר
	\[
		f(x) = \frac{\cos x}{\sqrt{x}}
	\]
	נחשב
	\[
		\int f(x) dx = \frac{\sin x}{\sqrt{x}} - \int \frac{\sin x}{\sqrt{x^3}} dx
	\]
	ונשים לב כי
	\[
		\frac{\sin x}{\sqrt{x^3}} \le \frac{1}{\sqrt{x^3}}
	\]
	ולכן ממבחן ההשוואה ולמה 3.12 נובע כי
	\[
		\int_1^\infty f(x) dx = \left( \lim_{x \to \infty} \frac{\sin x}{\sqrt{x}} \right) - \int_1^\infty \frac{\sin x}{\sqrt{x^3}} dx
	\]
	הוא אינטגרל מתכנס. נבחן את האינטגרל של $f^2(x)$
	\[
		\int f^2(x) dx
		= \int \frac{\cos^2 x}{x} dx
		= \ln x \cos^2 x - \int \ln x \cdot 2 \cos x \sin x dx
		= \ln x \cos^2 x - \int \ln x \sin 2x dx
	\]
	עוד נראה כי
	\[
		\ln x \sin 2x \le \ln x
	\]
	ולכן ממבחן ההשוואה נובע ישירות כי
	\[
		\ln x \cos^2 x - \int \ln x \sin 2x dx \ge \ln (\cos^2(x)) - \frac{1}{x}
	\]
	כמובן שהגבול $\lim_{x \to \infty} \ln (\cos^2(x)) - \frac{1}{x} = \infty$ ולכן ממבחן ההשוואה נובעת התבדרות האינטגרל
	\[
		\int_1^\infty f^2(x) dx
	\]
	מצאנו כי האינטגרל מתכנס בסתירה לטענה.
\end{proof}

\section{שאלה 5}
תהיינה $f(x), g(x)$ פונקציות רציפות ב־$[0, \infty)$ אשר עבורן מתקיים: \\* % chktex 9
$g(x)$ חסומה ובעלת נגזרת רציפה ב־$[0, \infty)$ כך ש־$g'(x) < g(x)$ לכל $x \ge 0$, \\* % chktex 9
קיים $M$ כך שמתקיים לכל $t \ge 0$
\[
	\left\lvert \int_0^t e^x f(x) dx \right\rvert \le M
\]
נוכיח את התכנסות האינטגרל
\[
	\int_0^\infty f(x) g(x) dx
\]
\begin{proof}
	נגדיר $g_0(x) = \frac{g(x)}{e^x}$, ונחשב את נגזרתה
	\[
		g_0'(x) = \frac{g'(x) e^x - g(x) e^x}{e^{2x}} = \frac{g'(x) - g(x)}{e^x}
	\]
	נתון כי $g'(x) - g(x) < 0$ ולכן $g_0$ יורדת לכל $x \ge 0$. \\*
	$g_0(x)$ היא חלוקת פונקציה חסומה בפונקציה מונוטונית עולה, ולכן
	\[
		\lim_{x \to \infty} g_0(x) = \frac{1}{\infty} = 0
	\]
	מצאנו כי $g_0(x)$ מקיימת את התנאי הראשון של מבחן דיריכלה. \\*
	נגדיר $f_0(x) = e^x f(x)$. נתון כי
	\[
		\left\lvert \int_0^t f_0(x) dx \right\rvert \le M
	\]
	דהינו האינטגרל של $f_0(x)$ חסום לכל ערך,
	ולכן $f_0$ ממלאת את התנאים למבחן דיריכלה יחד עם $g_0(x)$. \\*
	עתה נבחין כי
	\[
		f_0(x) g_0(x) = f(x) g(x) \frac{e^x}{e^x} = f(x) g(x)
	\]
	ולכן נובע כי מתכנס האינטגרל
	\[
		\int_0^\infty f(x) g(x) dx
	\]
\end{proof}

\section{שאלת רשות}
יהיו $0 < a < b$. נוכיח את התכנסות האינטגרל
\[
	\int_0^\infty \frac{e^{-ax} - e^{-bx}}{x}dx
\]
\begin{proof}
	נראה כי
	\begin{flalign*}
		\int_0^\infty \frac{e^{-ax} - e^{-bx}}{x}dx
		& = \int_0^\infty \frac{e^{-ax}}{x}dx - \int_0^\infty \frac{e^{-bx}}{x}dx && & \text{מהגדרת האינטגרל המורחב} \\
		& = \lim_{k \to 0} \int_k^\infty \frac{e^{-ax}}{x}dx - \int_k^\infty \frac{e^{-bx}}{x}dx \\
		\begin{vmatrix}
			t_1 = x/a & dt_1 = dx/a \\
			t_2 = x/b & dt_2 = dx/b \\
		\end{vmatrix}
		& = \lim_{k \to 0} \int_{ak}^\infty \frac{e^{-t_1}}{t_1}dt_1 - \int_{bk}^\infty \frac{e^{-t_2}}{t_2}dt_2 && & \text{נשים לב כי הביטויים שקולים}\\
		& = \lim_{k \to 0} \int_{ak}^{bk} \frac{e^{-t}}{t}dt && & \text{על־פי גבול אינטגרל מוכלל ניתן לראות כי} \\
		\begin{vmatrix}
			t = ku \\
			dt = k du
		\end{vmatrix}
		& = \lim_{k \to 0} \int_{a}^{b} \frac{e^{-k u}}{k u}k \, du && & \text{נשים לב כי $k$ קבוע באיטגרל} \\
		& = \int_{a}^{b} \lim_{k \to 0} \frac{e^{-k u}}{u} du && & \text{קיבלנו ביטוי עבורו הפונקציה הפנימית בלבד תלויה ב־$k$} \\
		& = \int_{a}^{b} \frac{1}{u} du = \left. \ln u \right|_a^b = \ln a - \ln b = \ln \frac{a}{b}
	\end{flalign*}
	מצאנו כי האינטגרל מוגדר וכי
	\[
		\int_0^\infty \frac{e^{-ax} - e^{-bx}}{x}dx = \ln \frac{a}{b}
	\]
\end{proof}
\end{document} % chktex 17
