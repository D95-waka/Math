\documentclass[a4paper]{article}

% packages
\usepackage{inputenc, fontspec, amsmath, amsthm, amsfonts, polyglossia, catchfile}
\usepackage[a4paper, margin=50pt, includeheadfoot]{geometry} % set page margins

% style
\AddToHook{cmd/section/before}{\clearpage}	% Add line break before section
\linespread{1.5}
\setcounter{secnumdepth}{0}		% Remove default number tags from sections
\setmainfont{Libertinus Serif}
\setsansfont{Libertinus Sans}
\setmonofont{Libertinus Mono}
\setdefaultlanguage{hebrew}
\setotherlanguage{english}

% operators
\DeclareMathOperator\cis{cis}
\DeclareMathOperator\Sp{Sp}
\DeclareMathOperator\tr{tr}
\DeclareMathOperator\im{Im}
\DeclareMathOperator\diag{diag}
\DeclareMathOperator*\lowlim{\underline{lim}}
\DeclareMathOperator*\uplim{\overline{lim}}

% commands
\renewcommand\qedsymbol{\textbf{משל}}
\newcommand{\NN}[0]{\mathbb{N}}
\newcommand{\ZZ}[0]{\mathbb{Z}}
\newcommand{\QQ}[0]{\mathbb{Q}}
\newcommand{\RR}[0]{\mathbb{R}}
\newcommand{\CC}[0]{\mathbb{C}}
\newcommand{\getenv}[2][] {
  \CatchFileEdef{\temp}{"|kpsewhich --var-value #2"}{\endlinechar=-1}
  \if\relax\detokenize{#1}\relax\temp\else\let#1\temp\fi
}
\newcommand{\explain}[2] {
	\begin{flalign*}
		 && \text{#2} && \text{#1}
	\end{flalign*}
}

% headers
\getenv[\AUTHOR]{AUTHOR}
\author{\AUTHOR}
\date\today

\title{פתרון ממ''ן 12 – חשבון אינפיניטסימלי 2 (20475)}

\begin{document}
\maketitle
\maketitleprint{}
\section{שאלה 1}
נחשב את האינטגרלים הבאים

\subsection{סעיף א'}
\begin{align*}
	\int x^3{(1 - 3x^2)}^{10} dx
	& = \int x^3 \left( \sum_{k = 0}^{10} \binom{10}{k} {(-3x^2)}^k \right) dx \\
	& = \int \left( \sum_{k = 0}^{10} \binom{10}{k} {(-3)}^k x^{2k + 3} \right) dx \\
	& = C + \sum_{k = 0}^{10} \binom{10}{k} \frac{1}{2k + 4} {(-3)}^k x^{2k + 4} \\
	& = C + x^4 \sum_{k = 0}^{10} \binom{10}{k} \frac{{(-3)}^k}{2k + 4} x^{2k}
\end{align*}

\subsection{סעיף ב'}
\begin{align*}
	\int \frac{x e^{\arcsin x}}{\sqrt{1 - x^2}}dx
	& = \int \frac{\sin \arcsin(x) e^{\arcsin x}}{\sqrt{1 - x^2}}dx \\
	& = \left. \int \sin(t) e^t dt \right|_{dt = \frac{dx}{\sqrt{1 - x^2}}}^{t = \arcsin x} \\
	& = -\cos(t) e^t + \int \cos(t) e^t dt && \text{אינטגרציה בחלקים} \\
	& = -\cos(t) e^t + \sin(x) e^t - \int \sin(t) e^t dt && \text{אינטגרציה בחלקים} \\
	2 \int \sin(t) e^t dt & = \sin(x) - \cos(t) e^t \\
	\int \frac{x e^{\arcsin x}}{\sqrt{1 - x^2}}dx & = \frac{1}{2} \left(x - \sqrt{1 - x^2} e^{\arcsin x}\right) + C
\end{align*}

\subsection{סעיף ג'}
\begin{align*}
	\int {(x - 1)}^2 e^{2x} dx
	& = \frac{1}{2} e^{2x} {(x - 1)}^2 - \int \frac{1}{2} e^{2x} (2x - 2) dx \\
	& = \frac{1}{2} e^{2x} {(x - 1)}^2 + \int e^{2x} dx - \int e^{2x} x dx \\
	& = \frac{1}{2} e^{2x} {(x - 1)}^2 + \frac{1}{2} e^{2x} - \frac{1}{2} e^{2x} x + \int \frac{1}{2} e^{2x} dx \\
	& = \frac{1}{2} e^{2x} (x^2 - 3x + 2) + \frac{1}{4} e^{2x} \\
	& = \frac{1}{2} e^{2x} (x^2 - 3x + 2\frac{1}{2}) + C
\end{align*}

\subsection{סעיף ד'}
\begin{align*}
	\int \frac{4x + 1}{\sqrt{{(x^2 + 4x + 5)}^3}} dx
	& = \begin{vmatrix}
		t = x + 2 \\
		dt = dx
	\end{vmatrix} \\
	& = \int \frac{4t - 7}{\sqrt{{(t^2 + 1)}^3}} dt \\
	& = 2\int \frac{2t}{\sqrt{{(t^2 + 1)}^3}} dt - 7\int \frac{1}{\sqrt{{(t^2 + 1)}^3}} dt \\
\end{align*}
נחשב
\begin{align*}
	& \int \frac{2t}{\sqrt{{(t^2 + 1)}^3}} dt \\
	& \begin{vmatrix}
		u = t^2 + 1 \\
		du = 2t
	\end{vmatrix} \\
	= & \int \frac{du}{u^{3/2}} \\
	= & \int u^{-3/2} du \\
	= & -2 u^{-1/2} \\
	= & \frac{-2}{\sqrt{t^2 + 1}}
\end{align*}
נחשב גם
\begin{align*}
	& \int \frac{1}{\sqrt{{(t^2 + 1)}^3}} dt \\
	& \begin{vmatrix}
		t = \tan u \\
		dt = \frac{du}{\cos^2 u}
	\end{vmatrix} \\
	= & \int \frac{1}{\cos^2 u \sqrt{{(\tan^2 u + 1)}^3}} du \\
	= & \int \frac{1}{\cos^2 u \sqrt{{(\cos^{-2})}^3}} du \\
	= & \sin u = \sin \arctan t = \frac{t}{\sqrt{t^2 + 1}}
\end{align*}
ולכן נובע כי
\[
	2\int \frac{2t}{\sqrt{{(t^2 + 1)}^3}} dt - 7\int \frac{1}{\sqrt{{(t^2 + 1)}^3}} dt \\
	= 2 \frac{-2}{\sqrt{t^2 + 1}} - 7 \frac{t}{\sqrt{t^2 + 1}}
	= \frac{-7t - 4}{\sqrt{t^2 + 1}}
	= \frac{-7x - 18}{\sqrt{x^2 + 4x + 5}}
\]
לכן
\[
	\int_{-2}^{-1} \frac{4x + 1}{\sqrt{{(x^2 + 4x + 5)}^3}} dx
	= \left. \frac{-7x - 18}{\sqrt{x^2 + 4x + 5}} \right|_{-2}^{-1}
	= \frac{-7(-2) - 18}{\sqrt{{(-2)}^2 + 4 (-2) + 5}} - \frac{-7(-1) - 18}{\sqrt{{(-1)}^2 + 4 (-1) + 5}}
	= -4 + \frac{11}{\sqrt{2}}
\]

\subsection{סעיף ה'}
\[
	\int_{-\ln 2}^{\ln 2} e^{|x|} (1 + x^3 + \sin x) dx
	= \int_{-\ln 2}^{\ln 2} e^{|x|} (x^3 + \sin x) dx
	+ \int_{-\ln 2}^{\ln 2} e^{|x|} dx
\]
נגדיר
\[
	f(x) = e^{|x|} (1 + \sin x)
\]
מחישוב עולה כי
\[
	f(-x) = e^{|-x|} (-x^3 - \sin x) = - f(x)
\]
ולכן
\[
	\int_{-\ln 2}^{\ln 2} f(x) dx
	= \int_{-\ln 2}^{0} f(x) dx + \int_{0}^{\ln 2} f(x) dx
	= -\int_{0}^{\ln 2} f(x) dx + \int_{0}^{\ln 2} f(x) dx
	= 0
\]
עוד נראה כי
\[
	\int_{-\ln 2}^{\ln 2} e^{|x|} dx
	= \int_{-\ln 2}^{0} e^{|x|} dx + \int_{0}^{\ln 2} e^{|x|} dx
	= \int_{-\ln 2}^{0} e^{-x} dx + \int_{0}^{\ln 2} e^{x} dx
	= 2\int_{0}^{\ln 2} e^{x} dx
	= 2 (e^{\ln 2} - 1) = 4
\]

\section{שאלה 2}
עבור כל אחד מהאינטגרלים הבאים נקבע אם הוא מתכנס בהחלט, בתנאי, או כלל לא.

\subsection{סעיף א'}
\[
	\int_0^\infty \frac{\cos^2 x}{\sqrt{x}}dx
\]
נשים לב כי $f(x) = \frac{1}{\sqrt{x}}$ היא פונקציה מונוטונית יורדת וכי $\lim_{x \to \infty} f(x) = 0$. \\*
עוד נשים לב כי $g(x) = \cos^2 x$ היא פונקציה חסומה המקיימת $0 \le g(x) \le 1$ לכל $0 \le x$. \\*
לכן ממבחן דיריכלה נובע כי האינטגרל הנתון, אשר מהווה אינטגרל מכפלת הפונקציות $f, g$ מתכנס. \\*
נשים לב כי גם $f(x) g(x) = |f(x) g(x)|$ ולכן האינטגרל גם מתכנס בהחלט.

\subsection{סעיף ב'}
\[
	\int_1^\infty \frac{x \ln x}{(x^2 - 1){(\ln(x + 1))}^3} dx
\]
נגדיר
\[
	f(x) = \frac{1}{x^2 - 1}
\]
נשים לב כי
\[
	f'(x) = \frac{x^2 - 1 - 2x^2}{{(x^2 - 1)}^2} < 0
\]
נראה גם כי
\[
	\lim_{x \to \infty} f(x) = 0
\]
ולכן $f(x)$ מונוטונית יורדת ואפסה. \\*
נגדיר גם
\[
	g(x) = \frac{\ln x}{\ln^3(x + 1)}
\]
נראה כי עבור $x \ge 1$:
\[
	0 < \frac{\ln x}{\ln^3(x + 1)}
	\le \frac{\ln(x + 1)}{\ln^3(x + 1)}
	= \frac{1}{\ln^2(x + 1)} \le 1
\]
מצאנו כי $g(x)$ חסומה בתחום $[1, \infty)$. \\* % chktex 9
מצאנו כי הפונקציות $f$ ו־$g$ מקיימות את מבחן דיריכלה ולכן
\[
	\int_1^\infty f(x) g(x) dx
\]
מתכנס. נשים לב כי $0 \le f(x)$ לכל $x > 1$ ולכן מתקיים $|f(x) g(x)| = f(x) g(x)$ ובהתאם האינטגרל מתכנס בהחלט.

\subsection{סעיף ג'}
\[
	\int_0^\infty \frac{1}{x} (\sqrt[4]{x} + 1) \sin \left( 2 \sqrt{x} \right) dx
\]
נגדיר
\[
	f(x) = \frac{\sqrt[4]{x} + 1}{x}, g(x) = \sin\left(2\sqrt{x}\right)
\]
בתחום $[1, \infty)$ הפונקציה $f$ מונוטונית יורדת וחיובית בכל התחום, ו־$|g|$ חסומה על־ידי $[0, 1]$. \\* % chktex 9
כמו־כן גם $\lim_{x \to \infty} f(x) = 0$ ולכן ממבחן דיריכלה האינטגרל
\[
	\int_1^\infty |f(x) g(x)| dx
\] 
מתכנס. ידוע מאדיטיביות האינטגרל כי
\[
	\int_0^\infty f(x) g(x) dx = \int_0^1 f(x) g(x) dx + \int_1^\infty f(x) g(x) dx
\]
נראה כי
\[
	f(x) g(x) = 2\frac{\frac{1}{\sqrt[4]{x}} + \frac{1}{\sqrt{x}}}{1} \frac{\sin(2\sqrt{x})}{2\sqrt{x}}
\]
ידוע כי בתחום $0 < x < \frac{\pi}{2}$ מתקיים $0 < \sin t < t$ ולכן גם בהכרח $0 < \frac{\sin(2 \sqrt{x})}{2 \sqrt{x}} < 1$. \\*
אז מצאנו כי
\[
	f(x) g(x)
	< \frac{2}{\sqrt[4]{x}} + \frac{2}{\sqrt{x}}
	= \frac{2(x^{1/2} + x^{1/4})}{x^{3/4}}
	= \frac{4x^{1/4}}{x^{3/4}}
	< \frac{4}{x^{1/4}}
\]
מלמה 3.2 נובע כי האינטגרל
\[
	\int_0^1 \frac{4}{x^{1/4}} dx
\]
הוא אינטגרל מתכנס ולכן ממשפט ההשוואה נובע שמתכנס גם
\[
	\int_0^1 f(x) g(x) dx
\]
ולכן האינטגרל
\[
	\int_0^\infty f(x) g(x) dx
\]
מתכנס בהחלט.

\section{שאלה 3}
תהי פונקציה $f$ אינטגרבילית בקטע $[0, a]$ לכל $a > 0$ ומתקיים $\lvert f(x) \rvert \le \sqrt{x}$ לכל $x \ge 0$. \\*
נגדיר $F(x) = \int_0^x f(t) dt$, ונוכיח כי האינטגרל המוגדר על־ידי
\[
	\int_0^\infty \frac{F(x)}{\pi + x^3} dx \tag{1}
\]
מתכנס.
\begin{proof}
	ידוע כי $\lvert f(x) \rvert \le \sqrt{x}$ לכל $x \ge 0$ ולכן מהמונוטוניות של האינטגרל המסוים (1.26) נובע
	\[
		F(x) = \int_0^x \lvert f(t) \rvert dt \le \int_0^x \sqrt{t} dt = \left. \frac{2}{3} \sqrt{t^3} \right|_0^x = \frac{2}{3} \sqrt{x^3}
	\]
	מאי־השוויון נובע כי לכל $x \ge 0$ גם
	\[
		\frac{F(x)}{\pi + x^3} \le \frac{2 \sqrt{x^3}}{3 \pi + 3 x^3} \tag{2}
	\]
	נשים לב כי מונה נגזרת הביטוי הימני הוא
	\[
		\sqrt{x} (\pi + x^3) - \frac{4}{3} x^2 \sqrt{x^3}
	\]
	ולכן קיים קבוע $c > 0$ עבורו הביטוי יורד לכל $x \ge c$, וכמו כן ניתן לראות כי זוהי פונקציה אפסה. \\*
	עוד נראה כי שתי הפונקציות המופיעות באי־השוויון הן רציפות לכל $x \ge 0$ ובהתאם ניתן להסיק כי הביטוי השמאלי חסום בתחום $[a, \infty)$. \\* % chktex 9
	אז ממבחן דיריכלה נובע כי האינטגרל
	\[
		\int_c^\infty \frac{F(x)}{\pi + x^3} dx
	\]
	מתכנס. נבחן את האינטגרל
	\[
		\int_0^c \frac{F(x)}{\pi + x^3} dx
	\]
	זהו כמובן אינטגרל מסוים של פונקציה רציפה ולכן הוא מתכנס ובעל ערך סופי, ומתקיים
	\[
		\int_c^\infty \frac{F(x)}{\pi + x^3} dx + \int_0^c \frac{F(x)}{\pi + x^3} dx
		= \int_0^\infty \frac{F(x)}{\pi + x^3} dx
	\]
\end{proof}

\section{שאלה 4}
\subsection{סעיף א'}
תהי $f(x)$ פונקציה רציפה וחיובית בקטע $[a, \infty)$. \\* % chktex 9
נוכיח כי אם האינטגרל $\int_a^\infty f(x) dx$ מתכנס אז קיים $0 < c < 1$ כך שהאינטגרל
\[
	\int_a^\infty {(f(x))}^p dx
\]
מתכנס לכל $c \le p \le 1$.
\begin{proof}
	מהגדרת האינטגרל אנו יכולים להסיק כי
	\[
		\lim_{x \to \infty} f(x) = 0
	\]
	מהגבול ומרציפותה בתחום ניתן להסיק ממשפט ויירשטרס השני כי יש לפונקציה מקסימום $M$, \\*
	ומהגבול ניתן להסיק כי אפשר לחסום את הפונקציה בפונקציה מהצורה $M_1 x^{-b}$, כאשר $b > 1$ קבוע כלשהו, וכאשר $M_1 \ge M$. \\*
	אז מתקיים
	\[
		0 < f(x) < M_1 x^{-b}
	\]
	לכל $x \ge a$. נבחין כי גם
	\[
		0 < f^p(x) < M_1^p x^{-bp}
	\]
	לכל $0 < p$, ומלמה 3.12 נסיק כי אם $bp > 1$ אז האינטגרל $\int_a^\infty M_1^p x^{-bp}dx$ מוגדר גם הוא ובעקבותיו גם $\int_a^\infty f^p(x) dx$. \\*
	אז מצאנו כי עבור $\frac{1}{b} < p \le 1$ האינטגרל מוגדר, נוכל להגדיר $\frac{1}{b} < c < 1$ ערך המקיים את התנאי.
\end{proof}

\subsection{סעיף ב'}
נפריך את הטענה כי
אם $f(x)$ רציפה ב־$[a, \infty)$, האינטגרל $\displaystyle\int_a^\infty f(x) dx$ מתכנס וקיים הגבול $\displaystyle\lim_{x \to \infty} f(x)$, \\* % chktex 9
אז האינטגרל $\displaystyle\int_a^\infty f^2(x) dx$ מתכנס.
\begin{proof}
	נגדיר
	\[
		f(x) = \frac{\cos x}{\sqrt{x}}
	\]
	נחשב
	\[
		\int f(x) dx = \frac{\sin x}{\sqrt{x}} - \int \frac{\sin x}{\sqrt{x^3}} dx
	\]
	ונשים לב כי
	\[
		\frac{\sin x}{\sqrt{x^3}} \le \frac{1}{\sqrt{x^3}}
	\]
	ולכן ממבחן ההשוואה ולמה 3.12 נובע כי
	\[
		\int_1^\infty f(x) dx = \left( \lim_{x \to \infty} \frac{\sin x}{\sqrt{x}} \right) - \int_1^\infty \frac{\sin x}{\sqrt{x^3}} dx
	\]
	הוא אינטגרל מתכנס. נבחן את האינטגרל של $f^2(x)$
	\[
		\int f^2(x) dx
		= \int \frac{\cos^2 x}{x} dx
		= \ln x \cos^2 x - \int \ln x \cdot 2 \cos x \sin x dx
		= \ln x \cos^2 x - \int \ln x \sin 2x dx
	\]
	עוד נראה כי
	\[
		\ln x \sin 2x \le \ln x
	\]
	ולכן
	\[
		\ln x \cos^2 x - \int \ln x \sin 2x dx \ge \ln (\cos^2(x) + 1)
	\]
	ולכן ממבחן ההשוואה נקבל כי האינטגרל
	\[
		\int_1^\infty f^2(x) dx
	\]
	איננו מתכנס, בסתירה לטענה.
\end{proof}

\section{שאלה 5}
תהיינה $f(x), g(x)$ פונקציות רציפות ב־$[0, \infty)$ אשר עבורן מתקיים: \\* % chktex 9
$g(x)$ חסומה ובעלת נגזרת רציפה ב־$[0, \infty)$ כך ש־$g'(x) < g(x)$ לכל $x \ge 0$, \\* % chktex 9
קיים $M$ כך שמתקיים לכל $t \ge 0$
\[
	\left\lvert \int_0^t e^x f(x) dx \right\rvert \le M
\]
נוכיח את התכנסות האינטגרל
\[
	\int_0^\infty f(x) g(x) dx
\]
\begin{proof}
	נגדיר $g_0(x) = \frac{g(x)}{e^x}$, ונחשב את נגזרתה
	\[
		g_0'(x) = \frac{g'(x) e^x - g(x) e^x}{e^{2x}} = \frac{g'(x) - g(x)}{e^x}
	\]
	ידוע כי $g'(x) - g(x) < 0$ ולכן $g_0$ יורדת לכל $x \ge 0$. \\*
	$g_0(x)$ היא חלוקת פונקציה חסומה בפונקציה מונוטונית עולה, ולכן
	\[
		\lim_{x \to \infty} g_0(x) = \frac{1}{\infty} = 0
	\]
	מצאנו כי $g_0(x)$ מקיימת את התנאי הראשון של מבחן דיריכלה. \\*
	נגדיר $f_0(x) = e^x f(x)$. נתון כי
	\[
		\left\lvert \int_0^t f_0(x) dx \right\rvert \le M
	\]
	ידוע כי $f_0(x)$ היא רציפה בתחום $[0, \infty)$, אילו מתקיים % chktex 9
	\[
		\lim_{x \to \infty} f(x) = L
	\]
	כאשר $L$ סופי אז ניתן להוכיח כי $f_0$ חסומה בתחום. נניח כי גבול זה איננו סופי, דהינו $L = \pm \infty$. \\*
	במצב זה האינטגרל
	\[
		\int_0^t f_0(x) dx
	\]
	מייצג פונקציה מונוטונית עולה או מונוטונית יורדת על־פי אינפי 1, בסתירה לטענה כי אינטגרל זה חסום, ולכן הגבול של $f_0$ הוא סופי והיא חסומה. \\*
	על־כן $f_0$ ממלאת את התנאים למבחן דיריכלה. עתה נבחין כי
	\[
		f_0(x) g_0(x) = f(x) g(x) \frac{e^x}{e^x} = f(x) g(x)
	\]
	ולכן נובע כי מתקיים האינטגרל
	\[
		\int_0^\infty f(x) g(x) dx
	\]
\end{proof}

\section{שאלת רשות}
יהיו $0 < a < b$. נוכיח את התכנסות האינטגרל
\[
	\int_0^\infty \frac{e^{-ax} - e^{-bx}}{x}dx \tag{1}
\]
\begin{proof}
	נראה כי
	\begin{align*}
		\int_0^\infty \frac{e^{-ax} - e^{-bx}}{x}dx
		& = \int_0^\infty \frac{e^{-ax}}{x}dx - \int_0^\infty \frac{e^{-bx}}{x}dx \\
		& = \lim_{k \to 0} \int_k^\infty \frac{e^{-ax}}{x}dx - \int_k^\infty \frac{e^{-bx}}{x}dx \\
		\begin{vmatrix}
			t_1 = x/a & dt_1 = dx/a \\
			t_2 = x/b & dt_2 = dx/b \\
		\end{vmatrix}
		& = \lim_{k \to 0} \int_{ak}^\infty \frac{e^{-t_1}}{t_1}dt_1 - \int_{bk}^\infty \frac{e^{-t_2}}{t_2}dt_2 \\
		& = \lim_{k \to 0} \int_{ak}^{bk} \frac{e^{-t}}{t}dt \\
		\begin{vmatrix}
			t = ku \\
			dt = k du
		\end{vmatrix}
		& = \lim_{k \to 0} \int_{a}^{b} \frac{e^{-k u}}{k u}k \, du \\
		& = \int_{a}^{b} \lim_{k \to 0} \frac{e^{-k u}}{u} du \\
		& = \int_{a}^{b} \frac{1}{u} du = \frac{1}{b} - \frac{1}{a}
	\end{align*}
\end{proof}
\end{document} % chktex 17
