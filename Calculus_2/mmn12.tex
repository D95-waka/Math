\documentclass[a4paper]{article}

% packages
\usepackage{inputenc, fontspec, amsmath, amsthm, amsfonts, polyglossia, catchfile}
\usepackage[a4paper, margin=50pt, includeheadfoot]{geometry} % set page margins

% style
\AddToHook{cmd/section/before}{\clearpage}	% Add line break before section
\linespread{1.5}
\setcounter{secnumdepth}{0}		% Remove default number tags from sections
\setmainfont{Libertinus Serif}
\setsansfont{Libertinus Sans}
\setmonofont{Libertinus Mono}
\setdefaultlanguage{hebrew}
\setotherlanguage{english}

% operators
\DeclareMathOperator\cis{cis}
\DeclareMathOperator\Sp{Sp}
\DeclareMathOperator\tr{tr}
\DeclareMathOperator\im{Im}
\DeclareMathOperator\diag{diag}
\DeclareMathOperator*\lowlim{\underline{lim}}
\DeclareMathOperator*\uplim{\overline{lim}}

% commands
\renewcommand\qedsymbol{\textbf{משל}}
\newcommand{\NN}[0]{\mathbb{N}}
\newcommand{\ZZ}[0]{\mathbb{Z}}
\newcommand{\QQ}[0]{\mathbb{Q}}
\newcommand{\RR}[0]{\mathbb{R}}
\newcommand{\CC}[0]{\mathbb{C}}
\newcommand{\getenv}[2][] {
  \CatchFileEdef{\temp}{"|kpsewhich --var-value #2"}{\endlinechar=-1}
  \if\relax\detokenize{#1}\relax\temp\else\let#1\temp\fi
}
\newcommand{\explain}[2] {
	\begin{flalign*}
		 && \text{#2} && \text{#1}
	\end{flalign*}
}

% headers
\getenv[\AUTHOR]{AUTHOR}
\author{\AUTHOR}
\date\today

\title{פתרון ממ''ן 12 – חשבון אינפיניטסימלי 2 (20475)}

\begin{document}
\maketitle
\maketitleprint{}
\section{שאלה 1}
נחשב את האינטגרלים הבאים

\subsection{סעיף א'}
\begin{align*}
	\int x^3{(1 - 3x^2)}^{10} dx
	& = \int x^3 \left( \sum_{k = 0}^{10} \binom{10}{k} {(-3x^2)}^k \right) dx \\
	& = \int \left( \sum_{k = 0}^{10} \binom{10}{k} {(-3)}^k x^{2k + 3} \right) dx \\
	& = C + \sum_{k = 0}^{10} \binom{10}{k} \frac{1}{2k + 4} {(-3)}^k x^{2k + 4} \\
	& = C + x^4 \sum_{k = 0}^{10} \binom{10}{k} \frac{{(-3)}^k}{2k + 4} x^{2k}
\end{align*}

\subsection{סעיף ב'}
\begin{align*}
	\int \frac{x e^{\arcsin x}}{\sqrt{1 - x^2}}dx
	& = \int \frac{\sin \arcsin(x) e^{\arcsin x}}{\sqrt{1 - x^2}}dx \\
	& = \left. \int \sin(t) e^t dt \right|_{dt = \frac{dx}{\sqrt{1 - x^2}}}^{t = \arcsin x} \\
	& = -\cos(t) e^t + \int \cos(t) e^t dt && \text{אינטגרציה בחלקים} \\
	& = -\cos(t) e^t + \sin(x) e^t - \int \sin(t) e^t dt && \text{אינטגרציה בחלקים} \\
	2 \int \sin(t) e^t dt & = \sin(x) - \cos(t) e^t \\
	\int \frac{x e^{\arcsin x}}{\sqrt{1 - x^2}}dx & = \frac{1}{2} \left(x - \sqrt{1 - x^2} e^{\arcsin x}\right) + C
\end{align*}

\subsection{סעיף ג'}
\begin{align*}
	\int {(x - 1)}^2 e^{2x} dx
	& = \frac{1}{2} e^{2x} {(x - 1)}^2 - \int \frac{1}{2} e^{2x} (2x - 2) dx \\
	& = \frac{1}{2} e^{2x} {(x - 1)}^2 + \int e^{2x} dx - \int e^{2x} x dx \\
	& = \frac{1}{2} e^{2x} {(x - 1)}^2 + \frac{1}{2} e^{2x} - \frac{1}{2} e^{2x} x + \int \frac{1}{2} e^{2x} dx \\
	& = \frac{1}{2} e^{2x} (x^2 - 3x + 2) + \frac{1}{4} e^{2x} \\
	& = \frac{1}{2} e^{2x} (x^2 - 3x + 2\frac{1}{2}) + C
\end{align*}

\subsection{סעיף ד'}
\begin{align*}
	\int_{-2}^{-1} \frac{4x + 1}{\sqrt{{(x^2 + 4x + 5)}^3}} dx
	& = \begin{vmatrix}
		x = t + 2 \\
		dx = dt
	\end{vmatrix} \\
	& = \int_{-4}^{-3} \frac{4t - 7}{\sqrt{{(t^2 + 1)}^3}} dt \\
	& = \begin{vmatrix}
		t = \tan u \\
		dt = \frac{du}{\cos^2 u}
	\end{vmatrix} \\
	& = \int_{\arctan -4}^{\arctan -3} \frac{4 \tan u - 7}{\cos^2 u \sqrt{ {(\tan^2 u + 1)}^3}} du \\
	& = \int_{\arctan -4}^{\arctan -3} \frac{4 \tan u - 7}{\cos^2 u \sqrt{ {(\frac{1}{\cos^2 u})}^3}} du \\
	& = \int_{\arctan -4}^{\arctan -3} \frac{4 \tan u - 7}{\cos^2 u \cos^{-3} u} du \\
	& = \int_{\arctan -4}^{\arctan -3} (4 \tan u - 7) \cos u \, du \\
	& = \int_{\arctan -4}^{\arctan -3} 4 \sin u - 7 \cos u \, du \\
	& = \left. -4 \cos u -7 \sin u \right|_{\arctan -4}^{\arctan -3} \\
	& = \left. \frac{-4x}{\sqrt{x^2 + 1}} - \frac{7}{\sqrt{x^2 + 1}} \right|_{-4}^{-3} \\
	& = \frac{-4 (-3)}{\sqrt{{(-3)}^2 + 1}} - \frac{7}{\sqrt{{(-3)}^2 + 1}} + \frac{4 (-4)}{\sqrt{{(-4)}^2 + 1}} + \frac{7}{\sqrt{{(-4)}^2 + 1}}
\end{align*}

\subsection{סעיף ה'}
\[
	\int_{-\ln 2}^{\ln 2} e^{|x|} (1 + x^3 + \sin x) dx
	= \int_{-\ln 2}^{\ln 2} e^{|x|} (x^3 + \sin x) dx
	+ \int_{-\ln 2}^{\ln 2} e^{|x|} dx
\]
נגדיר
\[
	f(x) = e^{|x|} (1 + \sin x)
\]
מחישוב עולה כי
\[
	f(-x) = e^{|-x|} (-x^3 - \sin x) = - f(x)
\]
ולכן
\[
	\int_{-\ln 2}^{\ln 2} f(x) dx
	= \int_{-\ln 2}^{0} f(x) dx + \int_{0}^{\ln 2} f(x) dx
	= -\int_{0}^{\ln 2} f(x) dx + \int_{0}^{\ln 2} f(x) dx
	= 0
\]
עוד נראה כי
\[
	\int_{-\ln 2}^{\ln 2} e^{|x|} dx
	= \int_{-\ln 2}^{0} e^{|x|} dx + \int_{0}^{\ln 2} e^{|x|} dx
	= \int_{-\ln 2}^{0} e^{-x} dx + \int_{0}^{\ln 2} e^{x} dx
	= 2\int_{0}^{\ln 2} e^{x} dx
	= 2 (e^{\ln 2} - 1) = 4
\]

\end{document}
