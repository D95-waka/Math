\documentclass[a4paper]{article}

% packages
\usepackage{inputenc, amsmath, amsthm, thmtools, amsfonts, amssymb, luacode, catchfile, tikzducks, hyperref}
\usepackage[a4paper, margin=50pt, includeheadfoot]{geometry} % set page margins
\usepackage[shortlabels]{enumitem}
\usepackage[skip=3pt, indent=0pt]{parskip}

% language
\usepackage[bidi=basic, layout=tabular, provide=*]{babel}
\babelprovide[main, import]{hebrew}
\babelprovide{rl}
\babelfont{rm}{Libertinus Serif}
\babelfont{sf}{Libertinus Sans}
\babelfont{tt}{Libertinus Mono}

% style
\AddToHook{cmd/section/before}{\clearpage}	% Add line break before section
\linespread{1.3}
\setcounter{secnumdepth}{0}		% Remove default number tags from sections, this won't do well with theorems
\AtBeginDocument{\setlength{\belowdisplayskip}{3pt}}
\AtBeginDocument{\setlength{\abovedisplayskip}{3pt}}

% operators
\DeclareMathOperator\cis{cis}
\DeclareMathOperator\Sp{Sp}
\DeclareMathOperator\tr{tr}
\DeclareMathOperator\im{Im}
\DeclareMathOperator\re{Re}
\DeclareMathOperator\diag{diag}
\DeclareMathOperator*\lowlim{\underline{lim}}
\DeclareMathOperator*\uplim{\overline{lim}}
\DeclareMathOperator\rng{rng}
\DeclareMathOperator\Sym{Sym}
\DeclareMathOperator\Arg{Arg}
\DeclareMathOperator\Log{Log}
\DeclareMathOperator\dom{dom}

% commands
%\renewcommand\qedsymbol{\textbf{מש''ל}}
%\renewcommand\qedsymbol{\fbox{\emoji{lizard}}}
\newcommand{\NN}[0]{\mathbb{N}}
\newcommand{\ZZ}[0]{\mathbb{Z}}
\newcommand{\QQ}[0]{\mathbb{Q}}
\newcommand{\RR}[0]{\mathbb{R}}
\newcommand{\CC}[0]{\mathbb{C}}
\newcommand{\FF}[0]{\mathbb{F}}
\newcommand{\PP}[0]{\mathbb{P}}
\newcommand{\TT}[0]{\mathbb{T}}
\newcommand{\acts}[0]{\circlearrowright}
\newcommand{\explain}[2] {
	\begin{flalign*}
		 && \text{#2} && \text{#1}
	\end{flalign*}
}
\newcommand{\maketitleprint}[0]{ \begin{center}
	\begin{tikzpicture}[scale=3]
		\duck[graduate=gray!20!black, tassel=red!70!black]
	\end{tikzpicture}	
\end{center}
}

% theorem commands
\newtheoremstyle{c_remark}
	{}	% Space above
	{}	% Space below
	{}% Body font
	{}	% Indent amount
	{\bfseries}	% Theorem head font
	{}	% Punctuation after theorem head
	{.5em}	% Space after theorem head
	{\thmname{#1}\thmnumber{ #2}\thmnote{ \normalfont{\text{(#3)}}}}	% head content
\newtheoremstyle{c_definition}
	{3pt}	% Space above
	{3pt}	% Space below
	{}% Body font
	{}	% Indent amount
	{\bfseries}	% Theorem head font
	{}	% Punctuation after theorem head
	{.5em}	% Space after theorem head
	{\thmname{#1}\thmnumber{ #2}\thmnote{ \normalfont{\text{(#3)}}}}	% head content
\newtheoremstyle{c_plain}
	{3pt}	% Space above
	{3pt}	% Space below
	{\itshape}% Body font
	{}	% Indent amount
	{\bfseries}	% Theorem head font
	{}	% Punctuation after theorem head
	{.5em}	% Space after theorem head
	{\thmname{#1}\thmnumber{ #2}\thmnote{ \text{(#3)}}}	% head content

\theoremstyle{c_plain}
\newtheorem{theorem}{משפט}[section]
\newtheorem{lemma}[theorem]{למה}
\newtheorem{proposition}[theorem]{טענה}
\newtheorem*{proposition*}{טענה}
%\newtheorem{corollary}[theorem]{אין חלופה עברית}

\theoremstyle{c_definition}
\newtheorem{definition}[theorem]{הגדרה}
\newtheorem*{definition*}{הגדרה}
\newtheorem{example}{דוגמה}[section]
\newtheorem{exercise}{תרגיל}[section]

\theoremstyle{c_remark}
\newtheorem*{remark}{הערה}
\newtheorem*{solution}{פתרון}
\newtheorem{conclusion}[theorem]{מסקנה}
\newtheorem{notation}[theorem]{סימון}

% Questions related commands
\newcounter{question}
\setcounter{question}{1}
\newcounter{sub_question}
\setcounter{sub_question}{1}

\newcommand{\question}[1][0]{
	\ifthenelse{#1 = 0}{}{\setcounter{question}{#1}}
	\subsection{שאלה \arabic{question}}
	\addtocounter{question}{1}
	\setcounter{sub_question}{1}
}

\newcommand{\subquestion}[1][0]{
	\ifthenelse{#1 = 0}{}{\setcounter{sub_question}{#1}}
	\subsubsection{סעיף \localecounter{letters.gershayim}{sub_question}}
	\addtocounter{sub_question}{1}
}

% import lua and start of document
\directlua{common = require ('../common')}

\GetEnv{AUTHOR}

% headers
\author{\AUTHOR}
\date\today

\title{פתרון ממ''ן 12 – חשבון אינפיניטסימלי 2 (20475)}

\begin{document}
\maketitle
\maketitleprint{}
\section{שאלה 1}
נחשב את האינטגרלים הבאים

\subsection{סעיף א'}
\begin{align*}
	\int x^3{(1 - 3x^2)}^{10} dx
	& = \int x^3 \left( \sum_{k = 0}^{10} \binom{10}{k} {(-3x^2)}^k \right) dx \\
	& = \int \left( \sum_{k = 0}^{10} \binom{10}{k} {(-3)}^k x^{2k + 3} \right) dx \\
	& = C + \sum_{k = 0}^{10} \binom{10}{k} \frac{1}{2k + 4} {(-3)}^k x^{2k + 4} \\
	& = C + x^4 \sum_{k = 0}^{10} \binom{10}{k} \frac{{(-3)}^k}{2k + 4} x^{2k}
\end{align*}

\subsection{סעיף ב'}
\begin{align*}
	\int \frac{x e^{\arcsin x}}{\sqrt{1 - x^2}}dx
	& = \int \frac{\sin \arcsin(x) e^{\arcsin x}}{\sqrt{1 - x^2}}dx \\
	& = \left. \int \sin(t) e^t dt \right|_{dt = \frac{dx}{\sqrt{1 - x^2}}}^{t = \arcsin x} \\
	& = -\cos(t) e^t + \int \cos(t) e^t dt && \text{אינטגרציה בחלקים} \\
	& = -\cos(t) e^t + \sin(x) e^t - \int \sin(t) e^t dt && \text{אינטגרציה בחלקים} \\
	2 \int \sin(t) e^t dt & = \sin(x) - \cos(t) e^t \\
	\int \frac{x e^{\arcsin x}}{\sqrt{1 - x^2}}dx & = \frac{1}{2} \left(x - \sqrt{1 - x^2} e^{\arcsin x}\right) + C
\end{align*}

\subsection{סעיף ג'}
\begin{align*}
	\int {(x - 1)}^2 e^{2x} dx
	& = \frac{1}{2} e^{2x} {(x - 1)}^2 - \int \frac{1}{2} e^{2x} (2x - 2) dx \\
	& = \frac{1}{2} e^{2x} {(x - 1)}^2 + \int e^{2x} dx - \int e^{2x} x dx \\
	& = \frac{1}{2} e^{2x} {(x - 1)}^2 + \frac{1}{2} e^{2x} - \frac{1}{2} e^{2x} x + \int \frac{1}{2} e^{2x} dx \\
	& = \frac{1}{2} e^{2x} (x^2 - 3x + 2) + \frac{1}{4} e^{2x} \\
	& = \frac{1}{2} e^{2x} (x^2 - 3x + 2\frac{1}{2}) + C
\end{align*}

\subsection{סעיף ד'}
\begin{align*}
	\int \frac{4x + 1}{\sqrt{{(x^2 + 4x + 5)}^3}} dx
	& = \begin{vmatrix}
		t = x + 2 \\
		dt = dx
	\end{vmatrix} \\
	& = \int \frac{4t - 7}{\sqrt{{(t^2 + 1)}^3}} dt \\
	& = 2\int \frac{2t}{\sqrt{{(t^2 + 1)}^3}} dt - 7\int \frac{1}{\sqrt{{(t^2 + 1)}^3}} dt \\
\end{align*}
נחשב
\begin{align*}
	& \int \frac{2t}{\sqrt{{(t^2 + 1)}^3}} dt \\
	& \begin{vmatrix}
		u = t^2 + 1 \\
		du = 2t
	\end{vmatrix} \\
	= & \int \frac{du}{u^{3/2}} \\
	= & \int u^{-3/2} du \\
	= & -2 u^{-1/2} \\
	= & \frac{-2}{\sqrt{t^2 + 1}}
\end{align*}
נחשב גם
\begin{align*}
	& \int \frac{1}{\sqrt{{(t^2 + 1)}^3}} dt \\
	& \begin{vmatrix}
		t = \tan u \\
		dt = \frac{du}{\cos^2 u}
	\end{vmatrix} \\
	= & \int \frac{1}{\cos^2 u \sqrt{{(\tan^2 u + 1)}^3}} du \\
	= & \int \frac{1}{\cos^2 u \sqrt{{(\cos^{-2})}^3}} du \\
	= & \sin u = \sin \arctan t = \frac{t}{\sqrt{t^2 + 1}}
\end{align*}
ולכן נובע כי
\[
	2\int \frac{2t}{\sqrt{{(t^2 + 1)}^3}} dt - 7\int \frac{1}{\sqrt{{(t^2 + 1)}^3}} dt \\
	= 2 \frac{-2}{\sqrt{t^2 + 1}} - 7 \frac{t}{\sqrt{t^2 + 1}}
	= \frac{-7t - 4}{\sqrt{t^2 + 1}}
	= \frac{-7x - 18}{\sqrt{x^2 + 4x + 5}}
\]
לכן
\[
	\int_{-2}^{-1} \frac{4x + 1}{\sqrt{{(x^2 + 4x + 5)}^3}} dx
	= \left. \frac{-7x - 18}{\sqrt{x^2 + 4x + 5}} \right|_{-2}^{-1}
	= \frac{-7(-2) - 18}{\sqrt{{(-2)}^2 + 4 (-2) + 5}} - \frac{-7(-1) - 18}{\sqrt{{(-1)}^2 + 4 (-1) + 5}}
	= -4 + \frac{11}{\sqrt{2}}
\]

\subsection{סעיף ה'}
\[
	\int_{-\ln 2}^{\ln 2} e^{|x|} (1 + x^3 + \sin x) dx
	= \int_{-\ln 2}^{\ln 2} e^{|x|} (x^3 + \sin x) dx
	+ \int_{-\ln 2}^{\ln 2} e^{|x|} dx
\]
נגדיר
\[
	f(x) = e^{|x|} (1 + \sin x)
\]
מחישוב עולה כי
\[
	f(-x) = e^{|-x|} (-x^3 - \sin x) = - f(x)
\]
ולכן
\[
	\int_{-\ln 2}^{\ln 2} f(x) dx
	= \int_{-\ln 2}^{0} f(x) dx + \int_{0}^{\ln 2} f(x) dx
	= -\int_{0}^{\ln 2} f(x) dx + \int_{0}^{\ln 2} f(x) dx
	= 0
\]
עוד נראה כי
\[
	\int_{-\ln 2}^{\ln 2} e^{|x|} dx
	= \int_{-\ln 2}^{0} e^{|x|} dx + \int_{0}^{\ln 2} e^{|x|} dx
	= \int_{-\ln 2}^{0} e^{-x} dx + \int_{0}^{\ln 2} e^{x} dx
	= 2\int_{0}^{\ln 2} e^{x} dx
	= 2 (e^{\ln 2} - 1) = 4
\]

\section{שאלה 2}
עבור כל אחד מהאינטגרלים הבאים נקבע אם הוא מתכנס בהחלט, בתנאי, או כלל לא.

\subsection{סעיף א'}
\[
	\int_0^\infty \frac{\cos^2 x}{\sqrt{x}}dx
\]
נשים לב כי $f(x) = \frac{1}{\sqrt{x}}$ היא פונקציה מונוטונית יורדת וכי $\lim_{x \to \infty} f(x) = 0$. \\*
עוד נשים לב כי $g(x) = \cos^2 x$ היא פונקציה חסומה המקיימת $0 \le g(x) \le 1$ לכל $0 \le x$. \\*
לכן ממבחן דיריכלה נובע כי האינטגרל הנתון, אשר מהווה אינטגרל מכפלת הפונקציות $f, g$ מתכנס. \\*
נשים לב כי גם $f(x) g(x) = |f(x) g(x)|$ ולכן האינטגרל גם מתכנס בהחלט.

\subsection{סעיף ב'}
\[
	\int_1^\infty \frac{x \ln x}{(x^2 - 1){(\ln(x + 1))}^3} dx
\]
נגדיר
\[
	f(x) = \frac{1}{x^2 - 1}
\]
נשים לב כי
\[
	f'(x) = \frac{x^2 - 1 - 2x^2}{{(x^2 - 1)}^2} < 0
\]
נראה גם כי
\[
	\lim_{x \to \infty} f(x) = 0
\]
ולכן $f(x)$ מונוטונית יורדת ואפסה. \\*
נגדיר גם
\[
	g(x) = \frac{\ln x}{\ln^3(x + 1)}
\]
נראה כי עבור $x \ge 1$:
\[
	0 < \frac{\ln x}{\ln^3(x + 1)}
	\le \frac{\ln(x + 1)}{\ln^3(x + 1)}
	= \frac{1}{\ln^2(x + 1)} \le 1
\]
מצאנו כי $g(x)$ חסומה בתחום $[1, \infty)$. \\* % chktex 9
מצאנו כי הפונקציות $f$ ו־$g$ מקיימות את מבחן דיריכלה ולכן
\[
	\int_1^\infty f(x) g(x) dx
\]
מתכנס. נשים לב כי $0 \le f(x)$ לכל $x > 1$ ולכן מתקיים $|f(x) g(x)| = f(x) g(x)$ ובהתאם האינטגרל מתכנס בהחלט.

\subsection{סעיף ג'}
\[
	\int_0^\infty \frac{1}{x} (\sqrt[4]{x} + 1) \sin \left( 2 \sqrt{x} \right) dx
\]
נגדיר
\[
	f(x) = \frac{\sqrt[4]{x} + 1}{x}, g(x) = \sin\left(2\sqrt{x}\right)
\]

\end{document} % chktex 17
