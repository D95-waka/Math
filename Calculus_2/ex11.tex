\documentclass[a4paper]{article}

% packages
\usepackage{inputenc, fontspec, amsmath, amsthm, amsfonts, polyglossia, catchfile}
\usepackage[a4paper, margin=50pt, includeheadfoot]{geometry} % set page margins

% style
\AddToHook{cmd/section/before}{\clearpage}	% Add line break before section
\linespread{1.5}
\setcounter{secnumdepth}{0}		% Remove default number tags from sections
\setmainfont{Libertinus Serif}
\setsansfont{Libertinus Sans}
\setmonofont{Libertinus Mono}
\setdefaultlanguage{hebrew}
\setotherlanguage{english}

% operators
\DeclareMathOperator\cis{cis}
\DeclareMathOperator\Sp{Sp}
\DeclareMathOperator\tr{tr}
\DeclareMathOperator\im{Im}
\DeclareMathOperator\diag{diag}
\DeclareMathOperator*\lowlim{\underline{lim}}
\DeclareMathOperator*\uplim{\overline{lim}}

% commands
\renewcommand\qedsymbol{\textbf{משל}}
\newcommand{\NN}[0]{\mathbb{N}}
\newcommand{\ZZ}[0]{\mathbb{Z}}
\newcommand{\QQ}[0]{\mathbb{Q}}
\newcommand{\RR}[0]{\mathbb{R}}
\newcommand{\CC}[0]{\mathbb{C}}
\newcommand{\getenv}[2][] {
  \CatchFileEdef{\temp}{"|kpsewhich --var-value #2"}{\endlinechar=-1}
  \if\relax\detokenize{#1}\relax\temp\else\let#1\temp\fi
}
\newcommand{\explain}[2] {
	\begin{flalign*}
		 && \text{#2} && \text{#1}
	\end{flalign*}
}

% headers
\getenv[\AUTHOR]{AUTHOR}
\author{\AUTHOR}
\date\today

\usepackage{tikz}
\DeclareMathOperator\arcsinh{arcsinh}
\title{פתרון מטלה 11 – חשבון אינפיניטסימלי 2 (80132)}
% chktex-file 9

\begin{document}
\maketitle
\maketitleprint{}

\Question{}
\Subquestion{}
יהי $0 < S \in \RR$ ויהיו האינטגרלים
\[
	\int_{0}^{1} e^{-t} t^{S - 1}\ dt,
	\qquad
	\int_{1}^{\infty} e^{-t} t^{S - 1}\ dt
\]
נוכיח כי הם מתכנסים.
\begin{proof}
	נבחין כי במבחן ההשוואה ל־$\int e^{-t}\ dt$ (אשר מתכנס) נקבל כי האינטגרלים הנתונים מתכנסים אם ורק אם $t^{S - 1} \xrightarrow{x \to D} L < \infty$. \\*
	כאשר $S \le 1$ כמובן נקבל התכנות מהגרסה המורחבת של מבחן ההשוואה הגבולי מהמטלות הקודמות. \\*
	עבור $S > 1$ נשתמש בהשוואה לפונקציה $t^{-2}$ אשר חוסם מלמעלה את הפונקציה הנתונה לכל $x$ רלוונטי, ולכן נסיק כי יש התכנסות גם עבור $S > 1$ ובכלל עבור $0 < S$ כללי.
\end{proof}

\Subquestion{}
נגדיר $\Gamma : \RR^+ \to \RR$ המוגדר על־ידי
\[
	\Gamma(s) = \int_{0}^{\infty} e^{-t} t^{s - 1}\ dt
\]
אשר מצאנו בסעיף הקודם שמתכנס, ונוכיח כי גם $\forall s \in (0, \infty) : \Gamma(s + 1) = s \Gamma(s)$.
\begin{proof}
	נגדיר $N \in (0, \infty)$ כלשהו ויהי $0 < \delta < N$ ונקבל
	\[
		\int_{\delta}^{N} e^{-t} t^{N - 1}
	\]
\end{proof}

\Subquestion{}
נוכיח כי לכל $n \in \NN$ מתקיים $r(n + 1) = n! $.
\begin{proof}
	
\end{proof}

\end{document} % chktex 17
