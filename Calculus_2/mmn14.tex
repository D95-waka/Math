\documentclass[a4paper]{article}

% packages
\usepackage{inputenc, amsmath, amsthm, thmtools, amsfonts, amssymb, luacode, catchfile, tikzducks, hyperref}
\usepackage[a4paper, margin=50pt, includeheadfoot]{geometry} % set page margins
\usepackage[shortlabels]{enumitem}
\usepackage[skip=3pt, indent=0pt]{parskip}

% language
\usepackage[bidi=basic, layout=tabular, provide=*]{babel}
\babelprovide[main, import]{hebrew}
\babelprovide{rl}
\babelfont{rm}{Libertinus Serif}
\babelfont{sf}{Libertinus Sans}
\babelfont{tt}{Libertinus Mono}

% style
\AddToHook{cmd/section/before}{\clearpage}	% Add line break before section
\linespread{1.3}
\setcounter{secnumdepth}{0}		% Remove default number tags from sections, this won't do well with theorems
\AtBeginDocument{\setlength{\belowdisplayskip}{3pt}}
\AtBeginDocument{\setlength{\abovedisplayskip}{3pt}}

% operators
\DeclareMathOperator\cis{cis}
\DeclareMathOperator\Sp{Sp}
\DeclareMathOperator\tr{tr}
\DeclareMathOperator\im{Im}
\DeclareMathOperator\re{Re}
\DeclareMathOperator\diag{diag}
\DeclareMathOperator*\lowlim{\underline{lim}}
\DeclareMathOperator*\uplim{\overline{lim}}
\DeclareMathOperator\rng{rng}
\DeclareMathOperator\Sym{Sym}
\DeclareMathOperator\Arg{Arg}
\DeclareMathOperator\Log{Log}
\DeclareMathOperator\dom{dom}

% commands
%\renewcommand\qedsymbol{\textbf{מש''ל}}
%\renewcommand\qedsymbol{\fbox{\emoji{lizard}}}
\newcommand{\NN}[0]{\mathbb{N}}
\newcommand{\ZZ}[0]{\mathbb{Z}}
\newcommand{\QQ}[0]{\mathbb{Q}}
\newcommand{\RR}[0]{\mathbb{R}}
\newcommand{\CC}[0]{\mathbb{C}}
\newcommand{\FF}[0]{\mathbb{F}}
\newcommand{\PP}[0]{\mathbb{P}}
\newcommand{\TT}[0]{\mathbb{T}}
\newcommand{\acts}[0]{\circlearrowright}
\newcommand{\explain}[2] {
	\begin{flalign*}
		 && \text{#2} && \text{#1}
	\end{flalign*}
}
\newcommand{\maketitleprint}[0]{ \begin{center}
	\begin{tikzpicture}[scale=3]
		\duck[graduate=gray!20!black, tassel=red!70!black]
	\end{tikzpicture}	
\end{center}
}

% theorem commands
\newtheoremstyle{c_remark}
	{}	% Space above
	{}	% Space below
	{}% Body font
	{}	% Indent amount
	{\bfseries}	% Theorem head font
	{}	% Punctuation after theorem head
	{.5em}	% Space after theorem head
	{\thmname{#1}\thmnumber{ #2}\thmnote{ \normalfont{\text{(#3)}}}}	% head content
\newtheoremstyle{c_definition}
	{3pt}	% Space above
	{3pt}	% Space below
	{}% Body font
	{}	% Indent amount
	{\bfseries}	% Theorem head font
	{}	% Punctuation after theorem head
	{.5em}	% Space after theorem head
	{\thmname{#1}\thmnumber{ #2}\thmnote{ \normalfont{\text{(#3)}}}}	% head content
\newtheoremstyle{c_plain}
	{3pt}	% Space above
	{3pt}	% Space below
	{\itshape}% Body font
	{}	% Indent amount
	{\bfseries}	% Theorem head font
	{}	% Punctuation after theorem head
	{.5em}	% Space after theorem head
	{\thmname{#1}\thmnumber{ #2}\thmnote{ \text{(#3)}}}	% head content

\theoremstyle{c_plain}
\newtheorem{theorem}{משפט}[section]
\newtheorem{lemma}[theorem]{למה}
\newtheorem{proposition}[theorem]{טענה}
\newtheorem*{proposition*}{טענה}
%\newtheorem{corollary}[theorem]{אין חלופה עברית}

\theoremstyle{c_definition}
\newtheorem{definition}[theorem]{הגדרה}
\newtheorem*{definition*}{הגדרה}
\newtheorem{example}{דוגמה}[section]
\newtheorem{exercise}{תרגיל}[section]

\theoremstyle{c_remark}
\newtheorem*{remark}{הערה}
\newtheorem*{solution}{פתרון}
\newtheorem{conclusion}[theorem]{מסקנה}
\newtheorem{notation}[theorem]{סימון}

% Questions related commands
\newcounter{question}
\setcounter{question}{1}
\newcounter{sub_question}
\setcounter{sub_question}{1}

\newcommand{\question}[1][0]{
	\ifthenelse{#1 = 0}{}{\setcounter{question}{#1}}
	\subsection{שאלה \arabic{question}}
	\addtocounter{question}{1}
	\setcounter{sub_question}{1}
}

\newcommand{\subquestion}[1][0]{
	\ifthenelse{#1 = 0}{}{\setcounter{sub_question}{#1}}
	\subsubsection{סעיף \localecounter{letters.gershayim}{sub_question}}
	\addtocounter{sub_question}{1}
}

% import lua and start of document
\directlua{common = require ('../common')}

\GetEnv{AUTHOR}

% headers
\author{\AUTHOR}
\date\today

\title{פתרון ממ''ן 14 – חשבון אינפיניטסימלי 2 (20475)}

\begin{document}
\maketitle
\maketitleprint{}

\section{שאלה 1}
\subsection{סעיף א'}
\[
	\sum_{n = 1}^\infty \left( \sin n \cdot \frac{ (2^n + 5^n) {(n + 1)}^2}{n!} + \frac{\cos(1/n)}{\sqrt{n}} \right)
\]
נשים לב כי
\begin{align*}
	& \lim_{n \to \infty} \sin n \cdot \frac{ (2^n + 5^n) {(n + 1)}^2}{n!} \\
	\iff & \frac{\sin(n + 1)}{\sin n} \frac{ (2 \cdot 2^n + 5 \cdot 5^n) {(n + 2)}^2}{(n + 1)!} / \frac{ (2^n + 5^n) {(n + 1)}^2}{n!} < 1 \\
	\iff & \frac{\sin(n + 1)}{\sin n} (2 + \frac{3 \cdot 5^n}{2^n + 5^n}) \frac{{(n + 2)}^2}{{(n + 1)}^2}(n + 1) < 1
\end{align*}
ניתן לשים לב כי כמעט לכל $n$ אי־השוויון לא מתקיים, ולכן מהתנאי ההכרחי נובע כי הרכיב הראשון בטור לא מתכנס כלל. \\*
ידוע כי $\cos x$ חיובי ואיננו מתאפס עבור $0 < x \le 1$ ולכן
\[
	\frac{\cos(1/n)}{\sqrt{n}} \ge 0
\]
הביטוי חיובי או מתאפס לכל $n$ ולכן נוכל להסיק כי הוא לא מתכנס או מתבדר לאינסוף או מתכנס לערך סופי. \\*
מצאנו כי הטור מורכב מחיבור של טור מתבדר וטור חיובי ולכן נוכל להסיק כי הטור השלם מתבדר.

\subsection{סעיף ב'}
\[
	\sum_{n = 1}^{\infty} \frac{\cos n \cdot {(n + 1)}^n}{n^{n + 1}}
	= \sum_{n = 1}^{\infty} \frac{\cos n}{n} \cdot {(1 + \frac{1}{n})}^n
\]
ידוע כי הפונקציה ${(1 + \frac{1}{n})}^n$ מונוטונית עולה ומתכנסת ל־$e$ ולכן לכל $n$
\[
	{(1 + \frac{1}{n})}^n < e
\]
ומתקיים
\[
	0 \le
	\left\lvert \sum_{n = 1}^{\infty} \frac{\cos n}{n} \cdot {(1 + \frac{1}{n})}^n \right\rvert
	= \sum_{n = 1}^{\infty} \left\lvert \frac{\cos n}{n} \cdot {(1 + \frac{1}{n})}^n \right\rvert
	= \sum_{n = 1}^{\infty} \frac{\lvert \cos n \rvert}{n} \cdot {(1 + \frac{1}{n})}^n
	< \sum_{n = 1}^{\infty} \frac{e \cos n}{n}
	\overset{5.10}{=} e \sum_{n = 1}^{\infty} \frac{\cos n}{n}
	\tag{1}
\]
קיים הגבול
\[
	\lim_{n \to \infty} \sqrt{\left\lvert \frac{\cos n}{n} \right\rvert}
	= \lim_{n \to \infty} \frac{\sqrt{\left\lvert \cos n \right\rvert}}{\sqrt{n}} = 0
\]
ולכן ממשפט 5.16** נובע ישירות כי הטור
\[
	\sum_{n = 1}^{\infty} \frac{\cos n}{n}
\]
הוא טור מתכנס, ולכן מאי־שוויון $(1)$ ומשפט ההשוואה הראשון נובע כי
\[
	\sum_{n = 1}^{\infty} \frac{\cos n}{n} \cdot {(1 + \frac{1}{n})}^n
\]
הוא טור מתכנס בהחלט.

\subsection{סעיף ג'}
\[
	\sum_{n = 1}^{\infty} \left( 1 - n \sin \frac{1}{n}\right)
\]
הטור מתכנס אם ורק אם האינטגרל הבא מתכנס על־פי מבחן ההתכנסות האינטגרלי:
\[
	\int_{1}^{\infty} \left( 1 - x \sin \frac{1}{x}\right) dx \tag{1}
\]
מאי־השוויון הידוע $\sin x < x$ לכל $x > 0$ נוכל להסיק בתחום $1 < x$ גם:
\[
	 1 - x \sin \frac{1}{x} \le \frac{1}{x^2}
\]
וידוע כי
\[
	\int_{1}^{\infty} x^{-2} dx = \left. -x^{-1} \right|_1^\infty = 0 + 1 = 1
\]
לכן ממשפט 3.16 נובע כי האינטגרל $(1)$ מתכנס ולכן גם הטור. \\*
נשים לב כי כלל איברי הטור הם חיוביים ולכן הטור מתכנס גם בהחלט.

\section{שאלה 2}
נתונה סדרה $(a_n)$ כך ש־$a_n > 0$ ו־$a_n \ne 1$ לכל $n$. \\*
נוכיח כי הטור $\sum_{n = 1}^{\infty} a_n$ מתכנס אם ורק אם הטור $\sum_{n = 1}^{\infty} \frac{a_n}{a_n - 1}$ מתכנס.
\begin{proof}
	נניח כי הטור $\sum_{n = 1}^{\infty} a_n$ מתכנס, על־פי משפט 5.5 הסדרה $(a_n)$ אפסה, ולכן לכמעט כל $n$ מתקיים $0 < a_n < 1$. \\*
	בהתאם גם $0 < 1 - a_n < 1$, נגדיר סדרה $(b_n)$ כך שמתקיים
	\[
		b_n = \frac{a_n}{1 - a_n}
	\]
	ולכן לכמעט כל $n$ אי־השוויון $0 < b_n < 1$ מתקיים. \\*
	עוד נשים לב כי
	\[
		\lim_{n \to \infty} \frac{a_n}{b_n}
		= \lim_{n \to \infty} 1 - a_n
		= 1 - \lim_{n \to \infty} a_n = 1
	\]
	ולכן תנאי מבחן ההשוואה השני מתקיימים והטור
	\[
		- \sum_{n = 1}^{\infty} b_n = \sum_{n = 1}^{\infty} \frac{a_n}{a_n - 1}
	\]
	מתכנס והוכחנו את הכיוון הראשון של הטענה. \\*
	נניח כי הטור $\sum_{n = 1}^{\infty} \frac{a_n}{a_n - 1}$ מתכנס. \\*
	נניח בשלילה כי $(a_n)$ איננה אפסה, ולכן היא מתכנסת לערך סופי שונה מאפס או לערך לא סופי. \\*
	אילו היא מתכנסת לערך סופי שונה מאפס אז גבול הסדרה $\frac{a_n}{a_n - 1}$ הוא מספר שאיננו אפס או אינסוף בסתירה למשפט 5.5 ולנתון. \\*
	נניח אם כך כי הגבול של $(a_n)$ הוא אינסוף או מינוס אינסוף, אך בשני מקרים אלה גבול $\frac{a_n}{a_n - 1}$ יהיה $1$ בסתירה לאפסות הסדרה. \\*
	לכן $\lim_{n \to \infty} a_n = 0$. ידוע כמובן מהנתון ומהגבול כי לכמעט כל $n$ מתקיים $0 < a_n < 1$,
	ונוכל להגדיר מחדש את $(b_n)$ כבחלק הראשון של ההוכחה, ולכן נתון כי הטור
	\[
		\sum_{n = 1}^{\infty} b_n
		= -\sum_{n = 1}^{\infty} \frac{a_n}{a_n - 1}
	\]
	הוא טור מתכנס. נוכל אפוא להשתמש במשפט ההשוואה השני ולהוכיח כי גם הטור $\sum_{n = 1}^{\infty} a_n$ מתכנס. \\*
	מצאנו כי שני הטורים מתכנסים ומתבדרים יחדיו.
\end{proof}

\section{שאלה 3}
הסדרה $(u_n)$ מוגדרת באופן הבא:
\[
	u_1 = 1, u_{n + 1} = u_n \cdot \frac{1 + u_n}{1 + 2u_n}
\]
נוכיח כי הטור
\[
	\sum_{n = 1}^{\infty} u_n^2
\]
הוא טור מתכנס.
\begin{proof}
	נוכיח באידוקציה כי $(u_n)$ מקיימת $0 < u_n \le 1$ לכל $n$: \\*
	\textbf{בסיס האינדוקציה:}
	נתון כי $0 < u_1 = 1 \le 1$ \\*
	\textbf{מהלך האינדוקציה:}
	נניח כי $0 < u_n \le 1$. \\*
	אז כמובן $1 < 1 + u_n \le 2$ וגם $1 < 1 + 2u_n \le 3$ ולכן בהתאם
	\[
		0 < \frac{1 + u_n}{1 + 2u_n} \le \frac{2}{3}
		\implies 0 < u_n \frac{1 + u_n}{1 + 2u_n} \le \frac{2}{3} u_n \le \frac{2}{3}
		\implies 0 < u_{n + 1} \le 1
	\]
	מצאנו כי $(u_n)$ חסומה ובמהלך ההוכחה אף ראינו כי לכל $n$ מתקיים $u_{n + 1} < u_n$, דהינו $(u_n)$ היא סדרה מונוטונית יורדת וחסומה. \\*
	על־פי אינפי 1 הסדרה כמובן מתכנסת ואפסה. \\*
	נבחין כי 
	\[
		u_{n + 1} \le \frac{2}{3} u_n \le {(\frac{2}{3})}^2 u_{n - 1} \le \dots \le {(\frac{2}{3})}^n u_1 = {(\frac{2}{3})}^n
	\]
	דהינו $u_n$ קטן מערך סדרה הנדסית שמנתה $2/3$ ובהתאם למשפט ההשוואה הראשון הטור $\sum_{n = 1}^{\infty} u_n$ מתכנס. \\*
	ממשפט 5.9 נובע כי גם הטור $\sum_{n = 1}^{\infty} u_n - u_{n + 1}$ אשר מורכב מסכום סדרות שטוריהן מתכנסים, הוא טור מתכנס. \\*
	מחישוב עולה כי
	\[
		u_{n + 1}
		= \frac{u_n + u_n^2}{1 + 2u_n}
		\ge \frac{u_n + u_n^2}{3}
		\implies 3u_{n + 1} \ge u_n + u_n^2
		\implies 3u_{n + 1} - u_n \ge u_n^2 > 0
	\]
	ולכן באופן דומה גם $\sum_{n = 1}^{\infty} u_n^2$ מתכנס.
\end{proof}

\section{שאלה 4}
תהי $(a_n)$ סדרה אפסה ויהי $k \in \NN$ כך ש־$k > 1$. \\*
נגדיר לכל $n \ge 1$:
\[
	b_1 = \sum_{n = 1}^{k} a_n,
	b_{n + 1} = \sum_{n = kn + 1}^{kn + k} a_n
\]
נוכיח שהטור $\sum_{n = 1}^{\infty} a_n$ מתכנס אם ורק אם הטור $\sum_{n = 1}^{\infty} b_n$ מתכנס.
\begin{proof}
	נוכיח תחילה באינדוקציה כי מתקיים
	\[
		\sum_{n = 1}^{m} b_n = \sum_{n = 1}^{mk} a_n
	\]
	\textbf{בסיס האינדוקציה:}
	השוויון מתקיים על־פי נתוני השאלה. \\*
	\textbf{מהלך האינדוקציה:}
	נניח כי התנאי מתקיים ולכן
	\[
		\sum_{n = 1}^{m + 1} b_n
		= \sum_{n = 1}^{m} b_n + b_{m + 1}
		= \sum_{n = 1}^{mk} a_n + \sum_{n = km + 1}^{k(m + 1)} a_n
		= \sum_{n = 1}^{m(k + 1)} a_n
	\]
	והשלמנו את מהלך האינדוקציה. \\*
	\textbf{כיוון ראשון:}
	נניח כי הטור $\sum_{n = 1}^{\infty} a_n$ הוא מתכנס. \\*
	מהגדרת התכנסות הטור נובע כי מתקיים הגבול
	\[
		\lim_{m \to \infty} \sum_{n = 1}^{m} a_n
	\]
	ומהגדרת היינה לסדרות נסיק כי גם הגבול
	\[
		\lim_{mk \to \infty} \sum_{n = 1}^{mk} a_n
	\]
	הוא גבול מתכנס, וכמובן ש־$mk \to \infty$ אם ורק אם $m \to \infty$ ולכן מתכנס גם הגבול
	\[
		\lim_{m \to \infty} \sum_{n = 1}^{mk} a_n = \sum_{n = 1}^{m} b_n
	\]
	ומהגדרת הגבול נובע כי הטור $\sum_{n = 1}^{\infty} b_n$ מתכנס. \\*
	\textbf{כיוון שני:}
	נניח כי הטור $\sum_{n = 1}^{\infty} b_n$ מתכנס. \\*
	מהגדרת הגבול נובע
	\[
		\lim_{m \to \infty} \sum_{n = 1}^{m} b_n
		= \lim_{m \to \infty} \sum_{n = 1}^{mk} a_n
	\]
	אז ממשפט 5.11 נובע ישירות כי הטור $\sum_{n = 1}^{\infty} a_n$ מתכנס.
\end{proof}

\section{שאלה 5}
\subsection{סעיף א'}
נוכיח כי אם טור $\sum_{n = 1}^{\infty} a_n$ מתכנס בהחלט
וטור $\sum_{n = 1}^{\infty} b_n$ מתכנס בתנאי אז הטור $\sum_{n = 1}^{\infty} (a_n + b_n)$ מתכנס בתנאי.
\begin{proof}
	ממשפט 5.9 נובע כי הטור $\sum_{n = 1}^{\infty} b_n$ מתכנס. \\*
	ממשפט 5.24 אנו למדים כי טור החיוביים והשליליים של $\sum_{n = 1}^{\infty} b_n$ מתכנסים, בעוד אלה של $\sum_{n = 1}^{\infty} a_n$ מתבדרים ל־$\infty$. \\*
	מאינפי 1 אנו יודעים כי גבול סכום סדרות מתכנסת ומתבדרת הוא מתבדר ולכן גם גבול הטור המתאים לסדרה $|a_n + b_n|$ מתבדר ל־$\infty$. \\*
	מצאנו כי טור סכומי אברי הסדרות מתכנס בתנאי.
\end{proof}

\subsection{סעיף ב'}
נסתור את הטענה כי אם
\[
	\sqrt[n]{\lvert a_n \rvert} \le 1 - \frac{1}{n}
\]
לכל $n$ אז הטור $\sum_{n = 1}^{\infty} a_n$ מתכנס.
\begin{proof}[סתירה]
	נגדיר
	\[
		a_n = {(1 - \frac{1}{n})}^n
	\]
	ולכן כמובן
	\[
		\sqrt[n]{\lvert a_n \rvert}
		= \sqrt[n]{{(1 - \frac{1}{n})}^n}
		= 1 - \frac{1}{n}
		\le 1 - \frac{1}{n}
	\]
	ממסקנה 6.19 באינפי 1 נובע
	\[
		\lim_{n \to \infty} a_n = e^{-1}
	\]
	בסתירה לתנאי ההכרחי להתכנסות טורים, ולכן $\sum_{n = 1}^{\infty} a_n$ לא מתכנס.
\end{proof}

\subsection{סעיף ג'}
נוכיח כי אם $(a_n)$ סדרה יורדת ואפסה,
אז הטור
\[
	\sum_{n = 1}^{\infty} \sin 3n \cdot \frac{a_1 + a_2 + \cdots + a_n}{n}
\]
הוא טור מתכנס.
\begin{proof}
	ממשפט 2.51 באינפי 1 אנו למדים כי הסדרה $b_n = \frac{1}{n} \sum_{k = 1}^{n} a_k$ היא סדרה אפסה,
	וידוע כי כל איבר ב־$(a_n)$ חיובי ולכן גם כל איבר ב־$(b_n)$ חיובי ונובע כי היא סדרה מונוטונית יורדת ואפסה. \\*
	הטור $\sum_{n = 1}^{\infty} \sin 3n$ הוא טור חסום על־פי שאלה 33 ביחידה 5. \\*
	אז הסדרה $\sin 3n$ והסדרה $(b_n)$ מקיימות את התנאים למבחן דיריכלה והטור
	\[
		\sum_{n = 1}^{\infty} \sin 3n \cdot \frac{a_1 + a_2 + \cdots + a_n}{n}
	\]
	הוא טור מתכנס.
\end{proof}

\subsection{סעיף ד'}
תהי $f(x)$ פונקציה רציפה ואי־שלילית בתחום $[1, \infty)$, וידוע כי מתקיים $\lim_{x \to \infty} f(x) = 0$. \\* % chktex 9
נוכיח כי הטור $\sum_{n = 1}^{\infty} f(n)$ מתכנס אם ורק אם האינטגרל $\int_{1}^{\infty} f(x) dx$ מתכנס.
\begin{proof}
	% מהותית מגדירים סדרה $x_n$ כך שלכל $n$ מתקיים $f(x_{n + 1}) < f(x_n)$, זה אפשרי בגלל הגבול לאפס. \\*
	% אחרי שעשינו את זה משתמשים בגדול בהוכחה של משפט 5.19 בעמוד 139. \\*

	נגדיר סדרה חדשה $(x_n)$ כך שמתקיים $0 < f(x_{n + 1}) < f(x_n)$. \\*
	ניתן להגדיר סדרה כזו כמובן על־ידי שימוש בחיוביות ואפסות הפונקציה באינסוף. \\*
	על־ידי שימוש בסדרה זו נוכל להסיק ישירות כי הטור הנתון מתכנס אם ורק אם האינטגרל הנתון מתכנס אף הוא כמסקנה ממבחן ההתכנסות האינטגרלי.
\end{proof}

\end{document} % chktex 17
