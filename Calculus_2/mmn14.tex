\documentclass[a4paper]{article}

% packages
\usepackage{inputenc, fontspec, amsmath, amsthm, amsfonts, polyglossia, catchfile}
\usepackage[a4paper, margin=50pt, includeheadfoot]{geometry} % set page margins

% style
\AddToHook{cmd/section/before}{\clearpage}	% Add line break before section
\linespread{1.5}
\setcounter{secnumdepth}{0}		% Remove default number tags from sections
\setmainfont{Libertinus Serif}
\setsansfont{Libertinus Sans}
\setmonofont{Libertinus Mono}
\setdefaultlanguage{hebrew}
\setotherlanguage{english}

% operators
\DeclareMathOperator\cis{cis}
\DeclareMathOperator\Sp{Sp}
\DeclareMathOperator\tr{tr}
\DeclareMathOperator\im{Im}
\DeclareMathOperator\diag{diag}
\DeclareMathOperator*\lowlim{\underline{lim}}
\DeclareMathOperator*\uplim{\overline{lim}}

% commands
\renewcommand\qedsymbol{\textbf{משל}}
\newcommand{\NN}[0]{\mathbb{N}}
\newcommand{\ZZ}[0]{\mathbb{Z}}
\newcommand{\QQ}[0]{\mathbb{Q}}
\newcommand{\RR}[0]{\mathbb{R}}
\newcommand{\CC}[0]{\mathbb{C}}
\newcommand{\getenv}[2][] {
  \CatchFileEdef{\temp}{"|kpsewhich --var-value #2"}{\endlinechar=-1}
  \if\relax\detokenize{#1}\relax\temp\else\let#1\temp\fi
}
\newcommand{\explain}[2] {
	\begin{flalign*}
		 && \text{#2} && \text{#1}
	\end{flalign*}
}

% headers
\getenv[\AUTHOR]{AUTHOR}
\author{\AUTHOR}
\date\today

\title{פתרון ממ''ן 14 – חשבון אינפיניטסימלי 2 (20475)}

\begin{document}
\maketitle
\maketitleprint{}

\section{שאלה 1}
\subsection{סעיף א'}
\[
	\sum_{n = 1}^\infty \left( \sin n \cdot \frac{ (2^n + 5^n) {(n + 1)}^2}{n!} + \frac{\cos(1/n)}{\sqrt{n}} \right)
\]
נשים לב כי
\begin{align*}
	& \lim_{n \to \infty} \sin n \cdot \frac{ (2^n + 5^n) {(n + 1)}^2}{n!} \\
	\iff & \frac{\sin(n + 1)}{\sin n} \frac{ (2 \cdot 2^n + 5 \cdot 5^n) {(n + 2)}^2}{(n + 1)!} / \frac{ (2^n + 5^n) {(n + 1)}^2}{n!} < 1 \\
	\iff & \frac{\sin(n + 1)}{\sin n} (2 + \frac{3 \cdot 5^n}{2^n + 5^n}) \frac{{(n + 2)}^2}{{(n + 1)}^2}(n + 1) < 1
\end{align*}
ולכן גבול הרכיב הראשון בטור איננו אפס. \\*
ידוע כי $\cos x$ חיובי ואיננו מתאפס עבור $0 < x \le 1$ ולכן
\[
	\lim_{n \to \infty} \frac{\cos(1/n)}{\sqrt{n}} = 0
\]
אז מצאנו כי גבול הסדרה איננו אפס, ולכן ממשפט 5.5 נובע כי הטור איננו מתכנס.

\subsection{סעיף ב'}
\[
	\sum_{n = 1}^{\infty} \frac{\cos n \cdot {(n + 1)}^n}{n^{n + 1}}
	= \sum_{n = 1}^{\infty} \frac{\cos n}{n} \cdot {(1 + \frac{1}{n})}^n
\]
ידוע כי הפונקציה ${(1 + \frac{1}{n})}^n$ מונוטונית עולה ומתכנסת ל־$e$ ולכן לכל $n$
\[
	{(1 + \frac{1}{n})}^n < e
\]
ומתקיים
\[
	0 \le
	\left\lvert \sum_{n = 1}^{\infty} \frac{\cos n}{n} \cdot {(1 + \frac{1}{n})}^n \right\rvert
	= \sum_{n = 1}^{\infty} \left\lvert \frac{\cos n}{n} \cdot {(1 + \frac{1}{n})}^n \right\rvert
	= \sum_{n = 1}^{\infty} \frac{\lvert \cos n \rvert}{n} \cdot {(1 + \frac{1}{n})}^n
	< \sum_{n = 1}^{\infty} \frac{e \cos n}{n}
	\overset{5.10}{=} e \sum_{n = 1}^{\infty} \frac{\cos n}{n}
	\tag{1}
\]
קיים הגבול
\[
	\lim_{n \to \infty} \sqrt{\left\lvert \frac{\cos n}{n} \right\rvert}
	= \lim_{n \to \infty} \frac{\sqrt{\left\lvert \cos n \right\rvert}}{\sqrt{n}} = 0
\]
ולכן ממשפט 5.16** נובע ישירות כי הטור
\[
	\sum_{n = 1}^{\infty} \frac{\cos n}{n}
\]
הוא טור מתכנס, ולכן מאי־שוויון $(1)$ ומשפט ההשוואה הראשון נובע כי
\[
	\sum_{n = 1}^{\infty} \frac{\cos n}{n} \cdot {(1 + \frac{1}{n})}^n
\]
הוא טור מתכנס בהחלט.

\subsection{סעיף ג'}
\[
	\sum_{n = 1}^{\infty} \left( 1 - n \sin \frac{1}{n}\right)
\]
הטור מתכנס אם ורק אם האינטגרל הבא מתכנס
\[
	\int_{1}^{\infty} \left( 1 - x \sin \frac{1}{x}\right) dx \tag{1}
\]
נוכל לגזור את הביטויים פעמיים ולראות כי על־פי משפט 8.17 מאינפי 1 נובע לכל $1 \le x \le \infty$
\[
	 1 - x \sin \frac{1}{x} \le \frac{1}{x^2}
\]
וידוע כי
\[
	\int_{1}^{\infty} x^{-2} dx = \left. -x^{-1} \right|_1^\infty = 0 + 1 = 1
\]
לכן ממשפט 3.16 נובע כי האינטגרל $(1)$ מתכנס ולכן גם הטור. \\*
נשים לב כי כלל איברי הטור הם חיוביים ולכן הטור מתכנס גם בהחלט.

\section{שאלה 2}
נתונה סדרה $(a_n)$ כך ש־$a_n > 0$ ו־$a_n \ne 1$ לכל $n$. \\*
נוכיח כי הטור $\sum_{n = 1}^{\infty} a_n$ מתכנס אם ורק אם הטור $\sum_{n = 1}^{\infty} \frac{a_n}{a_n - 1}$ מתכנס.
\begin{proof}
	נניח כי הטור $\sum_{n = 1}^{\infty} a_n$ מתכנס, על־פי משפט 5.5 הסדרה $(a_n)$ אפסה, ולכן לכמעט כל $n$ מתקיים $0 < a_n < 1$. \\*
	בהתאם גם $0 < 1 - s_n < 1$, נגדיר סדרה $(b_n)$ כך שמתקיים
	\[
		b_n = \frac{a_n}{1 - a_n}
	\]
	ולכן לכמעט כל $n$ אי־השוויון $0 < b_n < 1$ מתקיים. \\*
	עוד נשים לב כי
	\[
		\lim_{n \to \infty} \frac{a_n}{b_n}
		= \lim_{n \to \infty} 1 - a_n
		= 1 - \lim_{n \to \infty} a_n = 1
	\]
	ולכן תנאי מבחן ההשוואה השני מתקיימים והטור
	\[
		- \sum_{n = 1}^{\infty} b_n = \sum_{n = 1}^{\infty} \frac{a_n}{a_n - 1}
	\]
	מתכנס והוכחנו את הכיוון הראשון של הטענה. \\*
	נניח כי הטור $\sum_{n = 1}^{\infty} \frac{a_n}{a_n - 1}$ מתכנס. \\*
	נניח בשלילה כי $(a_n)$ איננה אפסה, ולכן היא מתכנסת לערך סופי שונה מאפס או לערך לא סופי. \\*
	אילו היא מתכנסת לערך סופי שונה מאפס אז גבול הסדרה $\frac{a_n}{a_n - 1}$ הוא מספר שאיננו אפס או אינסוף בסתירה למשפט 5.5 ולנתון. \\*
	נניח אם כך כי הגבול של $(a_n)$ הוא אינסוף או מינוס אינסוף, אך בשני מקרים אלה גבול $\frac{a_n}{a_n - 1}$ יהיה $1$ בסתירה לאפסות הסדרה. \\*
	לכן $\lim_{n \to \infty} a_n = 0$. ידוע כמובן מהנתון ומהגבול כי לכמעט כל $n$ מתקיים $0 < a_n < 1$,
	ונוכל להגדיר מחדש את $(b_n)$ כבחלק הראשון של ההוכחה, ולכן נתון כי הטור
	\[
		\sum_{n = 1}^{\infty} b_n
		= -\sum_{n = 1}^{\infty} \frac{a_n}{a_n - 1}
	\]
	הוא טור מתכנס. נוכל אפוא להשתמש במשפט ההשוואה השני ולהוכיח כי גם הטור $\sum_{n = 1}^{\infty} a_n$ מתכנס. \\*
	מצאנו כי שני הטורים מתכנסים ומתבדרים יחדיו.
\end{proof}

\section{שאלה 3}
הסדרה $(u_n)$ מוגדרת באופן הבא:
\[
	u_1 = 1, u_{n + 1} = u_n \cdot \frac{1 + u_n}{1 + 2u_n}
\]
נוכיח כי הטור
\[
	\sum_{n = 1}^{\infty} u_n^2
\]
הוא טור מתכנס.
\begin{proof}
	נוכיח באידוקציה כי $(u_n)$ מקיימת $0 < u_n \le 1$ לכל $n$: \\*
	\textbf{בסיס האינדוקציה:}
	נתון כי $0 < u_1 = 1 \le 1$ \\*
	\textbf{מהלך האינדוקציה:}
	נניח כי $0 < u_n \le 1$. \\*
	אז כמובן $1 < 1 + u_n \le 2$ וגם $1 < 1 + 2u_n \le 3$ ולכן בהתאם
	\[
		0 < \frac{1 + u_n}{1 + 2u_n} \le \frac{2}{3}
		\implies 0 < u_n \frac{1 + u_n}{1 + 2u_n} \le \frac{2}{3} u_n \le \frac{2}{3}
		\implies 0 < u_{n + 1} \le 1
	\]
	מצאנו כי $(u_n)$ חסומה ובמהלך ההוכחה אף ראינו כי לכל $n$ מתקיים $u_{n + 1} < u_n$, דהינו $(u_n)$ היא סדרה מונוטונית יורדת וחסומה. \\*
	על־פי אינפי 1 הסדרה כמובן מתכנסת ואפסה. \\*
	נבחין כי 
	\[
		u_{n + 1} \le \frac{2}{3} u_n \le {(\frac{2}{3})}^2 u_{n - 1} \le \dots \le {(\frac{2}{3})}^n u_1 = {(\frac{2}{3})}^n
	\]
	דהינו $u_n$ קטן מערך סדרה הנדסית שמנתה $2/3$ ובהתאם למשפט ההשוואה הראשון הטור $\sum_{n = 1}^{\infty} u_n$ מתכנס. \\*
	ממשפט 5.9 נובע כי גם הטור $\sum_{n = 1}^{\infty} u_n - u_{n + 1}$ אשר מורכב מסכום סדרות שטוריהן מתכנסים, הוא טור מתכנס. \\*
	מחישוב עולה כי
	\[
		u_{n + 1}
		= \frac{u_n + u_n^2}{1 + 2u_n}
		\ge \frac{u_n + u_n^2}{3}
		\implies 3u_{n + 1} \ge u_n + u_n^2
		\implies 3u_{n + 1} - u_n \ge u_n^2 > 0
	\]
	ולכן באופן דומה גם $\sum_{n = 1}^{\infty} u_n^2$ מתכנס.
\end{proof}

\section{שאלה 4}
תהי $(a_n)$ סדרה אפסה ויהי $k \in \NN$ כך ש־$k > 1$. \\*
נגדיר לכל $n \ge 1$:
\[
	b_1 = \sum_{n = 1}^{k} a_n,
	b_{n + 1} = \sum_{n = kn + 1}^{kn + k} a_n
\]
נוכיח שהטור $\sum_{n = 1}^{\infty} a_n$ מתכנס אם ורק אם הטור $\sum_{n = 1}^{\infty} b_n$ מתכנס.
\begin{proof}
	שלום
\end{proof}

\end{document}

