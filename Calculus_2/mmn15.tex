\documentclass[a4paper]{article}

% packages
\usepackage{inputenc, fontspec, amsmath, amsthm, amsfonts, polyglossia, catchfile}
\usepackage[a4paper, margin=50pt, includeheadfoot]{geometry} % set page margins

% style
\AddToHook{cmd/section/before}{\clearpage}	% Add line break before section
\linespread{1.5}
\setcounter{secnumdepth}{0}		% Remove default number tags from sections
\setmainfont{Libertinus Serif}
\setsansfont{Libertinus Sans}
\setmonofont{Libertinus Mono}
\setdefaultlanguage{hebrew}
\setotherlanguage{english}

% operators
\DeclareMathOperator\cis{cis}
\DeclareMathOperator\Sp{Sp}
\DeclareMathOperator\tr{tr}
\DeclareMathOperator\im{Im}
\DeclareMathOperator\diag{diag}
\DeclareMathOperator*\lowlim{\underline{lim}}
\DeclareMathOperator*\uplim{\overline{lim}}

% commands
\renewcommand\qedsymbol{\textbf{משל}}
\newcommand{\NN}[0]{\mathbb{N}}
\newcommand{\ZZ}[0]{\mathbb{Z}}
\newcommand{\QQ}[0]{\mathbb{Q}}
\newcommand{\RR}[0]{\mathbb{R}}
\newcommand{\CC}[0]{\mathbb{C}}
\newcommand{\getenv}[2][] {
  \CatchFileEdef{\temp}{"|kpsewhich --var-value #2"}{\endlinechar=-1}
  \if\relax\detokenize{#1}\relax\temp\else\let#1\temp\fi
}
\newcommand{\explain}[2] {
	\begin{flalign*}
		 && \text{#2} && \text{#1}
	\end{flalign*}
}

% headers
\getenv[\AUTHOR]{AUTHOR}
\author{\AUTHOR}
\date\today

\title{פתרון ממ''ן 15 – חשבון אינפיניטסימלי 2 (20475)}

\begin{document}
\maketitle
\maketitleprint{}

\section{שאלה 1}
לכל $x \in [0, 1]$ נגדיר את $f_n(x)$ בצורה הבאה:
\[
	f_1(x) = x, f_{n + 1}(x) = f_n(x) - f_n^3(x)
\]
נוכיח שהסדרה $(f_n(x))$ מתכנסת לכל $x \in [0, 1]$ ונבדוק אם היא מתכנסת במידה שווה בקטע זה.
\begin{proof}
	יהי $x_0 \in [0, 1]$, ונגדיר סדרת מספרים $(a_n)$ על־ידי:
	\[
		a_1 = f_1(x_0), a_{n} = f_n(x_0)
	\]
	אז כמובן שנובע גם
	\[
		a_{n + 1}
		= f_{n + 1}(x_0)
		= f_n(x_0) - f_n^3(x_0) 
		= a_n - a_n^3
	\]
	עתה נוכיח באינדוקציה כי $0 \le a_n \le a$ לכל $n$: \\*
	בסיס האינדוקציה: נתון כי $a_1 = f_1(x_0) = x_0$ והטענה מתקיימת. \\*
	מהלך האינדוקציה: נניח כי $0 \le a_n \le 1$. \\*
	מתקיים כמובן גם $0 \le a_n^3 \le a_n$, ואחרי העברת אגפים $0 \le a_n - a_n^3 \le a_n \le 1$ והשלמנו את מהלך האינדוקציה. \\*
	עוד נראה כי אילו $a_n > 0$ אז כמובן שגם $a_n - a_n^3 < a_n$.
	במהלך ההוכחה מצאנו כי הסדרה מונוטונית במובן הרחב, ואז ראינו כי היא גם מונוטונית או שווה לאפס,
	לכן $a_n$ סדרה מונוטונית יורדת וחסומה, ולכן מתכנסת, דהינו $(f_n(x_0))$ מתכנסת. \\*
	מצאנו כי הסדרה מונוטונית וחסומה ולכן $f(x_0) = \lim_{n \to \infty} f_n(x_0) = 0$. \\*
	כמובן שמהגדרת הגבול הזה נובע כי לכל $\epsilon > 0$ שנבחר כמעט לכל $n$ מתקיים 
	\[
		f_n(x_0) = \lvert 0 - f_n(x_0) \rvert = \lvert f(x_0) - f_n(x_0) \rvert < \epsilon
	\]
	ולכן על־פי הגדרה הסדרה $(f_n)$ מתכנסת לפונקציה $f(x)$ במידה שווה בתחום $[0, 1]$.
\end{proof}

\section{שאלה 2}
נגדיר
\[
	f_n(x) = \frac{1}{\sin^2 x + {(1 + x^2)}^n}
\]
תחילה נמצא את ערכיה של $f(x)$: \\*
לכל $x \ne 0$ מתקיים הגבול $\lim_{n \to \infty} f_n(x) = \frac{1}{\infty} = 0$ ומתקיים $f(x) = 0$. \\*
כאשר $x = 0$ אז מתקיים $\lim_{n \to \infty} f_n(x) = f_n(x) = \frac{1}{1}$ ולכן $f(0) = 1$. \\*
נבדוק את התכנסות סדרת הפונקציות $(f_n(x))$ במידה שווה בקטעים הנתונים:

\subsection{סעיף א'}
נבדוק את ההתכנסות במידה שווה בקטע $\RR$. \\*
על־פי הגדרת $f_n(x)$ נובע כי היא רציפה לכל $n$, נניח בשלילה רציפות במידה שווה ונקבל ממשפט 6.4 כי $f(x)$ רציפה בכל $\RR$, בסתירה לאי־רציפותה ב־$x = 0$. \\*
לכן ההתכנסות היא לא במידה שווה ב־$\RR$.

\subsection{סעיף ב'}
נבדוק את התכנסותה במידה שווה בקטע $(0, 1)$. \\*
בקטע זה הפונקציה $f_n(x)$ היא חיובית מונוטונית יורדת, ידוע כי לכל $x$ בתחום $f(x) = 0$ ולכן $\lvert f_n(x) - f(x) \rvert = f_n(x)$. \\*
נבחן את $c_n = \sup\{ f_n(x) : x \in (0, 1)\}$. אנו יודעים כי לכל $x \in (0, 1)$ מתקיים $f_n(x) < 1$ וכי לכל $x < 1$ קיים $x_0$ כך ש־$f_n(x_0) > x$.
לכן מתקיים $c_n = 1$ לכל $n$ ובהתאם $\lim_{n \to \infty} c_n = 1$ ועל־כן מלמה 6.3 נובע כי סדרת הפונקציות לא מתכנסת במידה שווה בקטע.

\subsection{סעיף ג'}
נבדוק את התכנסותה במידה שווה בקטע $[\frac{1}{2023}, \infty)$. \\* % chktex 9
מסעיף ב' אנו יודעים כי הפונקציה $f_n(x)$ היא מונוטונית יורדת בתחום, ולכן היא מקבלת מקסימום בקטע בנקודה $x = \frac{1}{2023}$. \\*
בהתאם גם $c_n = \sup\{ \lvert f_n(x) - f(x) \rvert : x \in [\frac{1}{2023}, \infty)\} = f_n(\frac{1}{2023})$. \\* % chktex 9
הגבול $\lim_{n \to \infty} c_n = \frac{1}{\infty} = 0$ כמובן ובהתאם ללמה 6.3 סדרת הפונקציות מתכנסת במידה שווה בקטע.

\section{שאלה 3}
נבדוק התכנסות במידה שווה בתחום עבור הטורים הבאים:

\subsection{סעיף א'}
\[
	u(x) = \sum_{n = 1}^{\infty} x^3e^{-nx}
\]
בתחום התכנסותו. \\*
נראה כי
\[
	u(x) = x^3 \sum_{n = 1}^{\infty} e^{-nx}
\]
נבדוק את מנת הטור:
\[
	\left\lvert \frac{e^{-(n + 1)x}}{e^{-nx}} \right\rvert
	= \left\lvert e^{-(n + 1)x} \cdot e^{nx} \right\rvert
	= \left\lvert e^{-x} \right\rvert
	=  e^{-x} 
\]
נוכל כמובן למצוא מספר $q$ המקיים $e^{-x} \le q < 1$ אם ורק אם $x > 0$ ובהתאם הטור מתכנס על־פי מבחן דאלמבר לכל $x > 0$. \\*
באופן דומה נקבל ממבחן דאלמבר כי לכל $x \le 0$ הטור מתבדר. \\*
נבחין כי כאשר הטור מתכנס, הוא מהווה חיבור של פונקציות רציפות ולכן רציף גם הוא,
ולכן ממשפט דיני לטורים נובע כי $u(x)$ מתכנסת במידה שווה בתחום התכנסותה.

\subsection{סעיף ב'}
\[
	u(x) = \sum_{n = 1}^{\infty} {(-1)}^n \frac{x^{2n}}{\sqrt{n} \cdot 9^n}
\]
בתחום התכנסותו. \\*
ממבחן לייבניץ נובע כי אם $\frac{x^{2n}}{\sqrt{n} \cdot 9^n}$ היא סדרה אפסה ויורדת אז הטור מתכנס. \\*
נחשב את הגבול
\[
	\lim_{n \to \infty} \frac{x^{2n}}{\sqrt{n} \cdot 9^n}
	\lim_{n \to \infty} \frac{1}{\sqrt{n}} {\left(\frac{x^2}{9}\right)}^n
\]
ולכן הגבול מתכנס לאפס ואף מונוטוני יורד כאשר $\frac{x^2}{9} \le 1$ ומתבדר אחרת, דהינו הטור מתכנס כאשר $-3 \le x \le 3$. \\*
ממבחן המנה של דאלמבר נובע כי עבור כל $x$ שאיננו בתחום זה הטור מתבדר, ומצאנו כי הוא מתכנס רק כאשר $x \in [-3, 3]$. \\*
נבדוק את התכנסותו במידה שווה בתחום זה. \\*
מהסעיף השני והשלישי של מבחן לייבניץ אנו מסיקים כי עבור $m > n > N$ כא־$N$ טבעי כלשהו:
\[
	\lvert S_m - S_n \rvert
	= \lvert S_m - S - S_n + S \rvert
	\overset{(1)}{\le} \lvert S_m - S \rvert + \lvert S_n - S \rvert
	\overset{(2)}{\le} a_{n + 1} + a_{m + 1}
	\overset{(3)}{\le} 2a_{N + 1}
\]
$(1)$ אי שוויון המשולש \\*
$(2)$ משפט לייבניץ \\*
$(3)$ סדרה מונוטונית יורדת \\*
לכל $\epsilon > 0$ נוכל למצוא $N$ כך ש־$2a_{N + 1} < \epsilon$ בשל אפסות הסדרה, \\*
ולכן ממבחן קושי להתכנסות במידה שווה נובע כי $u(x)$ מתכנסת במידה שווה בקטע $[-3, 3]$.

\subsection{סעיף ג'}
\[
	u(x) = \sum_{n = 1}^{\infty} x^2 {(1 - x^2)}^{n - 1}
\]
בקטע $[-1, 1]$. \\*
תחילה נראה כי כאשר $x = 0$ גם $u(x) = 0$, ובאופן דומה כאשר $x = \pm 1$ אז $1 - x^2 = 0$ ומתקבל באופן דומה כי $u(x) = 0$. \\*
נניח עתה כי $x \in (-1, 1)$ ונקבע $t = 1 - x^2$:
\[
	u(x)
	= \sum_{n = 1}^{\infty} x^2 {(1 - x^2)}^{n - 1}
	= \sum_{n = 1}^{\infty} (1 - t) t^{n - 1}
	= (1 - t) \sum_{n = 1}^{\infty} t^{n - 1}
	= (1 - t) \sum_{n = 0}^{\infty} t^n
\]
זהו כמובן סכום סדרה הנדסית, נבחין כי $x^2 < 1$ ולכן גם $0 < t < 1$ ולכן
\[
	(1 - t) \sum_{n = 0}^{\infty} t^n
	= (1 - t) \frac{1}{1 - t}
	= 1
\]
דהינו לכל $x \in (-1, 0), x \in (0, 1)$ מתקיים $u(x) = 1$. \\*
כמובן אם כך שבנקודה $x = 0$ הפונקציה $u(x)$ איננה רציפה, אלא מקבלת נקודת אי־רציפות סליקה,
וממשפט 6.4* נסיק כי $u(x)$ לא רציפה במידה שווה בקטע $[-1, 1]$.

\section{שאלה 4}
\begin{flalign*}
	&& g(x) = \begin{cases}
		\frac{\ln(1 + x)}{x} & x > -1, x \ne 0 \\
		1 & x = 0
	\end{cases}
	&& \text{נתון} \\
	&& f(x) = \int_{0}^{x} g(t) \, dt
	&& \text{נגדיר גם}
\end{flalign*}
נפתח את $f(x)$ לטור חזקות מהצורה $\sum_{n = 0}^{\infty} a_n x^n$ ונמצא את תחום ההתכנסות של הטור. \\*
על־פי שאלה 5.3 מתקיים
\[
	\frac{1}{x} \ln(1 + x)
	= \frac{1}{x} \left( \sum_{n = 1}^{\infty} \frac{{(-1)}^{n - 1} x^n}{n} \right)
	= \sum_{n = 1}^{\infty} \frac{{(-1)}^{n - 1} x^{n - 1}}{n}
	= \sum_{n = 0}^{\infty} \frac{{(-1)}^n}{n + 1} x^n
\]
נשים לב כי בהצבה $x = 0$ מתקבל $1$ ולכן התניית הפונקציה המקורית נשמרת ומתקיים
\[
	g(x) = \sum_{n = 0}^{\infty} \frac{{(-1)}^n}{n + 1} x^n
\]
נבדוק את הגבול המוגדר על־ידי למה 6.11
\[
	\lim_{n \to \infty} \left\lvert \frac{{(-1)}^{n - 1}}{n} \middle/ \frac{{(-1)}^{n}}{n + 1} \right\rvert
	= \lim_{n \to \infty} \left\lvert \frac{-1}{n} \middle/ \frac{1}{n + 1} \right\rvert
	= \lim_{n \to \infty} \frac{n + 1}{n}
	= 1
\]
אז נובע כי רדיוס ההתכנסות של הטור הוא $R = 1$. \\*
ממשפט 6.12 נובע כי $g(x)$ אינטגרבילית ב־$(-1, 1)$ ומתקיים
\[
	f(x)
	= \int_{0}^{x} g(t) \, dt
	= \sum_{n = 0}^{\infty} \frac{{(-1)}^n}{{(n + 1)}^2} x^{n + 1}
\]
כמובן ש־$f(x)$ מוגדר ב־$(-1, 1)$, נבדוק את התכנסותו ב־$x = \pm 1$ בהצבה:
\[
	f(1)
	= \sum_{n = 0}^{\infty} \frac{{(-1)}^n}{{(n + 1)}^2}
\]
וממשפט לייבניץ נובע ישירות כי הטור $f(1)$ מתכנס.
\[
	f(-1)
	= \sum_{n = 0}^{\infty} \frac{{(-1)}^n}{{(n + 1)}^2} (-1) {(-1)}^n
	= -\sum_{n = 0}^{\infty} \frac{1}{{(n + 1)}^2}
\]
וממשפט 5.16** אנו רואים כי הגבול
\[
	\lim_{n \to \infty} \sqrt{\frac{1}{{(n + 1)}^2}}
	= \lim_{n \to \infty} \frac{1}{n + 1}
	= 0
\]
הוא גבול מתכנס לאפס ולכן הטור מתכנס ובהתאם $f(-1)$ מתכנס ומוגדר. \\*
מצאנו כי $f(x)$ מתכנסת בתחום $[-1, 1]$ ומהערה ג' אודות הוכחת משפט אבל נובע כי זוהי התכנסות במידה שווה.

\section{שאלה 5}
נוכיח או נפריך את הטענות הבאות:

\subsection{סעיף א'}
נוכיח כי אם קיים $x \in \RR$ כך שהטור $\sum_{n = 1}^{\infty} a_n x^n$ מתבדר והטור $\sum_{n = 1}^{\infty} b_n x^n$ מתכנס, \\*
אז רדיוס ההתכנסות של הטור $\sum_{n = 1}^{\infty} (a_n + b_n) x^n$ שווה לרדיוס ההתכנסות של הטור $\sum_{n = 1}^{\infty} a_n x^n$.
\begin{proof}
	נגדיר רדיוס ההתכנסות של $\sum_{n = 1}^{\infty} a_n x^n$ הוא $R_a$ ושל $\sum_{n = 1}^{\infty} a_n x^n$ הוא $R_b$. \\*
	אילו $R_b < R_a$ אז לכל $x \in (-R_b, R_b)$ שהיינו בוחרים ממשפט 6.10 היה נובע כי שני הטורים היו מתכנסים ב־$x$ בסתירה ל־$x$ הנתון. \\*
	לכן אנו יכולים לקבוע כי $R_b \ge R_a$. \\*
	לכל $x_0 \in (-R_a, R_a)$ אנו יודעים כי שני הטורים מתכנסים ל־$x_0$, לכן ממשפט 5.9 נובע כי גם
	\[
		\sum_{n = 1}^{\infty} (a_n + b_n) x_0^n
	\]
	הוא טור מתכנס, ולכן מהגדרת רדיוס התכנסות נסיק ישירות כי $R_c \ge R_a$ כאשר $R_c$ הוא רדיוס ההתכנסות של סדרת המחוברים. \\*
	לכל $x_1 \in (-R_b, R_b)$ כך שגם $x_1 \notin [-R_a, R_a]$ ידוע כי $\sum_{n = 1}^{\infty} a_n x^n$ הוא טור מתבדר וכי $\sum_{n = 1}^{\infty} b_n x^n$ מתכנס. \\*
	נוכל אפוא להסיק כי גם $\sum_{n = 1}^{\infty} (a_n + b_n) x^n$ מתבדר, שכן הוא מורכב מסכום טור מתבדר וטור מתכנס, ולכן נובע גם $R_c \le R_a$. \\*
	מצאנו כי $R_c \ge R_a$ וגם $R_c \le R_a$ ולכן בהכרח $R_c = R_a$.
\end{proof}

\subsection{סעיף ב'}
נפריך את הטענה כי אם $\lim_{n \to \infty} f_n(x) = f(x)$ לכל $x \in I$, וגם
\[
	\sup\{f_n(x) : x \in I\} \rightarrow \sup\{ f(x) : x \in I\},
	\inf\{f_n(x) : x \in I\} \rightarrow \inf\{ f(x) : x \in I\}
\]
אז $(f_n(x))$ מתכנסת במידה שווה ב־$I$.
\begin{proof}[]
	נפריך את הטענה בעזרת דוגמה נגדית. \\*
	נגדיר $f_n(x) = \frac{1}{x^n}$ בקטע $I = [1, 2]$. \\*
	נראה כי $f(1) = \lim_{n \to \infty} \frac{1}{1^n} = 1$ וכי באופן דומה לכל $1 < x \le 2$ מתקיים $f(x) = 0$. \\*
	אז מתקיימת הטענה הראשונה על־ידי בנייה. \\*
	נשים לב כי בקטע הפונקציה $f_n(x)$ היא פונקציה מונוטונית יורדת לכל התחום, וכי התחום סגור, לכן מתקיים
	\[
		\sup\{ f_n(x) : x \in I \} = f_n(1) = 1,
		\inf\{ f_n(x) : x \in I \} = f_n(2) = \frac{1}{2^n}
	\]
	וכמובן $\sup\{ f(x) : x \in I\} = 1, \inf\{ f(x) : x \in I\} = 0$, לכן
	\[
		\lim_{n \to \infty} \sup\{ f_n(x) : x \in I \} = f(1) = 1
		\lim_{n \to \infty} \inf\{ f_n(x) : x \in I \} = f(2) = 0
	\]
	אז נובע מהטענה כי $f(x)$ מתכנסת במידה שווה בקטע, אבל מצאנו כי $f(x)$ איננה רציפה בקטע, בעוד $f_n(x)$ רציפה לכל $n$ בסתירה למשפט 6.4.
\end{proof}

\section{שאלת רשות}
נגדיר
\[
	f(x) = \sum_{n = 0}^{\infty} \frac{\sin(2^n x)}{n!}
\]

\subsection{סעיף א'}
נוכיח כי $f$ גזירה אינסוף פעמים ב־$\RR$ ונמצא את טור טיילור שלה סביב $0$.
\begin{proof}
	נגזור פעמיים את סדרת הטור:
	\[
		\frac{d}{dx} \frac{d}{dx} \frac{\sin(2^n x)}{n!}
		= \frac{d}{dx} \frac{2^n \cos(2^n x)}{n!}
		= -\frac{2^{2n} \sin(2^n x)}{n!}
	\]
	מצאנו כי ביטוי זה בגזירה פעמיים שווה לעצמו למעט מכפלה בקבוע, ולכן הוא גזיר אינסוף פעמים. \\*
	באופן דומה, ניתן לגזור את כל איברי הטור אינסוף פעמים, ובהתאם גם אותו עצמו. \\*
	עוד נשים לב כי $2^n \cdot 0 = 0$ וכי $\cos(2^n \cdot 0) = 1, \sin(2^n \cdot 0) = 0$,
	לכן נוכל להשתמש בפיתוח טיילור הידוע עבור $\sin(x)$ בהתחשבות במכפלה בקבוע בלבד.
	\[
		\frac{\sin(2^n x)}{n!}
		= \sum_{k = 0}^{\infty} \frac{2^{n(2k + 1)} {(-1)}^k}{(2k + 1)! n!} x^{2k + 1}
	\]
	ולכן גם
	\begin{align*}
		f(x)
		& = \sum_{n = 0}^{\infty} \sum_{k = 0}^{\infty} \frac{2^{n(2k+1)} {(-1)}^k}{(2k + 1)! n!} x^{2k + 1} \\
		& = \sum_{k = 0}^{\infty} \sum_{n = 0}^{\infty} \frac{2^{n(2k+1)} {(-1)}^k}{(2k + 1)! n!} x^{2k + 1} \\
		& = \sum_{k = 0}^{\infty} \frac{{(-1)}^k}{(2k + 1)!} x^{2k + 1} \sum_{n = 0}^{\infty} \frac{{(2^{2k+1})}^n}{n!} \\
		& = \sum_{k = 0}^{\infty} \frac{{(-1)}^k e^{2^{2k+1}}}{(2k + 1)!} x^{2k + 1}
	\end{align*}
\end{proof}

\subsection{סעיף ב'}
נמצא את תחום ההתכנסות של טור $f(x)$ כטור חזקות. \\*
על־פי משפט 6.10 אנו רוצים למצוא את הגבול העליון של הסדרה המגדירה את הטור שמצאנו בסעיף הקודם.
כמובן שהגבול העליון מורכב רק מאבריו שלא מתאפסים של הטור, ולכן נוכל להתעלם מהם בחישוב הגבול.
כמובן שביצוע שורש על ביטוי זה לא ממש משפיע, הוא עדיין שואף לאינסוף חזק מאוד. \\*
לכן נובע ממשפט 6.10 כי $R = 0$.

\end{document} % chktex 17
