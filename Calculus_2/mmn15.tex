\documentclass[a4paper]{article}

% packages
\usepackage{inputenc, amsmath, amsthm, thmtools, amsfonts, amssymb, luacode, catchfile, tikzducks, hyperref}
\usepackage[a4paper, margin=50pt, includeheadfoot]{geometry} % set page margins
\usepackage[shortlabels]{enumitem}
\usepackage[skip=3pt, indent=0pt]{parskip}

% language
\usepackage[bidi=basic, layout=tabular, provide=*]{babel}
\babelprovide[main, import]{hebrew}
\babelprovide{rl}
\babelfont{rm}{Libertinus Serif}
\babelfont{sf}{Libertinus Sans}
\babelfont{tt}{Libertinus Mono}

% style
\AddToHook{cmd/section/before}{\clearpage}	% Add line break before section
\linespread{1.3}
\setcounter{secnumdepth}{0}		% Remove default number tags from sections, this won't do well with theorems
\AtBeginDocument{\setlength{\belowdisplayskip}{3pt}}
\AtBeginDocument{\setlength{\abovedisplayskip}{3pt}}

% operators
\DeclareMathOperator\cis{cis}
\DeclareMathOperator\Sp{Sp}
\DeclareMathOperator\tr{tr}
\DeclareMathOperator\im{Im}
\DeclareMathOperator\re{Re}
\DeclareMathOperator\diag{diag}
\DeclareMathOperator*\lowlim{\underline{lim}}
\DeclareMathOperator*\uplim{\overline{lim}}
\DeclareMathOperator\rng{rng}
\DeclareMathOperator\Sym{Sym}
\DeclareMathOperator\Arg{Arg}
\DeclareMathOperator\Log{Log}
\DeclareMathOperator\dom{dom}

% commands
%\renewcommand\qedsymbol{\textbf{מש''ל}}
%\renewcommand\qedsymbol{\fbox{\emoji{lizard}}}
\newcommand{\NN}[0]{\mathbb{N}}
\newcommand{\ZZ}[0]{\mathbb{Z}}
\newcommand{\QQ}[0]{\mathbb{Q}}
\newcommand{\RR}[0]{\mathbb{R}}
\newcommand{\CC}[0]{\mathbb{C}}
\newcommand{\FF}[0]{\mathbb{F}}
\newcommand{\PP}[0]{\mathbb{P}}
\newcommand{\TT}[0]{\mathbb{T}}
\newcommand{\acts}[0]{\circlearrowright}
\newcommand{\explain}[2] {
	\begin{flalign*}
		 && \text{#2} && \text{#1}
	\end{flalign*}
}
\newcommand{\maketitleprint}[0]{ \begin{center}
	\begin{tikzpicture}[scale=3]
		\duck[graduate=gray!20!black, tassel=red!70!black]
	\end{tikzpicture}	
\end{center}
}

% theorem commands
\newtheoremstyle{c_remark}
	{}	% Space above
	{}	% Space below
	{}% Body font
	{}	% Indent amount
	{\bfseries}	% Theorem head font
	{}	% Punctuation after theorem head
	{.5em}	% Space after theorem head
	{\thmname{#1}\thmnumber{ #2}\thmnote{ \normalfont{\text{(#3)}}}}	% head content
\newtheoremstyle{c_definition}
	{3pt}	% Space above
	{3pt}	% Space below
	{}% Body font
	{}	% Indent amount
	{\bfseries}	% Theorem head font
	{}	% Punctuation after theorem head
	{.5em}	% Space after theorem head
	{\thmname{#1}\thmnumber{ #2}\thmnote{ \normalfont{\text{(#3)}}}}	% head content
\newtheoremstyle{c_plain}
	{3pt}	% Space above
	{3pt}	% Space below
	{\itshape}% Body font
	{}	% Indent amount
	{\bfseries}	% Theorem head font
	{}	% Punctuation after theorem head
	{.5em}	% Space after theorem head
	{\thmname{#1}\thmnumber{ #2}\thmnote{ \text{(#3)}}}	% head content

\theoremstyle{c_plain}
\newtheorem{theorem}{משפט}[section]
\newtheorem{lemma}[theorem]{למה}
\newtheorem{proposition}[theorem]{טענה}
\newtheorem*{proposition*}{טענה}
%\newtheorem{corollary}[theorem]{אין חלופה עברית}

\theoremstyle{c_definition}
\newtheorem{definition}[theorem]{הגדרה}
\newtheorem*{definition*}{הגדרה}
\newtheorem{example}{דוגמה}[section]
\newtheorem{exercise}{תרגיל}[section]

\theoremstyle{c_remark}
\newtheorem*{remark}{הערה}
\newtheorem*{solution}{פתרון}
\newtheorem{conclusion}[theorem]{מסקנה}
\newtheorem{notation}[theorem]{סימון}

% Questions related commands
\newcounter{question}
\setcounter{question}{1}
\newcounter{sub_question}
\setcounter{sub_question}{1}

\newcommand{\question}[1][0]{
	\ifthenelse{#1 = 0}{}{\setcounter{question}{#1}}
	\subsection{שאלה \arabic{question}}
	\addtocounter{question}{1}
	\setcounter{sub_question}{1}
}

\newcommand{\subquestion}[1][0]{
	\ifthenelse{#1 = 0}{}{\setcounter{sub_question}{#1}}
	\subsubsection{סעיף \localecounter{letters.gershayim}{sub_question}}
	\addtocounter{sub_question}{1}
}

% import lua and start of document
\directlua{common = require ('../common')}

\GetEnv{AUTHOR}

% headers
\author{\AUTHOR}
\date\today

\title{פתרון ממ''ן 15 – חשבון אינפיניטסימלי 2 (20475)}

\begin{document}
\maketitle
\maketitleprint{}

\section{שאלה 1}
לכל $x \in [0, 1]$ נגדיר את $f_n(x)$ בצורה הבאה:
\[
	f_1(x) = x, f_{n + 1}(x) = f_n(x) - f_n^3(x)
\]
נוכיח שהסדרה $(f_n(x))$ מתכנסת לכל $x \in [0, 1]$ ונבדוק אם היא מתכנסת במידה שווה בקטע זה.
\begin{proof}
	יהי $x_0 \in [0, 1]$, ונגדיר סדרת מספרים $(a_n)$ על־ידי:
	\[
		a_1 = f_1(x_0), a_{n} = f_n(x_0)
	\]
	אז כמובן שנובע גם
	\[
		a_{n + 1}
		= f_{n + 1}(x_0)
		= f_n(x_0) - f_n^3(x_0) 
		= a_n - a_n^3
	\]
	עתה נוכיח באינדוקציה כי $0 \le a_n \le a$ לכל $n$: \\*
	בסיס האינדוקציה: נתון כי $a_1 = f_1(x_0) = x_0$ והטענה מתקיימת. \\*
	מהלך האינדוקציה: נניח כי $0 \le a_n \le 1$. \\*
	מתקיים כמובן גם $0 \le a_n^3 \le a_n$, ואחרי העברת אגפים $0 \le a_n - a_n^3 \le a_n \le 1$ והשלמנו את מהלך האינדוקציה. \\*
	עוד נראה כי אילו $a_n > 0$ אז כמובן שגם $a_n - a_n^3 < a_n$.
	במהלך ההוכחה מצאנו כי הסדרה מונוטונית במובן הרחב, ואז ראינו כי היא גם מונוטונית או שווה לאפס,
	לכן $a_n$ סדרה מונוטונית יורדת וחסומה, ולכן מתכנסת, דהינו $(f_n(x_0))$ מתכנסת. \\*
	מצאנו כי הסדרה מונוטונית וחסומה ולכן $f(x_0) = \lim_{n \to \infty} f_n(x_0) = 0$. \\*
	כמובן שמהגדרת הגבול הזה נובע כי לכל $\epsilon > 0$ שנבחר כמעט לכל $n$ מתקיים 
	\[
		f_n(x_0) = \lvert 0 - f_n(x_0) \rvert = \lvert f(x_0) - f_n(x_0) \rvert < \epsilon
	\]
	ולכן על־פי הגדרה הסדרה $(f_n)$ מתכנסת לפונקציה $f(x)$ במידה שווה בתחום $[0, 1]$.
\end{proof}

\section{שאלה 2}
נגדיר
\[
	f_n(x) = \frac{1}{\sin^2 x + {(1 + x^2)}^n}
\]
תחילה נמצא את ערכיה של $f(x)$: \\*
לכל $x \ne 0$ הגבול $\lim_{n \to \infty} f_n(x) = \frac{1}{\infty} = 0$ ומתקיים $f(x) = 0$. \\*
כאשר $x = 0$ אז הגבול $\lim_{n \to \infty} f_n(x) = f_n(x) = \frac{1}{1}$ ולכן $f(0) = 1$. \\*
נבדוק את התכנסות סדרת הפונקציות $(f_n(x))$ במידה שווה בקטעים הנתונים.

\subsection{סעיף א'}
נבדוק את ההתכנסות במידה שווה בקטע $\RR$. \\*
על־פי הגדרת $f_n(x)$ נובע כי היא רציפה לכל $n$, נניח בשלילה רציפות במידה שווה ונקבל ממשפט 6.4 כי $f(x)$ רציפה בכל $\RR$, בסתירה לאי־רציפותה ב־$x = 0$. \\*
לכן ההתכנסות היא לא במידה שווה ב־$\RR$.

\subsection{סעיף ב'}
נבדוק את התכנסותה במידה שווה בקטע $(0, 1)$. \\*
בקטע זה הפונקציה $f_n(x)$ היא חיובית מונוטונית יורדת, ידוע כי לכל $x$ בתחום $f(x) = 0$ ולכן $\lvert f_n(x) - f(x) \rvert = f_n(x)$. \\*
נבחן את $c_n = \sup\{ f_n(x) : x \in (0, 1)\}$. אנו יודעים כי לכל $x \in (0, 1)$ מתקיים $f_n(x) < 1$ וכי לכל $x < 1$ קיים $x_0$ כך ש־$f_n(x_0) > x$.
לכן מתקיים $c_n = 1$ לכל $n$ ובהתאם $\lim_{n \to \infty} c_n = 1$ ועל־כן מלמה 6.3 נובע כי סדרת הפונקציות לא מתכנסת במידה שווה בקטע.

\subsection{סעיף ג'}
נבדוק את התכנסותה במידה שווה בקטע $[\frac{1}{2023}, \infty)$. \\* % chktex 9
מסעיף ב' אנו יודעים כי הפונקציה $f_n(x)$ היא מונוטונית יורדת בתחום, ולכן היא מקבלת מקסימום בקטע בנקודה $x = \frac{1}{2023}$. \\*
בהתאם גם $c_n = \sup\{ \lvert f_n(x) - f(x) \rvert : x \in [\frac{1}{2023}, \infty)\} = f_n(\frac{1}{2023})$. \\* % chktex 9
הגבול $\lim_{n \to \infty} c_n = \frac{1}{\infty} = 0$ כמובן ובהתאם ללמה 6.3 סדרת הפונקציות מתכנסת במידה שווה בקטע.

\section{שאלה 3}
נבדוק התכנסות במידה שווה בתחום עבור הטורים הבאים:

\subsection{סעיף א'}
\[
	u(x) = \sum_{n = 1}^{\infty} x^3e^{-nx}
\]
בתחום התכנסותו. \\*
נראה כי
\[
	u(x) = x^3 \sum_{n = 1}^{\infty} e^{-nx}
\]
נבדוק את מנת הטור:
\[
	\left\lvert \frac{e^{-(n + 1)x}}{e^{-nx}} \right\rvert
	= \left\lvert e^{-(n + 1)x} \cdot e^{nx} \right\rvert
	= \left\lvert e^{-x} \right\rvert
	=  e^{-x} 
\]
נוכל כמובן למצוא מספר $q$ המקיים $e^{-x} \le q < 1$ אם ורק אם $x > 0$ ובהתאם הטור מתכנס על־פי מבחן דאלמבר לכל $x > 0$. \\*
באופן דומה נקבל ממבחן דאלמבר כי לכל $x \le 0$ הטור מתבדר. \\*
נבחין כי כאשר הטור מתכנס, הוא מהווה חיבור של פונקציות רציפות ולכן רציף גם הוא,
ולכן ממשפט דיני לטורים נובע כי $u(x)$ מתכנסת במידה שווה בתחום התכנסותה.

\subsection{סעיף ב'}
\[
	u(x) = \sum_{n = 1}^{\infty} {(-1)}^n \frac{x^{2n}}{\sqrt{n} \cdot 9^n}
\]
בתחום התכנסותו. \\*
ממבחן לייבניץ נובע כי אם $\frac{x^{2n}}{\sqrt{n} \cdot 9^n}$ היא סדרה אפסה אז הטור מתכנס. נעיר כי הסדרה חיובית ולכן זהו התנאי היחיד ההכרחי. \\*
נחשב את הגבול
\[
	\lim_{n \to \infty} \frac{x^{2n}}{\sqrt{n} \cdot 9^n}
	\lim_{n \to \infty} \frac{1}{\sqrt{n}} {\left(\frac{x^2}{9}\right)}^n
\]
ולכן הגבול מתכנס לאפס כאשר $\frac{x^2}{9} \le 1$ ומתבדר אחרת, דהינו הטור מתכנס כאשר $-3 \le x \le 3$. \\*
ממבחן המנה של דאלמבר נובע כי עבור כל $x$ שאיננו בתחום זה הטור מתבדר, ומצאנו כי הוא מתכנס רק כאשר $x \in [-3, 3]$. \\*
נבדוק את התכנסותו במידה שווה בתחום זה. \\*
מהסעיף השני והשלישי של מבחן לייבניץ אנו מסיקים כי עבור $m > n > N$ כא־$N$ טבעי כלשהו:
\[
	\lvert S_m - S_n \rvert
	= \lvert S_m - S - S_n + S \rvert
	\overset{(1)}{\le} \lvert S_m - S \rvert + \lvert S_n - S \rvert
	\overset{(2)}{\le} a_{n + 1} + a_{m + 1}
	\overset{(3)}{\le} 2a_{N + 1}
\]
$(1)$ אי שוויון המשולש \\*
$(2)$ משפט לייבניץ \\*
$(3)$ סדרה מונוטונית יורדת \\*
לכל $\epsilon > 0$ נוכל למצוא $N$ כך ש־$2a_{N + 1} < \epsilon$ בשל אפסות הסדרה, \\*
ולכן ממבחן קושי להתכנסות במידה שווה נובע כי $u(x)$ מתכנסת במידה שווה בקטע $[-3, 3]$.

\subsection{סעיף ג'}
\[
	u(x) = \sum_{n = 1}^{\infty} x^2 {(1 - x^2)}^{n - 1}
\]
בקטע $[-1, 1]$. \\*
תחילה נראה כי כאשר $x = 0$ גם $u(x) = 0$, ובאופן דומה כאשר $x = \pm 1$ אז $1 - x^2 = 0$ ומתקבל באופן דומה כי $u(x) = 0$. \\*
נניח עתה כי $x \in (-1, 1)$ ונקבע $t = 1 - x^2$:
\[
	u(x)
	= \sum_{n = 1}^{\infty} x^2 {(1 - x^2)}^{n - 1}
	= \sum_{n = 1}^{\infty} (1 - t) t^{n - 1}
	= (1 - t) \sum_{n = 1}^{\infty} t^{n - 1}
	= (1 - t) \sum_{n = 0}^{\infty} t^n
\]
זהו כמובן סכום סדרה הנדסית, נבחין כי $x^2 < 1$ ולכן גם $0 < t < 1$ ולכן
\[
	(1 - t) \sum_{n = 0}^{\infty} t^n
	= (1 - t) \frac{1}{1 - t}
	= 1
\]
דהינו לכל $x \in (-1, 0), x \in (0, 1)$ מתקיים $u(x) = 1$. \\*
כמובן אם כך שבנקודה $x = 0$ הפונקציה $u(x)$ איננה רציפה, אלא מקבלת נקודת אי־רציפות סליקה,
וממשפט 6.4* נסיק כי $u(x)$ לא רציפה במידה שווה בקטע $[-1, 1]$.

\section{שאלה 4}
\begin{flalign*}
	&& g(x) = \begin{cases}
		\frac{\ln(1 + x)}{x} & x > -1, x \ne 0 \\
		1 & x = 0
	\end{cases}
	&& \text{נתון} \\
	&& f(x) = \int_{0}^{x} g(t) \, dt
	&& \text{נגדיר גם}
\end{flalign*}
נפתח את $f(x)$ לטור חזקות מהצורה $\sum_{n = 0}^{\infty} a_n x^n$ ונמצא את תחום ההתכנסות של הטור. \\*
על־פי שאלה 5.3 מתקיים
\[
	\frac{1}{x} \ln(1 + x)
	= \frac{1}{x} \left( \sum_{n = 1}^{\infty} \frac{{(-1)}^{n - 1} x^n}{n} \right)
	= \sum_{n = 1}^{\infty} \frac{{(-1)}^{n - 1} x^{n - 1}}{n}
	= \sum_{n = 0}^{\infty} \frac{{(-1)}^n}{n + 1} x^n
\]
נשים לב כי בהצבה $x = 0$ מתקבל $1$ ולכן התניית הפונקציה המקורית נשמרת ומתקיים
\[
	g(x) = \sum_{n = 0}^{\infty} \frac{{(-1)}^n}{n + 1} x^n
\]
נבדוק את הגבול המוגדר על־ידי למה 6.11
\[
	\lim_{n \to \infty} \left\lvert \frac{{(-1)}^{n - 1}}{n} \middle/ \frac{{(-1)}^{n}}{n + 1} \right\rvert
	= \lim_{n \to \infty} \left\lvert \frac{-1}{n} \middle/ \frac{1}{n + 1} \right\rvert
	= \lim_{n \to \infty} \frac{n + 1}{n}
	= 1
\]
אז נובע כי רדיוס ההתכנסות של הטור הוא $R = 1$. \\*
ממשפט 6.12 נובע כי $g(x)$ אינטגרבילית ב־$(-1, 1)$ ומתקיים
\[
	f(x)
	= \int_{0}^{x} g(t) \, dt
	= \sum_{n = 0}^{\infty} \frac{{(-1)}^n}{{(n + 1)}^2} x^{n + 1}
\]
כמובן ש־$f(x)$ מוגדר ב־$(-1, 1)$, נבדוק את התכנסותו ב־$x = \pm 1$ בהצבה:
\[
	f(1)
	= \sum_{n = 0}^{\infty} \frac{{(-1)}^n}{{(n + 1)}^2}
\]
וממשפט לייבניץ נובע ישירות כי הטור $f(1)$ מתכנס.
\[
	f(-1)
	= \sum_{n = 0}^{\infty} \frac{{(-1)}^n}{{(n + 1)}^2} (-1) {(-1)}^n
	= -\sum_{n = 0}^{\infty} \frac{1}{{(n + 1)}^2}
\]
וממשפט 5.16** אנו רואים כי הגבול
\[
	\lim_{n \to \infty} \sqrt{\frac{1}{{(n + 1)}^2}}
	= \lim_{n \to \infty} \frac{1}{n + 1}
	= 0
\]
הוא גבול מתכנס לאפס ולכן הטור מתכנס ובהתאם $f(-1)$ מתכנס ומוגדר. \\*
מצאנו כי $f(x)$ מתכנסת בתחום $[-1, 1]$ ומהערה ג' אודות הוכחת משפט אבל נובע כי זוהי התכנסות במידה שווה.

\section{שאלה 5}
נוכיח או נפריך את הטענות הבאות:

\subsection{סעיף א'}
נוכיח כי קיים $x \in \RR$ כך שהטור $\sum_{n = 1}^{\infty} a_n x^n$ מתבדר והטור $\sum_{n = 1}^{\infty} b_n x^n$ מתכנס, \\*
אז רדיוס ההתכנסות של הטור $\sum_{n = 1}^{\infty} (a_n + b_n) x^n$ שווה לרדיוס ההתכנסות של הטור $\sum_{n = 1}^{\infty} a_n x^n$.
\begin{proof}
	ש
\end{proof}


\end{document} % chktex 17
