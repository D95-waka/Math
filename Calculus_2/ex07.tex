\documentclass[a4paper]{article}

% packages
\usepackage{inputenc, fontspec, amsmath, amsthm, amsfonts, polyglossia, catchfile}
\usepackage[a4paper, margin=50pt, includeheadfoot]{geometry} % set page margins

% style
\AddToHook{cmd/section/before}{\clearpage}	% Add line break before section
\linespread{1.5}
\setcounter{secnumdepth}{0}		% Remove default number tags from sections
\setmainfont{Libertinus Serif}
\setsansfont{Libertinus Sans}
\setmonofont{Libertinus Mono}
\setdefaultlanguage{hebrew}
\setotherlanguage{english}

% operators
\DeclareMathOperator\cis{cis}
\DeclareMathOperator\Sp{Sp}
\DeclareMathOperator\tr{tr}
\DeclareMathOperator\im{Im}
\DeclareMathOperator\diag{diag}
\DeclareMathOperator*\lowlim{\underline{lim}}
\DeclareMathOperator*\uplim{\overline{lim}}

% commands
\renewcommand\qedsymbol{\textbf{משל}}
\newcommand{\NN}[0]{\mathbb{N}}
\newcommand{\ZZ}[0]{\mathbb{Z}}
\newcommand{\QQ}[0]{\mathbb{Q}}
\newcommand{\RR}[0]{\mathbb{R}}
\newcommand{\CC}[0]{\mathbb{C}}
\newcommand{\getenv}[2][] {
  \CatchFileEdef{\temp}{"|kpsewhich --var-value #2"}{\endlinechar=-1}
  \if\relax\detokenize{#1}\relax\temp\else\let#1\temp\fi
}
\newcommand{\explain}[2] {
	\begin{flalign*}
		 && \text{#2} && \text{#1}
	\end{flalign*}
}

% headers
\getenv[\AUTHOR]{AUTHOR}
\author{\AUTHOR}
\date\today

\usepackage{tikz}
\DeclareMathOperator\arcsinh{arcsinh}
\title{פתרון מטלה 7 – חשבון אינפיניטסימלי 2 (80132)}
% chktex-file 9

\begin{document}
\maketitle
\maketitleprint{}

\Question{}
יהיו $a, b \in \RR$ כך ש־$0 < a < b$ ו־$\alpha \in \NN$. לכל $n \in \NN$ נסמן $q = q_n = \sqrt[n]{\frac{b}{a}}$ ו־$P_n = \{a, aq, \dots, aq^n = b\}$.

\Subquestion{}
נחשב את $\lim_{n \to \infty} \triangle(P_n)$.

נבחן את $aq^{i + 1} - aq^i = \triangle x_i$, ונקבל $aq^i (q - 1)$, כמובן ש־$q - 1$ הוא ערך קבוע ולכן עלינו לבחון את $aq^i$ בלבד. כמובן ידוע כי $b > a$ ולכן $q > 1$ ונקבל כי $i = n - 1$ הוא המקסימום, דהינו
\[
	\triangle(P_n) = aq^{n - 1} (q - 1)
\]
ולכן גם
\[
	\lim_{n \to \infty} \triangle(P_n)
	= \lim_{n \to \infty} b - {(\frac{b}{a})}^{(n-1)/n}
	= \frac{ab - b}{a}
\]

\Subquestion{}
תהי $f : [a, b] \to \RR$ המוגדרת על־ידי $f(x) = x^\alpha$. \\*
נוכיח כי
\[
	L(f, P_n) = (b^{\alpha + 1} - a^{\alpha + 1}) \cdot \frac{q - 1}{q^{\alpha + 1} - 1},
	\qquad
	U(f, P_n) = q^\alpha \cdot L(f, P_n)
\]
\begin{proof}
	נבחין כי $\alpha \ge 1$ ולכן $x^\alpha$ מונוטונית עולה, ולכן בכל מקטע הערך המינימלי הוא גם $m_i$, ונקבל
	\begin{align*}
		L(f, P_n)
		& = \sum_{i = 1}^{n} {(a q^{i - 1})}^\alpha (x_i - x_{i - 1})
		& = \sum_{i = 1}^{n} a^{\alpha} q^{(i - 1) \alpha}aq^{i - 1}(q - 1) \\
		& = a^{\alpha + 1} (q - 1) \sum_{i = 1}^{n} q^{(i - 1)(\alpha + 1)}
		& = a^{\alpha + 1} (q - 1) \frac{q^{(\alpha + 1) n} - 1}{q^{\alpha + 1} - 1} \\
		& = a^{\alpha + 1} ({(\frac{b}{a})}^{(\alpha + 1)} - 1) \frac{q - 1}{q^{\alpha + 1} - 1}
		& = (b^{\alpha + 1} - a^{\alpha + 1}) \frac{q - 1}{q^{\alpha + 1} - 1}
	\end{align*}
	כמובן עבור $M_i$ עלינו לבחור על־פי המונוטוניות את הקצה הימני ובהתאם נקבל
	\[
		U(f, P_n) = q^\alpha L(f, P_n)
	\]
\end{proof}

\Subquestion{}
נוכיח כי $f$ אינטגרבילית בקטע $[a, b]$, ונחשב את ערך האינטגרל.
\begin{proof}
	נבחין כי
	\[
		\lim_{n \to \infty} L(f, P_n)
		= \lim_{q \to 1} (b^{\alpha + 1} - a^{\alpha + 1}) \frac{q - 1}{q^{\alpha + 1} - 1}
		= \lim_{q \to 1} (b^{\alpha + 1} - a^{\alpha + 1}) \frac{1}{(\alpha + 1) q^{\alpha}}
		= \frac{1}{\alpha + 1} (b^{\alpha + 1} - a^{\alpha + 1})
	\]
	ועוד נראה כי
	\[
		\lim_{n \to \infty} U(f, P_n)
		\lim_{n \to \infty} q^\alpha L(f, P_n)
		= \frac{1}{\alpha + 1} (b^{\alpha + 1} - a^{\alpha + 1})
	\]
	הגבולות מתכנסים לערך משותף ולכן על־פי הגדרת אינטגרל דרבו נקבל
	\[
		\int_{a}^{b} x^\alpha dx
		= \frac{1}{\alpha + 1} (b^{\alpha + 1} - a^{\alpha + 1})
	\]
\end{proof}

\Question{}
יהיו $a, b \in \RR$ כך ש־$a < b$ ותהי $f : [a, b] \to \RR$ פונקציה חסומה.

\Subquestion{}
נוכיח כי לכל חלוקה $P$ של $[a, b]$ קיימות פונקציות מדרגות $g, h$ בקטע כך שמתקיים
\[
	L(f, P) = \int_{a}^{b} g(x) dx,
	\qquad
	U(f, P) = \int_{a}^{b} h(x) dx
\]
\begin{proof}
	תהי חלוקה $P = \{x_1, \dots, x_n\}$ ויהי $i \in [n]$, אז נסיק כי $f$ חסומה בקטע $[x_i, x_{i + 1}]$. \\*
	נסיק מהחסימות בקטע כי קיימים
	\[
		M_i = \sup_{x \in [x_i, x_{i + 1}]} f(x),
		\quad
		m_i = \inf_{x \in [x_i, x_{i + 1}]} f(x)
	\]
	נגדיר כמובן $g$ פונקציית מדרגות על־ידי $m_i$ ו־$h$ על־פי $M_i$ ומהגדרת סכום תחתון ועליון, ומלמה על פונקציות מדרגות כי
	\[
		L(f, P) = \int_{a}^{b} g(x) dx,
		\qquad
		U(f, P) = \int_{a}^{b} h(x) dx
	\]
\end{proof}

\Subquestion{}
נוכיח כי $f$ אינטגרבילית בקטע $[a, b]$ אם ורק אם לכל $\epsilon > 0$ קיימות פונקציות מדרגות $g, h$ בקטע כך ש־$g(x) \le f(x) \le h(x)$ ו־$\int_{a}^{b} (h(x) - g(x)) dx \le \epsilon$.
\begin{proof}
	\textbf{כיוון ראשון:}
	נניח כי $f$ אינטגרבילית בקטע $[a, b]$, לכן על־פי הגדרת אינטגרל דרבו נקבל
	\[
		\lim_{n \to \infty} U(f, P_n) - L(f, P_n) = 0
	\]
	ולכן על־פי הסעיף הקודם נקבל
	\[
		\lim_{n \to \infty} \int_{a}^{b} h(x) dx - \int_{a}^{b} g(x) dx = 0
	\]
	ולכן הטענה נכונה מהגדרת הגבול ישירות.

	\textbf{כיוון שני:}
	נניח כי לכל $\epsilon > 0$ קיימות פונקציות מדרגות $g, h$ המקיימות את תנאי הטענה. \\*
	נוכל להשתמש בלמה על אינטגרל פונקציית מדרגות ובחלוקה $P_n$ מתאימה כי
	\[
		g(x) \le L(f, P_n) \le f(x) \le U(f, P_n)
	\]
	ולכן מהאינטגרלים הנתונים נוכל להסיק גם
	\[
		U(f, P_n) - L(f, P_n) < \epsilon
	\]
	וזו למעשה הגדרת אינטגרביליות דרבו.
\end{proof}

\Question{}
תהי $f : [0, 1] \to \RR$ המוגדרת על־ידי
\[
	f(x) = \begin{cases}
		1 & x \in \{ 1 / n \mid n \in \NN \} \\
		0 & x \in [0, 1] \setminus \{ 1 / n \mid n \in \NN \}
	\end{cases}
\]

\Subquestion{}
נוכיח כי $f$ אינטגרבילית ב־$[\alpha, 1]$ עבור כל $\alpha \in (0, 1)$ ונמצא את ערך אינטגרל זה.
\begin{proof}
	נוכיח הוכחה אינדוקטיבית על־פי $\alpha$. \\*
	נניח תחילה כי $\frac{1}{2} < \alpha < 1$, נקבל בהתאם כי $f(x) = 0$ לכל $x \in (\alpha, 1)$ ולכן בהתאם $f$ אינטגרבילית בקטע $(\alpha, 1)$ ובהתאם גם ב־$[\alpha, 1]$. \\*
	עתה נניח כי $f$ אינטגרבילית בקטע $[1/m, 1]$ עבור $m \in \NN$, ונבחר $\alpha \in (\frac{1}{m + 2}, \frac{1}{m + 1})$, לכן $f(x) = 0$ ל־$x \in (\alpha, \frac{1}{m})$, מלבד נקודה יחידה $\frac{1}{m + 1}$.
	לכן ממשפט שהוכחנו בתרגול נקבל כי $f$ אינטגרבילית בקטע $[\alpha, \frac{1}{m}]$, והנחנו תחת הנחת האינדוקציה כי היא אף אינטגרבילית בקטע $[\frac{1}{m}, 1]$ ולכן מאדיטיביות נקבל אינטגרביליות ב־$[\alpha, 1]$.
	נבחין כי $\frac{1}{m + 1} \in [\alpha, 1]$ ולמעשה השלמנו את מהלך האינדוקציה ומצאנו כי הטענה נכונה.

	בתהליך זה אף מצאנו כי ערך אינטגרל זה הוא $0$, דהינו
	\[
		\int_{\alpha}^{1} f(x)\ dx = 0
	\]
\end{proof}

\Subquestion{}
נוכיח כי $f$ אינטגרבילית ב־$[0, 1]$ ונחשב את האינטגרל בקטע זה.
\begin{proof}
	נראה כי $f$ אינטגרבילית בקטע $[0, \alpha]$ על־פי הגדרת דרבו לאינטגרביליות. \\*
	מצפיפות הממשיים נוכל לקבוע כי בכל סביבה של $x = \frac{1}{m}$ לכל $m \in \NN$ קיים $x \in \RR \setminus \QQ$, ולכן בחלק זה $m_i = 0$ בעוד $M_i = 1$. \\*
	לעומת זאת, נבחין כי בחלוקה בה ישנה סביבה לכל $\frac{1}{m}$ (הנשמרת תחת עידון), נקבל כי $\Delta x_i$ קטן בעוד $M_i, m_i$ קבועים, בעוד מתקבלים חלקים חדשים בחלוקה אשר אפסים. \\*
	נוכל לקבוע אם כן שהגבול של הסכום העליון של הפונקציה הוא אפס, ובהתאם היא אינטגרבילית בקטע $[0, \alpha]$, ומאדיטיביות נקבל
	\[
		\int_{0}^{1} f(x)\ dx = 0
	\]
\end{proof}

\Question{}
יהיו $a, b \in \RR$ כאשר $a \le b$ ויהי $D \subseteq \RR$ מקטע.

\Subquestion{}
תהי $f : [a, b] \to \RR$ רציפה, נוכיח כי לכל חלוקה $P$ קיימת קבוצת ערכים $P^*$ כך שסכום רימן $S(f, P, P^*) = U(f, P)$.
\begin{proof}
	תהי $P$ חלוקה כלשהי, ויהי $x_0, x_1$ קטע בחלוקה זו, אנו יודעים כי $M_i = \sup_{x \in [x_0, x_1]} f(x)$ ומוויירשטראס השני נקבל בפרט $M_i = \max_{x \in [x_0, x_1]} f(x)$,
	נקבע $M_i = f(c)$, אנו יכולים להסיק כי קיים כזה מוויירשטראס, ועתה נגדיר כי $c \in P^*$. \\*
	בשיטה זו נבנה את $P^*$ לכל קטע בחלוקה $P$, ונקבל $S(f, P, P^*) = U(f, P)$.
\end{proof}

\Subquestion{}
נפריך את הטענה כי אם $f, g : D \to \RR$ רציפות במידה שווה ב־$D$, אז גם מכפלתן רציפה במידה שווה ב־$D$.

נבחר $D = \RR_+$ ו־$f(x) = g(x) = x$, לכן $(fg)(x) = x^2$, ומצאנו כי $x^2$ איננה רציפה במידה שווה.

\Subquestion{}
נוכיח כי הפונקציה $f : [0, 1] \to \RR$ המוגדרת על־ידי
\[
	f(x) = \begin{cases}
		2x \sin(\frac{1}{x}) - \cos(\frac{1}{x}) & 0 < x \le 1 \\
		0 & x = 0
	\end{cases}
\]
אינטגרבילית בקטע $[0, 1]$.
\begin{proof}
	נשתמש בתוצאת שאלה 2 ונקבל שהפנקציה, אשר ידוע כי חסומה, היא אינטגרבילית בקטע החסום.
\end{proof}

\Question{}
תהי $f : \RR \to \RR$ פונקציה המוגדרת על־ידי $f(x) = \sin(x^2)$, נוכיח כי $f$ איננה רציפה במידה שווה ב־$[1, \infty)$.
\begin{proof}
	נגדיר שתי סדרות נקודות ${(x_n)}_{n = 1}^\infty, {(y_n)}_{n = 1}^\infty$ בתחום על־ידי
	\[
		x_n = \sqrt{2\pi n},
		\quad
		y_n = \sqrt{\frac{\pi}{2} + 2\pi n}
	\]
	ונבחין כי
	\[
		\lim_{n \to \infty} y_n - x_n
		= \lim_{n \to \infty} \frac{\frac{\pi}{2} + 2\pi n - 2\pi n}{\sqrt{\frac{\pi}{2} + 2\pi n} - \sqrt{2\pi n}}
		= 0
	\]
	נגדיר $\epsilon = \frac{1}{2}$, ונראה כי
	\[
		\forall n \in \NN : |f(y_n) - f(x_n)| = |1 - 0| = 1 \ge \epsilon
	\]
	ולכן ממשפט אשר למדנו בהרצאה נקבל כי הפונקציה לא רציפה במידה שווה בתחום.
\end{proof}

\Question{}
\Subquestion{}
יהי $a \in \RR$ ותהי $f : [a, \infty) \to \RR$ פונקציה רציפה. \\*
נוכיח שאם קיים $c \in [a, \infty)$ כך ש־$f$ רציפה במידה שווה ב־$[c, \infty)$ אז היא רציפה במדיה שווה ב־$[a, \infty)$.
\begin{proof}
	ממשפט קנטור נובע כי $f$ רציפה במידה שווה ב־$[a, c]$. \\*
	יהי $\epsilon > 0$, אז קיימות $\delta_0, \delta_1 > 0$ עבורן:
	\[
		\forall x_1, y_1 \in [a, c], x_2, y_2 \in [c, \infty) : |x_1 - y_1| \le \delta_0 \implies |f(x_1) - f(y_1)| \le \epsilon, |x_2 - y_2| \le \delta_1 \implies |f(x_2) - f(y_2)| \le \epsilon
	\]
	אילו נבחר $\delta = \min\{ \delta_0, \delta_1 \}$ נקבל גם
	\[
		\forall x_1, y_1 \in [a, c], x_2, y_2 \in [c, \infty) : |x_1 - y_1| \le \delta \implies |f(x_1) - f(y_1)| \le \epsilon, |x_2 - y_2| \le \delta \implies |f(x_2) - f(y_2)| \le \epsilon
	\]
	ולכן נוכל לאחד את התחומים ולקבל
	\[
		\forall x, y \in [a, c] \cup [c, \infty) : |x - y| \le \delta \implies |f(x) - f(y)| \le \epsilon
	\]
	דהינו, הפונקציה $f$ רציפה במידה שווה בקטע $[a, \infty)$.
\end{proof}

\Subquestion{}
נוכיח שאם $f$ מקיימת $\lim_{x \to \infty} f(x) = L \in \RR$ אז $f$ רציפה במידה שווה ב־$[a, \infty)$.
\begin{proof}
	יהי $\epsilon > 0$, אז קיים $M > 0$ כך שמתקיים
	\[
		\forall x > M : |f(x) - L| \le \frac{\epsilon}{2}
	\]
	עתה נראה כי מאי־שוויון המשולש נובע
	\[
		\forall x, y > M : |f(x) - f(y)| \le |f(x) - L| + |f(y) - L| \le \frac{\epsilon}{2} + \frac{\epsilon}{2} = \epsilon
	\]
	לכן בפרט הפונקציה $f$ רציפה במידה שווה ב־$[M, \infty)$, וכמובן ממשפט קנטור היא רציפה במידה שווה ב־$[a, M]$, לכן נסיק מהסעיף הקודם כי $f$ רציפה במידה שווה ב־$[a, \infty)$.
\end{proof}

\Subquestion{}
נוכיח כי $f(x) = \frac{1}{x} \sin(x^3)$ רציפה במידה שווה ב־$(0, \infty)$.
\begin{proof}
	מספיק שנוכיח כי $f$ רציפה במידה שווה בתחום $(0, 1]$ ונקבל מהסעיף הקודם כי היא רציפה במידה שווה בכל התחום. \\*
	נבחין כי
	\[
		\lim_{x \to 0} f(x)
		= \lim_{x \to 0} x^2 \frac{\sin(x^3)}{x^3}
		= 0 \cdot 1
		= 0
	\]
	לכן נגדיר $h : [0, 1] \to \RR$ על־ידי
	\[
		h(x) = \begin{cases}
			f(x) & x \ne 0 \\
			0 & x = 0
		\end{cases}
	\]
	אז $h$ רציפה בקטע הסגור $[0, 1]$ ולכן ממשפט קנטור נובע כי היא רציפה במידה שווה בו. \\*
	מכאן נסיק כי היא רציפה במידה שווה גם בקטע $(0, 1]$, ולכן גם $f$ רציפה במידה שווה בקטע זה.
\end{proof}

\Question{}
בכל סעיף מוגדרת $f : [0, 6] \to \RR$. נבדוק כי $f$ אינטגרבילית בתחום הגדרתה, נחשב ביטוי מפורש ל־$F : [0, 6] \to \RR$ המוגדרת על־ידי $F(x) = \int_{0}^{x} f(t)\ dt$, ונבדוק את תחום הגזירות והנגזרת של $F$.

\subsection{i.}
\[
	f(t) = t
\]
מצאנו בשאלה 1 כי פונקציה זו אינטגרבילית ואף קיבלנו כי
\[
	F(x) = \frac{1}{2} x^2
\]
קיבלנו פרבולה, אנו יודעים כי היא גזירה בכל התחום, ומתקיים
\[
	F'(x) = x
\]

\subsection{ii.}
\[
	f(t) = \begin{cases}
		0 & 0 \le t < 4 \\
		1 & t = 4 \\
		0 & 4 < t \le 6
	\end{cases}
\]
נבחין כי פונקציה זו שקולה לפונקציה $y = 0$ מלבד בנקודה יחידה, ולכן נסיק ממשפט מהתרגול כי היא אינטגרבילית וכי מתקיים
\[
	F(x) = 0
\]
כמובן פונקציה זו גזירה בכל נקודה, ומתקיים
\[
	F'(x) = 0
\]

\subsection{iii.}
\[
	f(t) = \begin{cases}
		\frac{1}{2} & 0 \le t < 4 \\
		3 & 4 \le t \le 6
	\end{cases}
\]
נבחין כי פונקציה זו היא פונקציית מדרגות ולכן היא אינטגרבילית. \\*
נשתמש בשאלה 1 ופירוק הפונקציה למקטעים ונקבל כי
\[
	F(x) = \begin{cases}
		\frac{1}{2}x & 0 \le x < 4 \\
		2 + 3x & 4 \le x \le 6
	\end{cases}
\]
פונקציה זו גזירה בכל נקודה בתחום מלבד $x = 4$, ובשאר הנקודות מתקיים
\[
	F'(x) = \begin{cases}
		\frac{1}{2} & 0 \le x < 4 \\
		3 & 4 < x \le 6
	\end{cases}
\]

\end{document} % chktex 17
