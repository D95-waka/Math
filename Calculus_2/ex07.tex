\documentclass[a4paper]{article}

% packages
\usepackage{inputenc, fontspec, amsmath, amsthm, amsfonts, polyglossia, catchfile}
\usepackage[a4paper, margin=50pt, includeheadfoot]{geometry} % set page margins

% style
\AddToHook{cmd/section/before}{\clearpage}	% Add line break before section
\linespread{1.5}
\setcounter{secnumdepth}{0}		% Remove default number tags from sections
\setmainfont{Libertinus Serif}
\setsansfont{Libertinus Sans}
\setmonofont{Libertinus Mono}
\setdefaultlanguage{hebrew}
\setotherlanguage{english}

% operators
\DeclareMathOperator\cis{cis}
\DeclareMathOperator\Sp{Sp}
\DeclareMathOperator\tr{tr}
\DeclareMathOperator\im{Im}
\DeclareMathOperator\diag{diag}
\DeclareMathOperator*\lowlim{\underline{lim}}
\DeclareMathOperator*\uplim{\overline{lim}}

% commands
\renewcommand\qedsymbol{\textbf{משל}}
\newcommand{\NN}[0]{\mathbb{N}}
\newcommand{\ZZ}[0]{\mathbb{Z}}
\newcommand{\QQ}[0]{\mathbb{Q}}
\newcommand{\RR}[0]{\mathbb{R}}
\newcommand{\CC}[0]{\mathbb{C}}
\newcommand{\getenv}[2][] {
  \CatchFileEdef{\temp}{"|kpsewhich --var-value #2"}{\endlinechar=-1}
  \if\relax\detokenize{#1}\relax\temp\else\let#1\temp\fi
}
\newcommand{\explain}[2] {
	\begin{flalign*}
		 && \text{#2} && \text{#1}
	\end{flalign*}
}

% headers
\getenv[\AUTHOR]{AUTHOR}
\author{\AUTHOR}
\date\today

\usepackage{tikz}
\DeclareMathOperator\arcsinh{arcsinh}
\title{פתרון מטלה 7 – חשבון אינפיניטסימלי 2 (80132)}

\begin{document}
\maketitle
\maketitleprint{}

\Question{}
יהיו $a, b \in \RR$ כך ש־$0 < a < b$ ו־$\alpha \in \NN$. לכל $n \in \NN$ נסמן $q = q_n = \sqrt[n]{\frac{b}{a}}$ ו־$P_n = \{a, aq, \dots, aq^n = b\}$.

\Subquestion{}
נחשב את $\lim_{n \to \infty} \triangle(P_n)$.

נבחן את $aq^{i + 1} - aq^i = \triangle x_i$, ונקבל $aq^i (q - 1)$, כמובן ש־$q - 1$ הוא ערך קבוע ולכן עלינו לבחון את $aq^i$ בלבד. כמובן ידוע כי $b > a$ ולכן $q > 1$ ונקבל כי $i = n - 1$ הוא המקסימום, דהינו
\[
	\triangle(P_n) = aq^{n - 1} (q - 1)
\]
ולכן גם
\[
	\lim_{n \to \infty} \triangle(P_n)
	= \lim_{n \to \infty} b - {(\frac{b}{a})}^{(n-1)/n}
	= \frac{ab - b}{a}
\]

\Subquestion{}
תהי $f : [a, b] \to \RR$ המוגדרת על־ידי $f(x) = x^\alpha$. \\*
נוכיח כי
\[
	L(f, P_n) = (b^{\alpha + 1} - a^{\alpha - 1}) \cdot \frac{q - 1}{q^{\alpha + 1} - 1},
	\qquad
	U(f, P_n) = q^\alpha \cdot L(f, P_n)
\]
\begin{proof}
	נבחר חלוקה $P_n$ אחידה ולכן מהחלוקה נקבל כי $f(P_n)$ מתלכדת עם החלוקה מהסעיף הקודם, ונוכל להסיק מיידית כי $U(f, P_n) = aq^{\alpha - 1} (q - 1)$
\end{proof}

\end{document}
