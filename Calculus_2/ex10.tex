\documentclass[a4paper]{article}

% packages
\usepackage{inputenc, fontspec, amsmath, amsthm, amsfonts, polyglossia, catchfile}
\usepackage[a4paper, margin=50pt, includeheadfoot]{geometry} % set page margins

% style
\AddToHook{cmd/section/before}{\clearpage}	% Add line break before section
\linespread{1.5}
\setcounter{secnumdepth}{0}		% Remove default number tags from sections
\setmainfont{Libertinus Serif}
\setsansfont{Libertinus Sans}
\setmonofont{Libertinus Mono}
\setdefaultlanguage{hebrew}
\setotherlanguage{english}

% operators
\DeclareMathOperator\cis{cis}
\DeclareMathOperator\Sp{Sp}
\DeclareMathOperator\tr{tr}
\DeclareMathOperator\im{Im}
\DeclareMathOperator\diag{diag}
\DeclareMathOperator*\lowlim{\underline{lim}}
\DeclareMathOperator*\uplim{\overline{lim}}

% commands
\renewcommand\qedsymbol{\textbf{משל}}
\newcommand{\NN}[0]{\mathbb{N}}
\newcommand{\ZZ}[0]{\mathbb{Z}}
\newcommand{\QQ}[0]{\mathbb{Q}}
\newcommand{\RR}[0]{\mathbb{R}}
\newcommand{\CC}[0]{\mathbb{C}}
\newcommand{\getenv}[2][] {
  \CatchFileEdef{\temp}{"|kpsewhich --var-value #2"}{\endlinechar=-1}
  \if\relax\detokenize{#1}\relax\temp\else\let#1\temp\fi
}
\newcommand{\explain}[2] {
	\begin{flalign*}
		 && \text{#2} && \text{#1}
	\end{flalign*}
}

% headers
\getenv[\AUTHOR]{AUTHOR}
\author{\AUTHOR}
\date\today

\usepackage{tikz}
\DeclareMathOperator\arcsinh{arcsinh}
\title{פתרון מטלה 10 – חשבון אינפיניטסימלי 2 (80132)}
% chktex-file 9

\begin{document}
\maketitle
\maketitleprint{}

\Question{}
\Subquestion{}
נוכיח כי אם ${(a_n)}_{n = 1}^\infty$ סדרה המקיימת $\lim_{n \to \infty} a_n = 1$ ותהי ${(b_n)}_{n = 1}^\infty$ סדרה חסומה אז $\uplim_{n \to \infty} a_n b_n = \uplim_{n \to \infty} b_n$
\begin{proof}
	מצאנו כי סדרה מתכנסת אם ורק אם הגבולות העליון והתחתון שלה שווים ולכן נסיק כי $\uplim_{n \to \infty} a_n = 1$. \\*
	לכן אם נבחר את סדרת הסופרמומים של $(a_n b_n)$ נקבל את מכפלת הסופרמומים ומצאנו כי אלו הן סדרות מוגדרות היטב. \\*
	נסיק אם כן שהגבול העליון של מכפלת הסדרות קיים במובן הרחב, ובפרט כאשר $(b_n)$ חסומה וקיים גבול עליון במובן המצומצם נקבל כי גבול המכפלות מוגדר.

	אנו יודעים כי הסופרמום הוא הגבול החלקי הגדול ביותר ונסיק כי תת־סדרה של $b_n$ הגדולה ביותר מקבלת גבול $1$ עבור $a_n$ ונסיק כי מתקיים
	\[
		\uplim_{n \to \infty} a_n b_n = \uplim_{n \to \infty} b_n
	\]
\end{proof}

\Subquestion{}
נוכיח כי קיימת סדרה שקבוצת הגבולות החלקיים שלה שווה ל־$\{ \frac{1}{n} \mid n \in \NN \}$.
\begin{proof}
	נגדיר סדרה ${(l_n)}_1^k$ על־ידי $l_n = \frac{1}{n}$ עבור $1 \le n \le k \in \NN$. \\*
	עתה נגדיר ${(a_n)}_{n = 1}^\infty$ על־ידי $l_1, l_2, \dots, l_k, \dots$ סדרה מהצורה
	\[
		1,
		1, \frac{1}{2},
		1, \frac{1}{2}, \frac{1}{3},
		\dots
	\]
	קל למצוא סדרת אינדקסים עולה ממש $(n_k)$ כך שנקבל $a_{n_k} = \frac{1}{m}$ עבור $m \in \NN$ קבוע כלשהו, ולכן זהו גבול חלקי שלה ומצאנו כי התנאי מתקיים.
\end{proof}

\Question{}
תהינה $f, g : [1, \infty) \to \RR$ פונקציות אינטגרביליות ב־$[1, N]$ עבור כל $1 < N$. \\*
נסתור את הטענה כי אם הסדרה ${( \int_{1}^{n} f(x)\ dx)}_{n = 1}^\infty$ מתכנסת אז $\int_{1}^{\infty} f(x)\ dx$ מתכנס אף הוא, על־ידי דגומה נגדית
\begin{proof}
	נגדיר $f(x) = \sin(2\pi x)$ וכלן לכל $N > 1$ נקבל $\int_1^N f(x)\ dx = \frac{1}{2 \pi} \cos(\pi x) \mid_1^N = 1 - 1 = 0$ ולכן נקבל סדרה קבועה וכמובן מתכנסת.
	לעומת זאת אנו יודעים כי $\int_{1}^{\infty} \sin (2\pi x)\ dx$ לא מתכנס, וקיבלנו סתירה.
\end{proof}

\Question{}
תהינה סדרות ${(a_n)}_{n = 1}^\infty, {(b_n)}_{n = 1}^\infty$ סדרות אי־שליליות כך ש־$b_n > 0$ לכל $n \in \NN$.

\Subquestion{}
נוכיח כי אם $\lim_{n \to \infty} \frac{a_n}{b_n} = 0$ אז הטור $\sum_n b_n$ מתכנס גורר שהטור $\sum_n a_n$ מתכנס.
\begin{proof}
	מהנתון נסיק כי לכמעט כל $n$ מתקיים $a_n < b_n$ ולכן ממבחן ההשוואה הראשון והאי־שליליות נסיק כי הטענה מתקיימת.
\end{proof}

\Subquestion{}
נסתור את הטענה כי אם $\lim_{n \to \infty} \frac{a_n}{b_n} = 0$ אז הטור $\sum_n a_n$ מתכנס גורר שהטור $\sum_n b_n$ מתכנס על־ידי דוגמה נגדית.

נבחר $a_n = \frac{1}{n^2}, b_n = 1$, ברור כי הטור $\sum_n a_n$ מתכנס וגם כי $\sum_n b_n$ מתבדר, אבל $\frac{a_n}{b_n} = \frac{1}{n^2} \xrightarrow{n \to \infty} 0$ בסתירה לטענה.

\Subquestion{}
הטענה של סעיף א' כאשר $0$ מוחלף ב־$\infty$ איננה נכונה.
הדוגמה הנגדית היא הדוגמה של סעיף ב' כאשר הופכים את הסדרות.

עבור הטענה של סעיף ב' וההחלפה נקבל כי הטענה נכונה, נוכיח באופו זהה להוכחת סעיף א', נקבל כי $b_n < a_n$ מהגבול ולכן $0 < b_n < a_n$ וממשפט ההשוואה נקבל התכנסות.

\Question{}
נבדוק את התכנסות הטורים הבאים

\Subquestion{}
\[
	\sum_n \frac{2^n}{3^n + 1}
\]
ממבחן ההשוואה הגבולי עם ${\left(\frac{2}{3}\right)}^n$ וכמובן הטור הזה מתכנס שכן ${(\frac{2}{3})}^{n + 1} / {(\frac{2}{3})}^n = \frac{2}{3} < 1$ נובעת התכנסות ממבחן דאלמבר.

\Subquestion{}
\[
	\sum_n \frac{3^n}{n!}
\]
נבדוק לפי מבחן המנה ונקבל
\[
	\frac{3^{n + 1}}{(n + 1)!}
	\cdot \frac{n!}{3^n}
	= \frac{3}{n + 1} \xrightarrow{n \to \infty} 0
\]
ולכן נסיק כי הטור מתכנס.

\Subquestion{}
\[
	\sum_n \frac{2n - 3}{n^2 - n + 4}
\]
ממבחן המנה הגבולי יחד עם $\frac{1}{n}$ נקבל כי הטור מתכנס אם ורק אם הטור ההרמוני מתכנס ולכן נסיק כי הטור מתבדר.

\Subquestion{}
\[
	\sum_n \frac{1}{\sqrt[n]{n}}
\]
אנו יודעים מאינפי 1 כי $\lim_{n \to \infty} \sqrt[n]{n} = 1$ ולכן הטור כמובן מתבדר.

\Subquestion{}
\[
	\sum_n \frac{1}{\sqrt{n^2 + 2n}}
\]
נבחין כי
\[
	\frac{\frac{1}{n}}{\frac{1}{\sqrt{n^2 + 2n}}} = \sqrt{\frac{n^2 + 2n}{n^2}} \xrightarrow{n \to \infty} 1
\]
ולכן ממבחן ההשוואה הגבולי יחד עם $\frac{1}{n}$ נסיק כי הטור מתבדר.

\Subquestion{}
\[
	\sum_n {(\sqrt[n]{n} - 1)}^n
\]
נבדוק ונקבל
\[
	\sqrt[n]{{(\sqrt[n]{n} - 1)}^n}
	= {(t - 1)}^t
	\xrightarrow{n \to \infty} 0
\]
ולכן ממבחן קושי להתכנסות נקבל שהטור מתכנס.

\Subquestion{}
\[
	\sum_n \frac{n^2}{2^n}
\]
נשתמש במבחן ההשוואה עבור $\frac{1}{n^2}$, אנו יודעים כי $\frac{n^2}{2^n} < \frac{1}{n^2}$ לכמעט כל $n$ ולכן נקבל מהתכנסות $\sum_n \frac{1}{n^2}$ שגם הטור הנתון מתכנס.

\Subquestion{}
\[
	\sum_n \frac{1}{n^{(1 + \frac{1}{n})}}
\]
נבחין כי
\[
	\frac{1}{n} / \frac{1}{n^{(1 + \frac{1}{n})}}
	= n^{(-1 + 1 + \frac{1}{n})}
	\xrightarrow{n \to \infty} 1
\]
ולכן מהתבדרות הטור ההרמוני ומבחן המנה הגבולי נקבל כי הטור מתבדר.

\Subquestion{}
\[
	\sum_n \frac{2^n n!}{n^n}
\]
נבדוק את מבחן דאלמבר
\[
	\frac{2^{n + 1} (n + 1)!}{{(n + 1)}^{n + 1}} / \frac{2^n n!}{n^n}
	= \frac{2 n^n}{{(n + 1)}^n}
	= 2 {(\frac{n}{n + 1})}^n
	= 2 {(1 - \frac{1}{n + 1})}^n
	\xrightarrow{n \to \infty} \frac{2}{e} < 1
\]
ונקבל מהמבחן כי הטור מתכנס.

\Subquestion{}
\[
	\sum_n \sqrt[n]{e} - 1
\]
נשתמש בתוצאת התרגול הגורסת כי $\lim_{n \to \infty} \frac{u^4}{e^u}$ עבור $u_n = -\frac{1}{n}$ ונקבל
\[
	\lim_{n \to \infty} \frac{\sqrt[n]{e}}{n^4} = 0
\]
ולכן נוכל להסיק כי הטור מתכנס על־ידי מבחן ההוואה הגבולי עם $\frac{1}{n^4}$.

\Subquestion{}
\[
	\sum_n \frac{1}{\sqrt[3]{(n + 3)(n^4 + 2n - 1)}}
\]
נבחין כי
\[
	\frac{1}{\sqrt[3]{(n + 3)(n^4 + 2n - 1)}} < \frac{1}{\sqrt[3]{(n)(n^4)}} = \frac{1}{n^{5/3}}
\]
ומצאנו כי הטור $\sum_n \frac{1}{n^{5/3}}$ מתכנס בתרגול ולכן נסיק ממבחן ההשוואה כי גם הטור הנתון מתכנס.

\Question{}
נגדיר
\[
	a_n = \begin{cases}
		\frac{1}{n} & \exists k \in \NN : n = 2^k \\
		0 & n \notin \{ 2^k \mid k \in \NN \}
	\end{cases},
	\qquad
	b_n = \begin{cases}
		\frac{1}{k} & \exists k \in \NN : n = 2^k \\
		0 & n \notin \{ 2^k \mid k \in \NN \}
	\end{cases}
\]
נבדוק את התכנסות הטורים $\sum_n a_n, \sum_n b_n$.

נבחין כי לכל $k \in \NN$ נקבל כי $\sum_k a_n = \sum_k \frac{1}{2^k}$ מההגדרה, וזהו כמובן טור מתכנס, ולכן נוכל להסיק גם את ההתכנסות של $\sum_n a_n$ עצמו.

לעומת זאת נקבל גם $\sum_k b_n = \sum_k \frac{1}{k}$, דהינו נוכל לבנות תת־סדרה של סכומים של $b_n$ כך שהם זהותית שווים לסכומים החלקיים של הטור ההרמוני, ונוכל להסיק כי הטור מתבדר.

\Question{}
תהינה ${(a_n)}_{n = 1}^\infty, {(b_n)}_{n = 1}^\infty$ סדרות ממשיות.

\Subquestion{}
נוכיח את הטענה כי אם $(a_n)$ אי־שלילית אז $\sum_n a_n$ מתכנס אם ורק אם $\sum_n {(1 + \frac{1}{n})}^n a_n$ מתכנס.
\begin{proof}
	\textbf{כיוון ראשון:}
	ידוע כי הטור מתכנס ולכן נסיק כי גם $\sum e a_n$ מתכנס, שכן מכפלה בסקלר לא משנה התכנסות. \\*
	עוד אנו יודעים כי $\lim_{n \to \infty} {(1 + \frac{1}{n})}^n = e$ ומונוטוני עולה ולכן $0 \le {(1 + \frac{1}{n})}^n a_n \le e a_n$ ונקבל כי הטור מתכנס.

	\textbf{כיוון שני:}
	נניח כי הטור השני מתכנס, ולכן ישירות נוכל להסיק כי $\frac{e}{2} a_n$ חסום על־ידי הסדרה ולכן ממבחן ההשוואה נקבל כי $\sum_n a_n$ מתכנס אף הוא.
\end{proof}

\Subquestion{}
נוכיח כי אם $\sum_n a_n$ מתבדר ו־$\sum_n b_n$ מתכנס אז $\sum_n a_n + b_n$ מתבדר.
\begin{proof}
	נשתמש במבחן ההשוואה הגבולי ונקבל מהעובדה ש־$\lim_{n \to \infty} b_n = 0$ שמתקיים
	\[
		\lim_{n \to \infty} \frac{a_n + b_n}{a_n} = 1
	\]
	ולכן הטורים מתכנסים או מתבדרים יחד, במקרה הזה כמובן מתבדרים.
\end{proof}

\Subquestion{}
נוכיח שאם $(a_n)$ אי־שלילית ואפסה אז קיימת תת־סדרה $(a_{n_k})$ כך ש־$\sum_k a_{n_k}$ מתכנסת.
\begin{proof}
ידוע כי הסדרה אפסה ולכן לכל $l \in \NN$ נוכל למצוא אינדקס עבורו $a_k < \frac{1}{l^2}$, נבנה סדרת אינדקסים כזו עבור $l = 1, 2, \dots$ ונקבל תת־סדרה $(a_{n_k})$ כך ש־$a_{n_k} \le \frac{1}{k^2}$ וזו כמובן מתכנסת.
\end{proof}

\Subquestion{}
נוכיח שאם $(a_n)$ אי־שלילית ומונוטונית יורדת אז $\sum_n a_n$ מתכנס אם ורק אם $\sum_n a_{2n}$ מתכנס.
\begin{proof}
	נניח כי $\sum_n a_n$ מתכנס ונבחין כי $a_{2n} < a_n$ לכל $n$ ולכן ממבחן ההתכנסות הראשון נובע מיידית כי הטור $\sum_n a_{2n}$ מתכנס אף הוא.

	נניח כי $\sum_n a_{2n}$ מתכנס. לכל $n$ נוכל לראות כי $a_n + a_{n + 1} < 2 a_n$ מהמונוטוניות ולכן $a_{2n - 1} + a_{2n} < 2 a_{2n}$ ולמעשה תנאי מבחן ההשוואה מתקיימים שוב ונסיק כי $\sum_n a_n$ מתכנס.
\end{proof}

\Subquestion{}
נוכיח כי אם $(a_n)$ אי־שלילית והטורים $\sum_n a_{2n - 1}, \sum_n a_{2n}$ מתכנסים אז גם הטור $\sum_n a_n$ מתכנס.
\begin{proof}
	נובע ישירות מתוצאת הסעיף הקודם.
\end{proof}

\Question{}
יהי $N \in \NN$, וידוע כי התכנסות $\sum_n a_n$ גוררת את התכנסות $\sum_n a_{n + N}$. \\*
נוכיח שאם $\sum_n a_n$ מתכנס אז מתקיים
\[
	\lim_{N \to \infty} \sum_{n = N + 1}^{\infty} a_n = \lim_{N \to \infty} \lim_{k \to \infty} \sum_{n = N + 1}^{k} a_n = 0
\]
\begin{proof}
	נבחין כי
	\[
		\sum_{n = N + 1}^\infty a_n = \left(\sum_{n = 1}^{\infty} a_n\right) - \left( \sum_{n = 1}^{N} a_n \right)
	\]
	טענה זו נכונה על־פי תהליך ההוכחה של הטענה שהוצגה בתחילת ההוכחה. \\*
	עתה נראה כי נתון $\sum_n a_n = L$ ערך סופי, ולכן נקבל על־פי הגדרה כי
	\[
		\lim_{N \to \infty} \sum_{n = 1}^{N} a_n = L
	\]
	ולכן נקבל
	\[
		\lim_{N \to \infty} \sum_{n = N + 1}^\infty a_n
		= \lim_{N \to \infty} \left( \left(\sum_{n = 1}^{\infty} a_n\right) - \left( \sum_{n = 1}^{N} a_n \right) \right)
		= L - L
		= 0
	\]
\end{proof}

\end{document} % chktex 17
