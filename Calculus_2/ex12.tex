\documentclass[a4paper]{article}

% packages
\usepackage{inputenc, fontspec, amsmath, amsthm, amsfonts, polyglossia, catchfile}
\usepackage[a4paper, margin=50pt, includeheadfoot]{geometry} % set page margins

% style
\AddToHook{cmd/section/before}{\clearpage}	% Add line break before section
\linespread{1.5}
\setcounter{secnumdepth}{0}		% Remove default number tags from sections
\setmainfont{Libertinus Serif}
\setsansfont{Libertinus Sans}
\setmonofont{Libertinus Mono}
\setdefaultlanguage{hebrew}
\setotherlanguage{english}

% operators
\DeclareMathOperator\cis{cis}
\DeclareMathOperator\Sp{Sp}
\DeclareMathOperator\tr{tr}
\DeclareMathOperator\im{Im}
\DeclareMathOperator\diag{diag}
\DeclareMathOperator*\lowlim{\underline{lim}}
\DeclareMathOperator*\uplim{\overline{lim}}

% commands
\renewcommand\qedsymbol{\textbf{משל}}
\newcommand{\NN}[0]{\mathbb{N}}
\newcommand{\ZZ}[0]{\mathbb{Z}}
\newcommand{\QQ}[0]{\mathbb{Q}}
\newcommand{\RR}[0]{\mathbb{R}}
\newcommand{\CC}[0]{\mathbb{C}}
\newcommand{\getenv}[2][] {
  \CatchFileEdef{\temp}{"|kpsewhich --var-value #2"}{\endlinechar=-1}
  \if\relax\detokenize{#1}\relax\temp\else\let#1\temp\fi
}
\newcommand{\explain}[2] {
	\begin{flalign*}
		 && \text{#2} && \text{#1}
	\end{flalign*}
}

% headers
\getenv[\AUTHOR]{AUTHOR}
\author{\AUTHOR}
\date\today

\usepackage{tikz}
\DeclareMathOperator\arcsinh{arcsinh}
\title{פתרון מטלה 12 – חשבון אינפיניטסימלי 2 (80132)}
% chktex-file 9

\begin{document}
\maketitle
\maketitleprint{}

\Question{}
נוכיח או נפריך את הטענות הבאות.

\Subquestion{}
נוכיח ש קיימות ${(a_n)}_{n = 1}^\infty$ ו־${(b_n)}_{n = 1}^\infty$ כך ש־$b_n \ne 0$ לכל $n \in \NN$ וכך ש־$\lim_{n \to \infty} \frac{a_n}{b_n} = 0$ ו־$\sum_n b_n$ מתכנס אך $\sum_n a_n$ מתבדר.
\begin{proof}
	למעשה כבר הוכחנו את המשפט הזה עבור סדרות אי־שליליות, ולכן נסיק כי אם התנאי מתקיים אז הטורים מתכנסים ומתבדרים ביחד בהחלט, ולכן בפרט גם מתכנסים ומתבדרים ביחד.
\end{proof}

\Subquestion{}
נוכיח כי אם $\sum_n a_n$ מתכנס בתנאי, אז קיימת סדרה ${(\epsilon_n)}_{n = 1}^\infty$ כך ש־$\epsilon_n = \pm 1$ לכל $n \in \NN$ כך ש־$\sum_{n = 1}^{\infty} a_n \epsilon_n = -\infty$.
\begin{proof}
	נתונה התכנסות בתנאי ולכן $\sum_n |a_n|$ מתבדרת, וזוהי כמובן סדרה מונוטונית חיובית, ולכן $\sum_{n = 1}^{\infty} |a_n| = \infty$. \\*
	לכן כמובן $\sum_{n = 1}^{\infty} -|a_n| = -\infty$, ונגדיר $\epsilon_n = \frac{-|a_n|}{a_n}$ ולכן $a_n \epsilon_n = -|a_n|$ ונקבל כי הטענה מתקיימת.
\end{proof}

\Question{}
נמצא בכל סעיף את רדיוס ההתכנסות ותחום ההתכנסות של הטורים הנתונים.

\Subquestion{}
\[
	\sum_{n = 1}^{\infty} \frac{4n^3 + 2n - 1}{7n^5 + 4n^2 - 3} x^n
\]
נבחין כי $a_n / \frac{1}{n^2} \xrightarrow{n \to \infty} 1$ ולכן גם $\sqrt[n]{|a_n|} / \frac{1}{\sqrt[n]{n^2}} \xrightarrow{n \to \infty} 1$ ונוכל להסיק כי $\sqrt[n]{|a_n|} \xrightarrow{n \to \infty} 1$, לכן $R = 1$.
נסיק מנקודות הקצה וממבחן ההשוואה הגבולי שראינו הרגע כי $R = [-1, 1]$.

\Subquestion{}
\[
	\sum_{n = 0}^{\infty} a^{n^2} x^n
\]
עבור $a \in \RR$ קבוע.
נראה כי
\[
	\sqrt[n]{|a^{n^2}|}
	= {|a|}^{n^2 / n}
	= {|a|}^n
\]
ולכן נסיק כי אם $-1 < a < 1$ אז $R = \RR$, אם $a = \pm 1$ אז $R = (-1, 1)$ ואם $|a| > 1$ אז $R = 0$.

\Subquestion{}
\[
	\sum_{n = 0}^{\infty} \frac{3^n + 4^n}{5^n + 6^n} {(x + 2)}^n
\]
נראה כי
\[
	a_n / {(\frac{4}{6})}^n \xrightarrow{n \to \infty} 1
	\implies \sqrt[n]{|a_n|} / \frac{2}{3} \xrightarrow{n \to \infty} 1
	\implies R = \frac{3}{2}
\]
ולכן תחום ההתכנסות הוא $(-\frac{7}{2}, -\frac{1}{2})$ ונשאר לבדוק את נקודות הקצה. עבור $x = -\frac{1}{2}$ נקבל את הטור $\sum_n 1$ והוא כמובן מתבדר, ועבור $x = -\frac{7}{2}$ נקבל $\sum_n (-1)$ וגם הוא מתבדר.

\Subquestion{}
\[
	\sum_{n = 1}^{\infty} 5^{n \ln n} x^n
\]
נראה כי
\[
	\lim_{n \to \infty} \sqrt[n]{|5^{n \ln n}|}
	\lim_{n \to \infty} 5^{\ln n}
	= \infty
\]
ולכן $R = 0$ ובהתאם הטור מתכנס ב־$x = 0$ בלבד.

\Subquestion{}
\[
	\sum_{n = 1}^{\infty} \frac{{(-1)}^n}{n} x^{2n + 1}
\]
ולכן הסדרה מורכבת מאפסים במקומות הזוגיים וסדרת לייבניץ במקומות האי־זוגיים, ולכן
\[
	\uplim_{n \to \infty} \sqrt[n]{|a_n|}
	= \lim_{n \to \infty} \sqrt[2n + 1]{\frac{1}{n}}
	= 1
\]
ולכן $R = 1$ ויש לבדוק נקודות קצה, בשתיהן נקבל מהאי־זוגיות של $x$ את טור לייבניץ והתכנסות, לכן תחום ההתכנסות הוא $[-1, 1]$.

\Subquestion{}
\[
	\sum_{n = 0}^{\infty} a^n x^{n^2}
\]
עבור $a \in \RR$ קבוע.
כמו בסעיף הקודם נקבל
\[
	\uplim_{n \to \infty} \sqrt[n]{|a_n|}
	= \lim_{n \to \infty} {|a|}^{n / n^2}
	= 1
\]
אם $a = 0$ אז נקבל $R = \infty$, לכן נניח $a \ne 0$ ולכן $R = 1$ ובהתאם בנקודות הקצה נקבל התכנסות אם ורק אם $a < 1$.

\Subquestion{}
\[
	\sum_{n = 0}^{\infty} \frac{3^{n^2} + 4^n}{5^n + 6^n} x^n
\]
נראה כי
\[
	a_n / 3^{n^2} \xrightarrow{n \to \infty} 1
	\implies \sqrt[n]{|a_n|} / 3^n \xrightarrow{n \to \infty} 1
	\implies \sqrt[n]{|a_n|} \xrightarrow{n \to \infty} \infty
\]
ולכן רדיוס ההתכנסות הוא $0$.

\Question{}
נוכיח או נפריך את הטענות הבאות.

\Subquestion{}
נוכיח כי קיים טור חזקות סביב $0$ המתכנס בתנאי ב־$\pm 1$.
\begin{proof}
	נבחין כי אם $a_n = \frac{1}{n}$ אז נקבל טור חזקות העומד בתנאים אך שמתבדר ב־$1$, לכן נבחר את הסידור מחדש של טור לייבניץ
	\[
		b_n = a_{a + {(-1)}^n}
	\]
	ונקבל שהוא מתכנס וכך גם הכפלתו ב־${(-1)}^n$, בתנאי בשני המקרים.
\end{proof}

\Subquestion{}
נסתור את הטענה כי קיים טור חזקות סביב $0$ המתכנס בהחלט ב־$1$ ומתבדר ב־$-1$.

נניח בשלילה כי קיים טור חזקות כזה $\sum_n a_n$, לכן מהנתון נקבל כי $\sum_n |a_n|$ מתכנס וגם $\sum_n {(-1)}^n a_n$ מתבדר, ולכן ממשפט התכנסות בהחלט נקבל $\sum_n |{(-1)}^n|$ מתבדר בסתירה להתכנסות שנתונה.

\Question{}
בתרגיל זה נעבוד בחוג $\RR_\infty = \RR \cup \{ \infty \}$ כאשר מוגדר $0 \cdot \infty = 1$, ותהי סדרה ${(a_n)}_{n = 1}^\infty$ כך ש־$\forall n \in \NN, a_n \ne 0$.

\Subquestion{}
נוכיח שאם $\lim_{n \to \infty} \frac{|a_{n + 1}|}{|a_n|} = L$ קיים ב־$\RR_\infty$, אז רדיוס ההתכנסות של $\sum_{n = 0}^{\infty} a_n {(x - x_0)}^n$ הוא $R = \frac{1}{L}$.
\begin{proof}
	נחשב
	\[
		\left\lvert \frac{a_{n + 1} {(x - x_0)}^{n + 1}}{a_n {(x - x_0)}^n} \right\rvert
		= \frac{|a_{n + 1}|}{|a_n|} \cdot |x - x_0|
		\xrightarrow{n \to \infty} L |x - x_0|
	\]
	ממבחן דאלמבר נסיק כי אם $L|x - x_0| < 1$ אז הטור מתכנס, דהינו אם $|x - x_0| = R = \frac{1}{L}$.
\end{proof}

\Subquestion{}
נוכיח כי ניתן להחליש את הטענה מהסעיף הקודם להיות התכנסות הטור כאשר $\uplim_{n \to \infty} \frac{|a_{n + 1}|}{|a_n|}$.
\begin{proof}
	נניח כי התנאי מתקיים, וידוע כי $0 < \left\lvert \frac{a_{n + 1}}{a_n} \right\rvert < M$ ולכן נסיק כי כלל  הגבולות החלקיים $L_i$ מקיימים $0 \le L_i \le L$ כאשר $L$ הגבול העליון.
	לכל גבול חלקי נקבל $R_i = \frac{1}{L_i}$ ולכן $R_i \ge R$ ונסיק כי ברדיוס $R$ כל תת־הסדרות מתכנסות ובהתאם הטור אף הוא מתכנס.
\end{proof}

\Question{}
תהי $f : \RR \to \RR$ המוגדרת על־ידי $f(x) = \sin x$.

\Subquestion{}
נראה כי $f$ חלקה.
\begin{proof}
	אנו יודעים כי $f^{(4)} = f$ ולכן נוכל להסיק כי היא חלקה.
\end{proof}

\Subquestion{}
נסמן כרגיל $T_{f, 0}(x) = \sum_{n = 0}^{\infty} a_n x^n$ טור טיילור של $f$ סביב $0$. \\*
נמצא נוסחה מפורשת ל־$a_n$ עבור כל $n \in \NN \cup \{ 0 \}$ ונחשב את רדיוס ההתכנסות $R$ של $T_{f, 0}$.

למעשה כבר מצאנו נוסחה מפורשת כזו והיא
\[
	T_{f, 0}(x) = \sum_{n = 0}^{\infty} \frac{{(-1)}^n}{(2n + 1)!} x^{2n + 1}
\]
ואם נשתמש במבחן ההתכנסות לטורי חזקות נקבל
\[
	\lim_{n \to \infty} \sqrt[n]{\frac{1}{(2n + 1)!}} = 0
\]
ולכן $T_{f, 0} : \RR \to \RR$.

\Subquestion{}
נוכיח כי $T_{f, 0} = f$.
\begin{proof}
	למעשה נוכל להשתמש בשארית טיילור, ונקבל
	\[
		\lim_{n \to \infty} R_{n, f, 0}
		= \lim_{n \to \infty} \frac{f^{(n + 1)}(c)}{(n + 1)!} {(x - x_0)}^{n + 1} = 0
	\]
	ולכן נוכל להסיק כי $f = \lim_{n \to \infty} P_{n, f, 0} = T_{f, 0}$.
\end{proof}

\Subquestion{}
נסדור עבור אילו $x \in \RR$ הטור $T_{f, 0}$ הוא טור לייבניץ.

למעשה, הסדרה היא סדרת לייבניץ כבר, ומקיום הערך במקומות האי־זוגיים בלבד נקבל כי גם עבור ערכי $x$ שליליים הטור הוא טור לייבניץ.

\Subquestion{}
נמצא קירוב רציונלי ל־$f(\frac{1}{2})$ המדויק עד כדי $10^{-6}$.

למעשה, נוכל להשתמש בשארית טיילור כפי שעשינו במטלות קודמות, נראה כי
\[
	R_{n, f, 0} \le 10^{-6}
	\implies \frac{f^{(n + 1)}(c)}{(n + 1)!} {(\frac{1}{2} - 0)}^{n + 1}
	\le \frac{1}{2^{n + 1} (n + 1)!}
	\le 2^{-n - 1}
	\le 2^{-20}
	\le 10^{-6}
\]
ולכן נגדיר $n = 19$ ובהתאם
\[
	\sin \frac{1}{2} \approx \sum_{n = 0}^{19} \frac{{(-1)}^n}{(2n + 1)!} {(\frac{1}{2})}^{2n + 1}
\]

\Question{}
יהיו $\sum_{n = 0}^{\infty} a_n x^n$ ו־$\sum_{n = 0}^{\infty} b_n x^n$ טורי חזקות עם רדיוסי התכנסות $R_1, R_2$ סופיים. \\*
יהי $R_3$ רדיוס ההתכנסות של $\sum_{n = 0}^{\infty} (a_n + b_n) x^n$.

\Subquestion{}
נוכיח כי $R_3 \ge \min(R_1, R_2)$.
\begin{proof}
	יהי $x \in (-R_1, R_1) \cap (-R_2, R_2)$, אז כמובן $\sum_{n = 0}^{\infty} a_n x^n, \sum_{n = 0}^{\infty} b_n x^n$ שניהם מתכנסים ולכן גם סכומם מתכנס ולכן נוכל להסיק כי $R_3 > |x|$, אבל $|x| < R_1, R_2$.
\end{proof}

\Subquestion{}
נוכיח שאם $R_1 \ne R_2$ אז $R_3 = \min(R_1, R_2)$.
\begin{proof}
	בלי הגבלת הכלליות נקבע $R_1 < R_2$ ולכן $\forall x \in (-R_1, R_1) \implies x \in (-R_2, R_2)$ ונסיק כי גם סכום הטורים מתכנס וכן נוכל להסיק כי $R_3 \ge R_1$. \\*
	נניח עתה כי $R_1 \le |x| < R_2$ ונקבל כי הטור הראשון מתבדר והשני מתכנס ולכן מהמטלות הקודמות נוכל להסיק כי סכום הטורים מתבדר אף הוא ונקבל $R_3 \le R_1$ וקיבלנו כי $R_1 = R_3$.
\end{proof}

\Subquestion{}
נמצא טורים $\sum_{n = 0}^{\infty} a_n x^n$ ו־$\sum_{n = 0}^{\infty} b_n x^n$ שעבורם מתקיים אי־שוויון חזק $R_3 > \min(R_1, R_2)$.

למעשה עבור כל $a_n = -b_n$ הטענה תתקיים, לדוגמה $a_n = n, b_n = -n$ מתקיים $R_1 = R_2 = 1$ אבל $R_3 = \infty$.

\end{document} % chktex 17
