\documentclass[a4paper]{article}

% packages
\usepackage{inputenc, fontspec, amsmath, amsthm, amsfonts, polyglossia, catchfile}
\usepackage[a4paper, margin=50pt, includeheadfoot]{geometry} % set page margins

% style
\AddToHook{cmd/section/before}{\clearpage}	% Add line break before section
\linespread{1.5}
\setcounter{secnumdepth}{0}		% Remove default number tags from sections
\setmainfont{Libertinus Serif}
\setsansfont{Libertinus Sans}
\setmonofont{Libertinus Mono}
\setdefaultlanguage{hebrew}
\setotherlanguage{english}

% operators
\DeclareMathOperator\cis{cis}
\DeclareMathOperator\Sp{Sp}
\DeclareMathOperator\tr{tr}
\DeclareMathOperator\im{Im}
\DeclareMathOperator\diag{diag}
\DeclareMathOperator*\lowlim{\underline{lim}}
\DeclareMathOperator*\uplim{\overline{lim}}

% commands
\renewcommand\qedsymbol{\textbf{משל}}
\newcommand{\NN}[0]{\mathbb{N}}
\newcommand{\ZZ}[0]{\mathbb{Z}}
\newcommand{\QQ}[0]{\mathbb{Q}}
\newcommand{\RR}[0]{\mathbb{R}}
\newcommand{\CC}[0]{\mathbb{C}}
\newcommand{\getenv}[2][] {
  \CatchFileEdef{\temp}{"|kpsewhich --var-value #2"}{\endlinechar=-1}
  \if\relax\detokenize{#1}\relax\temp\else\let#1\temp\fi
}
\newcommand{\explain}[2] {
	\begin{flalign*}
		 && \text{#2} && \text{#1}
	\end{flalign*}
}

% headers
\getenv[\AUTHOR]{AUTHOR}
\author{\AUTHOR}
\date\today

\usepackage{tikz}
\DeclareMathOperator\arcsinh{arcsinh}
\title{פתרון מטלה 3 – חשבון אינפיניטסימלי 2 (80132)}

\begin{document}
\maketitle
\maketitleprint{}

\Question{}
נוכיח כי $\forall x \in (-1, 0) \cup (0, \infty) : \frac{x}{1 + x} < \ln(1 + x) < x$.
\begin{proof}
	תהי $g(x) = \ln(1 + x)$ פונקציה המוגדרת לכל $x > -1$ וגזירה בתחום כאשר $g'(x) = \frac{1}{1 + x}$ על־פי נוסחות גזירה. \\*
	נבחר $x_0 = 0$ ו־$x > 0$ כך שגם $x \ne 0$ וממשפט הערך הממוצע נובע
	\[
		\exists c \in (0, x) : g'(c) = \frac{g(x) - g(x_0)}{x - x_0} = \frac{\ln(1 + x)}{x}
	\]
	ידוע כי $g'$ פונקציה מונוטונית יורדת עבור $x > -1$ ולכן $g'(x_0) > g'(c) > g'(x)$ ובהתאם:
	\[
		\frac{1}{1 + x} < \frac{\ln(1 + x)}{x} < 1 \implies \frac{x}{1 + x} < \ln(1 + x) < x
	\]
	עבור $-1 < x < 0$ עדיין מתקיים
	\[
		\exists c \in (x, 0): g'(c) = \frac{g(0) - g(x)}{0 - x} = \frac{\ln(1 + x)}{x}
	\]
	ומתקיים $g'(x) > g'(c) > g(0)$ ולכן משליליות $x$ נסיק
	\[
		\frac{1}{1 + x} > \frac{\ln(1 + x)}{x} > 1 \implies \frac{x}{1 + x} < \ln(1 + x) < x
	\]
	ומצאנו כי הטענה נכונה לכל $x \in (-1, 0) \cup (0, \infty)$.
\end{proof}

\Question{}
יהיו $a, b, c \in \RR$ כך ש־$a < b < c$. תהי $f$ פונקציה רציפה ב־$[a, b]$ וגזירה ב־$(a, b) \setminus \{c\}$.

\Subquestion{}
נוכיח כי אם $f'(x) \ge 0$ לכל $x \in (a, b) \setminus \{c\}$ אז $f$ מונוטונית עולה ב־$(a, b)$.
\begin{proof}
	נניח בשלילה כי $f$ איננה מונוטונית עולה בתחום, ולכן קיימים $x < y$ בתחום הנתון המקיימים $f(x) > f(y)$. \\*
	ידוע כי הנגזרת חיובית ולכן נובע שבכל נקודה בה הנגזרת מוגדרת הפונקציה עולה, ולכן $x < c < y$. \\*
	עתה נבחין כי לכל $x < c$ הנגזרת כן מוגדרת ואי־שלילית ולכן נסיק $f(x) \le f(c)$, ובאופן דומה נקבל כי $f(c) \le f(y)$ ומכאן נסיק $f(x) \le f(y)$ בסתירה להנחה. \\*
	לכן $f$ פונקציה מונוטונית עולה בתחום.
\end{proof}

\Subquestion{}
נוכיח כי אם $f'(x) > 0$ עבור כל $x \in (a, b) \setminus \{c\}$ אז $f$ מונוטונית עולה ממש ב־$(a, b)$.
\begin{proof}
	נניח בשלילה כי $f$ איננה מונוטונית עולה בתחום, ולכן קיימים $x < y$ בתחום הנתון המקיימים $f(x) \ge f(y)$. \\*
	ידוע כי הנגזרת חיובית ולכן נובע שבכל נקודה בה הנגזרת מוגדרת הפונקציה עולה, ולכן $x < c < y$. \\*
	עתה נבחין כי לכל $x < c$ הנגזרת כן מוגדרת וחיובית ולכן נסיק $f(x) < f(c)$, ובאופן דומה נקבל כי $f(c) < f(y)$ ומכאן נסיק $f(x) < f(y)$ בסתירה להנחה. \\*
	לכן $f$ פונקציה מונוטונית עולה ממש בתחום.
\end{proof}

\Subquestion{}
נוכיח שאם $\forall x \in (a, c) : f'(x) < 0$ וגם $\forall x \in (c, b) : f'(x) > 0$ אז $c$ מינימום מקומי של $f$.
\begin{proof}
	על־פי הנתונים $f$ מונוטונית עולה ממש ב־$(c, b)$ ויורדת ממש ב־$(a, c)$, דהינו לכל $x \in (a, b)$ מתקיים $f(c) < f(x)$ ולכן זהו מינימום מקומי.
\end{proof}

\Question{}
\Subquestion{}
נתונים $p, q \in \RR$ כאשר $0 < p < 1$. נוכיח כי למשוואה $x - p \cdot \sin(x) = q$ יש פתרון ממשי יחיד.
\begin{proof}
	נגדיר $f(x) = x - p \sin(x) - q$ ולכן פתרונות המשוואה שקולים לשורשי הפונקציה. \\*
	נשים לב כי היא גזירה ב־$\RR$ ואף $f'(x) = 1 - p \cos x$. תמונת $\cos x$ היא $[-1, 1]$ ובהתאם תמונת הפונקציה היא $[1 - p, 1 + p]$ ועל־פי הגדרת $p$ נובע ש־$f'(x) > 0$ לכל $x \in \RR$.
	עוד נבחין כי $f(x) \xrightarrow{x \to -\infty} -\infty$ וגם $f(x) \xrightarrow{x \to \infty} = \infty$ ולכן נוכל להסיק כי היא יש לה לכל הפחות שורש יחיד $c \in \RR$. \\*
	ניעזר במונוטויות ונקבל כי לכל $x \in \RR, x \ne c$ מתקיים $f(x) \ne f(c) = 0$ וקיבלנו כי קיים שורש יחיד.
\end{proof}

\Subquestion{}
נוכיח כי גם אם $p = 1$ עדיין יש ל־$f$ שורש יחיד ב־$\RR$.
\begin{proof}
	במקרה זה תמונת $f'$ היא $[0, 1]$ ולכן $f$ פונקציה מונוטונית עולה ולא עולה ממש. \\*
	ההצדקה לקיום שורש $c \in \RR$ עודנה נכונה במקרה זה, ונבדוק אם שורש זה הוא יחיד. \\*
	אילו $f'(c) > 0$ אז בסביבה של $c$ הפונקציה מונוטונית עולה ממש ונקבל כי זהו שורש יחיד בדומה לסעיף הקודם. \\*
	נניח אם כך ש־$f'(c) = 0$. מערך פונקציה הנגזרת שמצאנו ומתכונת פונקציית $\cos$ נסיק כי $f'(x) > 0$ בסביבה מנוקבת סביב $x = c$.
	ולכן $f$ עולה ממש בסביבה מנוקבת של $c$ ונקבל כי $f(c) \ne f(x)$ בתחום, וקיבלנו כי השורש הוא יחיד.
\end{proof}

\Question{}
תהי $f : (1, \infty) \to \RR$ פונקציה גזירה. \\*
נוכיח כי אם $\lim_{x \to \infty} f(x) = L \in \RR$ אז $\lim_{x \to \infty} f'(x) = 0$.
\begin{proof}
	יהי $\epsilon > 0$ ו־$M > 0$ המקיים שלכל $x > M$ מתקיים $|f(x) - L| < \epsilon$, זאת נסיק מהגבול הנתון. \\*
	נבחר עוד נקודה $y > 0$ ונקבל כי $|f(y) - L| < \epsilon$ ולכן נקבל $|f(x) - f(y)| \le |f(x) - L| + |f(y) - L| = 2\epsilon$. \\*
	בהתאם $\frac{|f(x) - f(y)|}{|x - y|} = 2\epsilon |x - y| \le 2 \epsilon$ כאשר $|x - y| < 1$ הגדרת הנגזרת מתקיימת ונובע $|f'(x)| \le |x - y|$ ומתקיים
	\[
		\lim_{x \to \infty} f'(x) = 0
	\]
\end{proof}

\Question{}
נגדיר $f : [0.\pi] \to [-1, 1]$ על־ידי $f(x) = \cos x$.

\Subquestion{}
נוכיח ש־$f$ חד־חד ערכית ועל.
\begin{proof}
	\[
		\forall x, y \in [0, \pi]: f(x) = f(y) \iff \cos x = \cos y \iff x = y + 2\pi k \lor x = -y + 2 \pi k \forall k \in \ZZ \implies x = y
	\]
	ולכן $f$ חד־חד ערכית. \\*
	נשים לב ש־$f(x)$ מונוטונית יורדת בכל תחומה, שכן $f'(x) = \cos'(x) = -\sin x$ וידוע כי $0 \le \sin x$ לכל $x \in [0, \pi]$. \\*
	עוד נראה ש־$f(0) = 1, f(\pi) = 0$ ולכן נוכל להסיק $f([0, \pi]) = [0, 1]$ ולכן היא על.
\end{proof}

\Subquestion{}
נמצא את תחום הגזירות של $f^{-1}$ וביטוי מפורש לערך נגזרתה בתחום.
\begin{proof}[פתרון]
	נשים לב תחילה ש־$f^{-1} : [0, 1] \to [0, \pi]$. \\*
	עוד נבחין שהפונקציה $f$ לא מקבלת אפס אלא בנקודת הקצה $x = \pi$ ולכן נסיק שנגזרת $f^{-1}$ מוגדרת עבור $(0, 1)$. \\*
	נשתמש בנוסחת נגזרת הופכית עבור $y = \cos x$ ונקבל
	\[
		(f^{-1})'(y) = \frac{1}{f'(x)} = \frac{1}{-\sin x} = -\frac{1}{\sin(\cos^{-1}(x))}
	\]
	נשתמש בזהות $\cos^2 \theta + \sin^2 \theta = 1$ יחד עם $\theta = \cos^{-1} x$ ונקבל $x^2 + \sin^2(\cos^{-1}(x)) = 1 \implies \sin(\cos^{-1}(x)) = \sqrt{1 - x^2}$ ונקבל
	\[
		(f^{-1})'(x) = \frac{-1}{\sqrt{1 - x^2}}
	\]
	ופונקציה זו אכן מוגדרת ב־$(0, 1)$.
\end{proof}

\Subquestion{}
נסמן $\arccos = f^{-1}$ ונסרטט את את גרפה על־ידי סרטוט $\cos x$ המוכר לנו בשיקוף על $y = x$ כנלמד בכיתה.
\begin{center}
	\begin{tikzpicture}[scale=3]
			\draw[->] (0, 0) -- (1, 0) node[left] {$x$};
			\draw[->] (0, 0) -- (0, pi/2) node[above] {$y$};
			\draw[domain=0:1, smooth, variable=\x, blue] plot ({\x}, {rad(acos(\x))});
	\end{tikzpicture}
\end{center}

\Question{}
\Subquestion{}
נוכיח ש־$\arcsinh$ הפיכה.
\begin{proof}
	ניזכר כי
	\[
		\sinh(x) = \frac{e^x - e^{-x}}{2} = y
		\implies 2y = e^x - e^{-x}
		= e^{\ln t} - e^{- \ln t}
		= t - \frac{1}{t}
	\]
\end{proof}

\end{document}
