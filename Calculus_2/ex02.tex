\documentclass[a4paper]{article}

% packages
\usepackage{inputenc, amsmath, amsthm, thmtools, amsfonts, amssymb, luacode, catchfile, tikzducks, hyperref}
\usepackage[a4paper, margin=50pt, includeheadfoot]{geometry} % set page margins
\usepackage[shortlabels]{enumitem}
\usepackage[skip=3pt, indent=0pt]{parskip}

% language
\usepackage[bidi=basic, layout=tabular, provide=*]{babel}
\babelprovide[main, import]{hebrew}
\babelprovide{rl}
\babelfont{rm}{Libertinus Serif}
\babelfont{sf}{Libertinus Sans}
\babelfont{tt}{Libertinus Mono}

% style
\AddToHook{cmd/section/before}{\clearpage}	% Add line break before section
\linespread{1.3}
\setcounter{secnumdepth}{0}		% Remove default number tags from sections, this won't do well with theorems
\AtBeginDocument{\setlength{\belowdisplayskip}{3pt}}
\AtBeginDocument{\setlength{\abovedisplayskip}{3pt}}

% operators
\DeclareMathOperator\cis{cis}
\DeclareMathOperator\Sp{Sp}
\DeclareMathOperator\tr{tr}
\DeclareMathOperator\im{Im}
\DeclareMathOperator\re{Re}
\DeclareMathOperator\diag{diag}
\DeclareMathOperator*\lowlim{\underline{lim}}
\DeclareMathOperator*\uplim{\overline{lim}}
\DeclareMathOperator\rng{rng}
\DeclareMathOperator\Sym{Sym}
\DeclareMathOperator\Arg{Arg}
\DeclareMathOperator\Log{Log}
\DeclareMathOperator\dom{dom}

% commands
%\renewcommand\qedsymbol{\textbf{מש''ל}}
%\renewcommand\qedsymbol{\fbox{\emoji{lizard}}}
\newcommand{\NN}[0]{\mathbb{N}}
\newcommand{\ZZ}[0]{\mathbb{Z}}
\newcommand{\QQ}[0]{\mathbb{Q}}
\newcommand{\RR}[0]{\mathbb{R}}
\newcommand{\CC}[0]{\mathbb{C}}
\newcommand{\FF}[0]{\mathbb{F}}
\newcommand{\PP}[0]{\mathbb{P}}
\newcommand{\TT}[0]{\mathbb{T}}
\newcommand{\acts}[0]{\circlearrowright}
\newcommand{\explain}[2] {
	\begin{flalign*}
		 && \text{#2} && \text{#1}
	\end{flalign*}
}
\newcommand{\maketitleprint}[0]{ \begin{center}
	\begin{tikzpicture}[scale=3]
		\duck[graduate=gray!20!black, tassel=red!70!black]
	\end{tikzpicture}	
\end{center}
}

% theorem commands
\newtheoremstyle{c_remark}
	{}	% Space above
	{}	% Space below
	{}% Body font
	{}	% Indent amount
	{\bfseries}	% Theorem head font
	{}	% Punctuation after theorem head
	{.5em}	% Space after theorem head
	{\thmname{#1}\thmnumber{ #2}\thmnote{ \normalfont{\text{(#3)}}}}	% head content
\newtheoremstyle{c_definition}
	{3pt}	% Space above
	{3pt}	% Space below
	{}% Body font
	{}	% Indent amount
	{\bfseries}	% Theorem head font
	{}	% Punctuation after theorem head
	{.5em}	% Space after theorem head
	{\thmname{#1}\thmnumber{ #2}\thmnote{ \normalfont{\text{(#3)}}}}	% head content
\newtheoremstyle{c_plain}
	{3pt}	% Space above
	{3pt}	% Space below
	{\itshape}% Body font
	{}	% Indent amount
	{\bfseries}	% Theorem head font
	{}	% Punctuation after theorem head
	{.5em}	% Space after theorem head
	{\thmname{#1}\thmnumber{ #2}\thmnote{ \text{(#3)}}}	% head content

\theoremstyle{c_plain}
\newtheorem{theorem}{משפט}[section]
\newtheorem{lemma}[theorem]{למה}
\newtheorem{proposition}[theorem]{טענה}
\newtheorem*{proposition*}{טענה}
%\newtheorem{corollary}[theorem]{אין חלופה עברית}

\theoremstyle{c_definition}
\newtheorem{definition}[theorem]{הגדרה}
\newtheorem*{definition*}{הגדרה}
\newtheorem{example}{דוגמה}[section]
\newtheorem{exercise}{תרגיל}[section]

\theoremstyle{c_remark}
\newtheorem*{remark}{הערה}
\newtheorem*{solution}{פתרון}
\newtheorem{conclusion}[theorem]{מסקנה}
\newtheorem{notation}[theorem]{סימון}

% Questions related commands
\newcounter{question}
\setcounter{question}{1}
\newcounter{sub_question}
\setcounter{sub_question}{1}

\newcommand{\question}[1][0]{
	\ifthenelse{#1 = 0}{}{\setcounter{question}{#1}}
	\subsection{שאלה \arabic{question}}
	\addtocounter{question}{1}
	\setcounter{sub_question}{1}
}

\newcommand{\subquestion}[1][0]{
	\ifthenelse{#1 = 0}{}{\setcounter{sub_question}{#1}}
	\subsubsection{סעיף \localecounter{letters.gershayim}{sub_question}}
	\addtocounter{sub_question}{1}
}

% import lua and start of document
\directlua{common = require ('../common')}

\GetEnv{AUTHOR}

% headers
\author{\AUTHOR}
\date\today

\title{פתרון מטלה 2 – חשבון אינפיניטסימלי 2 (80132--2)}

\begin{document}
\maketitle
\maketitleprint{}

\Question{}
\Subquestion{}
\subsubsection{i.}
נמצא את תחום הגזירות של הפונקציה
\[
	f(x) = \sqrt{1 - x^2}
\]
נחשב את ערך הנגזרת על־פי חוקי גזירה:
\[
	f'(x) = \frac{(1 - x^2)'}{2 \sqrt{1 - x^2}} = \frac{-x}{\sqrt{1 - x^2}}
\]
ונראה כי הביטוי מוגדר כאשר $\sqrt{1 - x^2} > 0$, ולכן נובע ישירות $1 - x^2 > 0$ ולכן $x \in (-1, 1)$.

\subsubsection{ii.}
יהי $x_0 \in (-1, 1)$ ומשיק $\Gamma_f$ לגרף של $f$ בנקודה $(x_0, f(x_0))$ ונראה כי הוא חותך את גרף הפונקציה $f$ בנקודה אחת ויחידה. \\*
נשים לב כי המשיק עובר בנקודה הנתונה ושיפועו נתון על־ידי $f'(x_0)$ ולכן נובע
\[
	\Gamma_f : f'(x_0)(x - x_0) + f(x_0)
	= \frac{-x_0 (x - x_0)}{\sqrt{1 - x_0^2}} + \sqrt{1 - x_0^2}
	= \frac{1 - x_0 x}{\sqrt{1 - x_0^2}}
\]
ונבדוק מתי מתקיים $\Gamma_f = f$:
\begin{align*}
	\Gamma_f = f
	& = \frac{1 - x_0 x}{\sqrt{1 - x_0^2}} = \sqrt{1 - x^2} \\
	& \implies \frac{1 - 2x_0 x + x_0^2x^2}{1 - x_0^2} = 1 - x^2 \\
	& \implies 1 - 2x_0 x + x_0^2x^2 = 1 - x_0^2 - x^2 + x^2 x_0^2 \\
	& \implies -2x_0 x = -x_0^2 - x^2 \\
	& \implies {(x - x_0)}^2 = 0
\end{align*}
ומצאנו כי השוויון מתקיים רק כאשר $x = x_0$.

\Subquestion{}
נגדיר פונקציה $g : \RR \to \RR$ על־ידי
\[
	g(x) = \begin{cases}
		x^2 \sin \frac{1}{x}, & x \ne 0 \\
		0, & x = 0
	\end{cases}
\]
נשים לב כי בנקודה $x_0 = 0$ מתקיים
\[
	g'(0)
	= \lim_{x \to x_0} \frac{g(x) - g(x_0)}{x - x_0}
	= \lim_{x \to 0} \frac{x^2 \sin \frac{1}{x} - 0}{x - 0}
	= \lim_{x \to 0} x \sin \frac{1}{x}
	= 0
\]
לכן הישר המשיק לפונקציה $g$ בנקודה $(0, 0)$ הוא $y = 0$, והוא כמובן נחתך אינסוף פעמים עם גרף הפונקציה $g$ בכל סיבהב מנוקבת של $x_0 = 0$.

\Question{}
\Subquestion{}
\subsubsection{i.}
ההגדרה לפונקציה זוגית וההגדרה לסימטריה סביב ציר ה־$y$ הלכה למעשה מתלכדות, שכן הגדרת הסימטריה סביב ישר $x = x_0$ היא $f(-x + x_0) = f(x + x_0)$.

\subsubsection{ii.}
נבחין כי ראשית הצירים היא מרכז סימטריה סיבובית ($180^\circ$) של פונקציה אי־זוגית. \\*
זאת שכן לכל נקודה שנבחר $(x, y) \in g$ נבחין כי גם $(-x, -y) \in g$ בעקבות הגדרת האי־סימטריה.

\Subquestion{}
תהי $g : \RR \to \RR$ גזירה בכל $\RR$, ונוכיח שאם $g$ זוגית אז נגזרתה $g'$ היא פונקציה אי־זוגית.
\begin{proof}
	נניח כי $g$ זוגית ולכן לכל $x \in \RR$ מתקיים $g(x) = g(-x)$.
	\begin{align*}
		g'(x_0) & = \lim_{x \to x_0} \frac{g(x) - g(x_0)}{x - x_0} \\
				& = \lim_{x \to x_0} \frac{g(x) - g(x_0)}{x - x_0} \\
				& = \lim_{x \to x_0} \frac{g(-x) - g(-x_0)}{-x - x_0} \\
				& = \lim_{x \to x_0} -\frac{g(-x) - g(-x_0)}{x - (-x_0)} \\
		g'(x_0) & = -g'(-x_0)
	\end{align*}
	ומצאנו כי הנגזרת מקיימת את ההגדרה לאי־סימטריה לכל $x \in \RR$.
\end{proof}

\Question{}
נוכיח כי פונקציית הנגזרת $f' : \RR \to \RR$ של הפונקציה שהגדרנו בשאלה 1 סעיף ב' היא לא רציפה.
\begin{proof}
	ראינו קודם כי מתקיים $f'(0) = 0$, ולכן נניח בשלילה כי גם $\lim_{x \to 0} f'(x) = 0$. \\*
	עתה נחשב את ערך הנגזרת בנקודה על־פי נוסחות גזירה:
	\[
		f'(x) = 2x \cdot \sin \frac{1}{x} + x^2 \cdot \cos \frac{1}{x} \cdot \frac{-1}{x^2} = 2x \sin \frac{1}{x} - \cos \frac{1}{x}
	\]
	בעוד אנו יודעים כי $2x \sin \frac{1}{x} \xrightarrow{x \to 0} 0$, גם ידוע כי $\cos \frac{1}{x}$ לא מתכנסת בנקודה, ולכן הגבול $\lim_{x \to 0} f'(x)$ לא מתקיים אף הוא, בסתירה להנחה. \\*
	לכן פונקציית הנגזרת איננה רציפה.
\end{proof}

\Question{}
בסעיפים הבאים נמצא תחומי רציפות וגזירות וערך נגזרת בנקודה לפונקציות נתונות.

\Subquestion{}
נגדיר $g = \ln(|f|)$ כאשר $f$ פונקציה גזירה ב־$\RR$ כך ש־$f(x) \ne 0$ לכל $x \in \RR$. \\*
נובע מההגדרות כי $f$ רציפה בכל $\RR$ ולכן ממשפט ערך הביניים נסיק כי $f(x) < 0$ לכל $x \in \RR$ או $f(x) > 0$ לכל $x \in \RR$, ולכן נניח בלי הגבלת הכלליות $f(x) > 0$. \\*
נובע אם כן כי $g(x) = \ln(f(x))$ ומתחום ההגדרה של $\ln$ נובע ישירות כי הפונקציה $g$ מוגדרת לכל $x \in \RR$. \\*
נחשב את הנגזרת על־פי חוקי גזירה, ונקבל כי $g'(x) = \frac{f'(x)}{f(x)}$. ידוע כי $f(x)$ חיובית, ולכן $g'(x)$ מוגדרת אף היא לכל $x \in \RR$.

\Subquestion{}
נגדיר $g(x) = x^\beta$ כאשר $\beta \in \RR \setminus \ZZ$. \\*
נשים לב כי הפונקציה בהכרח מוגדרת לכל $x > 0$, בהכרח לא מוגדרת עבור $x < 0$ ומוגדרת ב־$x = 0$ אם ורק אם $\beta > 0$. \\*
מחישוב עולה כי $g'(x) = \beta x^{\beta - 1}$ ולכן באופן דומה גם $g'$ מוגדרת לכל $x > 0$, לא מוגדרת ל־$x < 0$, ומוגדרת ב־$x = 0$ אם ורק אם $\beta > 1$.

\Subquestion{}
נגדיר $g(x) = \sqrt[3]{x^3 - x^4}$. \\*
הפונקציה $g$ מוגדרת בכל $x \in \RR$, שכן פונקציית השורש השלישי היא רציפה ואי־זוגית לכל $\RR$, וגם פולינומים מוגדרים ורציפים, ובסך הכול גם הפונקציה השלמה. \\*
נשים לב כי על־פי חוקי גזירה אשר נלמדו
\[
	g'(x) = \frac{3x^2 - 4x^3}{3 \sqrt[3]{{x^3(1 - x)}^2}}
	= \frac{3 - 4x}{3 \sqrt[3]{{(1 - x)}^2}}
\]
מביטוי זה נסיק כי פונקציית הנגזרת לא מוגדרת כאשר $3 \sqrt[3]{{(1 - x)}^2} = 0 \iff x = \pm 1$. \\*
לכן פונקציית הנגזרת מוגדרת ב־$\RR \setminus \{-1, 1\}$.

\Question{}
נוכיח כי לכל $k \in \NN$ ולכל $x > -1$ מתקיים
\[
	\ln^{(k)}(1 + x) = \frac{{(-1)}^{k - 1}(k - 1)!}{{(1 + x)}^k}
\]
\begin{proof}
	נוכיח באינדוקציה על $k$. \\*
	\textbf{בסיס האינדוקציה:} עבור $k = 1$ מתקיים $(\ln(x + 1))' = \frac{1}{x + 1}$. \\*
	\textbf{מהלך האינדוקציה:} נניח כי הטענה נכונה עבור $k$ כלשהו, ונראה כי מתקיים:
	\begin{align*}
		\ln^{(k + 1)}(1 + x)
		& = (\ln^{(k)}(1 + x))' \\
		& = (({(-1)}^{k - 1}(k - 1)!){(1 + x)}^{-k})' \\
		& = ({(-1)}^{k - 1}(k - 1)!)({(1 + x)}^{-k})' \\
		& = ({(-1)}^{k - 1}(k - 1)!) \cdot (-k) {(1 + x)}^{-k - 1} \\
		& = ({(-1)}^k(k)!) \cdot {(1 + x)}^{-k - 1} \\
		& = \frac{{(-1)}^k(k)!}{{(1 + x)}^{k + 1}} \\
	\end{align*}
	ומצאנו כי הטענה מתקיימת עבור $\ln^{(k + 1)}(1 + x)$.
\end{proof}

\Question{}
בסעיפים הבאים נחשב את נקודות הקיצון המקומי לפונקציות הנתונות.

\Subquestion{}
\[
	f(x) = \begin{cases}
		x^4, & x \ne 0 \\
		2, & x = 0
	\end{cases}
\]
אנו יודעים כי הפונקציה $f$ מונוטונית עולה ב־$x > 0$ ומונוטונית יורדת ב־$x < 0$ ולכן אין לה בתחומים אלה קיצון מקומי ועלינו לבחוןרק את $x = 0$.
נשים לב כי בסביבה המנוקבת $x \in {(-1, 1)}^*$ מתקיים $f(x) < 1 < f(0) = 2$ ולכן $(0, 2) \in f$ נקודת מקסימום מקומי.

\Subquestion{}
\[
	g(x) = \begin{cases}
		x^4, & x \ne 0 \\
		-2, & x = 0
	\end{cases}
\]
נשים לב כי באופן דומה לפונקציה אין מקסימום או מינימום משום סוג עבור $|x| > 0$, ולכן נבחן את $x = 0$. \\*
נראה כי $f(0) = -2 < 0 < x^4$ לכל $x \in \RR\setminus\{0\}$ ולכן $(0, -2) \in g$ נקודת מינימום מוחלטת.

\Subquestion{}
\[
	h(x) = \begin{cases}
		-x, & x \in \QQ \\
		x, & x \in \RR \setminus \QQ \\
	\end{cases}
\]
תהי נקודה $x_0 \in \RR$. מהגדרת הפונקציה וצפיפות הרציונליים והאי־רציונליים, לכל סביבה מנוקבת של $x_0$ נוכל למצוא $x_1 \ge x_0$ כך ש־$x_1 \in \RR \setminus \QQ$ ולכן $h(x_0) < h(x_1)$ והיא לא נקודת מקסימום מקומי.
באופן דוגמה מצפיפות הרציונליים נובע שבכל סביבה כזו ישנו גם $x_2 > x_0$ כך ש־$x_2 \in \QQ$ ולכן $f(x_2) < f(x_0)$ ובהתאם הוא גם לא מינימום מקומי. \\*
מצאנו כי לכל נקודה היא לא מינימום או מקסימום מקומי ולכן לפונקציה $h$ אין בכלל קיצון מקומי.

\Subquestion{}
\[
	\begin{cases}
		x + 3, & x \le 0 \\
		-x + 3, & 0 < x \le 3 \\
		0, & x > 3
	\end{cases}
\]
תחילה נשים לב שהפונקציה מונוטונית עולה לכל $x < 0$ ולכן אין בקטע זה בכלל נקודות מקסימום או מינימום. \\*
בנקודה $x = 0$ אנו רואים כי הפונקציה $h$ מונוטונית עולה בסביבה השמאלית ומונוטונית יורדת בסביבה ימנית ולכן $(0, 3) \in h$ נקודת מקסימום מקומי. \\*
בקטע $(0, 3)$ הפונקציה יורדת ולכן אין שם מינימום או מקסימום. \\*
עבור $x \ge 3$ מתקיים $h(x) = 0$ ולכן על־פי הגדרת מינימום ומקסימום אוסף הנקודות האלה מהוות מינימום ומקסימום מקומיים.

\Subquestion{}
\[
	\mathcal{D}(x) = \begin{cases}
		1, & x \in \QQ \\
		0, & x \in \RR\setminus\QQ
	\end{cases}
\]
נבחר נקודה $x_0 \in \QQ$, אז נוכל למצוא בכל סביבה שלה ערך אי־רציונלי אשר ערכו קטן מערך הפונקציה בנקודה, לכן $(x_0, 1)$ לא נקודת מינימום, אבל ידוע כי $f(x_0) = 1 \ge 1 > 0$ ולכן זוהי נקודת מקסימום מקומי. \\*
באופן דומה לכל נקודה $x_1 \in \RR\setminus\QQ$ נובע ש־$(x_1, 0)$ היא נקודת מינימום מקומי אך לא נקודת מקסימום.

\Question{}
נגדיר
\[
	\sinh(x) = \frac{e^x - e^{-x}}{2},
	\cosh(x) = \frac{e^x + e^{-x}}{2}
\]

\Subquestion{}
נוכיח ש־$\sinh$ היא אי־זוגית.
\begin{proof}
	לכל $x \in \RR$ מתקיים:
	\[
		\sinh(-x) = \frac{e^{-x} - e^x}{2} = - \frac{e^x - e^{-x}}{2} = - \sinh x
	\]
\end{proof}

\Subquestion{}
נוכיח ש־$\cosh$ היא פונקציה זוגית.
\begin{proof}
	לכל $x \in \RR$ מתקיים:
	\[
		\cosh(-x) = \frac{e^{-x} + e^{-(-x)}}{2} = \cosh x
	\]
\end{proof}

\Subquestion{}
נוכיח שמתקיים $\forall x, y \in \RR : \cosh(x + y) = \cosh(x) \cosh(y) + \sinh(x) \sinh(y)$.
\begin{proof}
	\begin{align*}
		\cosh(x + y)
		& = \frac{e^{x + y} + e^{-x-y}}{2} \\
		& = \frac{e^x e^y + e^{-x} e^{-y}}{2} \\
		& = \frac{2e^x e^y + 2e^{-x} e^{-y} - e^x e^{-y} - e^{-x}e^y + e^x e^{-y} + e^{-x}e^y }{4} \\
		& = \frac{(e^x + e^{-x})(e^y + e^{-y}) + e^x e^y + e^{-x} e^{-y} - e^x e^{-y} - e^{-x}e^y }{4} \\
		& = \frac{(e^x + e^{-x})(e^y + e^{-y}) + (e^x - e^{-x})(e^y - e^{-y}) }{4} \\
		\cosh(x + y) & = \cosh(x) \cosh(y) + \sinh(x) \sinh(y)
	\end{align*}
\end{proof}

\Subquestion{}
נוכיח כי $\cosh^2(x) - \sinh^2(x) = 1$ לכל $x \in \RR$.
\begin{proof}
	נשתמש בזהות מהסעיף הקודם עבור $\cosh(x - x)$ ונקבל מזוגיות ואי־זוגיות כי
	\[
		1 = \cosh(0) = \cosh(x - x)
		= \cosh^2(x) - \sinh^2(x)
	\]
\end{proof}
נסיק מהעברת אגפים ושורש כי $\cosh(x) = \sqrt{1 + \sinh^2(x)}$.

\Subquestion{}
נוכיח כי $\sinh' = \cosh, \cosh' = \sinh$.
\begin{proof}
	נשתמש בחוקי גזירה:
	\[
		\sinh'(x) = \left( \frac{e^x - e^{-x}}{2}\right)'
		= \frac{e^x - (-1)e^{-x}}{2}
		= \cosh(x)
	\]
	באופן דומה נקבל גם
	\[
		\cosh'(x) = \left( \frac{e^x + e^{-x}}{2}\right)'
		= \frac{e^x + (-1)e^{-x}}{2}
		= \sinh(x)
	\]
\end{proof}

\Subquestion{}
נוכיח כי
\[
	\lim_{x \to \infty} \cosh(x)
	= \lim_{x \to -\infty} \cosh(x)
	= \lim_{x \to \infty} \sinh(x)
	= \infty,
	\lim_{x \to -\infty} \sinh(x)
	= -\infty
\]
\begin{proof}
	תחילה ניזכר בגבולות הבאים:
	\[
		\lim_{x \to \infty} e^x = \infty, \lim_{x \to -\infty} e^x = 0
	\]
	לכן גם $\lim_{x \to \infty} e^x \pm e^{-x} = \infty$ ומכאן נובע ישירות כי $\lim_{x \to \infty} \cosh(x) = \lim_{x \to \infty} \sinh(x) = \infty$. \\*
	$\cosh(x)$ פונקציה סימטרית ולכן נובע גם $\lim_{x \to -\infty} \cosh(x) = \infty$ ומאי־זוגיות $\sinh(x)$ נובע $\lim_{x \to -\infty} \sinh(x) = -\infty$.
\end{proof}

\end{document}
