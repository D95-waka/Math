\documentclass[a4paper]{article}

% packages
\usepackage{inputenc, fontspec, amsmath, amsthm, amsfonts, polyglossia, catchfile}
\usepackage[a4paper, margin=50pt, includeheadfoot]{geometry} % set page margins

% style
\AddToHook{cmd/section/before}{\clearpage}	% Add line break before section
\linespread{1.5}
\setcounter{secnumdepth}{0}		% Remove default number tags from sections
\setmainfont{Libertinus Serif}
\setsansfont{Libertinus Sans}
\setmonofont{Libertinus Mono}
\setdefaultlanguage{hebrew}
\setotherlanguage{english}

% operators
\DeclareMathOperator\cis{cis}
\DeclareMathOperator\Sp{Sp}
\DeclareMathOperator\tr{tr}
\DeclareMathOperator\im{Im}
\DeclareMathOperator\diag{diag}
\DeclareMathOperator*\lowlim{\underline{lim}}
\DeclareMathOperator*\uplim{\overline{lim}}

% commands
\renewcommand\qedsymbol{\textbf{משל}}
\newcommand{\NN}[0]{\mathbb{N}}
\newcommand{\ZZ}[0]{\mathbb{Z}}
\newcommand{\QQ}[0]{\mathbb{Q}}
\newcommand{\RR}[0]{\mathbb{R}}
\newcommand{\CC}[0]{\mathbb{C}}
\newcommand{\getenv}[2][] {
  \CatchFileEdef{\temp}{"|kpsewhich --var-value #2"}{\endlinechar=-1}
  \if\relax\detokenize{#1}\relax\temp\else\let#1\temp\fi
}
\newcommand{\explain}[2] {
	\begin{flalign*}
		 && \text{#2} && \text{#1}
	\end{flalign*}
}

% headers
\getenv[\AUTHOR]{AUTHOR}
\author{\AUTHOR}
\date\today

\title{פתרון ממ''ן 16 – חשבון אינפיניטסימלי 2 (20475)}

\begin{document}
\maketitle
\maketitleprint{}

\section{שאלה 1}
\subsection{סעיף א'}
נחשב את הגבולות הבאים או נראה כי אינם מתקיימים

\subsubsection{1}
\[
	\lim_{(x, y) \to (0, 0)} (x + y) \ln(x^4 + y^4)
\]
נגדיר $f(x, y) = (x + y) \ln(x^4 + y^4)$ ונניח על־פי היינה כי הפונקציה שואפת ב־$0$ ל־$0$, \\*
לכן ישנן שתי סדרות נקודות $x_n \underset{n \to \infty}{\to} 0$ ו־$y_n \underset{n \to \infty}{\to} 0$. \\*
לכן
\[
	\lim_{n \to \infty} f(x_n, y_n)
	= \lim_{n \to \infty} (x_n + y_n) \ln(x_n^4 + y_n^4)
	= (\lim_{n \to \infty} x_n + y_n) \ln(\lim_{n \to \infty} x_n^4 + y_n^4)
\]
נשים לב כי כמעט לכל $n$ מתקיים ${(x_n + y_n)}^4 > x_n^4 + y_n^4$ ולכן נוכל על־ידי שימוש בכלל הסנדוויץ'
\[
	(\lim_{n \to \infty} x_n + y_n) \ln(\lim_{n \to \infty} {(x_n + y_n)}^4)
\]
אנו יודעים כי $\lim_{n \to \infty} x_n + y_n = 0$ ולכן הגבול שקול לגבול
\[
	\lim_{n \to \infty} n \ln n = 0
\]
ומצאנו כי הגבול מתכנס.

\subsubsection{2}
\[
	\lim_{(x, y) \to (0, 0)} \frac{xy^3}{x^2 + y^6}
\]
נניח כי סדרת הנקודות $(p_n = (x_n, y_n))$ מתכנסת ל־$L$,
ולכן מהגדרת היינה לגבולות בשני משתנים נובע שהגבול מתכנס עבור $p_n = (\frac{1}{n}, 0)$:
\[
	\lim_{n \to \infty} f(p_n)
	= \frac{\frac{0^3}{n}}{\frac{1}{n^2} + 0^6}
	= L = 0
\]
כמו־כן, הגבול מתכנס עבור הסדרה $p_n = (\frac{1}{n^3}, \frac{1}{n})$:
\[
	\lim_{n \to \infty} f(p_n)
	= \frac{\frac{1}{n^3 \cdot n^3}}{\frac{1}{n^6} + \frac{1}{n^6}}
	= L = \frac{1}{2}
\]
אז מצאנו כי עבור סדרות שונות הגבול מתכנס לערכים שונים ובהתאם להגדרת היינה הגבול לא מתכנס בנקודה.

\subsection{סעיף ב'}
נבדוק את רציפות הפונקציות הבאות ב־$\RR^2$:

\subsubsection{1}
\[
	f(x, y) = \begin{cases}
		\frac{xy}{x^2 + y^2} & (x, y) \ne (0, 0) \\
		0 & (x, y) = (0, 0) \\
	\end{cases}
\]
לכל נקודה $p_0 = (x_0, y_0)$ כאשר $x_0, y_0 \ne 0$ נראה כי נובע מרציפות והגדרת היינה להתכנסות כי
\[
	\lim_{p \to p_0} f(p)
	= \frac{\lim_{(x, y) \to p_0} xy}{\lim_{(x, y) \to p_0} x^2 + y^2}
	= \frac{x_0y_0}{x_0^2 + y_0^2}
	= f(p_0)
\]
אז מצאנו כי $f$ רציפה לכל $p \ne \textbf{0}$. \\*
מדוגמה 7.13 עולה כי ל־$f$ אין גבול בנקודה $\textbf{0}$ ולכן בהכרח גם איננה מתכנסת בנקודה.

\subsubsection{2}
\[
	\begin{cases}
		\frac{\sin(2x^2 + 2y^2)}{e^{x^2 + y^2} - 1} & (x, y) \ne (0, 0)\\
		2 & (x, y) = (0, 0)
	\end{cases}
\]
תחילה נשים לב כי הפונקציה תלויה רק בערך $x^2 + y^2 = \lVert (x, y) \rVert^2$, ולכן נוכל להגדיר:
\[
	g(x) = \begin{cases}
		\frac{\sin(2x)}{e^{x} - 1} & x > 0 \\
		2 & x = 0
	\end{cases},
	f(p) = g(\lVert p \rVert^2)
\]
ידוע כי פעולת הנורמה היא פעולה רציפה (כנראה) ולכן ממשפט הרכבת פונקציות נובע כי אם $g$ רציפה בכל תחומה אז גם $f$ רציפה בכל המישור. \\*
$g$ היא כמובן רציפה לכל $x > 0$, ולכן נבדוק את גבולה ב־$0$:
\[
	\lim_{x \to \infty} g(x)
	= \lim_{x \to \infty} \frac{\sin(2x)}{e^x - 1}
	= \lim_{x \to \infty} \frac{2\cos(2x)}{e^x}
	= \frac{2}{1}
	= 2
\]
אז מצאנו כי $g$ רציפה לכל $x \ge 0$ ובהתאם $f$ רציפה בכל המישור.

\section{שאלה 2}
נמצא את כל הנקודות $(x, y) \in \RR^2$ בהן הפונקציה $f(x, y) = \sqrt{\lvert xy^3 \rvert}$ היא דיפרנציאבילית. \\*
נמצא את הנגזרות החלקיות של $f$ כאשר $xy$ חיובי וכאשר הוא שלילי:
\begin{align*}
	xy \ge 0, f_x(x, y)
	& = \frac{y^3}{2 \sqrt{xy^3}}
	= \frac{y^2}{2 \sqrt{xy}} \\
	xy < 0, f_x(x, y)
	& = \frac{-y^3}{2 \sqrt{-xy^3}}
	= \frac{-y^2}{2 \sqrt{-xy}} \\
	xy \ge 0, f_y(x, y)
	& = \frac{3xy^2}{2 \sqrt{xy^3}}
	= \frac{3xy}{2 \sqrt{xy}}
	= \frac{3}{2} \sqrt{xy} \\
	xy < 0, f_y(x, y)
	& = \frac{-3xy}{2 \sqrt{-xy}}
	= \frac{-3}{2} \sqrt{-xy} \\
\end{align*}
נשים לב כי עבור $x \ne 0, y \ne 0$ כל נקודה שייכת לזוג נגזרות חלקיות ורציפה בהן ובהתאם לתנאי המספיק לדיפרנציאביליות $f$ דיפרנציאבילית בנקודה $(x, y)$. \\*
נבדוק את התכנסות הנגזרות כאשר $y$ הוא ערך קבוע לא אפס ו־$x = 0$: \\*
מונה $f_x$ הוא רציף ושואף לערך סופי, בעוד המכנה שואף לאינסוף לכל $y$ ולכן $f_x$ לא רציפה בנקודות אלה,
וממשפט 7.63 אנו יכולים להסיק כי $f$ לא דיפרנציאבילית בתחום זה. \\*
נבדוק את התכנסות הנגזרות כאשר $x$ הוא ערך קבוע ו־$y = 0$: \\*
בנקודות אלה $f_x$ כמובן רציפה, שכן המונה שלה מתאפס והמכנה רציף ולא מתאפס. כמו־כן גם $f_y$ פונקציה רציפה ומתאפסת באוסף הנקודות.
לכן $f$ דיפרנציאבילית באוסף הנקודות $(x, 0)$. \\*
מצאנו כי $f$ דיפרנציאבילית לכל $\RR^2$ חוץ מהנקודות $(0, t)$ כאשר $t \ne 0$.

\section{שאלה 3}
\subsection{סעיף א'}
נוכיח כי $u(x, y) = \ln(x^2 + y^2)$ היא הרמונית לכל $\RR^2 \backslash \{(0, 0)\}$.
\begin{proof}
	נחשב את נגזרותיה החלקיות מסדר שני של $u$ בתחום:
	\begin{align*}
		u_x(x, y) & = \frac{2x}{x^2 + y^2} \\
		u_{xx}(x, y) & = \frac{2x^2 + 2y^2 - 4x^2}{{(x^2 + y^2)}^2} = 2\frac{y^2 - x^2}{{(x^2 + y^2)}^2} \\
		u_y(x, y) & = \frac{2y}{x^2 + y^2} \\
		u_{yy}(x, y) & = \frac{2x^2 + 2y^2 - 4y^2}{{(x^2 + y^2)}^2} = 2\frac{x^2 - y^2}{{(x^2 + y^2)}^2} \\
	\end{align*}
	נבדוק
	\[
		f_{xx}(x, y) + f_{yy}(x, y)
		= 2\frac{y^2 - x^2}{{(x^2 + y^2)}^2} + 2\frac{x^2 - y^2}{{(x^2 + y^2)}^2}
		= 0
	\]
	ומצאנו כי $f$ הרמונית בתחום $\RR^2\backslash\{(0, 0)\}$.
\end{proof}

\subsection{סעיף ב'}
נגדיר $f(x, y) = u(x^2 - y^2, y^2 - x^2)$ כאשר $u(t, s)$ פונקציה דיפרנציאבילית בכל נקודה. \\*
נוכיח כי $x \, f_y + y \, f_x = 0$ לכל $x, y$.
\begin{proof}
	נגדיר $t(u, v) = u^2 - v^2$, ולכן $f(p) = u(t(p), -t(p))$.
	נחשב את נגזרותיה החלקיות של $f$ על־פי כלל השרשרת המוכלל מהתוספת לחומר של יחידה 7:
	\[
		f_x(x, y)
		= u_x t_u + u_y t_v
		= u_x(x, y) \cdot 2x + u_y(x, y) \cdot (-2y)
		= 2x (u_x(x, y) - u_y(x, y))
	\]
	ובאופן דומה נקבל גם
	\[
		f_y(x, y)
		= u_x t_v + u_y (-t_v)
		= (-2x) u_x - (-2x) u_y
		= 2y (-u_x(x, y) + u_y(x, y))
	\]
	ונקבל כי
	\[
		x f_y + y f_x
		= 2xy (u_x(x, y) - u_y(x, y))
		+ 2yx (-u_x(x, y) + u_y(x, y))
		= 0
	\]
	ומצאנו כי השוויון מתקיים.
\end{proof}

\subsection{סעיף ג'}
נמצא את הקצב בו משתנה שטח מלבן אשר ברגע נתון אורכו 15 מטרים והוא קטן ב־3 מטרים לשנייה, \\*
ואשר רוחבו הוא 6 מטרים והוא גדל ב־2 מטרים לשנייה. \\*
נגדיר $f(x, y) = xy$ פונקציית השטח של מלבן. \\*
עוד נגדיר $h(t)$ פונקציית הגובה של המלבן לפי זמן ו־$w(t)$ פונקציית הרוחב של המלבן לפי זמן. \\*
נגדיר $F(t) = f(h(t), w(t))$ פונקציית השטח של המלבן הנתון לפי זמן. \\*
נגדיר את הרגע הנתון בשאלה בתור $t = 0$ ולכן $h(0) = 15, h'(0) = -3, w(0) = 6, w'(0) = 2$. \\*
נשים לב כי $f_x(x, y) = y, f_y(x, y) = x$ ולכן מכלל השרשרת בשני משתנים נובע כי
\[
	F'(t) = f_x(h(t), w(t)) \cdot h'(t) + f_y(h(t), w(t)) \cdot w'(t)
	= w(t) h'(t) + h(t) w'(t)
\]
נציב
\[
	F'(0) = 6 \cdot (-3) + 15 \cdot 2 = 12
\]

\clearpage
\subsection{סעיף ד'}
תהי $f(x, y)$ דיפרנציאבילית במישור כולו. \\*
נוכיח כי לכל שתי נקודות $p_1 = (x_1, y_1)$ ו־$p_2 = (x_2, y_2)$,
בקטע המחבר בין $p_1$ ל־$p_2$ קיימת נקודה $p_c$ כך שמתקיים
\[
	f(p_2) - f(p_1) = f_x(p_c) (x_2 - x_1) + f_y(p_c) (y_2 - y_1)
\]
\begin{proof}
	תהי $l : \RR \to \RR^2$ העתקה לינארית המוגדרת על־ידי
	\[
		l(t) = p_1 + (p_2 - p_1) t = (x_1 + (x_2 - x_1) t, y_1 + (y_2 - y_1) t)
	\]
	ונגדיר גם $g(t) = f(l(t))$.
	נשים לב כי $l$ היא הישר העובר דרך הנקודות $p_1$ ו־$p_2$, והפונקציה אף מקיימת $l(0) = p_1, l(1) = p_2$. \\*
	בד בבד הפונקציה $g$ היא ''החתך'' של $f$ עבור ערכי הנקודות על הישר $t$, דהינו זו פונקציה ממשתנה אחד. \\*
	ממשפט הערך הממוצע של החשבון הדיפרנציאלי נובע כי קיים ערך $t_c$ אשר מקיים
	\[
		g'(t_c) (1 - 0) = g(1) - g(0) = f(p_2) - f(p_1) \tag{1}
	\]
	כאשר $0 < t_c < 1$. \\*
	נחשב את $g'(t)$ על־פי כלל השרשרת בשני משתנים:
	\begin{align*}
		g'(t)
		= & \frac{d}{dx} f(x_1 + (x_2 - x_1)t, y_1 + (y_2 - y_1)t) \\
		= & f_x(x_1 + (x_2 - x_1)t, y_1 + (y_2 - y_1)t) \cdot (x_1 + (x_2 - x_1) t)' \\
		  & + f_y(x_1 + (x_2 - x_1)t, y_1 + (y_2 - y_1)t) \cdot (y_1 + (y_2 - y_1) t)' \\
		= & f_x(l(t)) (x_2 - x_1) + f_y(l(t)) (y_2 - y_1)
	\end{align*}
	ועל־ידי שימוש בנוסחה זו ב־$(1)$ נקבל
	\[
		f_x(l(t_c)) (x_2 - x_1) + f_y(l(t_c)) (y_2 - y_1) = f(p_2) - f(p_1)
	\]
	נגדיר $p_c = l(t_c)$, ברור כי הנקודה נמצאת בין $p_1$ לבין $p_2$, ונקבל כי
	\[
		f_x(p_c) (x_2 - x_1) + f_y(p_c) (y_2 - y_1) = f(p_2) - f(p_1)
	\]
	והוכחנו את הטענה.
\end{proof}

\section{שאלה 4}
\subsection{סעיף א'}
מטייל עולה על הר שצורתו נתונה על־ידי הנוסחה
\[
	h(x, y) = 1000 - 0.05 x^2 - 0.04 y^2
\]
נתון כי המטייל נמצא בנקודה $(60, 100)$, נמצא את הכיוון עליו ללכת כדי להגיע לפסגה מהר ככל האפשר. \\*
המטייל יגיע לפסגה במהירות הגבוהה ביותר אם ילך בדרך התלולה ביותר, ולכן עלינו למצוא את הכיוון בו הנגזרת הכיוונית היא הגבוהה ביותר. \\*
נשים לב כי
\[
	f_x(x, y) = -0.1x,
	f_y(x, y) = -0.08y
\]
נחשב את הגרדיאנט של $h$:
\[
	\nabla f(60, 100)
	= (f_x(60, 100), f_y(60, 100))
	= (-6, -8)
\]
נחשב את הווקטור הנורמלי:
\[
	u = \frac{(-6, -8)}{\lVert(-6, -8)\rVert} = (-0.6, -0.8)
\]
ולכן הכיוון שעל המטייל ללכת בו הוא $(-0.6, -0.8)$.

\subsection{סעיף ב'}
נמצא את הנקודה על המישור $x + 2y + z = 1$ הקרובה ביותר לראשית הצירים. \\*
נשים לב כי $z = 1 - x - 2y$ ולכן נגדיר פונקציה $z(x, y) = 1 - x - 2y$ ונקבל כי לכל $x, y \in \RR$ הנקודה $(x, y, z(x, y))$ על המישור הנתון. \\*
על־פי נוסחת המרחק בין נקודות (פיתגורס) בין נקודה על המישור לבין נקודת האפס הוא:
\[
	d_0(x, y)
	= \sqrt{x^2 + y^2 + z^2(x, y)}
	= \sqrt{x^2 + y^2 + {(1 - x - 2y)}^2}
\]
כמובן שהפונקציה $d_0$ מקבלת מינימום ומקסימום באותן הנקודות כמו $d_0^2$, ולכן נגדיר $d(x, y) = d_0^2(x, y)$ ונחקור אותה. \\*
נמצא את נקודת המינימום של הפונקציה הנתונה בעזרת משפט 7.58, דהינו נמצא נקודות בהן שתי הנגזרות החלקיות מתאפסות.
\[
	d_x(x, y) = 2x - 2(1 - x - 2y) = -2 + 4x + 4y,
	d_y(x, y) = 2y - 4(1 - x - 2y) = -4 + 4x + 10y,
\]
נמיר למטריצת מקדמים ונפתור
\[
	\begin{pmatrix}
		4 & 4 & \vline & 2 \\
		4 & 10 & \vline & 4
	\end{pmatrix}
	\rightarrow
	\begin{pmatrix}
		4 & 4 & \vline & 2 \\
		0 & 1 & \vline & \frac{1}{3}
	\end{pmatrix}
	\rightarrow
	\begin{pmatrix}
		4 & 0 & \vline & \frac{2}{3} \\
		0 & 1 & \vline & \frac{1}{3}
	\end{pmatrix}
\]
ומצאנו כי יש למערכת פתרון יחיד $(\frac{1}{6}, \frac{1}{3})$. \\*
נחשב את הנגזרת השנייה $d_{xx}(x, y) = 4$, ולכן ממשפט 7.72 נובע כי
$(\frac{1}{6}, \frac{1}{3})$ היא נקודת מינימום של הפונקציה ובה המרחק מהראשית הוא הקטן ביותר על מישור.

\end{document}
