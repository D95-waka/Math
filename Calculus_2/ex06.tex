\documentclass[a4paper]{article}

% packages
\usepackage{inputenc, amsmath, amsthm, thmtools, amsfonts, amssymb, luacode, catchfile, tikzducks, hyperref}
\usepackage[a4paper, margin=50pt, includeheadfoot]{geometry} % set page margins
\usepackage[shortlabels]{enumitem}
\usepackage[skip=3pt, indent=0pt]{parskip}

% language
\usepackage[bidi=basic, layout=tabular, provide=*]{babel}
\babelprovide[main, import]{hebrew}
\babelprovide{rl}
\babelfont{rm}{Libertinus Serif}
\babelfont{sf}{Libertinus Sans}
\babelfont{tt}{Libertinus Mono}

% style
\AddToHook{cmd/section/before}{\clearpage}	% Add line break before section
\linespread{1.3}
\setcounter{secnumdepth}{0}		% Remove default number tags from sections, this won't do well with theorems
\AtBeginDocument{\setlength{\belowdisplayskip}{3pt}}
\AtBeginDocument{\setlength{\abovedisplayskip}{3pt}}

% operators
\DeclareMathOperator\cis{cis}
\DeclareMathOperator\Sp{Sp}
\DeclareMathOperator\tr{tr}
\DeclareMathOperator\im{Im}
\DeclareMathOperator\re{Re}
\DeclareMathOperator\diag{diag}
\DeclareMathOperator*\lowlim{\underline{lim}}
\DeclareMathOperator*\uplim{\overline{lim}}
\DeclareMathOperator\rng{rng}
\DeclareMathOperator\Sym{Sym}
\DeclareMathOperator\Arg{Arg}
\DeclareMathOperator\Log{Log}
\DeclareMathOperator\dom{dom}

% commands
%\renewcommand\qedsymbol{\textbf{מש''ל}}
%\renewcommand\qedsymbol{\fbox{\emoji{lizard}}}
\newcommand{\NN}[0]{\mathbb{N}}
\newcommand{\ZZ}[0]{\mathbb{Z}}
\newcommand{\QQ}[0]{\mathbb{Q}}
\newcommand{\RR}[0]{\mathbb{R}}
\newcommand{\CC}[0]{\mathbb{C}}
\newcommand{\FF}[0]{\mathbb{F}}
\newcommand{\PP}[0]{\mathbb{P}}
\newcommand{\TT}[0]{\mathbb{T}}
\newcommand{\acts}[0]{\circlearrowright}
\newcommand{\explain}[2] {
	\begin{flalign*}
		 && \text{#2} && \text{#1}
	\end{flalign*}
}
\newcommand{\maketitleprint}[0]{ \begin{center}
	\begin{tikzpicture}[scale=3]
		\duck[graduate=gray!20!black, tassel=red!70!black]
	\end{tikzpicture}	
\end{center}
}

% theorem commands
\newtheoremstyle{c_remark}
	{}	% Space above
	{}	% Space below
	{}% Body font
	{}	% Indent amount
	{\bfseries}	% Theorem head font
	{}	% Punctuation after theorem head
	{.5em}	% Space after theorem head
	{\thmname{#1}\thmnumber{ #2}\thmnote{ \normalfont{\text{(#3)}}}}	% head content
\newtheoremstyle{c_definition}
	{3pt}	% Space above
	{3pt}	% Space below
	{}% Body font
	{}	% Indent amount
	{\bfseries}	% Theorem head font
	{}	% Punctuation after theorem head
	{.5em}	% Space after theorem head
	{\thmname{#1}\thmnumber{ #2}\thmnote{ \normalfont{\text{(#3)}}}}	% head content
\newtheoremstyle{c_plain}
	{3pt}	% Space above
	{3pt}	% Space below
	{\itshape}% Body font
	{}	% Indent amount
	{\bfseries}	% Theorem head font
	{}	% Punctuation after theorem head
	{.5em}	% Space after theorem head
	{\thmname{#1}\thmnumber{ #2}\thmnote{ \text{(#3)}}}	% head content

\theoremstyle{c_plain}
\newtheorem{theorem}{משפט}[section]
\newtheorem{lemma}[theorem]{למה}
\newtheorem{proposition}[theorem]{טענה}
\newtheorem*{proposition*}{טענה}
%\newtheorem{corollary}[theorem]{אין חלופה עברית}

\theoremstyle{c_definition}
\newtheorem{definition}[theorem]{הגדרה}
\newtheorem*{definition*}{הגדרה}
\newtheorem{example}{דוגמה}[section]
\newtheorem{exercise}{תרגיל}[section]

\theoremstyle{c_remark}
\newtheorem*{remark}{הערה}
\newtheorem*{solution}{פתרון}
\newtheorem{conclusion}[theorem]{מסקנה}
\newtheorem{notation}[theorem]{סימון}

% Questions related commands
\newcounter{question}
\setcounter{question}{1}
\newcounter{sub_question}
\setcounter{sub_question}{1}

\newcommand{\question}[1][0]{
	\ifthenelse{#1 = 0}{}{\setcounter{question}{#1}}
	\subsection{שאלה \arabic{question}}
	\addtocounter{question}{1}
	\setcounter{sub_question}{1}
}

\newcommand{\subquestion}[1][0]{
	\ifthenelse{#1 = 0}{}{\setcounter{sub_question}{#1}}
	\subsubsection{סעיף \localecounter{letters.gershayim}{sub_question}}
	\addtocounter{sub_question}{1}
}

% import lua and start of document
\directlua{common = require ('../common')}

\GetEnv{AUTHOR}

% headers
\author{\AUTHOR}
\date\today

\usepackage{tikz}
\DeclareMathOperator\arcsinh{arcsinh}
\title{פתרון מטלה 6 – חשבון אינפיניטסימלי 2 (80132)}

\begin{document}
\maketitle
\maketitleprint{}

\Question{}
תהי פונקציה $f$ הרציפה בקטע $[a, b]$ ו־$K$־ליפשיצית בקטע הפתוח $(a, b)$.

\Subquestion{}
נוכיח כי $f$ היא גם $K$־ליפשיצית בקטע הסגור $[a, b]$.
\begin{proof}
	יהיו $a < x_1 < x_2 < b$ ולכן נקבל כי $|f(x_1) - f(x_2)| \le K|x_1 - x_2|$. \\*
	נבחין כי שני האגפים בשוויון המתקבל מייצגים פונקציות רציפות (בקיבוע אחד המשתנים) ולכן ניעזר בהגדרת הרציפות ונקבל את הגבול
	\[
		\lim_{x_2 \to b} |f(x_1) - f(x_2)| \le \lim_{x_2 \to \infty} K|x_1 - x_2|
	\]
	ונסיק מהרציפות
	\[
		|f(x_1) - f(b)| \le K |x_1 - b|
	\]
	נעשה את התהליך גם על $x_1$ עבור קיבוע של $x_2 \le b$ ונקבל כי $f$ אכן $K$־ליפשיצית ב־$[a, b]$.
\end{proof}

\Subquestion{}
נוכיח כי לכל חלוקה $P = \{ x_0, x_1, \dots, x_n \}$ של $[a, b]$ מתקיים
\[
	U(f, P) - L(f, P) \le K (b - a) \Delta(P)
\]
\begin{proof}
	נראה כי
	\[
		U(f, P) - L(f, P)
		= \sum_{i = 0}^{n - 1} (M_i - m_i)(x_{i + 1} - x_i)
	\]
	ואנו יודעים כי קיימים $m_i^x, M_i^x$ כך ש־$f(m_i^x) = m_i$ וגם $f(M_i^x) = M_i$ כנביעה מהגדרה ולכן
	\[
		U(f, P) - L(f, P)
		= \sum_{i = 0}^{n - 1} (f(M_i^x) - f(m_i^x))(x_{i + 1} - x_i)
		\le \sum_{i = 0}^{n - 1} K(M_i^x - m_i^x)(x_{i + 1} - x_i)
	\]
	נשים לב כי $M_i^x - m_i^x \le x_{i + 1} - x_i$ ולכן מספיק שנבחר קטע חלוקה מקסימלי ומאי־שוויון משולש נקבל
	\[
		U(f, P) - L(f, P)
		\le K \sum_{i = 0}^{n - 1} \Delta(P) (x_{i + 1} - x_i)
		\le K \Delta(P) \sum_{i = 0}^{n - 1} (x_{i + 1} - x_i)
		= K(b - a) \Delta(P)
	\]
	וקיבלנו כי אי־השוויון מתקיים.
\end{proof}

\Subquestion{}
נמצא חלוקה שווה $P$ של הקטע $[0, 1]$ כך ש־$U(\cos, P) - L(\cos, P) < 0.001$.

אנו יודעים כי $\cos$ היא $1$־ליפשיצית ולכן נקבל מתוצאת הסעיף הקודם כי
\[
	U(\cos, P) - L(\cos, P) \le 1 \cdot (1 - 0) \Delta(P) = \frac{1}{n}
\]
כאשר $n$ הוא כמות הנקודות בחלוקה $P$. \\*
לכן אם כן מספיק ש־$n = 1000$ והתנאי יתקיים.

\Question{}
תהי $f : [a, b] \to \RR$ אינטגרבילית ב־$[a, b]$, ונגדיר פונקציות חדשות $f_-, f_+$ בקטע זה על־ידי
\[
	\forall x \in [a, b]
	\quad
	f_+(x) = \max\{f(x), 0\},
	\qquad
	f_-(x) = -\min\{f(x), 0\}
\]

\Subquestion{}
\subsubsection{i.}
נוכיח כי $0 \le f_+(x)$.
\begin{proof}
	נחלק למקרים:
	\begin{itemize}
		\item אם $f(x) \ge 0$ אז נקבל $f_+(x) = f(x) \ge 0$.
		\item אם $f(x) < 0$ אז נקבל $f_+(x) = 0 \ge 0$
	\end{itemize}
	ולכן בכל מקרה $f_+(x) \ge 0$.
\end{proof}

\subsubsection{ii.}
נוכיח כי $f_-(x) \ge 0$.
\begin{proof}
	גם הפעם נראה שאם $f(x) \le 0$ אז $f_-(x) = -f(x) \ge 0$ ואילו $f(x) > 0$ אז $f_-(x) = 0 \ge 0$.
\end{proof}

\subsubsection{iii.}
נוכיח כי $f = f_+ - f_-$.
\begin{proof}
	נחלק שוב למקרים, תחילה נראה כי אם $f(x) = 0$ אז $f_+(x) = f_-(x) = 0$ ונקבל כי השוויון מתקיים. \\*
	כאשר $f(x) > 0$ נקבל כי $f_+(x) = f(x)$ ואילו כי $f_-(x) = 0$ ולכן השוויון עודנו מתקיים, \\*
	כאשר $f(x) < 0$ אז $f_+(x) = 0$ אבל $f_-(x) = -f(x)$ ולכן $f(x) = 0 - (-f(x))$. \\*
	מצאנו כי השוויון מתקיים לכל $x \in [a, b]$.
\end{proof}

\subsubsection{iv.}
נוכיח כי $|f| = f_+ + f_-$.
\begin{proof}
	הפעם אנחנו נהיה יצירתיים במיוחד, ונחלק למקרים. \\*
	אם $f(x) = 0$ אז מהשוויון בהוכחת הסעיף הקודם נקבל כי השוויון מתקיים. \\*
	אם $f(x) > 0$ אז $f_-(x) = 0$ ונקבל $|f(x)| = f(x) = f_+(x)$ והשייון מתקיים. \\*
	עבור $f(x) < 0$ נקבל $f_+(x) = 0$ וגם $f_-(x) = -f(x) \ge 0$ ולכן $|f(x)| = 0 - f(x)$. \\*
	מצאנו כי השוויון אף הוא מתקיים לכל $x \in [a, b]$.
\end{proof}

\Subquestion{}
תהי חלוקה $P = \{a = x_0 < \cdots < x_n = b \}$ ונסמן
\[
	m_i = \inf_{x_{i - 1} \le x \le x_i} \{ f(x) \},
	M_i = \sup_{x_{i - 1} \le x \le x_i} \{ f(x) \},
	m_i^+ = \inf_{x_{i - 1} \le x \le x_i} \{ f_+(x) \},
	M_i^+ = \sup_{x_{i - 1} \le x \le x_i} \{ f_+(x) \}
\]
ונוכיח כי $M_i^+ - m_i^+ \le M_i - m_i$.
\begin{proof}
	אנו יודעים כי כאשר $f(x) \ge 0$ אז $f(x) = f_+(x)$ ולכן נוכל להסיק כי אם $M_i \ge 0$ אז נוכל לבחור את אותו ה־$x$ ולקבל גם $M_i^+ = M_i$.
	באופן דומה אם $m_i \ge 0$ אז $m_i^+ = m_i$ ונקבל כי $M_i^+ - m_i^+ = M_i - m_i$ ובפרט אי־השוויון מתקיים. \\*
	אם $M_i \le 0$ אז נוכל להסיק $M_i^+ = 0$ מהגדרתו, וכך גם $m_i \le M_i \le 0$ יוביל ל־$m_i^+ = 0$ וידוע כי $0 \le M_i - m_i$ ונקבל כי אי־השוויון חל. \\*
	המקרה האחרון הוא כאשר $M_i > 0$ אבל $m_i \le 0$, במקרה זה מצאנו כי $M_i^+ = M_i, m_i^+ = 0$ ונקבל $M_i - 0 \le M_i - m_i$. \\*
	אי־השוויון מתקיים בכל מקרה ולכן נכון לכל החלוקה. \\*
	אנו יודעים כי $f$ אינטגרבילית, ולכן נקבל מאי־השוויון $0 \le M_i^+ - m_i^+ \le M_i - m_i$ כי גם $f^+$ אינטגרבילית ב־$[a, b]$.
\end{proof}

\Subquestion{}
נוכיח כי גם $f_-, |f|$ אינטגרביליות ב־$[a, b]$.
\begin{proof}
	מצאנו כי $f_- = f_+ - f$ בסעיף א' ובסעיף ב' מצאנו כי $f, f_+$ הן שתיהן אינטגרביליות, ולכן נסיק מלינאריות של האינטגרל כי גם $f_-$ אינטגרבילית ב־$[a, b]$. \\*
	עתה כשמצאנו כי $f_-, f_+$ אינטגרביליות, נקבל כי גם חיבורן --- הוא $|f|$ --- היא אינטגרבילית ב־$[a, b]$.
\end{proof}

\Subquestion{}
נוכיח את אי־שוויון משולש האינטגרלי
\[
	\left\lvert \int_{a}^{b} f(x) dx \right\rvert
	\le \int_{a}^{b} |f(x)| dx
\]
\begin{proof}
	מצאנו כי $f(x) \le f_+(x) \le |f(x)|$, ולכן גם
	\[
		\int_{a}^{b} f(x) dx \le \int_{a}^{b} |f(x)| dx
	\]
	אנו יודעים כי הביטוי הימני תמיד חיובי, בעוד $f(x) = f_+(x) - f_-(x)$ ולכן נוכל להסיק גם
	\[
		\left\lvert \int_{a}^{b} f(x) dx \right\rvert \le \int_{a}^{b} |f(x)| dx
	\]
\end{proof}

\Question{}
תהי $f : [a, b] \to \RR$ פונקציה אינטגרבילית ב־$[a, b]$ ויהי $c > 0$ קבוע ממשי. \\*
נגדיר פונקציה $h : [a - c, b - c] \to \RR$ על־ידי $h(x) = f(x + c)$ לכל $x \in [a - c, b - c]$.

\Subquestion{}
לכל חלוקה $P = \{ x_0 < \cdots < x_n \}$ של $[a, b]$ נתאים חלוקה $P - c = \{ x_0 - c < \cdots < x_n - c \}$ של $[a - c, b - c]$ ונראה כי $L(f, P) = L(h, P - c)$ וגם כי $U(f, P) = U(h, P - c)$.
\begin{proof}
	ניעזר בהזזה של הפונקציה ומהעובדה ש־$h(x) = f(x + c)$ ונקבל כי $M_i$ משותפת לשתי החלוקות, שכן עבור $[x_{i + 1} - c, x_i - c]$ נקבל $h(x) = f(x + c)$, ואם נניח ש־$M_i$ חסם עליון לערכי הפונקציה בחלוקה,
	אז זה גם החסם העליו עבור $h$ בקטע המוזז. נקבל גם כי $x_{i + 1} - c - x_i + c = x_{i + 1} - x_i$ ולכן $U(f, P) = U(h, P - c)$. \\*
	נוכל לבצע תהליך זהה ונקבל כי גם $L(f, P) = L(h, P - c)$.
\end{proof}

\Subquestion{}
נסיק כי $h$ אינטגרבילית בקטע $[a - c, b - c]$ וכי מתקיים
\[
	\int_{a - c}^{b - c} h(x)\ dx
	= \int_{a - c}^{b - c} f(x + c)\ dx
	= \int_{a}^{b} f(x)\ dx
\]
\begin{proof}
	מצאנו בסעיף הקודם כי הסכומים העליונים והתחתונים של $h$ ושל $f$ לאחר הזזה מתאימה הם שווים, ולכן נקבל כי $h$ אינטגרבילית אם ורק אם $f$ אינטגרבילית בהתאמה להזזה. \\*
	נסיק אם כן ישירות כי
	\[
		\int_{a - c}^{b - c} h(x)\ dx
		= \int_{a - c}^{b - c} f(x + c)\ dx
		= \int_{a}^{b} f(x)\ dx
	\]
\end{proof}

\Question{}
תהי $f : [a, b] \to \RR$ פונקציה אינטגרבילית בקטע $[a, b]$ ויהי $m \ge 0$ קבוע ממשי. \\*
נגדיר פונקציה $h : [a/m, b/m] \to \RR$ על־ידי $h(x) = f(mx)$ לכל $x \in [a/m, b/m]$.

\Subquestion{}
נתאים לכל חלוקה $P = \{x_0 < \cdots < x_n\}$ של $[a, b]$ את החלוקה $P/m = \{x_0/m < \cdots < x_n/m\}$ של $[a/m, b/m]$ ונראה כי $L(h, P/m) = \frac{1}{m} L(f, P)$ וכי $U(h, P/m) = \frac{1}{m} U(f, P)$.
\begin{proof}
	תהי $0 \le i < n$ ונבחן את $M_i = \sup_{x_i < x < x_{i + 1}} f(x)$, ידוע כי $h(x) = f(mx)$ ולכן עבור $x_M$ המקיים $f(x_M) = M_i$ נקבל גם $h(\frac{x_M}{m}) = f(x_M) = M_i$. 
	מצאנו כי הפונקציות, עבור החלוקות השקולות, חולקות חסמיםם מקסימליים, ונוכל להראות באותה הדרך כי גם מינימליים. \\*
	נבחן עתה את $U(h, P/m)$:
	\[
		U(h, P/m)
		= \frac{1}{m} \sum_{i = 0}^{n - 1} M_i (x_{i + 1} - x_i)
		= \frac{1}{m} U(f, P)
	\]
	ומתהליך דומה נקבל גם כי $L(h, P/m) = \frac{1}{m} L(f, P)$.
\end{proof}

\Subquestion{}
נוכיח כי $h$ אינטגרבילית ב־$[a/m, b/m]$ ושמתקיים
\[
	\int_{a/m}^{b/m} h(x)\ dx
	= \int_{a/m}^{b/m} f(mx)\ dx
	= \frac{1}{m} \int_{a}^{b} f(x)\ dx
\]
\begin{proof}
	מצאנו כי $U(h, P/m) - L(h, P/m) = \frac{1}{m}( U(f, P) - L(f, P))$ ומאינטגרביליות $f$ ב־$[a, b]$ נסיק את השוויון ישירות.
\end{proof}

\Subquestion{}
נוכיח כי השוויון הנתון נכון גם עבור $m < 0$.
\begin{proof}
	נגדיר $g(x) = h(-x)$ ולכן $g$ מייצגת את $h$ עבור $m < 0$, בתרגול מצאנו כי
	\[
		\int_{-b}^{-a} g(x)\ dx = \int_{a}^{b} h(x)\ dx = \int_{a/m}^{b/m} f(mx)\ dx
	\]
	ולמעשה מצאנו כי השוויון חל גם עבור בחירת $m < 0$.
\end{proof}

\Question{}
\Subquestion{}
יהי $0 < b \in \RR$ ותהי $f : [-b, b] \to \RR$ פונקציה אי־זוגית. \\*
נוכיח שאם $f$ אינטגרבילית ב־$[-b, b]$ אז גם $\int_{-b}^{b} f(x)\ dx = 0$.
\begin{proof}
	נשתמש בתכונת האדיטיביות ונקבל
	\[
		\int_{-b}^{b} f(x)\ dx
		= \int_{-b}^{0} f(x)\ dx + \int_{0}^{b} f(x)\ dx
	\]
	ומשיקוף על ציר ה־$y$ שהוכחנו בתרגול נקבל
	\[
		\int_{-b}^{0} f(x)\ dx + \int_{0}^{b} f(x)\ dx
		= \int_{0}^{b} f(-x)\ dx + \int_{0}^{b} f(x)\ dx
		= \int_{0}^{b} -f(x)\ dx + \int_{0}^{b} f(x)\ dx
		= \int_{0}^{b} -f(x) f(x)\ dx
		= 0
	\]
\end{proof}

\Subquestion{}
יהיו $n, m \in \ZZ$, ונוכיח כי
\[
	\int_{-\pi}^{\pi} \cos(mx) \sin(nx) dx = 0
\]
\begin{proof}
	נראה כי $\cos(mx) \sin(nx)$ היא אי־זוגית:
	\[
		\cos(-mx) \sin(-nx) = \cos(mx) (-1) \sin(nx) = -\cos(mx)\sin(nx)
	\]
	ולכן מהסעיף הקודם נקבל ישירות כי
	\[
		\int_{-\pi}^{\pi} \cos(mx) \sin(nx) dx = 0
	\]
\end{proof}

\Question{}
תהי $f : \RR \to \RR$ פונקציה מחזורית עם מחזור $T > 0$, ו־$f$ אינטגרבילית ב־$[0, T]$. \\*
נוכיח כי לכל $a \in \RR$ $f$ אינטגרבילית בקטע $[a, a + T]$ וכי מתקיים
\[
	\int_{a}^{a + T} f(x)\ dx = \int_{0}^{T} f(x)\ dx
\]
\begin{proof}
	בסעיף ב' של שאלה 3 קיבלנו כי אינטגרלים נשמרים תחת הזזות, ולכן נקבל עבור ההזזה $h(x) = f(x - a)$ כי
	\[
		\int_{a}^{a + T} h(x) = \int_{a}^{a + T} f(x - a) = \int_{0}^{T} f(x)\ dx
	\]
	נבחין כי מהמחזוריות של $f$ נקבל $h(x) = h(x + T)$ וכן גם $f(a) = f(a + T)$ ואנו רואים כי בחירת $a$ לא משפיעה על התחום שיכלול יותר או פחות ממחזור יחיד, לכן
	\[
		\int_{a}^{a + T} f(x) = \int_{0}^{T} f(x)\ dx
	\]
\end{proof}

\Question{}
נחשב את פולינומי טיילור מסדר 5 סביב 0 של $f(x) = e^x \sin x$ ו־$g(x) = e^{\sin x}$.

אנו יודעים כבר כי
\[
	P_{5, \sin, 0}(x) = x - \frac{1}{3!} x^3 + \frac{1}{5!} x^5
\]
וגם כי
\[
	P_{5, \exp, 0} = 1 + x + \frac{x^2}{2} + \frac{x^3}{3!} + \frac{x^4}{4!} + \frac{x^5}{5!}
\]

מצאנו בתרגול כי
\[
	P_{n, g \cdot f, a} = {[P_{n, g, a} \cdot P_{n, f, a}]}_{n, a}
\]
ולכן נקבל כי
\[
	P_{5, \exp \cdot \sin, 0}
	= {[P_{5, \exp, 0} \cdot P_{5, \sin, 0}]}_{5, 0}
	= x - \frac{1}{3!} x^3 + \frac{1}{5!} x^5 + x^2 - \frac{1}{3!} x^4 + \frac{1}{3!} (x^3 - \frac{1}{3!} x^5) + \frac{1}{4!} x^4 + \frac{1}{5!} x^5
\]

בתרגול מצאנו כי לכל $f, g$ גזירות $n$ פעמים מתקיים
\[
	P_{n, g \circ f, a}(x) = {[P_{n, g, f(a)}(P_{n, f, a}(x))]}_{n, a}
\]
ונקבל מהנוסחה כי
\begin{align*}
	& P_{5, \exp \circ \sin, 0} = {[P_{5, \exp, 0}(P_{5, \sin, 0}(x))]}_{5, 0} \\
	= & [x - \frac{1}{3!} x^3 + \frac{1}{5!} x^5
	+ x^2 - \frac{1}{3!} x^4 + \frac{1}{5!} x^6
	+ \frac{1}{{(2!)}^2} (x^3 - \frac{1}{3!} x^5 + \frac{1}{5!} x^7) \\
	& + \frac{1}{{(3!)}^2} (x^4 - \frac{1}{3!} x^6 + \frac{1}{5!} x^8)
	+ \frac{1}{{(4!)}^2} (x^5 - \frac{1}{3!} x^7 + \frac{1}{5!} x^9)
	+ \frac{1}{{(5!)}^2} (x^6 - \frac{1}{3!} x^8 + \frac{1}{5!} x^{10})]{}_{5, 0} \\
	= & x - \frac{1}{3!} x^3 + \frac{1}{5!} x^5
	+ x^2 - \frac{1}{3!} x^4
	+ \frac{1}{{(2!)}^2} (x^3 - \frac{1}{3!} x^5)
	+ \frac{1}{{(3!)}^2} (x^4)
	+ \frac{1}{{(4!)}^2} (x^5)
\end{align*}

\end{document}
