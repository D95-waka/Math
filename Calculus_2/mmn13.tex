\documentclass[a4paper]{article}

% packages
\usepackage{inputenc, fontspec, amsmath, amsthm, amsfonts, polyglossia, catchfile}
\usepackage[a4paper, margin=50pt, includeheadfoot]{geometry} % set page margins

% style
\AddToHook{cmd/section/before}{\clearpage}	% Add line break before section
\linespread{1.5}
\setcounter{secnumdepth}{0}		% Remove default number tags from sections
\setmainfont{Libertinus Serif}
\setsansfont{Libertinus Sans}
\setmonofont{Libertinus Mono}
\setdefaultlanguage{hebrew}
\setotherlanguage{english}

% operators
\DeclareMathOperator\cis{cis}
\DeclareMathOperator\Sp{Sp}
\DeclareMathOperator\tr{tr}
\DeclareMathOperator\im{Im}
\DeclareMathOperator\diag{diag}
\DeclareMathOperator*\lowlim{\underline{lim}}
\DeclareMathOperator*\uplim{\overline{lim}}

% commands
\renewcommand\qedsymbol{\textbf{משל}}
\newcommand{\NN}[0]{\mathbb{N}}
\newcommand{\ZZ}[0]{\mathbb{Z}}
\newcommand{\QQ}[0]{\mathbb{Q}}
\newcommand{\RR}[0]{\mathbb{R}}
\newcommand{\CC}[0]{\mathbb{C}}
\newcommand{\getenv}[2][] {
  \CatchFileEdef{\temp}{"|kpsewhich --var-value #2"}{\endlinechar=-1}
  \if\relax\detokenize{#1}\relax\temp\else\let#1\temp\fi
}
\newcommand{\explain}[2] {
	\begin{flalign*}
		 && \text{#2} && \text{#1}
	\end{flalign*}
}

% headers
\getenv[\AUTHOR]{AUTHOR}
\author{\AUTHOR}
\date\today

\title{פתרון ממ''ן 12 – חשבון אינפיניטסימלי 2 (20475)}

\begin{document}
\maketitle
\maketitleprint{}
\section{שאלה 1}
את הפונקציה $f(x) = \ln x$ מקרבים בקטע $I = [e^2 - 1, e^2 + 1]$ על־ידי הפולינום
\[
	P(x) = \frac{1}{2} + \frac{2x}{e^2} - \frac{x^2}{2e^4}
\]
נראה כי
\[
	\lvert f(x) - P(x) \rvert < \frac{1}{3{(e^2 - 1)}^3}
\]
לכל $x \in I$.
\begin{flalign*}
	\ln x
	= \ln(e^2 e^{-2} x)
	= \ln(e^{-2} x) + 2
	= \ln((e^{-2} x - 1) + 1) + 2 && \text{נראה כי} & \\
	t = \frac{x}{e^2} - 1 && \text{לכן נגדיר} & \\
	\ln x = g(t) = \ln(t + 1) + 2 && \text{אז מההגדרה נובע} & \\
\end{flalign*}
ידוע כי $x \in [e^2 - 1, e^2 + 1]$ ולכן
\begin{align*}
	& e^2 - 1 \le x \le e^2 + 1 \\
	& e^2 - 1 \le e^2 (t + 1) \le e^2 + 1 \\
	& 1 - e^{-2} \le t + 1 \le 1 + e^{-2} \\
	& -1 < - e^{-2} \le t \le e^{-2} < 1
\end{align*}
דהינו $t \in (-1, 1)$ ולכן על־פי פיתוח טיילור של $\ln(t + 1)$ בתחום $(-1, 1)$ אשר מוגדר בעמוד 65 כרך ב' פולינום טיילור מסדר 2 של $g(t)$ הוא
\[
	P_2(t) = g(0) + t - \frac{1}{2} t^2
	= 2 + \frac{x}{e^2} - 1 - \frac{1}{2}{(\frac{x}{e^2} - 1)}^2
	= 1 + \frac{x}{e^2} - \frac{1}{2}(\frac{x^2}{e^4} - \frac{2x}{e^2} + 1)
	= \frac{1}{2} + \frac{2x}{e^2} - \frac{x^2}{2e^4} = P(x)
\]
לכן על־פי הגדרת השארית
\[
	R_2(t) = g(t) - P_2(t)
\]
על־פי דוגמה 4.4 לכל $t \in (-1, 1)$ מתקיים
\[
	\lvert R_2(t) \rvert < \frac{{\lvert t \rvert}^3}{1 - |t|}
	= \frac{{( \frac{x}{e^2} - 1 )}^3}{\frac{x}{e^2}}
	= \frac{{( x - e^2 )}^3}{x e^4}
\]
מחקירת הפונקציה עולה כי היא מקבלת מקסימום ב־$x = e^2 + 1$ ולכן
\[
	\lvert R_2(t) \rvert = \lvert f(x) - P(x) \rvert
	< \frac{{(e^2 + 1 - e^2)}^3}{(e^2 + 1) e^4}
	= \frac{1}{e^6 + e^4}
\]
ניתן לבדוק ולראות כי מתקיים לכל $x \in [e^2 - 1, e^2 + 1]$
\[
	\lvert f(x) - P(x) \rvert < \frac{1}{3{(e^2 - 1)}^3}
\]

\section{שאלה 2}
תהי $f(x)$ פונקציה גזירה $n + 1$ פעמים בקטע $[a, b]$ ו־$f^{(n + 1)}(x)$ רציפה ב־$[a, b]$. \\*
נקבע נקודה $x_0 \in [a, b]$ ונסמן ב־$R_n(x)$ את השארית מסדר $n$ של $f$ ב־$x_0$. \\*
נוכיח כי לכל $x \in [a, b]$
\[
	R_n(x) = \frac{1}{n!} \int_{x_0}^x f^{(n + 1)}(t){(x - t)}^n \, dt
\]
\begin{proof}
	% https://www2.math.upenn.edu/~kazdan/508F14/Notes/Taylor-integral.pdf
	% I coppied the proof from here
	נגדיר $P_i(x)$ פיתוח טיילור של $f(x)$ סביב $x_0$. נשים לב כי הפונקציה $f(x)$ עומדת בכל הדרישות לפיתוח זה עבור $0 \le i \le n$. \\*
	 נוכיח באינדוקציה את הטענה: \\*
	 \textbf{בסיס האינדוקציה:}
	 מהמשפט היסודי של החשבון האינפיניטסמלי נובע כי
	 \[
		 \int_{x_0}^x f'(t) dt = f(x) - f(x_0) = f(x) - P_0(x) = R_0(x)
	 \]
	 ומצאנו כי הטענה נכונה עבור $n = 0$. \\*
	 \textbf{מהלך האינדוקציה:}
	 נניח כי הטענה נכונה עבור $0 < i \le n$, לכן מתקיים
	 \[
		  f(x) = P_i(x) + \frac{1}{i!} \int_{x_0}^x f^{(i + 1)}(t) {(x - t)}^i dt
	 \]
	 עבור הביטוי נבצע אינטגרציה בחלקים, כאשר
	 \begin{align*}
		 & u = f^{(i + 1)}(t) & dv & = {(x - t)}^i \\
		 & du = f^{(i + 2)}(t) dt & v & = -\frac{1}{i + 1} {(x - t)}^{i + 1}
	 \end{align*}
	ולכן
	 \begin{align*}
		 A & = \int_{x_0}^x u(t) v(t) dt \\
		   & = \left. -\frac{1}{i + 1} f^{(i + 1)}(t){(x - t)}^{(i + 1)} \right|_{x_0}^x \\
		   & = -\frac{1}{i + 1} f^{(i + 1)}(x){(x - x)}^{(i + 1)} + \frac{1}{i + 1} f^{(i + 1)}(x_0){(x - x_0)}^{(i + 1)} \\
		   & = \frac{1}{i + 1} f^{(i + 1)}(x_0){(x - x_0)}^{(i + 1)} \\
		 f(x) & = P_i(x) + \frac{1}{i!} \left( A - \frac{1}{i + 1} \int_{x_0}^x (-1) f^{(i + 2)}(t) {(x - t)}^{(i + 1)} dt \right) \\
		 & = P_i(x) + \frac{1}{(i + 1)!} \left( f^{(i + 1)}(x_0){(x - x_0)}^{(i + 1)} + \int_{x_0}^x f^{(i + 2)}(t) {(x - t)}^{(i + 1)} dt \right) \\
		 & = P_{i + 1}(x) + \frac{1}{(i + 1)!} \int_{x_0}^x f^{(i + 2)}(t) {(x - t)}^{(i + 1)} dt \\
	 \end{align*}
	 והשלמנו את מהלך האינדוקציה.
\end{proof}

\end{document}
