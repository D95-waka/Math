\documentclass[a4paper]{article}

% packages
\usepackage{inputenc, amsmath, amsthm, thmtools, amsfonts, amssymb, luacode, catchfile, tikzducks, hyperref}
\usepackage[a4paper, margin=50pt, includeheadfoot]{geometry} % set page margins
\usepackage[shortlabels]{enumitem}
\usepackage[skip=3pt, indent=0pt]{parskip}

% language
\usepackage[bidi=basic, layout=tabular, provide=*]{babel}
\babelprovide[main, import]{hebrew}
\babelprovide{rl}
\babelfont{rm}{Libertinus Serif}
\babelfont{sf}{Libertinus Sans}
\babelfont{tt}{Libertinus Mono}

% style
\AddToHook{cmd/section/before}{\clearpage}	% Add line break before section
\linespread{1.3}
\setcounter{secnumdepth}{0}		% Remove default number tags from sections, this won't do well with theorems
\AtBeginDocument{\setlength{\belowdisplayskip}{3pt}}
\AtBeginDocument{\setlength{\abovedisplayskip}{3pt}}

% operators
\DeclareMathOperator\cis{cis}
\DeclareMathOperator\Sp{Sp}
\DeclareMathOperator\tr{tr}
\DeclareMathOperator\im{Im}
\DeclareMathOperator\re{Re}
\DeclareMathOperator\diag{diag}
\DeclareMathOperator*\lowlim{\underline{lim}}
\DeclareMathOperator*\uplim{\overline{lim}}
\DeclareMathOperator\rng{rng}
\DeclareMathOperator\Sym{Sym}
\DeclareMathOperator\Arg{Arg}
\DeclareMathOperator\Log{Log}
\DeclareMathOperator\dom{dom}

% commands
%\renewcommand\qedsymbol{\textbf{מש''ל}}
%\renewcommand\qedsymbol{\fbox{\emoji{lizard}}}
\newcommand{\NN}[0]{\mathbb{N}}
\newcommand{\ZZ}[0]{\mathbb{Z}}
\newcommand{\QQ}[0]{\mathbb{Q}}
\newcommand{\RR}[0]{\mathbb{R}}
\newcommand{\CC}[0]{\mathbb{C}}
\newcommand{\FF}[0]{\mathbb{F}}
\newcommand{\PP}[0]{\mathbb{P}}
\newcommand{\TT}[0]{\mathbb{T}}
\newcommand{\acts}[0]{\circlearrowright}
\newcommand{\explain}[2] {
	\begin{flalign*}
		 && \text{#2} && \text{#1}
	\end{flalign*}
}
\newcommand{\maketitleprint}[0]{ \begin{center}
	\begin{tikzpicture}[scale=3]
		\duck[graduate=gray!20!black, tassel=red!70!black]
	\end{tikzpicture}	
\end{center}
}

% theorem commands
\newtheoremstyle{c_remark}
	{}	% Space above
	{}	% Space below
	{}% Body font
	{}	% Indent amount
	{\bfseries}	% Theorem head font
	{}	% Punctuation after theorem head
	{.5em}	% Space after theorem head
	{\thmname{#1}\thmnumber{ #2}\thmnote{ \normalfont{\text{(#3)}}}}	% head content
\newtheoremstyle{c_definition}
	{3pt}	% Space above
	{3pt}	% Space below
	{}% Body font
	{}	% Indent amount
	{\bfseries}	% Theorem head font
	{}	% Punctuation after theorem head
	{.5em}	% Space after theorem head
	{\thmname{#1}\thmnumber{ #2}\thmnote{ \normalfont{\text{(#3)}}}}	% head content
\newtheoremstyle{c_plain}
	{3pt}	% Space above
	{3pt}	% Space below
	{\itshape}% Body font
	{}	% Indent amount
	{\bfseries}	% Theorem head font
	{}	% Punctuation after theorem head
	{.5em}	% Space after theorem head
	{\thmname{#1}\thmnumber{ #2}\thmnote{ \text{(#3)}}}	% head content

\theoremstyle{c_plain}
\newtheorem{theorem}{משפט}[section]
\newtheorem{lemma}[theorem]{למה}
\newtheorem{proposition}[theorem]{טענה}
\newtheorem*{proposition*}{טענה}
%\newtheorem{corollary}[theorem]{אין חלופה עברית}

\theoremstyle{c_definition}
\newtheorem{definition}[theorem]{הגדרה}
\newtheorem*{definition*}{הגדרה}
\newtheorem{example}{דוגמה}[section]
\newtheorem{exercise}{תרגיל}[section]

\theoremstyle{c_remark}
\newtheorem*{remark}{הערה}
\newtheorem*{solution}{פתרון}
\newtheorem{conclusion}[theorem]{מסקנה}
\newtheorem{notation}[theorem]{סימון}

% Questions related commands
\newcounter{question}
\setcounter{question}{1}
\newcounter{sub_question}
\setcounter{sub_question}{1}

\newcommand{\question}[1][0]{
	\ifthenelse{#1 = 0}{}{\setcounter{question}{#1}}
	\subsection{שאלה \arabic{question}}
	\addtocounter{question}{1}
	\setcounter{sub_question}{1}
}

\newcommand{\subquestion}[1][0]{
	\ifthenelse{#1 = 0}{}{\setcounter{sub_question}{#1}}
	\subsubsection{סעיף \localecounter{letters.gershayim}{sub_question}}
	\addtocounter{sub_question}{1}
}

% import lua and start of document
\directlua{common = require ('../common')}

\GetEnv{AUTHOR}

% headers
\author{\AUTHOR}
\date\today

\title{פתרון ממ''ן 13 – חשבון אינפיניטסימלי 2 (20475)}

\begin{document}
\maketitle
\maketitleprint{}
\section{שאלה 1}
את הפונקציה $f(x) = \ln x$ מקרבים בקטע $I = [e^2 - 1, e^2 + 1]$ על־ידי הפולינום
\[
	P(x) = \frac{1}{2} + \frac{2x}{e^2} - \frac{x^2}{2e^4}
\]
נראה כי
\[
	\lvert f(x) - P(x) \rvert < \frac{1}{3{(e^2 - 1)}^3}
\]
לכל $x \in I$.
\begin{flalign*}
	\ln x
	= \ln(e^2 e^{-2} x)
	= \ln(e^{-2} x) + 2
	= \ln((e^{-2} x - 1) + 1) + 2 && \text{נראה כי} & \\
	t = \frac{x}{e^2} - 1 && \text{לכן נגדיר} & \\
	\ln x = g(t) = \ln(t + 1) + 2 && \text{אז מההגדרה נובע} & \\
\end{flalign*}
ידוע כי $x \in [e^2 - 1, e^2 + 1]$ ולכן
\begin{align*}
	& e^2 - 1 \le x \le e^2 + 1 \\
	& e^2 - 1 \le e^2 (t + 1) \le e^2 + 1 \\
	& 1 - e^{-2} \le t + 1 \le 1 + e^{-2} \\
	& -1 < - e^{-2} \le t \le e^{-2} < 1
\end{align*}
דהינו $t \in (-1, 1)$ ולכן על־פי פיתוח טיילור של $\ln(t + 1)$ בתחום $(-1, 1)$ אשר מוגדר בעמוד 65 כרך ב' פולינום טיילור מסדר 2 של $g(t)$ הוא
\[
	P_2(t) = g(0) + t - \frac{1}{2} t^2
	= 2 + \frac{x}{e^2} - 1 - \frac{1}{2}{(\frac{x}{e^2} - 1)}^2
	= 1 + \frac{x}{e^2} - \frac{1}{2}(\frac{x^2}{e^4} - \frac{2x}{e^2} + 1)
	= \frac{1}{2} + \frac{2x}{e^2} - \frac{x^2}{2e^4} = P(x)
\]
לכן על־פי הגדרת השארית
\[
	R_2(t) = g(t) - P_2(t)
\]
על־פי דוגמה 4.4 לכל $t \in (-1, 1)$ מתקיים
\[
	\lvert R_2(t) \rvert < \frac{{\lvert t \rvert}^3}{1 - |t|}
	= \frac{{( \frac{x}{e^2} - 1 )}^3}{\frac{x}{e^2}}
	= \frac{{( x - e^2 )}^3}{x e^4}
\]
מחקירת הפונקציה עולה כי היא מקבלת מקסימום ב־$x = e^2 + 1$ ולכן
\[
	\lvert R_2(t) \rvert = \lvert f(x) - P(x) \rvert
	< \frac{{(e^2 + 1 - e^2)}^3}{(e^2 + 1) e^4}
	= \frac{1}{e^6 + e^4}
\]
ניתן לבדוק ולראות כי מתקיים לכל $x \in [e^2 - 1, e^2 + 1]$
\[
	\lvert f(x) - P(x) \rvert < \frac{1}{3{(e^2 - 1)}^3}
\]

\section{שאלה 2}
תהי $f(x)$ פונקציה גזירה $n + 1$ פעמים בקטע $[a, b]$ ו־$f^{(n + 1)}(x)$ רציפה ב־$[a, b]$. \\*
נקבע נקודה $x_0 \in [a, b]$ ונסמן ב־$R_n(x)$ את השארית מסדר $n$ של $f$ ב־$x_0$. \\*
נוכיח כי לכל $x \in [a, b]$
\[
	R_n(x) = \frac{1}{n!} \int_{x_0}^x f^{(n + 1)}(t){(x - t)}^n \, dt
\]
\begin{proof}
	% https://www2.math.upenn.edu/~kazdan/508F14/Notes/Taylor-integral.pdf
	% I coppied the proof from here
	נגדיר $P_i(x)$ פיתוח טיילור של $f(x)$ סביב $x_0$. נשים לב כי הפונקציה $f(x)$ עומדת בכל הדרישות לפיתוח זה עבור $0 \le i \le n$. \\*
	 נוכיח באינדוקציה את הטענה: \\*
	 \textbf{בסיס האינדוקציה:}
	 מהמשפט היסודי של החשבון האינפיניטסמלי נובע כי
	 \[
		 \int_{x_0}^x f'(t) dt = f(x) - f(x_0) = f(x) - P_0(x) = R_0(x)
	 \]
	 ומצאנו כי הטענה נכונה עבור $n = 0$. \\*
	 \textbf{מהלך האינדוקציה:}
	 נניח כי הטענה נכונה עבור $0 < i \le n$, לכן מתקיים
	 \[
		  f(x) = P_i(x) + \frac{1}{i!} \int_{x_0}^x f^{(i + 1)}(t) {(x - t)}^i dt
	 \]
	 עבור הביטוי נבצע אינטגרציה בחלקים, כאשר
	 \begin{align*}
		 & u = f^{(i + 1)}(t) & dv & = {(x - t)}^i \\
		 & du = f^{(i + 2)}(t) dt & v & = -\frac{1}{i + 1} {(x - t)}^{i + 1}
	 \end{align*}
	ולכן
	 \begin{align*}
		 A & = u(t) v(t) dt \\
		   & = \left. -\frac{1}{i + 1} f^{(i + 1)}(t){(x - t)}^{(i + 1)} \right|_{x_0}^x \\
		   & = -\frac{1}{i + 1} f^{(i + 1)}(x){(x - x)}^{(i + 1)} + \frac{1}{i + 1} f^{(i + 1)}(x_0){(x - x_0)}^{(i + 1)} \\
		   & = \frac{1}{i + 1} f^{(i + 1)}(x_0){(x - x_0)}^{(i + 1)} \\
		 f(x) & = P_i(x) + \frac{1}{i!} \left( A - \frac{1}{i + 1} \int_{x_0}^x (-1) f^{(i + 2)}(t) {(x - t)}^{(i + 1)} dt \right) \\
		 & = P_i(x) + \frac{1}{(i + 1)!} \left( f^{(i + 1)}(x_0){(x - x_0)}^{(i + 1)} + \int_{x_0}^x f^{(i + 2)}(t) {(x - t)}^{(i + 1)} dt \right) \\
		 & = P_{i + 1}(x) + \frac{1}{(i + 1)!} \int_{x_0}^x f^{(i + 2)}(t) {(x - t)}^{(i + 1)} dt \\
	 \end{align*}
	 והשלמנו את מהלך האינדוקציה.
\end{proof}

\section{שאלה 3}
נשתמש בפיתוח מקלורן ונחשב את הגבולות הבאים:

\subsection{סעיף א'}
\[
	\lim_{x \to 0} \frac{e^x \cos x - (x + 1)}{\tan x - \sin x} \tag{1}
\]
נגזור ונחשב פולינומים עבור חלקי הביטוי
\[
	f(x) = e^x \cos x, f'(x) = e^x \cos x - e^x \sin x, f''(x) = -2 e^x \sin x, f^{(3)}(x) = -2 e^x ( \sin x + \cos x)
\]
ונחשב
\[
	f(0) = 1, f'(1) = 1, f''(0) = 0, f^{(3)}(0) = -2
\]
ולכן ערך המכנה הוא
\[
	e^x \cos x - (x + 1) = 1 + x - 2 \frac{1}{3!} x^3 + R_3(x) - x - 1 = -\frac{1}{3} x^3 + R_3(x)
\]
נגזור את הביטוי
\begin{align*}
	& g(x) = \tan x - \sin x,
	g'(x) = \cos^{-2}(x) - \cos x, \\
	& g''(x) = 2\sin x \cos^{-3}(x) + \sin x,
	g^{(3)}(x) = 2( \cos^{-2}(x) + -3 \sin^2 x \cos^{-4}(x) ) + \cos x
\end{align*}
ונחשב
\[
	g(0) = 0, g'(0) = 0, g''(0) = 0, g^{(3)}(0) = 3
\]
ולכן מכנה הביטוי מקיים
\[
	\tan x - \sin x = \frac{1}{2}x^3 + S_3(x)
\]
ולכן על־פי משפט 4.7 גבול $(1)$ שקול לגבול
\[
	\lim_{x \to 0} \frac{-\frac{1}{3} x^3 + R_3(x)}{\frac{1}{2} x^3 + S_3(x)}
	= \lim_{x \to 0} \frac{-2 \frac{x^3}{x^3} + \frac{R_3(x)}{x^3}}{3 \frac{x^3}{x^3} + \frac{S_3(x)}{x^3}}
	= \lim_{x \to 0} \frac{-2 + 0}{3 + 0}
	= -\frac{2}{3}
\]

\subsection{סעיף ב'}
\[
	\lim_{n \to \infty} \frac{\ln(n^2 + 1) + \ln(n^2 - 1) - 4 \ln n}{1 - \cos(1/n^2)} \tag{2}
\]
% תעביר לפונקציה על־ידי הגדרת היינה
% בשלב השני תעביר לגבול לאפס עם אחד חלקי אן
% את המונה כבר פתרנו, את המכנה יקח שנייה אחרי ההיינה
נחשב
\[
	\ln(n^2 + 1) + \ln(n^2 - 1) - 4 \ln n
	= \ln(\frac{n^4 - 1}{n^4})
	= \ln(1 - \frac{1}{n^4})
\]
לכן ערך גבול $(2)$ שקול לערך הגבול
\[
	\lim_{n \to \infty} \frac{\ln(1 - \frac{1}{n^4})}{1 - \cos(\frac{1}{n^2})} \tag{3}
\]
על־פי הגדרת היינה לגבול פונקציה ערך גבול $(3)$ לערך גבול לפונקציה זהה. על הגבול בתצורת פונקציה נחיל את משפט הרכבת פונקציות מאינפי 1 עבור $f(x) = \frac{1}{x^2}$:
\[
	\lim_{x \to 0} \frac{\ln(1 - x^2)}{1 - \cos x}
\]
הוא גבול המתכנס לאותו ערך כמו גבול $(2)$. \\*
נחשב נגזרות עבור המונה
\[
	f(x) = \ln(1 - x^2),
	f'(x) = -2\frac{x}{1 - x^2},
	f''(x) = -2\frac{1 + x^2}{{(1 - x^2)}^2}
\]
ולכן
\[
	f(0) = 0, f'(0) = 0, f''(0) = -2
\]
נחשב את נגזרות המכנה
\[
	g(x) = 1 - \cos x,
	g'(x) = \sin x,
	g''(x) = \cos x
\]
ומחישוב עולה כי
\[
	g(0) = 0, g'(0) = 0, g''(0) = 1
\]
ונובע כי הגבול שקול על־פי משפט 4.7 לביטוי
\[
	\lim_{x \to 0} \frac{-\frac{2}{2!} x^2 + R_2(x)}{\frac{1}{2!}x^2 + S_2(x)}
	= \lim_{x \to 0} \frac{-1 + \frac{R_2(x)}{x^2}}{\frac{1}{2} + \frac{S_2(x)}{x^2}}
	= \frac{-1}{\frac{1}{2} }
	= -2
\]
אז גבול $(2)$ ערכו הוא $-2$.

\section{שאלה 4}
תהי $f(x)$ פונקציה אי־שלילית גזירה פעמיים בקטע $[-2, 2]$ וידוע שיש ל־$f(x)$ אפס בקטע זה. עוד ידוע כי $f(-2) = f(2) = \pi$. \\*
נוכיח כי קיימת נקודה $c \in (-2, 2)$ כך שמתקיים $f''(c) \ge \pi/2$
\begin{proof}
	ידוע כי קיימת נקודה $x_0 \in (-2, 2)$ עבורה $f(x_0) = 0$, וידוע כי זוהי נקודת מינימום של הפונקציה ולכן ממשפט פרמה נובע כי $f'(x_0) = 0$ גם כן. \\*
	לכן פיתוח טיילור של $f$ סביב $x_0$ הוא
	\[
		P(x) = 0 + 0(x - x_0) + R_1(x) = R_1(x)
	\]
	כאשר $R_1(x)$ פונקציית השארית. \\*
	נציג את השארית בצורת לגראנז' בנקודה $x = 2$ ונקבל
	\[
		P(2) = \pi = R_1(2) = \frac{1}{2} f''(\xi_1) {(2 - x_0)}^2
	\]
	ולכן
	\[
		f''(\xi_1) = \frac{2 \pi}{{(2 - x_0)}^2}
	\]
	באופן דומה נקבל עבור הצגת לאגרנז' בנקודה $x = -2$ כי
	\[
		f''(\xi_2) = \frac{2 \pi}{{(2 + x_0)}^2}
	\]
	נבחין כי על־פי הצגת לאגרנז' $\xi_1, \xi_2 \in (-2, 2)$. \\*
	ידוע כי $-2 < x_0 < 2$, נבחן את התחומים הבאים:
	\begin{enumerate}
		\item כאשר $-2 < x_0 < 0$ \\*
			מתקיים $0 < {(x_0 + 2)}^2 < 4$ ולכן גם
			\[
				f''(\xi_2) = \frac{2 \pi}{{(2 + x_0)}^2} > \frac{\pi}{2}
			\]
		\item כאשר $x_0 = 0$ \\*
			מתקיים
			\[
				f''(\xi_2) = \frac{2\pi}{{(2 + 0)}^2} = \frac{\pi}{2}
			\]
		\item כאשר $0 < x_0 < 2$ \\*
			מתקיים $0 < {(2 - x_0)}^2 < 4$ ולכן גם
			\[
				f''(\xi_1) = \frac{2 \pi}{{(2 - x_0)}^2} > \frac{\pi}{2}
			\]
	\end{enumerate}
	אז מצאנו כי קיים מספר $\xi \in (-2, 2)$ עבורו
	\[
		f''(\xi) \ge \frac{\pi}{2}
	\]
\end{proof}

\section{שאלה 5}
תהי $f(x)$ גזירה פעמיים בנקודה $x = 1$ ומתקיים
\[
	\lim_{x \to 1} \frac{ f(x) - x }{ {(x - 1)}^2 } = 1
\]
נמצא את הערכים של $f$ ושל שתי נגזרותיה הראשונות בנקודה $x = 1$. \\*
תחילה נבחין כי תנאי כלל לופיטל מתקיימים פעמיים, ולכן מתקיים
\[
	\lim_{x \to 1} \frac{ f(x) - x }{ {(x - 1)}^2 }
	= \lim_{x \to 1} \frac{ f'(x) - 1 }{ 2x - 2 }
	= \lim_{x \to 1} \frac{ f''(x) }{ 2 }
	= f''(1) / 2 = 1
\]
וקיבלנו $f''(1) = 2$. \\*
נבחן את הגבול
\[
	\lim_{x \to 1} \frac{ f'(x) - 1 }{ 2x - 2 } = 1
\]
הוא כמובן שקול על־פי פיתוח טיילור סביב $x = 1$
\[
	\lim_{x \to 1} \frac{ f'(1) + f''(1)(x - 1) + R_1(x) - 1 }{ 2x - 2 }
	= \lim_{x \to 1} \frac{ f'(1) + 2x - 3 + R_1(x) }{ 2x - 2 }
	= \lim_{x \to 1} \frac{ 2(x - 1) + f'(1) - 1 + R_1(x) }{ 2x - 2 }
\]
ממשפט 4.7 נובע כי $R_1(x)$ איננו משפיע על הגבול, אך אילו $f'(1) - 1 \ne 0$ אז המונה לא יתאפס בנקודה בסתירה לסופיות הגבול, ולכן נובע כי $f'(1) = 1$. \\*
לכן לפי פיתוח טיילור ב־$x = 1$ של הגבול:
\[
	\lim_{x \to 1} \frac{f(1) + (x - 1) + {(x - 1)}^2 - x + R_2(x)}{ {(x - 1)}^2 }
	= \lim_{x \to 1} \frac{f(1) - 1 + {(x - 1)}^2 + R_2(x)}{ {(x - 1)}^2 } = 1
\]
באופן דומה $R_2(x)$ מתאפס בגבול, אך אילו $f(1) - 1 \ne 0$ אז המונה איננו מתאפס בסתירה לסופיות הגבול ולכן $f(1) = 1$. \\*
מצאנו כי $f(1) = f'(1) = 1, f''(1) = 2$.

\section{שאלת רשות}
נמצא פולינום מקלורן ממעלה $3$ של הפונקציה
\[
	f(x) = {(1 + x)}^{1/x}, f(0) = \lim_{x \to 0} f(x) = e
\]
תחילה נשים לב כי
\[
	f(x) = e^{\ln f(x)} = e^{\frac{1}{x} \ln(1 + x)}
\]
על־פי כרך ב' של הספר אנו למדים כי
\[
	\ln(1 + x) = x - \frac{x^2}{2} + \frac{x^3}{3} - \frac{x^4}{4} + R_4(x)
\]
ולכן
\[
	f(x) = e^{\frac{1}{x} (x - \frac{x^2}{2} + \frac{x^3}{3} - \frac{x^4}{4} + R_4(x))}
	= e^{1 - \frac{x}{2} + \frac{x^2}{3} - \frac{x^3}{4} + \frac{R_4(x)}{x}}
	= e^{1 - \frac{x}{2} + \frac{x^2}{3} - \frac{x^3}{4} + R_3(x)}
	= e(e^\frac{-x}{2})(e^\frac{x^2}{3})(e^{\frac{-x^3}{4}}) + R_3(x)
\]
מפיתוח מקלורן עולה כי
\begin{align*}
	& e^{-x/2} = 1 - \frac{1}{2} x + \frac{1}{8}x^2 - \frac{1}{48}x^3 + R_3(x) \\
	& e^{x^2/3} = 1 + 0x + \frac{1}{3}x^2 + 0 x^3 + R_3(x) \\
	& e^{-x^3/4} = 1 + 0x + 0x^2 - \frac{1}{4}x^3 + R_3(x)
\end{align*}
ולכן
\begin{align*}
	e(e^\frac{-x}{2})(e^\frac{x^2}{3})(e^{\frac{-x^3}{4}}) + R_3(x)
	& = e( 1 - \frac{1}{2} x + \frac{1}{8} x^2 - \frac{1}{48}x^3 )( 1 + \frac{1}{3} x^2 )(1 - \frac{1}{4}x^3) + R_3(x) \\
	& = e( 1 - \frac{1}{2} x + \frac{1}{8} x^2 + \frac{1}{3} x^2 - \frac{1}{48}x^3 - \frac{1}{6}x^3 - \frac{1}{4}x^3 ) + R_3(x) \\
	& = e( 1 - \frac{1}{2} x + \frac{11}{24} x^2 - \frac{21}{48}x^3 ) + R_3(x) \\
\end{align*}
ולכן מצאנו כי
\[
	f(x) = e - \frac{e}{2} x + \frac{11e}{24} x^2 - \frac{7e}{16}x^3 + R_3(x)
\]

\end{document}
