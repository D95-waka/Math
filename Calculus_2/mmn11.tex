\documentclass[a4paper]{article}

% packages
\usepackage{inputenc, fontspec, amsmath, amsthm, amsfonts, polyglossia, catchfile}
\usepackage[a4paper, margin=50pt, includeheadfoot]{geometry} % set page margins

% style
\AddToHook{cmd/section/before}{\clearpage}	% Add line break before section
\linespread{1.5}
\setcounter{secnumdepth}{0}		% Remove default number tags from sections
\setmainfont{Libertinus Serif}
\setsansfont{Libertinus Sans}
\setmonofont{Libertinus Mono}
\setdefaultlanguage{hebrew}
\setotherlanguage{english}

% operators
\DeclareMathOperator\cis{cis}
\DeclareMathOperator\Sp{Sp}
\DeclareMathOperator\tr{tr}
\DeclareMathOperator\im{Im}
\DeclareMathOperator\diag{diag}
\DeclareMathOperator*\lowlim{\underline{lim}}
\DeclareMathOperator*\uplim{\overline{lim}}

% commands
\renewcommand\qedsymbol{\textbf{משל}}
\newcommand{\NN}[0]{\mathbb{N}}
\newcommand{\ZZ}[0]{\mathbb{Z}}
\newcommand{\QQ}[0]{\mathbb{Q}}
\newcommand{\RR}[0]{\mathbb{R}}
\newcommand{\CC}[0]{\mathbb{C}}
\newcommand{\getenv}[2][] {
  \CatchFileEdef{\temp}{"|kpsewhich --var-value #2"}{\endlinechar=-1}
  \if\relax\detokenize{#1}\relax\temp\else\let#1\temp\fi
}
\newcommand{\explain}[2] {
	\begin{flalign*}
		 && \text{#2} && \text{#1}
	\end{flalign*}
}

% headers
\getenv[\AUTHOR]{AUTHOR}
\author{\AUTHOR}
\date\today

\title{פתרון ממ''ן 11 – חשבון אינפיניטסימלי 2 (20475)}

\begin{document}
\maketitle
\section{שאלה 1}
\subsection{סעיף א'}
נוכיח כי
\[
	\frac{2}{3\pi} \le \int_{2\pi}^{3\pi} \frac{\sin x}{x} dx \le \frac{1}{\pi}
\]
\begin{proof}
	תחילה נגדיר
	\[
		f(x) = \frac{\sin x}{x}
	\]
	זוהי כמובן פונקציה רציפה ולכן אינטגרבילית.
	נגדיר גם $g(x) = \frac{x^2}{2\pi}$ ונראה כי מתקיים $g(x) \ge f(x)$ לכל $x \ge 2\pi$.
	ממשפט 1.26 נובע כי
	\[
		\int_{2\pi}^{3\pi} f(x) dx \le \int_{2\pi}^{3\pi} g(x) dx = \frac{x}{\pi} \Big|_{2\pi}^{3\pi} = \frac{1}{\pi}
	\]
	נגדיר גם
	\[
		h(x) = \frac{\sin x}{3 \pi}
	\]
	כמובן שבתחום האינטגרל המונה של $f$ ו־$h$ זהים, אך $x \le 3\pi$ בתחום, ולכן גם $h(x) \le f(x)$ לכל התחום, ומתקיים
	\[
		\int_{2\pi}^{3\pi} f(x) dx \ge \int_{2\pi}^{3\pi} h(x) dx = \frac{-\cos x}{3\pi} \Big|_{2\pi}^{3\pi} = \frac{2}{3\pi}
	\]
	אז מצאנו כי
	\[
		\frac{2}{3\pi} \le \int_{2\pi}^{3\pi} \frac{\sin x}{x} dx \le \frac{1}{\pi}
	\]
\end{proof}

\subsection{סעיף ב'}
תהי $f(x)$ פונקציה רציפה בקטע $[0, 2]$. לכל $n \in \NN$ נגדיר
\[
	a_n = \int_{1/n}^{2/n} f(x) dx
\]
נוכיח כי $\lim_{n \to \infty} n a_n = f(0)$
\begin{proof}
	נגדיר $F(x)$ הפונקציה הקדומה של $f(x)$, לכן על־פי הנוסחה היסודית של החשבון האינפיניטסמלי מתקיים
	\[
		\int_{1/n}^{2/n} f(x) dx = a_n = F(2/n) - F(1/n)
	\]
	מהגדרת הגבול לפי היינה ועל־פי משפט הרכבת פונקציות בגבול מאינפי 1 נקבל עבור הרכבת הפונקציה $g(t) = \frac{1}{t}$
	\begin{align*}
		\lim_{n \to \infty} n a_n
		& = \lim_{t \to 0} \frac{F(2t) - F(t)}{t} \\
		& = \lim_{t \to 0} \frac{F(2t) - F(2 \cdot 0) - F(t) + F(0)}{t} \\
		& = \lim_{t \to 0} 2\frac{F(2t) - F(2 \cdot 0)}{2t} - \lim_{t \to 0} \frac{F(t) - F(0)}{t} \\
		& = 2 f(0) - f(0) \\ 
		\lim_{n \to \infty} n a_n
		& = f(0)
	\end{align*}
\end{proof}

\subsection{סעיף ג'}
נגדיר
\[
	f(x) = \int_{x}^{x + 1} \sqrt{\arctan t} \, dt
\]
נוכיח כי $f(x) < \sqrt{\arctan(x + 1)}$ לכל $x \ge 0$.
\begin{proof}
	ידוע כי $g(x) = \sqrt{\arctan x}$ היא פונקציה עולה וחסומה, לכן $g(x) < g(x + 1)$ \\*
	להסביר למה המשיק לפונקציה $g$ תמיד מעל הפונקציה עצמה.
	\begin{align*}
		\int_{x}^{x + 1} g(t) \, dt
		& \le
		\int_{x}^{x + 1} g'(x + 1)(t - x - 1) + g(x + 1) \, dt && \text{נשים לב כי $x$ ערך קבוע באינטגרל} \\
		& = g'(x + 1) \frac{t^2}{2} - g'(x + 1)(x + 1)t + g(x + 1) t \Big|_x^{x + 1} \\
		& = g'(x + 1) (2x + 1)/2 - g'(x + 1) (x + 1) + g(x + 1) \\
		& = \frac{-g'(x + 1)}{2} + g(x + 1) \\
		& < g(x + 1)
	\end{align*}
	מצאנו כי $f(x) < g(x + 1)$.
\end{proof}

\section{שאלה 2}
נחשב את
\[
	\lim_{x \to \infty} \frac{\displaystyle\int_0^{\sqrt{x}} t^2 \arctan(e^t) dt}{\sqrt{x^3}}
\]
\end{document}
