\documentclass[a4paper]{article}

% packages
\usepackage{inputenc, fontspec, amsmath, amsthm, amsfonts, polyglossia, catchfile}
\usepackage[a4paper, margin=50pt, includeheadfoot]{geometry} % set page margins

% style
\AddToHook{cmd/section/before}{\clearpage}	% Add line break before section
\linespread{1.5}
\setcounter{secnumdepth}{0}		% Remove default number tags from sections
\setmainfont{Libertinus Serif}
\setsansfont{Libertinus Sans}
\setmonofont{Libertinus Mono}
\setdefaultlanguage{hebrew}
\setotherlanguage{english}

% operators
\DeclareMathOperator\cis{cis}
\DeclareMathOperator\Sp{Sp}
\DeclareMathOperator\tr{tr}
\DeclareMathOperator\im{Im}
\DeclareMathOperator\diag{diag}
\DeclareMathOperator*\lowlim{\underline{lim}}
\DeclareMathOperator*\uplim{\overline{lim}}

% commands
\renewcommand\qedsymbol{\textbf{משל}}
\newcommand{\NN}[0]{\mathbb{N}}
\newcommand{\ZZ}[0]{\mathbb{Z}}
\newcommand{\QQ}[0]{\mathbb{Q}}
\newcommand{\RR}[0]{\mathbb{R}}
\newcommand{\CC}[0]{\mathbb{C}}
\newcommand{\getenv}[2][] {
  \CatchFileEdef{\temp}{"|kpsewhich --var-value #2"}{\endlinechar=-1}
  \if\relax\detokenize{#1}\relax\temp\else\let#1\temp\fi
}
\newcommand{\explain}[2] {
	\begin{flalign*}
		 && \text{#2} && \text{#1}
	\end{flalign*}
}

% headers
\getenv[\AUTHOR]{AUTHOR}
\author{\AUTHOR}
\date\today

\title{פתרון ממ''ן 11 – חשבון אינפיניטסימלי 2 (20475)}

\begin{document}
\maketitle
\maketitleprint{}
\section{שאלה 1}
\subsection{סעיף א'}
נוכיח כי
\[
	\frac{2}{3\pi} \le \int_{2\pi}^{3\pi} \frac{\sin x}{x} dx \le \frac{1}{\pi}
\]
\begin{proof}
	תחילה נגדיר
	\[
		f(x) = \frac{\sin x}{x}
	\]
	זוהי כמובן פונקציה רציפה ולכן אינטגרבילית.
	נגדיר גם $g(x) = \frac{x^2}{2\pi}$ ונראה כי מתקיים $g(x) \ge f(x)$ לכל $x \ge 2\pi$.
	ממשפט 1.26 נובע כי
	\[
		\int_{2\pi}^{3\pi} f(x) dx \le \int_{2\pi}^{3\pi} g(x) dx = \frac{x}{\pi} \Big|_{2\pi}^{3\pi} = \frac{1}{\pi}
	\]
	נגדיר גם
	\[
		h(x) = \frac{\sin x}{3 \pi}
	\]
	כמובן שבתחום האינטגרל המונה של $f$ ו־$h$ זהים, אך $x \le 3\pi$ בתחום, ולכן גם $h(x) \le f(x)$ לכל התחום, ומתקיים
	\[
		\int_{2\pi}^{3\pi} f(x) dx \ge \int_{2\pi}^{3\pi} h(x) dx = \left. \frac{-\cos x}{3\pi} \right|_{2\pi}^{3\pi} = \frac{2}{3\pi}
	\]
	מצאנו כי
	\[
		\frac{2}{3\pi} \le \int_{2\pi}^{3\pi} \frac{\sin x}{x} dx \le \frac{1}{\pi}
	\]
\end{proof}

\subsection{סעיף ב'}
תהי $f(x)$ פונקציה רציפה בקטע $[0, 2]$. לכל $n \in \NN$ נגדיר
\[
	a_n = \int_{1/n}^{2/n} f(x) dx
\]
נוכיח כי $\lim_{n \to \infty} n a_n = f(0)$
\begin{proof}
	נגדיר $F(x)$ הפונקציה הקדומה של $f(x)$, לכן על־פי הנוסחה היסודית של החשבון האינפיניטסמלי מתקיים
	\[
		\int_{1/n}^{2/n} f(x) dx = a_n = F(2/n) - F(1/n)
	\]
	מהגדרת הגבול לפי היינה ועל־פי משפט הרכבת פונקציות בגבול מאינפי 1 נקבל עבור הרכבת הפונקציה $g(t) = \frac{1}{t}$
	\begin{align*}
		\lim_{n \to \infty} n a_n
		& = \lim_{t \to 0} \frac{F(2t) - F(t)}{t} \\
		& = \lim_{t \to 0} \frac{F(2t) - F(2 \cdot 0) - F(t) + F(0)}{t} \\
		& = \lim_{t \to 0} 2\frac{F(2t) - F(2 \cdot 0)}{2t} - \lim_{t \to 0} \frac{F(t) - F(0)}{t} \\
		& = 2 f(0) - f(0) \\ 
		\lim_{n \to \infty} n a_n
		& = f(0)
	\end{align*}
\end{proof}

\subsection{סעיף ג'}
נגדיר
\[
	f(x) = \int_{x}^{x + 1} \sqrt{\arctan t} \, dt
\]
נוכיח כי $f(x) < \sqrt{\arctan(x + 1)}$ לכל $x \ge 0$.
\begin{proof}
	ידוע כי $g(x) = \sqrt{\arctan x}$ היא פונקציה עולה וחסומה, לכן $g(x) < g(x + 1)$ \\*
	מחישוב ישיר של הנגזרת של $g(x)$ נוכל להסיק כי לכל $x > 0$ לכל $y > x$ מתקיים $g'(x) > g'(y)$
	ולכן שיפוע משיק לפונצקיה בנקודה גדול משל הפונקציה המקורית לכל נקודה לאחר נקודת ההשקה, ובהתאם בכל תחום זה המשיק גדול מ־$g$ עצמה. לכן
	\begin{flalign*}
		\int_{x}^{x + 1} g(t) \, dt
		& \le
		\int_{x}^{x + 1} g'(x + 1)(t - x - 1) + g(x + 1) \, dt && & \text{נשים לב כי $x$ ערך קבוע באינטגרל} \\
		& = \left. g'(x + 1) \frac{t^2}{2} - g'(x + 1)(x + 1)t + g(x + 1) t \right|_x^{x + 1} && & \text{נציב ב־$t$} \\
		& = g'(x + 1) (2x + 1)/2 - g'(x + 1) (x + 1) + g(x + 1) \\
		& = \frac{-g'(x + 1)}{2} + g(x + 1) && & \text{ידוע כי הפונקציה עולה ולכן הנגזרת חיובית} \\
		& < g(x + 1)
	\end{flalign*}
	לכן $f(x) < g(x + 1)$, דהינו $f(x) < \sqrt{\arctan(x + 1)}$
\end{proof}

\section{שאלה 2}
נחשב את
\[
	\lim_{x \to \infty} \frac{\displaystyle\int_0^{\sqrt{x}} t^2 \arctan(e^t) dt}{\sqrt{x^3}}
\]
אנו יודעים כי $\arctan t$ היא פונקציה עולה וחסומה, ואנו יודעים כי \[
	\lim_{t \to \infty} \arctan t = \frac{\pi}{2}
\]
לכן לכל $M > 0$ לכל $t > M$ מתקיים $\frac{\alpha_M \pi}{2} < \arctan{t}$ כאשר $\alpha_M = \arctan{M} \cdot \frac{2}{\pi}$ ונובע כי $0 < \alpha_M < 1$. \\*
אז לכמעט כל $x$ מתקיים גם $\frac{\alpha \pi}{2} t^2 \le t^2 \arctan(e^t) \le \frac{\pi}{2} t^2$ וממשפט 1.26 נובע
\[
	\int_0^{\sqrt{x}} \frac{\alpha \pi}{2} t^2 dt
	\le \int_0^{\sqrt{x}} t^2 \arctan(e^t) dt
	\le \int_0^{\sqrt{x}} \frac{\pi}{2} t^2 dt
\]
ידוע כי $(t^3/3)' = t^2$ ולכן מהמשפט היסודי של החשבון האינפיניטסמלי נובע
\[
	\frac{\alpha \pi}{2} \sqrt{x^3}
	\le \int_0^{\sqrt{x}} t^2 \arctan(e^t) dt
	\le \frac{\pi}{2} \sqrt{x^3}
\]
מהגדרת הגבול לפונקציות ומהגדרת $\alpha$ נובע כי $\lim_{x \to \infty} \alpha_M = 1$. \\*
לכן ממשפט הסנדוויץ' לגבולות נובע כי
\[
	\lim_{x \to \infty} \frac{\frac{\alpha \pi}{2} \sqrt{x^3}}{\sqrt{x^3}}
	= \lim_{x \to \infty} \frac{\int_0^{\sqrt{x}} t^2 \arctan(e^t) dt}{\sqrt{x^3}}
	= \lim_{x \to \infty} \frac{\frac{\pi}{2} \sqrt{x^3}}{\sqrt{x^3}} = \frac{\pi}{2}
\]

\section{שאלה 3}
תהי $f(x)$ פונקציה אינטגרבילית בכל קטע סגור חלקי ל־$\RR$, ולכל $x \in \RR$ מתקיים
\[
	f(x) = \int_0^x f(t) dt
\]
נוכיח כי $f \equiv 0$ ב־$\RR$.
\begin{proof}
	על־פי הנוסחה היסודית של החשבון האינפיניטסמלי מתקיים
	\[
		f(x) = F(x) - F(0)
	\]
	כאשר $F(x)$ היא הפונקציה הקדומה של $f(x)$, נגזור את השוויון ונקבל כי $f(x) = f'(x)$. \\*
	נגדיר $g(x) = \frac{f(x)}{e^x}$, ונגזור אותה:
	\[
		g'(x) = \frac{f'(x) e^x - f(x) e^x}{e^{2x}} = \frac{f(x) - f(x)}{e^x} = 0
	\]
	לכן $g(x)$ היא פונצקיה קבועה, לכן בהכרח $f(x) = c e^x$. בהתאם גם $F(x) = c e^x$ ומתקיים
	\[
		f(x) = F(x) - F(0) = f(x) - c \rightarrow c = 0
	\]
	ולכן $f(x) = 0 e^x = 0$.
\end{proof}

\section{שאלה 4}
נגדיר
\[
	f(x) = \begin{cases}
		\lfloor \cos \pi x \rfloor & 0 \le x \le 1 \\
		\lvert x - 2 \rvert & x > 1
	\end{cases}
\]

\subsection{סעיף א'}
נוכיח כי $f$ אינטגרבילית בקטע $[0, 3]$ אך אין לה פונקציה קדומה בקטע זה.
\begin{proof}
	נשים לב כי בשל הגדרתה מתקיים
	\[
		f(x) = \begin{cases}
			1 & x = 0 \\
			0 & 0 < x \le \frac{1}{2} \\
			-1 & \frac{1}{2} < x \le 1 \\
			2 - x & 1 < x < 2 \\
			x - 2 & 2 \le x \le 3
		\end{cases}
	\]
	אז על־פי הגדרה 1.15 הפונקציה $f$ מונוטונית למקוטעין בקטע $(0, 3]$ ולכן ממשפט 1.17 נובע כי היא אינטגרבילית בקטע זה. \\* % chktex 9
	מלמה 1.25 עבור הפונקציה $f(x)$ והפונקציה $g(x)$ המוגדרת על־ידי
	\[
		g(x) = \begin{cases}
			0 & x = 0 \\
			f(x) & x > 0
		\end{cases}
	\]
	נובע כי $f(x)$ אינטגרבילית בקטע הסגור $[0, 3]$. \\*
	נניח בשלילה כי קיימת פונקציה קדומה $F(x)$ ל־$f(x)$, ממשפט 8.12 באינפי 1 נובע כי כלל נקודות אי־הרציפות של $f$ הן ממין שני. \\*
	נבחין כי $x = 1$ היא נקודת אי־רציפות ב־$f(x)$ ממין ראשון, בסתירה לטענה, ולכן לא קיימת פונקציה $F$ כזו.
\end{proof}

\subsection{סעיף ב'}
ידוע כי בתחום $[0, \frac{1}{2}]$ מתקיים
\[
	F(x) = \int_0^x f(x) dx = \int_0^x g(x) dx = \int_0^x 0dx = \left. 0 \right|_0^x = 0
\]
בתחום $(\frac{1}{2}, 1]$ מתקיים $f(x) = -1$ ולכן % chktex 9
\[
	F(x) = \int_0^x f(x) dx = \int_0^\frac{1}{2} f(x) dx + \int_\frac{1}{2}^x -1 dx = \left. -x \right|_\frac{1}{2}^x = -x + \frac{1}{2}
\]
בתחום $(1, 2)$ מתקיים $f(x) = 2 - x$ ובהתאם
\[
	F(x) = \int_0^x f(x) dx = F(1) + \int_1^x (2 - x) dx = \left. 2x - \frac{x^2}{2} \right|_1^x + F(1) = - \frac{x^2}{2} + 2x - 2
\]
בתחום $[2, 3]$ ידוע כי $f(x) = x - 2$ ולכן
\[
	F(x) = \int_0^x f(x) dx = F(2) + \int_2^x (x - 2) dx = \left. \frac{x^2}{2} - 2x \right|_2^x + F(2) = \frac{x^2}{2} - 2x + 2
\]

\section{שאלה 5}
\subsection{סעיף א'}
נוכיח כי אם $f, g$ פונקציות לא יורדות ואי־שליליות בקטע $[a, b]$ אז $fg$ אינטגרבילית ב־$[a, b]$.
\begin{proof}
	לכל $x, y \in [a, b]$ מתקיים $0 < f(x) \le f(y)$ וגם $0 < g(x) \le g(y)$ ולכן בהתאם גם $0 < f(x) g(x) \le f(y) g(y)$. \\*
	אז גם $fg$ היא עולה במובן הרחב וממשפט 1.15 נובע כי היא אינטגרבילית ב־$[a, b]$.
\end{proof}

\subsection{סעיף ב'}
נפריך את הטענה כי אם $f(x)$ פונקציה רציפה ב־$[a, b]$ המקיימת $\int_a^b f(x) dx = 0$ אז קיימת נקודה $c \in (a, b)$ כך ש־$f(c) = 0$ בעזרת דוגמה נגדית.
\begin{proof}
	נגדיר $f(x) = 1$, בקטע $[0, 0]$, קטע מנוון אך מוגדר מקיים
	\[
		\int_0^0 f(x) dx = 0
	\]
	אך אין נקודה $c \in \RR$ כך ש־$f(c) = 0$, ובפרט אין נקודה כזו ב־$[0, 0]$.
\end{proof}

\subsection{סעיף ג'}
תהי $f(x)$ פונקציה אינטגרבילית בקטע $[a, b]$ ונגדיר $F(x) = \int_a^x f(t) dt$, ידוע כי $F$ איננה גזירה בקטע. \\*
נוכיח כי אין ל־$f$ פונקציה קדומה בקטע $[a, b]$
\begin{proof}
	נניח בשלילה כי $G(x)$ היא פונקציה המקיימת $G'(x) = f(x)$, דהינו $G$ היא פונקציה קדומה של $f$. \\*
	אז ממשפט 1.31 נובע כי הפונקציה $G$ ו־$F$ נבדלות ביניהן בקבוע, $F(x) = G(x) + C$. \\*
	אבל ידוע כי $F$ איננה גזירה ולכן גם $G(x) + C$ איננה גזירה בסתירה להנחה כי $G'(x) = f(x)$.
\end{proof}

\subsection{סעיף ד'}
תהי $f(x)$ מוגדרת בקטע $[a, b]$ ואינטגרבילית בקטע $[c, b]$ לכל $c \in (a, b)$. \\*
נוכיח כי $f$ אינטגרבילית בקטע $[a, b]$.
\begin{proof}
	לכל קטע $[a, c]$ ניתן להגדיר $[a, \frac{a + c}{2}]$ קטע חלקי ממש לקטע הראשון ואשר ידוע כי $(\frac{a + c}{2}, c) \subseteq [\frac{a + c}{2}, b]$. \\*
	דהינו בהפרש הקטעים $f$ היא אינטגרבילית. נוכל אם כן לבנות סדרת קטעים אינסופית של קטעים מוכלים ממש אשר בהפרשם $f$ אינטגרבילית. \\*
	על־פי הלמה של קנטור בחיתוך סדרת הקטעים נמצא ערך יחיד $c_0$, וידוע כי $f$ אינטגרבילית בכל הקטע $[a, b]$ מלבד ב־$c_0$. \\*
	עוד נשים לב כי בכל קטע מתקיים $a \in [a, c]$ ולכן נוכל להניח כי $a = c_0$, אז מלמה 1.25 נובע כי $f$ אינטגרבילית בכל הקטע.
\end{proof}

\section{שאלה 6}
תהיינה
\[
	f(x) = \sin x,
	g(x) = \begin{cases}
		x & 0 \le x < \frac{\pi}{2} \\
		\pi - x & \frac{\pi}{2} \le x < \frac{3 \pi}{2} \\
		x - 2\pi & \frac{3 \pi}{2} \le x \le 2 \pi
	\end{cases}
\]
נחשב את השטחה הכלוא בין הגרפים של הפונקציות עבור $0 \le x \le 2 \pi$. \\*
נחשב את האינטגרלים הלא מסוימים תחילה
\[
	F(x) = \int_0^x f(x) dx = - \cos x
\]
נחשב את האינטגרל של $g$ על־פי חלקיו:
\[
	G(x) = \int_0^x g(x) = \begin{cases}
		\frac{x^2}{2} & 0 \le x < \frac{\pi}{2} \\
		\pi x - \frac{x^2}{2} & \frac{\pi}{2} \le x < \frac{3 \pi}{2} \\
		\frac{x^2}{2} - 2\pi x & \frac{3 \pi}{2} \le x \le 2 \pi
	\end{cases}
\]
עוד נשים לב כי
\[
	\max\{ F(x), G(x) \} = \begin{cases}
		G(x) & 0 \le x < \frac{\pi}{2} \\
		F(x) & \frac{\pi}{2} \le x < \pi \\
		G(x) & \pi \le x < \frac{3 \pi}{2} \\
		F(x) & \frac{3 \pi}{2} \le x \le 2 \pi
	\end{cases}
\]
ולכן השטח הכלוא בין הגרפים הוא
\begin{align*}
	S
	& = \int_{0}^{\frac{\pi}{2}} (g(x) - f(x)) dx
	+ \int_{\frac{\pi}{2}}^{\pi} (f(x) - g(x)) dx
	+ \int_{\pi}^{\frac{3\pi}{2}} (g(x) - f(x)) dx
	+ \int_{\frac{3\pi}{2}}^{2\pi} (f(x) - g(x)) dx
	\\
	&=\left\lbrack G(x) - F(x) \right\rbrack_{0}^{\frac{\pi}{2}}
	+ \left\lbrack F(x) - G(x) \right\rbrack_{\frac{\pi}{2}}^{\pi}
	+ \left\lbrack G(x) - F(x) \right\rbrack_{\pi}^{\frac{3\pi}{2}}
	+ \left\lbrack F(x) - G(x) \right\rbrack_{\frac{3\pi}{2}}^{2\pi}
		\\
	& = G(\frac{\pi}{2}) - F(\frac{\pi}{2}) - G(0) + F(0)
	+ F(\pi) - G(\pi) - F(\frac{\pi}{2}) + G(\frac{\pi}{2}) \\
	& + G(\frac{3\pi}{2}) - F(\frac{3\pi}{2}) - G(\pi) + F(\pi)
	+ F(2\pi) - G(2\pi) - F(\frac{3\pi}{2}) + G(\frac{3\pi}{2})
		\\
	& = 2G(\frac{\pi}{2}) - 2F(\frac{\pi}{2}) - G(0) + F(0) + 2F(\pi) - 2G(\pi) + 2G(\frac{3\pi}{2}) - 2F(\frac{3\pi}{2}) + F(2\pi) - G(2\pi) \\
	& = \frac{1}{2}\pi^2 - 4
\end{align*}

\end{document} % chktex 17
