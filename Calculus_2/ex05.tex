\documentclass[a4paper]{article}

% packages
\usepackage{inputenc, fontspec, amsmath, amsthm, amsfonts, polyglossia, catchfile}
\usepackage[a4paper, margin=50pt, includeheadfoot]{geometry} % set page margins

% style
\AddToHook{cmd/section/before}{\clearpage}	% Add line break before section
\linespread{1.5}
\setcounter{secnumdepth}{0}		% Remove default number tags from sections
\setmainfont{Libertinus Serif}
\setsansfont{Libertinus Sans}
\setmonofont{Libertinus Mono}
\setdefaultlanguage{hebrew}
\setotherlanguage{english}

% operators
\DeclareMathOperator\cis{cis}
\DeclareMathOperator\Sp{Sp}
\DeclareMathOperator\tr{tr}
\DeclareMathOperator\im{Im}
\DeclareMathOperator\diag{diag}
\DeclareMathOperator*\lowlim{\underline{lim}}
\DeclareMathOperator*\uplim{\overline{lim}}

% commands
\renewcommand\qedsymbol{\textbf{משל}}
\newcommand{\NN}[0]{\mathbb{N}}
\newcommand{\ZZ}[0]{\mathbb{Z}}
\newcommand{\QQ}[0]{\mathbb{Q}}
\newcommand{\RR}[0]{\mathbb{R}}
\newcommand{\CC}[0]{\mathbb{C}}
\newcommand{\getenv}[2][] {
  \CatchFileEdef{\temp}{"|kpsewhich --var-value #2"}{\endlinechar=-1}
  \if\relax\detokenize{#1}\relax\temp\else\let#1\temp\fi
}
\newcommand{\explain}[2] {
	\begin{flalign*}
		 && \text{#2} && \text{#1}
	\end{flalign*}
}

% headers
\getenv[\AUTHOR]{AUTHOR}
\author{\AUTHOR}
\date\today

\usepackage{tikz}
\DeclareMathOperator\arcsinh{arcsinh}
\title{פתרון מטלה 5 – חשבון אינפיניטסימלי 2 (80132)}

\begin{document}
\maketitle
\maketitleprint{}

\Question{}
ניעזר בפולינום טיילור כדי למצוא קירוב רציונלי ל־$\cos \frac{1}{4}$ כך שהשגיאה לא תהיה מעל $10^{-12}$.
\begin{proof}[פתרון]
	בכיתה מצאנו כי
	\[
		P_{n, \cos, 0} = \sum_{k = 0}^{n} \frac{{(-1)}^k}{(2k)!} x^{2k}
	\]
	ונבחן את השארית בצורת לגרנז'
	\[
		R_n = \frac{\cos^{(k + 1)}(c)}{(k + 1)!}x^{k + 1}
	\]
	ובהתאם
	\[
		|R_n| \le \frac{x^{k + 1}}{(k + 1)!} = \frac{1}{4^{k + 1}(k + 1)!}
	\]
	אם כן נמצא $k \in \NN$ עבורו $4^{k + 1} (k + 1)! \ge 10^{12}$, נבחין כי תנאי זה מתקיים עבור $k = 9$ ולכן
	\[
		\cos \frac{1}{4} = \sum_{k = 0}^{5} \frac{{(-1)}^k}{(2n)!} x^{2k} + R_n,
		\qquad |R_n| < 10^{-12}
	\]
\end{proof}

\Question{}
נחשב את הגבול
\[
	\lim_{x \to 0} \frac{\cos(x) \sin(x) - x}{\sin^3(x)}
	= \lim_{x \to 0} \frac{\sin(2x) - 2x}{2\sin^3(x)}
\]

נבחין כי נגזרותיה הראשונות של $\sin(2x)$ הן $\sin(2x), 2\cos(2x), -4\sin(2x)$ ובהתאם עבור $x = 0$ נקבל $0, 2, 0$
ולכן הגבול שקול לביטוי
\[
	\lim_{x \to 0} \frac{0 + 2x - 0 + R_{3,\sin(2x), 0}(x) - 2x}{2 \sin^3(x)}
	= \lim_{x \to 0} \frac{o(x^3)}{2 \sin^3(x)}
	= \lim_{x \to 0} \frac{o(x^3)/x^3}{2 \sin^3(x)/x^3}
	= \frac{0}{2} = 0
\]

\Question{}
תהי $f : \RR \to \RR$ המוגדרת על־ידי $f(x) = \sin(x^{10})$ ונחשב את $f^{(2024)}(0),f^{(2025)}(0),f^{(2026)}(0)$.

נשתמש בתוצאת סעיף 6א' מהמטלה הקודמת ונקבל כי $\forall k \ge 0 : {(f(x))}^{(k)} = f^{(k)}(x) h(x)$ כאשר $h$ פולינום כלשהו. \\*
נציב ונקבל אף ש־$f^{(k)}(0) = \sin^{(k \mod 4)}(0) h(0)$. נבחין עתה כי $2024 \mod 4 = 0, 2026 \mod 4 = 2$ ולכן נקבל $f^{(2024)} = f^{(2026)} = \sin(0) h(0) = 0$. \\*
נשאר עתה לחשב את $f^{(2025)}$.

\end{document}
