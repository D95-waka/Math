\documentclass[a4paper]{article}

% packages
\usepackage{inputenc, amsmath, amsthm, thmtools, amsfonts, amssymb, luacode, catchfile, tikzducks, hyperref}
\usepackage[a4paper, margin=50pt, includeheadfoot]{geometry} % set page margins
\usepackage[shortlabels]{enumitem}
\usepackage[skip=3pt, indent=0pt]{parskip}

% language
\usepackage[bidi=basic, layout=tabular, provide=*]{babel}
\babelprovide[main, import]{hebrew}
\babelprovide{rl}
\babelfont{rm}{Libertinus Serif}
\babelfont{sf}{Libertinus Sans}
\babelfont{tt}{Libertinus Mono}

% style
\AddToHook{cmd/section/before}{\clearpage}	% Add line break before section
\linespread{1.3}
\setcounter{secnumdepth}{0}		% Remove default number tags from sections, this won't do well with theorems
\AtBeginDocument{\setlength{\belowdisplayskip}{3pt}}
\AtBeginDocument{\setlength{\abovedisplayskip}{3pt}}

% operators
\DeclareMathOperator\cis{cis}
\DeclareMathOperator\Sp{Sp}
\DeclareMathOperator\tr{tr}
\DeclareMathOperator\im{Im}
\DeclareMathOperator\re{Re}
\DeclareMathOperator\diag{diag}
\DeclareMathOperator*\lowlim{\underline{lim}}
\DeclareMathOperator*\uplim{\overline{lim}}
\DeclareMathOperator\rng{rng}
\DeclareMathOperator\Sym{Sym}
\DeclareMathOperator\Arg{Arg}
\DeclareMathOperator\Log{Log}
\DeclareMathOperator\dom{dom}

% commands
%\renewcommand\qedsymbol{\textbf{מש''ל}}
%\renewcommand\qedsymbol{\fbox{\emoji{lizard}}}
\newcommand{\NN}[0]{\mathbb{N}}
\newcommand{\ZZ}[0]{\mathbb{Z}}
\newcommand{\QQ}[0]{\mathbb{Q}}
\newcommand{\RR}[0]{\mathbb{R}}
\newcommand{\CC}[0]{\mathbb{C}}
\newcommand{\FF}[0]{\mathbb{F}}
\newcommand{\PP}[0]{\mathbb{P}}
\newcommand{\TT}[0]{\mathbb{T}}
\newcommand{\acts}[0]{\circlearrowright}
\newcommand{\explain}[2] {
	\begin{flalign*}
		 && \text{#2} && \text{#1}
	\end{flalign*}
}
\newcommand{\maketitleprint}[0]{ \begin{center}
	\begin{tikzpicture}[scale=3]
		\duck[graduate=gray!20!black, tassel=red!70!black]
	\end{tikzpicture}	
\end{center}
}

% theorem commands
\newtheoremstyle{c_remark}
	{}	% Space above
	{}	% Space below
	{}% Body font
	{}	% Indent amount
	{\bfseries}	% Theorem head font
	{}	% Punctuation after theorem head
	{.5em}	% Space after theorem head
	{\thmname{#1}\thmnumber{ #2}\thmnote{ \normalfont{\text{(#3)}}}}	% head content
\newtheoremstyle{c_definition}
	{3pt}	% Space above
	{3pt}	% Space below
	{}% Body font
	{}	% Indent amount
	{\bfseries}	% Theorem head font
	{}	% Punctuation after theorem head
	{.5em}	% Space after theorem head
	{\thmname{#1}\thmnumber{ #2}\thmnote{ \normalfont{\text{(#3)}}}}	% head content
\newtheoremstyle{c_plain}
	{3pt}	% Space above
	{3pt}	% Space below
	{\itshape}% Body font
	{}	% Indent amount
	{\bfseries}	% Theorem head font
	{}	% Punctuation after theorem head
	{.5em}	% Space after theorem head
	{\thmname{#1}\thmnumber{ #2}\thmnote{ \text{(#3)}}}	% head content

\theoremstyle{c_plain}
\newtheorem{theorem}{משפט}[section]
\newtheorem{lemma}[theorem]{למה}
\newtheorem{proposition}[theorem]{טענה}
\newtheorem*{proposition*}{טענה}
%\newtheorem{corollary}[theorem]{אין חלופה עברית}

\theoremstyle{c_definition}
\newtheorem{definition}[theorem]{הגדרה}
\newtheorem*{definition*}{הגדרה}
\newtheorem{example}{דוגמה}[section]
\newtheorem{exercise}{תרגיל}[section]

\theoremstyle{c_remark}
\newtheorem*{remark}{הערה}
\newtheorem*{solution}{פתרון}
\newtheorem{conclusion}[theorem]{מסקנה}
\newtheorem{notation}[theorem]{סימון}

% Questions related commands
\newcounter{question}
\setcounter{question}{1}
\newcounter{sub_question}
\setcounter{sub_question}{1}

\newcommand{\question}[1][0]{
	\ifthenelse{#1 = 0}{}{\setcounter{question}{#1}}
	\subsection{שאלה \arabic{question}}
	\addtocounter{question}{1}
	\setcounter{sub_question}{1}
}

\newcommand{\subquestion}[1][0]{
	\ifthenelse{#1 = 0}{}{\setcounter{sub_question}{#1}}
	\subsubsection{סעיף \localecounter{letters.gershayim}{sub_question}}
	\addtocounter{sub_question}{1}
}

% import lua and start of document
\directlua{common = require ('../common')}

\GetEnv{AUTHOR}

% headers
\author{\AUTHOR}
\date\today

\usepackage{tikz}
\DeclareMathOperator\arcsinh{arcsinh}
\title{פתרון מטלה 5 – חשבון אינפיניטסימלי 2 (80132)}

\begin{document}
\maketitle
\maketitleprint{}

\Question{}
ניעזר בפולינום טיילור כדי למצוא קירוב רציונלי ל־$\cos \frac{1}{4}$ כך שהשגיאה לא תהיה מעל $10^{-12}$.
\begin{proof}[פתרון]
	בכיתה מצאנו כי
	\[
		P_{n, \cos, 0} = \sum_{k = 0}^{n} \frac{{(-1)}^k}{(2k)!} x^{2k}
	\]
	ונבחן את השארית בצורת לגרנז'
	\[
		R_n = \frac{\cos^{(k + 1)}(c)}{(k + 1)!}x^{k + 1}
	\]
	ובהתאם
	\[
		|R_n| \le \frac{x^{k + 1}}{(k + 1)!} = \frac{1}{4^{k + 1}(k + 1)!}
	\]
	אם כן נמצא $k \in \NN$ עבורו $4^{k + 1} (k + 1)! \ge 10^{12}$, נבחין כי תנאי זה מתקיים עבור $k = 9$ ולכן
	\[
		\cos \frac{1}{4} = \sum_{k = 0}^{5} \frac{{(-1)}^k}{(2n)!} x^{2k} + R_n,
		\qquad |R_n| < 10^{-12}
	\]
\end{proof}

\Question{}
נחשב את הגבול
\[
	\lim_{x \to 0} \frac{\cos(x) \sin(x) - x}{\sin^3(x)}
	= \lim_{x \to 0} \frac{\sin(2x) - 2x}{2\sin^3(x)}
\]

נבחין כי נגזרותיה הראשונות של $\sin(2x)$ הן $\sin(2x), 2\cos(2x), -4\sin(2x), -8\cos(2x)$ ובהתאם עבור $x = 0$ נקבל $0, 2, 0, -8$
ולכן הגבול שקול לביטוי
\[
	\lim_{x \to 0} \frac{0 + 2x - 0 - \frac{8}{6} x^3 + R_{3,\sin(2x), 0}(x) - 2x}{2 \sin^3(x)}
	= \lim_{x \to 0} \frac{-\frac{2}{3}x^3 + o(x^3)}{\sin^3(x)}
	= \lim_{x \to 0} \frac{\frac{-2}{3} + o(x^3)/x^3}{\sin^3(x)/x^3}
	= -\frac{2}{3}
\]

\Question{}
תהי $f : \RR \to \RR$ המוגדרת על־ידי $f(x) = \sin(x^{10})$ ונחשב את $f^{(2024)}(0),f^{(2025)}(0),f^{(2026)}(0)$.

נשתמש בתוצאת סעיף 6א' מהמטלה הקודמת ונקבל כי $\forall k \ge 0 : {(f(x))}^{(k)} = f^{(k)}(x) h(x)$ כאשר $h$ פולינום כלשהו. \\*
עתה נבחין כי כל מונום ב־$h$ מוכפל ב־$\sin(x^{10})$ או ב־$\cos(x^{10})$, וכי הם מתחלפים בזוגיות (להוציא סימן) ולכן חזקת המונום זוגית רק במקרה של מכפלה ב־$\cos$ וזוגית במכפלה ב־$\sin$, ונקבל כי האיבר החופשי מוכפל תמיד ב־$\sin$. \\*
לכן לכל נגזרת באפס כלל המונומים מתאפסים בשל הכפולה ב־$x$, והאיבר החופשי מתאפס שכן $\sin(0^k) = 0$ לכל $k \in \NN$.
בהתאם נובע
\[
	f^{(2024)}(0) = f^{(2025)}(0) = f^{(2026)}(0) = 0
\]

\Question{}
\Subquestion{}
נוכיח כי לכל $n \in \NN$ מתקיים
\[
	0 < e - S_n < \frac{3}{(n + 1)!}
\]
כאשר $S_n = \sum_{k = 0}^{n} \frac{1}{k!}$.
\begin{proof}
	נבחין כי על־פי הנלמד בכיתה מתקבל $S_n = P_{n, \exp, 0}(1)$. \\*
	עוד ידוע ש־$\exp(1) = e$ ולכן $R_{n, \exp, 0}(1) = e - S_n$. \\*
	נציג את השארית בצורת לגרנז' ונקבל
	\[
		R_{n, \exp, 0}(1) = \frac{\exp^{(n + 1)}(c)}{(n + 1)!}{(1 - 0)}^{n + 1}
		= \frac{e^c}{(n + 1)!}
	\]
	עתה נבחין כי $0 < c < 1$ על־פי משפט השארית ובהתאם $1 < e^c < e < 3$ ומכאן נסיק ישירות כי
	\[
		0 < R_{n, \exp, 0}(1) < \frac{3}{(n + 1)!}
	\]
	ומצאנו כי אי־השוויון נכון.
\end{proof}

\Subquestion{}
נמצא קירוב רציונלי ל־$e$ כך שהשגיאה לא תעלה על $10^{-3}$. \\*
לאחר מכן נשווה תוצאה זו לקירוב ${(1 + \frac{1}{1000})}^{1000}$.

מצאנו בסעיף הקודם חסם לשארית הקירוב הרציונלי לפי פולינום טיילור ל־$e$, ולכן מספיק שנמצא $n \in \NN$ עבורו
\[
	3 \cdot 10^3 = 3000 < (n + 1)!
\]
מבדיקה מהירה נקבל כי $n = 6$ מקיים את השוויון והוא הטבעי המינימלי הכזה (נשבע שבדקתי בדף ולא במחשבון), \\*
לכן $e = \sum_{k = 0}^{6} \frac{1}{k!}$ בקירוב הרלוונטי.
\begin{align*}
	e & = 2.71828182846\dots \\
	& \approx \sum_{k = 0}^{6} \frac{1}{k!} = 2.718055555\dots \\
	& \approx {(1 + \frac{1}{1000})}^{1000} = 2.71692393224
\end{align*}
דהינו הקירוב שמצאנו הוא הרבה יותר מדויק ביחס לכמות החישוב הנדרשת.

\Subquestion{}
נוכיח כי $e$ לא רציונלי.
\begin{proof}
	נניח בשלילה כי $e$ רציונלי, ולכן קיימים $p, q \in \NN$ עבורם מתקיים $e = \frac{p}{q}$.
	עתה נשים לב כי
	\[
		e = \frac{p}{q}
	\]
	נבחין כי מסעיף 1 נובע
	\begin{align*}
		0 < \frac{p}{q} - \sum_{k = 0}^{q} \frac{1}{k!} < \frac{3}{(q + 1)!} \\
		0 < \frac{p(q - 1)!}{q!} - \sum_{k = 0}^{q} \frac{1}{k!} < \frac{3}{(q + 1)!} \\
		0 < \frac{p(q - 1)! - \sum_{k = 0}^{q} k!}{q!} < \frac{3}{(q + 1)!} \\
		0 < p(q - 1)! - \sum_{k = 0}^{q} k! < \frac{3}{q + 1}
	\end{align*}
	ולכן $C = p(q - 1)! - \sum_{k = 0}^{q} k! $ מקיים $0 < C < 1$ עבור $q le 2$ בסתירה לזה שהוא מספר שלם מהגדרתו, ולכן $0 < q < 2$. \\*
	אבל מצאנו בסעיף הראשון כי לא יתכן ש־$e$ הוא מתחלק ב־$2$ והגענו לסתירה. \\*
	לכן $e$ לא רציונלי.
\end{proof}

\Question{}
תהי $f : \RR \to \RR$ פונקציה גזירה אינסוף פעמים בכל נקודה. \\*
ידוע גם כי לכל $b > 0$ קיים $M$ קבוע כך שלכל $n \in \NN$ ולכל $x \in [0, b]$ מתקיים $|f^{(n)}(x)| \le M$. \\*
נוכיח שלכל $b > 0$ מתקיים $\lim_{n \to \infty} P_n(b) = f(b)$ כאשר $P_n(x) = P_{n, f, 0}(x)$.
\begin{proof}
	יהי $b > 0$ אז קיים $M$ קבוע כך שלכל $x \in [0, b]$ מתקיים
	\[
		\frac{|f^{(n + 1)}(b)|}{(n + 1)!} < \frac{M}{(n + 1)!}
	\]
	ובהתאם גם
	\[
		0 \le R_n(x) = \frac{|f^{(n + 1)}(b)|}{(n + 1)!}x^{n + 1} < \frac{M b^{n + 1}}{(n + 1)!}
	\]
	כאשר $R_n$ שארית בצורת לגרנז'.
	כמובן
	\[
		\lim_{n \to \infty} \frac{M b^{n + 1}}{(n + 1)!} = 0
	\]
	ולכן מכלל הסנדוויץ' נובע כי $R_n(x) \xrightarrow[n \to \infty]{} 0$ ולכן נסיק כי
	\[
		\lim_{n \to \infty} P_n(x) = \lim_{n; \to \infty} f(x) - R_n(x) = f(x)
	\]
	ובפרט הגבול מתקיים גם עבור $x = b$.
\end{proof}

\Question{}
יהי $\alpha \in \QQ$ ונגדיר $f : (-1, \infty) \to \RR$ על־ידי $f(x) = {(x + 1)}^\alpha$.

\Subquestion{}
יהי $n \in \NN \cup \{0\}$ ונחשב את $P_{n, f, 0}$.

תחילה נבחין כי $f^{(n)}(x) = (\prod_{k = 0}^{n - 1} \alpha^k) {(x + 1)}^{\alpha - n}$ על־פי חוקי גזירה. \\*
עתה נחשב את פולינום טיילור:
\[
	P_n(x) = \sum_{k = 0}^{n} \frac{f^{(n)}(0)}{k!} x^k
	= \sum_{k = 0}^{n} (\prod_{k = 0}^{n - 1} \alpha^k) {(0 + 1)}^{\alpha - n} \frac{1}{k!} 1^{\alpha - k} x^k
	= \sum_{k = 0}^{n} \binom{\alpha}{k} x^k
\]

\Subquestion{}
נחשב את $P_4$ עבור $f$ כאשר $\alpha = \frac{1}{2}$.

קיבלנו
\[
	P_4(x) = \sum_{k = 0}^{4} \binom{\frac{1}{2}}{k} x^k
	= 1 + \frac{1}{4} x + \frac{1}{32} x^2 + \dots
\]

\Subquestion{}
\subsubsection{i.}
נקבל כי
\[
	P_0(9) = 1,
	P_1(9) = 5.5,
	P_2(9) = -4.65,
	P_3(9) = 40.9375,
	P_4(9) = -215\dots
\]
ונבחין כי הסדרה הזו לא מתכנסת, אף לא במובן הרחב.

\subsubsection{ii.}
נקבל הפעם כי
\[
	3 \cdot P_0(9) = 3,
	3 \cdot P_1(9) = 3.166\dots,
	3 \cdot P_2(9) = 3.1620\dots,
	3 \cdot P_3(9) = 3.1622\dots,
	3 \cdot P_4(9) = 3.1622\dots
\]
הפעם ניתן לראות כי סדרת המספרים נזכירה את ערכו של $\sqrt{10}$ ונראה כי היא נוטה להתכנסות.

\Subquestion{}
ננסה לחסום מלעיל את הפיתוח של $y = \sqrt{\frac{6}{5}}$ הנתון על־ידי $A = 1.095 = 1 + \frac{1}{10} - \frac{1}{200}$.

נתבקשנו לחסום מלעיל, ולא למצוא חסם עליון, ולכן נבחר $1000000000000000000000000$ כחסם ונראה כי הוא אכן תקף.

\Question{}
יהיו $L, U \subseteq \RR$ לא ריקות כך ש־$L \le U$.

\Subquestion{}
נוכיח כי $\sup(L) = \inf(U)$ אם ורק אם קיימות סדרות ${(l_n)}_{n = 1}^\infty, {(u_n)}_{n = 1}^\infty$ כך ש־$u_n \in U, l_n \in L$ לכל $n \in \NN$ ומתקיים $\lim_{n \to \infty} (u_n - l_n) = 0$. \\*
נראה גם כי $\lim_{n \to \infty} l_n = \sup(L) = \inf(U) = \lim_{n \to \infty} u_n$.
\begin{proof}
	\textbf{כיוון ראשון:}
	נניח כי $\sup(L) = \inf(U)$, דהינו קיים $M$ כך שמתקיים $\forall u \in U, l \in L : l \le M \le u$. \\*
	מההגדרה של אינפימום וסופרמום נקבל גם כי לכל $M - \frac{1}{n}$ ו־$M + \frac{1}{n}$ בהתאמה $\exists l \in L, u \in U : M - \frac{1}{n} \le u \le M \le l \le M + \frac{1}{n}$. \\*
	לכל $n$ נגדיר $u_n, l_n$ איברים המקיימים את אי־השוויון האחרון כפי שהוא מבטיח שקיימים. \\*
	מהגדרה זו נובע ישירות כי
	\[
		\lim_{n \to \infty} u_n = \lim_{n \to \infty} l_n = M
	\]
	ולכן גם $\lim_{n \to \infty} (u_n - l_n) = 0$.

	\textbf{כיוון שני:}
	נניח כי קיימות סדרות $(l_n), (u_n)$ עבורן
	\[
		\lim_{n \to \infty} (u_n - l_n) = 0
	\]
	נגדיר $M = \inf(U)$ ולכן מאי־השוויון $L \le U$ נסיק $l_n \le M \le u_n$. \\*
	נחסר ונקבל $0 \le M - l_n \le u_n - l_n$ ולכן מכלל הסנדוויץ' נקבל ישירות $M - l_n \xrightarrow[n \to \infty]{} 0$. \\*
	בהתאם $l_n \xrightarrow[n \to \infty]{} M$. נבחין כי $\forall l \in L : l \le M$ מאי־שוויון הקבוצות, וקיבלנו כי לכל $M - \delta$ קיים $l \in L$ המקיים $M - \delta \le l_n$ ולכן זהו אינפימום. \\*
	קיבלנו כי $M = \inf U = \sup L$.
\end{proof}

\Question{}
ניתן דוגמה לפונקציות $f, g$ החסומות ב־$[a, b]$ כך שמתקיים
\[
	\underline{\int_{a}^{b}} f(x)dx
	+ \underline{\int_{a}^{b}} g(x)dx
	< \underline{\int_{a}^{b}} (f(x) + g(x))dx
\]

נבחר $f(x) = D(x)$ ו־$g(x) = 1 - D(x)$ כאשר $D$ היא פונקציית דיריכלה. \\*
בכיתה הראינו כי
\[
	\underline{\int_{a}^{b}} D(x)dx
	= \underline{\int_{a}^{b}} f(x)dx
	= 0
\]
ולכן נסיק גם
\[
	\underline{\int_{a}^{b}} 1 - D(x)dx
	= \underline{\int_{a}^{b}} g(x)dx
	= 0
\]
אבל $g(x) + f(x) = 1$ ולכן
\[
	\underline{\int_{a}^{b}} (f(x) + g(x))dx = a - b
\]
ומצאנו כי הפונקציות עומדות בתנאי.

\Question{}
תהי $f : [0, 1] \to \RR$ פונקציה המוגדרת על־ידי
\[
	f(x) = \begin{cases}
		x & x \in \QQ \cap [0, 1] \\
		0 & x \in [0, 1] \setminus \QQ
	\end{cases}
\]

\Subquestion{}
תהי $P = \{x_0, \dots, x_n\}$ חלוקה של $[0, 1]$, ונחשב את ערכם של $m_i, M_i$ המוגדרים על־ידי
\[
	m_i = \inf_{x_{i - 1} \le x \le x_{i + 1}} \{ f(x) \},
	M_i = \sup_{x_{i - 1} \le x \le x_{i + 1}} \{ f(x) \}
\]

נבחין כי לכל $x \in [0, 1]$ מתקיים $f(x) = 0$ או $f(x) = x$ ולכן בהתאם נוכל לקבוע כי $m_i = 0$. \\*
את הסופרמום נקבע על־פי צפיפות הרציונליים ולכן $M_i = x_{i + 1}$.

\Subquestion{}
נוכיח כי $f$ לא אינטגרבילית ב־$[0, 1]$.
\begin{proof}
	מצאנו בסעיף הקודם כי על־פי הגדרה מתקיים
	\[
		\underline{\int_0^1} f(x) dx = 0
	\]
	וגם כי
	\[
		\overline{\int_0^1} f(x) dx > 0
	\]
	ולכן הפונקציה לא עומדת בהגדרה לאינטגרביליות.
\end{proof}

\end{document}
