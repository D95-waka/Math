\documentclass[a4paper]{article}

% packages
\usepackage{inputenc, fontspec, amsmath, amsthm, amsfonts, polyglossia, catchfile}
\usepackage[a4paper, margin=50pt, includeheadfoot]{geometry} % set page margins

% style
\AddToHook{cmd/section/before}{\clearpage}	% Add line break before section
\linespread{1.5}
\setcounter{secnumdepth}{0}		% Remove default number tags from sections
\setmainfont{Libertinus Serif}
\setsansfont{Libertinus Sans}
\setmonofont{Libertinus Mono}
\setdefaultlanguage{hebrew}
\setotherlanguage{english}

% operators
\DeclareMathOperator\cis{cis}
\DeclareMathOperator\Sp{Sp}
\DeclareMathOperator\tr{tr}
\DeclareMathOperator\im{Im}
\DeclareMathOperator\diag{diag}
\DeclareMathOperator*\lowlim{\underline{lim}}
\DeclareMathOperator*\uplim{\overline{lim}}

% commands
\renewcommand\qedsymbol{\textbf{משל}}
\newcommand{\NN}[0]{\mathbb{N}}
\newcommand{\ZZ}[0]{\mathbb{Z}}
\newcommand{\QQ}[0]{\mathbb{Q}}
\newcommand{\RR}[0]{\mathbb{R}}
\newcommand{\CC}[0]{\mathbb{C}}
\newcommand{\getenv}[2][] {
  \CatchFileEdef{\temp}{"|kpsewhich --var-value #2"}{\endlinechar=-1}
  \if\relax\detokenize{#1}\relax\temp\else\let#1\temp\fi
}
\newcommand{\explain}[2] {
	\begin{flalign*}
		 && \text{#2} && \text{#1}
	\end{flalign*}
}

% headers
\getenv[\AUTHOR]{AUTHOR}
\author{\AUTHOR}
\date\today

\usepackage{tikz}
\DeclareMathOperator\arcsinh{arcsinh}
\title{פתרון מטלה 9 – חשבון אינפיניטסימלי 2 (80132)}

\begin{document}
\maketitle
\maketitleprint{}

\Question{}
נבדוק את התכנסות וערך האינטגרלים הבאים.

\subsection{i.}
\[
	\int_{0}^{1} \ln(x)\ dx
\]
נבחין כי בתחום מתקבל
\[
	\int \ln(x)\ dx = \frac{1}{x}
\]
ולכן אם נניח שהאינטגרל מוגדר נקבל כי
\[
	\lim_{C \to 0} \frac{1}{1} - \frac{1}{C} = -\infty
\]
מתכנס, בסתירה כמובן להתבדרותו.

\subsection{ii.}
\[
	\int_{0}^{\infty} \frac{\arctan(x)}{x^2 + 1} dx
\]
נגדיר $t = \arctan x, dt = \frac{dx}{1 + x^2}$ ולכן נקבל
\[
	\int_{0}^{\pi} t\ dt = \pi
\]

\Question{}
יהי $a \in \RR$ ותהינה $f, g : [a, \infty) \to \RR$ פונקציות אינטגרביליות ב־$[a, N]$ עבור כל $a < N$. \\* % chktex 9
נתון כי קיים $c \in [a, \infty)$ כך שלכל $c \le x$ מתקיים $0 < f(x)$ ו־$0 < g(x)$. % chktex 9

\end{document} % chktex 17
