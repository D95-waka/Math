\documentclass[a4paper]{article}

% packages
\usepackage{inputenc, fontspec, amsmath, amsthm, amsfonts, polyglossia, catchfile}
\usepackage[a4paper, margin=50pt, includeheadfoot]{geometry} % set page margins

% style
\AddToHook{cmd/section/before}{\clearpage}	% Add line break before section
\linespread{1.5}
\setcounter{secnumdepth}{0}		% Remove default number tags from sections
\setmainfont{Libertinus Serif}
\setsansfont{Libertinus Sans}
\setmonofont{Libertinus Mono}
\setdefaultlanguage{hebrew}
\setotherlanguage{english}

% operators
\DeclareMathOperator\cis{cis}
\DeclareMathOperator\Sp{Sp}
\DeclareMathOperator\tr{tr}
\DeclareMathOperator\im{Im}
\DeclareMathOperator\diag{diag}
\DeclareMathOperator*\lowlim{\underline{lim}}
\DeclareMathOperator*\uplim{\overline{lim}}

% commands
\renewcommand\qedsymbol{\textbf{משל}}
\newcommand{\NN}[0]{\mathbb{N}}
\newcommand{\ZZ}[0]{\mathbb{Z}}
\newcommand{\QQ}[0]{\mathbb{Q}}
\newcommand{\RR}[0]{\mathbb{R}}
\newcommand{\CC}[0]{\mathbb{C}}
\newcommand{\getenv}[2][] {
  \CatchFileEdef{\temp}{"|kpsewhich --var-value #2"}{\endlinechar=-1}
  \if\relax\detokenize{#1}\relax\temp\else\let#1\temp\fi
}
\newcommand{\explain}[2] {
	\begin{flalign*}
		 && \text{#2} && \text{#1}
	\end{flalign*}
}

% headers
\getenv[\AUTHOR]{AUTHOR}
\author{\AUTHOR}
\date\today

\usepackage{tikz}
\DeclareMathOperator\arcsinh{arcsinh}
\title{פתרון מטלה 9 – חשבון אינפיניטסימלי 2 (80132)}
% chktex-file 9

\begin{document}
\maketitle
\maketitleprint{}

\Question{}
נבדוק את התכנסות וערך האינטגרלים הבאים.

\subsection{i.}
\[
	\int_{0}^{1} \ln(x)\ dx
\]
נבחין כי בתחום מתקבל
\[
	\int \ln(x)\ dx = x \ln x - x
\]
ולכן נקבל
\[
	\int_{0}^{1} \ln(x)\ dx
	= \lim_{h \to 0} x \ln x - x \mid_h^1
	= \lim_{h \to 0} -1 - h \ln h - h
	= -1 - 0 \ln 0 - 0
	= -1
\]
ומצאנו כי האינטגרל מתכנס ל־$-1$.

\subsection{ii.}
\[
	\int_{0}^{\infty} \frac{\arctan(x)}{x^2 + 1} dx
\]
נגדיר $t = \arctan x, dt = \frac{dx}{1 + x^2}$ ולכן נקבל
\[
	\int_{0}^{\pi} t\ dt = \pi
\]

\Question{}
יהי $a \in \RR$ ותהינה $f, g : [a, \infty) \to \RR$ פונקציות אינטגרביליות ב־$[a, N]$ עבור כל $a < N$. \\*
נתון כי קיים $c \in [a, \infty)$ כך שלכל $c \le x$ מתקיים $0 < f(x)$ ו־$0 < g(x)$. \\*
נוכיח כי אם הגבול $\lim_{x \to \infty} \frac{f(x)}{g(x)}$ קיים במובן הצר והוא גדול ממש מאפס, אז האינטגרלים $\int_{a}^{\infty} f(x)\ dx$ ו־$\int_{a}^{\infty} g(x)\ dx$ מתבדרים ומתכנסים יחד.
\begin{proof}
	נסמן $\lim_{x \to \infty} \frac{f(x)}{g(x)} = L < \infty$ ולכן עבור $\epsilon = \frac{L}{2}$ לכמעט כל $x$ מתקיים
	\[
		\left| \frac{f(x)}{g(x)} - L \right| < \frac{L}{2}
		\iff
		L - \frac{L}{2} < \frac{f(x)}{g(x)} < \frac{3L}{2}
		\iff
		0 < \frac{L}{2} g(x) < f(x) < \frac{3L}{2} f(x)
	\]
	אם כמובן נניח כי $\int_{a}^{\infty} g(x)\ dx$ מתכנס אז נקבל ישירות ממבחן ההשוואה כי $\int_{a}^{\infty} f(x)\ dx$ מתכנס אף הוא. \\*
	אם נניח כי $\int_{a}^{\infty} g(x)\ dx$ מתבדר נקבל כי גם $\int_{a}^{\infty} f(x)\ dx$ מתבדר.
\end{proof}

\Question{}
נבדוק את התכנסות האינטגרלים הבאים:

\Subquestion{}
\[
	\int_{0}^{\infty} \frac{\cos x}{x}\ dx
\]
נשים לב כי $\cos x > \frac{1}{2}$ בסביבה של $0$ ולכן האינטגרל מתכנס אם
\[
	\int_{0}^{\delta} \frac{1}{x} dx
\]
מתכנס, ואנו כבר יודעים כי הוא מתבדר, ולכן נסיק כי גם האינטגרל המקורי מתבדר.

\Subquestion{}
\[
	\int_{0}^{1} \frac{\sin^2(x)}{x^2}\ dx
\]
אנו יודעים כי
\[
	\lim_{x \to 0} {(\frac{\sin x}{x})}^2 = 1
\]
ולכן ממשפט ההתכנסות האינטגרלי בגרסה הגבולית נסיק כי האינטגרל מתכנס, וכמובן גם בהחלט מהחיוביות בתחום.

\Subquestion{}
\[
	\int_{1}^{\infty} \exp(-\sqrt{x})\ dx
\]
ממבחן ההשוואה יחד עם $g(x) = \frac{1}{x}$ נקבל כי האינטגרל מתכנס.

\Subquestion{}
\[
	\int_{1}^{2} \frac{dx}{\ln x}
\]
אילו נגדיר $f(x) = \ln x, g(x) = \frac{1}{x - 1}$ נקבל מלופיטל
\[
	\lim_{x \to 1} \frac{f(x)}{g(x)}
	= \lim_{x \to 1} \frac{\frac{1}{x}}{1}
	= 1
\]
ולכן מהגרסה הגבולית של מבחן ההשוואה נקבל כי האינטגרל מתכנס.

\Subquestion{}
\[
	\int_{0}^{\infty} \frac{\sin x}{x^{3/2}}\ dx
\]
בסביבה של $0$ נקבל כי האינטגרל מתכנס על־פי מבחן ההשואה הגבולי יחד עם $g(x) = \frac{1}{\sqrt{x}}$ ובדיקה ישירה של האינטגרל של $g$. \\*
בסביבה של $\infty$ נקבל ממבחן כי הערך המוחלט של הפונקציה חסום על־ידי $h(x) = \frac{1}{x^{3/2}}$ וזה כמובן מתכנס על־פי אינטגרציה ישירה, ולכן קיבלנו כי הפונקציה מתכנסת בהחלט.

\Subquestion{}
\[
	\int_{0}^{1} {(\ln x)}^7\ dx
\]
נגדיר $x = e^t, dx = e^t dt$ ולכן האינטגרל שקול לאינטגרל
\[
	\int_{-\infty}^{0} t^7 e^t\ dt
\]
ומצאנו במטלה הקודמת כי אינטגרל מהצורה הזו ניתן לחישוב וערכו הוא פולינום כלשהו כפול אקספוננט ולכן נוכל להסיק כי האינטגרל מתכנס, כמובן בתחום הוא לא משנה סימן ונסיק עי הוא מתכנס בהחלט.

\Question{}
תהי $f : [1, \infty) \to \RR$ פונקציה רציפה ויהיו $a, b \in \RR$ עם $1 \le a < b$.

\Subquestion{}
נוכיח כי אם $\int_{1}^{\infty} f(x)\ dx$ מתכנס בהחלט אז $\int_{1}^{\infty} f(x) \sin(x)\ dx$ מתכנס.
\begin{proof}
	נתון כי $\int_{1}^{\infty} |f(x)|\ dx$ מתכנס, ונבחין כי
	\[
		|f(x) \sin x| = |f(x)| \cdot |\sin x| \le |f(x)|
	\]
	ולכן נסיק ממבחן ההשוואה הראשון כי $\int_{1}^{\infty} |f(x) \sin x|\ dx$ מתכנס, ולכן $\int_{1}^{\infty} f(x) \sin x\ dx$ מתכנס בהחלט ובפרט מתכנס.
\end{proof}

\Subquestion{}
נפריך את הטענה כי אם $f$ מונוטונית עולה ב־$[1, \infty)$ אז $\int_{1}^{\infty} f(x)\ dx$ מתבדר.

נבחר את הדוגמה הנגדית $f(x) = -\frac{1}{x^2}$.
נבחין כי הפונקציה אכן מונוטונית עולה ממש בתחום, ונשים לב גם כי $\int f(x)\ dx = -2 \frac{1}{x} + C$ ולכן נסיק כי האינטגרל שלה בתחום מתכנס.

\Question{}
נחשב את הפונקציות הקדומות של הפונקציות הרציונליות הבאות.

\Subquestion{}
\begin{align*}
	\int \frac{2x - 5}{x^2 + 6}\ dx
	& = \int \frac{2x}{x^2 + 6} - \frac{5}{x^2 + 6}\ dx \\
	& = \ln |x^2 + 6| - \frac{5}{6} \int \frac{dx}{{(\frac{x}{\sqrt{6}})}^2 + 1} \\
	& = \ln |x^2 + 6| - \frac{5}{\sqrt{6}} \int \frac{dt}{t^2 + 1} \\
	& = \ln |x^2 + 6| - \frac{5}{\sqrt{6}} \arctan(\frac{x}{\sqrt{6}})
\end{align*}

\Subquestion{}
\begin{align*}
	\int \frac{x^3 + 2x^2 - 2}{(x + 1)(x + 2)}dx
	& = \int x - \frac{x}{x + 2} - \frac{1}{x + 1} dx \\
	& = \frac{1}{2} x^2 - \ln|x + 1| - \int 1 - \frac{1}{x + 2} dx \\
	& = \frac{1}{2} x^2 - \ln|x + 1| - x + \ln|x + 2|
\end{align*}

\Subquestion{}
\begin{align*}
	\int \frac{4x^2 - 3x - 4}{{(x - 1)}^3 (x + 2)} dx
	= \int \frac{4(x^2 + x - 2) - 7x + 4}{{(x - 1)}^3 (x + 2)} dx \\
	= \frac{4}{1 - x} + \int \frac{- 7x + 4}{{(x - 1)}^3 (x + 2)} dx \\
	= \frac{4}{1 - x} + \int \frac{- 7(x + 2)}{{(x - 1)}^3 (x + 2)} + \frac{18}{{(x - 1)}^3 (x + 2)} dx \\
	= \frac{4}{1 - x} + \frac{7}{2{(x - 1)}^2} + \int \frac{18}{{(x - 1)}^3 (x + 2)} dx \\
\end{align*}

\Question{}
יהיו $b, c \in \RR$ כך ש־$x^2 + bx + c$ פולינום ריבועי אי־פריק, ויהי $n \in \NN$.

\Subquestion{}
נביע את האינטגרל
\[
	\int \frac{1}{{(x^2 + bx + c)}^n} dx
\]
על־ידי $I_n$.

\[
	\int \frac{1}{{(x^2 + bx + c)}^n} dx
	= \int \frac{1}{{({(x + \frac{b}{2})}^2 + c - \frac{b^2}{4})}^n} dx
	= {(c - \frac{b^2}{4})}^{-n} \int \frac{1}{{({(\frac{x + \frac{b}{2}}{\sqrt{c - b^2/4}})}^2 + 1)}^n} dx
	= {(c - \frac{b^2}{4})}^{-n} I_n
\]

\Subquestion{}
נחשב את האינטגרל הבא כתלות ב־$I_n$.
\[
	\int \frac{Ax + B}{{(x^2 + bx + c)}^n} dx
\]

מלא חישוב לא מעניין.

\Question{}
תהי סדרה חסומה ${(a_n)}_{n = 1}^\infty$ ונסמן $u_k = \sup\{ a_n \mid n \ge k \}$ ו־$l_k = \inf\{ a_n \mid n \ge k \}$. \\*
נחשב את $u_k, l_k$ ונחשב את הגבול העליון והתחתון של הסדרה $(a_n)$.

\Subquestion{}
$(a_n)$ סדרה מונוטונית עולה המתכנסת ל־$A \in \RR$.

נבחין כי $a_k \le a_n$ לכל $n \in \NN$ ולכן נסיק $l_k = a_k$. \\*
באופן דומה נראה כי $a_n < L$ לכל $n \in \NN$, ומהגדרת הגבול נסיק כי זהו הגבול עצמו, דהינו $u_k = A$.

\Subquestion{}
$a_n = {(-1)}^n (1 + \frac{1}{n})$.

נבחין כי $a_{2n} = 1 + \frac{1}{n}$, וגם $a_{2n - 1} = -1 - \frac{1}{n}$. \\*
לכן נקבל כי $a_n \xrightarrow{n \to \infty} 1$ ולכן נסיק $u_k = 1 + \frac{1}{n}$, ונסיק באופן דומה כי $u_k = -1 - \frac{1}{n}$.

\Subquestion{}
$a_n = {(-1)}^n + \frac{1}{n}$

נקבל $a_{2n} = 1 + \frac{1}{n}, a_{2n + 1} = -1 + \frac{1}{n}$. \\*
לכן נבחין כי $1 + \frac{1}{k}$ הוא הערך הגדול ביותר שנקבל עבור $n \ge k$, ונקבל $u_k = 1 + \frac{1}{k}$, ונבחין כי $a_{2n + 1}$ מונוטונית יורדת ולכן נסיק $l_k = -1$.

\Question{}
תהי ${(a_n)}_{n = 1}^\infty$ כלשהי.

\Subquestion{}
נוכיח כי אם $(a_n)$ חסומה אז מתקיים
\[
	\uplim_{n \to \infty} (-a_n) = - \lowlim_{n \to \infty}(a_n)
\]
\begin{proof}
	ידוע שהקבוצה חסומה ולכן נוכל להסיק כי כלל הגבולות החלקיים חסומים אף הם, נניח
	\[
		\lowlim_{n \to \infty} a_n = l,
		\qquad
		\uplim_{n \to \infty} a_n = L
	\]
	עתה, אם נבחן את $(-a_n)$ נקבל כי הגבולות החלקיים שלה כולם הם הגבולות החלקיים של $(a_n)$ עם היפוך סימן, ולכן נוכל להסיק כי גם
	\[
		\lowlim_{n \to \infty} -a_n = -L,
		\qquad
		\uplim_{n \to \infty} -a_n = -l
	\]
	ולמעשה הוכחנו עתה את הטענה.
\end{proof}

\Subquestion{}
נוכיח שאם $\lowlim_{n \to \infty} a_n < \infty$ אז יש $L \in \RR$ כך ש־$a_n < L$ מתקיים באופן שכיח.
\begin{proof}
	נגדיר $\lowlim_{n \to \infty} a_n = K$ ואנו יודעים כי $K \in \RR$, אז אנו יודעים כי קיימת $n_k$ כך ש־$a_{n_k} \xrightarrow{k \to \infty} K$. \\*
	לכן מהגדרת הגבול נקבל כי $-1 < a_{n_k} - K < 1$ ולכן $K - 1 < a_{n_k} < K + 1$ ונוכל לבחור $L = K + 1$ וקיבלנו כי התכונה אכן שכיחה עבור $n$.
\end{proof}

\end{document} % chktex 17
