\documentclass[a4paper]{article}

% packages
\usepackage{inputenc, fontspec, amsmath, amsthm, amsfonts, polyglossia, catchfile}
\usepackage[a4paper, margin=50pt, includeheadfoot]{geometry} % set page margins

% style
\AddToHook{cmd/section/before}{\clearpage}	% Add line break before section
\linespread{1.5}
\setcounter{secnumdepth}{0}		% Remove default number tags from sections
\setmainfont{Libertinus Serif}
\setsansfont{Libertinus Sans}
\setmonofont{Libertinus Mono}
\setdefaultlanguage{hebrew}
\setotherlanguage{english}

% operators
\DeclareMathOperator\cis{cis}
\DeclareMathOperator\Sp{Sp}
\DeclareMathOperator\tr{tr}
\DeclareMathOperator\im{Im}
\DeclareMathOperator\diag{diag}
\DeclareMathOperator*\lowlim{\underline{lim}}
\DeclareMathOperator*\uplim{\overline{lim}}

% commands
\renewcommand\qedsymbol{\textbf{משל}}
\newcommand{\NN}[0]{\mathbb{N}}
\newcommand{\ZZ}[0]{\mathbb{Z}}
\newcommand{\QQ}[0]{\mathbb{Q}}
\newcommand{\RR}[0]{\mathbb{R}}
\newcommand{\CC}[0]{\mathbb{C}}
\newcommand{\getenv}[2][] {
  \CatchFileEdef{\temp}{"|kpsewhich --var-value #2"}{\endlinechar=-1}
  \if\relax\detokenize{#1}\relax\temp\else\let#1\temp\fi
}
\newcommand{\explain}[2] {
	\begin{flalign*}
		 && \text{#2} && \text{#1}
	\end{flalign*}
}

% headers
\getenv[\AUTHOR]{AUTHOR}
\author{\AUTHOR}
\date\today

\usepackage{tikz}
\DeclareMathOperator\arcsinh{arcsinh}
\title{פתרון מטלה 8 – חשבון אינפיניטסימלי 2 (80132)}

\begin{document}
\maketitle
\maketitleprint{}

\Question{}
תהי $f : \RR \to \RR$ המוגדרת על־ידי
\[
	f(x) = \begin{cases}
		x^2 \sin(\frac{1}{x^2}) & x \ne 0 \\
		0 & x = 0
	\end{cases}
\]
נוכיח כי $f$ גזירה בקטע הסגור $[0, 1]$ אך לא ליפשיצית בקטע זה.
\begin{proof}
	ברור כי $f$ גזירה בכל הנקודות שהן לא $x = 0$, ולכן נבדוק את הנגזרת בנקודה זו בלבד:
	\[
		\lim_{x \to 0^+} \frac{f(x) - f(0)}{x - 0}
		= \lim_{x \to 0^+} x \sin(\frac{1}{x^2})
		= 0
	\]
	ומצאנו כי $f'(0) = 0$.

	נניח בשלילה כי $f$ היא $M$־ליפשיצית, לכן בפרט מתקיים
	\[
		|f(\frac{1}{\sqrt{2 \pi k}}) - f(\frac{1}{\sqrt{2 \pi k + \frac{1}{2} \pi}})| \le M | \frac{1}{\sqrt{2 \pi k}} - \frac{1}{\sqrt{2 \pi k + \frac{1}{2} \pi}}|
	\]
	דהינו
	\[
		\frac{1}{2 \pi k + \frac{1}{2} \pi} \le M \frac{1}{2} \pi \cdot \frac{1}{\sqrt{4 \pi^2 k^2 + \pi^2 k}}
	\]
	ונוכל לראות שעבור $k$ עולים, גם $M$ נחסם מלמטה על־יד מספרים הולכים וגדלים, ולכן נוכל להסיק כי $M$ סופי כזה לא קיים והפונקציה איננה ליפשיצית.
\end{proof}

\Question{}
תהי $f : [-1, 1] \to \RR$ המוגדרת על־ידי
\[
	f(x) = \begin{cases}
		0 & x \in [-1, 1] \setminus \{ \frac{1}{n} \mid n \in \NN \} \\
		1 & x \in \{ \frac{1}{n} \mid n \in \NN \}
	\end{cases}
\]

\Subquestion{}
נוכיח כי $f$ אינטגרבילית ב־$[-1, 1]$ ונחשב את $\int_{-1}^{1} f(t)\ dt$.

נשים לב כי בקטע $[0, 1]$ הפונקציה שווה לפונקציה משאלה 3 במטלה 7, ולכן נוכל להסיק כי היא אינטגרבילית וכי ערך האינטגרל שלה בקטע הוא $0$. \\*
עוד נראה כי $\forall x \in [-1, 0] : f(x) = 1$ ולכן נוכל להסיק כי גם $\int_{-1}^{0} f(x)\ dx = 1$, ונקבל מאדיטיביות האינטגרל
\[
	\int_{-1}^{1} f(x)\ dx
	= \int_{-1}^{0} f(x)\ dx + \int_{0}^{1} f(x)\ dx
	= 1
\]

\Subquestion{}
תהי $F : [-1, 1] \to \RR$ המוגדרת על־ידי
\[
	F(x) = \int_{-1}^{x} f(t)\ dt
\]

\subsubsection{i.}
נראה כי הפונקציה מוגדרת היטב, שכן מצאנו כי הפונרציה $f$ היא אינטגרבילית, ולכן נוכל להסיק כי $F$ מוגדרת בכל התחום.

\subsubsection{ii.}
נחשב נוסחה מפורשת ל־$F$:
\[
	F(x) = \begin{cases}
		-x & -1 \le x \le 0 \\
		1 & 0 < x \le 1
	\end{cases}
\]
על־פי הסעיפים הקודמים.

\subsubsection{iii.}
נוכיח כי $f$ בעלת נקודת אי־רציפות מסוג שני בנקודה $0$.
\begin{proof}
	אנו יודעים כבר כי
	\[
		\lim_{x \to 0^-} f(x)
		= \lim_{x \to 0^-} 1
		= 1
	\]
	ואילו נגדיר ${(a_n)}_{n = 1}^\infty, {(b_n)}_{n = 1}^\infty$ על־ידי
	\[
		a_n = \frac{1}{n},
		\qquad
		b_n = \frac{\sqrt{2}}{n}
	\]
	אז נקבל כי $f(a_n) \xrightarrow{n \to \infty} 1$ ואילו $f(b_n) \xrightarrow{n \to \infty} 0$ ולכן נוכל להסיק כי $f$ כלל לא רציפה מימין בנקודה.
\end{proof}
נבחין כי $F$ איננה גזירה ב־$x = 0$, זאת נסיק ישירות מנוסחתה המפורשת שמצאנו.

\Question{}
נשתמש בסכומי רימן ונחשב את הערך של
\[
	\lim_{n \to \infty} \left( \sum_{k = 1}^{n} \frac{k^p}{n^{p + 1}} \right)
\]
עבור $0 < p \in \RR$ כלשהו.

נבחין כי מתקיים
\[
	\sum_{k = 1}^{n} \frac{k^p}{n^{p + 1}}
	= \frac{1}{n} \sum_{k = 1}^{n} \frac{k^p}{n^p}
	= \frac{1}{n} \sum_{k = 1}^{n} {(\frac{k}{n})}^p
\]
דהינו, מצאנו כי זהו סכום רימן על החלוקה השווה של $[0, 1]$ של $x^p$, ולכן נוכל להסיק כי
\[
	\lim_{n \to \infty} \left( \sum_{k = 1}^{n} \frac{k^p}{n^{p + 1}} \right)
	= \int_{0}^{1} x^p\ dx
	= \frac{1}{p + 1}
\]

\Question{}
תהי $f : \RR \to \RR$ פונקציה אינטגרבילית על כל תת־קטע סגור וחסום של $\RR$ המקיימת $\int_{a}^{a + 1} f(x)\ dx = 0$ לכל $a \in \RR$.

\Subquestion{}
נמצא דוגמה ל־$f$ אי־שלילית המקיימת את הנתון ואיננה מתאפסת באף קטע של $\RR$.

נבחר
\[
	f(x) = \begin{cases}
		0 & x \in \QQ \\
		\cos^2 x & x \in \RR \setminus \QQ
	\end{cases}
\]
אז מהתרגילים הקודמים אנו יודעים כי $\int_{a}^{b} f(x)\ dx = 0$ לכל $a \le b$, וכמובן אין קטע חלקי ל־$\RR$ בו $f$ היא זהותית אפס.

\Subquestion{}
נמצא דוגמה ל־$f$ רציפה המקיימת את הנתון ושאיננה פונקציית האפס.

נגדיר $f(x) = \sin(\pi x)$, ולכן
\[
	\int_{a}^{a + 1} \sin(\pi x)\ dx
	= -\frac{1}{\pi} \cos(\pi x) \mid_a^{a + 1}
	= - \frac{1}{\pi} ( \cos(\pi a + \pi) - \cos(\pi a))
	= 0
\]
וכמובן שהפונקציה רציפה ולא אפס.

\Subquestion{}
נוכיח כי לא תיתכן $f$ רציפה המקיימת את הנתון ובנוסף $f(x) \ne 0$ לכל $x \in \RR$.
\begin{proof}
	מערך הביניים נוכל להסיק כי $f(x) < 0$ לכל $x \in \RR$ או $f(x) > 0$ לכל $x \in \RR$, על־ידי בחירת השלילי לפונקציה נוכל להניח כי $\forall x \in \RR : f(x) > 0$. \\*
	יהי $a \in \RR$, אז בקטע הסגור $[a, a + 1]$ מוויירשטראס השני נסיק כי ישנו מינימום $m > 0$ עבורו $f(x) > m$ לכל $x$ בתחום. \\*
	עתה נבחן את $h(x) = f(x) - m$, היא כמובן חיובית בתחום, ולכן נסיק $\int_{a}^{a + 1} h(x)\ dx > 0$, אילו נניח בשלילה $\int_{a}^{a + 1} f(x)\ dx = 0$ אז נקבל סתירה ללינאריות האינטגרל. \\*
	לכן לא יתכן כי $f$ עומדת בתנאים.
\end{proof}

\Subquestion{}
נמצא דוגמה לפונקציה המקיימת את הנתון כך שבנוסף גם $f(x) \ne 0$ לכל $x \in \RR$.

נגדיר
\[
	f(x) = \begin{cases}
		\cos(\pi x) & x \in \RR \setminus \{ k \mid k \in \NN \} \\
		1 & x \in \{ k \mid k \in \NN \} \\
	\end{cases}
\]
נבחין כי הפונקציה אכן לא מתאפסת באף נקודה, היא אינטגרבילית בכל קטע סגור וחסום שכן היא בעצמה חסומה, \\*
וכמובן בכל קטע $[a, a + 1]$ היא זהה ל־$\cos(\pi x)$ מלבד לכל היותר שתי נקודות, ולכן האינטגרלים של הפונקציות זהים, ומצאנו כי הטענה מתקיימת עבור פונקציה זו.

\Subquestion{}
נניח כי $f$ פולינום, ונוכיח כי $\forall x \in \RR : f(x) = 0$.
\begin{proof}
	פולינומים סגורים לגזירה ולכן גם לאינטגרציה, ונוכל להניח כי ל־$f$ פונקציה קדומה $F$ אשר היא בעצמה גם פולינום, ונוכיח כי אם $F(a + 1) - F(a) = 0$ לכל $a \in \RR$ אז בהכרח $F(x) = 0$. \\*
	יהי $k$ מעלת הפולינום, אז $\lim_{x \to \infty} \frac{f(x)}{x^k} = b$, עבור $b \in \RR$ כלשהו, אילו $b \ne 0$ נקבל כי הפונקציה חיובית לחלוטין או שלילית לחלוטין לכמעט כל $x$, ולכן נסיק $b = 0$. \\*
	נקבל כי מעלת הפולינום היא $k - 1$, וכך נוכל להראות באינדוקציה כי $k = 0$ וכמובן $F(x) = 0$.
\end{proof}

\Question{}
נמצא את תחום הגזירות של הפונקציה
\[
	G(x) = \int_{\sin x}^{x^2} e^{t^2}\ dt
\]
ונמצא ביטוי מפורש ללא סימן אינטגרל לנגזרת של $G$.

אנו יודעים כי האינטגרל של פונקציה הוא תמיד רציף, ואנו יודעים כי $e^{t^2}$ חיובית רציפה, לכן נוכל להסיק משיקולי חלוקה כי גם האינטגרל שלה הוא חיובי, וכמובן גזיר ל־$e^{t^2}$ עצמה. \\*
עבור $x > 1$ נקבל $x^2 > \sin x$ ולכן נוכל להסיק כי $G$ גזירה, ובאופן דומה נקבל כי היא גזירה אף ב־$x < 0$, נותר לבדוק את $x = 0$ עצמה. \\*
נבחין כי הנגזרת של $\int e^{t^2}\ dt$ שואפת ל־$1$ משני הצדדים, לכן מספיק שנבחר $0 < x < 1$, אז $\sin x > x^2$ ונקבל כי הסימן מתהפך, והפונקציה לא תהיה גזירה בנקודות $x = 0, 1$.

נגדיר $F(t) = \int e^{t^2}\ dt$, ולכן $F'(t) = e^{t^2}$, נראה כי $G(x) = F(x^2) - F(\sin x)$, ולכן נגזור לפי כלל השרשרת ונקבל
\[
	G'(x) = 2x F'(x^2) - \cos(x) F'(\sin(x)) = e^{x^4} - \cos(x) e^{\sin^2(x)}
\]

\Subquestion{}
\subsubsection{i.}
נגדיר
\[
	G(x) = \int_{0}^{x} \frac{dt}{1 + t^2} + \int_{0}^{1/x} \frac{dt}{1 + t^2}
\]
נוכיח כי $G$ קבועה בקרן $(0, \infty)$ ונחשב את ערכה שם.

נוכל להשתמש בשיטה דומה לסעיף הקודם ולקבל כי
\[
	G'(x) = \frac{1}{1 + x^2} + \frac{1}{1 + \frac{1}{x^2}} \cdot \frac{-1}{x^2}
	= \frac{1}{1 + x^2} - \frac{1}{1 + x^2}
	= 0
\]
מצאנו כי הנגזרת היא אפס תמיד ולכן נוכל להסיק כי הפונקציה $G$ קבועה בקטע זה.

אנו גם יודעים ש־$\arctan(x)' = \frac{1}{1 + x^2}$ ולכן נוכל להסיק כי
\[
	G(1) = 2 \int_{0}^{1} \frac{dt}{1 + t^2}
	= 2\arctan(x) \mid_0^1
	= 2\frac{\pi}{2}
\]
ידוע כי $G$ קבועה ולכן נוכל להסיק כי $G(x) = \pi$ לכל $x \in (0, \infty)$.

\subsubsection{ii.}
נוכיח כי $G$ קבועה גם בקרן $(-\infty, 0)$ וכי ערכה שם הוא $-\pi$.
\begin{proof}
	נראה כי $G(-x) = -G(x)$ מחוקי אינטגרלים, ולכן נוכל כמובן להניח כי $\forall x < 0 : G(x) = -G(-x) = -\pi$.
\end{proof}

\Question{}
\Subquestion{}
נוכיח כי לכל $-1 < x \in \RR$ מתקיים
\[
	\ln(1 + x) = \int_{0}^{x} \frac{1}{1 + t}\ dt
\]
\begin{proof}
	נשתמש במשפט היסודי של החשבון האינפינטסמלי, ונקבל כי
	\[
		\ln(x + 1)' = \frac{1}{1 + x}
	\]
	ולכן מאינטגרציה לשני האגפים נקבל
	\[
		\ln(x + 1) + C = \int \frac{dt}{1 + t}
	\]
	נציב כמובן $[0, x]$ בקטע ונקבל
	\[
		\ln(x + 1) - \ln(1) = \int_{0}^{x} \frac{dt}{1 + t}
	\]
\end{proof}
נבחין כי ערך הסכום
\[
	\lim_{n \to \infty} \left( \sum_{k = 1}^{n} \frac{1}{n + k} \right)
	= \lim_{n \to \infty} \left( \frac{1}{n} \sum_{k = 1}^{n} \frac{1}{1 + \frac{k}{n}} \right)
	= \int_{0}^{1} \frac{dt}{1 + t}
	= \ln(2) - \ln(1)
\]

\Subquestion{} 
נוכיח כי $\forall x > -1 : \ln(1 + x) \le x$.
\begin{proof}
	נפרק למקרים, תחילה נניח כי $x > 0$ ונקבל נקבל $x + 1 > 1$ ובהתאם גם $0 < \frac{1}{1 + x} < 1$, וממונוטוניות האינטגרל נקבל גם
	\[
		\int_{0}^{x} \frac{dt}{t + 1} \le \int_{0}^{t} dt
		\implies \ln(1 + x) \le x
	\]
	נניח עתה כי $-1 < x < 0$ ונקבל גם $0 < x + 1 < 1$ ולכן $1 < \frac{1}{x + 1}$, ונקבל
	\[
		\int_{0}^{x} dt \le \int_{0}^{x} \frac{dt}{t + 1}
		\implies \ln(1 + x) \le x
	\]
\end{proof}

\Subquestion{}
נוכיח כי $\forall 0 < x : x - \frac{x^2}{2} \le \ln(1 + x)$.
\begin{proof}
	יהי $x > 0$, אז $1 - x < 1 < \frac{1}{x + 1}$ כפי שמצאנו בסעיף הקודם, ומתהליך זהה לסעיף הקודם נקבל ישירות כי $x - \frac{x^2}{2} \le \ln(1 + x)$.
\end{proof}

\Question{}
נחשב את האינטגרלים הבאים:

\Subquestion{}
\[
	\int \frac{\sin x}{1 + \cos^2 x} dx
\]
נשתמש בשיטת ההצבה עבור $t = \cos x, dt = -\sin x\ dx$ ונקבל
\[
	-\int \frac{dt}{1 + t^2} = -\arctan(t)
	= -\arctan(\cos x)
\]

\Subquestion{}
\[
	\int x {(x + 1)}^{95}\ dx
\]
נגדיר $t = x - 1, dt = dx$ ונקבל
\[
	\int (t - 1) t^{95}\ dt = \int t^{96} - t^{95}\ dt
	= \frac{1}{97} t^{97} - \frac{1}{96} t^{96}
	= \frac{1}{97} {(x - 1)}^{97} - \frac{1}{96} {(x - 1)}^{96}
\]

\Subquestion{}
\[
	\int \frac{1}{e^x + 1} dx
\]
נגדיר $t = e^x, dt = e^x dx$ ולכן נקבל
\[
	\int \frac{1}{t + 1} \cdot \frac{dt}{t}
	\int \frac{1}{\frac{1}{t} + 1} \cdot \frac{dt}{t^2}
\]
נגדיר גם $u = \frac{1}{t}, du = \frac{-dt}{t^2}$ ולכן
\[
	\int \frac{1}{u + 1} du = \ln(u + 1)
	= \ln(\frac{1}{t} + 1)
	= \ln(\frac{1}{e^x} + 1)
\]

\Subquestion{}
\[
	\int \sin^5(x) dx
	= \int {(1 - \cos^2 x)}^2 \sin x\ dx
\]
נגדיר $\cos x = t, -\sin x dx = dt$ ונקבל
\[
	-\int {(1 - t^2)}^2\ dt
	= \int -t^4 + 2t^2 - 1\ dt
	= \frac{-t^5}{5} + \frac{2t^3}{3} - 1
	= \frac{-\cos^5(x)}{5} + \frac{2\cos^3(x)}{3} - 1
\]

\Subquestion{}
\[
	\int \frac{dx}{\sqrt{x} (x + 1)}
\]
נגדיר $t = \sqrt{x}, dt = \frac{1}{2 \sqrt{x}}dx$ ונקבל
\[
	\int \frac{2dt}{1 + t^2}
	= 2 \arctan(t)
	= 2 \arctan(\sqrt{x})
\]

\Subquestion{}
\[
	\int \arctan(x)\ dx
	= x \arctan(x) - \int \frac{x}{1 + x^2}\ dx
\]
נגדיר $t = x^2, dt = 2x\ dx$ ונקבל
\[
	x \arctan(x) - \frac{1}{2} \int \frac{dt}{1 + t}
	= x \arctan(x) - \frac{1}{2}\ln(1 + t)
	= x \arctan(x) - \frac{1}{2}\ln(1 + x^2)
\]

\end{document}
