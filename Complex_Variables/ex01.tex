\documentclass[a4paper]{article}

% packages
\usepackage{inputenc, amsmath, amsthm, thmtools, amsfonts, amssymb, luacode, catchfile, tikzducks, hyperref}
\usepackage[a4paper, margin=50pt, includeheadfoot]{geometry} % set page margins
\usepackage[shortlabels]{enumitem}
\usepackage[skip=3pt, indent=0pt]{parskip}

% language
\usepackage[bidi=basic, layout=tabular, provide=*]{babel}
\babelprovide[main, import]{hebrew}
\babelprovide{rl}
\babelfont{rm}{Libertinus Serif}
\babelfont{sf}{Libertinus Sans}
\babelfont{tt}{Libertinus Mono}

% style
\AddToHook{cmd/section/before}{\clearpage}	% Add line break before section
\linespread{1.3}
\setcounter{secnumdepth}{0}		% Remove default number tags from sections, this won't do well with theorems
\AtBeginDocument{\setlength{\belowdisplayskip}{3pt}}
\AtBeginDocument{\setlength{\abovedisplayskip}{3pt}}

% operators
\DeclareMathOperator\cis{cis}
\DeclareMathOperator\Sp{Sp}
\DeclareMathOperator\tr{tr}
\DeclareMathOperator\im{Im}
\DeclareMathOperator\re{Re}
\DeclareMathOperator\diag{diag}
\DeclareMathOperator*\lowlim{\underline{lim}}
\DeclareMathOperator*\uplim{\overline{lim}}
\DeclareMathOperator\rng{rng}
\DeclareMathOperator\Sym{Sym}
\DeclareMathOperator\Arg{Arg}
\DeclareMathOperator\Log{Log}
\DeclareMathOperator\dom{dom}

% commands
%\renewcommand\qedsymbol{\textbf{מש''ל}}
%\renewcommand\qedsymbol{\fbox{\emoji{lizard}}}
\newcommand{\NN}[0]{\mathbb{N}}
\newcommand{\ZZ}[0]{\mathbb{Z}}
\newcommand{\QQ}[0]{\mathbb{Q}}
\newcommand{\RR}[0]{\mathbb{R}}
\newcommand{\CC}[0]{\mathbb{C}}
\newcommand{\FF}[0]{\mathbb{F}}
\newcommand{\PP}[0]{\mathbb{P}}
\newcommand{\TT}[0]{\mathbb{T}}
\newcommand{\acts}[0]{\circlearrowright}
\newcommand{\explain}[2] {
	\begin{flalign*}
		 && \text{#2} && \text{#1}
	\end{flalign*}
}
\newcommand{\maketitleprint}[0]{ \begin{center}
	\begin{tikzpicture}[scale=3]
		\duck[graduate=gray!20!black, tassel=red!70!black]
	\end{tikzpicture}	
\end{center}
}

% theorem commands
\newtheoremstyle{c_remark}
	{}	% Space above
	{}	% Space below
	{}% Body font
	{}	% Indent amount
	{\bfseries}	% Theorem head font
	{}	% Punctuation after theorem head
	{.5em}	% Space after theorem head
	{\thmname{#1}\thmnumber{ #2}\thmnote{ \normalfont{\text{(#3)}}}}	% head content
\newtheoremstyle{c_definition}
	{3pt}	% Space above
	{3pt}	% Space below
	{}% Body font
	{}	% Indent amount
	{\bfseries}	% Theorem head font
	{}	% Punctuation after theorem head
	{.5em}	% Space after theorem head
	{\thmname{#1}\thmnumber{ #2}\thmnote{ \normalfont{\text{(#3)}}}}	% head content
\newtheoremstyle{c_plain}
	{3pt}	% Space above
	{3pt}	% Space below
	{\itshape}% Body font
	{}	% Indent amount
	{\bfseries}	% Theorem head font
	{}	% Punctuation after theorem head
	{.5em}	% Space after theorem head
	{\thmname{#1}\thmnumber{ #2}\thmnote{ \text{(#3)}}}	% head content

\theoremstyle{c_plain}
\newtheorem{theorem}{משפט}[section]
\newtheorem{lemma}[theorem]{למה}
\newtheorem{proposition}[theorem]{טענה}
\newtheorem*{proposition*}{טענה}
%\newtheorem{corollary}[theorem]{אין חלופה עברית}

\theoremstyle{c_definition}
\newtheorem{definition}[theorem]{הגדרה}
\newtheorem*{definition*}{הגדרה}
\newtheorem{example}{דוגמה}[section]
\newtheorem{exercise}{תרגיל}[section]

\theoremstyle{c_remark}
\newtheorem*{remark}{הערה}
\newtheorem*{solution}{פתרון}
\newtheorem{conclusion}[theorem]{מסקנה}
\newtheorem{notation}[theorem]{סימון}

% Questions related commands
\newcounter{question}
\setcounter{question}{1}
\newcounter{sub_question}
\setcounter{sub_question}{1}

\newcommand{\question}[1][0]{
	\ifthenelse{#1 = 0}{}{\setcounter{question}{#1}}
	\subsection{שאלה \arabic{question}}
	\addtocounter{question}{1}
	\setcounter{sub_question}{1}
}

\newcommand{\subquestion}[1][0]{
	\ifthenelse{#1 = 0}{}{\setcounter{sub_question}{#1}}
	\subsubsection{סעיף \localecounter{letters.gershayim}{sub_question}}
	\addtocounter{sub_question}{1}
}

% import lua and start of document
\directlua{common = require ('../common')}

\GetEnv{AUTHOR}

% headers
\author{\AUTHOR}
\date\today

\title{פתרון מטלה 01 --- פונקציות מרוכבות, 80519}

\begin{document}
\maketitle
\maketitleprint{}

\Question{}
נצייר את הקבוצות הבאות במישור המרוכב:

\Subquestion{}
$\{z \in \CC \mid 1 \le |z - 1| < 2\}$. \\*
זהו למעשה עיגול שמרכזו הוא $(1, 0)$ שרדיוסו 2 לא כולל, ללא עיגול פנימי ברדיוס 1.

\Subquestion{}
$\{z \in \CC \mid \frac{\pi}{3} < \text{Arg}(z) \le \frac{\pi}{2} \}$. \\*
זוהי קרן חסומה על־ידי $y = x$ מצד אחד ו־$y = \arctan(\frac{\pi}{3})x$, בכיוון החיובי.

\Subquestion{}
$\{ z \in \CC \mid 2 < \im(z) < 4 \}$. \\*
שקול ל־$2 < y < 4$ לכל $x$.

\Subquestion{}
$\{ z \in \CC \mid |z^2 + 1| \ge |z + i| \}$. \\*
נחשב
\[
	|z^2 + 1| \ge |z + i|
	\iff |z + i| \cdot |z - i| \ge |z + i|
	\iff |z - i| \ge 1
	\iff \sqrt{x^2 + {(y - 1)}^2} \ge 1
\]
דהינו זהו כל המישור למעט עיגול פתוח שמרכזו $(0, 1)$ ורדיוסו 1.

\Subquestion{}
$A = \{z \in \CC \mid \frac{1}{z} = \overline{z}\}$. \\*
נבחין כי $0 \notin A$, ועתה נוכל לקבוע כי
\[
	\frac{1}{z} = \overline{z}
	\iff z \overline{z} = 1
	\iff x^2 + y^2 = 1
\]
ולכן זהו מעגל היחידה, ובכל מקרה $0$ לא במעגל היחידה.

\Question{}
\Subquestion{}
המספר $z_1 = -1 + i$ נמצא ב־$(-1, 1) \in \RR^2$, ומתקיים $\re(z_1) = -1, \im(z_1) = 1, |z_1| = \sqrt{2}, \text{Arg}(z_1) = \frac{3\pi}{4}$.

\Subquestion{}
המספר $z_2 = 4 \exp(\frac{4\pi}{3} i)$ נמצא ב־$(4 \cdot \cos(\frac{4\pi}{3}), 4 \cdot \sin(\frac{4\pi}{3}))$,
ומתקיים $\re(z_2) = 4 \cdot \cos(\frac{4\pi}{3}), \im(z_2) = 4 \cdot \sin(\frac{4\pi}{3}), |z_2| = 4, \text{Arg}(z_2) = \frac{4\pi}{3}$.

\Subquestion{}
המספר $z_3 = 7i$ נמצא ב־$(0, 7)$,
ומתקיים $\re(z_3) = 0, \im(z_3) = 7, |z_3| = 7, \text{Arg}(z_2) = \frac{\pi}{2}$.

\Subquestion{}
המספר $z_4 = e^2$ נמצא ב־$(e^2, 0)$,
ומתקיים $\re(z_4) = e^2, \im(z_4) = 0, |z_4| = e^2, \text{Arg}(z_2) = 0$.

\Subquestion{}
המספר $z_5 = (\frac{\sqrt{2}}{2} + \frac{\sqrt{2}}{2}i)(\frac{1}{2} - \frac{\sqrt{3}}{2}i) = e^{i \pi / 4} e^{-i\pi/3} = e^{i \pi / 12}$ נמצא ב־$(\cos(\pi/12), -\sin(\pi/12))$, \\*
ומתקיים $\re(z_5) = \cos(\pi/12), \im(z_5) = -\sin(\pi/12), |z_4| = 1, \text{Arg}(z_2) = -\pi/12$.

\Question{}
תהי ${(z_n)}_{n = 1}^\infty \subseteq \CC$.

\Subquestion{}
נוכיח כי $z_n \to z$ אם ורק אם $\re(z_n) \to \re(z) \land \im(z_n) \to \im(z)$.
\begin{proof}
	תחילה נגדיר $x_n = \re(z_n), y_n = \im(z_n)$.

	נניח כי $z_n \to z$.
	לכן מתקיים $\lim_{n \to \infty} |z_n - z| = 0$, אז נקבל $|x_n - x + i(y_n - y)| \to 0$ או בהתאם ${(x_n - x)}^2 + {(y_n - y)}^2 \to 0$. \\*
	אילו נניח בשלילה שלפחות אחת הסדרות לא מתכנסת, נקבל כי האינפימום הוא חיובי ובהתאם הגבול לא יכול להיות אפס, וסיימנו.

	בכיוון ההפוך נניח $x_n \to x, y_n \to y$ ולכן בהתאם גם ${(x - x_n)}^2 + {(y - y_n)}^2 \to 0$, אבל זה שקול לגבול $|z_n - z| to 0$ וקיבלנו את הסדרה המקורית.
\end{proof}

\Subquestion{}
נניח כי $z_n \to z$ ונסמן 4$z_n = r_n e^{i \theta_n}$ ובהתאם $z = r e^{i\theta}$. \\*
נוכיח כי בהכרח $r_n \to r$ ונמצא דוגמה נגדית הסותרת את הטענה $\theta_n \to \theta$.
\begin{proof}[הוכחה ($r_n \to r$)]
	מהסעיף הקודם נסיק כי אם $z_n \to z$ אז גם $|z_n| \to |z|$, אבל $|z_n| = r_n, |z| = r$ ולכן ישירות $r_n \to r$.
\end{proof}
\begin{solution}[הפרכה $\theta_n \to \theta$]
	נניח $r_n = \frac{1}{n}$, $\theta_n = n \pi$, וגם $r = 0, \theta = 1$, אז נקבל $|z_n - z| = |\frac{1}{n} e^{i\pi n} - 0| = \frac{1}{n} \to 0$ וקיבלנו את ההתכנסות $z_n \to z$ למרות שברור כי $\theta_n$ לא מתכנסת כלל.
\end{solution}

\Question{}
נמצא נוסחה סגורה לשני הסכומים הבאים עבור $\theta \in \RR$ ו־$N \in \NN$.

\Subquestion{}
\[
	\sum_{n = 0}^N \cos(n \theta)
	= \sum_{n = 0}^N \re(e^{in\theta})
	= \re( \sum_{n = 0}^N {(e^{i\theta})}^n )
	= \re (\frac{e^{i\theta (N + 1)} - 1^N}{e^{i\theta} - 1})
\]
מצאנו נוסחה סגורה.

\Subquestion{}
נפעל באופן דוגמה ונקבל
\[
	\sum_{n = 0}^N \sin(n \theta)
	= \im (\frac{e^{i\theta (N + 1)} - 1}{e^{i\theta} - 1})
\]

\Question{}
לכל $c \in \CC$ נגדיר $f_c(z) = z^2 + c$, והגדרנו את קבוצת מנדלברוט על־ידי $M = \{ c \in \CC \mid {(f_c^n(0))}_{n = 1}^\infty \text{ is bound}\}$.

\Subquestion{}
נוכיח כי לכל $z \in \CC$ ו־$\epsilon > 0$ כך ש־$|z| \ge \max\{ 2 + \epsilon, |c|\}$ מתקיים $|f_c(z)| \ge (1 + \epsilon)|z|$.
\begin{proof}
	נשתמש בהגדרת $|z|$ ונקבל
	\[
		|f_c(z)| = |z^2 + c| \ge |z^2| - |c| = |z| \cdot |z| - |c| \ge |z| (2 + \epsilon) - |c|
		= |z| (1 + \epsilon) + |z| - |c| \ge |z| (1 + \epsilon)
	\]
\end{proof}

\Subquestion{}
נוכיח כי $M \subseteq \overline{B}(0, 2) \subseteq \CC$.
\begin{proof}
	יהי $c \notin \overline{B}(0, 2)$, אז בהכרח $|c| > 2$. \\*
	מחישוב ישיר $z_1 = f_c^1(0) = 0^2 + c$ ולכן גם $|z_1| > 2$.
	בהתאם לטענת סעיף א' מתקיים שאם $z_2 = f_c^2(0)$ אז $|z_2| \ge |z_1|(1 + \epsilon)$ עבור $\epsilon = \frac{1}{2}$, דהינו $|z_2| \ge |z_1| \frac{3}{2}$. \\*
	נגדיר סדרה ${(z_n)}_{n = 1}^\infty \subseteq \CC$ על־ידי תהליך זה ונקבל שגם $|z_{n + 1}| \ge \frac{3}{2} |z_n|$ ולכן נוכל להסיק כי $z_n \to \infty$, ובהתאם $c \notin M$. \\*
	לכן נוכל להסיק כי אם $c \in M$ אז גם $c \in \overline{B}(0, 2)$.
\end{proof}

\Subquestion{}
לכל $n \in \NN$ נגדיר $F_n = \{ c \in \CC \mid |f_c^n(0) \le 2 \}$, ונגדיר $F = \bigcap_{n \in N} F_n$. \\*
נוכיח כי $M = F$.
\begin{proof}
	תחילה נבחין כי $M \supseteq F$ שכן כל איבר בקבוצה זו מהגדרתה חסום לכל $n \in \NN$ וסיימנו.

	נראה כי $M \subseteq F$. \\*
	יהי $c \in M$, אז כמובן $c \in F_1$ מסעיף ב'.
	נניח כי $m \in \NN$ המינימלי כך ש־$c \notin F_m$.
	אז נקבל $f_c^m(0) > 2$, אבל שוב מסעיף ב' וההנחה נקבל כי $f_c^{m - 1}(0) \in \overline{B}(0, 2)$ ולכן גן $f_c^m(0) \in \overline{B}(0, 2)$, וזו כמובן סתירה, לכן אין $m$ כזה, ובהתאם $c \in F_n$ לכל $n \in \NN$. \\*
	נסיק כמובן ש־$c \in F$ ולכן $M \subseteq F$.
\end{proof}
נבחין כי חיתוך של קבוצות סגורות וקומפקטיות אף הוא חסום וקומפקטי ולכן $F$ קומפקטית, אבל $M = F$ ולכן קיבלנו כי $M$ קומפקטית.

\Question{}
\Subquestion{}
נוכיח כי כל ישר או מעגל ב־$\CC$ נשלח למעגל ב־$\mathbb{S}^2$ תחת ההעתקה ההופכית להטלה הסטריאוגרפית המוגדרת על־ידי
\[
	\varphi(z) = \left( \frac{2 \re(z)}{1 + {|z|}^2}, \frac{2 \im(z)}{1 + {|z|}^2}, \frac{-1 + {|z|}^2}{1 + {|z|}^2} \right)
\]
\begin{proof}
	יהי $c = a \re(z) + b \im(z)$ ישר כלשהו ב־$\CC$, נבחר שלוש נקודות שונות על ישר זה ונבחן את המישור שהן יוצרות לאחר הטלה סטריאוגרפית, נגדיר מישור זה להיות $P_C$.
	אבל מטענה מהכיתה נובע כי החיתוך $\mathbb{S}^2 \cap P_C$ הוא ישר או מעגל בלבד ב־$\CC$, ומהגדרתו שלוש נקודות שונות של הישר שהגדרנו מונחות עליו, ולכן לא יתכן שהוא מעגל.
	נסיק אם כן שהוא ישר, ומהתלכדות שלוש נקודות שונות בו נסיק ששני הישרים שווים.
	נשים לב כי יכולנו לבחור שתי נקודות בלבד ולבחור את הנקודה $\infty \in \CC*$ כנקודה שלישית, שני ישירם זהים אם שתי נקודות שלהם מתלכדות ולכן טענה זו עדיין חלה.

	נעבור אם כך לבחינת מעגל ב־$\CC$, גם הפעם עבור נעגל נבחר שלוש נקודות ונגדיר מישור $P_C \subseteq \RR^3$ המוגדר על־ידי הנקודות הללו אחרי העתקה סטריאוגרפית.
	גם הפעם אנו יודעים כי $\mathbb{S}^2 \cap P_C$ מעגל ב־$\RR^3$ וגם הפעם נקבל שהוא משליך מעגל או ישר על המישור המרוכב, אבל שלוש הנקודות שבחרנו פוסלות את היותו ישר ומקבעים מעגל יחיד, שכן מעגל נוצר ביחידות על־ידי שלוש נקודות.
\end{proof}

\Subquestion{}
נגדיר $\rho : \CC \to \RR_+$ המטריקה המושרית מההעתקה הסטריאוגרפית, ידוע כי
\[
	\rho(z, w) = \frac{2|z - w|}{\sqrt{(1 + {|z|}^2)(1 + {|w|}^2)}}
\]
נוכיח כי אם ${(z_n)}_{n = 1}^\infty \subseteq \CC$ אז $z_n \to z$ עבור $z \in \CC$ אם ורק אם $\rho(z_n, z) \to 0$.
\begin{proof}
	נבחין כי $\sqrt{(1 + {|z|}^2)(1 + {|w|}^2)} \ge 1$ לכל $z, w \in \CC$. \\*
	נניח כי $z_n \to z$, לכן לפי ההגדרה $|z - z_n| \to 0$, וכן גם $2|z - z_n| \to 0$ מאריתמטיקה, לכן
	\[
		0 \le \rho(z_n, z) \le 2|z_n - z|
	\]
	לכל $n \in \NN$ ובהתאם נסיק ש־$\rho(z_n, z) \to 0$.

	בכיוון ההפוך נניח כי $\rho(z_n, z) \to 0$ ובשלילה ש־$z_n \not\to z$.
	קיים אם כך $\epsilon > 0$ כך ש־$|z - z_n| \ge \epsilon$ היא תכונה שכיחה על $n$, ובהתאם נקבל
	\[
		\rho(z_n, z) = \frac{2|z_n - z|}{\sqrt{(1 + {|z_n|}^2)(1 + {|z|}^2)}}
		\ge \frac{2\epsilon}{\sqrt{(1 + {|z_n|}^2)(1 + {|z|}^2)}}
		\ge \frac{2\epsilon}{\sqrt{(1 + \sup {|z_n|}^2)(1 + {|z|}^2)}}
	\]
	ובהתאם קיבלנו חסם תחתון חיובי למרחק ובהתאם קיבלנו סתירה ל־$\rho(z_n, z) \to 0$.
\end{proof}

\Subquestion{}
נוכיח כי לכל $z, w \in \CC$ מתקיים $\rho(z, w) = \rho(\overline{z}, \overline{w}) = \rho(\frac{1}{z}, \frac{1}{w})$.
\begin{proof}
	נוכיח טענה כללית יותר תחילה, נוכיח כי לכל $z \in \CC$ מתקיים $|z| = |\overline{z}|$.
	נזכור כי הגדרנו $|z| = \sqrt{z \cdot \overline{z}}$, וראינו בכיתה כי $\overline{\overline{z}} = z$ ולכן נסיק כי $|\overline{z}| = \sqrt{\overline{z} \cdot z} = |z|$.
	לכן בהכרח גם
	\[
		\rho(\overline{z}, \overline{w})
		= \frac{2|\overline{z} - \overline{w}|}{\sqrt{(1 + {|\overline{z}|}^2)(1 + {|\overline{w}|}^2)}}
		= \frac{2|\overline{z - w}|}{\sqrt{(1 + {|z|}^2)(1 + {|w|}^2)}}
		= \frac{2|z - w|}{\sqrt{(1 + {|z|}^2)(1 + {|w|}^2)}}
		= \rho(z, w)
	\]
	נעבור לבדיקת החלק האחרון בשוויון:
	\begin{align*}
		 \rho(\frac{1}{z}, \frac{1}{w})
		& = \frac{2|\frac{1}{z} - \frac{1}{w}|}{\sqrt{(1 + {|\frac{1}{z}|}^2)(1 + {|\frac{1}{w}|}^2)}}
		& = \frac{2|\frac{z - w}{zw}|}{\sqrt{(1 + \frac{1}{{|z|}^2})(1 + \frac{1}{{|w|}^2}|)}} \\
		& = \frac{2|z - w|}{\sqrt{{|zw|}^2 (1 + \frac{1}{{|z|}^2})(1 + \frac{1}{{|w|}^2}|)}}
		& = \frac{2|z - w|}{\sqrt{(1 + {|z|}^2)(1 + {|w|}^2)}} \\
		& = \rho(z, w)
	\end{align*}
	ומצאנו כי השוויון אכן מתקיים.
\end{proof}

\end{document}
