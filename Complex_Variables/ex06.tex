\documentclass[a4paper]{article}

% packages
\usepackage{inputenc, fontspec, amsmath, amsthm, amsfonts, polyglossia, catchfile}
\usepackage[a4paper, margin=50pt, includeheadfoot]{geometry} % set page margins

% style
\AddToHook{cmd/section/before}{\clearpage}	% Add line break before section
\linespread{1.5}
\setcounter{secnumdepth}{0}		% Remove default number tags from sections
\setmainfont{Libertinus Serif}
\setsansfont{Libertinus Sans}
\setmonofont{Libertinus Mono}
\setdefaultlanguage{hebrew}
\setotherlanguage{english}

% operators
\DeclareMathOperator\cis{cis}
\DeclareMathOperator\Sp{Sp}
\DeclareMathOperator\tr{tr}
\DeclareMathOperator\im{Im}
\DeclareMathOperator\diag{diag}
\DeclareMathOperator*\lowlim{\underline{lim}}
\DeclareMathOperator*\uplim{\overline{lim}}

% commands
\renewcommand\qedsymbol{\textbf{משל}}
\newcommand{\NN}[0]{\mathbb{N}}
\newcommand{\ZZ}[0]{\mathbb{Z}}
\newcommand{\QQ}[0]{\mathbb{Q}}
\newcommand{\RR}[0]{\mathbb{R}}
\newcommand{\CC}[0]{\mathbb{C}}
\newcommand{\getenv}[2][] {
  \CatchFileEdef{\temp}{"|kpsewhich --var-value #2"}{\endlinechar=-1}
  \if\relax\detokenize{#1}\relax\temp\else\let#1\temp\fi
}
\newcommand{\explain}[2] {
	\begin{flalign*}
		 && \text{#2} && \text{#1}
	\end{flalign*}
}

% headers
\getenv[\AUTHOR]{AUTHOR}
\author{\AUTHOR}
\date\today

\title{פתרון מטלה 06 --- פונקציות מרוכבות, 80519}

\begin{document}
\maketitle
\maketitleprint{}

\question{}
נחשב את האינטגרלים המסילתיים הנתונים.

\subquestion{}
\[
	\int_\gamma (2z - 3\overline{z} + 1)\ dz
\]
עבור $\gamma = 3 \cos t + 2i \sin t$ ב־$t \in [0, 2\pi]$.
\begin{solution}
	נתחיל בחישוב הכרחי:
	\[
		\gamma'(t) = -3 \sin t + 2i \cos t
	\]
	ונעבור לחישוב האינטגרל
	\begin{align*}
		\int_\gamma (2z - 3\overline{z} + 1)\ dz
		& = \int_0^{2\pi} (2(3 \cos t + 2i \sin t) - 3(3 \cos t - 2i \sin t) + 1)(-3 \sin t + 2i \cos t)\ dt \\
		& = \int_0^{2\pi} (-3 \cos t + 10i \sin t + 1)(-3 \sin t + 2i \cos t)\ dt \\
		& = \int_0^{2\pi} \frac{9}{2} \sin(2t) - 6 - 24i \sin^2 t - 10 \sin(2t) - 3 \sin t + 2i \cos t\ dt \\
		& = \int_0^{2\pi} - 6 - 6 {(e^{it} - e^{-it})}^2\ dt \\
		& = -6 \int_0^{2\pi} -1 + e^{2it} + e^{-2it}\ dt \\
		& = -6 {\left[ -t - \frac{i}{2} e^{2it} + \frac{i}{2} e^{-2it}\right]}_0^{2\pi}
	\end{align*}
	כאשר במעבר האחרון השתמשנו באינטגרל של פונקציה טריגונומטרית אפס בתחום.
\end{solution}

\subquestion{}
\[
	\int_\gamma \cos(\re(z))\ dz
\]
עבור $\gamma(t) = i + e^{it}$ עבור $t \in [-\pi, \pi]$.
\begin{solution}
	נובע $\gamma'(t) = i e^{it}$ ונבחין כי $\cos(\cos(t)) = -\cos(\cos(t))$ והפונקציה הזו זוגית, לכן
	\begin{align*}
		\int_\gamma \cos(\re(z))\ dz
		& = \int_{-\pi}^{\pi} \cos(\re(i + e^{it})) \cdot i e^{it}\ dt \\
		& = \int_{-\pi}^{\pi} \cos(\cos(t)) \cdot i e^{it}\ dt \\
		& = \int_{-\pi}^{\pi} \cos(\cos(t)) (i \cos t - \sin t)\ dt \\
		& = i \int_{-\pi}^{\pi} \cos(\cos(t)) \cos t\ dt - \int_{-\pi}^{\pi} \cos(\cos(t)) \sin t\ dt \\
		& = i \int_{-\pi}^{\pi} \cos(\cos(t)) \cos t\ dt - 0 \\
		& = 2i \int_{0}^{\pi} \cos(\cos(t)) \cos t\ dt \\
		& = 2i \int_{-\pi/2}^{\pi/2} \cos(\cos(u - \pi/2)) \cos(u - \pi/2)\ dt \\
		& = 2i \int_{-\pi/2}^{\pi/2} \cos(\sin(u)) \sin(u)\ dt \\
		& = 0
	\end{align*}
\end{solution}

\subquestion{}
\[
	\int_\gamma {\left(\frac{\Log(z)}{z}\right)}^2\ dz
\]
עבור $\gamma = e^{-it}$ ב־ $t \in [-\frac{\pi}{2}, \frac{\pi}{2}]$.
\begin{solution}
	הפעם $\gamma'(t) = -i e^{-it}$ ולכן
	\begin{align*}
		\int_\gamma {\left(\frac{\Log(z)}{z}\right)}^2\ dz
		& = \int_{-\frac{\pi}{2}}^{\frac{\pi}{2}} {\left(\frac{\Log(e^{-it})}{e^{it}}\right)}^2 \cdot (-i e^{-it})\ dt \\
		& = i \int_{-\frac{\pi}{2}}^{\frac{\pi}{2}} t^2 e^{-3it}\ dt \\
		& = {\left[ t^2 \frac{e^{-3it}}{-3}\right]}_{-\frac{\pi}{2}}^{\frac{\pi}{2}} + \frac{2}{3} \int_{-\frac{\pi}{2}}^{\frac{\pi}{2}} t e^{-3it}\ dt \\
		& = \frac{1}{2} \cdot \frac{1}{-3} + {\left[ t \frac{e^{-3it}}{-3i}\right]}_{-\frac{\pi}{2}}^{\frac{\pi}{2}} + \frac{2}{9i} \int_{-\frac{\pi}{2}}^{\frac{\pi}{2}} e^{-3it}\ dt \\
		& = \frac{1}{6} + 0 + \frac{1}{27 \cdot 2}
	\end{align*}
	זאת בשל חישוב ישיר, והעובדה ש־$e^{-\frac{\pi}{6}i} - e^{-\frac{\pi}{6}i} = \sin(-\frac{\pi}{6}) = -\frac{1}{2}$.
\end{solution}

\question{}
תהי $f : G \to \CC$ רציפה ותהי $\gamma : [a, b] \to G$ מסילה.

\subquestion{}
נסמן $M = \max_{t \in [a, b]} |f(\gamma(t))|$ ו־$L(\gamma)$ אורך המסילה $\gamma$. \\*
נראה כי מתקיים
\[
	\left\lvert \int_\gamma f(z)\ dz\right\rvert \le M \cdot L(\gamma)
\]
\begin{proof}
	\begin{align*}
		\left\lvert \int_\gamma f(z)\ dz \right\rvert
		& = \left\lvert \int_a^b f(\gamma(t)) \gamma'(t)\ dt \right\rvert \\
		& \le \int_a^b \left\lvert f(\gamma(t)) \gamma'(t) \right\rvert\ dt \tag{1} \\
		& = \int_a^b \left\lvert f(\gamma(t)) \right\rvert \cdot \left\lvert \gamma'(t) \right\rvert\ dt \\
		& \le \int_a^b M \cdot \left\lvert \gamma'(t) \right\rvert\ dt \tag{2} \\
		& = M \cdot \int_a^b \left\lvert \gamma'(t) \right\rvert\ dt \\
		& = M \cdot L(\gamma)
	\end{align*}
	כאשר
	\begin{enumerate}
		\item טענה ששאבנו מאינטגרלים ממשיים מרובי משתנים, ההוכחה עבור המקרה המרוכב זהה לחלוטין.
		\item מאינפי 3.
	\end{enumerate}
	ובכך הראינו כי אי־השוויון אכן חל.
\end{proof}

\subquestion{}
נוכיח כי לכל פונקציה על, מונוטונית עולה וגזירה ברציפות למקוטעין $\mu : [c, d] \to [a, b]$ מתקיים
\[
	\int_\gamma f(z)\ dz
	= \int_{\gamma \circ \mu} f(z)\ dz
\]
\begin{proof}
	נבחין תחילה ש־$\mu$ היא פונקציה הפיכה מההגדרה שלה. \\*
	נעבור לבחינת האינטגרל
	\[
		\int_{\gamma \circ \mu} f(z)\ dz
		= \int_c^d f(\gamma(\mu(t))) \cdot (\gamma \circ \mu)'(t)\ dt
		= \int_c^d f(\gamma(\mu(t))) \cdot \gamma'(\mu(t)) \cdot \mu'(t)\ dt
	\]
	אבל מכלל ההצבה האינטגרלי עבור $\mu^{-1}$ נובע
	\[
		\int_c^d f(\gamma(\mu(t))) \cdot \gamma'(\mu(t)) \cdot \mu'(t)\ dt
		= \int_a^b f(\gamma(\mu(\mu^{-1}(t)))) \cdot \gamma'(\mu(\mu^{-1}(t)))\ dt
		= \int_a^b f(\gamma(t)) \cdot \gamma'(t)\ dt
		= \int_\gamma f(z)\ dz
	\]
	כפי שרצינו.
\end{proof}

\subquestion{}
נניח ש־$f$ אנליטית ב־$G$ ונוכיח שמתקיים
\[
	\int_\gamma f'(z)\ dz
	= f(\gamma(b)) - f(\gamma(a))
\]
\begin{proof}
	נחלק את המסילה ל־$N$ תת־מסילות באורך שווה ולכן
	\[
		\int_\gamma f'(z)\ dz
		= \sum_{i = 1}^N \int_{\gamma_i} f'(z)\ dz
	\]
	ובהתאם
	\[
		\left\lvert \int_\gamma f'(z)\ dz \right\rvert
		\le \sum_{i = 1}^N \left\lvert  \int_{\gamma_i} f'(z)\ dz \right\rvert
		\le N M_i L(\gamma_1)
	\]
\end{proof}

\end{document}
