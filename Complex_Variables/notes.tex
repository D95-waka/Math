\documentclass[a4paper]{article}

% packages
\usepackage{inputenc, fontspec, amsmath, amsthm, amsfonts, polyglossia, catchfile}
\usepackage[a4paper, margin=50pt, includeheadfoot]{geometry} % set page margins

% style
\AddToHook{cmd/section/before}{\clearpage}	% Add line break before section
\linespread{1.5}
\setcounter{secnumdepth}{0}		% Remove default number tags from sections
\setmainfont{Libertinus Serif}
\setsansfont{Libertinus Sans}
\setmonofont{Libertinus Mono}
\setdefaultlanguage{hebrew}
\setotherlanguage{english}

% operators
\DeclareMathOperator\cis{cis}
\DeclareMathOperator\Sp{Sp}
\DeclareMathOperator\tr{tr}
\DeclareMathOperator\im{Im}
\DeclareMathOperator\diag{diag}
\DeclareMathOperator*\lowlim{\underline{lim}}
\DeclareMathOperator*\uplim{\overline{lim}}

% commands
\renewcommand\qedsymbol{\textbf{משל}}
\newcommand{\NN}[0]{\mathbb{N}}
\newcommand{\ZZ}[0]{\mathbb{Z}}
\newcommand{\QQ}[0]{\mathbb{Q}}
\newcommand{\RR}[0]{\mathbb{R}}
\newcommand{\CC}[0]{\mathbb{C}}
\newcommand{\getenv}[2][] {
  \CatchFileEdef{\temp}{"|kpsewhich --var-value #2"}{\endlinechar=-1}
  \if\relax\detokenize{#1}\relax\temp\else\let#1\temp\fi
}
\newcommand{\explain}[2] {
	\begin{flalign*}
		 && \text{#2} && \text{#1}
	\end{flalign*}
}

% headers
\getenv[\AUTHOR]{AUTHOR}
\author{\AUTHOR}
\date\today

\title{פונקציות מרוכבות --- סיכום}
\setcounter{secnumdepth}{2}

\hypersetup{}
\begin{document}
\maketitle
\maketitleprint{}

\tableofcontents

\section{שיעור 1 --- 31.10.2024}
למרצה קוראים עדי. המייל הוא adi.glucksam@mail.huji.ac.il \\*
שיעורי הבית הפעם הם 20 אחוזים מהציון, גם פה עם התחשבות במטלות הטובות ביותר.
שעת קבלה של עדי היא בימי ראשון אחרי השיעור, דהינו ב־12:00, במנצ'סטר 303.

\subsection{מבוא}
נגדיר מספרים מרוכבים על־ידי ההתאמה $(x, y) \mapsto z = x + i y$ כאשר $i = \sqrt{-1}$, הקבוע המדומה.
נגדיר מספר סימונים שיעזרו לנו בהמשך.
\begin{definition}[חלק שלם וחלק מדומה]
	עבור $z = x + iy$ נגדיר $\re(z) = x$ ו־$\im(z) = y$, החלק הממשי והחלק המדומה בהתאמה.
\end{definition}
נעבור להגדרת הפעולות בשדה המרוכב:
\begin{definition}[חיבור וחיסור מרוכבים]
	אם $z = x + i y$ ו־$w = a + i b$ אז נגדיר $z \pm w = (x \pm a) + i (y \pm b)$.
\end{definition}
\begin{definition}[כפל]
	כפל בסקלר $\alpha \in \RR$ נגדיר על־ידי $\alpha \cdot z = \alpha x + i \alpha y$. \\*
	כפל של מרוכב במרוכב נגדיר על־ידי $z \cdot w = (x + i y)(a + i b) = xa + xib + iya + iy ib = xa - yb + i(xb + ya)$.
\end{definition}
\begin{definition}[הצמדה]
	נגדיר פעולה חדשה שלא קיימת בממשיים, היא הצמדה (conjugation), נסמן $\overline{z} = \overline{x + iy} = x - y$. \\*
	נקבל גם $\overline{\overline{z}} = z$.
\end{definition}
במקרה בו $z \in \RR$ אז נקבל $\overline{z} = x$ ולמעשה השוויון מתקיים אם ורק אם $z \in \RR$.
\begin{definition}[ערך מוחלט]
	נגדיר ערך מוחלט על־ידי $|z| = \sqrt{z \cdot \overline{z}}$. \\*
	פעולה זו מייצגת את המרחק מהראשית במישור המרוכב, בדומה לאופן פעולת הערך המוחלט בממשיים.
\end{definition}
\begin{definition}[חלוקה]
	חלוקה נגדיר על־ידי $\frac{z}{w} = \frac{z \cdot \overline{w}}{w \cdot \overline{w}} = \frac{z \overline{w}}{{|w|}^2} = \frac{1}{{|w|}^2} z \cdot \overline{w}$.
\end{definition}
\begin{remark}[מרוכבים כמרחב וקטורי מעל הממשיים]
	ניתן לבחון את המרוכבים כמרחב וקטורי מעל $\RR^2$ על־ידי $\CC \to \RR^2$ המוגדר
	\[
		z = x \begin{pmatrix} 1 \\ 0 \end{pmatrix} + y \begin{pmatrix} 0 \\ 1 \end{pmatrix}
	\]
\end{remark}
ראינו כי אפשר לייצג את המרוכבים על־ידי מרחב וקטורי ממשי, ובאותו אופן ניתן לייצג את המרוכבים גם על־ידי מטריצות ועל־ידי תצוגה פולארית. בתרגול נעסוק בתצוגת המטריצות, ועתה נתעמק בהצגה פולארית.

נוכל לבחון כל מספר כווקטור, דהינו על־ידי עוצמה וזווית.
בקורס שלנו זווית היא ב־$(-\pi, \pi]$ והיא מודדת מרחק זוויתי מהכיוון החיובי של ציר ה־$x$. %chktex 9
כל מספר $z = x + i y$ ניתן לייצג על־ידי $(r, \theta)$, כאשר $r = |z|$ ו־$\theta = \text{Arg}(z)$.
סימון לזה (ובהמשך הקורס הוא יהפוך להגדרה) הוא $e^{i\theta} = \cos(\theta) + i \sin(\theta)$.
בהתאם $z = r \cdot e^{i\theta}$.
\begin{exercise}
	\begin{enumerate}
		\item הראו כי $e^{i \theta_1} \cdot e^{i \theta_2} = e^{i(\theta_1 + \theta_2)}$
		\item האם נכון תמיד ש־$Arg(z \cdot w) = Arg(z) + Arg(w)$?
		\item אם התשובה היא לא, איך זה לא מתנגש עם סעיף 1?
	\end{enumerate}
\end{exercise}
\begin{exercise}
	מצאו את כל הפתרונות של המשוואה $\sqrt[n]{z} = w$.
\end{exercise}
\begin{solution}
	\[
		\sqrt[n]{z} = w \iff z = w^n = {(r \cdot e^{i\theta})}^n = r^n {(e^{i\theta})}^n
	\]
	אז נקבל ${|w|}^n = r^n$ ולכן נקבל $|w| = {|z|}^\frac{1}{n}$. \\*
	נקבל בנוסף על־ידי נוסחת דה־מואר (שתגיע בהמשך הקורס)
	\[
		{(e^{i\theta})}^n = e^{i\theta} {(e^{i\theta})}^{n - 1} = e^{i n\theta}
	\]
	דהינו $Arg(w) n = Arg(z)$ ולכן $Arg(w) = \frac{Arg(z)}{n} + \frac{2\pi k}{n}$ עבור $k = \{0, 1, \dots, n - 1\}$.
\end{solution}

\subsection{תזכורת למטריקות}
נוכל להגדיר מטריקה על המרוכבים על־ידי שימוש בערך המוחלט שהגדרנו, דהינו נגדיר $d(z, w) = |z - w|$, והגדרה זו משרה טופולוגיה על המרוכבים ומאפשרת לנו לדון במספר תכונות נוספות:
\begin{definition}[כדור פתוח]
	נגדיר כדור פתוח במרוכבים על־ידי $B(z, r) = \{ w \in \CC \mid d(z, w) < r\}$.
\end{definition}
ניזכר בהגדרה של קבוצות פתוחות וסגורות:
\begin{definition}[קבוצה פתוחה וסגורה]
	קבוצה $U \subseteq \CC$ תיקרא \textbf{פתוחה} אם $\forall z \in U \exists r \in \RR, B(z, r) \subseteq U$. \\*
	קבוצה $F \subseteq \CC$ תיקרא \textbf{סגורה} אם המשלים שלה $F^C = \CC \setminus F$ הוא קבוצה פתוחה.
\end{definition}
\begin{definition}[פנים של קבוצה]
	פנים של $A \subseteq \CC$ מוגדר על־ידי $\text{int}(A) = \{ z \in A \mid \exists r > 0, B(z, r) \subseteq A \}$.
\end{definition}
\begin{definition}[חוץ של קבוצה]
	החוץ של $A$ מוגדר על־ידי $\text{Ext}(A) = \text{int}(\CC \setminus A)$.
\end{definition}
\begin{definition}[שפה של קבוצה]
	השפה של $A$ תוגדר להיות $\partial A = \CC \setminus (\text{int}(A) \cup \text{Ext}(A))$.
\end{definition}
\begin{definition}[סגור של קבוצה]
	הסגור של $A$ הוא $A \cup \partial A$, 
\end{definition}
\begin{definition}[קבוצה חסומה וקבוצה קומפקטית]
	קבוצה $A$ היא חסומה אם קיים $R > 0$ כך ש־$A \subseteq B(0, R)$ וקבוצה $K$ היא קומפקטית אם היא סגורה וחסומה.
\end{definition}

\section{שיעור 2 --- 3.11.2024}

\subsection{התכנסות ורציפות}
\begin{definition}[התכנסות סדרות מרוכבים]
	תהי סדרה ${\{z_n\}}_{n = 1}^\infty \subseteq \CC$,
	נאמר ש־$z_n \to z$ אם $\lim_{n \to \infty} |z_n - z| = 0$.
\end{definition}
נבחין כי זהו גבול מעל הממשיים.
\begin{exercise}
	תהי ${\{z_n\}}_{n = 1}^\infty \subseteq \CC$ ונסמן $x_n = \re(z_n), y_n = \im(z_n)$ אז $z_n \to z \iff x_n \to x \land y_n \to y$.
	כאשר $z = x + iy$.
\end{exercise}
\begin{example}
	תהי $z_n = 2^{1/n} + i2^{-n}$ ונבדוק אם היא מתכנסת.
	נבחן את הערך הממשי שלה, $x_n = \re(z_n) = 2^{1/n} \xrightarrow[n \to \infty]{} 1$. \\*
	באופן דומה $y_n = \im(z_n) = 2^{-n} \xrightarrow[x \to \infty]{} 0$.
	ולכן $z_n \to 1$.
\end{example}
\begin{example}
	לעומת זאת $z_n = {(-1)}^n 2^{1/n} + i 2^{-n}$,
	$x_{2n} = \re(z_{2n}) = 2^{1/2n} \xrightarrow{x \to \infty} 1$ \\*
	אבל $x_{2n + 1} = \re(z_{2n + 1}) = - 2^{1/(2n + 1)} \xrightarrow{x \to \infty} -1$.
	ולכן $z_n$ לא מתכנסת.
\end{example}
\begin{definition}
	תהי $G \subseteq \CC$, ותהי $f : G \to \CC$.
	נאמר ש־$f$ רציפה בסביבת $z$ אם לכל סדרה ${\{z_n\}}_{n = 1}^\infty$ כך ש־$z_n \to z$ מקיימת $f(z_n) \to f(z)$. \\*
	נאמר ש־$f$ רציפה (רציפה ב־$G$) אם לכל $z \in G$ מתקיים ש־$f$ רציפה ב־$z$.
\end{definition}
\begin{example}
	נגדיר $f(z) = \re(z) \cdot \im(z) + i \frac{\re(z)}{\im(z)}$ ונגדיר $G = \{ z \in \CC \mid \im(z) \ne 0 \}$. \\*
	אנו יודעים שיש התכנסות אם ורק אם יש התכנסות בחלק הממשיים ובחלק המדומה בנפרד, נקבל
	\[
		\re(f) = \re(z) \cdot \im(z) = x \cdot y
	\]
	ולכן $f$ רציפה כפונקציה מ־$\RR^2$ ל־$\RR$.
	נבדוק את החלק המדומה
	\[
		\im(f) = \frac{\re z}{\im z} = \frac{x}{y}
	\]
	ולכן $f$ רציפה לכל $y \ne 0$.
	נסיק מהתרגיל כי $f$ אכן רציפה ב־$G$.
\end{example}
ניזכר בהגדרת הקשירות
\begin{definition}[קשירות]
	תהי $G \subseteq \CC$ קבוצה פתוחה, התנאים הבאים שקולים:
	\begin{enumerate}
		\item אם $U \subseteq G$ פתוחה וגם $G \setminus U$ פתוחה אז $U = G$ או $U = \emptyset$
		\item קשירות פוליגונלית: לכל $z, w \in G$ קיים $z \le a_1, \dots, a_n \le w$ כך ש־$G \supseteq [a_j, a_{j + 1}]$. \\*
			נבחין כי $[z, w] = \{ t \cdot z + (1 - t) \cdot w \mid t \in [0, 1]\}$. \\*
			הרעיון הוא שלכל שתי נקודות, נוכל לבחור סדרת נקודות, כל שתי נקודות מחוברות בקטע ישר, ובסך הכול קיים מסלול של קטעים ישרים כאלה שמחבר את הנקודות, והחובה היא שכל הקטעים האלה מוכלים בקבוצה.
		\item כל פונקציה קבועה מקומית היא קבועה. \\*
			ההגדרה לפונקציה קבועה מקומית היא $\forall z \in G, \exists r, B(z, r) \subseteq G \land f \mid_{B(z, r)} = c$.
			בפועל המשמעות היא שבכל קבוצה מבודדת הערך קבוע.
	\end{enumerate}
\end{definition}
\begin{definition}[תחום]
	תחום הוא קבוצה פתוחה וקשירה.
\end{definition}
\begin{remark}
	כל $G \subseteq \CC$ פתוחה ניתן לכתוב $G = \biguplus G_i$ ו־$G_i$ תחום.
\end{remark}

\subsection{הטלה סטריאוגרפית}
המטרה היא להטיל את המרחב שמורכב מהמישור המרוכב וציר נוסף לספירת היחידה.
נגדיר את ספירת היחידה להיות $S^2$.
במצב זה הצירים $x, y$ מייצגים את המישור המרוכב עצמו, על־ידי $z = x + i y, z = 0$.
נגדיר $N = (0, 0, 1)$ הנקודה העליונה של ספירת היחידה.
לכל $z \in \CC$ נסמן ב־$P_z = (x, y, 0) = (\re(z), \im(z), 0)$ ונסמן $L_z = \{ t \cdot P_z + (1 - t) N \mid t \in \RR \}$.
בהטלה מהסוג הזה אנו מסתכלים על $N$ בתור אינסוף.

עתה נגדיר $\phi : \CC \to S^2$.
כל נקודה על $L_z$ היא מהצורה $t P_z + (1 - t) N = (tx, ty, (1 - t))$
פריט המידע השני הוא שהנקודה צריכה להיות על ספירת היחידה, דהינו $t P_z + (1 - t) N \in S^2$,
לכן $1 = {\lVert t P_z + (1 - t) N \rVert}^2 = t^2 x^2 + t^2 y^2 + {(1 - t)}^2 \iff 2t = t^2 (1 + {|z|}^2)$, כאשר ${|z|}^2 = x^2 + y^2$.
במקרה $t = 0$ נקבל את $N$ ולכן הוא לא מעניין, אם $t \ne 0$ אז $t = \frac{2}{1 + {|z|}^2}$.
נקבל
\[
	t \cdot P_z + (1 - t)N = (\frac{2x}{1 + {|z|}^2},\frac{2y}{1 + {|z|}^2}, 1 - \frac{2}{1 + {|z|}^2})
\]

נחשב את $\phi^{-1} : S^2 \to \CC$. \\*
עתה $P = (x_0, y_0, z_0) \in S^2$, דהינו $x_0^2 + y_0^2 + z_0^2 = 1$, אך עדיין אם $\phi^{-1}(P) = z_0$ אז $P \in L_{z_0}$ ובהתאם $z_0 \in L_p$, כאשר
\[
	L_p = \{ t P + (1 - t) N \mid t \in \RR \}
\]
ולכן
\[
	\re(z_0) = t x_0,
	\quad \im(z_0) = t y_0,
	\quad \{ z = 0 \} \subseteq \RR^3
\]
אם כך נקבל
\[
	t z_0 + (1 - t) = 0 \iff t(z_0 - 1) = -1 \implies t = \frac{1}{1 - z_0}
\]
אז
\[
	\re(z) = \frac{x_0}{1 - z_0},
	\quad \im(z) = \frac{y_0}{1 - z_0}
\]
אם $z_0 = 1$ אז הנקודה מתאימה לאינסוף, ולכן נשתמש ב־$\CC^* = \CC \cup \{ \infty \}$ במקום ב־$\CC$ עצמו.

במקרה זה גם נוכל לקבל מטריקה חדשה.
\begin{definition}
	$z, w \in \CC$, אז נגדיר $\rho(z, w) = \lVert \phi(z) - \phi(w) \rVert$. \\*
	במקרה זה נקבל $\rho^2(z, w) = \cdots = \frac{|z - w|}{\sqrt{1 + {|z|}^2} \sqrt{1 + {|w|}^2}}$. \\*
	גם
	\[
		\rho(z, \infty) = \lim_{w \to \infty} \rho(z, w)
		= \lim_{w \to \infty} \frac{2| \frac{z}{w} - 1|}{\sqrt{1 + {|z|}^2} \sqrt{1 + {|\frac{1}{w}|}^2}}
		= \frac{2}{\sqrt{1 + {|z|}^2}}
	\]
\end{definition}
\begin{exercise}
	אם $w_n \to \infty$ לא חסום אז $\rho(w_n, \infty) \to 0$.
\end{exercise}

מה קורה למעגלים ב־$S^2$ תחת $\phi_{-1}$? \\*
נשים לב שאם $C$ מעגל על $S^2$ אז בהכרח $C = S^2 \cap P_C$ עבור $P_C$ מישור כלשהו.
\[
	P_C = \{ (x, y, z) \mid ax + by + cz = d, a, b, c, d \in \RR \}
\]
תהי $z \in \CC$ המקיימת $\phi(z) \in C$ אז בפרט $\phi(z) \in P_C$, אז נציג $\phi(z)$ במשוואת המישור
\[
	d = a \cdot \frac{2 \re(z)}{1 + {|z|}^2} + b \cdot \frac{2 \im(z)}{1 + {|z|}^2} + c \frac{{|z|}^2 - 1}{{|z|}^2 + 1}
	\implies d + c = 2a \re(z) + 2b \im(z) + {|z|}^2 (c - d)
\]
נבחן את המקרה ש־$c = d$, נקבל
\[
	c = a \re(z) + b \im(z) = ax + by
\]
וזהו למעשה ישר במישור.
אם $c \ne d$ אז מקבלים משוואת מעגל
\[
	c + d = a2 x + 2b y + (x^2 + y^2)(c - d)
	\iff {(x - A)}^2 + {(y - B)}^2 = C^2
\]
המשמעות היא שאם $c = d$ אז $N \in P_C$ ואם $c \ne d$ אז $N \notin P_C$.

\subsection{דיפרנציאביליות}
מעכשיו $F : U_z \to \CC$ כאשר $U_z$ סביבה פתוחה של $z$, לדוגמה כדור פתוח סביב $z$.
נראה תזכורת מ־$\RR^2$.
\[
	\frac{\partial f}{\partial x}(x_0, y_0) = \lim_{x \to x_0} \frac{f(x, y_0) - f(x_0, y_0)}{x - x_0}
\]
$f$ דיפרנציאבילית ב־$(x_0, y_0)$ אם ניתן לחקור את הפונקציה על־ידי חקירת קירוב לינארי שלה, דהינו
\[
	\lim_{(x, y) \to (x_0, y_0)}  \frac{1}{\lVert (x, y) - (x_0, y_0) \rVert}\lVert f(x, y) - f(x_0, y_0) + f_x(x_0, y_0) (x - x_0) + f_y(x_0, y_0)(y - y_0) \rVert = 0
\]

\section{תרגול 1 --- 4.11.2024}
\subsection{מנהלות}
למתרגל קוראים יונתן.
יש 12 תרגילים בסמסטר הזה, כולם להגשה ונלקחים 10 הטובים ביותר.
הם 20\% מהציון, אז חשוב להשקיע בהם.
תהיה ליונתן שעת קבלה אבל הוא עוד לא קבע אותה.
המייל שלו הוא yonatan.bachar@mail.huji.ac.il.

\subsection{שדה המרוכבים}
הגדרנו את שדה המרוכבים על־ידי
\[
	\CC = \{ a + bi \mid a, b \in \RR \},
	\qquad i^2 = -1
\]
כפי שראינו בשיעור 1, יש לנו מספר פעולות על המרוכבים.

אנו גם יודעים כי $\CC \cong \RR^2$ עם בסיס $\{ 1, i \}$ כמרחב וקטורי, נקבע $z \in \CC$ ונגדיר $M_z : \CC \to \CC$ על־ידי $M_z(w) = z \cdot w$.
נבחן את המטריצה המייצגת של ההעתקה הזו, נניח $z = a + bi$ ונבדוק את הפעולה על הבסיס שלנו.
\[
	M_z = \begin{pmatrix}
		a & -b \\
		b & a
	\end{pmatrix}
\]
העתקה זו מייצגת את הכפל ב־$z$. \\*
מה אנחנו יכולים להגיד על ההעתקה הזו?
תכונות:
\begin{enumerate}
	\item $M_{z + w} = M_z + M_w$
	\item $M_{z \cdot w} = M_z \cdot M_w$
	\item $M_{\overline{z}} = M_z^T$
\end{enumerate}
נגדיר את $z$ בתצורה פולארית:
\[
	z = r e^{i \theta}
\]
ונקבל
\[
	M_z = M_{r e^{i\theta}}
	= \begin{pmatrix}
		r \\
		r
	\end{pmatrix}
	\begin{pmatrix}
		\cos \theta & - \sin \theta \\
		\sin \theta & \cos \theta
	\end{pmatrix}
\]
זוהי למעשה מטריצה סקלרית כפול מטריצת סיבוב.

\subsection{טופולוגיה וסדרות}
נבחן את המטריקה על $\CC$:
\[
	d(z, w) = |z - w|
\]
היא מגדירה לנו טופולוגיה עם קבוצות פתוחות בסיסיות מהצורה $B(z, r)$. \\*
כמרחב טופולוגי או נורמי, $\CC \cong \RR^2$ ולכן התכונות מושרות. \\*
נגדיר שסדרה $(z_n)$ מתכנסת ל־$z$ אם $|z_n - z| \to 0$ ונאמר שסדרה היא חסומה אם קיים $r > 0$ כך ש־$|z_n| < r$ לכל $n \in \NN$.
ניזכר בטענה מההרצאה:
\begin{proposition}
	$z_n \to z$ אם ורק אם $\re(z_n) \to \re(z) \land \im(z_n) \to \im(z)$.
\end{proposition}
ונראה דוגמה לשימוש בטענה זו.
\begin{example}
	נגדיר $z_n = n(e^{-n} + i \sin \frac{1}{n})$. \\*
	נקבל מהטענה כי $z_n \to i$.
\end{example}
נעבור לסדרה מעניינת יותר
\begin{example}
	נקבע $c \in \CC$ ונגדיר $f_c(z) = z^2 + c$, נתבונן בסדרה
	\[
		z_0 = 0,
		\quad z_1 = f_c(0),
		\quad z_n = f_c(z_{n - 1})
	\]
	עבור $c = 0$ נקבל $z_n \equiv 0$.
	עבור $c = 1$ נקבל $0, 1, 2, 5, 26, \dots$.
	עבור $c = i$ נקבל $0, i, -1 + i, -i, -1 + i, \dots$. \\*
	הסדרה הזו מתנהגת בצורה מאוד משונה בהתאם לנקודת ההתחלה, וקשה להבין את ההתנהגות באופן כללי. \\*
	סדרה זו מהווה הבסיס להגדרה של קבוצת מנדלברוט ופרקטל מנדלברוט, קבוצה זו מוגדרת על־ידי המספרים המרוכבים שהסדרה שלהם חסומה:
	\[
		M = \{ c \in \CC \mid \exists r > 0, \forall n \in \NN |f_c^n(0)| < r \}
	\]
\end{example}

נסיים בתזכורת בהטלה הסטריאוגרפית שראינו בשיעור 2. \\*
הגדרנו את הספירה של רימן, $\CC^* = \CC \cup \{ \infty \}$ בתור קומפקטיזציה חד־מימדית והגדרנו את $S^2$ על־ידי
\[
	S^2 = \{ (x_0, y_0, z_0) \in \RR^3 \mid x_0^2 + y_0^2 + z_0^2 = 1 \}
\]
ראינו את $\pi : S^2 \to \CC^*$ ההטלה, מצאנו כי היא נתונה על־ידי
\[
	\pi(x_0, y_0, z_0) = \frac{x_0}{1 - z_0} + i \frac{y_0}{1 - z_0}
\]
נראה עתה שתי טענות מעניינות
\begin{proposition}
	לכל $N \in S^2$ מתקיים
	\[
		\pi(N) \overline{\pi(-N)} = -1
	\]
\end{proposition}
\begin{proof}
	נסמן $N = (x_0, y_0, z_0)$ ונקבל
	\begin{align*}
		\pi(N) \cdot \overline{\pi(-N)}
		& = \left(\frac{x_0}{1 - z_0} + i \frac{y_0}{1 - z_0}\right) \left(-\frac{x_0}{1 + z_0} + i \frac{y_0}{1 + z_0}\right) \\
		& = \frac{-x_0^2 - y_0^2}{1 - z_0^2} + i \frac{x_0 y_0 - x_0 y_0}{1 - z_0^2} \\
		& = \frac{-(1 - z_0^2)}{1 - z_0^2} \\
		& = -1
	\end{align*}
\end{proof}
\begin{proposition}
	לכל $\theta$ מתקיים:
	\[
		\sin(2\theta) = 2 \sin \theta \cos \theta
	\]
\end{proposition}
\begin{proof}
	נוכיח באמצעות מרוכבים
	\begin{align*}
		\sin(2\theta) = \im(e^{i2\theta})
		& = \im({(e^{i\theta})}^2) \\
		& = \im({(\cos\theta + i \sin \theta)}^2) \\
		& = \im((\cos^2 \theta - \sin^2 \theta) + i (2\sin\theta \cos \theta)) \\
		& = 2 \sin \theta \cos \theta
	\end{align*}
\end{proof}

\end{document} % chktex 17
