\documentclass[a4paper]{article}

% packages
\usepackage{inputenc, fontspec, amsmath, amsthm, amsfonts, polyglossia, catchfile}
\usepackage[a4paper, margin=50pt, includeheadfoot]{geometry} % set page margins

% style
\AddToHook{cmd/section/before}{\clearpage}	% Add line break before section
\linespread{1.5}
\setcounter{secnumdepth}{0}		% Remove default number tags from sections
\setmainfont{Libertinus Serif}
\setsansfont{Libertinus Sans}
\setmonofont{Libertinus Mono}
\setdefaultlanguage{hebrew}
\setotherlanguage{english}

% operators
\DeclareMathOperator\cis{cis}
\DeclareMathOperator\Sp{Sp}
\DeclareMathOperator\tr{tr}
\DeclareMathOperator\im{Im}
\DeclareMathOperator\diag{diag}
\DeclareMathOperator*\lowlim{\underline{lim}}
\DeclareMathOperator*\uplim{\overline{lim}}

% commands
\renewcommand\qedsymbol{\textbf{משל}}
\newcommand{\NN}[0]{\mathbb{N}}
\newcommand{\ZZ}[0]{\mathbb{Z}}
\newcommand{\QQ}[0]{\mathbb{Q}}
\newcommand{\RR}[0]{\mathbb{R}}
\newcommand{\CC}[0]{\mathbb{C}}
\newcommand{\getenv}[2][] {
  \CatchFileEdef{\temp}{"|kpsewhich --var-value #2"}{\endlinechar=-1}
  \if\relax\detokenize{#1}\relax\temp\else\let#1\temp\fi
}
\newcommand{\explain}[2] {
	\begin{flalign*}
		 && \text{#2} && \text{#1}
	\end{flalign*}
}

% headers
\getenv[\AUTHOR]{AUTHOR}
\author{\AUTHOR}
\date\today

\title{פונקציות מרוכבות --- סיכום}
\setcounter{secnumdepth}{2}

\hypersetup{}
\begin{document}
\maketitle
\maketitleprint{}

\tableofcontents

\section{שיעור 1 --- 31.10.2024}
למרצה קוראים עדי. המייל הוא adi.glucksam@mail.huji.ac.il \\*
שיעורי הבית הפעם הם 20 אחוזים מהציון, גם פה עם התחשבות במטלות הטובות ביותר.
שעת קבלה של עדי היא בימי ראשון אחרי השיעור, דהינו ב־12:00, במנצ'סטר 303.

\subsection{מבוא}
נגדיר מספרים מרוכבים על־ידי ההתאמה $(x, y) \mapsto z = x + i y$ כאשר $i = \sqrt{-1}$, הקבוע המדומה.
נגדיר מספר סימונים שיעזרו לנו בהמשך.
\begin{definition}[חלק שלם וחלק מדומה]
	עבור $z = x + iy$ נגדיר $\re(z) = x$ ו־$\im(z) = y$, החלק הממשי והחלק המדומה בהתאמה.
\end{definition}
נעבור להגדרת הפעולות בשדה המרוכב:
\begin{definition}[חיבור וחיסור מרוכבים]
	אם $z = x + i y$ ו־$w = a + i b$ אז נגדיר $z \pm w = (x \pm a) + i (y \pm b)$.
\end{definition}
\begin{definition}[כפל]
	כפל בסקלר $\alpha \in \RR$ נגדיר על־ידי $\alpha \cdot z = \alpha x + i \alpha y$. \\*
	כפל של מרוכב במרוכב נגדיר על־ידי $z \cdot w = (x + i y)(a + i b) = xa + xib + iya + iy ib = xa - yb + i(xb + ya)$.
\end{definition}
\begin{definition}[הצמדה]
	נגדיר פעולה חדשה שלא קיימת בממשיים, היא הצמדה (conjugation), נסמן $\overline{z} = \overline{x + iy} = x - y$. \\*
	נקבל גם $\overline{\overline{z}} = z$.
\end{definition}
במקרה בו $z \in \RR$ אז נקבל $\overline{z} = x$ ולמעשה השוויון מתקיים אם ורק אם $z \in \RR$.
\begin{definition}[ערך מוחלט]
	נגדיר ערך מוחלט על־ידי $|z| = \sqrt{z \cdot \overline{z}}$. \\*
	פעולה זו מייצגת את המרחק מהראשית במישור המרוכב, בדומה לאופן פעולת הערך המוחלט בממשיים.
\end{definition}
\begin{definition}[חלוקה]
	חלוקה נגדיר על־ידי $\frac{z}{w} = \frac{z \cdot \overline{w}}{w \cdot \overline{w}} = \frac{z \overline{w}}{{|w|}^2} = \frac{1}{{|w|}^2} z \cdot \overline{w}$.
\end{definition}
\begin{remark}[מרוכבים כמרחב וקטורי מעל הממשיים]
	ניתן לבחון את המרוכבים כמרחב וקטורי מעל $\RR^2$ על־ידי $\CC \to \RR^2$ המוגדר
	\[
		z = x \begin{pmatrix} 1 \\ 0 \end{pmatrix} + y \begin{pmatrix} 0 \\ 1 \end{pmatrix}
	\]
\end{remark}
ראינו כי אפשר לייצג את המרוכבים על־ידי מרחב וקטורי ממשי, ובאותו אופן ניתן לייצג את המרוכבים גם על־ידי מטריצות ועל־ידי תצוגה פולארית. בתרגול נעסוק בתצוגת המטריצות, ועתה נתעמק בהצגה פולארית.

נוכל לבחון כל מספר כווקטור, דהינו על־ידי עוצמה וזווית.
בקורס שלנו זווית היא ב־$(-\pi, \pi]$ והיא מודדת מרחק זוויתי מהכיוון החיובי של ציר ה־$x$. %chktex 9
כל מספר $z = x + i y$ ניתן לייצג על־ידי $(r, \theta)$, כאשר $r = |z|$ ו־$\theta = \text{Arg}(z)$.
סימון לזה (ובהמשך הקורס הוא יהפוך להגדרה) הוא $e^{i\theta} = \cos(\theta) + i \sin(\theta)$.
בהתאם $z = r \cdot e^{i\theta}$.
\begin{exercise}
	\begin{enumerate}
		\item הראו כי $e^{i \theta_1} \cdot e^{i \theta_2} = e^{i(\theta_1 + \theta_2)}$
		\item האם נכון תמיד ש־$Arg(z \cdot w) = Arg(z) + Arg(w)$?
		\item אם התשובה היא לא, איך זה לא מתנגש עם סעיף 1?
	\end{enumerate}
\end{exercise}
\begin{exercise}
	מצאו את כל הפתרונות של המשוואה $\sqrt[n]{z} = w$.
\end{exercise}
\begin{solution}
	\[
		\sqrt[n]{z} = w \iff z = w^n = {(r \cdot e^{i\theta})}^n = r^n {(e^{i\theta})}^n
	\]
	אז נקבל ${|w|}^n = r^n$ ולכן נקבל $|w| = {|z|}^\frac{1}{n}$. \\*
	נקבל בנוסף על־ידי נוסחת דה־מואר (שתגיע בהמשך הקורס)
	\[
		{(e^{i\theta})}^n = e^{i\theta} {(e^{i\theta})}^{n - 1} = e^{i n\theta}
	\]
	דהינו $Arg(w) n = Arg(z)$ ולכן $Arg(w) = \frac{Arg(z)}{n} + \frac{2\pi k}{n}$ עבור $k = \{0, 1, \dots, n - 1\}$.
\end{solution}

\subsection{תזכורת למטריקות}
נוכל להגדיר מטריקה על המרוכבים על־ידי שימוש בערך המוחלט שהגדרנו, דהינו נגדיר $d(z, w) = |z - w|$, והגדרה זו משרה טופולוגיה על המרוכבים ומאפשרת לנו לדון במספר תכונות נוספות:
\begin{definition}[כדור פתוח]
	נגדיר כדור פתוח במרוכבים על־ידי $B(z, r) = \{ w \in \CC \mid d(z, w) < r\}$.
\end{definition}
ניזכר בהגדרה של קבוצות פתוחות וסגורות:
\begin{definition}[קבוצה פתוחה וסגורה]
	קבוצה $U \subseteq \CC$ תיקרא \textbf{פתוחה} אם $\forall z \in U \exists r \in \RR, B(z, r) \subseteq U$. \\*
	קבוצה $F \subseteq \CC$ תיקרא \textbf{סגורה} אם המשלים שלה $F^C = \CC \setminus F$ הוא קבוצה פתוחה.
\end{definition}
\begin{definition}[פנים של קבוצה]
	פנים של $A \subseteq \CC$ מוגדר על־ידי $\text{int}(A) = \{ z \in A \mid \exists r > 0, B(z, r) \subseteq A \}$.
\end{definition}
\begin{definition}[חוץ של קבוצה]
	החוץ של $A$ מוגדר על־ידי $\text{Ext}(A) = \text{int}(\CC \setminus A)$.
\end{definition}
\begin{definition}[שפה של קבוצה]
	השפה של $A$ תוגדר להיות $\partial A = \CC \setminus (\text{int}(A) \cup \text{Ext}(A))$.
\end{definition}
\begin{definition}[סגור של קבוצה]
	הסגור של $A$ הוא $A \cup \partial A$, 
\end{definition}
\begin{definition}[קבוצה חסומה וקבוצה קומפקטית]
	קבוצה $A$ היא חסומה אם קיים $R > 0$ כך ש־$A \subseteq B(0, R)$ וקבוצה $K$ היא קומפקטית אם היא סגורה וחסומה.
\end{definition}

\end{document} % chktex 17
