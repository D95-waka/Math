\documentclass[a4paper]{article}

% packages
\usepackage{inputenc, fontspec, amsmath, amsthm, amsfonts, polyglossia, catchfile}
\usepackage[a4paper, margin=50pt, includeheadfoot]{geometry} % set page margins

% style
\AddToHook{cmd/section/before}{\clearpage}	% Add line break before section
\linespread{1.5}
\setcounter{secnumdepth}{0}		% Remove default number tags from sections
\setmainfont{Libertinus Serif}
\setsansfont{Libertinus Sans}
\setmonofont{Libertinus Mono}
\setdefaultlanguage{hebrew}
\setotherlanguage{english}

% operators
\DeclareMathOperator\cis{cis}
\DeclareMathOperator\Sp{Sp}
\DeclareMathOperator\tr{tr}
\DeclareMathOperator\im{Im}
\DeclareMathOperator\diag{diag}
\DeclareMathOperator*\lowlim{\underline{lim}}
\DeclareMathOperator*\uplim{\overline{lim}}

% commands
\renewcommand\qedsymbol{\textbf{משל}}
\newcommand{\NN}[0]{\mathbb{N}}
\newcommand{\ZZ}[0]{\mathbb{Z}}
\newcommand{\QQ}[0]{\mathbb{Q}}
\newcommand{\RR}[0]{\mathbb{R}}
\newcommand{\CC}[0]{\mathbb{C}}
\newcommand{\getenv}[2][] {
  \CatchFileEdef{\temp}{"|kpsewhich --var-value #2"}{\endlinechar=-1}
  \if\relax\detokenize{#1}\relax\temp\else\let#1\temp\fi
}
\newcommand{\explain}[2] {
	\begin{flalign*}
		 && \text{#2} && \text{#1}
	\end{flalign*}
}

% headers
\getenv[\AUTHOR]{AUTHOR}
\author{\AUTHOR}
\date\today

\title{פתרון מטלה 07 --- פונקציות מרוכבות, 80519}

\begin{document}
\maketitle
\maketitleprint{}

\question{}
נחשב את האינטגרלים הבאים על־ידי שימוש במשפט קושי.

\subquestion{}
\[
	J = \int_0^\infty \frac{1 - \cos(x)}{x^2}\ dx
\]
על־ידי שימוש בהרכבת המסילות
\begin{align*}
	& \gamma_1(t) = (1 - t) \epsilon + t r & t \in [0, 1] \\
	& \gamma_2(t) = r e^{it} & t \in [0, \pi] \\
	& \gamma_3(t) = (1 - t) (-r) + t (-\epsilon) & t \in [0, 1] \\
	& \gamma_4(t) = \epsilon e^{i(\pi - t)} & t \in [0, \pi] \\
\end{align*}
\begin{solution}
	נגדיר $f(z) = \frac{1 - e^{iz}}{z^2}$ ונבחין כי $J = \re(\int_0^\infty f)$. \\
	לכל $\epsilon, r \in \RR$ האינטגרל מתאפס ממשפט קושי, אנו גם יודעים שהאינטגרל מתכנס מאינפי 2 ולכן נעבור לחישובו. \\
	נבדוק את האינטגרל על $\gamma_2$:
	\[
		I_2 = \int_{\gamma_2} f(z)\ dz
	\]
	נבחין כי
	\[
		\max_{z \in \gamma_2} |f(z)|
		= \max_{z \in \gamma_2} \left\lvert \frac{1 - e^{iz}}{z^2} \right\rvert
		\le \max_{z \in \gamma_2} \frac{|1 - e^{iz}|}{r^2}
		\le \frac{1 + 1 \cdot \max_{t \in [0, \pi]} e^{-\sin t}}{r^2}
		= \frac{2}{r^2}
	\]
	ולכן מאי־שוויון ML,
	\[
		I_2 \le \frac{2}{r^2} \cdot \frac{2\pi r}{2} = \frac{2}{r}
	\]
	לכן $I_2 \to 0$. \\
	נעבור לבחינת
	\[
		I_4 = \int_{\gamma_4} f(z)\ dz
	\]
	הפעם מקירוב לינארי כפי שראינו בהרצאה
	\[
		\int_{\gamma_4} f
		= \int_{\gamma_4} \frac{1 - (1 + iz + o(|z|))}{z^2}\ dz
		= -\int_0^\pi \frac{i e^{i(\pi - t)} + o(|\epsilon e^{i(\pi - t)})}{\epsilon e^{i(\pi - t)} \cdot \epsilon e^{i(\pi - t)}} (-i \epsilon) e^{i(\pi - t)}\ dt
		= i \int_0^\pi \frac{i \epsilon e^{i(\pi - t)} + o(|\epsilon|)}{\epsilon e^{i(\pi - t)}}\ dt
		\to -\pi
	\]
	נבחין כי גם
	\[
		I_1 = \int_{\gamma_1} f(z)\ dz
		= \int_0^1 \frac{1 - e^{i(1 - t)\epsilon + itr}}{{((1 - t)\epsilon + tr)}^2}\ dt
		= \int_\epsilon^r \frac{1 - e^{it}}{t^2}\ dt
	\]
	וכן
	\[
		I_3
		= \int_{-r}^{-\epsilon} \frac{1 - e^{it}}{t^2}\ dt
		= \int_{r}^{\epsilon} -\frac{1 - e^{-it}}{t^2}\ dt
		= \int_{\epsilon}^r \frac{1 - e^{-it}}{t^2}\ dt
	\]
	ולכן
	\[
		I_1 + I_3
		= 2 \re(\int_\epsilon^r \frac{1 - e^{it}}{t^2}\ dt)
		= 2 \int_\epsilon^r \frac{1 - \cos t}{t^2}\ dt
	\]
	ונוכל להסיק $I_1 + I_3 \to 2 J$. \\
	ממשפט קושי נוכל להסיק
	\[
		I_1 + I_2 + I_3 + I_4 = 0
		\implies
		-\pi + 2 J = 0
	\]
	ולכן $J = \frac{\pi}{2}$
\end{solution}

\subquestion{}
נחשב את
\[
	J = \int_0^\infty \frac{x^{m - 1}}{x^n + 1}\ dx
\]
עבור $n > m$ טסעיים על־ידי שימוש במסילה הבאה
\begin{align*}
	& \gamma_1(t) = t & t \in [0, r] \\
	& \gamma_2(t) = r e^{it} & t \in [0, \frac{2\pi}{n}] \\
	& \gamma_3(t) = e^{i \frac{2\pi}{n}}(r - t) & t \in [0, r] \\
	& \gamma_4(t) = e^{\frac{\pi i}{n}} + \epsilon e^{-it} & t \in [0, 2\pi]
\end{align*}
\begin{solution}
	נגדיר
	\[
		f(z) = \frac{z^{m - 1}}{z^n + 1}
	\]
	ונבחין כי הפונקציה מוגדרת ורציפה בכל נקודה פרט ל־$z = e^{\frac{\pi i}{n}}$, לכן התחום הסגור שהמסילות מגדירות הוא תחום בו $f$ רציפה כך שהוא מקיים את תנאי משפט קושי המורחב (את תנאי רגל שמאל) ולכן הוא תקף. \\
	כמובן מאינפי 2 האינטגרל המבוקש מתכנס ויש הצדקה לדבר על ערכו, ונגדיר
	\[
		I_1 = \int_{\gamma_1} f(z)\ dz, \quad
		I_2 = \int_{\gamma_2} f(z)\ dz, \quad
		I_3 = \int_{\gamma_3} f(z)\ dz, \quad
		I_4 = \int_{\gamma_4} f(z)\ dz
	\]
	אז מהמשפט נקבל $I_1 + I_2 + I_3 + I_4 = 0$ לכל $\epsilon, r > 0$. \\
	נעבור לבדיקת האינטגרלים הללו.
	\[
		I_1
		= \int_0^r \frac{t^{m - 1}}{t^n + 1}\ dt
		\to J
	\]
	וכן
	\[
		I_3
		= \int_0^r \frac{e^{i \frac{2\pi (m - 1)}{n}} {(r - t)}^{m - 1}}{{(r - t)}^n + 1}\ dt
		= e^{i \frac{2\pi (m - 1)}{n}} \int_0^r \frac{t^{m - 1}}{t^n + 1}\ dt
		\to e^{i \frac{2\pi (m - 1)}{n}} J
	\]
	נחסום את $f$ ב־$\gamma_2$ ונקבל
	\[
		\sup_{t \in [0, \frac{2\pi}{n}]} |f(\gamma_2(t))|
		\le \sup_{t \in [0, \frac{2\pi}{n}]} \frac{r^{m - 1}}{r^n + 1}
		= \frac{r^{m - 1}}{r^n + 1}
	\]
	ולכן מאי־שוויון ML והנתון אודות $n > m$,
	\[
		I_2
		\le \frac{r^{m - 1}}{r^n + 1} \cdot \frac{2\pi r}{n}
		= \frac{r^m}{r^n + 1} \cdot \frac{2\pi}{n}
		\to 0
	\]
	נעבור לחישוב $I_4$,
	\[
		I_4
		= \int_{0}^{2\pi} \frac{{(e^{\frac{\pi i}{n}} + \epsilon e^{it})}^{m - 1}}{{(e^{\frac{\pi i}{n}} + \epsilon e^{it})}^n + 1}\ dt
		= \int_{0}^{2\pi} \frac{e^{\frac{(m - 1) \pi i}{n}} + o(\epsilon)}{2 + o(\epsilon)}\ dt
		\to \int_{0}^{2\pi} \frac{e^{\frac{(m - 1) \pi i}{n}}}{2}\ dt 
		= \pi e^{\frac{(m - 1) \pi i}{n}}
	\]
	ולכן
	\[
		J + e^{i \frac{2\pi(m - 1)}{n}} J + 0 + \pi e^{i \frac{\pi(m - 1)}{n}} = 0
	\]
	ובפרט
	\[
		J + \cos(\frac{2\pi(m - 1)}{n}) J = - \pi \cos(\frac{\pi(m - 1)}{n})
	\]
	ולכן
	\[
		J = \frac{- \pi \cos(\frac{\pi(m - 1)}{n})}{1 + \cos(\frac{2\pi(m - 1)}{n})}
	\]
\end{solution}

\question{}
יהי $r > 0$ ו־$\gamma_r : [0, \pi] \to \CC$ על־ידי $\gamma_r(t) = r e^{it}$. \\
אם $g : \overline{B}(0, r) \to \CC$ רציפה בכל תחומה ואנליטית ב־$B(0, r)$ אז נוכיח שמתקיים
\[
	\left\lvert \int_{\gamma_r} e^{i a z} g(z)\ dz \right\rvert \le M \cdot \frac{\pi}{a}
\]
עבור $M = \max_{t \in [0, \pi]} |g(\gamma_r(t))|$ לכל $a > 0$.
\begin{proof}
	נרצה להשתמש באי־שוויון ML, ולכן נחפש חסם לפונקציה הפנימית,
	\begin{align*}
		\sup_{0 \le t \le \pi} |e^{i a \gamma_r(t)} g(\gamma_r(t))|
		& = \sup_{0 \le t \le \pi} |e^{i a r e^{it}}| \cdot |g(z)| \\
		& \le M \sup_{0 \le t \le \pi} |e^{i a r (\cos t + i \sin t)}| \\
		& = M \sup_{0 \le t \le \pi} |e^{i a r \cos t}| \cdot |e^{-ar \sin t}| \\
		& = M \sup_{0 \le t \le \pi} e^{-ar \sin t} \\
		& = \frac{M}{e^{ar}} \\
		& \le \frac{M}{a r}
	\end{align*}
	וכן $L(\gamma_r) = \pi r$, ונובע
	\[
		M L(\gamma_r) \le M \cdot \frac{\pi r}{a r} = M \frac{\pi}{a}
	\]
	כפי שרצינו להראות.
\end{proof}

\question{}
יהי $G \subseteq \CC$ תחום כוכבי. \\
נוכיח כי לכל פונקציה אנליטית $f : G \to \CC$ קיימת פונקציה קדומה.
\begin{proof}
	תהי $z_0 \in G$ נקודה כך שלכל $z \in G$ גם $[z_0, z] \subseteq G$, קיימת כזו מהיות $G$ כוכבי. \\
	נגדיר מסילה $\gamma_z(t) = z_0 (1 - t) + z t$ וכן את הפונקציה $F(z) = \int_{\gamma_z} f(w)\ dw$. \\
	אנו רוצים להראות ש־$F$ רציפה וכן ש־$F' = f$. \\
	תהי $z$ ותהי $z' \in B(z, \epsilon)$, אז ממשפט קושי למשולשים מתקיים
	\[
		|F(z) - F(z')| = \left\lvert \int_{[z', z]} f(w)\ dw \right\rvert \le M |z' - z| = M \epsilon
	\]
	עבור $M = \sup_{w \in [z, z']} f(w)$, אז גם
	\[
		\lim_{z' \to z} \frac{|F(z') - F(z) - f(z)(z' - z)|}{|z' - z|}
		\le \lim_{z' \to z} \sup_{w \in [z, z']} f(w) - f(z)
		= 0
	\]
	ומצאנו כי הפונקציה $F$ גזירה כך ש־$f$ נגזרתה.
\end{proof}

\question{}
ץהי $f$ פונקציה הנתונה על־ידי טור חזקות מתכנס עם רדיוס התכנסות $r$ ונסמן
\[
	f(z) = \sum_{n = 0}^{\infty} a_n z^n
\]

\subquestion{}
נוכיח כי רדיוס ההתכנסות של טור הנגזרות הנתון על־ידי
\[
	g(z) = \sum_{n = 1}^{\infty} n a_n z^{n - 1}
\]
הוא $r$.
\begin{proof}
	יהי $z \in B(0, r)$, אז $f(z)$ מוגדר ואף מתכנס בהחלט בנקודה זו, כלומר
	\[
		\sum_{n = 0}^{\infty} |a_n| \cdot {|z|}^n
	\]
	מוגדר אף הוא, זאת ראינו בתחילת הקורס. \\
	נבחין שזהו טור ממשי ולכן נוכל לגזור אותו איבר איבר ויתקבל
	\[
		\sum_{n = 1}^{\infty} n |a_n| {|z|}^{n - 1}
	\]
	טור מתכנס, אבל זוהי התכנסות בהחלט של הטור אותו אנו מחפשים, לכן $g(z)$ מוגדר ומתכנס בהחלט לכל $z \in B(0, r)$, לכן רדיוס ההתכנסות של $g$ הוא $r$.
\end{proof}

\subquestion{}
נוכיח שמתקיים
\[
	f(z) = f(0) + \int_{[0, z]} g(w)\ dw
\]
\begin{proof}
	נבחין כי מהשאלה הקודמת והעובדה שכדור הוא תחום כוכבי קיימת $G$ פונקציה הולומורפית כך ש־$G' = g$, נבחין כי קיימת יחידה כזו כך שגם $G(0) = f(0)$, ונראה ש־$f = G$.
	\[
		\left\lvert f(0) + \int_{[0, z]} g(w)\ dw - f(z) \right\rvert
		\le \left\lvert \int_{[0, z]} \sum_{n = 1}^{\infty} n a_n z^n\ dw - \sum_{n = 1}^{\infty} a_n z^n \right\rvert
		\le \int_{[0, z]} |g(w)|\ dw
	\]
	יהי $N \in \NN$, אז נגדיר $f_N, g_N$ התחיליות של הטורים, כלומר $f_N \to f, g_N \to g$, ולכן
	\[
		\left\lvert f_N(z) - f(0) - \int_{[0, z]} g_N(w)\ dw \right\rvert
		= \left\lvert \sum_{n = 1}^{N} a_n z^n - \int_{[0, z]} \sum_{n = 1}^{N} n a_n z^{n - 1}\ dw \right\rvert
		= \left\lvert \sum_{n = 1}^{N} a_n z^n - \sum_{n = 1}^{N} \int_{[0, z]} n a_n z^{n - 1}\ dw \right\rvert
		= 0
	\]
	כאשר המהלך האחרון נובע מאינטגרביליות פולינום שראינו כבר, ועתה נוכל להסיק גם
	\[
		\lim_{N \to \infty} |f(z) - f(0) - \int_{[0, z]} g(w)\ dw| = 0
	\]
	ולכן ממשפט התכנסות במידה שווה נסיק $f(z) = f(0) + \int_{[0, z]} g(w)\ dw$.
\end{proof}

\subquestion{}
נוכיח ללא שימוש במשפט קושי שלכל מסילה סגורה $\gamma : I \to B(0, r)$ מתקיים
\[
	\int_\gamma g(z)\ dz = 0
\]
\begin{proof}
	יהי $N \in \NN$ ונגדיר חלוקה למסילה $\gamma$ על־ידי $[\gamma_0, \dots, \gamma_{n - 1}]$ כך ש־$\gamma = \gamma_0 + \cdots + \gamma_{n - 1}$. \\
	עוד נגדיר
	\[
		I = \int_\gamma g(z)\ dz
	\]
	נסמן את נקודות הקצה של $\gamma_i$ ב־$a_i, b_i$ ונגדיר
	\[
		I_i = \int_{\gamma_i} g(z)\ dz + \int_{[0, a_i]} f(z)\ dz + \int_{[b_i, 0]} f(z)\ dz
	\]
	לבסוף נבחין כי
	\[
		I = I_1 + \cdots I_{n - 1}
	\]
	ישירות מהגדרה ומהעובדה ש־$a_{i + 1} = b_i$. \\
	מתוצאת הסעיף הקודם נובע גם
	\[
		I_i = f(b_i) - f(a_i) + \int_{\gamma_i} g(z)\ dz
	\]
	ולכן מרציפות $f$ נובע ש־$N \to \infty$ גורר $I_i = \int_{\gamma_i} g(z)\ dz$. \\
	כאשר $N$ מספיק קטן $I_i$ שואף לקירוב הפוליגונלי של $\gamma$ ולכן משפט קושי למשולשים (אותו אנו יכולים להוכיח מחדש) תקף ונובע ש־$I_i \to 0$,
	ולכן $\int_{\gamma_i} g(z)\ dz \to 0$ לכל $0 \le i < n$ וכן $I \to 0$, כלומר $I = 0$ כמבוקש.
\end{proof}

\subquestion{}
נשתמש בהגדרת הגבול ונוכיח כי $f$ אנליטית ב־$B(0, r)$ כך ש־$f' = g$.
\begin{proof}
	\begin{align*}
		f'(z_0)
		& = \lim_{z \to z_0} \frac{f(z) - f(z_0)}{z - z_0} \\
		& = \lim_{z \to z_0} \frac{\int_{[0, z_0]} g(w)\ dw - \int_{[0, z]} g(w)\ dw}{z - z_0} \\
		& = \lim_{z \to z_0} \frac{\int_{[z, z_0]} g(w)\ dw}{z - z_0} \\
		& = \lim_{z \to z_0} \int_{[z, z_0]} \frac{g(w)}{z - z_0}\ dw \\
		& = \lim_{z \to z_0} \int_0^1 \frac{g(\gamma(t)) \cdot (z - z_0)}{z - z_0}\ dw \\
		& = \lim_{z \to z_0} \int_0^1 g(\gamma(t))\ dt \\
		& = \lim_{z \to z_0} g(z_0) - g(z - z_0) \\
		& = g(z_0)
	\end{align*}
\end{proof}

\question{}
תהי $K \subseteq \CC$ קבוצה קמורה קומפקטית.

\subquestion{}
לכל $\delta > 0$ תהי הקבוצה $K_\delta = \{z \in \CC \mid d(z, K) < \delta\}$. \\
נראה ש־$K_\delta$ קבוצה פתוחה וקמורה.
\begin{proof}
	תהי נקודה $z \in K$, אז ברור ש־$z \in K_\delta^\circ$ מהגדרה, ואילו $z \in K_\delta \setminus K$ אז קיימת $w \in K$ כך ש־$d(z, w) < \delta$, ולכן נוכל לבנות כדור פתוח עם $r < \delta - d(z, w)$ ונקבל שהנקודה פנימית. \\
	נעבור להוכחת הקמירות.
	תהינה שתי נקודות $z, z' \in K_\delta$, אז קיימות $w, w' \in K$ כך ש־$d(z, w), d(z', w') < \delta$. \\
	מקמירות $K$ נסיק $[w, w'] \subseteq K$ ונרצה להראות שגם $[z, z'] \subseteq K_\delta$.
	תהי $z (1 - t) + z' t$ כלשהי עבור $t \in [0, 1]$ ולכן
	\begin{align*}
		d(z (1 - t) + z' t, [w, w'])
		& = \inf_{0 \le s \le 1} d(z (1 - t) + z' t, w (1 - s) + w' s) \\
		& \le \inf_{0 \le s \le 1} |z (1 - t) + z' t - (w (1 - s) + w' s)| \\
		& \le |(z - w) (1 - t) + (z' - w') t| \\
		& \le |z - w| (1 - t) + |z' - w'| t \\
		& < \delta (1 - t) + \delta t \\
		& = \delta
	\end{align*}
	ומצאנו כי $K_\delta$ אכן קמורה.
\end{proof}

\subquestion{}
נוכיח כי לכל קבוצה פתוחה $K \subseteq U \subseteq \CC$ קיים $\delta > 0$ כך ש־$K_\delta \subseteq U$.
\begin{proof}
	מידת לבג של השפה של $K$ היא אפס ולכן קיים $\epsilon$־כיסוי לשפה, ולכן נובע ישירות ש־$K_{\epsilon / 2} \subseteq U$ ממשפטים רלוונטיים שהוכחו באינפי 3.
\end{proof}

\subquestion{}
נאמר שפונקציה $f : K \to \CC$ היא אנליטית אם קיימת קבוצה פתוחה $K \subseteq U \subseteq \CC$ ופונקציה אנליטית  $\tilde{f} : U \to \CC$ כך ש־$\tilde{f} \mid_K = f$. \\
נסיק שלכל פונקציה $f$ כזאת ניתן לבחור $U$ כך של־$\tilde{f}$ תהיה קיימת פונקציה קדומה.
\begin{proof}
	ממשפט קושי לקבוצות קמורות אנו יודעים שקיימת קדומה $F : K \to \CC$ כך ש־$F' = f$. \\
	עוד אנו יודעים ש־$F$ היא אנליטית ולכן ניתן לייצוג כטור חזקות בעל רדיוס התכנסות כלשהו המכסה את $K$, ונבחר מספר $\delta$ קטן מרדיוס זה, ולכן $\tilde{F} : K_\delta \to \CC$ רציפה ו־$\tilde{F} \mid_K = F$.
	נגדיר $\tilde{f} = \tilde{F}'$ ולכן $\tilde{f} : K_\delta \to \CC$ ו־$\tilde{f} \mid_K = f$.
\end{proof}

\end{document}
