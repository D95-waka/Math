\documentclass[a4paper]{article}

% packages
\usepackage{inputenc, fontspec, amsmath, amsthm, amsfonts, polyglossia, catchfile}
\usepackage[a4paper, margin=50pt, includeheadfoot]{geometry} % set page margins

% style
\AddToHook{cmd/section/before}{\clearpage}	% Add line break before section
\linespread{1.5}
\setcounter{secnumdepth}{0}		% Remove default number tags from sections
\setmainfont{Libertinus Serif}
\setsansfont{Libertinus Sans}
\setmonofont{Libertinus Mono}
\setdefaultlanguage{hebrew}
\setotherlanguage{english}

% operators
\DeclareMathOperator\cis{cis}
\DeclareMathOperator\Sp{Sp}
\DeclareMathOperator\tr{tr}
\DeclareMathOperator\im{Im}
\DeclareMathOperator\diag{diag}
\DeclareMathOperator*\lowlim{\underline{lim}}
\DeclareMathOperator*\uplim{\overline{lim}}

% commands
\renewcommand\qedsymbol{\textbf{משל}}
\newcommand{\NN}[0]{\mathbb{N}}
\newcommand{\ZZ}[0]{\mathbb{Z}}
\newcommand{\QQ}[0]{\mathbb{Q}}
\newcommand{\RR}[0]{\mathbb{R}}
\newcommand{\CC}[0]{\mathbb{C}}
\newcommand{\getenv}[2][] {
  \CatchFileEdef{\temp}{"|kpsewhich --var-value #2"}{\endlinechar=-1}
  \if\relax\detokenize{#1}\relax\temp\else\let#1\temp\fi
}
\newcommand{\explain}[2] {
	\begin{flalign*}
		 && \text{#2} && \text{#1}
	\end{flalign*}
}

% headers
\getenv[\AUTHOR]{AUTHOR}
\author{\AUTHOR}
\date\today

\title{פתרון מטלה 08 --- פונקציות מרוכבות, 80519}

\begin{document}
\maketitle
\maketitleprint{}

\question{}
נמצא את פיתוחי הטיילור ואת רדיוסי ההתכנסות לפונקציות הנתונות.

\subquestion{}
\[
	f(z) = \frac{1}{1 + \sqrt{1 - z}}
\]
סביב $z = 0$.
\begin{solution}
	מתקיים
	\[
		\frac{1}{1 - \sqrt{1 - z}}
		= \frac{1 - \sqrt{1 - z}}{1 - (1 - z)}
		= \frac{1 - \sqrt{1 - z}}{z}
	\]
	ועוד אנו יודעים כי
	\[
		1 - \sqrt{1 - z}
		= 1 - {((-z) + 1)}^\frac{1}{2}
		= 1 - \sum_{n = 0}^{\infty} \binom{\frac{1}{2}}{n} {(-z)}^n
		= 1 + \sum_{n = 0}^{\infty} \binom{\frac{1}{2}}{n} {(-1)}^{n + 1} z^n
		= \sum_{n = 1}^{\infty} \binom{\frac{1}{2}}{n} {(-1)}^{n + 1} z^n
	\]
	ולכן
	\[
		f(z)
		= \frac{\sum_{n = 1}^{\infty} \binom{\frac{1}{2}}{n} {(-1)}^{n + 1} z^n}{z}
		= \sum_{n = 1}^{\infty} \binom{\frac{1}{2}}{n} {(-1)}^{n + 1} \frac{z^n}{z}
		= \sum_{n = 0}^{\infty} \binom{\frac{1}{2}}{n + 1} {(-1)}^n z^n
	\]

	נבחין כי הפונקציה רציפה ב־$B(0, 1)$, ולכן ממסקנה מהתרגול רדיוס ההתכנסות הוא $r = 1$.
\end{solution}

\subquestion{}
\[
	f(z) = \Log(z)
\]
סביב $z = -1 + i$.
\begin{solution}
	מתקיים
	\[
		\Log(z)
		= \Log(-1 + i + (z + 1 - i)) - \Log(-1 + i) + \Log(-1 + i)
		= \Log(1 + \frac{z + 1 - i}{-1 + i}) + \Log(-1 + i)
	\]
	ומפיתוח טיילור של לוגריתם נסיק
	\[
		f(z)
		= \Log(-1 + i) + \sum_{n = 1}^{\infty} \frac{{(-1)}^{n - 1}}{n} {(\frac{z - (-1 + i)}{-1 + i})}^n
		= \Log(-1 + i) + \sum_{n = 1}^{\infty} \frac{{(-1)}^{n - 1}}{n {(-1 + i)}^n} {(z - (-1 + i))}^n
	\]

	כמובן הלוגריתם המרוכב הולומורפי בכל $\CC \setminus \{ 0 \}$, ולכן $r = 1$.
\end{solution}

\subquestion{}
\[
	f(z) = \frac{z^2 + z - 1}{z^3 - z}
\]
סביב $z = i$.
\begin{solution}
	נבחין כי
	\begin{align*}
		f(z)
		& = \frac{z^2 + z - 1}{z(z - 1)(z + 1)} \\
		& = \frac{A}{z} + \frac{B}{z - 1} + \frac{C}{z + 1} \\
		& = \frac{A(z^2 - 1) + B(z + z^2) + C(z^2 - z)}{z^3 - z} \\
		& = \frac{z^2 (A + B + C) + z(B - C) - A}{z^3 - z}
	\end{align*}
	ולכן נובע $A = 1, B = \frac{1}{2}, C = -\frac{1}{2}$ וכן
	\[
		f(z)
		= \frac{1}{z} + \frac{1}{2(z - 1)} - \frac{1}{2(z + 1)}
		= \frac{1}{i} \cdot \frac{1}{1 - \frac{z - i}{-i}} + \frac{1}{2(-1 + i)} \cdot \frac{1}{1 - \frac{z - i}{1 - i}} - \frac{1}{2(1 + i)} \frac{1}{1 - \frac{z - i}{-1 - i}}
	\]
	ולכן מפיתוח טיילור של $\frac{1}{1 - z}$ נובע
	\begin{align*}
		f(z)
		& = \frac{1}{i} \sum_{n = 0}^{\infty} {\left(\frac{z - i}{-i}\right)}^n + \frac{1}{2(-1 + i)} \sum_{n = 0}^{\infty} {\left(\frac{z - i}{1 - i}\right)}^n
		- \frac{1}{2(1 + i)} \sum_{n = 0}^{\infty} {\left(\frac{z - i}{-1 - i}\right)}^n \\
		& = \sum_{n = 0}^{\infty} \frac{1}{i} \frac{1}{{(-i)}^n} {(z - i)}^n + \sum_{n = 0}^{\infty} \frac{1}{2(-1 + i)} \frac{1}{{(1 - i)}^n} {(z - i)}^n
		+ \sum_{n = 0}^{\infty} - \frac{1}{2(1 + i)} \frac{1}{{(-i - i)}^n} {(z - i)}^n \\
		& = \sum_{n = 0}^{\infty} \left(\frac{1}{i} \frac{1}{{(-i)}^n} + \frac{1}{2(-1 + i)} \frac{1}{{(1 - i)}^n} - \frac{1}{2(1 + i)} \frac{1}{{(-i - i)}^n}\right) {(z - i)}^n \\
	\end{align*}
	הפונקציה כמובן אנליטית ב־$\CC \setminus \{0, 1, -1\}$ ולכן $r = 1$.
\end{solution}

\question{}
הוכחת משפט מוררה.
נוכיח שאם $f : G \to \CC$ רציפה כל שלכל משולש מלא $T \subseteq G$ מתקיים
\[
	\int_{\partial T} f(z)\ dz = 0
\]
אז $f$ אנליטית.
\begin{proof}
	נניח ללא הגבלת הכלליות ש־$G$ קמורה, ותהי $z_0 \in G$ כך ש־$[z_0, z] \subseteq G$ לכל $z \in G$.
	נגדיר $F : G \to \CC$ על־ידי
	\[
		F(z) = \int_{[z_0, z]} f(w)\ dw
	\]
	פונקציה זו כמובן מוגדרת בכל נקודה בשל קמירות $G$.
	תהי נקודה $z' \in G$ כך ש־$|z - z'| < \epsilon$, ולכן מתקיים מהגדרת הפונקציה שמתקיים
	\[
		F(z) - F(z') + \int_{[z, z']} f(w)\ dw = 0
		\iff
		F(z) - F(z') = \int_{[z', z]} f(w)\ dw
	\]
	ולכן מאי־שוויון ML
	\[
		0
		\le \left\lvert \frac{F(z) - F(z') - f(z)}{z - z'} \right\rvert
		\le \max_{t \in [z, z']} |f(t) - f(z)|
		\xrightarrow[z' \to z]{} 0
	\]
	ולכן $F'(z) = f(z)$, ובפרט $F$ אנליטית ולכן ממשפט טיילור גזירה אינסוף פעמים, כלומר קיימת נגזרת ל־$f$ עצמה.
\end{proof}

\question{}
הוכחת משפט ויירשטראס.
נוכיח שאם ${\{f_n\}}_{n = 1}^\infty \subseteq C^1(G)$ כך שהסדרה מתכנסת במידה שווה מקומית ל־$f$, אז $f$ אנליטית.
נראה שלכל $k \in \NN$ סדרת הנגזרות $(f_n^{(k)})$ מתכנסת במידה שווה מקומית ל־$f^{(k)}$.
\begin{proof}
	יהי משולש $T \subseteq G$,
	\[
		\left\lvert \int_{\partial T} f(z)\ dz - \int_{\partial T} f_n(z)\ dz \right\rvert
		= \left\lvert \int_{\partial T} f(z) - f_n(z)\ dz \right\rvert
		\le \int_{\partial T} |f(z) - f_n(z)|\ dz 
		\xrightarrow[n \to \infty]{}
		\int_{\partial T} |f(z) - f(z)|\ dz 
		= 0
	\]
	אבל גם לכל $n \in \NN$ אנו יודעים שמתקיים
	\[
		\int_{\partial T} f_n(z)\ dz = 0
	\]
	ולכן נוכל להסיק
	\[
		\int_{\partial T} f(z)\ dz = 0
	\]
	לכן ממשפט מוררה $f$ אנליטית. \\
	ממשפט טיילור ושוויון האינטגרלים נוכל להסיק שגם הנגזרות מתכנסות לכל סדר גזירה.
\end{proof}

\question{}
תהי $f : \CC \to \CC$ פונקציה שלמה, נוכיח כי $f$ קבועה עבור התנאים הבאים.

\subquestion{}
$\re(f(z)) \le 0$ לכל $z \in \CC$.
\begin{proof}
	נגדיר $g(z) = e^z$ ולכן
	\[
		|(g \circ f)(z)|
		= e^{\re(f(z))} |e^{i\im(f(z))}|
		= e^{\re f(z)}
		\le e^0
		= 1
	\]
	ולכן ממשפט ליוביל $f \circ g$ היא פונקציה קבועה, ובהתאם גם $f$ עצמה קבועה (אחרת נקבל ש־$g$ קבועה וזו סתירה).
\end{proof}

\subquestion{}
$|f(z)| \ne 1$ לכל $z \in \CC$.
\begin{proof}
	מרציפות כפונקציה דו־משתנית אנו יכולים להסיק ש־$|f| < 1$ או $|f| > 1$. \\
	אם $|f(z)| < 1$ אז ממשפט ליוביל סיימנו, לכן נניח $|f(z)| > 1$ תמיד, ונגדיר $g(z) = \frac{1}{z}$, לכן $|g \circ f| < 1$ תמיד ונוכל להסיק ש־$f$ קבועה בהכרח.
\end{proof}

\subquestion{}
$f(z) \notin (-\infty, 0]$ לכל $z \in \CC$. % chktex 9
\begin{proof}
	נגדיר $g(z) = z^i = e^{i \Log z}$, ולכן מהגדרת $f$ נוכל להסיק שמתקיים
	\[
		|g(f(z))|
		= |e^{i \log|z| - \Arg(z)}|
		= |e^{i \log|z|}| \cdot |e^{-\Arg(z)}|
		\le 1 \cdot e^{\pi}
	\]
	ולכן נובע ש־$f$ קבועה.
\end{proof}

\subquestion{}
$f(z) \notin [0, 1]$ לכל $z \in \CC$.
\begin{proof}
	במטלה 4 תרגיל 4 מצאנו שהעתקת מביוס נקבעת ביחידות על־ידי הנקודות $z_1 \to 0, z_2 \to 1, z_3 \to \infty$, נבחר את הנקודות $i, 0, \frac{1}{2}$ בהתאמה ונקבל את ההעתקה
	\[
		g(z)
		= \frac{0 - \frac{1}{2}}{0 - i} \cdot \frac{z - i}{z - \frac{1}{2}}
		= \frac{z - i}{2i z - i}
	\]
	מהגדרתה נוכל להסיק ש־$g \circ f$ חסומה על־ידי $|0 + i| = 1$ ולכן $f$ פונקציה קבועה.
\end{proof}

\question{}
תהי $f : \overline{B}(z_0, r) \to \CC$ פונקציה רציפה בכל תחומה ואנליטית ב־$B(z_0, r)$. \\
נגדיר את $R_k$ להיות שארית טור טיילור מסדר $k$ סביב $z = z_0$, דהינו
\[
	R_k(z)
	= f(z) - \sum_{n = 0}^{k} \frac{f^{(n)}(z_0)}{n!} {(z - z_0)}^n
	= \sum_{n = k + 1}^{\infty} \frac{f^{(n)}(z_0)}{n!} {(z - z_0)}^n
\]

\subquestion{}
נוכיח כי לכל $z \in B(z_0, r)$ מתקיים
\[
	R_k(z)
	= \frac{{(z - z_0)}^{k + 1}}{2\pi i} \int_{\partial B(z_0, r)} \frac{f(w)}{(w - z){(w - z_0)}^{k + 1}}\ dw
\]
\begin{proof}
	מההגדרה של $R_k$,
	\[
		R_k(z)
		= \sum_{n = k + 1}^{\infty} \frac{f^{(n)}(z_0)}{n!} {(z - z_0)}^n
		= \sum_{n = 0}^{\infty} \frac{f^{(k + 1 + n)}(z_0)}{(k + 1 + n)!} {(z - z_0)}^{k + 1 + n}
		= {(z - z_0)}^{k + 1} \sum_{n = 0}^{\infty} \frac{f^{(k + 1 + n)}(z_0)}{(k + 1 + n)!} {(z - z_0)}^n
	\]
	ממשפט טיילור נובע
	\begin{align*}
		& {(z - z_0)}^{k + 1} \sum_{n = 0}^{\infty} \frac{f^{(k + 1 + n)}(z_0)}{(k + 1 + n)!} {(z - z_0)}^n \\
		= & {(z - z_0)}^{k + 1} \sum_{n = 0}^{\infty} \frac{(n + k + 1)!}{2\pi i} \int_{\partial B(z_0, r)} \frac{f(w)}{{(w - z_0)}^{k + n + 2}} \frac{1}{(k + 1 + n)!} {(z - z_0)}^n\ dw \\
		= & \frac{{(z - z_0)}^{k + 1}}{2\pi i} \sum_{n = 0}^{\infty} \int_{\partial B(z_0, r)} \frac{f(w)}{{(w - z_0)}^{k + n + 2}} {(z - z_0)}^n\ dw \\
		= & \frac{{(z - z_0)}^{k + 1}}{2\pi i} \int_{\partial B(z_0, r)} \frac{f(w)}{{(w - z_0)}^{k + 2}} \cdot \frac{1}{2 - z_0} \sum_{n = 0}^{\infty} \frac{{(z - z_0)}^n}{{(w - z_0)}^n}\ dw \\
		\overset{(1)}{=} & \frac{{(z - z_0)}^{k + 1}}{2\pi i} \int_{\partial B(z_0, r)} \frac{f(w)}{{(w - z_0)}^{k + 1}} \cdot \frac{1}{w - z}\ dw
	\end{align*}
	כאשר $(1)$ נובע מהזהות $\frac{1}{w - z} = \frac{1}{w - z_0} \sum_{n = 0}^{\infty} {\left(\frac{z - z_0}{w - z_0}\right)}^n$ הידועה.
\end{proof}

\subquestion{}
נסיק שמתקיים
\[
	|R_k(z)|
	\le \left( \frac{r}{r - |z - z_0|} \cdot \max_{|w - z_0| = r} |f(w)| \right) {\left(\frac{|z - z_0|}{r}\right)}^{k + 1}
\]
\begin{proof}
	\begin{align*}
		|R_k(z)|
		& = \left\lvert \frac{{(z - z_0)}^{k + 1}}{2\pi i} \int_{\partial B(z_0, r)} \frac{f(w)}{(w - z){(w - z_0)}^{k + 1}}\ dw \right\rvert \\
		& = \frac{{|z - z_0|}^{k + 1}}{2\pi} \left\lvert \int_{\partial B(z_0, r)} \frac{f(w)}{(w - z){(w - z_0)}^{k + 1}}\ dw \right\rvert \\
		& \overset{\text{ML}}{\le} \frac{{|z - z_0|}^{k + 1}}{2\pi} L(\partial B(z_0, r)) \cdot \max_{w \in \partial B(z_0, r)} \left\lvert \frac{f(w)}{(w - z){(w - z_0)}^{k + 1}} \right\rvert \\
		& = {|z - z_0|}^{k + 1} r \cdot \max_{w \in \partial B(z_0, r)} \frac{|f(w)|}{|w - z| \cdot {|w - z_0|}^{k + 1}}  \\
		& \le \left(r \cdot \max_{w \in \partial B(z_0, r)} \frac{|f(w)|}{|w - z_0 + z_0 - z|}\right) {\left(\frac{|z - z_0|}{r}\right)}^{k + 1} \\
		& \le \left( \frac{r}{r - |z - z_0|} \cdot \max_{|w - z_0| = r} |f(w)| \right) {\left(\frac{|z - z_0|}{r}\right)}^{k + 1}
	\end{align*}
	ומצאנו כי אי־השוויון אכן מתקיים.
\end{proof}

\end{document} % chktex 17
