\documentclass[a4paper]{article}

% packages
\usepackage{inputenc, amsmath, amsthm, thmtools, amsfonts, amssymb, luacode, catchfile, tikzducks, hyperref}
\usepackage[a4paper, margin=50pt, includeheadfoot]{geometry} % set page margins
\usepackage[shortlabels]{enumitem}
\usepackage[skip=3pt, indent=0pt]{parskip}

% language
\usepackage[bidi=basic, layout=tabular, provide=*]{babel}
\babelprovide[main, import]{hebrew}
\babelprovide{rl}
\babelfont{rm}{Libertinus Serif}
\babelfont{sf}{Libertinus Sans}
\babelfont{tt}{Libertinus Mono}

% style
\AddToHook{cmd/section/before}{\clearpage}	% Add line break before section
\linespread{1.3}
\setcounter{secnumdepth}{0}		% Remove default number tags from sections, this won't do well with theorems
\AtBeginDocument{\setlength{\belowdisplayskip}{3pt}}
\AtBeginDocument{\setlength{\abovedisplayskip}{3pt}}

% operators
\DeclareMathOperator\cis{cis}
\DeclareMathOperator\Sp{Sp}
\DeclareMathOperator\tr{tr}
\DeclareMathOperator\im{Im}
\DeclareMathOperator\re{Re}
\DeclareMathOperator\diag{diag}
\DeclareMathOperator*\lowlim{\underline{lim}}
\DeclareMathOperator*\uplim{\overline{lim}}
\DeclareMathOperator\rng{rng}
\DeclareMathOperator\Sym{Sym}
\DeclareMathOperator\Arg{Arg}
\DeclareMathOperator\Log{Log}
\DeclareMathOperator\dom{dom}

% commands
%\renewcommand\qedsymbol{\textbf{מש''ל}}
%\renewcommand\qedsymbol{\fbox{\emoji{lizard}}}
\newcommand{\NN}[0]{\mathbb{N}}
\newcommand{\ZZ}[0]{\mathbb{Z}}
\newcommand{\QQ}[0]{\mathbb{Q}}
\newcommand{\RR}[0]{\mathbb{R}}
\newcommand{\CC}[0]{\mathbb{C}}
\newcommand{\FF}[0]{\mathbb{F}}
\newcommand{\PP}[0]{\mathbb{P}}
\newcommand{\TT}[0]{\mathbb{T}}
\newcommand{\acts}[0]{\circlearrowright}
\newcommand{\explain}[2] {
	\begin{flalign*}
		 && \text{#2} && \text{#1}
	\end{flalign*}
}
\newcommand{\maketitleprint}[0]{ \begin{center}
	\begin{tikzpicture}[scale=3]
		\duck[graduate=gray!20!black, tassel=red!70!black]
	\end{tikzpicture}	
\end{center}
}

% theorem commands
\newtheoremstyle{c_remark}
	{}	% Space above
	{}	% Space below
	{}% Body font
	{}	% Indent amount
	{\bfseries}	% Theorem head font
	{}	% Punctuation after theorem head
	{.5em}	% Space after theorem head
	{\thmname{#1}\thmnumber{ #2}\thmnote{ \normalfont{\text{(#3)}}}}	% head content
\newtheoremstyle{c_definition}
	{3pt}	% Space above
	{3pt}	% Space below
	{}% Body font
	{}	% Indent amount
	{\bfseries}	% Theorem head font
	{}	% Punctuation after theorem head
	{.5em}	% Space after theorem head
	{\thmname{#1}\thmnumber{ #2}\thmnote{ \normalfont{\text{(#3)}}}}	% head content
\newtheoremstyle{c_plain}
	{3pt}	% Space above
	{3pt}	% Space below
	{\itshape}% Body font
	{}	% Indent amount
	{\bfseries}	% Theorem head font
	{}	% Punctuation after theorem head
	{.5em}	% Space after theorem head
	{\thmname{#1}\thmnumber{ #2}\thmnote{ \text{(#3)}}}	% head content

\theoremstyle{c_plain}
\newtheorem{theorem}{משפט}[section]
\newtheorem{lemma}[theorem]{למה}
\newtheorem{proposition}[theorem]{טענה}
\newtheorem*{proposition*}{טענה}
%\newtheorem{corollary}[theorem]{אין חלופה עברית}

\theoremstyle{c_definition}
\newtheorem{definition}[theorem]{הגדרה}
\newtheorem*{definition*}{הגדרה}
\newtheorem{example}{דוגמה}[section]
\newtheorem{exercise}{תרגיל}[section]

\theoremstyle{c_remark}
\newtheorem*{remark}{הערה}
\newtheorem*{solution}{פתרון}
\newtheorem{conclusion}[theorem]{מסקנה}
\newtheorem{notation}[theorem]{סימון}

% Questions related commands
\newcounter{question}
\setcounter{question}{1}
\newcounter{sub_question}
\setcounter{sub_question}{1}

\newcommand{\question}[1][0]{
	\ifthenelse{#1 = 0}{}{\setcounter{question}{#1}}
	\subsection{שאלה \arabic{question}}
	\addtocounter{question}{1}
	\setcounter{sub_question}{1}
}

\newcommand{\subquestion}[1][0]{
	\ifthenelse{#1 = 0}{}{\setcounter{sub_question}{#1}}
	\subsubsection{סעיף \localecounter{letters.gershayim}{sub_question}}
	\addtocounter{sub_question}{1}
}

% import lua and start of document
\directlua{common = require ('../common')}

\GetEnv{AUTHOR}

% headers
\author{\AUTHOR}
\date\today

\title{פתרון מטלה 09 --- פונקציות מרוכבות, 80519}

\begin{document}
\maketitle
\maketitleprint{}

\question{}
בכל סעיף נוכיח שלא קיימת פונקציה שלמה המקיימת את הטענה או נביא דוגמה לפונקציה כזו.

\subquestion{}
לכל $n \ge 1$,
\[
	f(\frac{1}{n}) = \frac{{(-1)}^n}{n}
\]
\begin{proof}
	נניח בשלילה שקיימת פונקציה כזו, ונגדיר
	\[
		g(z) = f(z) - z
	\]
	לכן
	\[
		g(\frac{1}{2n})
		= f(\frac{1}{2n}) - \frac{1}{2n}
		= 0
		\xrightarrow[n \to \infty]{} 0
	\]
	וממשפט היחידות השני נובע $g \equiv 0$, לכן $f(z) = z$, אבל זו סתירה ל־$f(1) = -1$ ולכן לא קיימת $f$ כזו.
\end{proof}

\subquestion{}
עבור $n \ge 1$,
\[
	f(n) = {(-1)}^n n
\]
\begin{solution}
	נגדיר
	\[
		f(z) = e^{i\pi z} z
	\]
	ולכן נובע ישירות כי
	\[
		f(n)
		= {(-1)}^n n
	\]
	כפי שרצינו, ונבחין כי $f$ אכן שלמה, כהרכבת פונקציות שלמות.
\end{solution}

\subquestion{}
עבור $n \ge 1$,
\[
	f(\frac{1}{n^2}) = \frac{1}{n}
\]
\begin{proof}
	נניח בשלילה ש־$f$ כזו קיימת ונגדיר
	\[
		g(z) = f^2(z) - z
	\]
	זוהי כמובן פונקציה שלמה, ונבחין כי גם
	\[
		g(\frac{1}{n^2})
		= f^2(\frac{1}{n^2}) - \frac{1}{n^2}
		= 0
		\xrightarrow[n \to \infty]{} 0
	\]
	וממשפט היחידות השני נקבל $g \equiv 0$, כלומר $f^2 = z \iff f(z) = \sqrt{z}$.
	אבל $f$ לא גזירה ב־$0$ בסתירה לשלמותה.
\end{proof}

\subquestion{}
עבור $n \ge 1$,
\[
	f(\frac{1}{n})
	= \frac{n}{n + 1}
\]
\begin{proof}
	נניח בשלילה ש־$f$ כזו קיימת ונגדיר
	\[
		g(z)
		= (1 + z)f(z)
	\]
	נבחין כי
	\[
		g(\frac{1}{n})
		= (1 + \frac{1}{n})\frac{n}{n + 1}
		= \frac{n + 1}{n}\frac{n}{n + 1}
		= 1
		\xrightarrow[n \to \infty]{} 1
	\]
	ולכן שוב $g \equiv 1 \implies f(z) = \frac{1}{z + 1}$ לכל $z \ne -1$. \\
	אבל עבור $z = -1$ מתקיים $1 = 0 \cdot f(z)$ וזו כמובן סתירה.
\end{proof}

\question{}
תהי פונקציה שלמה המקיימת שלכל $z_0 \in \CC$ קיים $n \in \NN$ כך ש־$f^{(n)}(z_0) = 0$. \\
נוכיח כי $f$ היא פולינום.
\begin{proof}
	תהי $z_0 \in \CC$ כלשהי, אז ממשפט טיילור והעובדה שקיים סדר ממנו הנגזרת מתאפסת נוכל לקבוע כי קיים פולינום $P_0$ כך ש־$f \equiv P_0$ סביב הנקודה. \\
	נבחר נקודה $z_1$ כך שערכה לא נקבע על־ידי $P_0$ ונקבל שקיים פולינום סביבה $P_1$. כל סביבה כזו לא מנוונת ולכן נוכל לבנות כיסוי ל־$\CC$ על־ידי $\langle P_i \mid i < \omega \rangle$. \\
	מהלמה של צורן נובע כמעט מיד שקיים $n$ מקסימלי בקבוצה זו, כלומר קיים $n \in \NN$ כך ש־$f^{(n)}(z) = 0$ לכל $z \in \CC$, ומכאן נסיק ישירות ש־$f$ אכן פולינום.
\end{proof}

\question{}
נגדיר $S = \partial B(0, 1)$ ונוכיח כי לכל $z_1, \dots, z_n \in S$ קיימת נקודה $z \in S$ כך שמתקיים
\[
	\prod_{k = 1}^n |z - z_k| = 1
\]
\begin{proof}
	תהינה קבוצת נקודות כזו ונגדיר את הפונקציה $f : \CC \to \CC$ על־ידי
	\[
		f(z) = \prod_{k = 1}^n (z - z_k)
	\]
	נבחין כי $f$ היא פונקציה הולומורפית, וכן
	\[
		|f(0)| = |z_1| \cdots |z_n| = 1 \cdots 1 = 1
	\]
	מעיקרון המקסימום נוכל להסיק שבהכרח $\forall z \in S, |f(z)| \ge 1$.
	בנוסף מתקיים $f(z_0) = 0 \cdot (z_1 - z_0) \cdots (z_n - z_0) = 0$ ולכן גם $|f(z_0)| = 0$.
	לבסוף נגדיר את המסילה $\gamma(t) = e^{it}$ בתחום $[-\pi, \pi]$ ונקבל $g(t) = |f(\gamma(t))|$ פונקציה $[-\pi, \pi] \to \RR^+$ רציפה כך שקיים $g(t_0) = 0$ ו־$g(t_1) \ge 1$ ולכן מערך הביניים קיים $t$ כך ש־$g(t) = 1$.
\end{proof}

\question{}
נגדיר $S = \{ z \in \CC \mid 0 < \re(z) < 1\}$ ותהי $f : \overline{S} \to \CC$ פונקציה לא קבועה רציפה ב־$\overline{S}$ ואנליטית ב־$S$.

\subquestion{}
נוכיח את הגרסה הבאה של עקרון פרגמן־לינדלוף, \\
נניח ש־$\log |f(z)| \le C {|\im(z)|}^b$ עבור $C > 0$ ו־$0 \le b < 2$ קבועים, כאשר $|\im(z)| \to \infty$, \\
אז אם $|f(z)| \le 1$ על $\partial S$ מתקיים $|f(z)| \le 1$ על $S$.
\begin{proof}
	נגדיר $f_\epsilon : [0, 1] \times [-r, r]$ על־ידי $f_\epsilon(z) = f(z) e^{\epsilon (z^2 - 1)}$ ונבחין כי
	\[
		|e^{\epsilon (z^2 - 1)}|
		\le e^{\epsilon \re(z^2 - 1)}
		\le e^{\epsilon({|\re z|}^2 - {|\im z|}^2 - 1)}
		\le e^{-\epsilon {|\im z|}^2}
	\]
	ולכן נסיק ש־$|f_\epsilon(z)| \le |f(z)|$ וגם
	\[
		|f_\epsilon(z)|
		\le \exp(-\epsilon {|\im z|}^2 + C {|\im z|}^b)
		\xrightarrow[z \to \infty]{} 0
	\]
	לכן עבור ערכים גדולים מספיק נקבל $|f_\epsilon(z)| < 1$, אבל אז מעיקרון המקסימום והעובדה ש־$|f(z)| \le 1$ עבור $z \in \partial S$ נסיק ש־$|f_\epsilon(z)| \le 1$ לכל $z \in \dom f_\epsilon$.
	אבל נבחין ש־$f_\epsilon \xrightarrow[\epsilon \to 0]{} f$ ולכן גם $|f(z)| \le 1$ לכל $z \in S^\circ$ כמבוקש.
\end{proof}

\subquestion{}
נניח ש־$f$ חסומה.
נגדיר $M : [0, 1] \to \RR$ על־ידי
\[
	M(x) = \sup_{y \in \RR} |f(x + iy)|
\]
נוכיח שמתקיים
\[
	M(x) \le {(M(0))}^{1 - x} {(M(1))}^x
\]
\begin{proof}
	נגדיר $g(z) = f(z) {(M(0))}^{z - 1} {(M(1))}^{-z}$. \\
	נרצה להראות ש־$|g(z)| \le 1$ על $\partial S$, ולכן
	\[
		|g(0 + yi)|
		= |f(0 + yi)| M^{\re(0 + yi - 1)}(0) M^{\re(yi)}(1)
		\le M^{-1}(0)
	\]
	אם נניח ש־$M(0) = 0$ נקבל $f(0 + yi) = 0$ ולכן ממשפט היחידות $f \equiv 0$ ונסיק ש־$M(0) > 0$ ולכן $|g(0 + yi)| \le 1$ בהכרח. \\
	באופן דומה נקבל שגם $|g(1 + yi)| \le 1$ ולכן $|g(z)| \le 1$ לכל $z \in \partial S$ ונקבל ש־$|g(z)| \le 1$ לכל $z \in S$. \\
	לכן נובע
	\[
		|g(x + 0i)|
		= |f(x + 0i)| |{(M(0))}^{x + 0i - 1} {(M(1))}^{-(x + 0i)}|
		= |f(x)| {M(0)}^{x - 1} {M(1)}^{-x}
		\le 1
		\iff
		|f(x)|
		\le {M(0)}^{1 - x} {M(1)}^x
	\]
	ולבסוף
	\[
		M(x)
		= \sup_{y \in \RR} |f(x + iy)|
		\le |f(x + 0i)|
		\le {M(0)}^{1 - x} {M(1)}^x
	\]
	כפי שרצינו.
\end{proof}

\subquestion{}
נסיק כי הפונקציה $\log M$ קמורה, \\
כלומר לכל $x_0, x_1, t \in [0, 1]$ מתקיים
\[
	g((1 - t)x_0 + tx_1) \le (1 - t) g(x_0) + t g(x_1)
\]
\begin{proof}
	נקבע $x_0, x_1$ ונגדיר $h(z) = g(x_0 + (x_1 - x_0)z)$, וכן $N(x) = \sup |h(x + iy)|$ אז נקבל $N(x) \le {N(0)}^{1 - x} {N(1)}^x$ כמו בסעיף הקודם ונובע
	\begin{align*}
		\log N(t)
		& = \log(M(x_0 + (x_1 - x_0)t)) \\
		& \le \log({N(0)}^{1 - t} {N(1)}^t) \\
		& = (1 - t) \log(N(0)) + t \log(N(1)) \\
		& = (1 - t) \log(M(x_0)) + t \log(M(x_1))
	\end{align*}
	ומצאנו ש־$M$ אכן קמורה.
\end{proof}

\end{document} % chktex 17
