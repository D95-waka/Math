\documentclass[a4paper]{article}

% packages
\usepackage{inputenc, amsmath, amsthm, thmtools, amsfonts, amssymb, luacode, catchfile, tikzducks, hyperref}
\usepackage[a4paper, margin=50pt, includeheadfoot]{geometry} % set page margins
\usepackage[shortlabels]{enumitem}
\usepackage[skip=3pt, indent=0pt]{parskip}

% language
\usepackage[bidi=basic, layout=tabular, provide=*]{babel}
\babelprovide[main, import]{hebrew}
\babelprovide{rl}
\babelfont{rm}{Libertinus Serif}
\babelfont{sf}{Libertinus Sans}
\babelfont{tt}{Libertinus Mono}

% style
\AddToHook{cmd/section/before}{\clearpage}	% Add line break before section
\linespread{1.3}
\setcounter{secnumdepth}{0}		% Remove default number tags from sections, this won't do well with theorems
\AtBeginDocument{\setlength{\belowdisplayskip}{3pt}}
\AtBeginDocument{\setlength{\abovedisplayskip}{3pt}}

% operators
\DeclareMathOperator\cis{cis}
\DeclareMathOperator\Sp{Sp}
\DeclareMathOperator\tr{tr}
\DeclareMathOperator\im{Im}
\DeclareMathOperator\re{Re}
\DeclareMathOperator\diag{diag}
\DeclareMathOperator*\lowlim{\underline{lim}}
\DeclareMathOperator*\uplim{\overline{lim}}
\DeclareMathOperator\rng{rng}
\DeclareMathOperator\Sym{Sym}
\DeclareMathOperator\Arg{Arg}
\DeclareMathOperator\Log{Log}
\DeclareMathOperator\dom{dom}

% commands
%\renewcommand\qedsymbol{\textbf{מש''ל}}
%\renewcommand\qedsymbol{\fbox{\emoji{lizard}}}
\newcommand{\NN}[0]{\mathbb{N}}
\newcommand{\ZZ}[0]{\mathbb{Z}}
\newcommand{\QQ}[0]{\mathbb{Q}}
\newcommand{\RR}[0]{\mathbb{R}}
\newcommand{\CC}[0]{\mathbb{C}}
\newcommand{\FF}[0]{\mathbb{F}}
\newcommand{\PP}[0]{\mathbb{P}}
\newcommand{\TT}[0]{\mathbb{T}}
\newcommand{\acts}[0]{\circlearrowright}
\newcommand{\explain}[2] {
	\begin{flalign*}
		 && \text{#2} && \text{#1}
	\end{flalign*}
}
\newcommand{\maketitleprint}[0]{ \begin{center}
	\begin{tikzpicture}[scale=3]
		\duck[graduate=gray!20!black, tassel=red!70!black]
	\end{tikzpicture}	
\end{center}
}

% theorem commands
\newtheoremstyle{c_remark}
	{}	% Space above
	{}	% Space below
	{}% Body font
	{}	% Indent amount
	{\bfseries}	% Theorem head font
	{}	% Punctuation after theorem head
	{.5em}	% Space after theorem head
	{\thmname{#1}\thmnumber{ #2}\thmnote{ \normalfont{\text{(#3)}}}}	% head content
\newtheoremstyle{c_definition}
	{3pt}	% Space above
	{3pt}	% Space below
	{}% Body font
	{}	% Indent amount
	{\bfseries}	% Theorem head font
	{}	% Punctuation after theorem head
	{.5em}	% Space after theorem head
	{\thmname{#1}\thmnumber{ #2}\thmnote{ \normalfont{\text{(#3)}}}}	% head content
\newtheoremstyle{c_plain}
	{3pt}	% Space above
	{3pt}	% Space below
	{\itshape}% Body font
	{}	% Indent amount
	{\bfseries}	% Theorem head font
	{}	% Punctuation after theorem head
	{.5em}	% Space after theorem head
	{\thmname{#1}\thmnumber{ #2}\thmnote{ \text{(#3)}}}	% head content

\theoremstyle{c_plain}
\newtheorem{theorem}{משפט}[section]
\newtheorem{lemma}[theorem]{למה}
\newtheorem{proposition}[theorem]{טענה}
\newtheorem*{proposition*}{טענה}
%\newtheorem{corollary}[theorem]{אין חלופה עברית}

\theoremstyle{c_definition}
\newtheorem{definition}[theorem]{הגדרה}
\newtheorem*{definition*}{הגדרה}
\newtheorem{example}{דוגמה}[section]
\newtheorem{exercise}{תרגיל}[section]

\theoremstyle{c_remark}
\newtheorem*{remark}{הערה}
\newtheorem*{solution}{פתרון}
\newtheorem{conclusion}[theorem]{מסקנה}
\newtheorem{notation}[theorem]{סימון}

% Questions related commands
\newcounter{question}
\setcounter{question}{1}
\newcounter{sub_question}
\setcounter{sub_question}{1}

\newcommand{\question}[1][0]{
	\ifthenelse{#1 = 0}{}{\setcounter{question}{#1}}
	\subsection{שאלה \arabic{question}}
	\addtocounter{question}{1}
	\setcounter{sub_question}{1}
}

\newcommand{\subquestion}[1][0]{
	\ifthenelse{#1 = 0}{}{\setcounter{sub_question}{#1}}
	\subsubsection{סעיף \localecounter{letters.gershayim}{sub_question}}
	\addtocounter{sub_question}{1}
}

% import lua and start of document
\directlua{common = require ('../common')}

\GetEnv{AUTHOR}

% headers
\author{\AUTHOR}
\date\today

\title{פתרון מטלה 02 --- פונקציות מרוכבות, 80519}

\begin{document}
\maketitle
\maketitleprint{}

\Question{}
תהינה $f, g : U \to \CC$ פונקציות אנליטיות המוגדרות על קבוצה פתוחה $U \subseteq \CC$.

\Subquestion{}
נוכיח כי $(f + g)'(z) = f'(z) + g'(z)$.
\begin{proof}
	מהגדרת הנגזרת נובע
	\[
		(f + g)'(z_0)
		= \lim_{z \to z_0} \frac{(f + g)(z) - (f + g)(z_0)}{z - z_0}
		= \lim_{z \to z_0} \frac{f(z) - f(z_0)}{z - z_0} + \frac{g(z) - g(z_0)}{z - z_0}
	\]
	אבל שני הביטויים הללו נתונים ולכן מאריתמטיקה
	\[
		(f + g)'(z_0)
		= \lim_{z \to z_0} \frac{f(z) - f(z_0)}{z - z_0} + \lim_{z \to z_0}  \frac{g(z) - g(z_0)}{z - z_0}
		= f'(z_0) + g'(z_0)
	\]
\end{proof}

\Subquestion{}
נוכיח כי $(f \cdot g)'(z) = f'(z) g(z) + f(z) g'(z)$.
\begin{proof}
	מהגדרת הנגזרת ומרציפות פונקציות גזירות נובע
	\begin{align*}
		(f \cdot g)'(z_0)
		& = \lim_{z \to z_0} \frac{(f \cdot g)(z) - (f \cdot g)(z_0)}{z - z_0} \\
		& = \lim_{z \to z_0} \frac{f(z) \cdot g(z) - f(z_0) \cdot g(z_0)}{z - z_0} \\
		& = \lim_{z \to z_0} \frac{(f(z) - f(z_0)) \cdot g(z) + f(z_0) g(z) - f(z_0) \cdot g(z_0)}{z - z_0} \\
		& = \lim_{z \to z_0} \frac{f(z) - f(z_0)}{z - z_0} \cdot g(z) + \frac{f(z_0) g(z) - f(z_0) \cdot g(z_0)}{z - z_0} \\
		& = f'(z_0) g(z_0) + f(z_0) \lim_{z \to z_0} \frac{g(z) - g(z_0)}{z - z_0} \\
		& = f'(z_0) g(z_0) + f(z_0) g'(z_0)
	\end{align*}
\end{proof}

\Subquestion{}
נוכיח כי $(f \circ g)'(z) = f'(g(z)) \cdot g'(z)$.
\begin{proof}
	מהגדרת הנגזרת נקבל
	\begin{align*}
		(f \circ g)'(z_0)
		& = \lim_{z \to z_0} \frac{(f \circ g)(z) - (f \circ g)(z_0)}{z - z_0} \\
		& = \lim_{z \to z_0} \frac{f(g(z)) - f(g(z_0))}{g(z) - g(z_0)} \cdot \frac{g(z) - g(z_0)}{z - z_0} \\
		& = f'(g(z_0)) \cdot g'(z_0)
	\end{align*}
\end{proof}

\Question{}
עבור הפונקציות הבאות נמצא את נקודות הגזירות ונקבע באילו נקודות היא אנליטית.

\Subquestion{}
\[
	f(x + iy) = (x^2 + y^2) + i (-x^2 + y^2)
\]
\begin{solution}
	נבחין כי אם נגזור את החלק מממשי נקבל
	\[
		\re(f(x + iy))' = (x^2 + y^2)' = (2x, 2y)
	\]
	ובאופן דומה
	\[
		\im(f(x + iy))' = (-x^2 + y^2)' = (-2x, 2y)
	\]
	כמובן $2y = 2y$ בכל התחום, אך $2x = - 2x \iff x = 0$ ולכן היא גזירה על הציר המדומה בלבד. \\*
	בהתאם אין נקודה פנימית בתחום הגזירות ולכן $f$ לא אנליטית לאף נקודה.
\end{solution}

\Subquestion{}
\[
	g(x + iy) = x^2 + 3iy
\]
גם הפעם נחשב
\[
	\re(g(x + iy))' = (3x^2, 0),
	\qquad
	\im(g(x + iy))' = (0, 3)
\]
אין נקודה בה $3 = 0$ ולכן $g$ לא גזירה באף נקודה.

\Subquestion{}
\[
	h(x + iy) = |x^2 - y^2| + 2i xy
\]
\begin{solution}
	נבחן את הנגזרות החלקיות כשאר $x^2 \ge y^2$:
	\[
		\re(h)' = (2x , -2y),
		\qquad
		\im(h)' = (2, 2)
	\]
	ונקבל גזירות כאשר $x = 1, y = -1$ בלבד. \\*
	נבחן את הנגזרות החלקיות בשאר המקרים:
	\[
		\re(h)' = (-2x, 2y),
		\qquad
		\im(h)' = (2, 2)
	\]
	ונקבל $x = -1, y = 1$ בלבד, אך לא מתקיים ${(-1)}^2 < 1^2$ ולכן נוכל להסיק כי $1 - i$ נקודת גזירות יחידה.
\end{solution}

\Subquestion{}
\[
	\forall a, b \in \CC,
	k(z) = az + b \overline{z}
\]
\begin{solution}
	ראינו כבר כי אם $b = 0$ אז $k$ גזירה ואנליטית בכל תחומה. \\*
	אילו $a = 0$ נקבל
	\[
		\lim_{z \to z_0} b\frac{\overline{z} - \overline{z_0}}{z - z_0}
		= b \lim_{z \to z_0} \frac{\overline{z - z_0}}{z - z_0}
	\]
	ומכאן ניתן לראות כי הצבה של סדרות ממשיות תניב נגזרת $1$ והצבת סדרות מדומות תניב $-1$ ו־$k$ איננה גזירה באף נקודה. \\*
	נניח $a, b \ne 0$ ונניח שקיימים ערכים עבורם $k$ גזירה, נקבל אם כך $(a z + b \overline{z})'$ גזיר, ומשאלה 1 נסיק $az' + b\overline{z}'$ ביטוי מוגדר, וזו כמובן סתירה לתוצאה שקיבלנו זה עתה. \\*
	נסיק שכל עוד $b = 0$ אז $k$ הולומורפית.
\end{solution}

\Question{}
תהי $f : G \to \CC$ פונקציה אנליטית המוגדרת על תחום $G \subseteq \CC$.

\Subquestion{}
נוכיח כי אם $\forall z \in G, f'(z) = 0$ אז $f$ בהכרח קבועה.
\begin{proof}
	תהינה שתי נקודות $z_0, z_1 \in G$, נניח גם $z_0 \ne z_1$. \\*
	נגדיר מסילה $l = [z_0, z_1]$, אז כמובן $l : \RR \to \CC$, ונבחן את הגרסה שלה בשני משתנים $l : \RR \to \RR^2$ במקום. \\*
	בגרסה זו נגזרתה מתלכדת עם הנגזרת הכיוונית $Df_{z_1 - z_0} = 0$, ולכן גם $\nabla l = {(0, 0)}^t$. \\*
	ידוע ש־$z_0 \ne z_1 \iff (\re(z_0) \ne \re(z_1) \lor \im(z_0) \ne \im(z_1))$ נניח בלי הגבלת הכלליות ש־$\re(z_0) \ne \re(z_1)$ ולכן אם $l = (l_1, l_2)$ מספיק שנבחן את $l_1$. \\*
	קיבלנו ש־$l_1 : \RR \to \RR$ וגם $l'(t) = 0$ לכל $0 \le t \le 1$, ולכן $l_1(0) = l_1(1)$ וזאת בסתירה להנחתנו, לכן $z_0 = z_1$ בלבד.
\end{proof}

\Subquestion{}
נוכיח כי אם $f(G) \subseteq \RR$ אז $f$ בהכרח קבועה.
\begin{proof}
\end{proof}

\end{document}
