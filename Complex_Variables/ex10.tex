\documentclass[a4paper]{article}

% packages
\usepackage{inputenc, fontspec, amsmath, amsthm, amsfonts, polyglossia, catchfile}
\usepackage[a4paper, margin=50pt, includeheadfoot]{geometry} % set page margins

% style
\AddToHook{cmd/section/before}{\clearpage}	% Add line break before section
\linespread{1.5}
\setcounter{secnumdepth}{0}		% Remove default number tags from sections
\setmainfont{Libertinus Serif}
\setsansfont{Libertinus Sans}
\setmonofont{Libertinus Mono}
\setdefaultlanguage{hebrew}
\setotherlanguage{english}

% operators
\DeclareMathOperator\cis{cis}
\DeclareMathOperator\Sp{Sp}
\DeclareMathOperator\tr{tr}
\DeclareMathOperator\im{Im}
\DeclareMathOperator\diag{diag}
\DeclareMathOperator*\lowlim{\underline{lim}}
\DeclareMathOperator*\uplim{\overline{lim}}

% commands
\renewcommand\qedsymbol{\textbf{משל}}
\newcommand{\NN}[0]{\mathbb{N}}
\newcommand{\ZZ}[0]{\mathbb{Z}}
\newcommand{\QQ}[0]{\mathbb{Q}}
\newcommand{\RR}[0]{\mathbb{R}}
\newcommand{\CC}[0]{\mathbb{C}}
\newcommand{\getenv}[2][] {
  \CatchFileEdef{\temp}{"|kpsewhich --var-value #2"}{\endlinechar=-1}
  \if\relax\detokenize{#1}\relax\temp\else\let#1\temp\fi
}
\newcommand{\explain}[2] {
	\begin{flalign*}
		 && \text{#2} && \text{#1}
	\end{flalign*}
}

% headers
\getenv[\AUTHOR]{AUTHOR}
\author{\AUTHOR}
\date\today

\title{פתרון מטלה 10 --- פונקציות מרוכבות, 80519}

\begin{document}
\maketitle
\maketitleprint{}

\question{}
עבור כל אחת מן הפונקציות הנתונות, נמצא את כל התחומים בהם ניתן לפתח טור לורן סביב נקודה נתונה, ונמצא את הפיתוחים.

\subquestion{}
$f(z) = \frac{1}{{(z^2 + 1)}^2}$ סביב $z = i$.
\begin{solution}
	נבחין כי $f$ לא מוגדרת ב־$\pm i$ בלבד, לכן נוכל לחלק את הפיתוח לתחומים $A_0^2(i)$ ו־$A_2^\infty(i)$. \\
	בתחום הראשון ישנה התלכדות עם טור טיילור של ולכן נפרק את הפונקציה ונחשב,
	\begin{align*}
		f(z)
		& = \frac{1}{{(z - i)}^2} \cdot \frac{1}{{(z + i)}^2} \\
		& = {(z - i)}^{-2} \cdot \int -\frac{1}{z + i} \\
		& = {(z - i)}^{-2} \cdot \int \frac{-1}{2i} \frac{1}{1 + (z - i)/2i} \\
		& = {(z - i)}^{-2} \cdot \int \frac{-1}{2i} \sum_{n = 0}^{\infty} {(-1)}^n \frac{1}{{(2i)}^n} {(z - i)}^n \\
		& = \sum_{n = 0}^{\infty} {(-1)}^{n - 1} \frac{1}{{(2i)}^{n - 1}} \frac{1}{n + 1} {(z - i)}^{n - 1} \\
		& = \sum_{n = -1}^{\infty} {(-1)}^n \frac{1}{{(2i)}^n} \frac{1}{n + 2} {(z - i)}^n
	\end{align*}
	נעבור לתחום השני ממהלך זהה למהלך בתרגיל נקבל
	\begin{align*}
		f(z)
		& = \sum_{n = -\infty}^{-2} {(-1)}^n \frac{1}{{(2i)}^n} \frac{1}{n + 2} {(z - i)}^n
	\end{align*}
\end{solution}

\subquestion{}
$f(z) = \frac{z^2 - 6z + 10}{z^2 - 7z + 12}$, סביב $z = 2$.
\begin{solution}
	נגדיר $w = z - 2$ ונקבל
	\[
		z(w) = \frac{w^2 - 2w + 2}{w^2 - 3w + 2}
		= 1 + \frac{w}{(w - 1)(w - 2)} 
		= 1 + \frac{1}{1 - w} - \frac{2}{1 - \frac{w}{2}}
	\]
	עבור $w$ אנו צריכים לפתח סביב $w = 0$, ונבחין כי $w = 1, 2$ נקודות אי הגדרה, לכן נפתח בתחומים $A_0^1, A_1^2, A_2^\infty$. \\
	עבור $A_0^1$ מתקבל
	\[
		f(w)
		= 1 + \sum_{n = 0}^{\infty} w^n - \sum_{n = 0}^{\infty} {(\frac{w}{2})}^n
	\]
	כלומר $c_0 = 1, c_n = 1 - \frac{1}{2^n}$ לכל $n$ חיובי.
	באופן דומה בתחום $A_1^2$ מתקבל
	\[
		f(w)
		= 1 + \frac{1}{w(\frac{1}{w} - 1)} - \frac{2}{1 - \frac{w}{2}}
		= 1 - \frac{1}{w} \sum_{n = 0}^{\infty} w^{-n} - \sum_{n = 0}^{\infty} {(\frac{w}{2})}^n
		= 1 - \sum_{n = -\infty}^{-1} w^{n} - \sum_{n = 0}^{\infty} {(\frac{w}{2})}^n
	\]
	ולבסוף בתחום $A_2^\infty$,
	\[
		f(w)
		= 1 - \frac{1}{w(1 - \frac{1}{w})} - \frac{1}{w} \frac{2}{1 - 2w}
		= 1 - \sum_{n = -\infty}^{-1} w^{n} - \sum_{n = -\infty}^{-1} {(\frac{w}{2})}^n
	\]
\end{solution}

\subquestion{}
$f(z) = \Log(\frac{z}{z - 1})$ סביב $z = 0$.
\begin{solution}
	נבחין כי הפעם הפונקציה לא מוגדרת ב־$z = 0, 1$ ולכן נחלק לתחומים $A_0^1, A_1^\infty$. \\
	בתחום $A_1^\infty$ נקבל
	\[
		f(z) = -\Log(1 - \frac{1}{z})
		= - \sum_{n = 1}^{\infty} \frac{{(-1)}^{n - 1}}{n} \frac{1}{{(-z)}^n}
		= \sum_{n = -\infty}^{-1} \frac{1}{n} z^n
	\]
	ובתחום $A_0^1$ נקבל
	\[
		f(z)
		= - \Log(\frac{z - 1}{z})
		= -(\Log(z  - 1) - z)
		= -\Log(1 + (-z)) + z
		= z + \sum_{n = 1}^{\infty} \frac{{(-1)}^{n - 1}}{n} {(-z)}^n
		= z - \sum_{n = 1}^{\infty} \frac{1}{n} z^n
	\]
\end{solution}

\question{}
עבור כל אחת מן הפונקציות הבאות נמצא את כל נקודות הסינגולריות, נסווגן ונחשב את השארית שלהן.

\subquestion{}
$f(z) = \frac{z}{\sin z}$
\begin{solution}
	$z$ חסרת נקודות סינגולריות, ולכן יש סינגולריות אם ורק אם $\sin z = 0 \iff z = \pi k$. \\
	בנקודה $z = 0$ ומרציפות נקבל
	\[
		\lim_{z \to 0} \frac{z}{\sin z}
		= \lim_{x \to 0} \frac{x}{\sin x}
		= 1
	\]
	ולכן בסביבה זו הפונקציה חסומה וזוהי סינגולריות סליקה. \\
	בכל נקודה אחרת מתקיים
	\[
		\lim_{z \to \pi k} |\frac{z}{\sin z}|
		= \lim_{x \to \pi k} |\frac{x}{\sin x}|
		= \infty
	\]
	ולכן ממשפט מההרצאה אלו הן נקודות סינגולריות קוטב, ומהליך דומה לחישוב הנקודה $z = 0$ נסיק שזהו קוטב מסדר $1$, ונחשב את השארית,
	\[
		\operatorname{res}_f(\pi k)
		= \frac{1}{(1 - 1)!} (z - \pi k) \frac{z}{\sin z}
		\to \pi k \frac{z - \pi k}{\sin z}
		= {(-1)}^n \pi k
	\]
\end{solution}

\subquestion{}
$f(z) = \frac{z^{2n}}{{(z + 1)}^n}$ עבור כל $n \in \NN$.
\begin{solution}
	ישנה נקודה יחידה בה הפונקציה לא מוגדרת, $z = -1$, בה נבדוק חשד לסינגולריות קוטב,
	\[
		\lim_{z \to -1} {(z + 1)}^n f(z)
		\lim_{z \to -1} z^{2n}
		= 1
	\]
	ולכן נקודה זו היא אכן קוטב, וסדר הקוטב $n$.
	נחשב את השארית בנקודה זו,
	\[
		\operatorname{res}_f(-1)
		= \frac{1}{(n - 1)!} \frac{d^{n - 1}}{dz^{n - 1}} {(z + 1)}^n f(z) \mid_{z = -1}
		= \frac{1}{(n - 1)!} \frac{(2n)!}{(n + 1)!} z^{n + 1} \mid_{z = -1}
		= \frac{1}{(n - 1)!} \frac{(2n)!}{(n + 1)!} {(-1)}^{n + 1}
	\]
\end{solution}

\subquestion{}
$f(z) = z^2 \cos(\frac{1}{z - 2})$.
\begin{solution}
	פונקציה זו מוגדרת בכל התחום פרט למקרה $z = 2$, נבדוק נקודה זו
	\[
		\lim_{z \to 2} z^2 \cos \frac{1}{z - 2}
		= 4 \lim_{z \to 0} \cos \frac{1}{z}
	\]
	אבל נבחין כי
	\[
		\lim_{n \to \infty} \cos 2 \pi n = 1
	\]
	בעוד
	\[
		\lim_{n \to \infty} \cos i n
		= \lim_{n \to \infty} \frac{e^{-n} + e^n}{2}
		= \infty
	\]
	ולכן זוהי סינגולריות עיקרית.
	נעבור לחישוב השארית על־ידי פיתוח טור לורן סביב $z = 2$ עבור $w = z - 2$,
	\[
		f(w)
		= {(w + 2)}^2 \cos \frac{1}{w}
		= {(w + 2)}^2 \sum_{n = 0}^{\infty} \frac{{(-1)}^n}{(n2)!} {(\frac{1}{w})}^{2n}
		= (w^2 + 4w + 4) \sum_{n = 0}^{\infty} \frac{{(-1)}^n}{(n2)!} w^{-2n}
	\]
	ולכן $c_{-1} = 4 \cdot \frac{-1}{2!} = -2$.
\end{solution}

\question{}
תהי $f$ פונקציה שלמה לא קבועה.

\subquestion{}
נוכיח כי $f(\CC)$ צפופה ב־$\CC$.
\begin{proof}
	נבחין כי באינסוף הפונקציה $f$ לא חסומה אחרת ממשפט ליוביל היא קבועה. \\
	אם ל־$f$ הייתה סינגולריות עיקרית ב־$\infty$ אז ממשפט קזרוטי ויירשטראס היינו מקבלים ש־$\overline{f(\CC)} = \CC$, כלומר ש־$f$ צפופה, וסיימנו. \\
	נניח אם כן של־$f$ אין סינגולריות עיקרית ב־$\infty$, לכן נובע שיש לה קוטב באינסוף ובהתאם ממשפט מההרצאה $f$ היא פולינום ותמונתה המישור המרוכב.
\end{proof}

\subquestion{}
נוכיח כי לכל פונקציה אנליטית $g : U_a^* \to \CC$ בעלת סינגולריות עיקרית ב־$z = a$, גם ל־$f \circ g$ יש סינגולריות עיקרית ב־$z = a$.
\begin{proof}
	נבחין של־$f \circ g$ אכן יש סינגולריות ב־$z = a$, וכן שהרכבה זו היא הרכבת פונקציות אנליטיות ולכן אנליטית. \\
	יהי $B_r = B(a, r) \subseteq U_a^*$.
	ממשפט קזרוטי ויירשטראס $\overline{g(B_r)} = \CC$.
	אבל מהצפיפות שמצאנו בסעיף הקודם $\overline{f(\CC)} = \CC$, כלומר מרציפות נובע
	\[
		\overline{(f \circ g)(B_r)}
		= \overline{f(\overline{g(B_r)})}
		= \overline{f(\CC)} = \CC
	\]
	אילו נניח ש־$z = a$ נקודה סליקה או קוטב של $f \circ g$ נקבל סתירה למשפט קזרוטי ויירשטראס, ולכן זוהי סינגולריות עיקרית.
\end{proof}

\question{}
תהינה $f, g$ פונקציות שלמות המקיימות $|f(z)| \le |g(z)|$ לכל $z \in \CC$. \\
נראה ש־$f = \lambda g$ עבור קבוע $|\lambda| \le 1$.
\begin{proof}
	נגדיר
	\[
		h(z) = \frac{f(z)}{g(z)}
	\]
	ולכן $|h(z)| \le 1$ לכל $z$.
	אילו $h(z) \ne 0$ לכל $z \in \CC$ אז ממשפט ליוביל $h$ קבועה ובהתאם $f(z) = h(0) g(z)$, לכן נניח ש־$f$ מתאפסת במספר כלשהו של נקודות.
	בכל נקודה כזאת הפונקציה חסומה, ולכן נוכל להסיק שנקודה זו היא סינגולריות סליקה של $h$, ונגדיר פונקציה חדשה $\bar{h} \equiv h$ המשכה אנליטית של $h$.
	נבחין שממשפט ליוביל שוב $\bar{h}$ היא פונקציה קבועה, ונגדיר $\lambda = \im \bar{h}$.
	לכל $z \in \CC, g(z) \ne 0$ מתקבל $f(z) = \lambda g(z)$.
	בנקודות בהן $g(z) = 0$ נובע גם $|f(z)| \le 0$, כלומר $f(z) = 0 = \lambda g(z)$, ולכן שוויון זה נכון בכל נקודה.
\end{proof}

\question{}
תהי $f : U_0^* \to \CC$ אנליטית לא קבועה בסביבה מנוקבת של $z = 0$. \\
נראה שאם $|f(\frac{1}{n})| = O(\frac{1}{n!})$ אז ל־$f$ יש סינגולריות עיקרית ב־$z = 0$. \\
נמצא דוגמה לפונקציה כזו.
\begin{proof}
	מהנתון $0$ סינגולריות של $f$, וכן
	\[
		\exists M \in \RR_+, \lim_{n \to \infty} |f(\frac{1}{n})|
		\le \lim_{n \to \infty} M \frac{1}{n!}
		= 0
	\]
	ולכן $f$ חסומה בסביבה של $z = 0$, כלור זוהי לא סינגולריות קוטב, לכן היא עיקרית או סליקה, ונניח בשלילה שהיא סליקה. \\
	מהגבול שמצאנו נובע בהכרח שאם $g$ המשכה אנליטית של $f$, אז $g(0) = 0$. \\
	נגדיר $g(z) = \sum_{l = 0}^{\infty} a_l z^l$ פיתוח טיילור שקיים בסביבה של $z = 0$, מצאנו ש־$a_0 = 0$.
	יהי $k \in \NN$ האיבר הראשון כך ש־$c_k \ne 0$ ולכן
	\[
		g(z)
		= \sum_{n = k}^{\infty} a_n z^n
		= z^k \sum_{n = 0}^{\infty} a_{n + k} z^n
	\]
	ולכן
	\[
		\frac{g(z)}{z^k} \sum_{n = 0}^{\infty} a_{n + k} z^n \xrightarrow[z \to 0]{} c_k
	\]
	אבל
	\[
		\lim_{n \to \infty} \left\lvert \frac{g(\frac{1}{n})}{{(\frac{1}{n})}^k} \right\rvert
		\le \lim_{n \to \infty} \left\lvert \frac{g(\frac{1}{n})}{\frac{1}{n!}} \right\rvert
		= 0
	\]
	ולכן $c_k = 0$ ובהתאם לא קיים $k$ כזה, כלומר $g \equiv 0$.
	זוהי כמובן סתירה לנתון ש־$f$ לא קבועה, ולכן הסינגולריות לא סליקה, דהינו היא עיקרית.
\end{proof}
לבסוף נראה דוגמה לפונקציה כזו, $f(z) = e^{-\frac{1}{z}}$, פונקציה זו היא פונקציה שאנו כבר יודעים שב־$z = 0$ יש לה סינגולריות עיקרית, ולכל $n \in \NN$,
\[
	|f(z)|
	= e^{-n}
	\le \frac{1}{n!}
\]
והתנאי אכן מתקיים.

\question{}
תהי $f : G \to \CC$ רציפה. \\
נאמר ש־$g : G \to \CC$ היא שורש רציף של $f$ אם היא רציפה ובנוסף $\forall z \in G, f(z) = {g(z)}^2$.

\subquestion{}
נוכיח שאם $f$ אנליטית ו־$f(z) \ne 0$ לכל $z \in G$ אז כל שורש רציף של $f$ הוא אנליטי.
\begin{proof}
	נגדיר $H = f(G)$, ידוע ש־$f$ רציפה ו־$G$ תחום ולכן גם $H$ תחום, וכן $0 \notin H$, ולכן קיים ענף של הלוגריתם על $H$, $\log$.
	בהתאם
	\[
		g(z) = e^{\frac{\log(f(z))}{2}}
	\]
	אבל אקספוננט אנליטית בכל תחום, וכן ידוע ש־$f$ אנליטית, ולכן מספיק לבדוק את $\log$ שהגדרנו.
	אבל ראינו שכל פונקציית לוגריתם רציפה ואנליטית בתחום בו היא מוגדרת (טענה שראינו בהרצאה) ולכן כמובן שגם פונקציה זו אנליטית, ולכן גם $g$.
\end{proof}

\subquestion{}
נוכיח כי הטענה שהוכחנו זה עתה חלה גם אם לא בהכרח $f(z) \ne 0$.
\begin{proof}
	אילו יש כמות בת־מניה של נקודות בהן $f(z) = 0$ אז נוכל לצמצם את התחום שלנו בנקודות אלה ולהשתמש בהמשכה אנליטית כדי להשתמש בהוכחה שראינו זה עתה. \\
	אם כמות זו לא בת־מניה, אז מקומפקטיות $G$ (התחום לא הוגדר בשום שלב, אבל נניח שהתחום הוא טוב) ולכן יש תת־סדרה מתכנסת כך ש־$g(z_n) = 0$ לכל $n \in \NN$ וממשפט היחידות השני היא קבועה וכמובן גם אנליטית.
\end{proof}

\end{document} % chktex 17
