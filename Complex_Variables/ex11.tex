\documentclass[a4paper]{article}

% packages
\usepackage{inputenc, fontspec, amsmath, amsthm, amsfonts, polyglossia, catchfile}
\usepackage[a4paper, margin=50pt, includeheadfoot]{geometry} % set page margins

% style
\AddToHook{cmd/section/before}{\clearpage}	% Add line break before section
\linespread{1.5}
\setcounter{secnumdepth}{0}		% Remove default number tags from sections
\setmainfont{Libertinus Serif}
\setsansfont{Libertinus Sans}
\setmonofont{Libertinus Mono}
\setdefaultlanguage{hebrew}
\setotherlanguage{english}

% operators
\DeclareMathOperator\cis{cis}
\DeclareMathOperator\Sp{Sp}
\DeclareMathOperator\tr{tr}
\DeclareMathOperator\im{Im}
\DeclareMathOperator\diag{diag}
\DeclareMathOperator*\lowlim{\underline{lim}}
\DeclareMathOperator*\uplim{\overline{lim}}

% commands
\renewcommand\qedsymbol{\textbf{משל}}
\newcommand{\NN}[0]{\mathbb{N}}
\newcommand{\ZZ}[0]{\mathbb{Z}}
\newcommand{\QQ}[0]{\mathbb{Q}}
\newcommand{\RR}[0]{\mathbb{R}}
\newcommand{\CC}[0]{\mathbb{C}}
\newcommand{\getenv}[2][] {
  \CatchFileEdef{\temp}{"|kpsewhich --var-value #2"}{\endlinechar=-1}
  \if\relax\detokenize{#1}\relax\temp\else\let#1\temp\fi
}
\newcommand{\explain}[2] {
	\begin{flalign*}
		 && \text{#2} && \text{#1}
	\end{flalign*}
}

% headers
\getenv[\AUTHOR]{AUTHOR}
\author{\AUTHOR}
\date\today

\title{פתרון מטלה 11 --- פונקציות מרוכבות, 80519}

\begin{document}
\maketitle
\maketitleprint{}

\question{}
נחשב את האינטגרלים הבאים באמצעות משפט השארית.

\subquestion{}
$I = \int_0^\infty \frac{\log x}{x^n + 1}\ dx$ לכל $n \ge 2$ טבעי.
\begin{solution}
	נגדיר
	\[
		f(z) = \frac{\log z}{z^n + 1}
	\]
	כאשר $\log$ הוא הענף המוגדר בתחום שנציין בהמשך (בתחום זה אכן יש ענף מוגדר).
	נבחין שמתקיים
	\[
		z^n + 1 = 0
		\iff
		z = \exp(\frac{\pi i}{n} + \frac{2\pi k}{n}), k \in \{0, \dots, n - 1\}
	\]
	לכן בפרט יש סינגולריות קוטב מסדר $1$ ב־$z = \exp(\frac{\pi i}{n})$. \\
	נגדיר את המסילות הבאות:
	\begin{align*}
		& \gamma_1(t) = t & t \in [\epsilon, r] \\
		& \gamma_2(t) = r e^{it} & t \in [0, \frac{2\pi}{n}] \\
		& \gamma_3(t) = t e^{\frac{2\pi i}{n}} & t \in [r, \epsilon] \\
		& \gamma_4(t) = \epsilon e^{\frac{2\pi i}{n} - it} & t \in [0, \frac{2\pi}{n}] \\
	\end{align*}
	התחום המוגדר על־ידי גבולות אלה לכל $\epsilon, r > 0$ הוא תחום טוב (בעל שפה סופית וחסום) ומחורר בנקודה היחידה $z = e^{\frac{\pi i}{n}}$.
	מההגדרה
	\[
		\int_{\gamma_1} f(z)\ dz \xrightarrow[\epsilon \to 0]{r \to \infty} I
	\]
	מאי־שוויון ML נובע,
	\[
		\int_{\gamma_2} f(z)\ dz
		= \int_0^{\frac{2\pi}{n}} \frac{\log(re^{it})}{r^n e^{int} + 1}\ dt
		\le \frac{2\pi r}{n} \log r \cdot \frac{1}{r^n + 1}
		\xrightarrow[r \to \infty]{} 0
	\]
	מבדיקה ישירה ותוצאת שאלה 1 סעיף ב' של מטלה 7 נובע,
	\[
		\int_{\gamma_3} f(z)\ dz
		= \int_r^\epsilon f(t e^{\frac{2\pi i}{n}}) e^{\frac{2\pi i}{n}}\ dt
		= e^{\frac{2\pi i}{n}} \int_r^\epsilon \frac{\log(t) + \frac{2\pi i}{n}}{t^n + 1}\ dt
		= e^{\frac{2\pi i}{n}} (I - \frac{\pi}{2})
	\]
	ולבסוף נבחין שגם
	\[
		\int_{\gamma_4} f(z)\ dz
		\le \frac{2\pi \epsilon}{n} \cdot \frac{e^\epsilon}{\epsilon^n + 1}
		\xrightarrow{\epsilon \to 0} 0
	\]
	ולכן נובע ממשפט השארית וטענה לחישוב שאריות קוטב פשוט,
	\[
		\int_{\partial G} f(z)\ dz
		= I + e^{\frac{\pi i}{n}} (I - \frac{\pi}{2})
		= I (1 + e^{\frac{\pi i}{n}}) - \frac{\pi}{2} e^{\frac{\pi i}{n}}
		= 2\pi i \operatorname{res}_f(e^{\frac{\pi i}{n}})
		= 2\pi i \left. \frac{\log z}{n z^{n - 1}} \right\rvert_{z = e^{\frac{\pi i}{n}}}
		= 2 \frac{\pi^2}{n^2} e^{\frac{\pi i}{n}}
	\]
	ובפרט נובע
	\[
		I (1 + e^{\frac{\pi i}{n}})
		= 2 \frac{\pi^2}{n^2} e^{\frac{\pi i}{n}} + \frac{\pi}{2} e^{\frac{\pi i}{n}}
	\]
	ומחילוץ הערך הממשי נקבל
	\[
		I (1 + \cos(\frac{\pi}{n}))
		= \cos(\frac{\pi}{n}) (2 \frac{\pi^2}{n^2} + \frac{\pi}{2})
	\]
	ולכן
	\[
		\int_0^\infty \frac{\log x}{x^n + 1}\ dx
		= \frac{\cos(\frac{\pi}{n}) (2 \frac{\pi^2}{n^2} + \frac{\pi}{2})}{1 + \cos(\frac{\pi}{n})}
	\]
\end{solution}

\subquestion{}
עבור כל $a, b \in \RR$,
\[
	I = \int_{0}^{\infty} \frac{\cos x}{(x^2 + a^2)(x^2 + b^2)}
\]
\begin{solution}
	נגדיר
	\[
		f(z)
		= \frac{e^z}{(z^2 + a^2)(z^2 + b^2)}
	\]
	ולכן $I = \re(\int_{0}^{\infty} f(z)\ dz)$.
	נבחין כי אם $a = 0$ או $b = 0$ אז מאינפי 2 האינטגרל לא מתכנס ולכן נניח $a, b \ne 0$. \\
	עוד נבחין ש־$f$ היא פונקציה זוגית ולכן,
	\[
		2I
		= \int_{-\infty}^{\infty} f(x)\ dx
	\]
	ונחשב אינטגרל זה במקום. 
	עוד נבחין שמתקיים
	\[
		\lim_{x \to \pm\infty} x f(x)
		= 0
	\]
	בשל חסימות המונה וגודל המכנה, לכן $f(x) = o(|\frac{1}{x}|)$ ומטענה מהכיתה נובע שאם $Z$ קבוצת הסינגולריות של $f$, אז
	\[
		2I
		= 2\pi i \sum_{z \in Z, \im z > 0} \operatorname{res}_f(z)
	\]
	אבל אנו גם יודעים ש־$Z = \{\pm ai, \pm bi \}$, לכן נניח בלי הגבלת הכלליות $a, b > 0$ ונקבל שהנקודות הקריטיות לחישוב האינטגרל הן $ai, bi$ בלבד, את ההנחה הזו מותר לנו לעשות בשל סימטריית הסינגולריות, במקרה שלילה נבחר $-a, -b$.
	אילו $a \ne b$, אז נובע שהסינגולריות היא קוטב מסדר 1 בשתי הנקודות, נחשב,
	\[
		\operatorname{res}_f(ai)
		= \left. \frac{e^z}{2z (2z^2 + a^2 + b^2)} \right\rvert_{z = ai}
		= \frac{e^{ai}}{2ai (b^2 - a^2)}
	\]
	ולכן באופן דומה גם
	\[
		\operatorname{res}_f(bi)
		= \frac{e^{bi}}{2bi (a^2 - b^2)}
	\]
	ובהתאם נובע
	\begin{multline*}
		I
		= \re\left(\pi i \left(\frac{e^{ai}}{2ai (b^2 - a^2)} + \frac{e^{ai}}{2bi (a^2 - b^2)}\right)\right) \\
		= \re\left(\pi \left(\frac{e^{ai}}{2a (b^2 - a^2)} + \frac{e^{ai}}{2b (a^2 - b^2)}\right)\right)
		= \pi \left(\frac{\cos(a)}{2a (b^2 - a^2)} + \frac{\cos(b)}{2b (a^2 - b^2)}\right)
	\end{multline*}
	כעת נניח ש־$a = b$, ולכן קיימת סינגולריות יחידה מסדר 2 ב־$ai$, ונחשב את השארית,
	\[
		\operatorname{res}_f(ai)
		= \lim_{z \to ai} \frac{d}{dz} \frac{e^z {(z - ai)}^2}{{(z^2 + a^2)}^2}
		= \lim_{z \to ai} \frac{d}{dz} \frac{e^z}{{(z + ai)}^2}
		= \lim_{z \to ai} \frac{e^z {(z + ai)}^2 - e^z 2 (z + ai)}{{(z + ai)}^4}
		= \frac{e^{ai} (-a^2 - ai)}{4 a^4}
	\]
	ולכן
	\[
		I
		= \re\left( \pi i \frac{e^{ai} (-a^2 - ai)}{4 a^4} \right)
		= \re\left( \pi \frac{e^{ai} (-a^2i + a)}{4 a^4} \right)
		= \pi \frac{\cos(a)}{4 a^3} 
	\]
\end{solution}

\subquestion{}
\[
	I = \int_{0}^{\infty} \frac{\sin^3 x}{x^3}\ dx
\]
\begin{solution}
	נבחין שהאינטגרל הנתון מתכנס ישירות מהזהות $\lim_{x \to 0} \frac{\sin x}{x} = 1$ וממבחני התכנסות.
	נבחין שמתקיים,
	\begin{align*}
		\sin^3(x)
		& = \im\left(i \frac{{(e^{ix} - e^{-ix})}^3}{{(2i)}^3}\right) \\
		& = \im\left(i \frac{e^{3ix} - 3e^{2ix - ix} + 3e^{ix - 2 ix} - e^{-3ix}}{-8i}\right) \\
		& = \im\left(\frac{e^{3ix} - 3e^{ix} + 3e^{-ix} - e^{-3ix} + 2\cos(3x) - 6\cos(x)}{-8}\right) \\
		& = \im\left(\frac{e^{3ix} - 3e^{ix} + 3e^{-ix} - e^{-3ix} + e^{3ix} + e^{-3ix} - 3e^{ix} - 3e^{-ix}}{-8}\right) \\
		& = \im\left(\frac{e^{3ix} - 3e^{ix}}{-4}\right) \\
		& = \im\left(\frac{(1 - e^{3ix}) - 3(1 - e^{ix})}{4}\right) \\
	\end{align*}
	ולכן נגדיר
	\[
		f(z)
		= \frac{(1 - e^{3iz}) - 3(1 - e^{iz})}{4z^3}
	\]
	ונובע
	\[
		I = \im\left(\int_0^\infty f(z)\ dz\right)
	\]
	נבחין ש־$f$ בעלת סינגולריות סליקה ב־$z = 0$ ולכן נתייחס אליה כאל ההשלמה האנליטית שלה בנקודה זו.
	נרצה אם כך לחשב את ערך האינטגרל, לכן נגדיר תחום טוב אשר שפתו מוגדרת על־ידי
	\begin{align*}
		& \gamma_1(t) = t && t \in [0, r] \\
		& \gamma_2(t) = r e^{it} && t \in [0, \frac{\pi}{2}] \\
		& \gamma_3(t) = it && t \in [r, 0] \\
	\end{align*}
	לכל $r > 0$.
	מאנליטיות $f$ נובע
	\[
		\im\left( \int_{\gamma_1} f(z)\ dz + \int_{\gamma_2} f(z)\ dz + \int_{\gamma_3} f(z)\ dz \right) = 0
	\]
	ונעבור לחישוב האינטגרלים.
	\[
		\im \int_{\gamma_1} f(z)\ dz
		\xrightarrow{r \to \infty} I
	\]
	נבחין כי אין חלק מדומה לאינטגרל הבא,
	\[
		\im \int_{\gamma_3} f(z)\ dz
		= \int_{r}^{0} \frac{1 - e^{-3t} - 3(1 - e^{-t})}{-it^3} \cdot i\ dt
		= 0
	\]
	אנו יודעים שמתקיים $e^z = 1 + z + \frac{z^2}{2} + o(z^3)$ ולכן נובע
	\[
		f(z)
		= \frac{-2 - 1 - 3iz - \frac{{(3iz)}^2}{2} + 1 + iz + \frac{{(iz)}^2}{2} + o(z^3)}{4z^3}
		= \frac{3z^2 + o(z^3)}{4z^3}
		= \frac{3}{4} \frac{1}{z} + o(\frac{1}{r})
	\]
	ולכן
	\[
		\int_{\gamma_2} f(z)\ dz
		= \int_0^{\pi/2} (\frac{3}{4} \frac{1}{r e^{it}} + \frac{o(r^3)}{r^3}) \cdot ire^{it}
		= \frac{3i}{4} \int_0^{\pi/2} 1 + \frac{o(\frac{1}{r})}{\frac{1}{r}}\ dt
		= \frac{3i}{4} \cdot (0 - \pi/2) + o(\frac{1}{r})/r
		\xrightarrow{r \to \infty} -\frac{3\pi i}{8}
	\]
	ולכן מאיחוד השוויונות שמצאנו נובע
	\[
		\im\left( \int_{\gamma_1} f(z)\ dz + \int_{\gamma_2} f(z)\ dz + \int_{\gamma_3} f(z)\ dz \right)
		= \im \left( 0 + iI - \frac{3\pi i}{8} \right)
		\implies I = \frac{3\pi}{8}
	\]
\end{solution}

\question{}
\subquestion{}
תהי $\frac{p(z)}{q(z)}$ פונקציה רציונלית, כך ש־$p, q \in \CC[z]$ פולינומים כך ש־$\deg q \ge \deg p + 2$. \\
נסמן $Z \subseteq \CC$ את קבוצת השורשים של $q$ ונניח ש־$Z \cap \RR = \emptyset$.
נוכיח שמתקיים
\[
	\int_{-\infty}^{\infty} \frac{p(x)}{q(x)}\ dx
	= 2\pi i \sum_{a \in Z \cap H} \operatorname{res}_{p/q}(a)
\]
כאשר $H = \{ z \in \CC \mid \im z > 0 \}$.
\begin{proof}
	נגדיר $G = \{ z \in \overline{B}(0, r) \mid \re z \ge 0 \}$ לכל $r > 0$, זהו כמובן תחום טוב, וכן $\partial G = \gamma_1 + \gamma_2$, כאשר
	\begin{align*}
		& \gamma_1(t) = t, && t \in [-r, r] \\
		& \gamma_2(t) = r e^{it}, && t \in [0, \pi]
	\end{align*}
	נגדיר $d = \deg q - \deg p$, לכן $d \ge 2$, עבור $r$ מספיק גדול נובע $\frac{p}{q} = o(\frac{1}{r^d})$ ולכן מ־ML,
	\[
		\int_{\gamma_2} \frac{p(z)}{q(z)}\ dz
		\le \pi r \cdot \max_{z \in \gamma_2} \frac{p(z)}{q(z)}
		= \pi r^{1 - d} \cdot \frac{o(\frac{1}{r^d})}{\frac{1}{r^d}}
		\xrightarrow{r \to \infty} 0
	\]
	ונסיק ממשפט השארית (תוך ההנחה שבחרנו $r$ כך ש־$Z \cap \partial G = \emptyset$),
	\[
		\int_{\partial G} \frac{p(z)}{q(z)}\ dz
		= \int_{-r}^r \frac{p(z)}{q(z)}\ dz
		= 2\pi i \sum_{a \in Z \cap G} \operatorname{res}_{p/q}(a)
	\]
	אבל $G \xrightarrow{r \to \infty} H$ ולכן
	\[
		\int_{-\infty}^{\infty} \frac{p(x)}{q(x)}\ dx
		= 2\pi i \sum_{a \in Z \cap H} \operatorname{res}_{p/q}(a)
	\]
\end{proof}

\subquestion{}
נסיק דרך נוספת לחישוב האינטגרל
\[
	\int_0^\infty \frac{x^2}{x^4 + 1}
\]
שהוצג בתרגיל 7.
\begin{solution}
	נגדיר $p(z) = z^2, q(z) = z^4 + 1$ ולכן $\deg q \ge \deg p + 2$ וכן $p, q \in \CC[z]$.
	נבחין ש־$z^4 + 1 = 0 \iff z = e^{\frac{\pi i + 2\pi i k}{4}}$ ל־$k \in \{0, \dots, 3\}$, לכן
	\[
		Z = \{ e^{\frac{\pi i + 2\pi i k}{4}} \mid k \in \{0, \dots, 3\} \}
	\]
	בפרט $Z \cap \RR = \emptyset$, ולכן תנאי הסעיף הקודם חלים ונסיק
	\begin{align*}
		\int_{-\infty}^{\infty} \frac{p(x)}{q(x)}\ dx
		& = 2\pi i \sum_{a \in Z \cap H} \operatorname{res}_{p/q}(a) \\
		& = 2\pi i (\operatorname{res}_{p/q}(e^{\frac{\pi i}{4}}) + \operatorname{res}_{p/q}(e^{\frac{3 \pi i}{4}})) \\
		& = 2\pi i \left({\left( \frac{z^2}{4z^3} \right)}_{z = e^{\frac{\pi i}{4}}} + {\left( \frac{z^2}{4z^3} \right)}_{z = e^{\frac{3 \pi i}{4}}}\right) \\
		& = 2\pi i \left(\frac{1}{4e^{\frac{\pi i}{4}}} + \frac{1}{4e^{\frac{3 \pi i}{4}}} \right)
	\end{align*}
	נבחין ש־$\frac{p(x)}{q(x)}$ היא פונקציה סימטרית מעל הממשיים, ולכן
	\[
		\int_0^{\infty} \frac{p(x)}{q(x)}\ dx
		= \pi i \left(\frac{1}{4e^{\frac{\pi i}{4}}} + \frac{1}{4e^{\frac{3 \pi i}{4}}} \right)
	\]
\end{solution}

\subquestion{}
נוכיח שמתקיים
\[
	\int_{-\infty}^{\infty} \frac{1}{{(x^2 + 1)}^{n + 1}}\ dx
	= \frac{(2n - 1)!!}{(2n)!!} \cdot \pi
\]
לכל $n \in \NN$.
\begin{proof}
	נגדיר $p(z) = 1, q(z) = {(z^2 + 1)}^{n + 1}$ ולכן תנאי סעיף א' חלים ועלינו לחשב את השארית בכל נקודות הסינגולריות.
	ישנה סינגולריות קוטב מסדר $n + 1$ בנקודות $z = \pm i$, אבל רק $z = i \in H$ ולכן מספיק שנחשב את ערך השארית בה.
	מטענה אודות ערך שארית בקוטב נקבל שעלינו לחשב את,
	\begin{align*}
		\operatorname{res}_{p/q}(i)
		& = \frac{1}{(n - 1 + 1)!} \lim_{z \to i} \frac{d^{n - 1 + 1}}{dz^{n - 1 + 1}} \left({(z - i)}^{n + 1} \frac{1}{{(z^2 + 1)}^{n + 1}}\right) \\
		& = \frac{1}{n!} \lim_{z \to i} \frac{d^n}{dz^n} {(z + i)}^{-n - 1} \\
		& = \frac{(-n - 1) \cdots (-2n)}{n!} \lim_{z \to i} {(z + i)}^{-2n - 1} \\
		& = \frac{{(-1)}^n (n + 1) \cdots 2n}{n! 2^{2n + 1} \cdot {(-1)}^n \cdot i} \\
		& = -\frac{i}{2} \frac{\frac{(2n)!}{n! 2^n}}{n! \cdot 2^n}
	\end{align*}
	נבחין שמהגדרת עצרת כפולה נובע $(2n)!! = n! 2^n$, ולכן
	\[
		-\frac{i}{2} \frac{\frac{(2n)!}{n! 2^n}}{n! \cdot 2^n}
		= -\frac{i}{2} \frac{\frac{(2n)!}{(2n)!!}}{(2n)!!}
	\]
	אנו גם יודעים ש־$(2n)! = (2n)!! \cdot (2n - 1)!! $ מבדיקה ישירה של ההגדרה, לכן
	\[
		\operatorname{res}_{p/q}(i)
		= -\frac{i}{2} \frac{(2n - 1)!!}{(2n)!!}
	\]
	ולבסוף מסעיף א',
	\[
		\int_{-\infty}^{\infty} \frac{1}{{(x^2 + 1)}^{n + 1}}
		= 2 \pi i \cdot (-1) \frac{i}{2} \frac{(2n - 1)!!}{(2n)!!}
		= \frac{(2n - 1)!!}{(2n)!!}
	\]
\end{proof}

\question{}
תהי
\[
	f(z) = \Log\left(\frac{z}{z - 1}\right)
\]

\subquestion{}
נראה ש־$f$ מוגדרת בתחום $\CC \setminus [0, 1]$.
\begin{proof}
	נבחין ש־$\Log$ מוגדר ב־$\CC \setminus \{ 0 \}$ ולכן מספיק שנבדוק מתי $\frac{z}{z - 1} \ne 0$.
	כמובן $\frac{z}{z - 1} = 0 \iff z = 0$, וכן $z - 1 = 0 \iff z = 1$, כלור בתחום $\CC \setminus [0, 1]$ הפונקציה $f$ מוגדרת וסופית.
\end{proof}

\subquestion{}
נחשב את האינטגרל
\[
	\int_{|z| = 4} f(z)\ dz
\]
\begin{solution}
	ממשפט השארית באינסוף נובע
	\[
		\int_{|z| = 4} f(z)\ dz
		= 2\pi i \cdot \operatorname{res}_f(\infty)
	\]
	נגדיר
	\[
		F(z)
		= -\frac{f(\frac{1}{z})}{z^2}
		= -\frac{\Log(\frac{\frac{1}{z}}{\frac{1}{z} - 1})}{z^2}
		= -\frac{1}{z^2} \Log\left(\frac{1}{1 - z}\right)
	\]
	אנו יודעים ש־$\frac{1}{1 - z} = 1 + z + o(z^2)$ וכן ש־$\Log(1 + z + o(z^2)) = z + o(z^2)$, ולכן $c_{-1} = -1$ עבור הסדרה המגדירה את טור לורן של $F$.
	בהתאם $\operatorname{res}_F(0) = -1$ ולכן ממשפט השארית באינסוף גם
	\[
		\int_{|z| = 4} f(z)\ dz
		= 2\pi i \operatorname{res}_f(\infty)
		= 2\pi i \operatorname{res}_F(0)
		= -2\pi i
	\]
\end{solution}

\question{}
יהי $u \in \CC$ ונגדיר
\[
	f(z)
	= \frac{\pi \cot(\pi z)}{{(z + u)}^2}
\]
עבור $\cot(z) = \frac{\cos(z)}{\sin(z)}$.

\subquestion{}
נסמן ב־$C_N$ לכל $N \in \NN$ את המסילה המקיפה את הריבוע ${[-N - \frac{1}{2}, N + \frac{1}{2}]}^2$ נגד כיוון השעון. \\
נראה שמתקיים
\[
	\lim_{N \to \infty} \int_{C_N} f(z)\ dz
	= 0
\]
\begin{proof}
	כתוצאה ממטלה 3 נבחין שלכל $z \in \CC, |\im z| \ge 1$ מתקיים $|\cot z| \le 1$, כלומר הפונקציה חסומה.
	בנוסף $\cot(\pi n + \frac{1}{2}) = 0$ ולכן הפונקציה חסומה גם בסביבות אלה בריבוע.
	לכל $N$ מספיק גדול ביחס ל־$u$ (כלומר ש־$u$ פנימית לריבוע),
	\[
		\int_{C_N} f(z)\ dz
		\le 4(N + \frac{1}{2}) \cdot \max_{z \in C_N} f(z)
		\le 4(N + \frac{1}{2}) \cdot \frac{\pi \cdot 1}{N^2}
		\xrightarrow{N \to \infty} 0
	\]
	ולכן נוכל להסיק שערך האינטגרל שואף לאפס כפי שרצינו.
\end{proof}

\subquestion{}
נשתמש במשפט השארית כדי להראות שלכל $u \in \CC \setminus \ZZ$ מתקיים,
\[
	\sum_{n = -\infty}^{\infty} \frac{1}{{(n + u)}^2}
	= \frac{\pi^2}{\sin^2(\pi u)}
\]
\begin{proof}
	נבחין של־$f$ סינגולריות בקבוצה $\{ -u \} \cup \{ n \in \ZZ \mid |n| \le N \}$ לכל $C_N$, ולכן נובע ממשפט השארית בתחום זה
	\[
		\int_{C_N} f(z)\ dz
		= 2\pi i \operatorname{res}_f(u) + 2\pi i \sum_{n = -N}^{N} \operatorname{res}_f(n)
		= 0
		\iff
		-\operatorname{res}_f(u) = \sum_{n = -N}^{N} \operatorname{res}_f(n)
	\]
	כל סינגולריות של $f$ היא קוטב מסדר 1 לפי זהויות על סינוס, ונקבל
	\[
		\operatorname{res}_f(n)
		= \left. \frac{\frac{\pi \cos(\pi z)}{{(z + u)}^2}}{\pi \cos(\pi z)} \right\rvert_{z = n}
		= \frac{\frac{\pi \cos(\pi n)}{{(n + u)}^2}}{\pi \cos(\pi n)}
		= \frac{1}{{(n + u)}^2}
	\]
	ב־$z = -u$ יש סינגולריות מסדר 2 ולכן
	\[
		\operatorname{res}_f(-u)
		= \lim_{z \to -u} \frac{d}{dz} \frac{\pi \cot(\pi z) {(z + u)}^2}{{(z + u)}^2}
		= \lim_{z \to -u} \frac{d}{dz} \pi \cot(\pi z)
		= \frac{-\pi}{\sin^2(-\pi u)}
		= \frac{-\pi}{\sin^2(\pi u)}
	\]
	ולכן מהשוויון ממשפט השארית נובע
	\[
		\sum_{n = -\infty}^{\infty} \frac{1}{{(n + u)}^2}
		= \frac{\pi^2}{\sin^2(\pi u)}
	\]
\end{proof}

\subquestion{}
נשתמש במשפט השארית כדי להראות שעבור $u = 0$ מתקיים
\[
	\sum_{n = 1}^{\infty} \frac{1}{n^2} = \frac{\pi^2}{6}
\]
\begin{proof}
	מהסעיף הקודם נובע שעבור $u$ בסביבה של $0$ מתקיים,
	\[
		\frac{1}{u^2} + \sum_{n \in \ZZ \setminus \{ 0 \}} \frac{1}{{(n + u)}^2}
		= \frac{\pi^2}{\sin^2(\pi u)}
		\iff
		\sum_{n \in \ZZ \setminus \{ 0 \}} \frac{1}{{(n + u)}^2}
		= \frac{\pi^2}{\sin^2(\pi u)} - \frac{1}{u^2}
		= \frac{\pi^2 u^2 - \sin^2(\pi u)}{u^2 \sin^2(\pi u)}
	\]
	ממבחן התכנסות אנו יודעים שהביטוי השמאלי מתכנס כאשר $u \to 0$ לביטוי
	\[
		2 \sum_{n = 1}^{\infty} \frac{1}{n^2}
	\]
	ונותר לבדוק את הביטוי הימני, נבחין שמתקיים $\sin^2(\pi u) = \pi^2 u^2 - \frac{\pi^4}{3} u^4 + o(u^5)$ ולכן,
	\[
		\frac{\pi^2 u^2 - \sin^2(\pi u)}{u^2 \sin^2(\pi u)}
		= \frac{\pi^2 u^2 - \pi^2 u^2 + \frac{\pi^4}{3} u^4 + o(u^5)}{\pi^2 u^4 + o(u^5)}
		\xrightarrow{u \to 0} \frac{\pi^2}{3}
	\]
	ונובע ישירות
	\[
		2 \sum_{n = 1}^{\infty} \frac{1}{n^2}
		= \frac{\pi^2}{3}
	\]
	כפי שרצינו.
\end{proof}

\end{document} % chktex 17
