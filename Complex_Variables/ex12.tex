\documentclass[a4paper]{article}

% packages
\usepackage{inputenc, fontspec, amsmath, amsthm, amsfonts, polyglossia, catchfile}
\usepackage[a4paper, margin=50pt, includeheadfoot]{geometry} % set page margins

% style
\AddToHook{cmd/section/before}{\clearpage}	% Add line break before section
\linespread{1.5}
\setcounter{secnumdepth}{0}		% Remove default number tags from sections
\setmainfont{Libertinus Serif}
\setsansfont{Libertinus Sans}
\setmonofont{Libertinus Mono}
\setdefaultlanguage{hebrew}
\setotherlanguage{english}

% operators
\DeclareMathOperator\cis{cis}
\DeclareMathOperator\Sp{Sp}
\DeclareMathOperator\tr{tr}
\DeclareMathOperator\im{Im}
\DeclareMathOperator\diag{diag}
\DeclareMathOperator*\lowlim{\underline{lim}}
\DeclareMathOperator*\uplim{\overline{lim}}

% commands
\renewcommand\qedsymbol{\textbf{משל}}
\newcommand{\NN}[0]{\mathbb{N}}
\newcommand{\ZZ}[0]{\mathbb{Z}}
\newcommand{\QQ}[0]{\mathbb{Q}}
\newcommand{\RR}[0]{\mathbb{R}}
\newcommand{\CC}[0]{\mathbb{C}}
\newcommand{\getenv}[2][] {
  \CatchFileEdef{\temp}{"|kpsewhich --var-value #2"}{\endlinechar=-1}
  \if\relax\detokenize{#1}\relax\temp\else\let#1\temp\fi
}
\newcommand{\explain}[2] {
	\begin{flalign*}
		 && \text{#2} && \text{#1}
	\end{flalign*}
}

% headers
\getenv[\AUTHOR]{AUTHOR}
\author{\AUTHOR}
\date\today

\title{פתרון מטלה 12 --- פונקציות מרוכבות, 80519}

\begin{document}
\maketitle
\maketitleprint{}

\question{}
תהי $f : U_a^* \to \CC$ אנליטית עם קוטב מסדר $m \ge 1$ ב־$z = a$. \\
נראה שקיימים $\epsilon, r > 0$ כך שלכל $|w| > r$ קיימים בדיוק $m$ פתרונות למשוואה $f(z) = w$ ב־$B^*(a, \epsilon)$.
\begin{proof}
	נגדיר $g(z) = f(z) {(z - a)}^m$, לכן $g \in Hol(U_a)$.
	הסדר של $z = a$ הוא $m$ ולכן מהגרסה המקומית של משפט רושה נובע שקיימים $\epsilon, r$ עבורם לכל $w \in B^*(g(a), r)$ יש בדיוק $m$ פתרונות למשוואה $g(z) = w$.
\end{proof}

\question{}
נמצא כמה פתרונות כולל ריבוי יש למשוואות הבאות בתחומים הנתונים.

\subquestion{}
$z^7 - 5z^4 + z^2 - 2 = 0$ ב־$B(0, 1)$.
\begin{solution}
	נגדיר $f(z) = z^7 - 5z^4 + z^4 - 2$ וגם $g(z) = z^7 - 5z^4$ ונבחין כי אם $|z| = 1$ אז,
	\[
		|f(z) - g(z)|
		\le |z^2| + |2|
		= 3
		\le |z^4| \cdot |z^2 - 5|
	\]
	ולכן ממשפט רושה בכדור $B(0, 1)$ אותו מספר אפסים לשתי הפונקציות, וכן
	\[
		g(z) = 0
		\iff z = 0, z^3 - 5 = 0
	\]
	אפס הוא שורש מריבוי 4, וכן ישנם שלושה שורשים שלא בתחום, לכן יש ארבעה שורשים ל־$f$ בתחום.
\end{solution}

\subquestion{}
$z^5 + 2z^3 - z^2 + z - \alpha = 0$ בתחום $\{ z \in \CC \mid \re z > 0 \}$, עבור $\alpha \in \RR$.
\begin{solution}
	נגדיר ריבועים שצלעם התחתונה מונחת על ציר ה־$x$ כך שגודל צלעם $r$ והם ממורכזים. \\
	נגדיר $f(z) = z^5 + 2z^3 - z^2 + z - \alpha$. \\
	נגדיר $g(z) = z^5 + 2z^3$ ונקבל שלכל $r$ מספיק גדול $|f(z) - g(z)| < |g(z)|$ על שפת הריבוע, ובהתאם משפט רושה חל ונובע שמספר השורשים בריבוע הוא $5$ שורשים לרבות ריבוי.
	מתוך שורשים אלה שלושה הם על הראשית והשניים הנותרים הם על הציר הממשי, ולכן יש בתחום אפס שורשים.
	נסיק אם כך שכאשר $r \to \infty$ אז מספר הפתרונות של המשוואה נשאר אפס.
\end{solution}

\subquestion{}
$e^z = 3z^n$ בחצי המישור $\{ z \in \CC \mid \re z < 1 \}$ עבור $n \in \NN$.
\begin{solution}
	נגדיר $f(z) = e^z - 3z^n$ ונחקור את שורשיה. \\
	נבחין שבמעגל היחידה מתקיים
	\[
		|f(z) + 3z^n|
		= |e^z|
		\le e
		\le 3
		= |-3z^n|
	\]
	ולכן לפונקציה אותה כמות שורשים כמו ל־$-3z^n$, אך לזו האחרונה שורש מריבוי $n$ באפס בלבד.
	נעבור לבחון ריבועים $[1 - 2r, 1] \times [-r, r]$.
	בריבועים אלה,
	\[
		|f(z) + 3z^n|
		= |e^z|
		\le 1
		\le |r|
		\le |-3z^n|
	\]
	ולכן גם בתחום זה הריבוי הוא זהה, ונוכל להסיק שכאשר $r \to \infty$ מספר השורשים נשאר $n$.
\end{solution}

\question{}
תהי $f : G \to \CC$ פונקציה אנליטית, ויהי $z_0 \in G$ כך ש־$f'(z_0) \ne 0$. \\
נראה שקיים $r > 0$ כך ש־$\overline{B}(z_0, r) \subseteq G$ ו־$f \mid_{B(z_0, r)}$ הפיכה עם הופכית הנתונה על־ידי
\[
	{f \mid_{B(z_0, r)}}^{-1}(w)
	= \frac{1}{2\pi i} \int_{\partial B(z_0, r)} \frac{z f'(z)}{f(z) - w}\ dz
\]
\begin{proof}
	נוכל להסיק ש־$f(z_0) \ne 0$, אחרת היינו יכולים להסיק ש־$f$ קבועה.
	משפט הפונקציה הפוכה חל ולכן קיים $r > 0$ כך ש־$f$ הפיכה ב־$B(z_0, r) \subseteq G$.
	ממשפט קושי נובע
	\[
		{f \mid_{B(z_0, r)}}^{-1}(w)
		= \frac{1}{2\pi i} \int_{\partial B(z_0, r)} \frac{{f \mid_{B(z_0, r)}}^{-1}(z)}{z - w}\ dz
	\]
	נשתמש במשפט ההצבה עם $z \rightarrow f \mid_{B(z_0, r)}(z)$ וכן $dz \rightarrow f \mid_{B(z_0, r)}(z)' dz$ ונקבל
	\[
		{f \mid_{B(z_0, r)}}^{-1}(w)
		= \frac{1}{2\pi i} \int_{\partial B(z_0, r)} \frac{{f \mid_{B(z_0, r)}}^{-1}(f \mid_{B(z_0, r)}(z))}{f \mid_{B(z_0, r)}(z) - w} \cdot f \mid_{B(z_0, r)}(z)'\ dz
		= \frac{1}{2\pi i} \int_{\partial B(z_0, r)} \frac{{f \mid_{B(z_0, r)}}^{-1}(z)}{z - w}\ dz
	\]
	כפי שרצינו.
\end{proof}

\question{}
נוכיח את משפט שוורץ־פיק על חצי המישור העליון $H$. \\
תהי $f : H \to H$ אנליטית לא קבועה, נראה שאי־השוויונות הבאים מתקיימים,
\begin{itemize}
	\item $\left\lvert \frac{f(z_2) - f(z_1)}{f(z_2) - \overline{f(z_1)}} \right\rvert \le \left\lvert \frac{z_2 - z_1}{z_2 - \overline{z_1}} \right\rvert$ לכל $z_1, z_2 \in H$.
	\item $|f'(z)| \le \frac{\im f(z)}{\im z}$ לכל $z \in H$.
\end{itemize}
נראה בנוסף שאם יש שוויון בנקודה מסוימת באחד מאי־השוויונות לעיל אז $f(z) = \frac{az + b}{cz + d}$ עבור $\begin{pmatrix} a & b \\ c & d \end{pmatrix} \in SL_2(\RR)$.
\begin{proof}
	ניזכר ב־$\varphi(z) = \frac{z - i}{z + i}$ העתקה המעבירה את $H$ ל־$D$, ונקבל ש־$\varphi \circ f$ מקיימת את משפט שוורץ־פיק.
	לכן מתקיים,
	\[
		\left\lvert \frac{\varphi(f(z_2)) - \varphi(f(z_1))}{1 - \varphi(f(z_1)) \overline{\varphi(f(z_2))}} \right\rvert
		\le \left\lvert \frac{\varphi(z_2) - \varphi(z_1)}{1 - \varphi(z_1) \overline{\varphi(z_2)}} \right\rvert
	\]
	מחישוב ישיר של $\varphi$ מתקבל,
	\[
		\left\lvert \frac{f(z_2) - f(z_1)}{f(z_2) - \overline{f(z_1)}} \right\rvert
		= \left\lvert \frac{\frac{f(z_2) - i}{f(z_2) + i} - \frac{f(z_1) - i}{f(z_1) + i}}{1 - \frac{f(z_1) - i}{f(z_1) + i} \frac{{(\overline{f(z_2)} + i)}^2}{(f(z_2) - i)(\overline{f(z_2)} + i)}} \right\rvert
		\le \left\lvert \frac{\varphi(z_2) - \varphi(z_1)}{1 - \varphi(z_2) \overline{\varphi(z_1)}} \right\rvert
		= \left\lvert \frac{\frac{f(z_2) - i}{f(z_2) + i} - \frac{f(z_1) - i}{f(z_1) + i}}{1 - \frac{f(z_1) - i}{f(z_1) + i} \frac{{(\overline{f(z_2)} + i)}^2}{(f(z_2) - i)(\overline{f(z_2)} + i)}} \right\rvert
		= \left\lvert \frac{z_2 - z_1}{z_2 - \overline{z_1}} \right\rvert
	\]
	באופן דומה נקבל מהמשפט עבור $\varphi \circ f$ שמתקיים,
	\[
		|(\varphi \circ f)'(z)| \le \frac{1 - {|\varphi(f(z))|}^2}{1 - {|z|}^2}
	\]
	נבחין כי $\varphi'(z) = \frac{2i}{{(z + i)}^2}$ ולכן,
	\[
		\left\lvert \varphi'(f(z)) f'(z) \right\rvert
		= \left\lvert \frac{-2i}{{(f(z) + i)}^2} f'(z) \right\rvert
		\le \frac{1 - {|\varphi(f(z))|}^2}{1 - {|\varphi(z)|}^2}
		= \frac{1 - \left\lvert \frac{{(f(z) - i)}^2}{{(f(z) + i)}^2} \right\rvert}{1 - \left\lvert \frac{{(z - i)}^2}{{(z + i)}^2} \right\rvert}
	\]
	ולכן,
	\begin{align*}
		\left\lvert 2 f'(z) \right\rvert
		& \le |{(z + i)}^2| \frac{{|(f(z) + i)}^2| - |{(f(z) - i)}^2| }{|{(z + i)}^2| - |{(z - i)}^2| } \\
		& = |{(z + i)}^2| \frac{\re^2(f(z) + i) + \im^2(f(z) + i) - \re^2(f(z) - i) - \im^2(f(z) - i)}{|{(z + i)}^2| - |{(z - i)}^2| } \\
		& = |{(z + i)}^2| \frac{2 \im f(z)}{2 \im z}
	\end{align*}
\end{proof}

\end{document} % chktex 17
