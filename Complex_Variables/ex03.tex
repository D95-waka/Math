\documentclass[a4paper]{article}

% packages
\usepackage{inputenc, fontspec, amsmath, amsthm, amsfonts, polyglossia, catchfile}
\usepackage[a4paper, margin=50pt, includeheadfoot]{geometry} % set page margins

% style
\AddToHook{cmd/section/before}{\clearpage}	% Add line break before section
\linespread{1.5}
\setcounter{secnumdepth}{0}		% Remove default number tags from sections
\setmainfont{Libertinus Serif}
\setsansfont{Libertinus Sans}
\setmonofont{Libertinus Mono}
\setdefaultlanguage{hebrew}
\setotherlanguage{english}

% operators
\DeclareMathOperator\cis{cis}
\DeclareMathOperator\Sp{Sp}
\DeclareMathOperator\tr{tr}
\DeclareMathOperator\im{Im}
\DeclareMathOperator\diag{diag}
\DeclareMathOperator*\lowlim{\underline{lim}}
\DeclareMathOperator*\uplim{\overline{lim}}

% commands
\renewcommand\qedsymbol{\textbf{משל}}
\newcommand{\NN}[0]{\mathbb{N}}
\newcommand{\ZZ}[0]{\mathbb{Z}}
\newcommand{\QQ}[0]{\mathbb{Q}}
\newcommand{\RR}[0]{\mathbb{R}}
\newcommand{\CC}[0]{\mathbb{C}}
\newcommand{\getenv}[2][] {
  \CatchFileEdef{\temp}{"|kpsewhich --var-value #2"}{\endlinechar=-1}
  \if\relax\detokenize{#1}\relax\temp\else\let#1\temp\fi
}
\newcommand{\explain}[2] {
	\begin{flalign*}
		 && \text{#2} && \text{#1}
	\end{flalign*}
}

% headers
\getenv[\AUTHOR]{AUTHOR}
\author{\AUTHOR}
\date\today

\title{פתרון מטלה 03 --- פונקציות מרוכבות, 80519}

\begin{document}
\maketitle
\maketitleprint{}

\Question{}
תהי $K \subseteq \CC \setminus \{ 0 \}$ קבוצה קומפקטית, ונניח כי קיים לוגריתם רציף $g : K \to \CC$.

\Subquestion{}
נגדיר $\epsilon = \inf \{ |g(z_1) - g(z_2) - 2 \pi i l| \mid l \in \ZZ \setminus \{ 0 \}, z_1, z_2 \in K \}$, \\*
ונוכיח כי $\epsilon > 0$.
\begin{proof}
	נבחין כי $\epsilon^2 = {(\log|z_1| - \log|z_2|)}^2 + {(\Arg(z_1) - \Arg(z_2) - 2 \pi l)}^2$ ולכן אם נניח בשלילה ש־$\epsilon = 0$ אז קיימת סדרה של מרוכבים כך שערכם המוחלט מתכנס והארגומנט שלהם שואף למרחק $2\pi k$. \\*
	עוד נתון כי $K$ קומפקטית ולכן סגורה וחסומה, לכן היא מכילה את כל הנקודות הגבוליות שלה ובהתאם הסדרות שמקיימות את $\epsilon = 0$ מתכנסות למספרים $z_1, z_2 \in K$. \\*
	אילו $|\arg(z_1) - \arg(z_2)| \ge 2\pi$ אז נקבל מהגדרת הארגומנט סתירה לרציפות של הענף של האגומנט שתומך ב־$g$, ולכן מתקבל שהארגומנטים מתכנסים, דהינו $z_1 = z_2$, אך אז נקבל $\epsilon \ge 2\pi$, וזו סתירה.
\end{proof}

\Subquestion{}
נוכיח כי לכל $z_0 \in K$ קיים $r > 0$ ולוגריתם אנליטי $h : B(z_0, r) \to \CC$ כך ש־$h(z_0) = g(z_0)$ ובנוסף $|h(z_1) - h(z_2)| < \frac{\epsilon}{2}$ לכל $z_1, z_2 \in B(z_0, r)$.
\begin{proof}
	תחילה נגדיר $0 < r < 2 \pi$ ולכן מהבנייה של לוגריתם אנליטי שראינו בתרגול נובע שקיים לוגריתם $h$, ואנו יודעים שהוא נקבע על־ידי נקודה יחידה (טענה מהתרגול), לכן נגדיר $h(z_0) = g(z_0)$. \\*
	עתה, קיבלנו כי $\epsilon$ של הסעיף הקודם הוא ערך ממשי, וכך גם $\frac{\epsilon}{2}$, נגדיר
	\[
		\sup_{z_1, z_2 \in \overline{B}(z_0, r)} |h(z_1) - h(z_2)| = \rho
	\]
	ולכן נוכל לקבוע את $r$ כך ש־$\rho$ יקטן בהתאם, ומטופולוגיה של כדורים סגורים במרחב אוקלידי נוכל להסיק כי קיים $r$ כך ש־$\rho < \frac{\epsilon}{2}$.
\end{proof}

\Subquestion{}
נוכיח כי קיימת קבוצה פתוחה $K \subseteq U \subseteq \CC \setminus \{ 0 \}$ ולוגריתם אנליטי $l : U \to \CC$ כך ש־$l \mid_K = g$.
\begin{proof}
	
\end{proof}

\Question{}
נוכיח בכל סעיף שניתן להגדיר לוגריתם אנליטי לפונקציה $f(z) = \cos(z) - 2$ בתחומים הנתונים.

\Subquestion{}
עבור $U_1 = \{ z \in \CC \mid 0 < \re(z) < 2\pi \}$
\begin{proof}
	
\end{proof}

\end{document}
