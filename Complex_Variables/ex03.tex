\documentclass[a4paper]{article}

% packages
\usepackage{inputenc, amsmath, amsthm, thmtools, amsfonts, amssymb, luacode, catchfile, tikzducks, hyperref}
\usepackage[a4paper, margin=50pt, includeheadfoot]{geometry} % set page margins
\usepackage[shortlabels]{enumitem}
\usepackage[skip=3pt, indent=0pt]{parskip}

% language
\usepackage[bidi=basic, layout=tabular, provide=*]{babel}
\babelprovide[main, import]{hebrew}
\babelprovide{rl}
\babelfont{rm}{Libertinus Serif}
\babelfont{sf}{Libertinus Sans}
\babelfont{tt}{Libertinus Mono}

% style
\AddToHook{cmd/section/before}{\clearpage}	% Add line break before section
\linespread{1.3}
\setcounter{secnumdepth}{0}		% Remove default number tags from sections, this won't do well with theorems
\AtBeginDocument{\setlength{\belowdisplayskip}{3pt}}
\AtBeginDocument{\setlength{\abovedisplayskip}{3pt}}

% operators
\DeclareMathOperator\cis{cis}
\DeclareMathOperator\Sp{Sp}
\DeclareMathOperator\tr{tr}
\DeclareMathOperator\im{Im}
\DeclareMathOperator\re{Re}
\DeclareMathOperator\diag{diag}
\DeclareMathOperator*\lowlim{\underline{lim}}
\DeclareMathOperator*\uplim{\overline{lim}}
\DeclareMathOperator\rng{rng}
\DeclareMathOperator\Sym{Sym}
\DeclareMathOperator\Arg{Arg}
\DeclareMathOperator\Log{Log}
\DeclareMathOperator\dom{dom}

% commands
%\renewcommand\qedsymbol{\textbf{מש''ל}}
%\renewcommand\qedsymbol{\fbox{\emoji{lizard}}}
\newcommand{\NN}[0]{\mathbb{N}}
\newcommand{\ZZ}[0]{\mathbb{Z}}
\newcommand{\QQ}[0]{\mathbb{Q}}
\newcommand{\RR}[0]{\mathbb{R}}
\newcommand{\CC}[0]{\mathbb{C}}
\newcommand{\FF}[0]{\mathbb{F}}
\newcommand{\PP}[0]{\mathbb{P}}
\newcommand{\TT}[0]{\mathbb{T}}
\newcommand{\acts}[0]{\circlearrowright}
\newcommand{\explain}[2] {
	\begin{flalign*}
		 && \text{#2} && \text{#1}
	\end{flalign*}
}
\newcommand{\maketitleprint}[0]{ \begin{center}
	\begin{tikzpicture}[scale=3]
		\duck[graduate=gray!20!black, tassel=red!70!black]
	\end{tikzpicture}	
\end{center}
}

% theorem commands
\newtheoremstyle{c_remark}
	{}	% Space above
	{}	% Space below
	{}% Body font
	{}	% Indent amount
	{\bfseries}	% Theorem head font
	{}	% Punctuation after theorem head
	{.5em}	% Space after theorem head
	{\thmname{#1}\thmnumber{ #2}\thmnote{ \normalfont{\text{(#3)}}}}	% head content
\newtheoremstyle{c_definition}
	{3pt}	% Space above
	{3pt}	% Space below
	{}% Body font
	{}	% Indent amount
	{\bfseries}	% Theorem head font
	{}	% Punctuation after theorem head
	{.5em}	% Space after theorem head
	{\thmname{#1}\thmnumber{ #2}\thmnote{ \normalfont{\text{(#3)}}}}	% head content
\newtheoremstyle{c_plain}
	{3pt}	% Space above
	{3pt}	% Space below
	{\itshape}% Body font
	{}	% Indent amount
	{\bfseries}	% Theorem head font
	{}	% Punctuation after theorem head
	{.5em}	% Space after theorem head
	{\thmname{#1}\thmnumber{ #2}\thmnote{ \text{(#3)}}}	% head content

\theoremstyle{c_plain}
\newtheorem{theorem}{משפט}[section]
\newtheorem{lemma}[theorem]{למה}
\newtheorem{proposition}[theorem]{טענה}
\newtheorem*{proposition*}{טענה}
%\newtheorem{corollary}[theorem]{אין חלופה עברית}

\theoremstyle{c_definition}
\newtheorem{definition}[theorem]{הגדרה}
\newtheorem*{definition*}{הגדרה}
\newtheorem{example}{דוגמה}[section]
\newtheorem{exercise}{תרגיל}[section]

\theoremstyle{c_remark}
\newtheorem*{remark}{הערה}
\newtheorem*{solution}{פתרון}
\newtheorem{conclusion}[theorem]{מסקנה}
\newtheorem{notation}[theorem]{סימון}

% Questions related commands
\newcounter{question}
\setcounter{question}{1}
\newcounter{sub_question}
\setcounter{sub_question}{1}

\newcommand{\question}[1][0]{
	\ifthenelse{#1 = 0}{}{\setcounter{question}{#1}}
	\subsection{שאלה \arabic{question}}
	\addtocounter{question}{1}
	\setcounter{sub_question}{1}
}

\newcommand{\subquestion}[1][0]{
	\ifthenelse{#1 = 0}{}{\setcounter{sub_question}{#1}}
	\subsubsection{סעיף \localecounter{letters.gershayim}{sub_question}}
	\addtocounter{sub_question}{1}
}

% import lua and start of document
\directlua{common = require ('../common')}

\GetEnv{AUTHOR}

% headers
\author{\AUTHOR}
\date\today

\title{פתרון מטלה 03 --- פונקציות מרוכבות, 80519}

\begin{document}
\maketitle
\maketitleprint{}

\Question{}
תהי $K \subseteq \CC \setminus \{ 0 \}$ קבוצה קומפקטית, ונניח כי קיים לוגריתם רציף $g : K \to \CC$.

\Subquestion{}
נגדיר $\epsilon = \inf \{ |g(z_1) - g(z_2) - 2 \pi i l| \mid l \in \ZZ \setminus \{ 0 \}, z_1, z_2 \in K \}$, \\*
ונוכיח כי $\epsilon > 0$.
\begin{proof}
	נבחין כי $\epsilon^2 = {(\log|z_1| - \log|z_2|)}^2 + {(\Arg(z_1) - \Arg(z_2) - 2 \pi l)}^2$ ולכן אם נניח בשלילה ש־$\epsilon = 0$ אז קיימת סדרה של מרוכבים כך שערכם המוחלט מתכנס והארגומנט שלהם שואף למרחק $2\pi k$. \\*
	עוד נתון כי $K$ קומפקטית ולכן סגורה וחסומה, לכן היא מכילה את כל הנקודות הגבוליות שלה ובהתאם הסדרות שמקיימות את $\epsilon = 0$ מתכנסות למספרים $z_1, z_2 \in K$. \\*
	אילו $|\arg(z_1) - \arg(z_2)| \ge 2\pi$ אז נקבל מהגדרת הארגומנט סתירה לרציפות של הענף של האגומנט שתומך ב־$g$, ולכן מתקבל שהארגומנטים מתכנסים, דהינו $z_1 = z_2$, אך אז נקבל $\epsilon \ge 2\pi$, וזו סתירה.
\end{proof}

\Subquestion{}
נוכיח כי לכל $z_0 \in K$ קיים $r > 0$ ולוגריתם אנליטי $h : B(z_0, r) \to \CC$ כך ש־$h(z_0) = g(z_0)$ ובנוסף $|h(z_1) - h(z_2)| < \frac{\epsilon}{2}$ לכל $z_1, z_2 \in B(z_0, r)$.
\begin{proof}
	תחילה נגדיר $0 < r < 2 \pi$ ולכן מהבנייה של לוגריתם אנליטי שראינו בתרגול נובע שקיים לוגריתם $h$, ואנו יודעים שהוא נקבע על־ידי נקודה יחידה (טענה מהתרגול), לכן נגדיר $h(z_0) = g(z_0)$. \\*
	עתה, קיבלנו כי $\epsilon$ של הסעיף הקודם הוא ערך ממשי, וכך גם $\frac{\epsilon}{3}$, נגדיר
	\[
		\sup_{z_1, z_2 \in \overline{B}(z_0, r)} |h(z_1) - h(z_2)| = \rho
	\]
	ולכן נוכל לקבוע את $r$ כך ש־$\rho$ יקטן בהתאם, ומטופולוגיה של כדורים סגורים במרחב אוקלידי נוכל להסיק כי קיים $r$ כך ש־$\rho < \frac{\epsilon}{3}$.
\end{proof}

\Subquestion{}
נוכיח כי קיימת קבוצה פתוחה $K \subseteq U \subseteq \CC \setminus \{ 0 \}$ ולוגריתם אנליטי $l : U \to \CC$ כך ש־$l \mid_K = g$.
\begin{proof}
	נבחר $\epsilon$־כיסוי פתוח של $K$ שאנו יודעים שקיים, ונקבל מהסעיף הקודם שקיימת $h$ רציפה במידה שווה על $K$ עבור הכיסוי הזה, ונגדיר $\tilde{h}$ על־ידי אוסף פונקציות זה. \\*
	מהרציפות נסיק כי קיימת פונקציה רציפה $\bar{h}$ כך שמתקיים $\bar{h} : U \to \CC$ אשר היא לוגריתם אנליטי וכך ש־$U$ פתוחה ו־$K \subseteq U$ ומקיימת את הטענה.
\end{proof}

\Question{}
נוכיח בכל סעיף שניתן להגדיר לוגריתם אנליטי לפונקציה $f(z) = \cos(z) - 2$ בתחומים הנתונים.

\Subquestion{}
עבור $U_1 = \{ z \in \CC \mid 0 < \re(z) < 2\pi \}$
\begin{proof}
	נגדיר $l_1(z) = \Log|f(z)| + \Arg(f(z)) + \pi i$, הארגומנט מקיים את שתי תכונות הארגומנט כפי שראינו בתרגול ובהתאם זהו לוגריתם אנליטי תקין.
	עוד נבחין כי
	\[
		f(z) = 0 \iff \cos z = 2 \iff e^{iz} + e^{-iz} = 4 \iff e^{2 iz} - 4 e^{iz} + 1 = 0 \iff e^{iz} = \frac{4 \pm \sqrt{12}}{2} \iff z = -i \log(2 \pm \sqrt{3})
	\]
	ולכן $0 \notin f(U_1)$ וההגדרה שנתנו רציפה, ובהתאם $l_1$ אכן לוגריתם אנליטי של $f$.
\end{proof}

\Subquestion{}
עבור התחום $U_2 = \{ z \in \CC \mid \im(z) < -3 \}$.
\begin{proof}
	גם הפעם נבחין כי $0 \notin f(U_2)$ ולכן נגדיר $l_2$ לוגריתם על $U_2$ של $f$ שמתחייב שקיים, וכמובן נוכל לקבוע אותו ביחידות על ידי הצבה $e^{l_2(-4i)} = \cos(-4i) - 2 = \frac{e^4 + e^{-4}}{2} - 2$ ולכן
	\[
		l_2(-4i) = \log(\frac{e^4 + e^{-4}}{2} - 2)
	\]
	ולכן הוא קיים ונקבע ביחידות.
\end{proof}

\Question{}
נמצא תחום $G \subseteq \CC \setminus \{ 0 \}$ וענף של הארגומנט $\arg : G \to \RR$ כך שיתקיים $\arg(G) = [0, \infty)$. % chktex 9
\begin{solution}
	נגדיר $G = \CC \setminus \{ t e^{t \theta} \mid t \ge 0 \}$. \\*
	בקבוצה זו נגדיר גם $\arg(1) = 0$ ולכן נובע $\arg$ כך שהוא רציף החל מ־$0$, זאת שכן שני ערכים $z_1, z_2$ כך ש־$\Arg(z_1) = \Arg(z_2), |z_1| = |z_2| + 2\pi k$ נמצאים משני צידי הספירלה, ולכן התנאי מתקיים.

	נבחין כי זו ההוכחה שראינו בתרגול, זוהי הוכחה גאומטרית.
\end{solution}

\Question{}
\Subquestion{}
תהינה $z_1, z_2, z_3 \in \CC$ נקודות שונות, ונוכיח כי קייצת העתקת מביוס יחידה $h(z) = \frac{az + b}{cz + d}$ המקיימת
\[
	h(z_1) = 0,
	\qquad
	h(z_2) = 1,
	\qquad
	h(z_3) = \infty
\]
\begin{proof}
	נניח את ההנחה ונסיק שמתקיים
	\[
		a z_1 + b = 0,
		\qquad
		c z_3 + d = 0,
		\qquad
		a z_2 + b = c z_2 + d
	\]
	נגדיר $a = k$ באופן שרירותי ולכן
	\[
		b = - k z_1,
		\qquad
		c z_2 + d = k z_2 - k z_1
	\]
	נחסר מהשוויון השני את אחד השוויונות הראשונים ואז
	\[
		c z_2 + d - c z_3 - d = c(z_2 - z_3) = -k (-z_2 + z_1) - 0
	\]
	ולכן
	\[
		c = -k \frac{-z_2 + z_1}{z_2 - z_3}
	\]
	ולבסוף
	\[
		d = -c z_3 = k z_3 \frac{-z_2 + z_1}{z_2 - z_3}
	\]
	אז ההעתקה מקיימת
	\[
		h(z)
		= \frac{k z - k z_1}{-k \frac{-z_2 + z_1}{z_2 - z_3} z + k z_3 \frac{-z_2 + z_1}{z_2 - z_3}}
		= \frac{z_2 - z_3}{z_2 - z_1} \cdot \frac{z - z_1}{z - z_3}
	\]
	ומצאנו ביטוי שקול ל־$h$ התלוי ב־$z_1, z_2, z_3$ בלבד, לכן כל ביטוי של $h$ שקול לביטוי זה ונקבע ביחידות על־ידי ערכים אלה.
\end{proof}

\Subquestion{}
נוכיח כי העתקות מביוס מעבירות מעגלים מוכללים למעגלים מוכללים על הספירה של רימן.
\begin{proof}
	אם $m z + n = 0$ ונבודק $m h(z) + n = 0$, נקבל $m(az + b) + n(cz + d) = 0$ וזה אכן מעגל מוכלל בספירת רימן. \\*
	בנוסף ${(z - n)}^2 = m$ מעגל ונקבל ${(h(z) - n)}^2 = m \iff {(az + b - n(cz + d))}^2 = n(cz + d)$ וזו משוואת מעגל מוכלל בספירה של רימן (מעבר כזה נעשה בהרצאה).
\end{proof}

\Subquestion{}
נמצא העתקת מביוס המעבירה את חצי המישור העליון $H = \{ z \in \CC \mid \im(z) > 0 \}$ לדיסק היחידה $B(0, 1)$.
\begin{solution}
	בכיתה נטענה הטענה שההעתקה $h(z) = i \frac{1 - z}{1 + z}$ מבצעת את המיפוי ההפוך בדיוק מזה שאנו נדרשים למצוא. \\*
	אנו גם יודעים כי העתקה זו נקבעת על־ידי
	\[
		A = \begin{pmatrix}
			1 & -1 \\
			i & i
		\end{pmatrix}
	\]
	זו מטריצה הפיכה ומהטענה שהוצגה אף היא בכיתה ${(h_A)}^{-1} = h_{A^{-1}}$ ולכן נחשב ונמצא כי
	\[
		A^{-1} = \frac{1}{2} \begin{pmatrix}
			1 & -i \\
			-1 & -i
		\end{pmatrix}
	\]
	ולכן ההעתקה שמקיימת את הדרישה היא
	\[
		h_{A^{-1}}(z)
		= \frac{z - i}{-z - i}
		= \frac{i - z}{z + i}
	\]
\end{solution}

\Subquestion{}
נמצא פונקציה אנליטית המעבירה את חצי דיסק היחידה העליון $B(0, 1) \cap H$ לדיסק היחידה $B(0, 1)$.
\begin{solution}
	תוצאת הסעיף הקודם תקפה גם פה, שכן $h$ שמצאנו אנליטית.
\end{solution}

\Question{}
נגדיר את הענף הראשי של $\arctan$ על־ידי
\[
	\arctan(z) = \frac{1}{2i} \Log(\frac{1 + iz}{1 - iz})
\]

\Subquestion{}
נמצא את תחום ההגדרה המקסימלי שבו $\arctan$ אנליטית.
\begin{solution}
	נבחין כי $\frac{1}{2i}$ קבוע ולא משפיע על הגזירות. \\*
	נעבור לייצוג הפונקציות על־ידי פונקציה דו־משתנית (לא כולל החלק שלא משפיע על הגזירות):
	\[
		\arctan(z)
		= -\frac{i}{2} (\log|\frac{1 + iz}{1 - iz}| + i \Arg(\frac{1 + iz}{1 - iz}))
		= -\frac{1}{2} (i \log|1 + iz| - i \log|1 - iz| - \Arg(\frac{1 + iz}{1 - iz}))
	\]
\end{solution}

\end{document} % chktex 17
