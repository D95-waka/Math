\documentclass[a4paper]{article}

% packages
\usepackage{inputenc, amsmath, amsthm, thmtools, amsfonts, amssymb, luacode, catchfile, tikzducks, hyperref}
\usepackage[a4paper, margin=50pt, includeheadfoot]{geometry} % set page margins
\usepackage[shortlabels]{enumitem}
\usepackage[skip=3pt, indent=0pt]{parskip}

% language
\usepackage[bidi=basic, layout=tabular, provide=*]{babel}
\babelprovide[main, import]{hebrew}
\babelprovide{rl}
\babelfont{rm}{Libertinus Serif}
\babelfont{sf}{Libertinus Sans}
\babelfont{tt}{Libertinus Mono}

% style
\AddToHook{cmd/section/before}{\clearpage}	% Add line break before section
\linespread{1.3}
\setcounter{secnumdepth}{0}		% Remove default number tags from sections, this won't do well with theorems
\AtBeginDocument{\setlength{\belowdisplayskip}{3pt}}
\AtBeginDocument{\setlength{\abovedisplayskip}{3pt}}

% operators
\DeclareMathOperator\cis{cis}
\DeclareMathOperator\Sp{Sp}
\DeclareMathOperator\tr{tr}
\DeclareMathOperator\im{Im}
\DeclareMathOperator\re{Re}
\DeclareMathOperator\diag{diag}
\DeclareMathOperator*\lowlim{\underline{lim}}
\DeclareMathOperator*\uplim{\overline{lim}}
\DeclareMathOperator\rng{rng}
\DeclareMathOperator\Sym{Sym}
\DeclareMathOperator\Arg{Arg}
\DeclareMathOperator\Log{Log}
\DeclareMathOperator\dom{dom}

% commands
%\renewcommand\qedsymbol{\textbf{מש''ל}}
%\renewcommand\qedsymbol{\fbox{\emoji{lizard}}}
\newcommand{\NN}[0]{\mathbb{N}}
\newcommand{\ZZ}[0]{\mathbb{Z}}
\newcommand{\QQ}[0]{\mathbb{Q}}
\newcommand{\RR}[0]{\mathbb{R}}
\newcommand{\CC}[0]{\mathbb{C}}
\newcommand{\FF}[0]{\mathbb{F}}
\newcommand{\PP}[0]{\mathbb{P}}
\newcommand{\TT}[0]{\mathbb{T}}
\newcommand{\acts}[0]{\circlearrowright}
\newcommand{\explain}[2] {
	\begin{flalign*}
		 && \text{#2} && \text{#1}
	\end{flalign*}
}
\newcommand{\maketitleprint}[0]{ \begin{center}
	\begin{tikzpicture}[scale=3]
		\duck[graduate=gray!20!black, tassel=red!70!black]
	\end{tikzpicture}	
\end{center}
}

% theorem commands
\newtheoremstyle{c_remark}
	{}	% Space above
	{}	% Space below
	{}% Body font
	{}	% Indent amount
	{\bfseries}	% Theorem head font
	{}	% Punctuation after theorem head
	{.5em}	% Space after theorem head
	{\thmname{#1}\thmnumber{ #2}\thmnote{ \normalfont{\text{(#3)}}}}	% head content
\newtheoremstyle{c_definition}
	{3pt}	% Space above
	{3pt}	% Space below
	{}% Body font
	{}	% Indent amount
	{\bfseries}	% Theorem head font
	{}	% Punctuation after theorem head
	{.5em}	% Space after theorem head
	{\thmname{#1}\thmnumber{ #2}\thmnote{ \normalfont{\text{(#3)}}}}	% head content
\newtheoremstyle{c_plain}
	{3pt}	% Space above
	{3pt}	% Space below
	{\itshape}% Body font
	{}	% Indent amount
	{\bfseries}	% Theorem head font
	{}	% Punctuation after theorem head
	{.5em}	% Space after theorem head
	{\thmname{#1}\thmnumber{ #2}\thmnote{ \text{(#3)}}}	% head content

\theoremstyle{c_plain}
\newtheorem{theorem}{משפט}[section]
\newtheorem{lemma}[theorem]{למה}
\newtheorem{proposition}[theorem]{טענה}
\newtheorem*{proposition*}{טענה}
%\newtheorem{corollary}[theorem]{אין חלופה עברית}

\theoremstyle{c_definition}
\newtheorem{definition}[theorem]{הגדרה}
\newtheorem*{definition*}{הגדרה}
\newtheorem{example}{דוגמה}[section]
\newtheorem{exercise}{תרגיל}[section]

\theoremstyle{c_remark}
\newtheorem*{remark}{הערה}
\newtheorem*{solution}{פתרון}
\newtheorem{conclusion}[theorem]{מסקנה}
\newtheorem{notation}[theorem]{סימון}

% Questions related commands
\newcounter{question}
\setcounter{question}{1}
\newcounter{sub_question}
\setcounter{sub_question}{1}

\newcommand{\question}[1][0]{
	\ifthenelse{#1 = 0}{}{\setcounter{question}{#1}}
	\subsection{שאלה \arabic{question}}
	\addtocounter{question}{1}
	\setcounter{sub_question}{1}
}

\newcommand{\subquestion}[1][0]{
	\ifthenelse{#1 = 0}{}{\setcounter{sub_question}{#1}}
	\subsubsection{סעיף \localecounter{letters.gershayim}{sub_question}}
	\addtocounter{sub_question}{1}
}

% import lua and start of document
\directlua{common = require ('../common')}

\GetEnv{AUTHOR}

% headers
\author{\AUTHOR}
\date\today

\title{פתרון מטלה 13 --- פונקציות מרוכבות, 80519}

\begin{document}
\maketitle
\maketitleprint{}

\question{}
תהי $\gamma : [0, 2\pi] \to \CC$ המוגדרת על־ידי $\gamma(t) = e^{4it} - 7e^{3it} + 2e^{it} + 6$. \\
נחשב את $\operatorname{Ind}_\gamma(3)$.
\begin{solution}
	מתכונות אינדקס ליפוף נובע,
	\[
		\operatorname{Ind}_\gamma(3)
		= \operatorname{Ind}_{\gamma - 3}(0)
	\]
	אבל אם מגדירים $f(z) = z^4 - 7z^3 + 2z + 3$ מקבלים מתכונות גם,
	\[
		\operatorname{Ind}_{\gamma - 3}(0)
		= \operatorname{Ind}_{f \circ e^{it}}(0)
		= \frac{1}{2\pi i} \Delta_\gamma f
	\]
	ומעקרון הארגומנט,
	\[
		\frac{1}{2\pi i} \Delta_\gamma f
		= |Z_f \cap D|
	\]
	כאשר $Z_f = \{ z \in \CC \mid f(z) = 0 \}$ ו־$D = \overline{B}(0, 1)$.
	נבחין כי עבור $z \in \partial D$,
	\[
		|f(z) + 7z^3|
		\le |z^4| + 2 |z| + 3
		= 6
		< 7
		= |-7z^3|
	\]
	ולכן ממשפט רושה ל־$f$ אותו מספר אפסים כמו ל־$-2z^3$, כלומר שלושה אפסים בדיוק, כלומר
	\[
		\operatorname{Ind}_{\gamma - 3}(0)
		= 3
	\]
\end{solution}

\question{}
תהי $\gamma : I \to \CC$ מסילה סגורה.

\subquestion{}
נוכיח כי ל־$\CC \setminus \gamma(I)$ יש רכיב קשירות יחיד לא חסום.
\begin{proof}
	נניח בשלילה שישנם לפחות שני רכיבי קשירות לא חסומים, $\Omega_1, \Omega_2$, נבחין כי מהגדרת מסילה סגורה קיים $r_0 > 0$ כך ש־$\CC \setminus B(0, r_0) = (\Omega_1 \setminus B(0, r_0)) \uplus (\Omega_2 \setminus B(0, r_0))$.
	יהי $r > r_0$, נראה ש־$\partial B(0, r) = (\partial B(0, r) \cap \Omega_1) \uplus (\partial B(0, r) \cap \Omega_2)$, ואם אחד משני התחומים ריק אז נובע שהתחום השני חסום בסתירה, לכן שניהם לא ריקים.
	נבחר $z_r \in \overline{(\partial B(0, r) \cap \Omega_1)} \cap \overline{(\partial B(0, r) \cap \Omega_2)}$, נבחין כי זו אכן קבוצה לא ריקה מהעובדה ששתי הקבוצות לא ריקות.
	נגדיר סדרה ${\{z_n\}}_{n = 1}^\infty \subseteq \CC$ על־ידי $z_n = z_r$ כך ש־$r = n$.
	הנקודות המחלקות את התחומים שייכות ל־$\gamma(I)$, כלומר ${\{z_n\}}_{n = 1}^\infty \subseteq \gamma(I)$, וכן $|z_n| = n$ לכל $n \in \NN$, לכן קבוצת הנקודות הזו לא חסומה, וזוהי כמובן סתירה לסגירות $\gamma$.
	נסיק אם כך שאין שני רכיבי קשירות שונים לא חסומים.
\end{proof}

\subquestion{}
נראה שהפונקציה $\operatorname{Ind}_\gamma : \CC \setminus \gamma(I) \to \ZZ$ רציפה ומתאפסת ברכיב הקשירות הלא חסום.
\begin{proof}
	תהי $z \in \CC \setminus \gamma(I)$, לכל $\epsilon > 0$ כך ש־$B(z, \epsilon) \subseteq \CC \setminus \gamma(I)$, אם $w \in B(z, \epsilon)$ אז,
	\begin{align*}
		|\operatorname{Ind}_\gamma(z) - \operatorname{Ind}_\gamma(w)|
		& = \left\lvert \frac{1}{2\pi i} \left( \int_\gamma \frac{1}{\zeta - z}\ d\zeta - \int_\gamma \frac{1}{\zeta - w}\ d\zeta \right) \right\rvert \\
		& = \frac{1}{2\pi} \left\lvert \int_\gamma \frac{w - z}{(\zeta - z)(\zeta - w)}\ d\zeta \right\rvert \\
		& \le \frac{\epsilon}{2\pi} L(\gamma) \cdot \max_{\zeta \in \gamma} \left\lvert \frac{1}{(\zeta - z)(\zeta - w)} \right\rvert
	\end{align*}
	נבחין שכאשר $z \to w$ נקבל $\max_{\zeta \in \gamma} \left\lvert \frac{1}{(\zeta - z)(\zeta - w)} \right\rvert \to \max_{\zeta \in \gamma} \left\lvert \frac{1}{{(\zeta - z)}^2} \right\rvert$, כלומר ביטוי קבוע. \\
	גם $L(\gamma)$ קבוע ולכן $|\operatorname{Ind}_\gamma(z) - \operatorname{Ind}_\gamma(w)| \xrightarrow{\epsilon \to 0} 0$, כלומר $\operatorname{Ind}_\gamma$ רציפה.

	נראה גם שברכיב הקשירות הלא חסום הפונקציה מתאפסת.
	מצאנו בסעיף הקודם כי רכיב הקשירות איננו חסום, ומרציפות נובע שבאותו רכיב קשירות ערך הפונקציה (שתמונתה איננה רציפה) הוא קבוע, כלומר לכל $r > r_0$ עבור $r_0$ החוסם את $\gamma(I)$ נובע,
	\[
		|\operatorname{Ind}_\gamma(z)|
		= \left\lvert \frac{1}{2\pi i} \int_\gamma \frac{1}{\zeta - z}\ d\zeta \right\rvert
		\le \frac{L(\gamma)}{2\pi} \max_{\zeta \in \gamma} \left\lvert \frac{1}{\zeta - z} \right\rvert
		= \frac{L(\gamma)}{2\pi} (r - r_0)
		\xrightarrow{r \to \infty} 0
	\]
	אבל מרציפות $\operatorname{Ind}_\gamma$ ומתמונתה (שאיננה רציפה) נובע שלכל $z, w \in \Omega_i$ כאשר $\Omega_i$ רכיב קשירות כלשהו,
	\[
		\operatorname{Ind}_\gamma(z) = \operatorname{Ind}_\gamma(w)
	\]
	ומצאנו שקיים $r > r_0$ עבורו $\operatorname{Ind}_\gamma(z) < 1$ לכל $z \in \CC \setminus B(0, r)$, ולכן $\operatorname{Ind}_\gamma(z) = 0$ לכל $z \in \CC \Omega_j$ רכיב הקשירות הלא חסום.
\end{proof}

\question{}
תהי $f : \overline{D} \to \RR$ פונקציה רציפה ב־$\overline{D}$ והרמונית ב־$D$.

\subquestion{}
נוכיח שאם $u \ge 0$ על $\partial D$ אז $u \ge 0$ ב־$D$.
\begin{proof}
	מתכונת הערך הממוצע נובע,
	\[
		u(z)
		= \frac{1}{2\pi} \int_0^{2\pi} u(e^{it})\ dt
		\ge 0
	\]
	זאת שכן $u(e^{it}) \ge 0$ לכל $t \in [0, 2\pi]$.
\end{proof}

\subquestion{}
נוכיח את אי־שוויון Harnack,
נניח ש־$u \ge 0$, ונראה שלכל $z \in D$ מתקיים,
\[
	\frac{1 - |z|}{1 + |z|} u(0) \le u(z) \le \frac{1 + |z|}{1 - |z|} u(0)
\]
\begin{proof}
	נבחין כי המשפט על גרעין פואסון חל ולכן
	\[
		u(z)
		= (P_D u)(z)
		= \frac{1}{2\pi} \int_0^{2\pi} \frac{1 - |z|^2}{{|e^{it} - z|}^2} u(e^{it})\ dt
	\]
	ולכן בפרט $u(0) = \frac{1}{2\pi} \int_0^{2\pi} u(e^{it})\ dt$.
	מאי־שוויון המשולש נובע,
	\[
		u(z)
		\le \frac{1}{2\pi} \int_0^{2\pi} \frac{(1 - |z|)(1 + |z|)}{{(|e^{it}| - |z|)}^2} u(e^{it})\ dt
		= \frac{1}{2\pi} \int_0^{2\pi} \frac{1 + |z|}{1 - |z|} u(e^{it})\ dt
		= \frac{1}{2\pi} \frac{1 + |z|}{1 - |z|} \int_0^{2\pi} u(e^{it})\ dt
		= \frac{1 + |z|}{1 - |z|}
	\]
	מהצד השני מתקיים,
	\[
		u(z)
		\ge \frac{1}{2\pi} \int_0^{2\pi} \frac{(1 - |z|)(1 + |z|)}{{(|e^{it}| + |z|)}^2} u(e^{it})\ dt
		\ge \frac{1}{2\pi} \int_0^{2\pi} \frac{1 - |z|}{1 + |z|} u(e^{it})\ dt
		= \frac{1 - |z|}{1 + |z|}
	\]
	ומצאנו את שני חלקי אי־השוויון.
\end{proof}

\question{}
נחשב את האינטגרל הבא בעזרת  נוסחת ינסן,
\[
	\int_0^{2\pi} \log(9 + 16 \sin^2 t)\ dt
\]
\begin{solution}
	נגדיר $f(z) = 4z^2 - 1$, נבדוק שזוהי פונקציה הרמונית.
	\[
		\frac{\partial^2 f}{\partial z \partial \overline{z}}
		= \frac{\partial}{\partial z} 0
		= 0
	\]
	ולכן $f$ היא הרמונית ב־$D$ ומקיימת את נוסחת ינסן,
	\[
		\log |f(0)|
		= \sum_{k = 1}^{n} m_k \log |a_k| + \frac{1}{2\pi} \int_0^{2\pi} \log |f(e^{it})|\ dt
	\]
	כאשר $a_1, \dots, a_n$ הם האפסים של $f$ ב־$D$ עם ריבויים $m_1, \dots, m_n$.
	נובע גם $\log |f(0)| = \log |-1| = 0$, ונבחין כי גם $z = \pm \frac{1}{2}$ האפסים היחידים של $f$, כלומר $a_1 = \frac{1}{2}$ ו־$m_1 = 1$, וכן גם $a_2 = -\frac{1}{2}$ ו־$m_2 = 1$.
	נציב ונקבל,
	\[
		0 = \log \left\lvert \frac{1}{2} \right\rvert + \log \left\lvert -\frac{1}{2} \right\rvert + \frac{1}{2\pi} \int_0^{2\pi} \log |4e^{2it} - 1|\ dt
	\]
	כלומר,
	\[
		\int_0^{2\pi} \log |4 e^{2it} - 1|\ dt = 2\pi \log 4
	\]
	נבחין כי גם,
	\begin{align*}
		& {|4 e^{2it} - 1|}^2
		&& = {(4 \cos(2t) - 1)}^2 + {(4\sin(2t))}^2 \\
		& = 16 \cos^2(2t) - 8 \cos(2t) + 1 + 16 \sin^2(2t)
		&& = -8 \cos(2t) + 17 \\
		& = -8 (1 - 2\sin^2 t) + 17
		&& = 16 \sin^2 t + 9
	\end{align*}
	ונובע,
	\[
		\int_0^{2\pi} \log(9 + 16\sin^2 t)\ dt = 8\pi \log 2
	\]
\end{solution}

\end{document} % chktex 17
